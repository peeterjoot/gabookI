%
% Copyright � 2016 Peeter Joot.  All Rights Reserved.
% Licenced as described in the file LICENSE under the root directory of this GIT repository.
%
%{
\input{../latex/blogpost.tex}
\renewcommand{\basename}{whatsTheDifference}
%\renewcommand{\dirname}{notes/phy1520/}
\renewcommand{\dirname}{notes/ece1228-electromagnetic-theory/}
%\newcommand{\dateintitle}{}
%\newcommand{\keywords}{}

\input{../latex/peeter_prologue_print2.tex}

\usepackage{peeters_layout_exercise}
\usepackage{peeters_braket}
\usepackage{peeters_figures}
\usepackage{siunitx}
%\usepackage{txfonts} % \ointclockwise

\beginArtNoToc

\generatetitle{XXX}
%\chapter{XXX}
%\label{chap:whatsTheDifference}

Exterior algebra defines an antisymmetric wedge product.  An example of the wedge product of two vectors, called a two-form (unit vectors in this case) is

\begin{dmath}\label{eqn:whatsTheDifference:20}
\Be_1 \wedge \Be_2 = -\Be_2 \wedge \Be_1.
\end{dmath}

An example of a wedge product of three (unit) vectors, a three-form, is

\begin{dmath}\label{eqn:whatsTheDifference:40}
\begin{aligned}
\Be_1 \wedge \Be_2 \wedge \Be_3
&= -\Be_2 \wedge \Be_1 \wedge \Be_3 \\
&= \Be_2 \wedge \Be_3 \wedge \Be_1 \\
&= -\Be_3 \wedge \Be_2 \wedge \Be_1.
\end{aligned}
\end{dmath}

A consequence of this antisymmetry is that any wedge product where one of the wedged vectors is colinear with another is zero.  Exterior algebra also has the concept of duality, which provides a mapping between k-forms and N-k forms, where N is the dimension of the underlying vector space.  For example, in a three dimensional Euclidean space the dual of the two form \( \Be_1 \wedge \Be_2 \), denoted \( *\lr{ \Be_1 \wedge \Be_2} \) is the quantity

\begin{dmath}\label{eqn:whatsTheDifference:60}
*\lr{\Be_1 \wedge \Be_2} \wedge \lr{ \Be_1 \wedge \Be_2} = \Be_1 \wedge \Be_2 \wedge \Be_3,
\end{dmath}

so
\begin{dmath}\label{eqn:whatsTheDifference:80}
*\lr{\Be_1 \wedge \Be_2} = \Be_3.
\end{dmath}

In an exterior algebra, one can add k-forms to other k-forms, but would not add forms of different rank.  This restriction is relaxed in Geometric Algebra (GA), where a quantity such as

\begin{dmath}\label{eqn:whatsTheDifference:100}
1 + 2 \Be_1 + 3 \Be_2 \wedge \Be_4 + 5 \Be_1 \wedge \Be_2 \wedge \Be_4,
\end{dmath}

is perfectly well formed.  The Geometric Algebra is built up of products of vectors, where the vector product is defined as an associative product

\begin{dmath}\label{eqn:whatsTheDifference:120}
\Ba (\Bb \Bc) = (\Ba \Bb) \Bc = \Ba \Bb \Bc,
\end{dmath}

and where the product of a vector with itself is defined as the squared length of that vector

\begin{dmath}\label{eqn:whatsTheDifference:140}
\Ba \Ba = \Ba \cdot \Ba = \Abs{\Ba}^2.
\end{dmath}

In an Euclidean space such length is always positive, but that mixed sign length metrics (such as that of the Minkowski space used in special relativity) are also allowed.

The product of two non-colinear vectors can be factored as

\begin{dmath}\label{eqn:whatsTheDifference:160}
\Ba \Bb = \inv{2} \lr{ \Ba \Bb + \Bb \Ba } + \inv{2} \lr{ \Ba \Bb - \Bb \Ba }.
\end{dmath}

The first (symmetric) term can be identified with the dot-product, whereas the second completely antisymmetric term can be identified as with the wedge product, so this complete vector product is denoted

\begin{dmath}\label{eqn:whatsTheDifference:180}
\Ba \Bb = \Ba \cdot \Bb + \Ba \wedge \Bb.
\end{dmath}

This is one of the simplest examples of what is called a multivector in GA, containing the sum of a scalar (grade zero) and a bivector (grade two).  There are a number of other consequences of the product axioms of GA.  One such consequence is that the product of two perpendicular vectors is antisymmetric, and that any unit vector has a unit square.  A number of specific algebraic structures can be represented with Geometric Algebras.  For example, one can identify the algebra spanned by a scalar and unit bivector, such as

\begin{dmath}\label{eqn:whatsTheDifference:200}
\Span \setlr{ 1, \Be_1 \Be_2 }
\end{dmath}

with complex numbers.  This is because any unit bivector of this form (in a Euclidean space) squares to unity

\begin{dmath}\label{eqn:whatsTheDifference:220}
\begin{aligned}
(\Be_1 \Be_2)^2
&= (\Be_1 \Be_2)(\Be_1 \Be_2) \\
&= \Be_1 (\Be_2 \Be_1) \Be_2 \\
&= -\Be_1 (\Be_1 \Be_2) \Be_2 \\
&= -(\Be_1 \Be_1) (\Be_2 \Be_2) \\
&= - (1)(1) \\
&= -1.
\end{aligned}
\end{dmath}

Other examples of algebraic structures that can have GA representations include quaternions, the Pauli (spin) algebra of quantum mechanics, and the Dirac algebra from QED.

The GA representation of dual vectors is through multiplication by a (unit) pseudoscalar, often denoted \( I \), for the vector space (an ordered product of all the unit vectors of the space).  For example, negative multiplication by the three dimensional pseudoscalar has the duality property illustrated in the exterior algebra duality example

\begin{dmath}\label{eqn:whatsTheDifference:240}
\begin{aligned}
-I \Be_1
&=
-\Be_1 \Be_2 \Be_3 \Be_1 \\
&=
\Be_1 \Be_2 \Be_1 \Be_3 \\
&=
\Be_2 \Be_3,
\end{aligned}
\end{dmath}

\begin{dmath}\label{eqn:whatsTheDifference:260}
\begin{aligned}
-I
\Be_2 \Be_3
&=
- \Be_1 \Be_2 \Be_3 \Be_2 \Be_3 \\
&=
\Be_1 \Be_2 \Be_2 \Be_3 \Be_3 \\
&=
\Be_1.
\end{aligned}
\end{dmath}

A number of fundamental geometric operations, such as projection, rotation, and reflection can all be represented using GA multivector product operations.

In GA the basis vectors for the space are typically real valued vectors.  Clifford algebras provide a further generalization, allowing those basis vectors to reside in a complex vector space.

All of these algebras are linear algebras.  In an exterior algebra

\begin{dmath}\label{eqn:whatsTheDifference:280}
\begin{aligned}
(c \Ba) \wedge \Bb
&=
c (\Ba \wedge \Bb) \\
&=
\Ba \wedge (c \Bb),
\end{aligned}
\end{dmath}

and
\begin{dmath}\label{eqn:whatsTheDifference:300}
\Ba \wedge (b \Bb + c \Bc) = b \Ba \wedge \Bb + c \Ba \wedge \Bc,
\end{dmath}

or in GA

\begin{dmath}\label{eqn:whatsTheDifference:320}
\Ba \lr{ b \Bb + c \Bc \Bd }
=
b \Ba \Bb + c \Ba \Bc \Bd.
\end{dmath}

%}
\EndArticle
%\EndNoBibArticle
