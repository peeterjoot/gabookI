%
% Copyright � 2012 Peeter Joot.  All Rights Reserved.
% Licenced as described in the file LICENSE under the root directory of this GIT repository.
%

%
%
\chapter{Spherical and hyperspherical parametrization}
\index{spherical parametrization}
\index{hyperspherical parametrization}
\label{chap:nvolume}
%\date{Feb 26, 2009.  nvolume.tex}

\section{Motivation}

In \citep{gabookII:PJ4dFourier} a 4D Fourier transform solution of Maxwell's equation yielded a Green's function of the form

\begin{equation}\label{eqn:nvolume:20}
\begin{aligned}
G(x) = \iiiint \frac{e^{i k_\mu x^\mu}}{k_\nu k^\nu} dk_1 dk_2 dk_3 dk_4
\end{aligned}
\end{equation}

To attempt to ``evaluate'' this integral, as done in
\citep{gabookII:PJpoisson}
to produce the retarded time potentials,
a hypervolume equivalent to spherical polar coordinate
parametrization is probably desirable.

Before attempting to tackle the problem of interest, the basic question
of how to do volume and weighted volume integrals over a hemispherical volumes
must be considered.  Doing this for both Euclidean and Minkowski metrics will have to be covered.

\section{Euclidean n-volume}
\index{n-volume}

\subsection{Parametrization}

The \href{http://en.wikipedia.org/wiki/Hypersphere}{wikipedia article on n-volumes} gives a parametrization, which I will write out explicitly for the first few dimensions

\begin{itemize}
\item 1-sphere (circle)

\begin{equation}\label{eqn:nvolume:40}
\begin{aligned}
x^1 &= r \cos\phi_1 \\
x^2 &= r \sin\phi_1
\end{aligned}
\end{equation}

\item 2-sphere (sphere)

\begin{equation}\label{eqn:nvolume:60}
\begin{aligned}
x^1 &= r \cos\phi_1 \\
x^2 &= r \sin\phi_1 \cos\phi_2 \\
x^3 &= r \sin\phi_1 \sin\phi_2
\end{aligned}
\end{equation}

\item 3-sphere (hypersphere)

\begin{equation}\label{eqn:nvolume:80}
\begin{aligned}
x^1 &= r \cos\phi_1 \\
x^2 &= r \sin\phi_1 \cos\phi_2 \\
x^3 &= r \sin\phi_1 \sin\phi_2 \cos\phi_3 \\
x^4 &= r \sin\phi_1 \sin\phi_2 \sin\phi_3 \\
\end{aligned}
\end{equation}

\end{itemize}

By inspection one can see that we have the desired \(r^2 = \sum_i (x^i)^2\) relation.  Each of these can be vectorized to produce
a parametrized vector that can trace out all the possible points on the volume

\begin{equation}\label{eqn:nvolume:100}
\begin{aligned}
\Br &= \sigma_k x^k
\end{aligned}
\end{equation}

\subsection{Volume elements}

We can form a parallelogram area (or parallelepiped volume, ...) element for any parametrized surface by taking wedge products, as in \cref{fig:tangent_area_form}.  This can also be done for this spherical parametrization too.

\imageFigure{../../physicsplay/figures/gabook/tangent_area_form}{Tangent vector along curves of parametrized vector}{fig:tangent_area_form}{0.4}

For example for the circle we have

\begin{equation}\label{eqn:nvolume:120}
\begin{aligned}
dV_{\Rm{2}}
&= \PD{r}{\Br} \wedge \PD{\phi_1}{\Br} dr d\phi_1 \\
&= \left(\PD{r}{}r(\cos\phi_1, \sin\phi_1) \right) \wedge \left(\PD{\phi_1}{}r(\cos\phi_1, \sin\phi_1) \right) dr d\phi_1 \\
&= (\cos\phi_1, \sin\phi_1) \wedge (-\sin\phi_1, \cos\phi_1) r dr d\phi_1 \\
&= (\cos^2\phi_1 \sigma_1 \sigma_2 - \sin^2\phi_1 \sigma_2 \sigma_1 ) r dr d\phi_1 \\
&= r dr d\phi_1 \sigma_1 \sigma_2 \\
\end{aligned}
\end{equation}

And for the sphere
\begin{equation}\label{eqn:nvolume:140}
\begin{aligned}
dV_{\Rm{3}}
&= \PD{r}{\Br} \wedge \PD{\phi_1}{\Br} \wedge \PD{\phi_2}{\Br} dr d\phi_1 d\phi_2 \\
&= (\cos\phi_1, \sin\phi_1 \cos\phi_2, \sin\phi_1 \sin\phi_2)  \\
&\wedge (-\sin\phi_1, \cos\phi_1 \cos\phi_2, \cos\phi_1 \sin\phi_2)  \\
&\wedge (0, -\sin\phi_1 \sin\phi_2, \sin\phi_1 \cos\phi_2) r^2 dr d\phi_1 d\phi_2 \\
&=
\begin{vmatrix}
\cos\phi_1 & \sin\phi_1 \cos\phi_2 & \sin\phi_1 \sin\phi_2  \\
-\sin\phi_1 & \cos\phi_1 \cos\phi_2 & \cos\phi_1 \sin\phi_2  \\
0 & -\sin\phi_1 \sin\phi_2 & \sin\phi_1 \cos\phi_2  \\
\end{vmatrix} r^2 dr d\phi_1 d\phi_2 \sigma_1 \sigma_2 \sigma_3 \\
&=
r^2 dr \sin\phi_1 d\phi_1 d\phi_2 \sigma_1 \sigma_2 \sigma_3 \\
\end{aligned}
\end{equation}

And finally for the hypersphere

\begin{equation}\label{eqn:nvolume:160}
\begin{aligned}
dV_{\Rm{4}} &=
\begin{vmatrix}
\cos\phi_1 & \sin\phi_1 \cos\phi_2 & \sin\phi_1 \sin\phi_2 \cos\phi_3 & \sin\phi_1 \sin\phi_2 \sin\phi_3  \\
-\sin\phi_1 & \cos\phi_1 \cos\phi_2 & \cos\phi_1 \sin\phi_2 \cos\phi_3 & \cos\phi_1 \sin\phi_2 \sin\phi_3  \\
         0 & -\sin\phi_1 \sin\phi_2 & \sin\phi_1 \cos\phi_2 \cos\phi_3 & \sin\phi_1 \cos\phi_2 \sin\phi_3  \\
         0 &                     0 & -\sin\phi_1 \sin\phi_2 \sin\phi_3 & \sin\phi_1 \sin\phi_2 \cos\phi_3  \\
\end{vmatrix} \\
&\quad r^3 dr d\phi_1 d\phi_2 d\phi_3 \sigma_1 \sigma_2 \sigma_3 \sigma_4 \\
&=
%
%
%C_1  &  S_1 C_2 &  S_1 S_2 C_3 & S_1 S_2 S_3  \\
%-S_1 &  C_1 C_2 &  C_1 S_2 C_3 & C_1 S_2 S_3  \\
%   0 & -S_1 S_2 &  S_1 C_2 C_3 & S_1 C_2 S_3  \\
%   0 &        0 & -S_1 S_2 S_3 & S_1 S_2 C_3  \\
%
%
%C_1^2 C_2 S_1^2 C_2 S_2
%+C_1 S_1 S_2^3 C_1 S_1
%+S_1^2 C_2 S_1^2 C_2 S_2
%+S_1^2 S_2 S_1^2 S_2^2
%
%
%+C_1^2 C_2^2 S_1^2 S_2
%+C_1^2 S_1^2 S_2^3
%+S_1^4 C_2^2 S_2
%+S_1^4 S_2^3
%
%
%+ S_1^2 S_2
r^3 dr \sin^2\phi_1 d\phi_1 \sin\phi_2 d\phi_2 d\phi_3 \sigma_1 \sigma_2 \sigma_3 \sigma_4 \\
\end{aligned}
\end{equation}

Each of these is consistent with the result in the wiki page.

\subsection{Some volume computations}

Let us apply the above results to compute the corresponding n-volume's.


\begin{itemize}
\item 1-sphere (circle)
\begin{equation}\label{eqn:nvolume:180}
\begin{aligned}
V_{\Rm{2}}
&= 4 \int_{0}^R r dr \int_{0}^{\pi/2} d\phi_1 \\
&= \pi R^2
\end{aligned}
\end{equation}

\item 2-sphere (sphere)

\begin{equation}\label{eqn:nvolume:200}
\begin{aligned}
V_{\Rm{3}}
&= 8 \int_0^R r^2 dr \int_0^{\pi/2} \sin\phi_1 d\phi_1 \int_0^{\pi/2} d\phi_2 \\
&= 8 \inv{3} R^3 \left( {\left. -\cos\phi_1 \right\vert}_0^{\pi/2} \right) \frac{\pi}{2} \\
&= \frac{4\pi R^3}{3} \\
\end{aligned}
\end{equation}

Okay, so far so good.

\item 3-sphere (hypersphere)

\begin{equation}\label{eqn:nvolume:220}
\begin{aligned}
V_{\Rm{3}}
&= 16 \int_0^R r^3 dr \int_0^{\pi/2} \sin^2\phi_1 d\phi_1 \int_0^{\pi/2} \sin\phi_2 d\phi_2 \int_0^{\pi/2} d\phi_3 \\
&= 2 \pi R^4 \int_0^{\pi/2} \sin^2\phi_1 d\phi_1 \\
&= \pi R^4 \left( {\left.  \phi_1 - \cos\phi_1 \sin\phi_1 \right\vert}_0^{\pi/2} \right) \\
&= \frac{\pi^2 R^4}{2} \\
\end{aligned}
\end{equation}

This is also consistent with the formula supplied in the wiki article.

\end{itemize}

\subsection{Range determination}

What I have done here though it integrate over only one of the quadrants, and multiply by \(2^n\).  This avoided the more tricky issue of what exact range of angles is required for a complete and non-overlapping cover of the surface.

The wiki article says that the range is \([0,2\pi]\) for the last angle and \([0,\pi]\) for the others.
Reevaluating the integrals above shows that this does work, but that is a bit of a cheat, and it is not obvious
to me past \R{3} that this should be the case.

How can this be rationalized?

\begin{itemize}
\item circle
For the case of the circle what are the end points in each of the quadrants?  These are (with \(r=1\))

\begin{equation}\label{eqn:nvolume:240}
\begin{aligned}
(\cos\phi_1, \sin\phi_1)_{\phi_1 = 0}        &= ( 1, 0 ) = \sigma_1 \\
(\cos\phi_1, \sin\phi_1)_{\phi_1 = \pi/2}    &= ( 0, 1 ) = \sigma_2 \\
(\cos\phi_1, \sin\phi_1)_{\phi_1 = \pi}      &= ( -1, 0 ) = -\sigma_1 \\
(\cos\phi_1, \sin\phi_1)_{\phi_1 = 3\pi/2}   &= ( 0, -1 ) = -\sigma_2
\end{aligned}
\end{equation}

As expected, each of the \(\pi/2\) increments traces out the points in successive quadrants.

\item sphere

Again with \(r=1\), some representative points on the circle are

\begin{tabular}{|l|l|l|l|}
\hline
\(\phi_1\) & \(\phi_2\) & \((\cos\phi_1, \sin\phi_1 \cos\phi_2, \sin\phi_1 \sin\phi_2)\) & \(\Br\) \\
\hline
0 & 0 & \((1,0,0)\) & \(\sigma_1\) \\
0 & \(\pi/2\) &  \((1,0,0)\) & \(\sigma_1\) \\
0 & \(\pi\) &  \((1,0,0)\) & \(\sigma_1\) \\
0 & \(3\pi/2\) &  \((1,0,0)\) & \(\sigma_1\) \\
%
% \((0, \cos\phi_2, \sin\phi_2)\)
%
% traces out a circle in y,z plane
%
\(\pi/2\) & 0 &  \((0,1,0)\) & \(\sigma_2\) \\
\(\pi/2\) & \(\pi/2\) &  \((0,0,1)\) & \(\sigma_3\) \\
\(\pi/2\) & \(\pi\) &  \((0,-1,0)\) & \(-\sigma_2\) \\
\(\pi/2\) & \(3\pi/2\) &  \((0,0,-1)\) & \(-\sigma_3\) \\
%
%
% \((-1, 0\cos\phi_2, \sin\phi_1 \sin\phi_2)\) & \(\Br\) \\
%
\(\pi\) & 0 &  \((-1,0,0)\) & \(-\sigma_1\) \\
\(\pi\) & \(\pi/2\) &  \((-1,0,0)\) & \(-\sigma_1\) \\
\(\pi\) & \(\pi\) &  \((-1,0,0)\) & \(-\sigma_1\) \\
\(\pi\) & \(3\pi/2\) & \((-1,0,0)\) & \(-\sigma_1\) \\
\hline
\end{tabular}

The most informative of these is for \(\phi_1 = \pi/2\), where we had \(\Br = (0, \cos\phi_2, \sin\phi_2)\), and our points trace out a path along the unit circle of the \(y,z\) plane.  At \(\phi_1 = 0\) our point \(\Br = \sigma_1\) did not move, and at \(\phi_1 = \pi\) we are at the other end of the sphere, also fixed.  A reasonable guess is that at each \(\phi_1\) we trace out a different circle in the \(y,z\) plane.

We can write, with \(\sigma_{23} = \sigma_1 \wedge \sigma_2 = \sigma_1 \sigma_2\),

\begin{equation}\label{eqn:nvolume:260}
\begin{aligned}
\Br
%&= \cos\phi_1 (1, \tan\phi_1 \cos\phi_2, \tan\phi_1 \sin\phi_2) \\
&= \cos\phi_1 \sigma_1 + \sin\phi_1 ( \cos\phi_2 \sigma_2 + \sin\phi_2 \sigma_3 ) \\
&= \cos\phi_1 \sigma_1 + \sin\phi_1 \sigma_2 ( \cos\phi_2 + \sin\phi_2 \sigma_2 \sigma_3 ) \\
\end{aligned}
\end{equation}

Or, in exponential form

\begin{equation}\label{eqn:nvolume:sphereVec}
\begin{aligned}
\Br &= \cos\phi_1 \sigma_1 + \sin\phi_1 \sigma_2 \exp( \sigma_{23} \phi_2 )
\end{aligned}
\end{equation}

Put this way the effects of the parametrization is clear.   For each fixed \(\phi_1\), the exponential traces out a circle in the \(y,z\) plane, starting at the point \(\Br = \cos\phi_1 \sigma_1 + \sin\phi_1 \sigma_2\).  \(\phi_1\) traces out a semi-circle in the \(x,y\) plane.

FIXME: picture.

This would have been easy enough to understand if starting from a picture and constructing the parametrization.  Seeing what the
geometry is from the algebra requires a bit more (or different) work.  Having done it, are we now prepared to understand the geometry
of the hypersphere parametrization.

\item hypersphere.

The vector form in the spherical case was convenient for extracting geometric properties.  Can we do that here too?

\begin{equation}\label{eqn:nvolume:280}
\begin{aligned}
\Br
&= \sigma_1 \cos\phi_1 + \sigma_2 \sin\phi_1 \cos\phi_2 + \sigma_3 \sin\phi_1 \sin\phi_2 \cos\phi_3 + \sigma_4 \sin\phi_1 \sin\phi_2 \sin\phi_3 \\
&= \sigma_1 \cos\phi_1 + \sigma_2 \sin\phi_1 \cos\phi_2 + \sigma_3 \sin\phi_1 \sin\phi_2 (\cos\phi_3 + \sigma_{34} \sin\phi_3) \\
&= \sigma_1 \cos\phi_1 + \sigma_2 \sin\phi_1 \cos\phi_2 + \sigma_3 \sin\phi_1 \sin\phi_2 \exp(\sigma_{34}\phi_3) \\
&= \sigma_1 \cos\phi_1 + \sigma_2 \sin\phi_1 (\cos\phi_2 + \sigma_{23} \sin\phi_2 \exp(\sigma_{34}\phi_3) ) \\
\end{aligned}
\end{equation}

Observe that if \(\phi_3 = 0\) we have

\begin{equation}\label{eqn:nvolume:300}
\begin{aligned}
\Br
&= \sigma_1 \cos\phi_1 + \sigma_2 \sin\phi_1 \exp( \sigma_{23} \phi_2 ) \\
\end{aligned}
\end{equation}

Which is exactly the parametrization of a half sphere (\(\phi_2 \in [0,\pi]\)).  Contrast this to the semi-circle that \(\phi_1\) traced out in the spherical case.

In the spherical case, the points \(\phi_1 = \pi/2\) were nicely representative.  For the hypersphere those points are

\begin{equation}\label{eqn:nvolume:320}
\begin{aligned}
\Br &= \sigma_2 \cos\phi_2 + \sigma_3 \sin\phi_2 \exp(\sigma_{34}\phi_3) \\
\end{aligned}
\end{equation}

We saw above that this is the parametrization of a sphere.

Also like the spherical case, we have \(\Br = \pm \sigma_1\) at \(\phi_1 = 0\), and \(\phi_1 = \pi\) respectively.

The geometrical conclusion is that for each \(\phi_1 \in [0,\pi/2]\) range the points \(\Br\) trace out increasingly
larger spheres, and after that decreasing sized spheres until we get to a point again at \(\phi_1 = \pi\).

\end{itemize}

\section{Minkowski metric sphere}

\subsection{2D hyperbola}

Our 1-sphere equation was all the points on the curve

\begin{equation}\label{eqn:nvolume:340}
\begin{aligned}
x^2 + y^2 = r^2
\end{aligned}
\end{equation}

The hyperbolic equivalent to this is

\begin{equation}\label{eqn:nvolume:360}
\begin{aligned}
x^2 - y^2 = r^2
\end{aligned}
\end{equation}

Although this is not a closed curve like the circle.  To put this in a
more natural physical context, lets write (with \(c=1\))

\begin{equation}\label{eqn:nvolume:380}
\begin{aligned}
\Br = \gamma_0 t + \gamma_1 x
\end{aligned}
\end{equation}

So the equation of the 1-hyperboloid becomes

\begin{equation}\label{eqn:nvolume:400}
\begin{aligned}
\Br^2 = t^2 - x^2 = r^2
\end{aligned}
\end{equation}

We can parametrize this with complex angles \(i\phi\)

\begin{equation}\label{eqn:nvolume:420}
\begin{aligned}
\Br = r (\gamma_0 \cosh\phi + \gamma_1 \sinh\phi)
\end{aligned}
\end{equation}

This gives us

\begin{equation}\label{eqn:nvolume:440}
\begin{aligned}
\Br^2 = r^2 (\cosh^2\phi - \sinh^2\phi) = r^2
\end{aligned}
\end{equation}

as desired.  Like the circle, writing \(\gamma_{01} = \gamma_0 \wedge \gamma_1\), an exponential form also works nicely

\begin{equation}\label{eqn:nvolume:460}
\begin{aligned}
\Br = r \gamma_0 \exp(\gamma_{01} \phi)
\end{aligned}
\end{equation}

Here the square is

\begin{equation}\label{eqn:nvolume:480}
\begin{aligned}
\Br^2
&= r^2 \gamma_0 \exp(\gamma_{01} \phi) \gamma_0 \exp(\gamma_{01} \phi) \\
&= r^2 \exp(-\gamma_{01} \phi) (\gamma_0)^2 \exp(\gamma_{01} \phi) \\
&= r^2 \exp(-\gamma_{01} \phi) \exp(\gamma_{01} \phi) \\
&= r^2 \\
\end{aligned}
\end{equation}

Again as desired.

\subsection{3D hyperbola}

Unlike the circle, a pure hyperbolic parametrization does not work to construct a Minkowski
square signature.  Consider for example

\begin{equation}\label{eqn:nvolume:500}
\begin{aligned}
\Br
&= \cosh \phi \gamma_0 + \gamma_1 \sinh\phi \cosh \psi + \gamma_2 \sinh\phi \sinh\psi \\
\end{aligned}
\end{equation}

Squaring this we have
\begin{equation}\label{eqn:nvolume:520}
\begin{aligned}
\Br^2
&= \cosh^2 \phi - \sinh^2\phi (\cosh^2 \psi + \sinh^2\psi) \\
\end{aligned}
\end{equation}

We would get the desired result if we chop off the \(h\) in all the \(\psi\) hyperbolic functions.  This
shows that an appropriate parametrization is instead

\begin{equation}\label{eqn:nvolume:540}
\begin{aligned}
\Br
&= \cosh \phi \gamma_0 + \gamma_1 \sinh\phi \cos\psi + \gamma_2 \sinh\phi \sin\psi \\
\end{aligned}
\end{equation}

This now squares to 1.  To see how to extend this to higher dimensions (of which we only need one more)
we can factor out a \(\gamma_0\)

\begin{equation}\label{eqn:nvolume:560}
\begin{aligned}
\Br
&= \gamma_0 (\cosh \phi - \sinh\phi (\mathLabelBox{\sigma_1 \cos\psi + \sigma_2 \sin\psi}{spatial vector parametrization of circle}) ) \\
\end{aligned}
\end{equation}

Now to extend this to three dimensions we have just to substituted the spherical parametrization from \eqnref{eqn:nvolume:sphereVec}

\begin{equation}\label{eqn:nvolume:580}
\begin{aligned}
\Br
&= r \gamma_0 (\cosh \phi_0 - \sinh\phi_0 ( \cos\phi_1 \sigma_1 + \sin\phi_1 \sigma_2 \exp( \sigma_{23} \phi_2 )) ) \\
&= r (\gamma_0 \cosh \phi_0 + \sinh\phi_0 ( \cos\phi_1 \gamma_1 + \sin\phi_1 \gamma_2 \exp( \gamma_{32} \phi_2 )) ) \\
\end{aligned}
\end{equation}

\subsection{Summarizing the hyperbolic vector parametrization}

Our parametrization in two, three, and four dimensions, respectively, are
\begin{equation}\label{eqn:nvolume:600}
\begin{aligned}
\Br_2 &= r (\gamma_0 \cosh \phi_0 + \sinh\phi_0 \gamma_1) \\
\Br_3 &= r (\gamma_0 \cosh \phi_0 + \sinh\phi_0 \gamma_1 \exp( \gamma_{21} \phi_1 ) ) \\
\Br_4 &= r (\gamma_0 \cosh \phi_0 + \sinh\phi_0 ( \cos\phi_1 \gamma_1 + \sin\phi_1 \gamma_2 \exp( \gamma_{32} \phi_2 )) )
\end{aligned}
\end{equation}

\subsection{Volume elements}

What are our volume elements using this parametrization can be calculated as above.

\subsubsection{For one spatial dimension we have}

\begin{equation}\label{eqn:nvolume:620}
\begin{aligned}
dV_2 \gamma_0 \gamma_1
&=
\begin{vmatrix}
\cosh \phi_0 & \sinh\phi_0  \\
\sinh \phi_0 & \cosh\phi_0
\end{vmatrix} r dr d\phi_0 \\
&=
r dr d\phi_0 \\
\end{aligned}
\end{equation}

\subsubsection{For two spatial dimensions we have}

%\begin{align*}
%dV_3 \gamma_0 \gamma_1 \gamma_2
%&= \left(\PD{r}{\Br} \wedge \PD{\phi_0}{\Br} \wedge \PD{\phi_1}{\Br} \right) dr d\phi_0 d\phi_1 \\
%&\langle \\
%& (\gamma_0 \cosh \phi_0 + \sinh\phi_0 \gamma_1 \exp( \gamma_{21} \phi_1 ) ) \times \\
%& (\gamma_0 \sinh \phi_0 + \cosh\phi_0 \gamma_1 \exp( \gamma_{21} \phi_1 ) ) \times \\
%& (\gamma_0 \cosh \phi_0 + \sinh\phi_0 \gamma_1 \gamma_{21} \exp( \gamma_{21} \phi_1 ) )  \\
%\rangle_{3} \\
%& r^2 dr d\phi_0 d\phi_1
%\end{align*}
%
%That first product within the grade selection is
%\begin{align*}
%&(\gamma_0 \cosh \phi_0 + \sinh\phi_0 \gamma_1 \exp( \gamma_{21} \phi_1 ) )
%(\gamma_0 \sinh \phi_0 + \cosh\phi_0 \gamma_1 \exp( \gamma_{21} \phi_1 ) )  \\
%&= \gamma_{01} ( \cosh^2\phi_0 \exp( \gamma_{21} \phi_1)  -\sinh^2\phi_0 \exp( -\gamma_{21} \phi_1 ) ) \\
%&= \gamma_{01} \cos\phi_1
%%+\gamma_{0121} \sin\phi_1 \cosh^2 \phi_0
%+\gamma_{02} \sin\phi_1 \cosh^2 \phi_0
%%-\gamma_{0112} \sin\phi_1 \sinh^2 \phi_0
%+\gamma_{02} \sin\phi_1 \sinh^2 \phi_0 \\
%&= \gamma_{01} \cos\phi_1 +\gamma_{02} \sin\phi_1 (\cosh^2 \phi_0 + \sinh^2 \phi_0) \\
%\end{align*}
%
%and the third factor is
%\begin{align*}
%\gamma_0 \cosh \phi_0 + \sinh\phi_0 \gamma_{2} \exp( \gamma_{21} \phi_1 )
%\end{align*}
%
%So this last product is
%\begin{align*}
%&(\gamma_{01} \cos\phi_1 +\gamma_{02} \sin\phi_1 (\cosh^2 \phi_0 + \sinh^2 \phi_0) )
%(\gamma_0 \cosh \phi_0 + \sinh\phi_0 \gamma_{2} \exp( \gamma_{21} \phi_1 ) ) \\
%&=
%(\gamma_{01} \cos\phi_1 +\gamma_{02} \sin\phi_1 (\cosh^2 \phi_0 + \sinh^2 \phi_0) ) \sinh\phi_0 (\gamma_{2} \cos \phi_1 -\gamma_{1} \sin\phi_1 ) \\
%&=
% \gamma_{012} \cos^2\phi_1
%-\gamma_{021} \sin^2\phi_1 (\cosh^2 \phi_0 + \sinh^2 \phi_0) ) \sinh\phi_0  \\
%&=
% \gamma_{012} (\cos^2\phi_1 +\sin^2\phi_1 (\cosh^2 \phi_0 + \sinh^2 \phi_0) ) \sinh\phi_0 ) \\
%\end{align*}
%
%Note that above in many spots the grade three selection was used to discard terms that have no contribution.  The final result, without much
%potential more for simplification appears to be
%
%\begin{align*}
%dV_3 &= (\cos^2\phi_1 +\sin^2\phi_1 (\cosh^2 \phi_0 + \sinh^2 \phi_0) ) \sinh\phi_0 r^2 dr d\phi_0 d\phi_1
%\end{align*}
%
%I do not trust this result since I expected something simpler.  Let us try again

\begin{equation}\label{eqn:nvolume:640}
\begin{aligned}
\Br_3 &= r (\gamma_0 \cosh \phi_0 + \gamma_1 \sinh\phi_0 \cos\phi_1 + \gamma_2 \sinh\phi_0 \sin\phi_2 ) \\
\end{aligned}
\end{equation}

The derivatives are
\begin{equation}\label{eqn:nvolume:660}
\begin{aligned}
\PD{r}{\Br_3} &= \gamma_0 \cosh \phi_0 + \gamma_1 \sinh\phi_0 \cos\phi_1 + \gamma_2 \sinh\phi_0 \sin\phi_1  \\
\inv{r} \PD{\phi_0}{\Br_3} &= \gamma_0 \sinh \phi_0 + \gamma_1 \cosh\phi_0 \cos\phi_1 + \gamma_2 \cosh\phi_0 \sin\phi_1  \\
\inv{r} \PD{\phi_1}{\Br_3} &= -\gamma_1 \sinh\phi_0 \sin\phi_1 + \gamma_2 \sinh\phi_0 \cos\phi_1
\end{aligned}
\end{equation}

Or
\begin{equation}\label{eqn:nvolume:680}
\begin{aligned}
\PD{r}{\Br_3} &= \gamma_0 \cosh \phi_0 + \gamma_1 \sinh\phi_0 \exp (\gamma_{21} \phi_1 ) \\
\inv{r} \PD{\phi_0}{\Br_3} &= \gamma_0 \sinh \phi_0 + \cosh\phi_0 \exp (-\gamma_{21} \phi_1 ) \gamma_1 \\
\inv{r} \PD{\phi_1}{\Br_3} &= \sinh\phi_0 \gamma_2 \exp(\gamma_{21} \phi_1 )
\end{aligned}
\end{equation}

Multiplying this out, discarding non-grade three terms we have

\begin{equation}\label{eqn:nvolume:700}
\begin{aligned}
&(\gamma_{10} \sinh^2\phi_0 \exp (\gamma_{21} \phi_1 ) +\gamma_{01} \cosh^2 \phi_0 \exp (\gamma_{21} \phi_1 ) ) \sinh\phi_0 \gamma_2 \exp(\gamma_{21} \phi_1 ) \\
&=\gamma_{01} \exp (\gamma_{21} \phi_1 ) \sinh\phi_0 \exp(-\gamma_{21} \phi_1 ) \gamma_2 \\
&=\gamma_{01} \sinh\phi_0 \gamma_2 \\
\end{aligned}
\end{equation}

This gives us

\begin{equation}\label{eqn:nvolume:720}
\begin{aligned}
dV_3 &= r^2 \sinh\phi_0 dr d\phi_0 d\phi_1 \\
\end{aligned}
\end{equation}

\subsubsection{For three spatial dimensions we have}

\begin{equation}\label{eqn:nvolume:740}
\begin{aligned}
\Br_4 &= r (\gamma_0 \cosh \phi_0 + \sinh\phi_0 ( \cos\phi_1 \gamma_1 + \sin\phi_1 \gamma_2 \exp( \gamma_{32} \phi_2 )) )
\end{aligned}
\end{equation}

So our derivatives are

\begin{equation}\label{eqn:nvolume:760}
\begin{aligned}
\PD{r}{\Br_4} &= \gamma_0 \cosh \phi_0 + \sinh\phi_0 ( \cos\phi_1 \gamma_1 + \sin\phi_1 \gamma_2 \exp( \gamma_{32} \phi_2 )) \\
\inv{r}\PD{\phi_0}{\Br_4} &= \gamma_0 \sinh \phi_0 + \cosh\phi_0 ( \cos\phi_1 \gamma_1 + \sin\phi_1 \gamma_2 \exp( \gamma_{32} \phi_2 )) \\
\inv{r}\PD{\phi_1}{\Br_4} &= \sinh\phi_0 ( -\sin\phi_1 \gamma_1 + \cos\phi_1 \gamma_2 \exp( \gamma_{32} \phi_2 )) \\
\inv{r}\PD{\phi_2}{\Br_4} &= \sinh\phi_0 \sin\phi_1 \gamma_3 \exp( \gamma_{32} \phi_2 )
\end{aligned}
\end{equation}

In shorthand, writing \(C\) and \(S\) for the trig and hyperbolic functions as appropriate, we have

\begin{equation}\label{eqn:nvolume:780}
\begin{aligned}
\gamma_0 C_0 + S_0 C_1 \gamma_1 + S_0 S_1 \gamma_2 \exp( \gamma_{32} \phi_2 )  \\
\gamma_0 S_0 + C_0 C_1 \gamma_1 + C_0 S_1 \gamma_2 \exp( \gamma_{32} \phi_2 )  \\
 -S_0 S_1 \gamma_1 + S_0 C_1 \gamma_2 \exp( \gamma_{32} \phi_2 )  \\
S_0 S_1 \gamma_3 \exp( \gamma_{32} \phi_2 )
\end{aligned}
\end{equation}

Multiplying these out and dropping terms that will not contribute grade four bits is needed to calculate the volume element.
The full product for the first two derivatives is

\begin{equation}\label{eqn:nvolume:800}
\begin{aligned}
&\gamma_0 C_0 \gamma_0 S_0
+\gamma_0 C_0 C_0 C_1 \gamma_1
+\gamma_0 C_0 C_0 S_1 \gamma_2 \exp( \gamma_{32} \phi_2 ) \\
&+S_0 C_1 \gamma_1 \gamma_0 S_0
+S_0 C_1 \gamma_1 C_0 C_1 \gamma_1
+S_0 C_1 \gamma_1 C_0 S_1 \gamma_2 \exp( \gamma_{32} \phi_2 ) \\
&+S_0 S_1 \gamma_2 \exp( \gamma_{32} \phi_2 ) \gamma_0 S_0
+S_0 S_1 \gamma_2 \exp( \gamma_{32} \phi_2 ) C_0 C_1 \gamma_1
+S_0 S_1 \gamma_2 \exp( \gamma_{32} \phi_2 ) C_0 S_1 \gamma_2 \exp( \gamma_{32} \phi_2 ) \\
\end{aligned}
\end{equation}

We can discard the scalar terms:
\begin{equation}\label{eqn:nvolume:820}
\begin{aligned}
\gamma_0 C_0 \gamma_0 S_0 +S_0 C_1 \gamma_1 C_0 C_1 \gamma_1 +S_0 S_1 \gamma_2 \exp( \gamma_{32} \phi_2 ) C_0 S_1 \exp( -\gamma_{32} \phi_2 ) \gamma_2
\end{aligned}
\end{equation}

Two of these terms cancel out
\begin{equation}\label{eqn:nvolume:840}
\begin{aligned}
S_0 C_0 C_1 S_1 \gamma_{12} \exp( \gamma_{32} \phi_2 )
+S_0 C_0 C_1 S_1 \gamma_{21} \exp( \gamma_{32} \phi_2 )
\end{aligned}
\end{equation}

and we are left with two bivector contributors to the eventual four-pseudoscalar
\begin{equation}\label{eqn:nvolume:860}
\begin{aligned}
\gamma_{01} C_1
+\gamma_{02} S_1 \exp( \gamma_{32} \phi_2 )
\end{aligned}
\end{equation}

Multiplying out the last two derivatives we have
\begin{equation}\label{eqn:nvolume:880}
\begin{aligned}
-S_0^2 S_1^2 \gamma_{13} \exp( \gamma_{32} \phi_2 )
%+ S_0^2 C_1 S_1 \gamma_2 \exp( \gamma_{32} \phi_2 ) \exp( -\gamma_{32} \phi_2 ) \gamma_3
+ S_0^2 C_1 S_1 \gamma_{23}
\end{aligned}
\end{equation}

Almost there.  A final multiplication of these sets of products gives

\begin{equation}\label{eqn:nvolume:900}
\begin{aligned}
&-\gamma_{01} C_1 S_0^2 S_1^2 \gamma_{13} \exp( \gamma_{32} \phi_2 )
-\gamma_{02} S_1 \exp( \gamma_{32} \phi_2 ) S_0^2 S_1^2 \gamma_{13} \exp( \gamma_{32} \phi_2 ) \\
&+\gamma_{01} C_1 S_0^2 C_1 S_1 \gamma_{23}
+\gamma_{02} S_1 \exp( \gamma_{32} \phi_2 ) S_0^2 C_1 S_1 \gamma_{23} \\
&=
\gamma_{03} C_1 S_0^2 S_1^2 \exp( \gamma_{32} \phi_2 )
-\gamma_{02} \exp( \gamma_{32} \phi_2 ) S_0^2 S_1^3 \exp( -\gamma_{32} \phi_2 ) \gamma_{13} \\
&\qquad +\gamma_{0123} C_1^2 S_0^2 S_1
-\gamma_{03} \exp( \gamma_{32} \phi_2 ) S_0^2 C_1 S_1^2 \\
&=
\gamma_{03} C_1 S_0^2 S_1^2 \exp( \gamma_{32} \phi_2 )
+\gamma_{0123} S_0^2 S_1 (S_1^2 + C_1^2)
-\gamma_{03} \exp( \gamma_{32} \phi_2 ) S_0^2 C_1 S_1^2 \\
&=
\gamma_{03} C_1 S_1^2 (S_0^2 -S_0^2)\exp( \gamma_{32} \phi_2 )
+\gamma_{0123} S_0^2 S_1  \\
\end{aligned}
\end{equation}

Therefore our final result is

\begin{equation}\label{eqn:nvolume:920}
\begin{aligned}
dV_4 &= \sinh^2 \phi_0 \sin\phi_1 r^3 dr d\phi_0 d\phi_1 d\phi_2
\end{aligned}
\end{equation}

\section{Summary}

\subsection{Vector parametrization}

N-Spherical parametrization

\begin{equation}\label{eqn:nvolume:940}
\begin{aligned}
\Br_2 &= \sigma_1 \exp( \sigma_{12} \phi_1 ) \\
\Br_3 &= \cos\phi_1 \sigma_1 + \sin\phi_1 \sigma_2 \exp( \sigma_{23} \phi_2 ) \\
\Br_4 &= \sigma_1 \cos\phi_1 + \sigma_2 \sin\phi_1 (\cos\phi_2 + \sigma_{23} \sin\phi_2 \exp(\sigma_{34}\phi_3) )
\end{aligned}
\end{equation}

N-Hypersphere parametrization

\begin{equation}\label{eqn:nvolume:960}
\begin{aligned}
\Br_2 &= r (\gamma_0 \cosh \phi_0 + \sinh\phi_0 \gamma_1) \\
\Br_3 &= r (\gamma_0 \cosh \phi_0 + \sinh\phi_0 \gamma_1 \exp( \gamma_{21} \phi_1 ) ) \\
\Br_4 &= r (\gamma_0 \cosh \phi_0 + \sinh\phi_0 ( \cos\phi_1 \gamma_1 + \sin\phi_1 \gamma_2 \exp( \gamma_{32} \phi_2 )) )
\end{aligned}
\end{equation}

\subsection{Volume elements}

To summarize the mess of algebra we have shown that our hyperbolic volume elements are given by

\begin{equation}\label{eqn:nvolume:980}
\begin{aligned}
dV_2 &= \left(r dr\right) d\phi_0 \\
dV_3 &= \left(r^2 dr\right) \left(\sinh\phi_0 d\phi_0\right) d\phi_1 \\
dV_4 &= \left(r^3 dr\right) \left(\sinh^2 \phi_0 d\phi_0\right) \left(\sin\phi_1 d\phi_1\right) d\phi_2
\end{aligned}
\end{equation}

Compare this to the volume elements for the n-spheres

\begin{equation}\label{eqn:nvolume:1000}
\begin{aligned}
dV_2 &= \left(r dr\right) d\phi_1 \\
dV_3 &= \left(r^2 dr\right) \left(\sin\phi_1 d\phi_1\right) d\phi_2 \\
dV_4 &= \left(r^3 dr\right) \left(\sin^2\phi_1 d\phi_1\right) \left(\sin\phi_2 d\phi_2\right) d\phi_3
\end{aligned}
\end{equation}

Besides labeling variations the only difference in the form is a switch from trig to hyperbolic functions for the first angle (which has an implied
range difference as well).
