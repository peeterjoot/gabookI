%
% Copyright � 2012 Peeter Joot.  All Rights Reserved.
% Licenced as described in the file LICENSE under the root directory of this GIT repository.
%

%
%
%\usepackage{txfonts}

\chapter{Stokes Law revisited with algebraic enumeration of boundary}\label{chap:PJStokes2}
%\date{Sept 27, 2008.  stokesRevisited.tex}
\section{Algebraic description of oriented boundaries}

Having used pictorial methods to enumerate the bounding loop and area elements
\chapcite{PJStokes1}
in the previous derivation
of the vector and bivector forms of Stokes's, makes the application
of these formulas harder.  Here this will be revisited, with the aim of remedying this, as well as
obtaining a proof for the general case, which was not possible because of a lack of exactly this
algebraic formulation.

\subsection{Parallelogram parametrization}

\imageFigure{../../figures/gabook/parallelogram_parameterized}{Two variable parametrization of \R{n} parallelogram}{fig:parallelogram}{0.4}

An oriented curve around a parallelogram in \R{n} is illustrated in \cref{fig:parallelogram}.  We want to evaluate the line integral around this path

\begin{equation}\label{eqn:stokesR:lineprep}
\begin{aligned}
\ointclockwise \Bf \cdot d\Br
&=
\int du_1 \left.{\Bf \cdot \PD{u_1}{\Br} }\right\vert_{u_2(0)}^{u_2(1)}
-\int du_2 \left.{\Bf \cdot \PD{u_2}{\Br} }\right\vert_{u_1(0)}^{u_1(1)} \\
\end{aligned}
\end{equation}

Now, we can put this in a more symmetric form utilizing a reciprocal
frame to enumerate the alternation.  Write

\begin{equation}\label{eqn:stokesRevisited:22}
\begin{aligned}
\Br_{u_i} &= \PD{u_i}{\Br} \\
\Br^{u_j} \cdot \Br_{u_i} &= {\delta^j}_i \\
I &= \Br_{u_1} \wedge \Br_{u_2} \\
I \Br^{u_1} &=
I \cdot \Br^{u_1} =
% (\Br_{u_1} \wedge \Br_{u_2}) \cdot \Br^{u_1}
- \Br_{u_2} \\
I \Br^{u_2} &=
I \cdot \Br^{u_2} =
% (\Br_{u_1} \wedge \Br_{u_2}) \cdot \Br^{u_2}
 \Br_{u_1}.
\end{aligned}
\end{equation}

We do not care to actually calculate the reciprocal frame vectors.  They just work well to describe the alternation in terms of the pseudoscalar for the plane.

Substituting back into \eqnref{eqn:stokesR:lineprep} we have

\begin{equation}\label{eqn:stokesRevisited:42}
\begin{aligned}
\ointclockwise \Bf \cdot d\Br
&=
\int du_1 \left.{\Bf \cdot \left(I \Br^{u_2}\right) }\right\vert_{u_2(0)}^{u_2(1)}
+\int du_2 \left.{\Bf \cdot \left(I \Br^{u_1}\right) }\right\vert_{u_1(0)}^{u_1(1)} \\
\end{aligned}
\end{equation}

Or
\begin{equation}\label{eqn:stokesR:lineintegral}
\ointclockwise \Bf \cdot d\Br
=
\sum_{i} \int \frac{du_1 du_2}{du_i} \left.{\Bf \cdot \left(I \Br^{u_i}\right) }\right\vert_{u_i(0)}^{u_i(1)}
\end{equation}

This completes the goal of expressing the line integral in a fashion that does not require drawing any pictures,
and gives a hint about how to do the same for general \({\bigwedge}^k \Rm{n}\) case.

As before this can be written in terms of its integrals

\begin{equation}\label{eqn:stokesRevisited:62}
\begin{aligned}
\ointclockwise \Bf \cdot d\Br
&= \sum_{i, j \ne i}
\int_{u_j(0)}^{u_j(1)} du_j
\int_{u_i(0)}^{u_i(1)}
 \PD{u_i}{} {\Bf \cdot \left(I \Br^{u_i}\right)} du_i \\
&= \iint du_1 du_2 \sum \PD{u_i}{} {\Bf \cdot \left(I \Br^{u_i}\right)}
\end{aligned}
\end{equation}

Evaluating the derivatives to prove the Stokes/Green's result will be deferred for now (may instead proving
the general case once formulated).

\subsection{Parallelepiped parametrization}

\imageFigure{../../figures/gabook/parallelopiped_parameterized}{Three variable parametrization of \R{n} parallelepiped}{fig:parallelepiped}{0.4}

Next, lets evaluate the bivector area dot products, as in \cref{fig:parallelepiped}.

\begin{equation}\label{eqn:stokesRevisited:82}
\begin{aligned}
\oiintclockwise \BF \cdot d\BA
&= \iint du_2 du_1 \left.{F \cdot (\Br_{u_2} \wedge \Br_{u_1})}\right\vert_{u_3(1)}^{u_3(0)} \\
& +\iint du_3 du_1 \left.{F \cdot (\Br_{u_3} \wedge \Br_{u_1})}\right\vert_{u_2(0)}^{u_2(1)} \\
& +\iint du_3 du_2 \left.{F \cdot (\Br_{u_3} \wedge -\Br_{u_2})}\right\vert_{u_1(0)}^{u_1(1)} \\
\end{aligned}
\end{equation}

Again introducing reciprocal vectors to enumerate the alternation, but now write \(I\) as a pseudoscalar
for the parallelepiped subspace that the area bounds

\begin{equation}\label{eqn:stokesRevisited:102}
\begin{aligned}
I &= \Br_{u_1} \wedge \Br_{u_2} \wedge \Br_{u_3} \\
I \Br^{u_1} &= \Br_{u_2} \wedge \Br_{u_3} \\
I \Br^{u_2} &= -\Br_{u_1} \wedge \Br_{u_3} \\
I \Br^{u_3} &= \Br_{u_1} \wedge \Br_{u_2} \\
\end{aligned}
\end{equation}

Substituting we have a form almost identical to the line integral of \eqnref{eqn:stokesR:lineintegral}.

\begin{equation}\label{eqn:stokesRevisited:122}
\begin{aligned}
\oiintclockwise \BF \cdot d\BA
&= \sum \iint \frac{du_1 du_2 du_3}{du_i} \left.{F \cdot (I \Br^{u_i})}\right\vert_{u_i(0)}^{u_i(1)} \\
&= \iiint du_1 du_2 du_3 \sum \PD{u_i}{} F \cdot (I \Br^{u_i})
\end{aligned}
\end{equation}

\subsection{General case}

Having found that the line integral and oriented area integrals can be expressed uniformly in the same algebraic form, it
is reasonable to define an integral with such structure as a directed hypervolume boundary for any grade blade, and then verify that
this yields the expected generalized Stokes result that has been proven for only the vector and area cases.

Writing
\begin{equation}\label{eqn:stokesRevisited:142}
\begin{aligned}
F &\in {\bigwedge}^{k-1} \Rm{n} \\
d^k \Bx &= \PD{u_1}{\Br} \wedge \PD{u_2}{\Br} \wedge \cdots \wedge \PD{u_k}{\Br} du_1 du_2 \cdots du_k = I du_1 du_2 \cdots du_k \\
\end{aligned}
\end{equation}

We wish to prove the general Stokes equation for a hyper-parallelepiped volume

\begin{equation}\label{eqn:stokesR:genstokes}
\int_V (\grad \wedge F) \cdot d^{k}\Bx = \int_{\partial V} F \cdot d^{k-1}\Bx
\end{equation}

With the presumption that this will algebraically be identical to the line integral and area integral cases for vectors and bivectors respectively
we want to evaluate

\begin{equation}\label{eqn:stokesRevisited:162}
\begin{aligned}
\int_{\partial V} F \cdot d^{k-1}\Bx
&= \sum \int \frac{du_1 du_2 \cdots du_k}{du_i} \left.{F \cdot (I \Br^{u_i})}\right\vert_{u_i(0)}^{u_i(1)} \\
&= \int_V du_1 du_2 \cdots du_k \sum \PD{u_i}{} F \cdot (I \Br^{u_i}) \\
\end{aligned}
\end{equation}

\begin{equation}\label{eqn:stokesR:toprove}
\int_{\partial V} F \cdot d^{k-1}\Bx
= \int_V du_1 du_2 \cdots du_k \sum \PD{u_i}{F} \cdot (I \Br^{u_i}) +F \cdot \left( \PD{u_i}{} I \Br^{u_i} \right).
\end{equation}

The last term here sums to zero.  The messy long proof of this can be found at the end.  Assuming that proven this leaves us with the following identity

\begin{equation}\label{eqn:stokesR:alsoProve}
\int_{\partial V} F \cdot d^{k-1}\Bx
= \int_V du_1 du_2 \cdots du_k \sum \PD{u_i}{F} \cdot (I \Br^{u_i})
\end{equation}

We wish to show that this equals

\begin{equation*}
\int_V du_1 du_2 \cdots du_k (\grad \wedge F) \cdot I,
\end{equation*}

after which point we have both formulated algebraically the boundary integral, and proven the general \(k-1\text{blade}\) Stokes theorem of \eqnref{eqn:stokesR:genstokes}.

\subsection{Is a coordinate free proof possible?}

Note that
\begin{equation}\label{eqn:stokesRevisited:182}
\begin{aligned}
\PD{u_i}{F} \cdot (I \Br^{u_i})
&= \gpgradezero{\PD{u_i}{F} I \Br^{u_i}} \\
&= \gpgradezero{ \Br^{u_i} \PD{u_i}{F} I } \\
&= \left(\Br^{u_i} \wedge \PD{u_i}{F} \right) \cdot I \\
\end{aligned}
\end{equation}

Can the reduction of this wedge product to curl form be done without coordinates?  It would also be fairly easy to go in circles here since the reciprocal frame vectors can be calculated in terms of the pseudoscalar \(I\).

\subsection{Notation for coordinate expansion}

I did not have any luck finding a coordinate free way as outlined above to prove the general result.  The dumb brute force way is still possible though, expanding
both sides and comparing.

The following will be used in the sections below

\begin{equation}\label{eqn:stokesRevisited:202}
\begin{aligned}
\Br &= \gamma_j x^j \\
\Br_{u_i} &= \gamma_j \PD{u_i}{x^j} \\
I \Br^{u_i}
&= (-1)^{k-i} \Br_{u_1} \wedge \Br_{u_2} \wedge \cdots \widehat{\Br_{u_i}} \cdots \wedge \Br_{u_k} \\
\end{aligned}
\end{equation}

\begin{equation}\label{eqn:stokesR:dA}
I \Br^{u_i}
= (-1)^{k-i}
\gamma_{j_1} \wedge \cdots \gamma_{j_{k-1}}
%\mathLabelBox{\PD{u_1}{x^{j_1}} \cdots \PD{u_k}{x^{j_{k-1}}}}{\(\partial_{u_i}\) \,omitted}
\PD{u_1}{x^{j_1}} \cdots \widehat{\PD{u_i}{}} \cdots \PD{u_k}{x^{j_{k-1}}}
\end{equation}

Here the overhat is used to indicate omission.

\subsection{Expanding the curl dot by coordinates}

One half of the comparison will based on the expansion of \((\grad \wedge F ) \cdot I\).  We calculate

\begin{equation*}
F = \inv{(k-1)!} F_{j_1 j_2 \cdots j_{k-1}} \gamma^{j_1} \wedge \gamma^{j_2} \cdots \wedge \gamma^{j_{k-1}}
\end{equation*}
\begin{equation*}
I = \gamma_{m_1} \wedge \gamma_{m_2} \cdots \wedge \gamma_{m_{k}} \PD{u_1}{x^{m_1}} \PD{u_2}{x^{m_2}} \cdots \PD{u_k}{x^{m_k}}
\end{equation*}
\begin{equation*}
\grad = \gamma^{j_k} \PD{x^{j_k}}{}
\end{equation*}
\begin{equation}\label{eqn:stokesRevisited:222}
\begin{aligned}
\grad \wedge F
&= \inv{(k-1)!} \PD{x^{j_{k}}}{F_{j_1 j_2 \cdots j_{k-1}}} \gamma^{j_k} \wedge \gamma^{j_1} \wedge \gamma^{j_2} \cdots \wedge \gamma^{j_{k-1}} \\
&= \frac{(-1)^{k-1}}{(k-1)!} \PD{x^{j_{k}}}{F_{j_1 j_2 \cdots j_{k-1}}}
\gamma^{j_1} \wedge \gamma^{j_2} \cdots \wedge \gamma^{j_{k-1}} \wedge \gamma^{j_k}.
\end{aligned}
\end{equation}

Now, put this all together
\begin{equation}\label{eqn:stokesRevisited:242}
\begin{aligned}
(\grad \wedge F ) \cdot I
&= \frac{(-1)^{k-1}}{(k-1)!}
\left( \gamma^{j_1} \wedge \gamma^{j_2} \cdots \wedge \gamma^{j_{k}} \right) \cdot
\left( \gamma_{m_1} \wedge \gamma_{m_2} \cdots \wedge \gamma_{m_{k}} \right) \\
&\PD{x^{j_{k}}}{F_{j_1 j_2 \cdots j_{k-1}}}
\PD{u_1}{x^{m_1}} \PD{u_2}{x^{m_2}} \cdots \PD{u_k}{x^{m_k}} \\
&= \frac{(-1)^{k-1}}{(k-1)!}
{\delta^{j_k}}_{m_1}
{\delta^{j_{k-1}}}_{m_2}
\cdots
{\delta^{j_{1}}}_{m_k}
\epsilon^{m_1 m_2 \cdots m_k}
\PD{x^{j_{k}}}{F_{j_1 j_2 \cdots j_{k-1}}}
\PD{u_1}{x^{m_1}} \PD{u_2}{x^{m_2}} \cdots \PD{u_k}{x^{m_k}} \\
&= \frac{(-1)^{k-1}}{(k-1)!}
\epsilon^{m_1 m_2 \cdots m_k}
\PD{x^{m_{1}}}{F_{m_k m_{k-1} \cdots m_{2}}}
\PD{u_1}{x^{m_1}} \PD{u_2}{x^{m_2}} \cdots \PD{u_k}{x^{m_k}} \\
\end{aligned}
\end{equation}

Now, to reverse a \(k\) vector, or its corresponding antisymmetric tensor as above we have to perform the following number of swaps

\begin{equation*}
k-1 + k-2 + \cdots + 1 = k(k-1)/2
\end{equation*}

We can use this to tidy the indices above

\begin{equation*}
k-1 + (k-1)(k-2)/2
%= (2k - 2 + k^2 -3k + 2)/2
%= (k^2 -k)/2
= k(k -1 )/2,
\end{equation*}

and thus write

\begin{equation}\label{eqn:stokesR:curldot}
\begin{aligned}
(\grad \wedge F ) \cdot I
= \frac{(-1)^{k(k-1)/2}}{(k-1)!}
\epsilon^{m_1 m_2 \cdots m_k}
\PD{x^{m_{1}}}{F_{m_2 \cdots m_{k}}}
\PD{u_1}{x^{m_1}} \PD{u_2}{x^{m_2}} \cdots \PD{u_k}{x^{m_k}}
\end{aligned}
\end{equation}

\subsection{Expanding the boundary integral by coordinates}

The remainder of the proof is to verify that the expression \eqnref{eqn:stokesR:curldot} matches the differential form in \eqnref{eqn:stokesR:alsoProve}.

To do so we have to expand
\begin{equation*}
\PD{u_i}{F} \cdot (I \Br^{u_i})
\end{equation*}

\begin{equation}\label{eqn:stokesRevisited:262}
\begin{aligned}
\PD{u_i}{F}
&=
\PD{u_i}{x^{m_1}}
\PD{x^{m_1}}{}
\inv{(k-1)!} F_{m_2 m_3 \cdots m_{k}} \gamma^{m_2} \wedge \gamma^{m_2} \cdots \wedge \gamma^{m_{k}} \\
\end{aligned}
\end{equation}

Dotting this with \eqnref{eqn:stokesR:dA} we have
\begin{equation}\label{eqn:stokesRevisited:282}
\begin{aligned}
\PD{u_i}{F} \cdot (I \Br^{u_i})
&=
\frac{(-1)^{k-i}}{(k-1)!}
\PD{x^{m_1}}{F_{m_2 m_3 \cdots m_{k}}}
\left( \gamma^{m_2} \wedge \gamma^{m_2} \cdots \wedge \gamma^{m_{k}} \right) \cdot
\left( \gamma_{j_1} \wedge \cdots \gamma_{j_{k-1}} \right) \\
&\PD{u_1}{x^{j_1}} \cdots \widehat{\PD{u_i}{}} \cdots \PD{u_k}{x^{j_{k-1}}} \PD{u_i}{x^{m_1}} \\
&=
\frac{(-1)^{k-i}}{(k-1)!}
\PD{x^{m_1}}{F_{m_2 m_3 \cdots m_{k}}}
{\delta^{m_k}}_{j_1}
{\delta^{m_{k-1}}}_{j_2}
\cdots
{\delta^{m_{2}}}_{j_{k-1}}
\epsilon^{j_1 j_2 \cdots j_{k-1}} \\
&\PD{u_1}{x^{j_1}} \cdots \widehat{\PD{u_i}{}} \cdots \PD{u_k}{x^{j_{k-1}}} \PD{u_i}{x^{m_1}} \\
&=
\frac{(-1)^{k-i}}{(k-1)!}
\PD{x^{m_1}}{F_{m_2 m_3 \cdots m_{k}}}
\epsilon^{m_k m_{k-1} \cdots m_{2}}
\PD{u_1}{x^{m_k}} \cdots \widehat{\PD{u_i}{}} \cdots \PD{u_k}{x^{m_{2}}} \PD{u_i}{x^{m_1}} \\
&=
\frac{-(-1)^{i + k(k-1)/2}}{(k-1)!}
\PD{x^{m_1}}{F_{m_2 m_3 \cdots m_{k}}}
\epsilon^{m_2 \cdots m_{k-1}}
\PD{u_1}{x^{m_k}} \cdots \widehat{\PD{u_i}{}} \cdots \PD{u_k}{x^{m_{2}}} \PD{u_i}{x^{m_1}} \\
\end{aligned}
\end{equation}

After nicely arranging the \(m_i\) indices to match \eqnref{eqn:stokesR:curldot}, the partials do not match.  Perhaps about a change of variables:
\begin{equation}\label{eqn:stokesRevisited:302}
\begin{aligned}
m_1 &= n_i \\
m_k &= n_1 \\
m_{k-1} &= n_2 \\
\hdots & \hdots \\
m_2 &= n_k \\
\end{aligned}
\end{equation}

(with appropriate adjustments for i=1)

\begin{equation}\label{eqn:stokesRevisited:322}
\begin{aligned}
\PD{u_i}{F} \cdot (I \Br^{u_i})
&=
\frac{-(-1)^{i + k(k-1)/2}}{(k-1)!}
\PD{x^{n_i}}{F_{n_k n_{k-1} \cdots \widehat{n_i} \cdots n_{1}}}
\epsilon^{n_k \cdots \widehat{n_i} \cdots n_{1}}
\PD{u_1}{x^{n_1}} \PD{u_2}{x^{n_2}} \cdots \PD{u_k}{x^{n_k}} \\
&=
\frac{-(-1)^{i + k(k-1)/2}}{(k-1)!}
\PD{x^{n_i}}{F_{n_1 n_{2} \cdots \widehat{n_i} \cdots n_{k}}}
\epsilon^{n_1 \cdots \widehat{n_i} \cdots n_{k}}
\PD{u_1}{x^{n_1}} \PD{u_2}{x^{n_2}} \cdots \PD{u_k}{x^{n_k}} \\
\end{aligned}
\end{equation}

Here we have the product of two completely antisymmetric tensors, both with the same set of indices, so any alternation of those indices has
no effect.  The only sign changes come from the \(-(-1)^i\) coefficient.

To verify consistency with \eqnref{eqn:stokesR:curldot} it remains
to prove that within the sum the following two are identical

\begin{equation}\label{eqn:stokesR:p1}
\begin{aligned}
\epsilon^{m_1 m_2 \cdots m_k} \PD{x^{m_{1}}}{F_{m_2 \cdots m_{k}}}
\end{aligned}
\end{equation}
\begin{equation}\label{eqn:stokesR:p2}
\begin{aligned}
(-1)^{i+1}
\PD{x^{n_i}}{F_{n_1 n_{2} \cdots \widehat{n_i} \cdots n_{k}}}
\epsilon^{n_1 \cdots \widehat{n_i} \cdots n_{k}}.
\end{aligned}
\end{equation}

Examination and a bit of thought shows this to be the case.
FIXME: this statement is intuition based, and I am having trouble describing
exactly why I say so.  Revisit this later (for now I had rather spend the time
working with the result than to complete the last details of the proof).

\section{Summary}

Summarizing, a proof has been given for the general multivector Stokes equation, that provides
equivalent volume and boundary integral expressions

\begin{equation}\label{eqn:stokesR:genstokesSummary}
\int_V (\grad \wedge F) \cdot d^{k}\Bx = \int_{\partial V} F \cdot d^{k-1}\Bx.
\end{equation}

The proof of this result was restricted to a hyper-parallelepiped volume and its corresponding boundary.  Additional
arguments are required to extend this to arbitrary shapes.  That argument follows the loop integral case where cancellation
of oppositely oriented surfaces in adjacent volumes can be used to build up an arbitrary shape in terms of small
parallelepiped volumes.

In addition to the proof of this result, a specific algebraic (non-pictorial) meaning has been given to the boundary
differential form \(d^{k-1}\Bx\).  We have used the following notation

\begin{equation}
\Br_{u_i} = \PD{u_i}{\Br}
\end{equation}
\begin{equation}
\Br^{u_i} \cdot \Br_{u_j} = {\delta^i}_j
\end{equation}
\begin{equation}\label{eqn:stokesRevisited:342}
\begin{aligned}
I &= \Br_{u_1} \wedge \Br_{u_2} \wedge \cdots \wedge \Br_{u_k} \\
  &= \gamma_{m_1} \wedge \gamma_{m_2} \cdots \wedge \gamma_{m_{k}} \PD{u_1}{x^{m_1}} \PD{u_2}{x^{m_2}} \cdots \PD{u_k}{x^{m_k}} \\
\end{aligned}
\end{equation}
\begin{equation}\label{eqn:stokesRevisited:362}
\begin{aligned}
I \Br^{u_i} &= I \cdot \Br^{u_i} \\
&= (-1)^{k-i} \Br_{u_1} \wedge \Br_{u_2} \wedge \cdots \widehat{\Br_{u_i}} \cdots \wedge \Br_{u_k} \\
&= (-1)^{k-i}
\gamma_{j_1} \wedge \cdots \gamma_{j_{k-1}}
\PD{u_1}{x^{j_1}} \cdots \widehat{\PD{u_i}{}} \cdots \PD{u_k}{x^{j_{k-1}}}
\end{aligned}
\end{equation}

Here \(\Br\), as parametrized by \(u_i\) spans the hyper-parallelepiped, and
\(\Br(u_1, \cdots, u_i(1), \cdots)\), and \(\Br(u_1, \cdots, u_i(0), \cdots)\) represent boundaries of the surface with respect to
parameter \(u_i\).  Putting things together we have the following algebraic description of the boundary

\begin{equation}
\int_{\partial V} F \cdot d^{k-1}\Bx
= \sum \int du_1 \cdots \widehat{du_i} \cdots du_k \left.{F \cdot (I \Br^{u_i})}\right\vert_{u_i(0)}^{u_i(1)} \\
\end{equation}

Observe that we have a Jacobian like relationship above due to the alternation provided by the wedge product.  For this
reason it would make sense to introduce vector differentials

\begin{equation}
d\Bx_i = \PD{u_i}{\Br} du_i,
\end{equation}

in order to suppress the explicit parametrization.

\begin{equation}\label{eqn:stokesRevisited:382}
\begin{aligned}
\int_{\partial V} F \cdot d^{k-1}\Bx
&=
 \int \sum (-1)^{k-i}
\left.{F \cdot
\left( d\Bx_1 \wedge \cdots \widehat{d\Bx_i} \cdots \wedge d\Bx_k \right)
}\right\vert_{u_i(0)}^{u_i(1)} \\
&=
\int \sum (-1)^{k-i}
\left.{F \cdot
\left( d\Bx_1 \wedge \cdots \widehat{d\Bx_i} \cdots \wedge d\Bx_k \right)
}\right\vert_{\partial \Bx_i}
\end{aligned}
\end{equation}

In the LHS of \eqnref{eqn:stokesR:genstokesSummary} we also have a specific meaning for the k-vector volume element

\begin{equation}\label{eqn:stokesRevisited:402}
\begin{aligned}
d^{k}\Bx
&= I du_1 du_2 \cdots du_k \\
&= d\Bx_1 \wedge d\Bx_2 \cdots \wedge d\Bx_k
\end{aligned}
\end{equation}

Also notable for the volume integral is its tensor formulation, where we have the volume Jacobian determinant explicitly

\begin{equation}
\int_V (\grad \wedge F) \cdot d^{k}\Bx
= \int_V
\frac{(-1)^{k(k-1)/2}}{(k-1)!}
\PD{x^{m_{1}}}{F_{m_2 \cdots m_{k}}}
\frac{\partial( x^{m_1}, \cdots, x^{m_k} )}{\partial( u_1, \cdots, u_k )} du_1 \cdots du_k
\end{equation}

The other interesting thing worth noting is the reciprocal expression for curl projected onto the integration subspace

\begin{equation}
(\grad \wedge F) \cdot I = \left( \sum_i \Br^{u_i} \wedge \PD{u_i}{F} \right) \cdot I
\end{equation}

I was not able to use this, but having mostly completed the proof, this is proved as a side effect.  Here mostly means that the
unsatisfactory treatments (really handwaving) marked with FIXMEs should be revisited to consider this multivector form of Stokes theorem fully proved here.

\section{Messy proof of zero sum}

Sum of \eqnref{eqn:stokesR:toprove} requires followup.

Here is the deferred proof that the sum of the differentials of the area elements are zero

\begin{equation}\label{eqn:stokesRevisited:422}
\begin{aligned}
\int du_1 du_2 \cdots du_k \sum F \cdot \left( \PD{u_i}{} I \Br^{u_i} \right).
\end{aligned}
\end{equation}

Although not elegant, the partials here can be expanded by coordinates as done in the previous line and area proofs.

We want to prove that
\begin{equation}
\sum (-1)^{k-i} \gamma_{j_1} \wedge \cdots \gamma_{j_{k-1}} \PD{u_i}{}
\PD{u_1}{x^{j_1}} \cdots \widehat{\PD{u_i}{}} \cdots \PD{u_k}{x^{j_{k-1}}}
= 0
\end{equation}

as was done previously in the vector and bivector cases.  Pick as an example the \(i=3\) case, and assume that \(k > 2\) since
the two simpler cases have been proven explicitly.  For that \(i\), we have the following terms
\begin{equation}\label{eqn:stokesRevisited:442}
\begin{aligned}
\sum (-1)^{k-3} \gamma_{j_1} \wedge \cdots \gamma_{j_{k-1}}
&\left(\PDD{u_3}{u_1}{x^{j_1}} \PD{u_2}{x^{j_{2}}} \PD{u_4}{x^{j_3}} \cdots \PD{u_k}{x^{j_{k-1}}}\right. \\
&+\PDD{u_3}{u_2}{x^{j_2}} \PD{u_1}{x^{j_1}} \PD{u_4}{x^{j_3}} \cdots \PD{u_k}{x^{j_{k-1}}} \\
&+\PDD{u_3}{u_4}{x^{j_3}} \PD{u_1}{x^{j_1}} \PD{u_2}{x^{j_2}} \cdots \PD{u_k}{x^{j_{k-1}}} \\
&+\cdots \\
&+\left.\PDD{u_3}{u_k}{x^{j_k}} \PD{u_1}{x^{j_1}} \PD{u_2}{x^{j_2}} \cdots \PD{u_{k-1}}{x^{j_{k-2}}}\right)
\end{aligned}
\end{equation}

Picking any mixed partial term we expect cancellation with the opposing mixed partial.  Two representative values of \(i\) should be sufficient to see that the sum
is zero.  First pick \(i=1\), so that \((-1)^{k-3} = (-1)^{k-1}\), and look at the
matching partial for the \(\PDD{u_1}{u_3}{}\) term above

\begin{equation}\label{eqn:stokesRevisited:462}
\begin{aligned}
\sum (-1)^{k-1} \gamma_{j_1} \wedge \cdots \gamma_{j_{k-1}}
&\left(\PDD{u_1}{u_2}{x^{j_1}} \PD{u_3}{x^{j_{2}}} \PD{u_4}{x^{j_3}} \cdots \PD{u_k}{x^{j_{k-1}}}\right. \\
&+\PDD{u_1}{u_3}{x^{j_2}} \PD{u_2}{x^{j_1}} \PD{u_4}{x^{j_3}} \cdots \PD{u_k}{x^{j_{k-1}}} \\
&+\cdots \\
&+\left.\PDD{u_1}{u_k}{x^{j_k}} \PD{u_2}{x^{j_1}} \PD{u_3}{x^{j_2}} \cdots \PD{u_{k-1}}{x^{j_{k-2}}}\right).
\end{aligned}
\end{equation}

Swapping dummy indices \(j_1\) and \(j_2\) here one can see that the
\(\PDD{u_1}{u_3}{}\) and \(\PDD{u_3}{u_1}{}\) terms cancel.

Now pick \(i=2\), so
that \((-1)^{k-1} = -(-1)^{k-2}\), and look at the
matching partial for the \(\PDD{u_1}{u_2}{}\) term above.

\begin{equation}\label{eqn:stokesRevisited:482}
\begin{aligned}
\sum (-1)^{k-2} \gamma_{j_1} \wedge \cdots \gamma_{j_{k-1}}
&\left(\PDD{u_2}{u_1}{x^{j_1}} \PD{u_3}{x^{j_{2}}} \PD{u_4}{x^{j_3}} \cdots \PD{u_k}{x^{j_{k-1}}}\right. \\
&+\cdots \\
&+\left.\PDD{u_2}{u_k}{x^{j_k}} \PD{u_1}{x^{j_1}} \PD{u_3}{x^{j_2}} \cdots \PD{u_{k-1}}{x^{j_{k-2}}}\right).
\end{aligned}
\end{equation}

No swap of indices is required and we see again that the mixed partials cancel.  Now this is perhaps a slightly lazy proof, but working with
indices in the abstract without assigning specific numbers gets confusing.  It is clear to me that the end result will be a zero sum for this term.

FIXME: A cleanup of this proof should be possible to eliminate the special case comparisons above.  The tough part is simply writing all the terms
in a manipulatable fashion.  Then proceed to split the sum into terms that differ by even and odd separation of indices.  Summing over indices greater and indices lesser, then swapping indices as appropriate should complete the proof.
Alternatively, perhaps I will figure out a clever way later to demonstrate this more directly without resorting to this messy coordinate expansion.
