%
% Copyright � 2012 Peeter Joot.  All Rights Reserved.
% Licenced as described in the file LICENSE under the root directory of this GIT repository.
%

%
%
%\input{../peeter_prologue_print.tex}
%\input{../peeter_prologue_widescreen.tex}

%\chapter{Derivation of the spherical polar Laplacian}
\index{Laplacian!spherical polar}
\label{chap:sphericalPolarLaplacian}

%\blogpage{http://sites.google.com/site/peeterjoot/math2010/sphericalPolarLaplacian.pdf}
%\date{Oct 20, 2010}
%\revisionInfo{sphericalPolarLaplacian.tex}

%\beginArtWithToc
%\beginArtNoToc

\section{Motivation}

In \chapcite{polarGradAndLaplacian} was a Geometric Algebra derivation of the 2D polar Laplacian by squaring the gradient.  In \chapcite{sphericalPolarUnit} was a factorization of the spherical polar unit vectors in a tidy compact form.  Here both these ideas are utilized to derive the spherical polar form for the Laplacian, an operation that is strictly algebraic (squaring the gradient) provided we operate on the unit vectors correctly.

\section{Our rotation multivector}

Our starting point is a pair of rotations.  We rotate first in the \(x,y\) plane by \(\phi\)

\begin{subequations}
\label{eqn:sphericalPolarLaplacian:101}
\begin{equation}\label{eqn:sphericalPolarLaplacian:421}
\begin{aligned}
\Bx &\rightarrow \Bx' = \tilde{R_\phi} \Bx R_\phi \\
i &\equiv \Be_1 \Be_2 \\
R_\phi &= e^{i \phi/2}
\end{aligned}
\end{equation}
\end{subequations}

Then apply a rotation in the \(\Be_3 \wedge (\tilde{R_\phi} \Be_1 R_\phi) = \tilde{R_\phi} \Be_3 \Be_1 R_\phi\) plane

\begin{subequations}
\label{eqn:sphericalPolarLaplacian:102}
\begin{equation}\label{eqn:sphericalPolarLaplacian:441}
\begin{aligned}
\Bx' &\rightarrow \Bx'' = \tilde{R_\theta} \Bx' R_\theta \\
R_\theta &= e^{ \tilde{R_\phi} \Be_3 \Be_1 R_\phi \theta/2 } = \tilde{R_\phi} e^{ \Be_3 \Be_1 \theta/2 } R_\phi
\end{aligned}
\end{equation}
\end{subequations}

The composition of rotations now gives us
%
\begin{equation}\label{eqn:sphericalPolarLaplacian:461}
\begin{aligned}
\Bx
&\rightarrow \Bx'' = \tilde{R_\theta} \tilde{R_\phi} \Bx R_\phi R_\theta = \tilde{R} \Bx R \\
R &= R_\phi R_\theta = e^{ \Be_3 \Be_1 \theta/2 } e^{ \Be_1 \Be_2 \phi/2 }.
\end{aligned}
\end{equation}
%
\section{Expressions for the unit vectors}

The unit vectors in the rotated frame can now be calculated.  With \(I = \Be_1 \Be_2 \Be_3\) we can calculate

\begin{subequations}
\label{eqn:sphericalPolarLaplacian:201}
\begin{equation}\label{eqn:sphericalPolarLaplacian:481}
\begin{aligned}
\phicap &= \tilde{R} \Be_2 R  \\
\rcap &= \tilde{R} \Be_3 R  \\
\thetacap &= \tilde{R} \Be_1 R
\end{aligned}
\end{equation}
\end{subequations}

Performing these we get
%
\begin{equation}\label{eqn:sphericalPolarLaplacian:501}
\begin{aligned}
\phicap
&= e^{ -\Be_1 \Be_2 \phi/2 } e^{ -\Be_3 \Be_1 \theta/2 } \Be_2 e^{ \Be_3 \Be_1 \theta/2 } e^{ \Be_1 \Be_2 \phi/2 } \\
&= \Be_2 e^{ i \phi },
\end{aligned}
\end{equation}
%
and
\begin{equation}\label{eqn:sphericalPolarLaplacian:521}
\begin{aligned}
\rcap
&= e^{ -\Be_1 \Be_2 \phi/2 } e^{ -\Be_3 \Be_1 \theta/2 } \Be_3 e^{ \Be_3 \Be_1 \theta/2 } e^{ \Be_1 \Be_2 \phi/2 } \\
&= e^{ -\Be_1 \Be_2 \phi/2 } (\Be_3 \cos\theta + \Be_1 \sin\theta ) e^{ \Be_1 \Be_2 \phi/2 } \\
&= \Be_3 \cos\theta +\Be_1 \sin\theta e^{ \Be_1 \Be_2 \phi } \\
&= \Be_3 (\cos\theta + \Be_3 \Be_1 \sin\theta e^{ \Be_1 \Be_2 \phi } ) \\
&= \Be_3 e^{I \phicap \theta},
\end{aligned}
\end{equation}
%
and
%
\begin{equation}\label{eqn:sphericalPolarLaplacian:541}
\begin{aligned}
\thetacap
&= e^{ -\Be_1 \Be_2 \phi/2 } e^{ -\Be_3 \Be_1 \theta/2 } \Be_1 e^{ \Be_3 \Be_1 \theta/2 } e^{ \Be_1 \Be_2 \phi/2 } \\
&= e^{ -\Be_1 \Be_2 \phi/2 } ( \Be_1 \cos\theta - \Be_3 \sin\theta ) e^{ \Be_1 \Be_2 \phi/2 } \\
&= \Be_1 \cos\theta e^{ \Be_1 \Be_2 \phi/2 } - \Be_3 \sin\theta \\
&= i \phicap \cos\theta - \Be_3 \sin\theta \\
&= i \phicap (\cos\theta + \phicap i \Be_3 \sin\theta ) \\
&= i \phicap e^{I \phicap \theta}.
\end{aligned}
\end{equation}
%
Summarizing these are

\begin{subequations}
\label{eqn:sphericalPolarLaplacian:205}
\begin{equation}\label{eqn:sphericalPolarLaplacian:561}
\begin{aligned}
\phicap &= \Be_2 e^{ i \phi } \\
\rcap &= \Be_3 e^{I \phicap \theta} \\
\thetacap &= i \phicap e^{I \phicap \theta}.
\end{aligned}
\end{equation}
\end{subequations}

\section{Derivatives of the unit vectors}

We will need the partials.  Most of these can be computed from \eqnref{eqn:sphericalPolarLaplacian:205} by inspection, and are

\begin{subequations}
\label{eqn:sphericalPolarLaplacian:206}
\begin{equation}\label{eqn:sphericalPolarLaplacian:581}
\begin{aligned}
\partial_r \phicap &= 0 \\
\partial_r \rcap &= 0 \\
\partial_r \thetacap &= 0 \\
\partial_\theta \phicap &= 0 \\
\partial_\theta \rcap &= \rcap I \phicap \\
\partial_\theta \thetacap &= \thetacap I \phicap \\
\partial_\phi \phicap &= \phicap i \\
%\partial_\phi \rcap &= \Be_3 I \phicap i \sin\theta = \phicap \sin\theta \\
\partial_\phi \rcap &= \phicap \sin\theta \\
\partial_\phi \thetacap &= \phicap \cos\theta
\end{aligned}
\end{equation}
\end{subequations}

\section{Expanding the Laplacian}

We note that the line element is \(ds = dr + r d\theta + r\sin\theta d\phi\), so our gradient in spherical coordinates is
%
\begin{equation}\label{eqn:sphericalPolarLaplacian:300}
\begin{aligned}
\spacegrad &= \rcap \partial_r + \frac{\thetacap}{r} \partial_\theta + \frac{\phicap}{r\sin\theta} \partial_\phi.
\end{aligned}
\end{equation}
%
We can now evaluate the Laplacian
%
\begin{equation}\label{eqn:sphericalPolarLaplacian:310}
\begin{aligned}
\spacegrad^2 &=
\left( \rcap \partial_r + \frac{\thetacap}{r} \partial_\theta + \frac{\phicap}{r\sin\theta} \partial_\phi \right) \cdot
\left( \rcap \partial_r + \frac{\thetacap}{r} \partial_\theta + \frac{\phicap}{r\sin\theta} \partial_\phi \right).
\end{aligned}
\end{equation}
%
Evaluating these one set at a time we have
\begin{equation}\label{eqn:sphericalPolarLaplacian:601}
\begin{aligned}
\rcap \partial_r \cdot \left( \rcap \partial_r + \frac{\thetacap}{r} \partial_\theta + \frac{\phicap}{r\sin\theta} \partial_\phi \right) &= \partial_{rr},
\end{aligned}
\end{equation}
%
and
\begin{equation}\label{eqn:sphericalPolarLaplacian:621}
\begin{aligned}
\inv{r} \thetacap \partial_\theta \cdot \left( \rcap \partial_r + \frac{\thetacap}{r} \partial_\theta + \frac{\phicap}{r\sin\theta} \partial_\phi \right)
&=
\inv{r} \gpgradezero{
\thetacap \left(
\rcap I \phicap \partial_r + \rcap \partial_{\theta r}
+ \frac{\thetacap}{r} \partial_{\theta\theta} + \inv{r} \thetacap I \phicap \partial_\theta
+ \phicap \partial_\theta \inv{r\sin\theta} \partial_\phi
\right)
} \\
&=
\inv{r} \partial_r
+\inv{r^2} \partial_{\theta\theta},
\end{aligned}
\end{equation}
%
and
%
\begin{equation}\label{eqn:sphericalPolarLaplacian:641}
\begin{aligned}
\frac{\phicap}{r\sin\theta} \partial_\phi &\cdot
\left( \rcap \partial_r + \frac{\thetacap}{r} \partial_\theta + \frac{\phicap}{r\sin\theta} \partial_\phi \right) \\
&=
\frac{1}{r\sin\theta}
\gpgradezero{
\phicap
\left(
%\partial_\phi \rcap \partial_r
%+ \rcap \partial_{\phi r}
%+ \partial_\phi \thetacap \frac{1}{r} \partial_\theta
%+ \frac{\thetacap}{r} \partial_{\phi \theta }
%+ \partial_\phi \phicap \frac{1}{r\sin\theta} \partial_\phi
%+ \phicap \frac{1}{r\sin\theta} \partial_{\phi \phi }
\phicap \sin\theta \partial_r
+ \rcap \partial_{\phi r}
+ \phicap \cos\theta \frac{1}{r} \partial_\theta
+ \frac{\thetacap}{r} \partial_{\phi \theta }
+ \phicap i \frac{1}{r\sin\theta} \partial_\phi
+ \phicap \frac{1}{r\sin\theta} \partial_{\phi \phi }
\right)
} \\
&=
\inv{r} \partial_r
+ \frac{\cot\theta}{r^2}\partial_\theta
+ \inv{r^2 \sin^2\theta} \partial_{\phi\phi}
\end{aligned}
\end{equation}
%
Summing these we have
%
\begin{equation}\label{eqn:sphericalPolarLaplacian:400}
\begin{aligned}
\spacegrad^2 &=
\partial_{rr}
+ \frac{2}{r} \partial_r
+\inv{r^2} \partial_{\theta\theta}
+ \frac{\cot\theta}{r^2}\partial_\theta
+ \inv{r^2 \sin^2\theta} \partial_{\phi\phi}
\end{aligned}
\end{equation}
%
This is often written with a chain rule trick to consolidate the \(r\) and \(\theta\) partials
%
\begin{equation}\label{eqn:sphericalPolarLaplacian:401}
\begin{aligned}
\spacegrad^2 \Psi &=
%\inv{r}\frac{\partial^2 (r \Psi)}{\partial r^2}
%+ \inv{r^2 \sin\theta} \PD{\theta}{} \left( \sin\theta \PD{\theta}{\Psi} \right)
%+ \inv{r^2 \sin^2\theta} \frac{\partial^2 \Psi}{\partial \psi^2}
\inv{r} \partial_{rr} (r \Psi)
+ \inv{r^2 \sin\theta} \partial_\theta \left( \sin\theta \partial_\theta \Psi \right)
+ \inv{r^2 \sin^2\theta} \partial_{\psi\psi} \Psi
\end{aligned}
\end{equation}
%
It is simple to verify that this is identical to \eqnref{eqn:sphericalPolarLaplacian:400}.

%\EndArticle
