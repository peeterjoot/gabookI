%
% Copyright � 2016 Peeter Joot.  All Rights Reserved.
% Licenced as described in the file LICENSE under the root directory of this GIT repository.
%
%{
%\input{../blogpost.tex}
%\renewcommand{\basename}{biotSavartGreens}
%%\renewcommand{\dirname}{notes/phy1520/}
%\renewcommand{\dirname}{notes/ece1228-electromagnetic-theory/}
%%\newcommand{\dateintitle}{}
%%\newcommand{\keywords}{}
%
%\input{../peeter_prologue_print2.tex}
%
%\usepackage{peeters_layout_exercise}
%\usepackage{peeters_braket}
%\usepackage{peeters_figures}
%\usepackage{siunitx}
%
%\beginArtNoToc
%
%\generatetitle{Green's function inversion of magnetostatic equation}
%\chapter{Green's function inversion of magnetostatic equation}
%\label{chap:biotSavartGreens}
% \citep{sakurai2014modern} pr X.Y
% \citep{pozar2009microwave}
% \citep{qftLectureNotes}
% \citep{doran2003gap}
% \citep{jackson1975cew}
% \citep{griffiths1999introduction}

\makeexample{Magnetostatics.}{example:biotSavartGreens:1}{

The magnetostatics equation in linear media has the Geometric Algebra form
%A previous example of inverting a gradient equation was the electrostatics equation.  We can do the same for the magnetostatics equation, which has the following Geometric Algebra form in linear media

\begin{dmath}\label{eqn:biotSavartGreens:20}
\spacegrad I \BB = - \mu \BJ.
\end{dmath}

The Green's inversion of this is
\begin{dmath}\label{eqn:biotSavartGreens:40}
I \BB(\Bx)
= \int_V dV' G(\Bx, \Bx') \spacegrad' I \BB(\Bx')
= \int_V dV' G(\Bx, \Bx') (-\mu \BJ(\Bx'))
= \inv{4\pi} \int_V dV' \frac{\Bx - \Bx'}{ \Abs{\Bx - \Bx'}^3 } (-\mu \BJ(\Bx')).
\end{dmath}

We expect the LHS to be a bivector, so the scalar component of this should be zero.  That can be demonstrated with some of the usual trickery
\begin{dmath}\label{eqn:biotSavartGreens:60}
-\frac{\mu}{4\pi} \int_V dV' \frac{\Bx - \Bx'}{ \Abs{\Bx - \Bx'}^3 } \cdot \BJ(\Bx')
= \frac{\mu}{4\pi} \int_V dV' \lr{ \spacegrad \inv{ \Abs{\Bx - \Bx'} }} \cdot \BJ(\Bx')
= -\frac{\mu}{4\pi} \int_V dV' \lr{ \spacegrad' \inv{ \Abs{\Bx - \Bx'} }} \cdot \BJ(\Bx')
= -\frac{\mu}{4\pi} \int_V dV' \lr{
\spacegrad' \cdot \frac{\BJ(\Bx')}{ \Abs{\Bx - \Bx'} }
-
\frac{\spacegrad' \cdot \BJ(\Bx')}{ \Abs{\Bx - \Bx'} }
}.
\end{dmath}

The current \( \BJ \) is not unconstrained.  This can be seen by premultiplying \cref{eqn:biotSavartGreens:20} by the gradient

\begin{dmath}\label{eqn:biotSavartGreens:80}
\spacegrad^2 I \BB = -\mu \spacegrad \BJ.
\end{dmath}

On the LHS we have a bivector so must have \( \spacegrad \BJ = \spacegrad \wedge \BJ \), or \( \spacegrad \cdot \BJ = 0 \).  This kills the \( \spacegrad' \cdot \BJ(\Bx') \) integrand numerator in \cref{eqn:biotSavartGreens:60}, leaving

\begin{dmath}\label{eqn:biotSavartGreens:100}
-\frac{\mu}{4\pi} \int_V dV' \frac{\Bx - \Bx'}{ \Abs{\Bx - \Bx'}^3 } \cdot \BJ(\Bx')
= -\frac{\mu}{4\pi} \int_V dV' \spacegrad' \cdot \frac{\BJ(\Bx')}{ \Abs{\Bx - \Bx'} }
= -\frac{\mu}{4\pi} \int_{\partial V} dA' \ncap \cdot \frac{\BJ(\Bx')}{ \Abs{\Bx - \Bx'} }.
\end{dmath}

This shows that the scalar part of the equation is zero, provided the normal component of \( \BJ/\Abs{\Bx - \Bx'} \) vanishes on the boundary of the infinite sphere.  This leaves the Biot-Savart law as a bivector equation

\begin{dmath}\label{eqn:biotSavartGreens:120}
I \BB(\Bx)
= \frac{\mu}{4\pi} \int_V dV' \BJ(\Bx') \wedge \frac{\Bx - \Bx'}{ \Abs{\Bx - \Bx'}^3 }.
\end{dmath}

Observe that the traditional vector form of the Biot-Savart law can be obtained by premultiplying both sides with \( -I \), leaving

\begin{dmath}\label{eqn:biotSavartGreens:140}
\BB(\Bx)
= \frac{\mu}{4\pi} \int_V dV' \BJ(\Bx') \cross \frac{\Bx - \Bx'}{ \Abs{\Bx - \Bx'}^3 }.
\end{dmath}

This checks against a trusted source such as \citep{griffiths1999introduction} (eq. 5.39).
} % example

%}
%\EndArticle
