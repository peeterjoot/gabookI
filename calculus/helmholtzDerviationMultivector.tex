%
% Copyright � 2016 Peeter Joot.  All Rights Reserved.
% Licenced as described in the file LICENSE under the root directory of this GIT repository.
%
%{
%\input{../blogpost.tex}
%\renewcommand{\basename}{helmholtzDerviationMultivector}
%%\renewcommand{\dirname}{notes/phy1520/}
%\renewcommand{\dirname}{notes/ece1228-electromagnetic-theory/}
%%\newcommand{\dateintitle}{}
%%\newcommand{\keywords}{}
%
%\input{../peeter_prologue_print2.tex}
%
%\usepackage{peeters_layout_exercise}
%\usepackage{peeters_braket}
%\usepackage{peeters_figures}
%\usepackage{siunitx}
%
%\beginArtNoToc
%
%\generatetitle{Helmholtz theorem}
%\chapter{Helmholtz theorem}
%\label{chap:helmholtzDerviationMultivector}
% \citep{sakurai2014modern} pr X.Y
% \citep{pozar2009microwave}
% \citep{qftLectureNotes}
% \citep{doran2003gap}
% \citep{jackson1975cew}
% \citep{griffiths1999introduction}

%\section{Appendix IV.  2nd Geometric Algebra solution to problem 5.}

%This is a problem from ece1228.  I attempted solutions in a number of ways.  One using Geometric Algebra, one devoid of that algebra, and then this method, which combined aspects of both.  Of the three methods I tried to obtain this result, this is the most compact and elegant.  It does however, require a fair bit of Geometric Algebra knowledge, including the Fundamental Theorem of Geometric Calculus, as detailed in \citep{doran2003gap}, \citep{sobczyk2011fundamental} and \citep{aMacdonaldVAGC}.

\makeproblem{Helmholtz theorem}{problem:helmholtzTakeIII}{
Prove the first Helmholtz's theorem, i.e. if vector \(\BM\) is defined by its divergence

\begin{dmath}\label{eqn:helmholtzDerviationMultivector:20}
\spacegrad \cdot \BM = s
\end{dmath}

and its curl
\begin{dmath}\label{eqn:helmholtzDerviationMultivector:40}
\spacegrad \cross \BM = \BC
\end{dmath}

within a region and its normal component \( \BM_{\txtn} \) over the boundary, then \( \BM \) is
uniquely specified.

%Note: Assume there is a vector \( \BN \) with its divergence and curl equal to \( s \) and \( \BC \) respectively, then show that \( \BM = \BN \) .
} % makeproblem

\makeanswer{problem:helmholtzTakeIII}{
%%
%%XX
%%
%%I attempted this problem in two different ways.  My first approach assembled the divergence and curl relations above into a single (Geometric Algebra) multivector gradient equation and applied the vector valued Green's function for the gradient to invert that equation.  That approach logically led from the differential equation for \( \BM \) to the solution for \( \BM \) in terms of \( s \) and \( \BC \).  However, this strategy introduced some complexities
%%that make me doubt the correctness of the associated boundary analysis.
%%
%%Even if the details of the boundary handling in my multivector approach is not correct, I thought that approach was interesting enough to share, and have placed it in an appendix to this problem set.  It is accompanied with a primer on Geometric Algebra that is hopefully enough to allow the reader to grasp the basic ideas of the approach, but is probably not sufficient to understand all the details without further study.
%%
%%The answer obtained in that first attempt at this problem, when \( \Abs{\BM}/r^2 \), and \( \Abs{\BC}/r \) both vanish on an infinite sphere, is that the field has a unique value determined by \( s \) and \( \BC \), namely
%
%\begin{dmath}\label{eqn:helmholtzDerviationMultivector:60}
%\BM =
%-\spacegrad \int_V dV' \frac{ s(\Bx')}{ 4 \pi \Abs{\Bx - \Bx'} }
%+\spacegrad \cross \int_V dV' \frac{ \BC(\Bx') }{ 4 \pi \Abs{\Bx - \Bx'} }.
%\end{dmath}
%
%It's possible to work backwards from this result to obtain second order gradient terms applied to \( \BM(\Bx')/\Abs{\Bx - \Bx'} \) .  This suggests that a Laplacian (i.e. scalar) representation of the delta function may be a superior way to tackle this problem, perhaps also yielding a simpler result for the boundary term.  This is in fact the case, and the logical starting point is a convolution representation of the vector function \( \BM \)
%

The gradient of the vector \( \BM \) can be written as a single even grade multivector

\begin{equation}\label{eqn:helmholtzDerviationMultivector:60}
\spacegrad \BM
= \spacegrad \cdot \BM + I \spacegrad \cross \BM
= s + I \BC.
\end{equation}

%Observe that the Laplacian of \( \BM \) is vector valued
%
%\begin{dmath}\label{eqn:helmholtzDerviationMultivector:760}
%\spacegrad^2 \BM = \spacegrad s + I \spacegrad \BC.
%\end{dmath}
%
%This means that \( \spacegrad \BC \) must be a bivector \( \spacegrad \BC = \spacegrad \wedge \BC \), or that \( \BC \) has zero divergence
%
%\begin{dmath}\label{eqn:helmholtzDerviationMultivector:780}
%\spacegrad \cdot \BC = 0.
%\end{dmath}

We will use this to attempt to discover the relation between the vector \( \BM \) and its divergence and curl.  We can express \( \BM \) at the point of interest as a convolution with the delta function at all other points in space

\begin{dmath}\label{eqn:helmholtzDerviationMultivector:80}
\BM(\Bx) = \int_V dV' \delta(\Bx - \Bx') \BM(\Bx').
\end{dmath}

The Laplacian representation of the delta function in \R{3} is

\begin{dmath}\label{eqn:helmholtzDerviationMultivector:100}
\delta(\Bx - \Bx') = -\inv{4\pi} \spacegrad^2 \inv{\Abs{\Bx - \Bx'}},
\end{dmath}

so \( \BM \) can be represented as the following convolution

\begin{dmath}\label{eqn:helmholtzDerviationMultivector:120}
\BM(\Bx) = -\inv{4\pi} \int_V dV' \spacegrad^2 \inv{\Abs{\Bx - \Bx'}} \BM(\Bx').
\end{dmath}

%As noted in \cref{eqn:helmholtzDerviationMultivector:460} the Laplacian of a vector can be factored as
%
%\begin{dmath}\label{eqn:helmholtzDerviationMultivector:140}
%\spacegrad^2 \Ba
%=
%\spacegrad (\spacegrad \cdot \Ba)
%-
%\spacegrad \cross (\spacegrad \cross \Ba).
%\end{dmath}
%
%Note that the last term can be written in cross product notation using \( \Bc \cdot (\Ba \wedge \Bb) = -\Bc \cross (\Ba \cross \Bb) \) if desired.

Using this relation and proceeding with a few applications of the chain rule, plus the fact that \( \spacegrad 1/\Abs{\Bx - \Bx'} = -\spacegrad' 1/\Abs{\Bx - \Bx'} \), we find
%
%I previously posted a Geometric Algebra attack on the Helmholtz theorem.  Here is
%
%Here's a third way of deriving the Helmholtz theorem inversion relation.  This is a refinement of the traditional vector algebra solution that led to \cref{eqn:helmholtzDerviationMultivector:200}, that uses a factorization of the Laplacian directly, deferring any expansion in terms of dot and cross (or wedge) products until the very end.
%
%Starting from the first line of \cref{eqn:helmholtzDerviationMultivector:160}, we have

\begin{dmath}\label{eqn:helmholtzDerviationMultivector:720}
-4 \pi \BM(\Bx)
= \int_V dV' \spacegrad^2 \inv{\Abs{\Bx - \Bx'}} \BM(\Bx')
= \gpgradeone{\int_V dV' \spacegrad^2 \inv{\Abs{\Bx - \Bx'}} \BM(\Bx')}
= -\gpgradeone{\int_V dV' \spacegrad \lr{ \spacegrad' \inv{\Abs{\Bx - \Bx'}}} \BM(\Bx')}
= -\gpgradeone{\spacegrad \int_V dV' \lr{
\spacegrad' \frac{\BM(\Bx')}{\Abs{\Bx - \Bx'}}
-\frac{\spacegrad' \BM(\Bx')}{\Abs{\Bx - \Bx'}}
} }
=
-\gpgradeone{\spacegrad \int_{\partial V} dA'
\ncap \frac{\BM(\Bx')}{\Abs{\Bx - \Bx'}}
 }
+\gpgradeone{\spacegrad \int_V dV'
\frac{s(\Bx') + I\BC(\Bx')}{\Abs{\Bx - \Bx'}}
 }
=
-\gpgradeone{\spacegrad \int_{\partial V} dA'
\ncap \frac{\BM(\Bx')}{\Abs{\Bx - \Bx'}}
 }
+\spacegrad \int_V dV'
\frac{s(\Bx')}{\Abs{\Bx - \Bx'}}
+\spacegrad \cdot \int_V dV'
\frac{I\BC(\Bx')}{\Abs{\Bx - \Bx'}}.
\end{dmath}

By inserting a no-op grade selection operation in the second step, the trivector terms that would show up in subsequent steps are automatically filtered out.
%the troublesome trivector term that shows up in my first purely Geometric Algebra
%attempt is eliminated.
This leaves us with a boundary term dependent on the surface and the normal and tangential components of \( \BM \).  Added to that is a pair of volume integrals that provide the unique dependence of \( \BM \) on its divergence and curl.
When the surface is taken to infinity, which requires \( \Abs{\BM}/\Abs{\Bx - \Bx'} \rightarrow 0 \), then the dependence of \( \BM \) on its divergence and curl is unique.

In order to express final result in traditional vector algebra form, a couple transformations are required.  The first is that

\begin{dmath}\label{eqn:helmholtzDerviationMultivector:800}
\gpgradeone{ \Ba I \Bb } = I^2 \Ba \cross \Bb = -\Ba \cross \Bb.
\end{dmath}

For the grade selection in the boundary integral, note that

\begin{dmath}\label{eqn:helmholtzDerviationMultivector:740}
\gpgradeone{ \spacegrad \ncap \BX }
=
\gpgradeone{ \spacegrad (\ncap \cdot \BX) }
+
\gpgradeone{ \spacegrad (\ncap \wedge \BX) }
=
\spacegrad (\ncap \cdot \BX)
+
\gpgradeone{ \spacegrad I (\ncap \cross \BX) }
=
\spacegrad (\ncap \cdot \BX)
-
\spacegrad \cross (\ncap \cross \BX).
\end{dmath}

These give

%\begin{dmath}\label{eqn:helmholtzDerviationMultivector:721}
\boxedEquation{eqn:helmholtzDerviationMultivector:721}{
\begin{aligned}
\BM(\Bx)
&=
\spacegrad \inv{4\pi} \int_{\partial V} dA' \ncap \cdot \frac{\BM(\Bx')}{\Abs{\Bx - \Bx'}}
-
\spacegrad \cross \inv{4\pi} \int_{\partial V} dA' \ncap \cross \frac{\BM(\Bx')}{\Abs{\Bx - \Bx'}} \\
&-\spacegrad \inv{4\pi} \int_V dV'
\frac{s(\Bx')}{\Abs{\Bx - \Bx'}}
+\spacegrad \cross \inv{4\pi} \int_V dV'
\frac{\BC(\Bx')}{\Abs{\Bx - \Bx'}}.
\end{aligned}
}
%\end{dmath}

%we recover the non-boundary integral of \cref{eqn:helmholtzDerviationMultivector:200}.  The boundary term is seen to have a particularly simple form using this technique.  Note that the dot and double cross product expression obtained with the vector algebra approach can be recovered from this directly if desired using an expansion of the following form

%Using this expansion in \cref{eqn:helmholtzDerviationMultivector:720} recovers \cref{eqn:helmholtzDerviationMultivector:200}.
%Of the three methods I tried to obtain this result, this is the most compact and elegant of all three solution attempts.
%, but also requires full knowledge of the Geometric Algebra toolbox to understand.
}

%\EndArticle
