%
% Copyright © 2016 Peeter Joot.  All Rights Reserved.
% Licenced as described in the file LICENSE under the root directory of this GIT repository.
%
The gradient of the vector \( \BM \) can be written as a single even grade multivector
%
\begin{equation}\label{eqn:helmholtzDerviationMultivector:60}
\spacegrad \BM
= \spacegrad \cdot \BM + I \spacegrad \cross \BM
= s + I \BC.
\end{equation}
%
%Observe that the Laplacian of \( \BM \) is vector valued
%
%\begin{dmath}\label{eqn:helmholtzDerviationMultivector:760}
%\spacegrad^2 \BM = \spacegrad s + I \spacegrad \BC.
%\end{dmath}
%
%This means that \( \spacegrad \BC \) must be a bivector \( \spacegrad \BC = \spacegrad \wedge \BC \), or that \( \BC \) has zero divergence
%
%\begin{dmath}\label{eqn:helmholtzDerviationMultivector:780}
%\spacegrad \cdot \BC = 0.
%\end{dmath}

This can be used to attempt to discover the relation between the vector \( \BM \) and its divergence and curl.  
The vector \( \BM \) can be expressed at the point of interest as a convolution with the delta function at all other points in space

\begin{dmath}\label{eqn:helmholtzDerviationMultivector:80}
\BM(\Bx) = \int_V dV' \delta(\Bx - \Bx') \BM(\Bx').
\end{dmath}

The Laplacian representation of the delta function in \R{3} is

\begin{dmath}\label{eqn:helmholtzDerviationMultivector:100}
\delta(\Bx - \Bx') = -\inv{4\pi} \spacegrad^2 \inv{\Norm{\Bx - \Bx'}},
\end{dmath}

so \( \BM \) can be represented as the following convolution

\begin{dmath}\label{eqn:helmholtzDerviationMultivector:120}
\BM(\Bx) = -\inv{4\pi} \int_V dV' \spacegrad^2 \inv{\Norm{\Bx - \Bx'}} \BM(\Bx').
\end{dmath}

%As noted in \cref{eqn:helmholtzDerviationMultivector:460} the Laplacian of a vector can be factored as
%
%\begin{dmath}\label{eqn:helmholtzDerviationMultivector:140}
%\spacegrad^2 \Ba
%=
%\spacegrad (\spacegrad \cdot \Ba)
%-
%\spacegrad \cross (\spacegrad \cross \Ba).
%\end{dmath}
%
%Note that the last term can be written in cross product notation using \( \Bc \cdot (\Ba \wedge \Bb) = -\Bc \cross (\Ba \cross \Bb) \) if desired.

Using this relation and proceeding with a few applications of the chain rule, plus the fact that \( \spacegrad 1/\Norm{\Bx - \Bx'} = -\spacegrad' 1/\Norm{\Bx - \Bx'} \), we find
%
%I previously posted a Geometric Algebra attack on the Helmholtz theorem.  Here is
%
%Here's a third way of deriving the Helmholtz theorem inversion relation.  This is a refinement of the traditional vector algebra solution that led to \cref{eqn:helmholtzDerviationMultivector:200}, that uses a factorization of the Laplacian directly, deferring any expansion in terms of dot and cross (or wedge) products until the very end.
%
%Starting from the first line of \cref{eqn:helmholtzDerviationMultivector:160}, we have

\begin{dmath}\label{eqn:helmholtzDerviationMultivector:720}
-4 \pi \BM(\Bx)
= \int_V dV' \spacegrad^2 \inv{\Norm{\Bx - \Bx'}} \BM(\Bx')
= \gpgradeone{\int_V dV' \spacegrad^2 \inv{\Norm{\Bx - \Bx'}} \BM(\Bx')}
= -\gpgradeone{\int_V dV' \spacegrad \lr{ \spacegrad' \inv{\Norm{\Bx - \Bx'}}} \BM(\Bx')}
= -\gpgradeone{\spacegrad \int_V dV' \lr{
\spacegrad' \frac{\BM(\Bx')}{\Norm{\Bx - \Bx'}}
-\frac{\spacegrad' \BM(\Bx')}{\Norm{\Bx - \Bx'}}
} }
=
-\gpgradeone{\spacegrad \int_{\partial V} dA'
\ncap \frac{\BM(\Bx')}{\Norm{\Bx - \Bx'}}
 }
+\gpgradeone{\spacegrad \int_V dV'
\frac{s(\Bx') + I\BC(\Bx')}{\Norm{\Bx - \Bx'}}
 }
=
-\gpgradeone{\spacegrad \int_{\partial V} dA'
\ncap \frac{\BM(\Bx')}{\Norm{\Bx - \Bx'}}
 }
+\spacegrad \int_V dV'
\frac{s(\Bx')}{\Norm{\Bx - \Bx'}}
+\spacegrad \cdot \int_V dV'
\frac{I\BC(\Bx')}{\Norm{\Bx - \Bx'}}.
\end{dmath}

By inserting a no-op grade selection operation in the second step, the trivector terms that would show up in subsequent steps are automatically filtered out.
%the troublesome trivector term that shows up in my first purely Geometric Algebra
%attempt is eliminated.
This leaves us with a boundary term dependent on the surface and the normal and tangential components of \( \BM \).  Added to that is a pair of volume integrals that provide the unique dependence of \( \BM \) on its divergence and curl.
When the surface is taken to infinity, which requires \( \Norm{\BM}/\Norm{\Bx - \Bx'} \rightarrow 0 \), then the dependence of \( \BM \) on its divergence and curl is unique.

In order to express final result in traditional vector algebra form, a couple transformations are required.  The first is that
%
\begin{equation}\label{eqn:helmholtzDerviationMultivector:800}
\gpgradeone{ \Ba I \Bb } = I^2 \Ba \cross \Bb = -\Ba \cross \Bb.
\end{equation}
%
For the grade selection in the boundary integral, note that

\begin{dmath}\label{eqn:helmholtzDerviationMultivector:740}
\gpgradeone{ \spacegrad \ncap \BX }
=
\gpgradeone{ \spacegrad (\ncap \cdot \BX) }
+
\gpgradeone{ \spacegrad (\ncap \wedge \BX) }
=
\spacegrad (\ncap \cdot \BX)
+
\gpgradeone{ \spacegrad I (\ncap \cross \BX) }
=
\spacegrad (\ncap \cdot \BX)
-
\spacegrad \cross (\ncap \cross \BX).
\end{dmath}

These give

%\begin{dmath}\label{eqn:helmholtzDerviationMultivector:721}
\boxedEquation{eqn:helmholtzDerviationMultivector:721}{
\begin{aligned}
\BM(\Bx)
&=
\spacegrad \inv{4\pi} \int_{\partial V} dA' \ncap \cdot \frac{\BM(\Bx')}{\Norm{\Bx - \Bx'}}
-
\spacegrad \cross \inv{4\pi} \int_{\partial V} dA' \ncap \cross \frac{\BM(\Bx')}{\Norm{\Bx - \Bx'}} \\
&-\spacegrad \inv{4\pi} \int_V dV'
\frac{s(\Bx')}{\Norm{\Bx - \Bx'}}
+\spacegrad \cross \inv{4\pi} \int_V dV'
\frac{\BC(\Bx')}{\Norm{\Bx - \Bx'}}.
\end{aligned}
}
%\end{dmath}
