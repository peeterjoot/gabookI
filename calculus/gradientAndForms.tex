%
% Copyright � 2012 Peeter Joot.  All Rights Reserved.
% Licenced as described in the file LICENSE under the root directory of this GIT repository.
%

%
%
%\chapter{Exterior derivative and chain rule components of the gradient}
\index{exterior derivative}
\index{chain rule}
\index{gradient}
\label{chap:gradientAndForms}
%\date{March 31, 2008.  gradientAndForms.tex}

\section{Gradient formulation in terms of reciprocal frames}

We have seen how to calculate reciprocal frames as a method to find
components of a vector with respect to an arbitrary basis (does not have
to be orthogonal).

This can be applied to any vector:
%
\begin{equation}\label{eqn:gradientAndForms:20}
\Bx = \sum \Ba_i (\Ba^i \cdot \Bx) = \sum \Ba^i (\Ba_i \cdot \Bx)
\end{equation}
%
so why not the gradient operator too.
%
\begin{equation}
\grad = \sum \Ba_i (\Ba^i \cdot \grad) = \sum \Ba^i (\Ba_i \cdot \grad)
\end{equation}
%
The dot product part:
%
\begin{equation}
(\Ba \cdot \grad) f(\Bu) = \lim_{\tau \rightarrow 0}
\frac{ f(\Bu + \Ba \tau) - f(\Bu ) }{ \tau }
\end{equation}
%
we know how to calculate explicitly (from NFCM) and is the direction
derivative.

So this gives us an explicit factorization of the gradient into components
in some arbitrary set of directions, all weighted appropriately.

\section{Allowing the basis to vary according to a parametrization}

Now, if one allows the vector basis \({\Ba_i}\) to vary along a curve, it
is interesting to observe the consequences of this to the gradient expressed
as a component along the curve and perpendicular components.

Suppose that one has a parametrization \(\phi(u_1, u_2, \cdots, u_{n-1}) \in \mathbb{R}^n\), defining a generalized surface of degree one less than the space.

Provided these surface direction vectors are linearly independent and non
zero, we can writing \(\Bphi_{u_i} = \frac{\partial \Bphi}{ \partial u_i}\),
and form a basis for the space by extension with a reciprocal frame vector:
%
\begin{equation}\label{eqn:gradientAndForms:40}
\{ \Bphi_{u_1}, \Bphi_{u_2}, \cdots, \Bphi_{u_{n-1}},
(\Bphi_{u_1} \wedge \Bphi_{u_2} \cdots \wedge \Bphi_{u_{n-1}}) \inv{\BI_n}
 \}
\end{equation}
%
\makeexample{Try it with the simplest case}{example:gradientAndForms:100}{

Let us calculate \(\grad\) in this basis.  Intuition says this will
produce something like the exterior derivative from differential forms
for the component that is normal to the surface.

To make things easy, consider the absolutely simplest case, a curve
in \R{2}, with parametrization \(\Br = \Bphi(t)\).  The basis associated
with this curve at some point is
%
\begin{equation}\label{eqn:gradientAndForms:60}
\{ \Ba_1, \Ba_2 \} = \{ \Bphi_t, \BI \Bphi_t \}
\end{equation}
%
with a reciprocal basis of:
\begin{equation}\label{eqn:gradientAndForms:80}
\{ \Ba^1, \Ba^2 \} = \setlr{ \inv{\Bphi_t}, -\inv{\Bphi_t} \BI }
\end{equation}
%
In terms of this components, the gradient along the curve at the specified
point is:
%
\begin{equation}\label{eqn:gradientAndForms:160}
\begin{aligned}
\grad f
&= \left(\Ba^1 \Ba_1 \cdot \grad +\Ba^2 \Ba_2 \cdot \grad \right) f \\
&= \left(\inv{\Bphi_t} \Bphi_t \cdot \grad + \left( \BI \inv{\Bphi_t} \right) \left(\BI \Bphi_t \right) \cdot \grad \right) f \\
&= \inv{\Bphi_t}\left(\Bphi_t \cdot \grad -\BI \left(\BI \cdot \Bphi_t \right) \cdot \grad \right) f \\
&= \inv{\Bphi_t}\left(\Bphi_t \cdot \grad -\BI \left(\BI \cdot \left(\Bphi_t \wedge \grad \right) \right) \right) f \\
&= \inv{\Bphi_t}\left(\Bphi_t \cdot \grad -\BI \left(\BI \left(\Bphi_t \wedge \grad \right) \right) \right) f \\
&= \inv{\Bphi_t}\left(\Bphi_t \cdot \grad + \Bphi_t \wedge \grad \right) f \\
\end{aligned}
\end{equation}
%
Lo and behold, we come full circle through a mass of identities back to the geometric product.
As with many things in math, knowing the answer we can be clever and start from the answer going backwards.  This would have allowed the standard factorization of the gradient vector into
orthogonal components in the usual fashion:
%
\begin{equation}\label{eqn:gradAndForms:grad_dot_wedge}
\grad
= \inv{\Bphi_t}\left( \Bphi_t \grad \right)
= \inv{\Bphi_t}\left( \Bphi_t \cdot \grad + \Bphi_t \wedge \grad \right)
\end{equation}
%
Let us continue writing \(\Bphi(t) = x(t) \Be_1 + y(t) \Be_2\).  Then
%
\begin{equation}\label{eqn:gradientAndForms:100}
\Bphi' = x' \Be_1 + y' \Be_2
\end{equation}
\begin{equation}\label{eqn:gradientAndForms:120}
\Bphi' \cdot \grad =
\frac{dx}{dt} \frac{\partial}{\partial x}
+\frac{dy}{dt} \frac{\partial}{\partial y}
\end{equation}
%
\begin{equation}\label{eqn:gradientAndForms:140}
\Bphi' \wedge \grad =
\BI
\left(\frac{dx}{dt} \frac{\partial}{\partial y} -\frac{dy}{dt} \frac{\partial}{\partial x}\right)
\end{equation}
%
Combining these and inserting back into \eqnref{eqn:gradAndForms:grad_dot_wedge} we have
%
\begin{equation}\label{eqn:gradientAndForms:180}
\begin{aligned}
\grad
&= \frac{\Bphi'}{\left(\Bphi'\right)^2}\left( \Bphi' \cdot \grad + \Bphi' \wedge \grad \right) \\
&=
\inv{(x')^2 + (y')^2}
(x',y')\left(
\frac{dx}{dt} \frac{\partial}{\partial x}
+\frac{dy}{dt} \frac{\partial}{\partial y}
\right)
+
(-y',x')
\left(\frac{dx}{dt} \frac{\partial}{\partial y} -\frac{dy}{dt} \frac{\partial}{\partial x}\right)
\end{aligned}
\end{equation}
%
Now, here it is worth pointing out that the choice of the parametrization can break some of the assumptions made.  In particular
the curve can be completely continuous, but the parametrization could allow it to be zero for some interval since \((x(t), y(t))\)
can be picked to be constant for a ``time'' before continuing.

This problem is eliminated by picking an arc length parametrization.  Provided the curve is not degenerate (ie: a point), then
we have at least one of \(dx/ds \ne 0\), or \(dy/ds \ne 0\).  Additionally, by parametrization using arc length we have
\((dx/ds)^2 + (dy/ds)^2 = (ds/ds)^2 = 1\).  This eliminates the denominator leaving the following decomposition of the \R{2} gradient
%
\begin{equation}
\grad =
\mathLabelBox{(x',y')}{Unit tangent vector}
\mathLabelBox
[
   labelstyle={below of=m\themathLableNode, below of=m\themathLableNode}
]
{
\left(
\frac{dx}{ds} \frac{\partial}{\partial x}
+\frac{dy}{ds} \frac{\partial}{\partial y}
\right)
}{Chain Rule in operator form.}
+
\mathLabelBox{(-y',x')}{Unit normal vector}
\mathLabelBox
[
   labelstyle={below of=m\themathLableNode, below of=m\themathLableNode}
]
{
\left(
\frac{dx}{ds} \frac{\partial}{\partial y}
-\frac{dy}{ds} \frac{\partial}{\partial x}
\right)
}{Exterior derivative operator.}
\end{equation}
%
Thus, loosely speaking we have the chain rule as the scalar component of the unit tangent vector along a parametrized curve,
and we have the exterior derivative as the component of the gradient that lies colinear to a unit normal to the curve (believe this
is the unit normal that points inwards to the curvature if there is any).

} % example

\makeexample{Extension to higher dimensional curves}{example:gradientAndForms:200}{

In \R{3} or above we can perform the same calculation.  The result is similar and straightforward to derive:
%
\begin{equation}
\nabla
=
\mathLabelBox{ \left(\frac{dx_1}{ds}, \frac{dx_2}{ds}, \cdots, \frac{dx_n}{ds}\right) }{Unit tangent.}
\sum_{i=1}^n \frac{dx_i}{ds}\frac{\partial}{\partial x_i}
+ \sum_{1 \le i < j \le n}
\mathLabelBox{
\left(
\mathbf{e}_j \frac{dx_i}{ds}
-\mathbf{e}_i \frac{dx_j}{ds}
\right)
}{Normal to unit tangent.}
\left(
\frac{dx_i}{ds}\frac{\partial}{\partial x_j}
-\frac{dx_j}{ds}\frac{\partial}{\partial x_i}
\right)
\end{equation}
%
Here we have a set of normals to the unit tangent.  For \R{3}, we have \(ij=\{12,13,23\}\).  One of these unit normals must
be linearly dependent on the other two (or zero).  The exterior scalar factors here loose some of their resemblance to the
exterior derivative here.  Perhaps a parametrized (hyper-)surface is required to get the familiar form for \R{3} or above.
} % example
