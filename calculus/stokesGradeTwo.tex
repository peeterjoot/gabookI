%
% Copyright � 2012 Peeter Joot.  All Rights Reserved.
% Licenced as described in the file LICENSE under the root directory of this GIT repository.
%

%
%
%\input{../peeter_prologue.tex}
%\usepackage{txfonts}

\chapter{Stokes theorem applied to vector and bivector fields}
\label{chap:stokesGradeTwo}

%%\date{July 17, 2009}
%\blogpage{http://sites.google.com/site/peeterjoot/math2009/stokesGradeTwo.pdf}

%\beginArtWithToc

\section{Vector Stokes Theorem}

I found my self forgetting stokes theorem once again.  Redo this for the simplest case of a parallelogram area element.

What I recall is that we have on one side the curl dotted into the plane of the surface area element

\begin{equation}\label{eqn:stokesGradeTwo:20}
\begin{aligned}
\int ( \grad \wedge A ) \cdot d^2 x
\end{aligned}
\end{equation}

and on the other side a loop integral

\begin{equation}\label{eqn:stokesGradeTwo:40}
\begin{aligned}
\ointctrclockwise A \cdot dx
\end{aligned}
\end{equation}

Comparing the two we should end up with the same form and thus determine the form of the grade two Stokes equation (i.e. for curl of a vector).

\subsection{Bivector product part}

\begin{equation}\label{eqn:stokesGradeTwo:60}
\begin{aligned}
( \grad \wedge A ) \cdot d^2 x
&=
( \grad \wedge A ) \cdot \left(\PD{\alpha}{x} \wedge \PD{\beta}{x}\right)
d\alpha d\beta \\
&=
\partial_\mu A_\nu \PD{\alpha}{x^\sigma} \PD{\beta}{x^\epsilon} (\gamma^\mu \wedge \gamma^\nu) \cdot (\gamma_\sigma \wedge \gamma_\epsilon)
d\alpha d\beta \\
&=
\partial_\mu A_\nu \PD{\alpha}{x^\sigma} \PD{\beta}{x^\epsilon} ( {\delta^\mu}_\epsilon {\delta^\nu}_\sigma - {\delta^\mu}_\sigma {\delta^\nu}_\epsilon )
d\alpha d\beta \\
&=
\partial_\mu A_\nu \left( \PD{\alpha}{x^\nu} \PD{\beta}{x^\mu} - \PD{\alpha}{x^\mu} \PD{\beta}{x^\nu} \right)
d\alpha d\beta \\
\end{aligned}
\end{equation}

So we have
\begin{equation}\label{eqn:stokesGradeTwo:withJac}
\begin{aligned}
( \grad \wedge A ) \cdot d^2 x &= -\partial_\mu A_\nu \frac{\partial (x^\mu, x^\nu)}{\partial (\alpha, \beta)} d\alpha d\beta
\end{aligned}
\end{equation}

\subsection{Loop integral part}

Integrating around a parallelogram spacetime area element with sides \(d\alpha \partial x/\partial \alpha\) and \(d\beta \partial x/\partial \beta\), as depicted in \cref{fig:surface_area_element}, we have

\imageFigure{../figures/gabook/surface_area_element}{Surface area element}{fig:surface_area_element}{0.4}

\begin{equation}\label{eqn:stokesGradeTwo:80}
\begin{aligned}
\ointctrclockwise
A \cdot dx
&=
\int
{\left. A \right\vert}_{\beta=\beta_0} \cdot \PD{\alpha}{x} d\alpha
+ {\left. A \right\vert}_{\alpha=\alpha_1} \cdot \PD{\beta}{x} d\beta
+ {\left. A \right\vert}_{\beta=\beta_1} \cdot \left( -\PD{\alpha}{x} d\alpha \right)
+ {\left. A \right\vert}_{\alpha=\alpha_0} \cdot \left( -\PD{\beta}{x} d\beta \right)
\\
&=
\int
\left( {\left. A \right\vert}_{\alpha=\alpha_1} - {\left. A \right\vert}_{\alpha=\alpha_0} \right) \cdot \PD{\beta}{x} d\beta
-\left( {\left. A \right\vert}_{\beta=\beta_1} - {\left. A \right\vert}_{\beta=\beta_0} \right) \cdot \PD{\alpha}{x} d\alpha
\\
&=
\int
\PD{\alpha}{A} \cdot \PD{\beta}{x} d\alpha d\beta
-\PD{\beta}{A} \cdot \PD{\alpha}{x} d\beta d\alpha
\end{aligned}
\end{equation}

Expanding the derivatives in terms of coordinates we have

\begin{equation}\label{eqn:stokesGradeTwo:100}
\begin{aligned}
\PD{\sigma}{A}
&=
\PD{\sigma}{A_\mu} \gamma^\mu \\
&=
\PD{x^\nu}{A_\mu}\PD{\sigma}{x^\nu} \gamma^\mu \\
&=
\partial_\nu A_\mu \PD{\sigma}{x^\nu} \gamma^\mu \\
\end{aligned}
\end{equation}

and
\begin{equation}\label{eqn:stokesGradeTwo:120}
\begin{aligned}
\PD{\sigma}{x} &= \PD{\sigma}{x^\nu} \gamma_\nu
\end{aligned}
\end{equation}

Assembling we have
\begin{equation}\label{eqn:stokesGradeTwo:140}
\begin{aligned}
\ointctrclockwise
A \cdot dx
&=
\int
\partial_\nu A_\mu \left( \PD{\alpha}{x^\nu} \PD{\beta}{x^\mu} - \PD{\beta}{x^\nu} \PD{\alpha}{x^\mu} \right) d\alpha d\beta
\end{aligned}
\end{equation}

In terms of the Jacobian used in \eqnref{eqn:stokesGradeTwo:withJac} we have

\begin{equation}\label{eqn:stokesGradeTwo:160}
\begin{aligned}
\ointctrclockwise
A \cdot dx &= \int \partial_\mu A_\nu \frac{\partial (x^\mu, x^\nu)}{\partial (\alpha, \beta)} d\alpha d\beta
\end{aligned}
\end{equation}

Comparing the two we have only a sign difference so the conclusion is that Stokes for a vector field (considering only a flat parallelogram area element) is

\begin{equation}\label{eqn:stokesGradeTwo:180}
\begin{aligned}
\int ( \grad \wedge A ) \cdot d^2 x &= \ointclockwise A \cdot dx
\end{aligned}
\end{equation}

Observe that there is an implied orientation of the area element on the LHS, required to match up with the orientation of the RHS integral.

\section{Bivector Stokes Theorem}

A parallelepiped volume element is depicted in \cref{fig:volume_element}.  Three parameters \(\alpha\), \(\beta\), \(\sigma\) generate a set of differential vector displacements spanning the three dimensional subspace

\imageFigure{../figures/gabook/volume_element}{Differential volume element}{fig:volume_element}{0.4}

Writing the displacements

\begin{equation}\label{eqn:stokesGradeTwo:200}
\begin{aligned}
dx_\alpha &= \PD{\alpha}{x} d\alpha \\
dx_\beta &= \PD{\beta}{x} d\beta \\
dx_\sigma &= \PD{\sigma}{x} d\sigma
\end{aligned}
\end{equation}

We have for the front, right and top face area elements

\begin{equation}\label{eqn:stokesGradeTwo:220}
\begin{aligned}
dA_F &= dx_\alpha \wedge dx_\beta \\
dA_R &= dx_\beta \wedge dx_\sigma \\
dA_T &= dx_\sigma \wedge dx_\alpha \\
\end{aligned}
\end{equation}

These are the surfaces of constant parametrization, respectively, \(\sigma = \sigma_1\), \(\alpha = \alpha_1\), and \(\beta = \beta_1\).  For a bivector, the flux through the surface is therefore

\begin{equation}\label{eqn:stokesGradeTwo:240}
\begin{aligned}
\int B \cdot dA
&=
(B_{\sigma_1} \cdot dA_F - B_{\sigma_0} \cdot dA_P )
+ (B_{\alpha_1} \cdot dA_R - B_{\alpha_0} \cdot dA_L)
+ (B_{\beta_1} \cdot dA_T - B_{\beta_0} \cdot dA_B) \\
&=
d \sigma \PD{\sigma}{B} \cdot (dx_\alpha \wedge dx_\beta )
+ d \alpha \PD{\alpha}{B} \cdot (dx_\beta \wedge dx_\sigma)
+ d \beta \PD{\beta}{B} \cdot (dx_\sigma \wedge dx_\alpha ) \\
\end{aligned}
\end{equation}

Written out in full this is a bit of a mess
\begin{equation}\label{eqn:stokesGradeTwo:mess}
\begin{aligned}
\int B \cdot dA
&=
d \alpha d\beta d\sigma
\partial_\mu B \cdot
\left(
\left(
- \PD{\sigma}{x^\mu} \PD{\beta}{x^\nu} \PD{\alpha}{x^\epsilon}
+ \PD{\alpha}{x^\mu} \PD{\beta}{x^\nu} \PD{\sigma}{x^\epsilon}
+ \PD{\beta}{x^\mu} \PD{\sigma}{x^\nu} \PD{\alpha}{x^\epsilon}
\right)
(\gamma_\nu \wedge \gamma_\epsilon )
\right)
\end{aligned}
\end{equation}

It should equal, at least up to a sign, \(\int (\grad \wedge B) \cdot d^3 x\).  Expanding the latter is probably easier than regrouping the mess, and doing so we have

\begin{equation}\label{eqn:stokesGradeTwo:260}
\begin{aligned}
(\grad \wedge B) \cdot d^3 x
&=
d\alpha d\beta d\sigma ( \gamma^\mu \wedge \partial_\mu B)  \cdot \left( \PD{\alpha}{x} \wedge \PD{\beta}{x} \wedge \PD{\sigma}{x} \right) \\
&=
d\alpha d\beta d\sigma \inv{2} ( \gamma^\mu \partial_\mu B + \partial_\mu B \gamma^\mu )  \cdot \left( \PD{\alpha}{x} \wedge \PD{\beta}{x} \wedge \PD{\sigma}{x} \right) \\
&=
d\alpha d\beta d\sigma \inv{2} \gpgradezero{
( \gamma^\mu \partial_\mu B + \partial_\mu B \gamma^\mu )  \left( \PD{\alpha}{x} \wedge \PD{\beta}{x} \wedge \PD{\sigma}{x} \right) }
\\
&=
d\alpha d\beta d\sigma \inv{2}
\partial_\mu B \cdot
\gpgradetwo{
\left( \PD{\alpha}{x} \wedge \PD{\beta}{x} \wedge \PD{\sigma}{x} \right) \gamma^\mu
+ \gamma^\mu \left( \PD{\alpha}{x} \wedge \PD{\beta}{x} \wedge \PD{\sigma}{x} \right) }
\\
&=
d\alpha d\beta d\sigma
\partial_\mu B \cdot
\left( \left( \PD{\alpha}{x} \wedge \PD{\beta}{x} \wedge \PD{\sigma}{x} \right) \cdot \gamma^\mu \right)
\\
\end{aligned}
\end{equation}

Expanding just that trivector-vector dot product

\begin{equation}\label{eqn:stokesGradeTwo:280}
\begin{aligned}
\left( \PD{\alpha}{x} \wedge \PD{\beta}{x} \wedge \PD{\sigma}{x} \right) \cdot \gamma^\mu
&=
\PD{\alpha}{x^\lambda} \PD{\beta}{x^\nu} \PD{\sigma}{x^\epsilon} \left( \gamma_\lambda \wedge \gamma_\nu \wedge \gamma_\epsilon \right) \cdot \gamma^\mu  \\
&=
\PD{\alpha}{x^\lambda} \PD{\beta}{x^\nu} \PD{\sigma}{x^\epsilon} \left(
\gamma_\lambda \wedge \gamma_\nu {\delta_\epsilon}^\mu
-\gamma_\lambda \wedge \gamma_\epsilon {\delta_\nu}^\mu
+\gamma_\nu \wedge \gamma_\epsilon {\delta_\lambda}^\mu
\right)
\end{aligned}
\end{equation}

So we have
\begin{equation}\label{eqn:stokesGradeTwo:300}
\begin{aligned}
(\grad \wedge B) \cdot d^3 x
&=
d\alpha d\beta d\sigma \PD{\alpha}{x^\lambda} \PD{\beta}{x^\nu} \PD{\sigma}{x^\epsilon} \partial_\mu B \cdot \left(
\gamma_\lambda \wedge \gamma_\nu {\delta_\epsilon}^\mu
-\gamma_\lambda \wedge \gamma_\epsilon {\delta_\nu}^\mu
+\gamma_\nu \wedge \gamma_\epsilon {\delta_\lambda}^\mu
\right)
\\
&=
d\alpha d\beta d\sigma
\partial_\mu B \cdot \left(
  \PD{\alpha}{x^\lambda} \PD{\beta}{x^\nu} \PD{\sigma}{x^\mu} \gamma_\lambda \wedge \gamma_\nu
%- \PD{\alpha}{x^\lambda} \PD{\beta}{x^\mu} \PD{\sigma}{x^\epsilon} \gamma_\lambda \wedge \gamma_\epsilon
+ \PD{\alpha}{x^\lambda} \PD{\beta}{x^\mu} \PD{\sigma}{x^\epsilon} \gamma_\epsilon \wedge \gamma_\lambda
+ \PD{\alpha}{x^\mu} \PD{\beta}{x^\nu} \PD{\sigma}{x^\epsilon} \gamma_\nu \wedge \gamma_\epsilon
\right)
\\
&=
d\alpha d\beta d\sigma
\partial_\mu B \cdot
\left(
\left(
  \PD{\alpha}{x^\nu} \PD{\beta}{x^\epsilon} \PD{\sigma}{x^\mu}
+ \PD{\alpha}{x^\epsilon} \PD{\beta}{x^\mu} \PD{\sigma}{x^\nu}
+ \PD{\alpha}{x^\mu} \PD{\beta}{x^\nu} \PD{\sigma}{x^\epsilon}
\right)
\gamma_\nu \wedge \gamma_\epsilon
\right)
\\
\end{aligned}
\end{equation}

Noting that an \(\epsilon\), \(\nu\) interchange in the first term inverts the sign, we have an exact match with \eqnref{eqn:stokesGradeTwo:mess}, thus fixing the sign for the
bivector form of Stokes theorem for the orientation picked in this diagram

\begin{equation}\label{eqn:stokesGradeTwo:320}
\begin{aligned}
\int (\grad \wedge B) \cdot d^3 x &= \int B \cdot d^2 x
\end{aligned}
\end{equation}

Like the vector case, there is a requirement to be very specific about the meaning given to the oriented surfaces, and the corresponding oriented volume element (which could be a volume subspace of a greater than three dimensional space).

%\EndNoBibArticle
