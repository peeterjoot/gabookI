%
% Copyright � 20123Peeter Joot.  All Rights Reserved.
% Licenced as described in the file LICENSE under the root directory of this GIT repository.
%
% pick one:
%\input{../assignment.tex}
%\input{../blogpost.tex}
%\renewcommand{\basename}{tangentAndNormalVectors}
%\renewcommand{\dirname}{notes/gabook/}
%%\newcommand{\dateintitle}{}
%\newcommand{\keywords}{tangent plane, surface normal, gradient, 3D, 4D, reciprocal basis, duality, pseudoscalar, geometric algebra, bivector, trivector, Minkowski space}
%
%\input{../peeter_prologue_print2.tex}
%
%\beginArtNoToc
%
%\generatetitle{Tangent planes and normals in three and four dimensions}
%\chapter{Tangent planes and normals in three and four dimensions}
\index{tangent plane}
\index{normal}
\label{chap:tangentAndNormalVectors}
\section{Motivation}

I was reviewing the method of Lagrange in my old first year calculus book \citep{salas1990coa} and found that I needed a review of some of the geometry ideas associated with the gradient (that it is normal to the surface).  The approach in the text used 3D level surfaces \(f(x, y, z) = c\), which is general but not the most intuitive.

If we define a surface in the simpler explicit form \(z = f(x, y)\), then how would you show this normal property?  Here we explore this in 3D and 4D, using geometric and wedge products to express the tangent planes and tangent volumes respectively.

In the 4D approach, with a vector \(x\) defined by coordinates \(x^\mu\) and basis \(\{\gamma_\mu\}\) so that

\begin{dmath}\label{eqn:tangentAndNormalVectors:20}
x = \gamma_\mu x^\mu,
\end{dmath}

the reciprocal basis \({\gamma^\mu}\) is defined implicitly by the dot product relations

\begin{dmath}\label{eqn:tangentAndNormalVectors:40}
\gamma^\mu \cdot \gamma_\nu = {\delta^\mu}_\nu.
\end{dmath}

Assuming such a basis makes the result general enough that the 4D (or a trivial generalization to N dimensions) holds for both Euclidean spaces as well as mixed metric (i.e. Minkowski) spaces, and avoids having to detail the specific metric in question.

\section{3D surface}

We start by considering \cref{fig:tangentAndNormalVectors:tangentAndNormalVectorsFig1}.  We wish to determine the bivector for the tangent plane in the neighborhood of the point \(\Bp\)

\imageFigure{../gabook-figures/tangentAndNormalVectorsFig1}{A portion of a surface in 3D}{fig:tangentAndNormalVectors:tangentAndNormalVectorsFig1}{0.3}

\begin{dmath}\label{eqn:tangentAndNormalVectors:60}
\Bp = ( x, y, f(x, y) ),
\end{dmath}

then using a duality transformation (multiplication by the pseudoscalar for the space) determine the normal vector to that plane at this point.  Holding either of the two free parameters constant, we find the tangent vectors on that surface to be

\begin{subequations}
\begin{dmath}\label{eqn:tangentAndNormalVectors:80}
\Bp_1
= \left( dx, 0, \PD{x}{f} dx \right)
\propto \left( 1, 0, \PD{x}{f} \right)
\end{dmath}
\begin{dmath}\label{eqn:tangentAndNormalVectors:100}
\Bp_2
= \left( 0, dy, \PD{y}{f} dy \right)
\propto \left( 0, 1, \PD{y}{f} \right)
\end{dmath}
\end{subequations}

The tangent plane is then

\begin{dmath}\label{eqn:tangentAndNormalVectors:120}
\Bp_1 \wedge \Bp_2 =
\left( 1, 0, \PD{x}{f} \right) \wedge
\left( 0, 1, \PD{y}{f} \right)
=
\left( \Be_1 + \Be_3 \PD{x}{f} \right)
\wedge
\left( \Be_2 + \Be_3 \PD{y}{f} \right)
=
\Be_1 \Be_2
+ \Be_1 \Be_3 \PD{y}{f}
+ \Be_3 \Be_2 \PD{x}{f}.
\end{dmath}

We can factor out the pseudoscalar 3D volume element \(I = \Be_1 \Be_2 \Be_3\), assuming a Euclidean space for which \(\Be_k^2 = 1\).  That is

\begin{dmath}\label{eqn:tangentAndNormalVectors:140}
\Bp_1 \wedge \Bp_2 =
\Be_1 \Be_2 \Be_3 \left(
\Be_3
- \Be_2 \PD{y}{f}
- \Be_1 \PD{x}{f}
\right)
\end{dmath}

Multiplying through by the volume element \(I\) we find that the normal to the surface at this point is

\begin{dmath}\label{eqn:tangentAndNormalVectors:160}
\Bn
\propto -I(\Bp_1 \wedge \Bp_2)
=
\Be_3
- \Be_1 \PD{x}{f}
- \Be_2 \PD{y}{f}.
\end{dmath}

Observe that we can write this as

\boxedEquation{eqn:tangentAndNormalVectors:180}{
\Bn = \spacegrad ( z - f(x, y) ).
}

Let's see how this works in 4D, so that we know how to handle the Minkowski spaces we find in special relativity.

\section{4D surface}

Now, let's move up to one additional direction, with

\begin{dmath}\label{eqn:tangentAndNormalVectors:200}
x^3 = f(x^0, x^1, x^2).
\end{dmath}

the differential of this is

\begin{equation}\label{eqn:tangentAndNormalVectors:220}
dx^3
= \sum_{k=0}^2 \PD{x^k}{f} dx^k
= \sum_{k=0}^2 \partial_k f dx^k .
\end{equation}

We are going to look at the 3-surface in the neighborhood of the point

\begin{equation}\label{eqn:tangentAndNormalVectors:240}
p =
\left(
x^0, x^1, x^2, x^3
\right),
\end{equation}

so that the tangent vectors in the neighborhood of this point are in the span of

\begin{equation}\label{eqn:tangentAndNormalVectors:260}
dp =
\left(
x^0, x^1, x^2, \sum_{k=0}^2 \partial_k dx^k
\right).
\end{equation}

In particular, in each of the directions we have

\begin{subequations}
\begin{equation}\label{eqn:tangentAndNormalVectors:280}
p_0 \propto ( 1, 0, 0, d_0 f)
\end{equation}
\begin{equation}\label{eqn:tangentAndNormalVectors:300}
p_1 \propto ( 0, 1, 0, d_1 f)
\end{equation}
\begin{equation}\label{eqn:tangentAndNormalVectors:320}
p_2 \propto ( 0, 0, 1, d_2 f)
\end{equation}
\end{subequations}

Our tangent volume in this neighborhood is

\begin{dmath}\label{eqn:tangentAndNormalVectors:340}
p_0 \wedge p_1 \wedge p_2
=
\left(
\gamma_0 + \gamma_3 \partial_0 f
\right)
\wedge
\left(
\gamma_1 + \gamma_3 \partial_1 f
\right)
\wedge
\left(
\gamma_2 + \gamma_3 \partial_2 f
\right)
=
\left(
\gamma_0 \gamma_1
+ \gamma_0 \gamma_3 \partial_1 f
+ \gamma_3 \gamma_1 \partial_0 f
\right)
\wedge
\left(
\gamma_2 + \gamma_3 \partial_2 f
\right)
=
\gamma_{012} - \gamma_{023} \partial_1 f + \gamma_{123} \partial_0 f + \gamma_{013} \partial_2 f.
\end{dmath}

Here the shorthand \(\gamma_{ijk} = \gamma_i \gamma_j \gamma_k\) has been used.  Can we factor out a 4D pseudoscalar from this and end up with a coherent result?  We have

\begin{subequations}
\begin{equation}\label{eqn:tangentAndNormalVectors:360}
\gamma_{0123} \gamma^3 = \gamma_{012}
\end{equation}
\begin{equation}\label{eqn:tangentAndNormalVectors:380}
\gamma_{0123} \gamma^1 = \gamma_{023}
\end{equation}
\begin{equation}\label{eqn:tangentAndNormalVectors:400}
\gamma_{0123} \gamma^0 = -\gamma_{123}
\end{equation}
\begin{equation}\label{eqn:tangentAndNormalVectors:420}
\gamma_{0123} \gamma^2 = -\gamma_{013}.
\end{equation}
\end{subequations}

This gives us

\begin{equation}\label{eqn:tangentAndNormalVectors:440}
d^3 p
=
p_0 \wedge p_1 \wedge p_2
=
\gamma_{0123} \left(
\gamma^3
- \gamma^1 \partial_1 f
- \gamma^0 \partial_0 f
- \gamma^2 \partial_2 f
\right).
\end{equation}

With the usual 4d gradient definition (sum implied)

\begin{dmath}\label{eqn:tangentAndNormalVectors:460}
\grad = \gamma^\mu \partial_\mu,
\end{dmath}

we have

\begin{dmath}\label{eqn:tangentAndNormalVectors:480}
\grad x^3
= \gamma^\mu \partial_\mu x^3
= \gamma^\mu {\delta_{\mu}}^3
= \gamma^3,
\end{dmath}

so we can write
\begin{dmath}\label{eqn:tangentAndNormalVectors:500}
d^3 p = \gamma_{0123} \grad \left( x^3 - f(x^0, x^1, x^2) \right),
\end{dmath}

so, finally, the ``normal'' to this surface volume element at this point is

\boxedEquation{eqn:tangentAndNormalVectors:520}{
n =
\grad \left( x^3 - f(x^0, x^1, x^2) \right).
}

This is just like the 3D Euclidean result, with the exception that we need to look at the dual of a 3-volume ``surface'' instead of our normal 2D surface.

It may seem curious that we had to specify an Euclidean metric for the 3D case, but did not here.  That doesn't mean this is a metric free result.  Instead, the metric choice is built into the definition of the gradient \eqnref{eqn:tangentAndNormalVectors:460} and its associated reciprocal basis.  For example with a \(1,3\) metric where \(\gamma_0^2 = 1, \gamma_k^2 = -1\), we have \(\gamma^0 = \gamma_0\) and \(\gamma^k = -\gamma_k\).

% this is to produce the sites.google url and version info and so forth (for blog posts)
%\vcsinfo
%\EndArticle
