%
% Copyright � 2012 Peeter Joot.  All Rights Reserved.
% Licenced as described in the file LICENSE under the root directory of this GIT repository.
%

%
%
%\usepackage{txfonts}
\chapter{Stokes law in wedge product form}\label{chap:PJStokes1}
%\date{Sept. 18, 2008.  vectorIntegralRelations.tex}

\section{A hodge podge of relations}

The aim of these notes is to work through proofs of the following integral equations

\begin{itemize}

\item Gradient line integral.

\begin{equation}\label{eqn:vector_integral_relations:lineintegral}
\int_C (\grad f) \cdot d\Br = f \vert_{\partial C}
\end{equation}

\item Jacobian area determinants.

Change of variables for a double integral

\begin{equation}
dA = dx dy =
\begin{vmatrix}
\PD{u}{x} & \PD{u}{y} \\
\PD{v}{x} & \PD{v}{y} \\
\end{vmatrix}
du dv
= \Abs{ \PD{(u,v)}{(x,y)} } du dv
\end{equation}

In Salus and Hille this is proved using Green's theorem, despite it
seeming like the more basic operation.  The greater than two dimensional cases are not proved at all.

\item Green's theorem.

\begin{equation}\label{eqn:vector_integral_relations:greens}
\iint \left(\PD{x}{Q} - \PD{y}{P}\right) dx dy = \ointctrclockwise P dx + Q dy
\end{equation}

\item Stokes theorem.

\begin{equation}\label{eqn:vector_integral_relations:stokes}
\iint (\grad \cross \Bf) \cdot \ncap\, dx dy = \ointctrclockwise \Bf \cdot d\Br
\end{equation}

\item Divergence theorem.

\begin{equation}\label{eqn:vector_integral_relations:divergenceplane}
\iint \grad \cdot \Bf\, dx dy = \oint \Bf \cdot \ncap\, ds
\end{equation}

\begin{equation}\label{eqn:vector_integral_relations:divergencevolume}
\iiint_V \grad \cdot \Bf\, dx dy dz = \iint_S \Bf \cdot \ncap\, dA
\end{equation}

\begin{equation}\label{eqn:vector_integral_relations:divergencegrad}
\iiint_V \grad \phi\, dV \iint_S \ncap \phi\, dA
\end{equation}

\begin{equation}\label{eqn:vector_integral_relations:divergencegradcross}
\iiint_V \grad \cross \Bf\, dV \iint_S \Bf \cross \ncap\, dA
\end{equation}


\end{itemize}

In particular I had like to relate these to the geometrical concepts of Clifford algebra now that I know how to work with that in a differential and algebraic fashion for many sorts of problems.  I am hoping that working through proofs of these basic identities will be enough that I can go on to the more general approaches in differential forms and the geometric calculus of Hestenes.

John Denker's
\href{ http://www.av8n.com/physics/straight-wire.pdf }{ article on the
magnetic field of a straight wire }
gives a simple looking high level description of vector form of Stokes'
theorem in its Clifford formulation

\begin{equation}\label{eqn:vector_integral_relations:stokesGA}
\int_S \grad \wedge F = \int_{\partial S} F
\end{equation}

This is simple enough looking, but there are some important details left
out.  In particular the grades do not match, so there must be some sort of
implied projection or dot product operations too.

I had say this suffers from some of the things that I had trouble with in
attempting to study differential forms.

The basic ideas of how to formulate
the curve, surface, volume, ... of integration is not specified.  How to do
that in greater than three dimensions is not trivial seeming to me since
none of the traditional methods of dotting with a normal work.

Knowing now about how subspaces can be expressed using blades is likely the
key.  The Clifford algebra ideas seem particularly suited to this as many
of these ideas can be formulated independent of the calculus applications.
One can learn the geometric and algebraic concepts first and then move on
to the Calculus.

\section{Gradient line integral}

This is the easiest of the identities to prove.  Introduction of a reciprocal frame \(\gamma^{\mu} \cdot \gamma_{\nu} = {\delta^{\mu}}_{\nu}\)
also means that we can do in full generality with a possibly
non-orthonormal basis of any dimension, and an arbitrary metric.

Write the gradient as normal

\begin{equation*}
\grad = \sum \gamma^{\mu} \PD{x^{\mu}}{} = \gamma^{\mu} \partial_{\mu}
\end{equation*}

Here summation convention with implied sum over mixed upper and lower indices is employed.

Express the position vector along the curve as
a parametrized path \(\Br = \Br(\lambda) = \gamma_{\mu} x^{\mu}\), and use
this to form the element of vector length along the path

\begin{equation*}
d\Br = \gamma_{\mu} \frac{d x^{\mu}}{d\lambda} d\lambda
\end{equation*}

Dotting the gradient and the path element we have
\begin{equation}\label{eqn:vectorIntegralRelations:20}
\begin{aligned}
\grad f \cdot d\Br
&= \left(\gamma^{\mu} \partial_{\mu} f\right) \cdot \left(\gamma_{\nu} \frac{d x^{\nu}}{d\lambda} \right) d\lambda \\
&= {\delta^{\mu}}_{\nu} \PD{x^{\mu}}{f} \frac{d x^{\nu}}{d\lambda} d\lambda \\
&= \sum \PD{x^{\mu}}{f} \frac{d x^{\mu}}{d\lambda} d\lambda \\
&= \frac{d f}{d \lambda} d\lambda
\end{aligned}
\end{equation}

\Eqnref{eqn:vector_integral_relations:lineintegral} follows immediately, which we see to be really not much more than the chain rule.

Additionally this can be put into correspondence with \eqnref{eqn:vector_integral_relations:stokesGA}, with the observation that one can write the gradient of a scalar function as a wedge product by the fundamental definition of wedge in terms of grade selection.  For blades \(A\) and \(B\) with grades \(a\) and \(b\) respectively, the wedge is

\begin{equation}\label{eqn:vectorIntegralRelations:40}
\begin{aligned}
A \wedge B = \gpgrade{AB}{a+b}
\end{aligned}
\end{equation}

Therefore for a scalar function \(f\)

\begin{equation}\label{eqn:vectorIntegralRelations:60}
\begin{aligned}
\grad \wedge f = \gpgrade{\grad f}{1+0} = \grad f
\end{aligned}
\end{equation}

Putting this back together one has the desired result

\begin{equation}\label{eqn:vector_integral_relations:lineintegralwedge}
\int_C (\grad \wedge f) \cdot d\Br = f \vert_{\partial C}
\end{equation}

\subsection{Motivating the non-orthonormal form of the gradient}

An additional note about the derivation of this line integral result.  Having done this with the gradient expressed for possibly non-orthonormal frames,
shows that if played backwards, it provides a nice motivation for the general form of the gradient, in terms
of a such a non-orthonormal basis.  That is a lot more obvious a way to get at this result than my previous way of observing that the Euler-Lagrange
equations when summed in vector form imply that this is the required form of the gradient.

\section{Jacobian area determinants}

Next in ease of proof is the Jacobian determinant.  This actually comes largely for free since we can utilize the wedge product to
express areas.

\imageFigure{../figures/gabook/planeParameterization}{Plane parametrization}{fig:planeParameterization}{0.4}

Introduce a two vector parametrization of the area as in \cref{fig:planeParameterization}

\begin{equation*}
\Br = \gamma_{i} \phi^i(u,v)
\end{equation*}

Provided that the partials are not collinear at the point of interest, we can compute the area of the parallelogram spanned by these

\begin{equation}\label{eqn:vectorIntegralRelations:80}
\begin{aligned}
d\BA
&= \left(\PD{u}{\Br} du\right) \wedge \left(\PD{v}{\Br} dv\right) \\
&= \left(\gamma_{i} \PD{u}{\phi^i}\right) \wedge \left(\gamma_{j} \PD{v}{\phi^j} \right) du dv \\
&= \gamma_{i} \wedge \gamma_{j} \PD{u}{\phi^i} \PD{v}{\phi^j} du dv \\
&= \sum_{i<j} \gamma_{i} \wedge \gamma_{j} \left( \PD{u}{\phi^i} \PD{v}{\phi^j} - \PD{u}{\phi^i} \PD{v}{\phi^i} \right) du dv \\
&= \sum_{i<j} \gamma_{i} \wedge \gamma_{j} \PD{(u,v)}{(\phi^i,\phi^j)} du dv \\
\end{aligned}
\end{equation}

Here \(d\BA\) is a bivector area element, so in the purely two dimensional case, where this is constrained to a plane, the scalar area element
is recovered by dividing by the plane unit pseudoscalar having the same orientation as this bivector.

One can also see how the same idea will be of use later in the Stokes' generalization of Green's theorem (considering a surface element small enough to be considered planar).

For now, considering just the 2D case we have, to divide through by the plane unit pseudoscalar \(i = \Be_1\Be_2\) produced by the product of two orthonormal vectors we want to calculate the product:

\begin{equation}\label{eqn:vectorIntegralRelations:100}
\begin{aligned}
\inv{i} \gamma_{1} \wedge \gamma_{2}
&= (\Be_2 \wedge \Be_1) \cdot (\gamma_{1} \wedge \gamma_{2}) \\
&= \Be_2 \cdot (\Be_1 \cdot (\gamma_{1} \wedge \gamma_{2})) \\
&= \Be_2 \cdot ( (\Be_1 \cdot \gamma_{1}) \gamma_{2} -(\Be_1 \cdot \gamma_{2}) \gamma_{1} ) \\
&= (\Be_1 \cdot \gamma_{1}) (\Be_2 \cdot \gamma_{2}) -(\Be_1 \cdot \gamma_{2}) (\Be_2 \cdot \gamma_{1}) \\
&=
\begin{vmatrix}
\Be_1 \cdot \gamma_{1} & \Be_1 \cdot \gamma_{2} \\
\Be_2 \cdot \gamma_{1} & \Be_2 \cdot \gamma_{2}
\end{vmatrix}
\end{aligned}
\end{equation}

Thus the (scalar) area element is

\begin{equation}\label{eqn:vector_integral_relations:jacobianframe}
\begin{aligned}
dA =
\begin{vmatrix}
\Be_1 \cdot \gamma_{1} & \Be_1 \cdot \gamma_{2} \\
\Be_2 \cdot \gamma_{1} & \Be_2 \cdot \gamma_{2}
\end{vmatrix}
\PD{(u,v)}{(\phi^1,\phi^2)} du dv
\end{aligned}
\end{equation}

This is a slightly more general form than we are used to seeing since the position vector parametrization was allowed to be expressed in terms
of an arbitrary (possibly non-orthonormal) basis.  Also observe that the coefficients in the determinant preceding the Jacobian are exactly those of the matrix of the linear transformation between the two sets of basis vectors.

\subsection{Orthonormal parametrization}

For the special (and usual) case of an orthonormal parametrization

\begin{equation}\label{eqn:vectorIntegralRelations:120}
\begin{aligned}
\Br = x(u,v) \Be_1 + y(u,v) \Be_2
\end{aligned}
\end{equation}

the product of determinants in \eqnref{eqn:vector_integral_relations:jacobianframe} takes the usual form

\begin{equation}\label{eqn:vector_integral_relations:jacobianarea}
\begin{aligned}
dx dy = \PD{(u,v)}{(x,y)} du dv.
\end{aligned}
\end{equation}

Now the danger of an expression like \eqnref{eqn:vector_integral_relations:jacobianarea} is that the differential notation for the determinant makes it seem almost
obvious.  Now, if you understand the wedge product origin you can state
that obviousness after a little bit of algebra.  However, in a book like Salus and Hille (used for Calculus I-III in UofT Engineering) they
can not even derive this two dimensional case til close to the end of the book, since they required Green's theorem to do so.  I had say that in that case it is not really so obvious.  The geometrical background just is not there.

Note that there are degrees of freedom to alter the sign given an arbitrary pseudoscalar.  This illustrates why the absolute value of the Jacobian determinant is used in some circumstances.  Less dodgy is to say the positive area element after change of variables in a specific region is produced by dividing out the pseudoscalar with the same orientation as the area element bivector.

It is also not too hard to see that this idea will also work for change of variables for volume and higher
dimensional volume elements, after wedging N partials.  We just have to divide by the spatial (or higher dimensional) pseudoscalar of the
same orientation associated with the parametrization.

\subsection{Surface area in higher dimensions}
\index{surface area}

As well as being able to use these ideas to express scalar area and volume, or higher dimensional generalizations, this can be used to calculate surface area in any number of dimensions.  For a two parameter vector parametrization of a surface \(\Br(u,v)\) we can write

\begin{equation}\label{eqn:vector_integral_relations:surfacearea}
A = \iint \inv{I_2(u,v)} \left(\PD{u}{\Br} \wedge \PD{v}{\Br}\right) du dv
\end{equation}

Here \(I_2(u,v)\) is the unit pseudoscalar for the tangent space of the surface at the point of interest with the orientation of the bivector
\(d\BA = \PD{u}{\Br} \wedge \PD{v}{\Br}\).

This is in fact equivalent to the familiar normal form in 3D expressed in terms
of a cross product

\begin{equation}
A = \iint \left(\PD{u}{\Br} \cross \PD{v}{\Br}\right) \cdot \ncap(u,v) du dv
\end{equation}

but the expression of \eqnref{eqn:vector_integral_relations:surfacearea}, holds for any number of dimensions \(N \ge 2\).  As with the wedge product form, we have a requirement that the parametrization is not degenerate at any point, so the 3D de-generalization of our requirement that
\(\PD{u}{\Br} \wedge \PD{v}{\Br} \ne 0\)
on the region of the surface of interest means that for 3D we simply require
\(\PD{u}{\Br} \cross \PD{v}{\Br} \ne 0\).

A consequence of non-degeneracy for the region of the surface area being integrated means that the sign of the bivector cannot change sign, so we have equivalence with the concept of outwards normal to the surface by picking the tangent space unit pseudoscalar to have the same orientation as the bivector area element.

\section{Green's theorem}

\subsection{Attempt to arrive at a more natural vector form for Green's theorem}
\index{Green's theorem}

It is pretty clear glancing at \eqnref{eqn:vector_integral_relations:greens}, that the left
hand side can likely be expressed as the curl of a vector.  By curl here
is meant the more natural bivector "curl", where we form the operator \(\grad \wedge\).

To get a feel for this operation, here is a dumb expansion of such a product,
where an orthonormal basis for the plane is assumed.  Introduce a vector

\begin{equation*}
\Bf = P \Be_1 + Q \Be_2
\end{equation*}

then compute
\begin{equation}\label{eqn:vectorIntegralRelations:140}
\begin{aligned}
\grad \wedge \Bf
&= \left( \Be_1 \partial_1 + \Be_2 \partial_2 \right) \wedge \left( \Be_1 P + \Be_2 Q \right) \\
&= (\Be_{1} \wedge \Be_2) \left( \partial_1 Q - \partial_2 P \right) \\
&= i \left( \partial_1 Q - \partial_2 P \right) \\
\end{aligned}
\end{equation}

This allows for writing the scalar alternating form as a vector relation

\begin{equation}\label{eqn:vector_integral_relations:altpartial}
\partial_1 Q - \partial_2 P = - i(\grad \wedge \Bf).
\end{equation}

Let us continue to put the Green's theorem area integral in complete vector form.
Since the area element can be expressed in vector form, introduce a vector parametrization \(\Br = x \Be_1 + y \Be_2\).  The element
of area expressed in terms of this parametrization is

\begin{equation*}
dA = \inv{i}\left( \PD{u}{\Br} \wedge \PD{v}{\Br} \right) du dv
\end{equation*}

Re-assembling the scalar alternating \eqnref{eqn:vector_integral_relations:altpartial}, we can put the area integral
completely in vector form

\begin{equation}\label{eqn:vector_integral_relations:greenlikeLHS}
\iint - i(\grad \wedge \Bf) \inv{i}\left( \PD{u}{\Br} \wedge \PD{v}{\Br} \right) du dv
= -\iint (\grad \wedge \Bf) \left( \PD{u}{\Br} \wedge \PD{v}{\Br} \right) du dv
\end{equation}

Considering the total differential of the position vector, it makes sense to introduce vector differential elements to
express this

\begin{equation*}
d\Br = \PD{u}{\Br} du + \PD{v}{\Br} dv = d\Bu + d\Bv
\end{equation*}

We can then rewrite \eqnref{eqn:vector_integral_relations:greenlikeLHS}
once more in a slightly cleaner form, independent of the specific parametrization

\begin{equation}\label{eqn:vector_integral_relations:greenlikeLHSFinal}
-\iint (\grad \wedge \Bf) \cdot (d\Bu \wedge d\Bv) = -\iint (\grad \wedge \Bf) \cdot d\BA
\end{equation}

Here we see that it becomes natural to work with the oriented bivector area element \(d\BA = d\Bu \wedge d\Bv\).

Having arrived at what is likely the most natural vector form \eqnref{eqn:vector_integral_relations:greenlikeLHSFinal} for the area
integral.

We should be able to integrate this in its most general form, dropping references to the original \(x\), and \(y\) coordinates.  If this is the correct form, we should end up with a vector line integral around a path after doing so, and thus prove Green's theorem.

FIXME: thought this, but am having trouble.  Will try from the loop integral instead.

\subsection{Expanding the area integral in terms of an arbitrary parametrization}

The integral expression of \eqnref{eqn:vector_integral_relations:greenlikeLHSFinal} is a form that can be examined independent of the original planar Green's theorem motivation.  Let us expand
this picking an arbitrary parametrization for both the area element and the vector.  There will also be no need for now to work with the original 2D vectors.

Given a two parameter vector parametrization of a surface, and a reciprocal frame representation of our curled vector:

\begin{equation}\label{eqn:vectorIntegralRelations:160}
\begin{aligned}
\Br(u,v) &= \gamma_i {x^i}(u,v) \\
\Bf &= \gamma^i {f_i}
\end{aligned}
\end{equation}

a bivector parallelogram surface element can be then be expressed as

\begin{equation}\label{eqn:vectorIntegralRelations:180}
\begin{aligned}
d\BA
&= \left(\PD{u}{\Br} du\right) \wedge \left(\PD{v}{\Br} dv\right) \\
&= \left(\gamma_i \wedge \gamma_j \right) \PD{u}{x^i} \PD{v}{x^j} du dv \\
\end{aligned}
\end{equation}

and our differential form is

\begin{equation}\label{eqn:vectorIntegralRelations:200}
\begin{aligned}
\left(\grad \wedge \Bf\right) \cdot d\BA
&= \left( \gamma^i \wedge \gamma^j \right) \cdot \left(\gamma_k \wedge \gamma_m \right) \PD{x^{i}}{f_j} \PD{u}{x^k} \PD{v}{x^m} du dv \\
&= \left( {\delta^i}_m {\delta^j}_k -{\delta^i}_k {\delta^j}_m \right) \PD{x^{i}}{f_j} \PD{u}{x^k} \PD{v}{x^m} du dv \\
&= \PD{x^{i}}{f_j} \left( \PD{u}{x^j} \PD{v}{x^i} -\PD{u}{x^i} \PD{v}{x^j} \right) du dv \\
&= -\PD{x^{i}}{f_j} \PD{(u,v)}{(x^i, x^j)} du dv \\
&= -\sum_{i<j} \left( \PD{x^{i}}{f_j} -\PD{x^{j}}{f_i} \right) \PD{(u,v)}{(x^i, x^j)} du dv \\
\end{aligned}
\end{equation}

The trailing differential form here is just the Jacobian form for change of variables, so we have

\begin{equation}\label{eqn:vector_integral_relations:areaintegral}
\begin{aligned}
-\left(\grad \wedge \Bf\right) \cdot d\BA
&= \sum_{i<j} \left( \PD{x^{i}}{f_j} -\PD{x^{j}}{f_i} \right) \PD{(u,v)}{(x^i, x^j)} du dv \\
&= \sum_{i<j} \left( \PD{x^{i}}{f_j} -\PD{x^{j}}{f_i} \right) dx^i dx^j
\end{aligned}
\end{equation}

A result that is independent of dimension or any particular parametrization of the area.

\subsection{Calculating the line integral}

The expectation is that calculation of the line integral

\begin{equation}
I = \ointctrclockwise \Bf \cdot d\Br,
\end{equation}

around any loop in a plane will match \eqnref{eqn:vector_integral_relations:areaintegral}.  This can
be verified with direct calculation.

FIXME: insert picture.

Again parametrizing the points around the loop with a vector \(\Br = \Br(u,v)\) the integral can be split into four parts

\begin{equation}\label{eqn:vectorIntegralRelations:220}
\begin{aligned}
I_1 &=  \int_{u=u_0}^{u_1} \Bf(\Br(u,v_0)) \cdot \PD{u}{\Br(u,v_0)} du \\
I_2 &=  \int_{v=v_0}^{v_1} \Bf(\Br(u_1,v)) \cdot \PD{v}{\Br(u_1,v)} dv \\
I_3 &= -\int_{u=u_0}^{u_1} \Bf(\Br(u,v_1)) \cdot \PD{u}{\Br(u,v_1)} du \\
I_4 &= -\int_{v=v_0}^{v_1} \Bf(\Br(u_0,v)) \cdot \PD{v}{\Br(u_0,v)} dv \\
\end{aligned}
\end{equation}

Summing these we have
\begin{equation}\label{eqn:vectorIntegralRelations:240}
\begin{aligned}
I
&= I_1 + I_3 + I_2 + I_4  \\
&=
\int_{u=u_0}^{u_1} du \left(
\Bf(\Br(u,v_0)) \cdot \PD{u}{\Br(u,v_0)}
-\Bf(\Br(u,v_1)) \cdot \PD{u}{\Br(u,v_1)}
\right) \\
&+\int_{v=v_0}^{v_1} dv
\left(
\Bf(\Br(u_1,v)) \cdot \PD{v}{\Br(u_1,v)}
-\Bf(\Br(u_0,v)) \cdot \PD{v}{\Br(u_0,v)}
\right)
\end{aligned}
\end{equation}

Writing out the vectors in components, utilizing reciprocal frames as in the area integral,
we have \(\Bf = \gamma^i f_i\), and \(\Br = \gamma_i x^i\) the dot products can be expanded and the sums
pulled out of the integral

\begin{equation}\label{eqn:vectorIntegralRelations:260}
\begin{aligned}
I
&=
\sum
\int_{u=u_0}^{u_1} du \left(
f_i(\Br(u,v_0)) \PD{u}{x^i(u,v_0)}
-f_i(\Br(u,v_1)) \PD{u}{x^i(u,v_1)}
\right) \\
&+\sum \int_{v=v_0}^{v_1} dv
\left(
f_i(\Br(u_1,v)) \PD{v}{x^i(u_1,v)}
-f_i(\Br(u_0,v)) \PD{v}{x^i(u_0,v)}
\right) \\
\end{aligned}
\end{equation}

The difference of functions here can be written as the integral of partials over the
\([u_0,u_1]\) or \([v_0,v_1]\) ranges.  This can be more obvious if one temporarily
introduces helper functions of one variable describing the difference (FIXME: do so like on paper notes).

Such an integration gives

\begin{equation}\label{eqn:vectorIntegralRelations:280}
\begin{aligned}
I
&=
\sum
\int_{u=u_0}^{u_1} du \left(
f_i(x^j(u,v_0)) \PD{u}{x^i(u,v_0)}
-f_i(x^j(u,v_1)) \PD{u}{x^i(u,v_1)}
\right) \\
&+\sum \int_{v=v_0}^{v_1} dv
\left(
f_i(x^j(u_1,v)) \PD{v}{x^i(u_1,v)}
-f_i(x^j(u_0,v)) \PD{v}{x^i(u_0,v)}
\right) \\
&=
\sum
-\int_{u=u_0}^{u_1} du
\int_{v=v_0}^{v_1} dv \PD{v}{} \left(f_i(x^j(u,v)) \PD{u}{x^i(u,v)} \right) \\
&\qquad +\sum
\int_{v=v_0}^{v_1} dv
\int_{u=u_0}^{u_1} du \PD{u}{} \left(f_i(x^j(u,v)) \PD{v}{x^i(u,v)} \right) \\
&=
\sum \iint \left(
\PD{u}{} \left(f_i \PD{v}{x^i} \right)
-\PD{v}{} \left(f_i \PD{u}{x^i} \right)
\right) du dv \\
&=
\sum \iint \left(
\PD{u}{f_i} \PD{v}{x^i}
+f_i \PD{u}{} \PD{v}{x^i}
-\PD{v}{f_i} \PD{u}{x^i}
-f_i \PD{v}{} \PD{u}{x^i}
\right)
du dv \\
&=
\sum \iint \left(
\PD{u}{f_i} \PD{v}{x^i}
-\PD{v}{f_i} \PD{u}{x^i}
\right)
du dv \\
\end{aligned}
\end{equation}

Sufficient continuity in the coordinates \(x^i\) has been assumed here for mixed partial equality.  Expanding out
the partials with respect to \(u\) and \(v\) in terms of the coordinates one has

\begin{equation}\label{eqn:vectorIntegralRelations:300}
\begin{aligned}
I
&=
\sum \iint
\PD{x^j}{f_i}
\left(
\PD{u}{x^j}
\PD{v}{x^i}
-
\PD{v}{x^j}
\PD{u}{x^i}
\right)
du dv \\
&= -\sum \iint \PD{x^j}{f_i} \PD{(u,v)}{(x^i,x^j)} du dv \\
\end{aligned}
\end{equation}

Summing over \(i<j\) and \(j>i\), with a switch of variables we have
\begin{equation}
I
=
\sum_{i<j} \iint
\left(
\PD{x^i}{f_j}
-\PD{x^j}{f_i}
\right)
\PD{(u,v)}{(x^i,x^j)} du dv
\end{equation}

which equals the area integral of \eqnref{eqn:vector_integral_relations:areaintegral}.  We have therefore proved a hybrid Green's-like
and Stokes-like theorem

\begin{equation}\label{eqn:vector_integral_relations:stokesplane}
\iint \left(\grad \wedge \Bf\right) \cdot d\BA = \ointclockwise \Bf \cdot d\Br
\end{equation}

Like the \R{2} Green's result this applies to a looping path integral in a plane, but this
form is valid for \(\Bf \in \Rm{N}\) as well.  In particular, like Stokes' law this this applies to \R{3}.

I set out only to prove
Green's theorem, but basically got the general proof without much extra work (using \(i,j\) instead of \(1,2\))
once the area integral was expressed as in terms of the wedge curl.

Note carefully that there is a
difference in the direction of the path integral compared to the cross product form of Stokes' law since
the squared bivector on the LHS introduces a negation.

To generalize this to a non-planar surface, the usual additional arguments to express a general
surface as a triangularized set of differential plane elements are required, with summation canceling opposing interior
contributions, is required to complete the proof.  The interesting (to me) part is the
plane to line integral Stokes law equation has been expressed in its \R{N} generality, and without
omission of the important loop orientation, and area sense.

\subsection{Application to formulate Stokes law for a plane loop}

We can obtain the \R{3} cross product form of Stokes law (assuming the triangularization generalization has been
done) with some basic algebraic manipulations.

Let \(\ncap d\BA = i dA\), where \(i = \Be_1\Be_2\Be_3\) is the \R{3} pseudoscalar.  Inserting back into
the differential form of the area integral of \eqnref{eqn:vector_integral_relations:stokesplane} we have

\begin{equation}\label{eqn:vectorIntegralRelations:320}
\begin{aligned}
\left(\grad \wedge \Bf\right) \cdot d\BA
&= \gpgradezero{ (\grad \wedge \Bf) \ncap i dA} \\
&= \gpgradezero{ i (\grad \cross \Bf) \ncap i dA} \\
&= -\gpgradezero{ (\grad \cross \Bf) \ncap } dA \\
&= - (\grad \cross \Bf) \cdot \ncap dA \\
\end{aligned}
\end{equation}

This recovers the cross product form of Stokes law

\begin{equation}
\iint (\grad \cross \Bf) \cdot \ncap dA = - \ointclockwise \Bf \cdot d\Br = \ointctrclockwise \Bf \cdot d\Br.
\end{equation}

Note that the surface here does not have to have any notion of outwards facing normal (this makes no sense for a plane for example), as is usually used
in the description of the \R{3} vector form of Stokes' law.
That means some care is required in the definition of the unit normal \(\ncap\).
There is however an orientation for this vector, and that is fixed by the
pseudoscalar.  Supposing that one picks \(\ncap\) such that \(\ncap d\BA \inv{i} = dA > 0\).  With such a selection, and
\(d\BA = \ucap \vcap dA\), the triplet of vectors is oriented such that \(\ncap \ucap \vcap = i\).  There
is no notion of handedness required, which is a very \R{3} concept, despite having a notion of
explicitly oriented vectors.

\subsection{Volume integral to Area integral}

Having calculated the scalar and vector variants of \eqnref{eqn:vector_integral_relations:stokesGA}, doing the same for the bivector case makes sense, especially since we need such a result for electromagnetism.

The aim will be to calculate

\begin{equation}\label{eqn:vector_integral_relations:volumetosolve}
\iiint (\grad \wedge F) \cdot d\BV = \pm \iint F \cdot d\BA
\end{equation}

where the relative orientation of the area elements has yet to be determined.

\subsubsection{Volume integral part}

Expanding the LHS using

\begin{equation}\label{eqn:vectorIntegralRelations:340}
\begin{aligned}
F &= \inv{2} F_{ij} \gamma^{ij} \\
\gamma^{ij} &= \gamma^i \wedge \gamma^j \\
\grad &= \gamma^i \partial_i \\
d\BV &= (\partial_u \Br \wedge \partial_v \Br \wedge \partial_w \Br) du dv dw \\
\Br &= x^i \gamma^i
\end{aligned}
\end{equation}

where the volume is a three variable \((u,v,w)\) parametrized parallelepiped volume element, we have

\begin{equation}\label{eqn:vectorIntegralRelations:360}
\begin{aligned}
(\grad \wedge F) \cdot d\BV
&=
\inv{2}
\gamma^{ijk} \cdot \gamma_{mnl}
\partial_i F_{jk}
\partial_u x^m
\partial_v x^n
\partial_w x^l
du dv dw
\\
\end{aligned}
\end{equation}

Expanding the dot product term we have

\begin{equation}\label{eqn:vectorIntegralRelations:380}
\begin{aligned}
\gamma^{ijk} \cdot \gamma_{mnl}
&= (((\gamma^{i} \wedge \gamma^j \wedge \gamma^k) \cdot \gamma_{m}) \cdot \gamma_n) \cdot \gamma_l \\
&=
(\gamma^{ij} {\delta^k}_m
-\gamma^{ik} {\delta^j}_m
+\gamma^{jk} {\delta^i}_m ) \cdot \gamma_{nl} \\
&= \delta^{i}_l {\delta^j}_n {\delta^k}_m -\delta^{j}_l {\delta^i}_n {\delta^k}_m \\
&- \delta^{i}_l {\delta^k}_n {\delta^j}_m +\delta^{k}_l {\delta^i}_n {\delta^j}_m \\
&+ \delta^{j}_l {\delta^k}_n {\delta^i}_m -\delta^{k}_l {\delta^j}_n {\delta^i}_m \\
\end{aligned}
\end{equation}

for short, write \(\partial^{ijk}_{uvw} = \partial_u x^i \partial_v x^j \partial_w x^k\).  Expanding the deltas in
%\(\partial_{uvw}^{mnl}\) term
\(\gamma^{ijk} \cdot \gamma_{mnl} \partial_{uvw}^{mnl}\)
we have

\begin{equation}\label{eqn:vectorIntegralRelations:400}
\begin{aligned}
\delta^{i}_l {\delta^j}_n {\delta^k}_m \partial_{uvw}^{mnl}
-\delta^{j}_l {\delta^i}_n {\delta^k}_m \partial_{uvw}^{mnl}
-\delta^{i}_l {\delta^k}_n {\delta^j}_m \partial_{uvw}^{mnl} \\
+\delta^{k}_l {\delta^i}_n {\delta^j}_m \partial_{uvw}^{mnl}
+\delta^{j}_l {\delta^k}_n {\delta^i}_m \partial_{uvw}^{mnl}
-\delta^{k}_l {\delta^j}_n {\delta^i}_m \partial_{uvw}^{mnl} \\
=
\partial_{u v w}^{k j i}
-\partial_{u v w}^{k i j}
-\partial_{u v w}^{j k i}
+\partial_{u v w}^{j i k}
+\partial_{u v w}^{i k j}
-\partial_{u v w}^{i j k} \\
\end{aligned}
\end{equation}
\begin{equation}\label{eqn:vectorIntegralRelations:420}
\begin{aligned}
\gamma^{ijk} \cdot \gamma_{mnl} \partial_{uvw}^{mnl}
&= \sum \epsilon_{i j k}\partial_{u v w}^{i j k} \\
&= \frac{\partial(x^i, x^j, x^k)}{\partial(u, v, w)}
\end{aligned}
\end{equation}

Therefore the final coordinate expression for the volume differential form is

\begin{equation}\label{eqn:vector_integral_relations:stokesvolume}
(\grad \wedge F) \cdot d\BV = -\inv{2} \partial_i F_{jk} \frac{\partial(x^i, x^j, x^k)}{\partial(u, v, w)} du dv dw.
\end{equation}

\subsubsection{Area integral part}

We next want to compare to an oriented area dot product

\begin{equation*}
\oiintctrclockwise F \cdot d\BA
\end{equation*}

What is meant by an oriented area element in such an integral?  I will try this calculation with loops drawn counterclockwise on each
face of a parallelepiped, such that the adjacent loops ``cancel out''.  This eliminates a requirement for an outward normal concept
which is only useful in \R{3}.  This is still a very geometric concept, and a good mathematical description independent of pictures
is required to get any further than a third degree volume subspace.  In fact, to formulate this description, and the results below, I have drawn arrows on a cube, and labeled these arrows \(\pm 1, \pm 2, \pm 3\) to enumerate them (also labeling the sides Front, Left, Right, Bottom, Top, and Posterior).

\imageFigure{../figures/gabook/cube_unfolded_004}{Oriented parallelepiped surfaces}{fig:cube_unfolded}{0.4}

With such a labeling we have the following table of paired area elements

\begin{itemize}
\item Front (\(w=w_0\)).  Posterior (\(w=w_1\)).

\begin{equation}\label{eqn:vectorIntegralRelations:440}
\begin{aligned}
d\BA_F &= \left( \PD{u}{\Br} \wedge \PD{v}{\Br} \right) du dv \\
d\BA_P &= \left( -\PD{u}{\Br} \wedge \PD{v}{\Br} \right) du dv
\end{aligned}
\end{equation}

\item Left (\(u=u_0\)).  Right (\(u=u_1\)).

\begin{equation}\label{eqn:vectorIntegralRelations:460}
\begin{aligned}
d\BA_L &= \left( -\PD{w}{\Br} \wedge \PD{v}{\Br} \right) dw dv \\
d\BA_R &= \left( \PD{w}{\Br} \wedge \PD{v}{\Br} \right) dw dv
\end{aligned}
\end{equation}

\item Top (\(v=v_1\)).  Bottom (\(v=v_0\)).

\begin{equation}\label{eqn:vectorIntegralRelations:480}
\begin{aligned}
d\BA_T &= \left( \PD{u}{\Br} \wedge \PD{w}{\Br} \right) dw dv \\
d\BA_B &= \left( \PD{u}{\Br} \wedge \left(-\PD{w}{\Br}\right) \right) dw dv
\end{aligned}
\end{equation}

\end{itemize}

With this particular enumeration of the oriented areas this geometrically described is a ``left handed'' triple \(\left\{ \PD{u}{\Br}, \PD{v}{\Br}, \PD{w}{\Br} \right\}\).

Summing the differential forms, writing our partial wedges in short like so \(\Br_{uv} = \PD{u}{\Br} \wedge \PD{v}{\Br}\), we have

\begin{equation}\label{eqn:vectorIntegralRelations:500}
\begin{aligned}
\sum F \cdot d\BA &=
(F \cdot \Br_{uv} \vert_{w=w_0}
-F \cdot \Br_{uv} \vert_{w=w_1}) du dv \\
&+(F \cdot \Br_{wv} \vert_{u=u_1}
-F \cdot \Br_{wv} \vert_{u=u_0}) dw dv \\
&+(F \cdot \Br_{uw} \vert_{v=v_1}
-F \cdot \Br_{uw} \vert_{v=v_0}) du dw
\end{aligned}
\end{equation}

\begin{equation}\label{eqn:vectorIntegralRelations:520}
\begin{aligned}
\implies
\oiintctrclockwise F \cdot d\BA
&= \iint du dv \int_{w=w_0}^{w_1} -\PD{w}{} (F \cdot \Br_{uv}) dw \\
&+\iint dw dv \int_{u=u_0}^{u_1} \PD{u}{} (F \cdot \Br_{wv}) du \\
&+\iint du dw \int_{v=v_0}^{v_1} \PD{v}{} (F \cdot \Br_{uw}) dv \\
&=
\iiint du dv dw
\left(
\PD{w}{F} \cdot \Br_{vu}
+\PD{u}{F} \cdot \Br_{wv}
+\PD{v}{F} \cdot \Br_{uw}
\right) \\
&+\iiint du dv dw
\, F \cdot \left( \PD{w}{ \Br_{vu} } +\PD{u}{ \Br_{wv} } +\PD{v}{ \Br_{uw} } \right) \\
\end{aligned}
\end{equation}

The last three terms here are expected to contribute zero.  Picking one to start

\begin{equation}\label{eqn:vectorIntegralRelations:540}
\begin{aligned}
\PD{w}{ \Br_{vu} }
&= \gamma^{ij} \PD{w}{} \left( \PD{v}{x^i} \PD{u}{x^j} \right) \\
&= \gamma^{ij} \left( \PDD{w}{v}{x^i} \PD{u}{x^j} +\PD{v}{x^i} \PDD{w}{u}{x^j} \right)
\end{aligned}
\end{equation}

and summing this and the rest
\begin{equation*}
\PD{w}{ \Br_{vu} } +\PD{u}{ \Br_{wv} } +\PD{v}{ \Br_{uw} }
\end{equation*}
\begin{equation}\label{eqn:vectorIntegralRelations:560}
\begin{aligned}
&=
\gamma^{ij} \left(
\PDD{w}{v}{x^i} \PD{u}{x^j} +\PD{v}{x^i} \PDD{w}{u}{x^j}
+\PDD{u}{w}{x^i} \PD{v}{x^j} +\PD{w}{x^i} \PDD{u}{v}{x^j}
+\PDD{v}{u}{x^i} \PD{w}{x^j} +\PD{u}{x^i} \PDD{v}{w}{x^j}
\right) \\
&=
\gamma^{ij} \left(
\left( \PDD{w}{v}{x^i} \PD{u}{x^j}
-\PD{u}{x^j} \PDD{v}{w}{x^i}\right)
+\left(\PD{v}{x^i} \PDD{w}{u}{x^j}
-\PDD{u}{w}{x^j} \PD{v}{x^i}\right)
+\left(\PD{w}{x^i} \PDD{u}{v}{x^j}
-\PDD{v}{u}{x^j} \PD{w}{x^i}\right)
\right) \\
&= 0,
\end{aligned}
\end{equation}

yields the expected zero, independent of \(F\).  This leaves our area integral as follows

\begin{equation}\label{eqn:vector_integral_relations:volumeareaAlmostDone}
\begin{aligned}
\implies
\oiintctrclockwise F \cdot d\BA
&=
-\iiint du dv dw
\left(
\PD{w}{F} \cdot \Br_{uv}
+\PD{u}{F} \cdot \Br_{vw}
+\PD{v}{F} \cdot \Br_{wu}
\right) \\
\end{aligned}
\end{equation}

To evaluate the sum to match with the volume integral we expand as in

\begin{equation}\label{eqn:vectorIntegralRelations:580}
\begin{aligned}
\PD{w}{F} \cdot \Br_{u v}
&= \inv{2} \PD{w}{F_{i j}} (\gamma^{ij} \cdot \gamma_{mn}) \partial_u x^m \partial_v x^n \\
&= \inv{2} \partial_w x^k \partial_{k}{F_{i j}} ( {\delta^{i}}_n {\delta^{j}}_m -{\delta^{j}}_n {\delta^{i}}_m ) \partial_u x^m \partial_v x^n \\
&= \inv{2} \partial_{k}{F_{i j}} ( {\delta^{i}}_n {\delta^{j}}_m -{\delta^{j}}_n {\delta^{i}}_m ) \partial_u x^m \partial_v x^n \partial_w x^k \\
&= \inv{2} \partial_{k}{F_{i j}} ( {\delta^{i}}_n {\delta^{j}}_m -{\delta^{j}}_n {\delta^{i}}_m ) \partial_{uvw}^{mnk} \\
\end{aligned}
\end{equation}

Summing the dot products in \eqnref{eqn:vector_integral_relations:volumeareaAlmostDone} we have
\begin{equation}\label{eqn:vectorIntegralRelations:600}
\begin{aligned}
&
( {\delta^{i}}_n {\delta^{j}}_m -{\delta^{j}}_n {\delta^{i}}_m )
\left(
\partial_u x^m \partial_v x^n \partial_w x^k
+\partial_v x^m \partial_w x^n \partial_u x^k
+\partial_w x^m \partial_u x^n \partial_v x^k
\right) \\
%&=
% {\delta^{i}}_n {\delta^{j}}_m \partial_u x^m \partial_v x^n \partial_w x^k
%+{\delta^{i}}_n {\delta^{j}}_m \partial_v x^m \partial_w x^n \partial_u x^k
%+{\delta^{i}}_n {\delta^{j}}_m \partial_w x^m \partial_u x^n \partial_v x^k \\
%&-{\delta^{j}}_n {\delta^{i}}_m \partial_u x^m \partial_v x^n \partial_w x^k
%-{\delta^{j}}_n {\delta^{i}}_m \partial_v x^m \partial_w x^n \partial_u x^k
%-{\delta^{j}}_n {\delta^{i}}_m \partial_w x^m \partial_u x^n \partial_v x^k \\
&=
 {\delta^{i}}_n {\delta^{j}}_m \partial_{uvw}^{m n k}
+{\delta^{i}}_n {\delta^{j}}_m \partial_{uvw}^{k m n}
+{\delta^{i}}_n {\delta^{j}}_m \partial_{uvw}^{n k m}
-{\delta^{j}}_n {\delta^{i}}_m \partial_{uvw}^{m n k}
-{\delta^{j}}_n {\delta^{i}}_m \partial_{uvw}^{k m n}
-{\delta^{j}}_n {\delta^{i}}_m \partial_{uvw}^{n k m} \\
%&=
% \partial_u x^j \partial_v x^i \partial_w x^k
%+\partial_v x^j \partial_w x^i \partial_u x^k
%+\partial_w x^j \partial_u x^i \partial_v x^k
%-\partial_u x^i \partial_v x^j \partial_w x^k
%-\partial_v x^i \partial_w x^j \partial_u x^k
%-\partial_w x^i \partial_u x^j \partial_v x^k \\
&= \partial_{uvw}^{j i k} +\partial_{uvw}^{k j i} +\partial_{uvw}^{i k j} -\partial_{uvw}^{i j k} -\partial_{uvw}^{k i j} -\partial_{uvw}^{j k i} \\
&= -\sum \epsilon_{ijk} \partial_{uvw}^{ijk} \\
&= - \frac{\partial(x^i, x^j, x^k)} {\partial(u,v,w)}
\end{aligned}
\end{equation}

Finally we can put the area integral back together

\begin{equation}\label{eqn:vectorIntegralRelations:620}
\begin{aligned}
\oiintctrclockwise F \cdot d\BA
&=
\iiint du dv dw
\inv{2} \partial_{k}{F_{i j}}
\frac{\partial(x^i, x^j, x^k)} {\partial(u,v,w)}
\end{aligned}
\end{equation}

A comparison to the volume \eqnref{eqn:vector_integral_relations:stokesvolume} shows a factor of \(-1\) difference, which completes the proof and fixes the orientation
of the surface area elements

\begin{equation}
\iiint (\grad \wedge F) \cdot d\BV = \oiintclockwise F \cdot d\BA
\end{equation}

\section{Divergence theorem}

The divergence theorem results follow directly from the Stokes variants after duality transformations.
Let us summarize all the Stokes equations proved so far to start

\begin{itemize}
\item \(f \in \Rm{1}\)

\begin{equation}\label{eqn:vector_integral_relations:summaryStokesLine}
\int_C (\grad \wedge f) \cdot d\Br = f \vert_{\partial C}
\end{equation}

\item \(f \in \Rm{N}\)

\begin{equation}\label{eqn:vector_integral_relations:summaryStokesArea}
\iint (\grad \wedge f) \cdot d\BA = \ointclockwise f \cdot d\Br
\end{equation}

It was demonstrated that Green's \eqnref{eqn:vector_integral_relations:greens} and the cross product Stokes \eqnref{eqn:vector_integral_relations:stokes} results are the \R{2} and \R{3} special cases of this respectively.

\item \(f \in {\bigwedge}^2 \Rm{N}\)

\begin{equation}\label{eqn:vector_integral_relations:summaryStokesVolume}
\iiint (\grad \wedge f) \cdot d\BV = \oiintclockwise f \cdot d\BA
\end{equation}
\end{itemize}

\subsection{Two variable divergence (Gauss's law)}

For the \R{2} divergence result we set \(f = I g\), for vectors \(f,g \in \Rm{2}\).  Calculating the area differential form of \eqnref{eqn:vector_integral_relations:summaryStokesArea} we have

\begin{equation}\label{eqn:vectorIntegralRelations:640}
\begin{aligned}
(\grad \wedge f ) \cdot d\BA
&= (\grad \wedge (I g) ) \cdot d\BA \\
&= \gpgradetwo{ \grad I g } \cdot d\BA \\
&= -\gpgradetwo{ I \grad g } \cdot d\BA \\
&= -\gpgradetwo{ I (\grad \cdot g + \grad \wedge g) } \cdot d\BA \\
&= - I (\grad \cdot g ) \cdot d\BA \\
&= - (\grad \cdot g ) I d\BA \\
\end{aligned}
\end{equation}

Expanding the line integral side of the equation we have
\begin{equation}\label{eqn:vectorIntegralRelations:660}
\begin{aligned}
f \cdot d\Br
&= (Ig) \cdot d\Br \\
&= \gpgradezero{ I g d\Br } \\
&= -\gpgradezero{ g I d\Br } \\
&= - g \cdot (I d\Br ) \\
\end{aligned}
\end{equation}

Therefore,
\begin{equation}\label{eqn:vectorIntegralRelations:680}
\begin{aligned}
\iint (\grad \cdot g ) I d\BA &= \ointclockwise g \cdot (I d\Br ) \\
\end{aligned}
\end{equation}

Letting \(dA = I d\BA\), and \(\ncap ds = I \cdot d\Br\), we have the usual form for two variable Gauss's
law:

\begin{equation}
\iint (\grad \cdot g ) dA = \ointclockwise g \cdot \ncap ds.
\end{equation}

Normally, the line integral side of this equation is expressed in terms of an outwards normal.  A specific
orientation for \(\ncap\) is also implied here, but depends on what bivector is used for the pseudoscalar \(I\)
and the metric for the space, so for generality this is left implicitly defined as above.

\subsection{Two variable divergence (Gauss's law)}

For a bivector \(F \in {\bigwedge}^2 \Rm{3}\), we introduce a duality defined vector \(f = IF\) (now \(I\) is an \R{3} pseudoscalar).

Expanding the two parts of the equation we have
\begin{equation}\label{eqn:vectorIntegralRelations:700}
\begin{aligned}
(\grad \wedge F) \cdot d\BV
&= (\grad \wedge (I f)) \cdot d\BV \\
&= \gpgradethree{\grad I f} \cdot d\BV \\
&= \gpgradethree{I \grad f} \cdot d\BV \\
&= \gpgradethree{I (\grad \cdot f + \grad \wedge f)} \cdot d\BV \\
&= I (\grad \cdot f) \cdot d\BV \\
&= (\grad \cdot f) I \cdot d\BV \\
&= (\grad \cdot f) (I d\BV) \\
\end{aligned}
\end{equation}

\begin{equation}\label{eqn:vectorIntegralRelations:720}
\begin{aligned}
F \cdot d\BA
&= (If) \cdot d\BA \\
&= \gpgradezero{ If d\BA } \\
&= \gpgradezero{ f (I d\BA) } \\
&= f \cdot (I \cdot d\BA ) \\
\end{aligned}
\end{equation}

With \(dV = I \cdot d\BV\), and \(\ncap d\sigma = I \cdot d\BA\) we have the \R{3} divergence equation

\begin{equation}
\iiint (\grad \cdot f) dV = \oiintclockwise f \cdot \ncap d\sigma
\end{equation}

\subsection{General divergence equation}

Now, if one assumes the correctness of the Stokes result for a k-blade \(T\) (proven here only for the scalar, vector, and bivector case) has the form

\begin{equation}\label{eqn:vector_integral_relations:generalStokes}
\int (\grad \wedge T) \cdot d^k \Bx = \int T \cdot d^{k-1} \Bx
\end{equation}

then the divergence result for vector \(\Bt = I T\) also follows without much more work than the two specific cases
above

\begin{equation}\label{eqn:vectorIntegralRelations:740}
\begin{aligned}
(\grad \wedge T) \cdot d^k \Bx
&= \gpgrade{\grad \inv{I} \Bt)}{k} \cdot d^k \Bx \\
&= (-1)^{k-1} \gpgrade{\inv{I} \grad \Bt}{k} \cdot d^k \Bx \\
&= (-1)^{k-1} \inv{I} (\grad \cdot \Bt) \cdot d^k \Bx \\
\end{aligned}
\end{equation}

\begin{equation}\label{eqn:vectorIntegralRelations:760}
\begin{aligned}
T \cdot d^{k-1} \Bx
&= \gpgradezero{\inv{I} \Bt d^{k-1} \Bx} \\
&= (-1)^{k-1} \Bt \cdot \left(\inv{I} d^{k-1} \Bx\right) \\
\end{aligned}
\end{equation}

therefore, with \(d^k x = I \cdot d^k \Bx\), and \(\ncap d^{k-1} x = I \cdot d^{k-1} \Bx\) we have

\begin{equation}\label{eqn:vector_integral_relations:generalDivergence}
\int (\grad \cdot \Bt) d^k x = \int \Bt \cdot \ncap d^{k-1} x
\end{equation}

Here \(\ncap\) is normal to all of the vectors in the span of the boundary surface (\(\ncap \wedge d^{k-1} \Bx \ne 0\)).  There are unspecified subtleties in \eqnref{eqn:vector_integral_relations:generalDivergence} since the orientation of the boundary in \eqnref{eqn:vector_integral_relations:generalStokes} has not been made explicit (doing so is likely the main part of the work
required to actually prove this result).

\subsection{Alternate duality identities}

The divergence equations have been seen to be consequences of Stokes theorem
\eqnref{eqn:vector_integral_relations:generalStokes}.  One can construct alternate variations that do not
have names that we are familiar with.

An example, consider \(f = IF\), in the following

\begin{equation*}
\iint (\grad \wedge f) \cdot d\BA = \ointclockwise f \cdot d\Br
\end{equation*}

for \(f \in \Rm{N}\).  For \(f \in \Rm{2}\), the object \(F\) is a vector, and
gives us the two variable divergence equation,
However, the other spaces produce other grade objects and some
unfamiliar formulas.
For example,
\(F\) is a bivector
in \R{3}
, and
is a trivector
in \R{4} \(F\)
, and so
forth.

\begin{equation}\label{eqn:vectorIntegralRelations:780}
\begin{aligned}
(\grad \wedge f) \cdot d\BA
&= \gpgradetwo{\grad IF} \cdot d\BA  \\
&= (-1)^{n-1} \gpgradetwo{I \grad F} \cdot d\BA  \\
&= (-1)^{n-1} \gpgradetwo{I (\mathLabelBox{\grad \cdot F}{\(n-2\) blade} +
\mathLabelBox
[
   labelstyle={below of=m\themathLableNode, below of=m\themathLableNode}
]
{\grad \wedge F}{\(n\) blade})} \cdot d\BA \\
&= (-1)^{n-1} (I (\grad \cdot F)) \cdot d\BA \\
&= (-1)^{n-1} \gpgradezero{I (\grad \cdot F) d\BA} \\
&= (-1)^{n-1} \gpgradezero{(\grad \cdot F) d\BA I} \\
&= (-1)^{n-1} (\grad \cdot F) \cdot (d\BA I) \\
\end{aligned}
\end{equation}

\begin{equation}\label{eqn:vectorIntegralRelations:800}
\begin{aligned}
f \cdot d\Br
&= \gpgradezero{ IF d\Br } \\
&= \gpgradezero{ F d\Br I } \\
&= (-1)^{n-1} \gpgradezero{ F I d\Br } \\
&= (-1)^{n-1} F \cdot (I d\Br ) \\
\end{aligned}
\end{equation}

This gives
\begin{equation}
\iint (\grad \cdot F) \cdot (d\BA I) = \ointclockwise F \cdot (I d\Br )
\end{equation}

As an example, for \R{3} where \(F\) is a bivector, \(d\BA I\) is a two variable parametrized vector ``line'' element (say \(d\Bs\)), and \(I d\Br\) is a single variable parametrized bivector ``surface'' element (say \(d\BB\)).  This gives us a peculiar (or unfamiliar) looking duality generated identity

\begin{equation*}
\iint (\grad \cdot F) \cdot d\Bs = \int F \cdot d\BB
\end{equation*}

\subsection{Summary remarks}

This mostly completes the aim of this examination.
There are a couple of identities still to derive in the intro table (equations \eqnref{eqn:vector_integral_relations:divergencegrad}, and \eqnref{eqn:vector_integral_relations:divergencegradcross}).
Everything else has been
shown to be a special case of the fundamental exterior derivative integral \eqnref{eqn:vector_integral_relations:stokesGA}, and we have assigned specific meanings and
orientations to the spatial volume elements on each side of that equation for the scalar, vector and bivector cases.  This may be enough for
electromagnetism calculations, unless a trivector and four-space result is also required.
