%
% Copyright © 2016 Peeter Joot.  All Rights Reserved.
% Licenced as described in the file LICENSE under the root directory of this GIT repository.
%

\makeexample{Application to Maxwell's equation}{example:fundamentalTheoremOfCalculus:1}{
Maxwell's equation is an example of a first order gradient equation
\begin{equation}\label{eqn:fundamentalTheoremOfCalculus:320}
\grad F = \inv{\epsilon_0 c} J.
\end{equation}

Integrating over a four-volume (where the vector derivative equals the gradient), and applying the Fundamental theorem, we have
\begin{equation}\label{eqn:fundamentalTheoremOfCalculus:340}
\inv{\epsilon_0 c} \int d^4 x J = \oint d^3 x F.
\end{equation}

Observe that the surface area element product with \( F \) has both vector and trivector terms.  This can be demonstrated by considering some examples
\begin{equation}\label{eqn:fundamentalTheoremOfCalculus:360}
\begin{aligned}
\gamma_{012} \gamma_{01} &\propto \gamma_2 \\
\gamma_{012} \gamma_{23} &\propto \gamma_{023}.
\end{aligned}
\end{equation}

On the other hand, the four volume integral of \( J \) has only trivector parts.  This means that the integral can be split into a pair of same-grade equations
\begin{equation}\label{eqn:fundamentalTheoremOfCalculus:380}
\begin{aligned}
\inv{\epsilon_0 c} \int d^4 x \cdot J &=
\oint \gpgradethree{ d^3 x F} \\
0 &=
\oint d^3 x \cdot F.
\end{aligned}
\end{equation}

The first can be put into a slightly tidier form using a duality transformation
\begin{equation}\label{eqn:fundamentalTheoremOfCalculus:400}
\begin{aligned}
\gpgradethree{ d^3 x F}
&=
-\gpgradethree{ d^3 x I^2 F} \\
&=
\gpgradethree{ I d^3 x I F} \\
&=
(I d^3 x) \wedge (I F).
\end{aligned}
\end{equation}

Letting \( n \Abs{d^3 x} = I d^3 x \), this gives
\begin{equation}\label{eqn:fundamentalTheoremOfCalculus:420}
\oint \Abs{d^3 x} n \wedge (I F) = \inv{\epsilon_0 c} \int d^4 x \cdot J.
\end{equation}

Note that this normal is normal to a three-volume subspace of the spacetime volume.  For example, if one component of that spacetime surface area element is \( \gamma_{012} c dt dx dy \), then the normal to that area component is \( \gamma_3 \).

A second set of duality transformations
\begin{equation}\label{eqn:fundamentalTheoremOfCalculus:440}
\begin{aligned}
n \wedge (IF)
&=
\gpgradethree{ n I F} \\
&=
-\gpgradethree{ I n F} \\
&=
-\gpgradethree{ I (n \cdot F)} \\
&=
-I (n \cdot F),
\end{aligned}
\end{equation}
and
\begin{equation}\label{eqn:fundamentalTheoremOfCalculus:460}
\begin{aligned}
I d^4 x \cdot J
&=
\gpgradeone{ I d^4 x \cdot J } \\
&=
\gpgradeone{ I d^4 x J } \\
&=
\gpgradeone{ (I d^4 x) J } \\
&=
(I d^4 x) J.
\end{aligned}
\end{equation}

We can further tidy things up, leaving us with
%\begin{dmath}\label{eqn:fundamentalTheoremOfCalculus:500}
\boxedEquation{eqn:fundamentalTheoremOfCalculus:500}{
\begin{aligned}
\oint \Abs{d^3 x} n \cdot F &= \inv{\epsilon_0 c} \int (I d^4 x) J \\
\oint d^3 x \cdot F &= 0.
\end{aligned}
}
%\end{dmath}

The Fundamental theorem of calculus immediately provides relations between normal projections of the Faraday bivector \( F \) and the four-current \( J \), as well as boundary value constraints on \( F \) coming from the source free components of Maxwell's equation.
} % example
