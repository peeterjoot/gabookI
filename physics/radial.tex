%
% Copyright � 2012 Peeter Joot.  All Rights Reserved.
% Licenced as described in the file LICENSE under the root directory of this GIT repository.
%

%
%
\chapter{Polar velocity and acceleration}
\index{velocity!polar}
\index{acceleration!polar}
\label{chap:radial}
%\date{Jan 13, 2009.  radial.tex}

\section{Motivation}

Have previously worked out the radial velocity and acceleration components a pile of different ways in
\chapcite{PJAngAcc},
\chapcite{PJAngAccCross},
\chapcite{PJAngVel},
\chapcite{PJKeRot},
\chapcite{PJRadialDer}, and
\chapcite{PJUnitDer}.

So, what is a couple more?

When the motion is strictly restricted to a plane we can get away with doing this either in complex numbers
(used in a number of the Tong Lagrangian solutions), or with a polar form \R{2} vector (a polar representation
I have not seen since High School).

\section{With complex numbers}

Let
\begin{equation}\label{eqn:radial:20}
\begin{aligned}
z = r e^{i\theta}
\end{aligned}
\end{equation}

So our velocity is

\begin{equation}\label{eqn:radial:40}
\begin{aligned}
\zdot = \rdot e^{i\theta} + i r \thetadot e^{i\theta}
\end{aligned}
\end{equation}

and the acceleration is
\begin{equation}\label{eqn:radial:60}
\begin{aligned}
\ddot{z}
&= \ddot{r} e^{i\theta} + i \dot{r} \thetadot e^{i\theta}
 + i \rdot \thetadot e^{i\theta}
 + i r \ddot{\theta} e^{i\theta}
 - r \thetadot^2 e^{i\theta} \\
&= (\ddot{r} - r \thetadot^2 ) e^{i\theta} + (2 \dot{r} \thetadot + r \ddot{\theta} ) i e^{i\theta}
\end{aligned}
\end{equation}

\section{Plane vector representation}

Also can do this with polar vector representation directly (without involving the complexity of rotation matrices or anything fancy)

\begin{equation}\label{eqn:radial:80}
\begin{aligned}
\Br
&= r
\begin{bmatrix}
\cos\theta \\
\sin\theta
\end{bmatrix}
\end{aligned}
\end{equation}

Velocity is then
\begin{equation}\label{eqn:radial:100}
\begin{aligned}
\Bv
&=
\rdot
\begin{bmatrix}
\cos\theta \\
\sin\theta
\end{bmatrix}
+r \thetadot
\begin{bmatrix}
-\sin\theta \\
\cos\theta
\end{bmatrix}
\end{aligned}
\end{equation}

and for acceleration we have

\begin{equation}\label{eqn:radial:120}
\begin{aligned}
\Ba
&=
\ddot{r}
\begin{bmatrix}
\cos\theta \\
\sin\theta
\end{bmatrix}
+\rdot \thetadot
\begin{bmatrix}
-\sin\theta \\
\cos\theta
\end{bmatrix}
+\rdot \thetadot
\begin{bmatrix}
-\sin\theta \\
\cos\theta
\end{bmatrix}
+r \ddot{\theta}
\begin{bmatrix}
-\sin\theta \\
\cos\theta
\end{bmatrix}
-r \thetadot^2
\begin{bmatrix}
\cos\theta \\
\sin\theta
\end{bmatrix} \\
&=
(\ddot{r} -r \thetadot^2)
\begin{bmatrix}
\cos\theta \\
\sin\theta
\end{bmatrix}
+(2\rdot \thetadot +r \ddot{\theta})
\begin{bmatrix}
-\sin\theta \\
\cos\theta
\end{bmatrix}
\end{aligned}
\end{equation}
