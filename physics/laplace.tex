%
% Copyright � 2012 Peeter Joot.  All Rights Reserved.
% Licenced as described in the file LICENSE under the root directory of this GIT repository.
%

%
%
%\chapter{Exponential Solutions to Laplace Equation in \texorpdfstring{\R{N}}{ND}}
\index{Laplace equation}
\label{chap:laplace}
%\date{Feb 28, 2008.  laplace.tex}

\section{The problem}

Want solutions of

\begin{equation}\label{eqn:gaLaplacianSol:laplacian}
\laplacian f = \sum_k \dsqxj{f}{k} = 0
\end{equation}

For real f.

\subsection{One dimension}

Here the problem is easy, just integrate twice:

\begin{equation}\label{eqn:laplace:22}
f = cx + d.
\end{equation}

\subsection{Two dimensions}

For the two dimensional case we want to solve:

\begin{equation}\label{eqn:laplace:42}
\dsqxj{f}{1} + \dsqxj{f}{2} = 0
\end{equation}

Using separation of variables one can find solutions of the form \(f = X(x_1)Y(x_2)\).  Differentiating we have:

\begin{equation}\label{eqn:laplace:62}
X''Y + XY'' = 0
\end{equation}

So, for \(X \ne 0\), and \(Y \ne 0\):
\begin{equation}\label{eqn:laplace:82}
\frac{X''}{X} = -\frac{Y''}{Y} = k^2
\end{equation}

\begin{equation}\label{eqn:laplace:102}
\implies
X = e^{kx}
\end{equation}
\begin{equation}\label{eqn:laplace:122}
Y = e^{k\Bi y}
\end{equation}

\begin{equation}\label{eqn:laplace:142}
\implies
f = XY = e^{k(x + \Bi y)}
\end{equation}

Here \(\Bi\) is anything that squares to -1.  Traditionally this is the
complex unit imaginary, but we are also free to use a geometric product unit bivector such as \(\Bi = \Be_1 \wedge \Be_2 = \Be_1\Be_2 = \Be_{12}\), or \(\Bi = \Be_{21}\).

With \(\Bi = \Be_{12}\) for example we have:

\begin{equation}\label{eqn:laplace:282}
\begin{aligned}
f = XY = e^{k(x + \Bi y)}
&= e^{k(x + \Be_{12} y)} \\
&= e^{k(x\Be_{1}\Be_1 + \Be_{12} y)} \\
&= e^{k\Be_1(x\Be_1 + \Be_2 y)} \\
\end{aligned}
\end{equation}

Writing \(\Bx = \sum x_i \Be_i\), all of the following are solutions
of the Laplacian

\begin{equation}\label{eqn:laplace:302}
\begin{aligned}
e^{k\Be_1\Bx} \\
e^{\Bx k\Be_1} \\
e^{k\Be_2\Bx} \\
e^{\Bx k\Be_2} \\
\end{aligned}
\end{equation}

Now there is not anything special about the use of the x and y axis so it is reasonable to expect that, given any constant vector \(\Bk\),
the following may also be solutions to the two dimensional Laplacian problem

\begin{equation}\label{eqn:gaLaplacianSol:expgeo1}
e^{\Bx\Bk} = e^{\Bx \cdot \Bk + \Bx \wedge \Bk}
\end{equation}
\begin{equation}\label{eqn:gaLaplacianSol:expgeo2}
e^{\Bk\Bx} = e^{\Bx \cdot \Bk - \Bx \wedge \Bk}
\end{equation}

\subsection{Verifying it is a solution}

To verify that equations \eqnref{eqn:gaLaplacianSol:expgeo1} and \eqnref{eqn:gaLaplacianSol:expgeo2} are Laplacian solutions, start with taking the first order partial with one of the coordinates.
Since there are conditions where this form of solution works in \R{N},
a two dimensional Laplacian will not be assumed here.

\begin{equation}\label{eqn:laplace:162}
\dxj{}{j}e^{\Bx\Bk}
\end{equation}

This can be evaluated without any restrictions, but introducing the restriction that the bivector part of \(\Bx\Bk\)
is coplanar with its derivative simplifies the result considerably.  That is introduce a restriction:

\begin{equation}\label{eqn:laplace:182}
\gpgradetwo{ \Bx \wedge \Bk \dxj{\Bx \wedge \Bk}{j} } = \gpgradetwo{ \Bx \wedge \Bk \Be_j \wedge \Bk } = 0
\end{equation}

With such a restriction we have

\begin{equation}\label{eqn:laplace:202}
\dxj{}{j}e^{\Bx\Bk} = \Be_j\Bk e^{\Bx\Bk} = e^{\Bx\Bk} \Be_j\Bk
\end{equation}

Now, how does one enforce a restriction of this form in general?  Some thought will show that one way to do so
is to require that
both \(\Bx\) and \(\Bk\) have only two components.  Say, components \(j\), and \(m\).  Then, summing second partials
we have:

\begin{equation}\label{eqn:laplace:322}
\begin{aligned}
\sum_{u=j,m}\dsqxj{}{u}e^{\Bx\Bk}
&= \left( \Be_j\Bk \Be_j\Bk + \Be_m\Bk \Be_m\Bk \right) e^{\Bx\Bk} \\
&= \left( \Be_j\Bk (-\Bk\Be_j + 2 \Bk \cdot \Be_j) + \Be_m\Bk (-\Bk\Be_m + 2 \Be_m \cdot \Bk) \right) e^{\Bx\Bk} \\
&= \left( -2\Bk^2 + 2 k_j^2 + 2 k_m k_j \Be_{jm} + 2 k_m^2 + 2 k_j k_m \Be_{mj} \right) e^{\Bx\Bk} \\
&= \left( -2\Bk^2 + 2 \Bk^2 + 2 k_j k_m (\Be_{mj} + \Be_{jm}) \right) e^{\Bx\Bk} \\
&= 0 \\
\end{aligned}
\end{equation}

This proves the result, but essentially just says that this form of
solution is only
valid when the constant parametrization vector \(\Bk\) and \(\Bx\) and its
variation are restricted to a specific plane.  That result could have
been obtained in much simpler ways, but I learned a lot about bivector
geometry in the approach! (not all listed here since it caused serious
digressions)

\subsection{Solution for an arbitrarily oriented plane}

Because the solution above is coordinate free, one would expect that this
works for any solution that is restricted to the plane with bivector \(\Bi\)
even when those do not line up with any specific pair of two coordinates.
This can be verified by performing a rotational
coordinate transformation of the
Laplacian operator, since one can always pick a pair of mutually orthogonal
basis vectors with corresponding coordinate vectors that lie in the plane
defined by such a bivector.

Given two arbitrary vectors in the space when both are projected onto the plane
with constant bivector \(\Bi\) their product is:

\begin{equation}\label{eqn:laplace:222}
\left(\Bx \cdot \Bi \inv{\Bi}\right)\left(\inv{\Bi} \Bi \cdot \Bk\right)
=
(\Bx \cdot \Bi)(\Bk \cdot \Bi)
\end{equation}

Thus one can express the general equation for a planar solution to the
homogeneous Laplace equation in the form

\begin{equation}
\exp((\Bx \cdot \Bi)(\Bk \cdot \Bi))
=
\exp((\Bx \cdot \Bi) \cdot (\Bk \cdot \Bi) +
     (\Bx \cdot \Bi) \wedge (\Bk \cdot \Bi) )
\end{equation}

\subsection{Characterization in real numbers}

Now that it has been verified that equations \eqnref{eqn:gaLaplacianSol:expgeo1} and \eqnref{eqn:gaLaplacianSol:expgeo2} are solutions
of \eqnref{eqn:gaLaplacianSol:laplacian} let us characterize this in terms of real numbers.

If \(\Bx\), and \(\Bk\) are colinear, the solution has the form

\begin{equation}
e^{\pm\Bx \cdot \Bk}
\end{equation}

(ie: purely hyperbolic solutions).

Whereas with \(\Bx\) and \(\Bk\) orthogonal we have can employ the unit bivector for the plane spanned by these vectors
\(\Bi = \frac{\Bx \wedge \Bk}{\abs{\Bx \wedge \Bk}}\):

\begin{equation}
e^{\pm\Bx \wedge \Bk} = \cos\abs{\Bx \wedge \Bk} \pm \Bi\sin\abs{\Bx \wedge \Bk}
\end{equation}

Or:
\begin{equation}
e^{\pm\Bx \wedge \Bk} = \cos\left(\frac{\Bx \wedge \Bk}{\Bi}\right) \pm \Bi\sin\left(\frac{\Bx \wedge \Bk}{\Bi}\right)
\end{equation}

(ie: purely trigonometric solutions)

Provided \(\Bx\), and \(\Bk\) are not colinear, the wedge product component of the above can be written in terms of a unit bivector
\(\Bi = \frac{\Bx \wedge \Bk}{\abs{\Bx \wedge \Bk}}\):

\begin{equation}\label{eqn:laplace:342}
\begin{aligned}
e^{\Bx\Bk} &= e^{\Bx \cdot \Bk + \Bx \wedge \Bk} \\
&= e^{\Bx \cdot \Bk} \left( \cos{\abs{\Bx \wedge \Bk}} + \Bi \sin{\abs{\Bx \wedge \Bk}} \right) \\
&= e^{\Bx \cdot \Bk} \left( \cos\left(\frac{\Bx \wedge \Bk}{\Bi}\right) + \Bi \sin\left(\frac{\Bx \wedge \Bk}{\Bi}\right) \right) \\
\end{aligned}
\end{equation}

And, for the reverse:
\begin{equation}\label{eqn:laplace:362}
\begin{aligned}
(e^{\Bx\Bk})^\dagger = e^{\Bk\Bx}
&= e^{\Bx \cdot \Bk} \left( \cos{\abs{\Bx \wedge \Bk}} - \Bi \sin\left(\abs{\Bx \wedge \Bk}\right) \right) \\
&= e^{\Bx \cdot \Bk} \left( \cos\left(\frac{\Bx \wedge \Bk}{\Bi}\right) - \Bi \sin\left(\frac{\Bx \wedge \Bk}{\Bi}\right) \right) \\
\end{aligned}
\end{equation}

This exponential however has both scalar and bivector parts, and we are looking for a strictly scalar result, so we can use linear combinations of the
exponential and its reverse to form a strictly real sum for the \(\Bx \wedge \Bk \ne 0\) cases:

\begin{equation}\label{eqn:laplace:382}
\begin{aligned}
\inv{2}\left(e^{\Bx\Bk} + e^{\Bk\Bx}\right) = e^{\Bx\cdot\Bk}\cos\left(\frac{\Bx \wedge \Bk}{\Bi}\right) \\
\inv{2\Bi}\left(e^{\Bx\Bk} - e^{\Bk\Bx}\right) = e^{\Bx\cdot\Bk}\sin{\frac{\Bx \wedge \Bk}{\Bi}} \\
\end{aligned}
\end{equation}

Also note that further linear combinations (with positive and negative variations of \(\Bk\)) can be taken, so we can
combine equations \eqnref{eqn:gaLaplacianSol:expgeo1} and \eqnref{eqn:gaLaplacianSol:expgeo2} into the following real valued, coordinate free, form:

\begin{equation}\label{eqn:laplace:402}
\begin{aligned}
\cosh(\Bx\cdot\Bk)\cos\left({\frac{\Bx \wedge \Bk}{\Bi}}\right) \\
\sinh(\Bx\cdot\Bk)\cos\left({\frac{\Bx \wedge \Bk}{\Bi}}\right) \\
\cosh(\Bx\cdot\Bk)\sin\left({\frac{\Bx \wedge \Bk}{\Bi}}\right) \\
\sinh(\Bx\cdot\Bk)\sin\left({\frac{\Bx \wedge \Bk}{\Bi}}\right)
\end{aligned}
\end{equation}

Observe that the ratio \(\frac{\Bx \wedge \Bk}{\Bi}\) is just a scalar
determinant

\begin{equation}\label{eqn:laplace:242}
\frac{\Bx \wedge \Bk}{\Bi}
=
x_j k_m - x_m k_j
\end{equation}

So one is free to choose \(k' = k_m \Be_j - k_j \Be_m\), in which case the
solution takes the alternate form:

\begin{equation}\label{eqn:laplace:422}
\begin{aligned}
\cos(\Bx\cdot\Bk')\cosh\left({\frac{\Bx \wedge \Bk'}{\Bi}}\right) \\
\sin(\Bx\cdot\Bk')\cosh\left({\frac{\Bx \wedge \Bk'}{\Bi}}\right) \\
\cos(\Bx\cdot\Bk')\sinh\left({\frac{\Bx \wedge \Bk'}{\Bi}}\right) \\
\sin(\Bx\cdot\Bk')\sinh\left({\frac{\Bx \wedge \Bk'}{\Bi}}\right)
\end{aligned}
\end{equation}

These sets of equations and the exponential form both remove the explicit reference to the pair of coordinates used in the original restriction

\begin{equation}\label{eqn:laplace:262}
\gpgradetwo{ \Bx \wedge \Bk \Be_j \wedge \Bk } = 0
\end{equation}

that was used in the proof that \(e^{\Bx\Bk}\) was a solution.
