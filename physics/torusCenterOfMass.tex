%
% Copyright � 2012 Peeter Joot.  All Rights Reserved.
% Licenced as described in the file LICENSE under the root directory of this GIT repository.
%

%
%
%\input{../peeter_prologue_print.tex}
%\input{../peeter_prologue_widescreen.tex}

\chapter{Center of mass of a toroidal segment}
\index{center of mass!toroidal segment}
\label{chap:torusCenterOfMass}

%\blogpage{http://sites.google.com/site/peeterjoot/math2010/torusCenterOfMass.pdf}
%\date{May 15, 2010}
%\revisionInfo{torusCenterOfMass.tex}

%\beginArtWithToc
\beginArtNoToc

\section{Motivation}

In \href{http://samjshah.com/2010/05/05/i-love-when-kids-stump-me/}{I love when kids stump me}, the center of mass of a toroidal segment is desired, and the simpler problem of a circular ring segment is considered.

Let us try the solid torus problem for fun using the geometric algebra toolbox.  To setup the problem, it seems reasonable to introduce two angle, plus radius, toroidal parametrization as shown in \cref{fig:toriodalSegment}.

\imageFigure{../../physicsplay/figures/gabook/toriodalSegment}{Toroidal parametrization}{fig:toriodalSegment}{0.5}

Our position vector to a point within the torus is then

\begin{subequations}
\begin{align}\label{eqn:torusCenterOfMass:1}
\Br(\rho, \theta, \phi) &= e^{-j\theta/2} \left( \rho \Be_1 e^{ i \phi } + R \Be_3 \right) e^{j \theta/2} \\
i &= \Be_1 \Be_3 \\
j &= \Be_3 \Be_2
\end{align}
\end{subequations}

Here \(i\) and \(j\) for the bivectors are labels picked at random.  They happen to have the quaternion-ic properties \(i j = -j i\), and \(i^2 = j^2 = -1\) which can be verified easily.

\section{Volume element}

Before we can calculate the center of mass, we will need the volume element.  I do not recall having ever seen such a volume element, so let us calculate it from scratch.

We want

\begin{align}\label{eqn:torusCenterOfMass:2}
dV = \pm \Be_1 \Be_2 \Be_3 \left( \PD{\rho}{\Br} \wedge \PD{\theta}{\Br} \wedge \PD{\phi}{\Br} \right) d\rho d\theta d\phi,
\end{align}

so the first order of business is calculation of the partials.  After some regrouping those are

\begin{subequations}
\begin{align}\label{eqn:torusCenterOfMass:3}
\PD{\rho}{\Br} &= e^{-j\theta/2} \Be_1 e^{ i \phi } e^{j \theta/2} \\
\PD{\theta}{\Br}
%&= e^{-j\theta/2} \left( \rho \inv{2} \left( -\Be_3 \Be_2 \Be_1 e^{ i \phi } + \Be_1 e^{ i \phi } \Be_3 \Be_2 \right) + R \Be_2 \right) e^{j \theta/2} \\
&= e^{-j\theta/2} \left( R + \rho \sin\phi \right) \Be_2 e^{j \theta/2} \\
\PD{\phi}{\Br} &= e^{-j\theta/2} \rho \Be_3 e^{ i \phi } e^{j \theta/2}.
\end{align}
\end{subequations}

For the volume element we want the wedge of each of these, and can instead select the trivector grades of the products, which conveniently wipes out a number of the interior exponentials

\begin{align}\label{eqn:torusCenterOfMass:4}
\PD{\rho}{\Br} \wedge \PD{\theta}{\Br} \wedge \PD{\phi}{\Br}
&=
\rho \left( R + \rho \sin\phi \right) \gpgradethree{ e^{-j\theta/2} \Be_1 e^{ i \phi } \Be_2 \Be_3 e^{ i \phi } e^{j \theta/2} }
\end{align}

Note that \(\Be_1\) commutes with \(j = \Be_3 \Be_2\), so also with \(e^{-j\theta/2}\).  Also \(\Be_2 \Be_3 = -j\) anticommutes with \(i\), so we have a conjugate commutation effect \(e^{i\phi} j = j e^{-i\phi}\).  Together the trivector grade selection reduces almost magically to just

\begin{align}\label{eqn:torusCenterOfMass:5}
\PD{\rho}{\Br} \wedge \PD{\theta}{\Br} \wedge \PD{\phi}{\Br}
&=
\rho \left( R + \rho \sin\phi \right) \Be_1 \Be_2 \Be_3
\end{align}

Thus the volume element, after taking the positive sign, is

\begin{align}\label{eqn:torusCenterOfMass:6}
dV = \rho \left( R + \rho \sin\phi \right) d\rho d\theta d\phi.
\end{align}

As a check we should find that we can use this to calculate the volume of the complete torus, and obtain the expected \(V = (2 \pi R) (\pi r^2)\) result.  That volume is

\begin{align}\label{eqn:torusCenterOfMass:7}
V = \int_{\rho=0}^r \int_{\theta=0}^{2\pi} \int_{\phi=0}^{2\pi} \rho \left( R + \rho \sin\phi \right) d\rho d\theta d\phi.
\end{align}

The sine term conveniently vanishes over the \(2\pi\) interval, leaving just
\begin{align}\label{eqn:torusCenterOfMass:8}
V = \inv{2} r^2 R (2 \pi)(2 \pi),
\end{align}

as expected.

\section{Center of mass}

With the prep done, we are ready to move on to the original problem.  Given a toroidal segment over angle \(\theta \in [-\Delta \theta/2, \Delta \theta/2]\), then the volume of that segment is

\begin{align}\label{eqn:torusCenterOfMass:9}
\Delta V = r^2 R \pi \Delta \theta.
\end{align}

Our center of mass position vector is then located at

\begin{align}\label{eqn:torusCenterOfMass:10}
\BR \Delta V
&=
\int_{\rho=0}^r \int_{\theta=-\Delta \theta/2}^{\Delta \theta/2} \int_{\phi=0}^{2\pi}
e^{-j\theta/2} \left( \rho \Be_1 e^{ i \phi } + R \Be_3 \right) e^{j \theta/2}
\rho \left( R + \rho \sin\phi \right) d\rho d\theta d\phi.
\end{align}

Evaluating the \(\phi\) integrals we loose the \(\int_0^{2\pi} e^{i\phi}\) and \(\int_0^{2\pi} \sin\phi\) terms and are left with \(\int_0^{2\pi} e^{i\phi} \sin\phi d\phi = i \pi /2\) and \(\int_0^{2\pi} d\phi = 2 \pi\).  This leaves us with

\begin{align}\label{eqn:torusCenterOfMass:11}
\BR \Delta V
&=
\int_{\rho=0}^r \int_{\theta=-\Delta \theta/2}^{\Delta \theta/2}
\left( e^{-j\theta/2} \rho^3 \Be_3 \frac{\pi}{2} e^{j \theta/2} + 2 \pi \rho R^2 \Be_3 e^{j \theta}  \right) d\rho d\theta \\
&=
\int_{\theta=-\Delta \theta/2}^{\Delta \theta/2}
\left( e^{-j\theta/2} r^4 \Be_3 \frac{\pi}{8} e^{j \theta/2} + 2\pi \inv{2} r^2 R^2 \Be_3 e^{j \theta}  \right) d\theta \\
&=
\int_{\theta=-\Delta \theta/2}^{\Delta \theta/2}
\left( e^{-j\theta/2} r^4 \Be_3 \frac{\pi}{8} e^{j \theta/2} + \pi r^2 R^2 \Be_3 e^{j \theta}  \right) d\theta.
\end{align}

Since \(\Be_3 j = -j \Be_3\), we have a conjugate commutation with the \(e^{-j \theta/2}\) for just

\begin{align}\label{eqn:torusCenterOfMass:12}
\BR \Delta V
&=
\pi r^2 \left( \frac{r^2}{8} + R^2 \right) \Be_3
\int_{\theta=-\Delta \theta/2}^{\Delta \theta/2}
e^{j \theta} d\theta \\
&=
\pi r^2 \left( \frac{r^2}{8} + R^2 \right) \Be_3
2 \sin(\Delta \theta/2).
\end{align}

A final reassembly, provides the desired final result for the center of mass vector

\begin{align}\label{eqn:torusCenterOfMass:13}
\BR &= \Be_3 \inv{R} \left( \frac{r^2}{8} + R^2 \right) \frac{\sin(\Delta \theta/2)}{ \Delta \theta/2 }.
\end{align}

Presuming no algebraic errors have been made, how about a couple of sanity checks to see if the correctness of this seems plausible.

We are pointing in the \(z\)-axis direction as expected by symmetry.  Good.  For \(\Delta \theta = 2 \pi\), our center of mass vector is at the origin.  Good, that is also what we expected.  If we let \(r \rightarrow 0\), and \(\Delta \theta \rightarrow 0\), we have \(\BR = R \Be_3\) as also expected for a tiny segment of ``wire'' at that position.  Also good.

\section{Center of mass for a circular wire segment}

As an additional check for the correctness of the result above, we should be able to compare with the center of mass of a circular wire segment, and get the same result in the limit \(r \rightarrow 0\).

For that we have

\begin{align}\label{eqn:torusCenterOfMass:14}
Z (R \Delta \theta) = \int_{\theta=-\Delta \theta/2}^{\Delta \theta/2} R i e^{-i\theta} R d\theta
\end{align}

So we have

\begin{align}\label{eqn:torusCenterOfMass:15}
Z
&= \inv{\Delta \theta} R i \inv{-i} (e^{-i\Delta \theta/2} - e^{i\Delta\theta/2}).
\end{align}

Observe that this is

\begin{align}\label{eqn:torusCenterOfMass:16}
Z &= R i \frac{\sin(\Delta\theta/2)}{\Delta\theta/2},
\end{align}

which is consistent with the previous calculation for the solid torus when we let that solid diameter shrink to zero.

In particular, for \(3/4\) of the torus, we have \(\Delta \theta = 2 \pi (3/4) = 3 \pi/2\), and

\begin{align}\label{eqn:torusCenterOfMass:17}
Z = R i \frac{4 \sin(3\pi/4)}{3 \pi} = R i \frac{2 \sqrt{2}}{3 \pi} \approx 0.3 R i.
\end{align}

We are a little bit up the imaginary axis as expected.

I had initially somehow thought I had been off by a factor of two compared to \href{http://samjshah.com/2010/05/05/i-love-when-kids-stump-me/#comment-2349}{the result by The Virtuosi}, without seeing a mistake in either.  But that now appears not to be the case, and I just screwed up plugging in the numbers.  Once again, I should go to my eight year old son when I have arithmetic problems, and restrict myself to just the calculus and algebra bits.

%%\EndArticle
%\EndNoBibArticle
