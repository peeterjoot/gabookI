%
% Copyright � 2012 Peeter Joot.  All Rights Reserved.
% Licenced as described in the file LICENSE under the root directory of this GIT repository.
%

%
%
%\chapter{Hyper complex numbers and symplectic structure}
\index{hypercomplex numbers}
\index{symplectic structure}
\label{chap:complex}
%\date{November 8, 2008.  complex.tex}

\section{On 4.2 Hermitian Norms and Unitary Groups}

These are some rather rough notes filling in some details
on the treatment of \citep{DoranHamiltonian}.

Expanding equation 4.17
%
\begin{equation}\label{eqn:complex:20}
\begin{aligned}
J &= e_i \wedge f_i \\
a &= u_i e_i + v_i f_i \\
b &= x_i e_i + y_i f_i \\
B &= a \wedge b + (a \cdot J) \wedge (b \cdot J)  \\
\end{aligned}
\end{equation}
%
\begin{equation}\label{eqn:complex:40}
\begin{aligned}
a \wedge b
&= (u_i e_i + v_i f_i) \wedge (x_j e_j + y_j f_j) \\
&=
u_i x_j e_i \wedge e_j
+ u_i y_j e_i \wedge f_j
+ v_i x_j f_i \wedge e_j
+ v_i y_j f_i \wedge f_j \\
\end{aligned}
\end{equation}
%
\begin{equation}\label{eqn:complex:60}
\begin{aligned}
a \cdot J
&=
u_i e_i \cdot ( e_j \wedge f_j )
+ v_i f_i \cdot ( e_j \wedge f_j ) \\
&= u_j f_j - v_j e_j
\end{aligned}
\end{equation}
%
Search and replace for \(b \cdot J\) gives
%
\begin{equation}\label{eqn:complex:80}
\begin{aligned}
b \cdot J
&=
x_i e_i \cdot ( e_j \wedge f_j )
+ y_i f_i \cdot ( e_j \wedge f_j ) \\
&= x_j f_j - y_j e_j
\end{aligned}
\end{equation}
%
So we have
%
\begin{equation}\label{eqn:complex:100}
\begin{aligned}
(a \cdot J) \wedge (b \cdot J)
&= (u_i f_i - v_i e_i) \wedge (x_j f_j - y_j e_j) \\
&=
 u_i x_j f_i \wedge f_j
-u_i y_j f_i \wedge e_j
- v_i x_j e_i \wedge f_j
+ v_i y_j e_i \wedge e_j
\end{aligned}
\end{equation}
%
For
\begin{equation}\label{eqn:complex:120}
\begin{aligned}
a \wedge b + (a \cdot J) \wedge (b \cdot J)
&=
 ( u_i y_j - v_i x_j ) (e_i \wedge f_j - f_i \wedge e_j)
+ ( u_i x_j + v_i y_j ) (e_i \wedge e_j + f_i \wedge f_j)
\end{aligned}
\end{equation}
%
This shows why the elements were picked as a basis
\begin{equation}\label{eqn:complex:140}
\begin{aligned}
e_i \wedge f_j - f_i \wedge e_j
\end{aligned}
\end{equation}
\begin{equation}\label{eqn:complex:160}
\begin{aligned}
e_i \wedge e_j + f_i \wedge f_j
\end{aligned}
\end{equation}
%
The first of which is a multiple of \(J_i = e_i \wedge f_i\) when \(i=j\), and the second of which is zero if \(i=j\).

\section{5.1 Conservation Theorems and Flows}

equation 5.10 is
%
\begin{equation}\label{eqn:complex:180}
\begin{aligned}
\fdot = \xdot \cdot \grad f = (\grad f \wedge \grad H) \cdot J
\end{aligned}
\end{equation}
%
This one is not obvious to me.  For \(\fdot\) we have
%
\begin{equation}\label{eqn:complex:200}
\begin{aligned}
\fdot = \PD{p_i}{f} \pdot_i +\PD{q_i}{f} \qdot_i + \mathLabelBox{\PD{t}{f}}{\(=0\)}
\end{aligned}
\end{equation}
%
compare to
%
\begin{equation}\label{eqn:complex:220}
\begin{aligned}
\xdot \cdot \grad f
&= (\pdot_i e_i + \qdot_i f_i) \cdot (e_j \PD{p_j}{f} + f_j \PD{q_j}{f}) \\
&= \pdot_i \PD{p_i}{f} + \qdot_i \PD{q_i}{f}
\end{aligned}
\end{equation}
%
Okay, this part matches the first part of (5.10).  Writing this in terms of the Hamiltonian relation (5.9) \(\xdot = \grad H \cdot J\) we have
%
\begin{equation}\label{eqn:complex:240}
\begin{aligned}
\fdot
&= (\grad H \cdot J) \cdot \grad f \\
&= \grad f \cdot (\grad H \cdot J) \\
\end{aligned}
\end{equation}
%
The relation \(a \cdot (b \cdot (c \wedge d)) = (a \wedge b) \cdot (c \wedge d)\),
can be used here to factor out the \(J\), we have
\begin{equation}\label{eqn:complex:260}
\begin{aligned}
\fdot
&= \grad f \cdot (\grad H \cdot J) \\
&= (\grad f \wedge \grad H) \cdot J \\
\end{aligned}
\end{equation}
%
which completes (5.10).

Also with \(f=H\) since H was also specified as having no explicit time dependence, one has
%
\begin{equation}\label{eqn:complex:280}
\begin{aligned}
\dot{H} &= (\grad H \wedge \grad H) \cdot J = 0 \cdot J = 0
\end{aligned}
\end{equation}
%
%\bibliographystyle{plainnat}
%\bibliography{myrefs}

%\end{document}
