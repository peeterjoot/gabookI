%
% Copyright � 2012 Peeter Joot.  All Rights Reserved.
% Licenced as described in the file LICENSE under the root directory of this GIT repository.
%

%
%
\chapter{Angular Velocity and Acceleration (Again)}\label{chap:PJAngAcc}
\index{angular velocity}
\index{angular acceleration}
%\date{June 10, 2008.  angularAcc.tex}

A more coherent derivation of angular velocity and acceleration than
my initial attempt while first learning geometric algebra.

\section{Angular velocity}

The goal is to take first and second derivatives of a vector expressed radially:

\begin{equation}
\Br = r \rcap.
\end{equation}

The velocity is the derivative of our position vector, which in terms of radial components is:

\begin{equation}\label{eqn:angular_acc:velocityasrcapprime}
\Bv = \Br' = r' \rcap + r \rcap'.
\end{equation}

We can also calculate the projection and rejection of the velocity by multiplication by \(1 = \rcap^2\), and expanding
this product in an alternate order taking advantage of the associativity of the geometric product:

\begin{equation}\label{eqn:angularAcc:380}
\begin{aligned}
\Bv &= \rcap \rcap \Bv \\
    &= \rcap \left ( \rcap \cdot \Bv + \rcap \wedge \Bv \right) \\
\end{aligned}
\end{equation}

Since \(\rcap \wedge (\rcap \wedge \Bv) = 0\), the total velocity in terms of radial components is:

\begin{equation}\label{eqn:angular_acc:velocityprojrej}
\Bv = \rcap \left(\rcap \cdot \Bv\right) + \rcap \cdot \left(\rcap \wedge \Bv\right).
\end{equation}

Here the first component above is the projection of the vector in the radial direction:

\begin{equation}\label{eqn:angularAcc:20}
\Proj_{\Br}(\Bv) = \rcap \left(\rcap \cdot \Bv\right)
\end{equation}

This projective term can also be rewritten in terms of magnitude:

\begin{equation}\label{eqn:angularAcc:40}
\left(r^2\right)' = 2 r r' = \left(\Br \cdot \Br\right)' = 2 \Br \cdot \Bv.
\end{equation}

So the magnitude variation can be expressed the radial coordinate of the velocity:

\begin{equation}\label{eqn:angular_acc:rprime}
r' = \rcap \cdot \Bv
\end{equation}

The remainder is the rejection of the radial component from the velocity, leaving just the part
portion perpendicular to the radial direction.

\begin{equation}\label{eqn:angularAcc:60}
\RejName_{\Br}(\Bv) = \rcap \cdot \left(\rcap \wedge \Bv\right)
\end{equation}

It is traditional to introduce an angular velocity vector normal to the plane of rotation
that describes this rejective component using a triple cross product.  With the formulation above,
one can see that it is more natural to directly use an angular velocity bivector instead:

\begin{equation}
\BOmega = \frac{\Br \wedge \Bv}{r^2}
\end{equation}

This bivector encodes the
angular velocity as a plane directly.  The
product of a vector with the bivector that contains it produces another vector
in the plane.  That product is a scaled and rotated by 90 degrees, much like the
multiplication by a unit complex imaginary.  That is no coincidence since
the square of a bivector is negative and directly encodes this complex structure
of an arbitrarily oriented plane.

Using this angular velocity bivector we have the following radial expression for velocity:

\begin{equation}\label{eqn:angular_acc:velocityomega}
\Bv = \rcap r' + \Br \cdot \BOmega.
\end{equation}

A little thought will show that \(\rcap'\) is also entirely perpendicular to \(\rcap\).  The \(\rcap\) vector describes
a path traced out on the unit sphere, and any variation of that vector must be tangential to the sphere.
It is thus not surprising that we can also express \(\rcap'\) using the rejective term of equation
\eqnref{eqn:angular_acc:velocityprojrej}.  Using the angular velocity bivector this is:

\begin{equation}\label{eqn:angular_acc:rcapprime}
\rcap' = \rcap \cdot \BOmega.
\end{equation}

This identity will be useful below for the calculation of angular acceleration.

\section{Angular acceleration}

Next we want the second derivatives of position

\begin{equation}\label{eqn:angularAcc:400}
\begin{aligned}
\Ba
&= \Br'' \\
&= r'' \rcap + 2r' \rcap' + r \rcap'' \\
&= r'' \rcap + \inv{r}\left( r^2 \rcap' \right)' \\
\end{aligned}
\end{equation}

This last step I found scribbled in a margin note in
my old mechanics book.  It is a trick that somebody clever once noticed and it simplifies this derivation to use it
since it avoids the generation of a number of terms that will just cancel out anyways after more tedious manipulation
(see examples section).

Expanding just this last derivative:

\begin{equation}\label{eqn:angularAcc:420}
\begin{aligned}
\left( r^2 \rcap' \right)'
&= \left( r^2 \rcap \cdot \BOmega \right)' \\
&= \left( \rcap \cdot \left(\Br \wedge \Bv\right) \right)' \\
&= \left( \rcap \cdot \left(\Br \wedge \Bv\right) \right)' \\
&= \rcap' \cdot \left(\Br \wedge \Bv\right) +\rcap \cdot (\mathLabelBox{\Bv \wedge \Bv}{\(=0\)}) + \rcap \cdot \left(\Br \wedge \Ba\right) \\
\end{aligned}
\end{equation}

Thus the acceleration is:
\begin{equation*}
\Ba = r'' \rcap + \left(\Br \cdot \BOmega\right) \cdot \BOmega + \rcap \cdot \left(\rcap \wedge \Ba\right)
\end{equation*}


Note that the action of taking two successive dot products with the plane bivector \(\BOmega\) just acts to rotate the
vector by 180 degrees (as well as scale it).

One can verify this explicitly using grade selection operators.  This allows the total acceleration to be expressed
in the final form:

\begin{equation*}
\Ba = r'' \rcap + \Br \BOmega^2 + \rcap \cdot \left(\rcap \wedge \Ba\right)
\end{equation*}

Note that the squared bivector \(\BOmega^2\) is a negative scalar, so the first two terms are radially directed.
The last term is perpendicular to the acceleration, in the plane formed by the vector and its second derivative.

Given the acceleration, the force on a particle is thus:

\begin{equation}\label{eqn:angularAcc:80}
\BF = m\Ba = m\rcap r'' + m \Br \BOmega^2 + \frac{\Br}{r^2} \left(\Br \wedge \Bp\right)'
\end{equation}

Writing the angular momentum as:

\begin{equation}\label{eqn:angularAcc:100}
\BL = \Br \wedge \Bp = m r^2 \BOmega
\end{equation}
%m \BOmega^2 = \BL^2/m r^4

the force is thus:

\begin{equation}\label{eqn:angularAcc:120}
\BF = m\Ba = m\rcap r'' + m \Br \BOmega^2 + \frac{1}{\Br} \cdot \frac{d\BL}{dt}
\end{equation}

The derivative of the angular momentum is called the torque \(\Btau\), also a bivector:

\begin{equation}\label{eqn:angularAcc:140}
\Btau = \frac{d\BL}{dt}
\end{equation}


When \(\Br\) is constant this has the radial arm times force form that we expect of torque:

\begin{equation}\label{eqn:angularAcc:160}
\Btau = \Br \wedge \frac{d \Bp}{dt} = \Br \wedge \BF
\end{equation}

%We can also write the equation of motion in terms of angular momentum and torque, in which case we have:
%
%\[
%\BF = m\rcap r'' + \inv{m\Br^3} \BL^2 + \frac{1}{\Br} \cdot \Btau
%\]
We can also write the equation of motion in terms of torque, in which case we have:

\begin{equation}\label{eqn:angularAcc:180}
\BF = m\rcap r'' + m \Br \Omega^2 + \frac{1}{\Br} \cdot \Btau
\end{equation}

As with all these plane quantities (angular velocity, momentum, acceleration), the torque as well is a bivector as it is natural to express this as a planar quantity.  This makes
more sense in many ways than a cross product, since all of these quantities should be perfectly well defined in a plane (or in spaces of degree greater than three), whereas the
cross product is a strictly three dimensional entity.

\section{Expressing these using traditional form (cross product)}

To compare with traditional results to see if I got things right, remove the geometric algebra constructs
(wedge products and bivector/vector products) in favor of cross products.  Do this by
using the duality relationships, multiplication by the three dimensional pseudoscalar
\(i = \Be_1\Be_2\Be_3\), to convert bivectors to vectors and wedge products to cross and dot products
(\(\Bu \wedge \Bv = \Bu \cross \Bv i\)).

First define some vector quantities in terms of the corresponding bivectors:

\begin{equation}\label{eqn:angularAcc:200}
\Bomega = \BOmega / i = \frac{\Br \wedge \Bv}{r^2 i} = \frac{\Br \cross \Bv}{r^2}
\end{equation}

\begin{equation}\label{eqn:angularAcc:220}
\Br \cdot \BOmega = \inv{2}(\Br \Bomega i - \Bomega i \Br ) = \Br \wedge \Bomega i = \Bomega \cross \Br
\end{equation}

Thus the velocity is:

\begin{equation}\label{eqn:angularAcc:240}
\Bv = \rcap r' + \Bomega \cross \Br.
\end{equation}

In the same way, write the angular momentum vector as the dual of the angular momentum bivector:

\begin{equation}\label{eqn:angularAcc:260}
\Bl = \BL /i = \Br \cross \Bp = m r^2 \Bomega
\end{equation}

And the torque vector \(\BN\) as the dual of the torque bivector \(\Btau\)

\begin{equation}\label{eqn:angularAcc:280}
\BN = \Btau /i = \frac{d\Bl}{dt} = \frac{d}{dt} \left(\Br \cross \Bp \right)
\end{equation}

The equation of motion for a single particle then becomes:

% r . t = r . N i = 1/2(r N i - N i r) = r ^ N i = r x N i^2 = N x r
\begin{equation}\label{eqn:angularAcc:300}
\BF = m\rcap r'' - m \Br \norm{\Bomega}^2 + \BN \cross \frac{\Br}{r^2}
\end{equation}

\section{Examples (perhaps future exercises?)}

\subsection{Unit vector derivative}
\index{derivative!unit vector}

Demonstrate by direct calculation the result of \eqnref{eqn:angular_acc:rcapprime}.

\begin{equation}\label{eqn:angularAcc:440}
\begin{aligned}
\rcap'
&= \left(\frac{\Br}{r}\right)' \\
&= \frac{\Br'}{r} - \frac{\Br r'}{r^2} \\
&= \inv{r} \left( \Bv - \rcap \left(\rcap \cdot \Bv \right) \right) \\
&= \frac{\rcap}{r} \left( \rcap \Bv - \rcap \cdot \Bv \right) \\
&= \frac{\rcap}{r} \left( \rcap \wedge \Bv \right) \\
\end{aligned}
\end{equation}

\subsection{Direct calculation of acceleration}
\index{acceleration}

It is more natural to calculate this acceleration directly by taking derivatives of \eqnref{eqn:angular_acc:velocityomega}, but as noted above this is messier.  Here is exactly that calculation for
comparison.

Taking second derivatives of the velocity we have:

\begin{equation}\label{eqn:angularAcc:320}
\Bv' = \Ba = \left(\rcap r' + \frac{\Br}{r^2} \left(\Br \wedge \Bv\right)\right)'
\end{equation}

%\rcap' = \inv{r^3} \Br \Br \wedge \Bv
\begin{equation}\label{eqn:angularAcc:460}
\begin{aligned}
\Ba
&= \rcap' r' + \rcap r'' + \frac{\Br}{r^2}
\mathLabelBox
[
   labelstyle={below of=m\themathLableNode, below of=m\themathLableNode}
]
{\left(\Bv \wedge \Bv\right)}{\(=0\)} + \frac{\Br}{r^2} \left(\Br \wedge \Ba\right) + \left(\frac{\rcap}{r}\right)' \left(\Br \wedge \Bv\right) \\
&=
\rcap r''
+\rcap'\left( r'  + \frac{1}{r} \Br \wedge \Bv \right)
- r' \frac{\rcap}{r^2} \left(\Br \wedge \Bv\right)
+ \rcap \left(\rcap \wedge \Ba\right)  \\
&=
\rcap r''
+\inv{r^3} \Br\left( \Br \wedge \Bv\right) \left( r'  + \frac{1}{r} \Br \wedge \Bv \right)
- r' \frac{\rcap}{r^2} \left(\Br \wedge \Bv\right)
+ \rcap \left(\rcap \wedge \Ba\right)  \\
\end{aligned}
\end{equation}

The \(r'\) terms cancel out, leaving just:

\begin{equation}\label{eqn:angularAcc:340}
\Ba = \rcap r'' + \Br \BOmega^2 +
\rcap \left(\rcap \wedge \Ba\right)
\end{equation}

\subsection{Expand the omega omega triple product}

\begin{equation}\label{eqn:angularAcc:480}
\begin{aligned}
\left(\Br \cdot \BOmega\right) \cdot \BOmega
&= \gpgradeone{ \left(\Br \cdot \BOmega\right) \BOmega } \\
&= \inv{2} \gpgradeone{ \Br \BOmega^2 - \BOmega \Br \BOmega } \\
&= \inv{2} \Br \BOmega^2 - \inv{2}\gpgradeone{ \BOmega \Br \BOmega } \\
&= \inv{2} \Br \BOmega^2 + \inv{2}\gpgradeone{ \Br \BOmega \BOmega } \\
&= \inv{2} \Br \BOmega^2 + \inv{2}\Br \BOmega^2 \\
&= \Br \BOmega^2 \\
\end{aligned}
\end{equation}

Also used above implicitly was the following:

\begin{equation}\label{eqn:angularAcc:360}
\Br \BOmega = \Br \cdot \BOmega + \mathLabelBox{\Br \wedge \BOmega}{\(=0\)} = -\BOmega \cdot \Br = -\BOmega \Br
\end{equation}

(ie: a vector anticommutes with a bivector describing a plane that contains it).
