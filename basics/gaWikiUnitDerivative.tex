%
% Copyright � 2012 Peeter Joot.  All Rights Reserved.
% Licenced as described in the file LICENSE under the root directory of this GIT repository.
%

%
%
%\mychapter{Derivatives of a unit vector}
%\label{chap:PJUnitDer}
\index{unit vector!derivative}
%\date{Oct 16, 2007.  gaWikiUnitDerivative.tex}

\section{First derivative of a unit vector}

\subsection{Expressed with the cross product}

It can be shown that a unit vector derivative can be expressed using the cross product.  Two cross product operations are required to get the result back into the plane of the rotation, since a unit vector is constrained to circular (really perpendicular to itself) motion.

\begin{equation}\label{eqn:gaWikiUnitDerivative:20}
\dt{}\left(\frac{\Br}{\Vert \Br \Vert}\right)
= \frac{1}{{\Vert \Br \Vert}^3}\left(\Br \times \dt{\Br}\right) \times \Br
= \left(\rcap \times \frac{1}{{\Vert \Br \Vert}} \dt{\Br}\right) \times \rcap
\end{equation}

This derivative is the rejective component of \(\dt{\Br}\) with respect to \(\rcap\), but is scaled by \(1/\Vert \Br \Vert\).

How to calculate this result can be found in other places, such as
\citep{salas1990coa}.

\section{Equivalent result utilizing the geometric product}

The equivalent geometric product result can be obtained by calculating the derivative of a vector \(\Br = r \rcap\).

\begin{equation}\label{eqn:gaWikiUnitDerivative:40}
\dt{\Br} = r \dt{\rcap} + \rcap \dt{r}
\end{equation}

\subsection{Taking dot products}
One trick is required first (as was also the case in the Salus and Hille derivation), which is expressing \(\dt{r}\) via the dot product.

\begin{equation}\label{eqn:gaWikiUnitDerivative:120}
\begin{aligned}
\dt{(r^2)} &= 2r \dt{r} \\
\dt{(\Br \cdot \Br)} &= 2 \Br \cdot \dt{\Br} \\
\end{aligned}
\end{equation}

Thus,
\begin{equation}\label{eqn:gaWikiUnitDerivative:60}
\dt{r} = \rcap \cdot \dt{\Br}
\end{equation}

Taking dot products of the derivative above yields

\begin{equation}\label{eqn:gaWikiUnitDerivative:140}
\begin{aligned}
\rcap \cdot \dt{\Br} &= \rcap \cdot r \dt{\rcap} + \rcap \cdot \rcap \dt{r} \\
                            &= \Br \cdot \dt{\rcap} + \dt{r} \\
                            &= \Br \cdot \dt{\rcap} + \rcap \cdot \dt{\Br}
\end{aligned}
\end{equation}

\begin{equation}\label{eqn:gaWikiUnitDerivative:80}
\implies
\Br \cdot \dt{\rcap} = \Bzero
\end{equation}

One could alternatively prove this with a diagram.


\subsection{Taking wedge products}

As in linear equation solution, the \(\rcap\) component can be eliminated by taking a wedge product

\begin{equation}\label{eqn:gaWikiUnitDerivative:160}
\begin{aligned}
\rcap \wedge \dt{\Br} &= \rcap \wedge r \dt{\rcap} + \rcap \wedge \rcap \dt{r} \\
                             &= r \rcap \wedge \dt{\rcap} \\
                             &= \Br \wedge \dt{\rcap}  \\
                             &= \Br \wedge \dt{\rcap} + \Br \cdot \dt{\rcap} \\
                             &= \Br \dt{\rcap}
\end{aligned}
\end{equation}

This allows expression of \(\dt{\rcap}\) in terms of \(\dt{\Br}\) in various ways (compare to the cross product results above)

\begin{equation}\label{eqn:gaWikiUnitDerivative:180}
\begin{aligned}
\dt{\rcap} &= \frac{1}{{ \Br }}\left(\rcap \wedge \dt{\Br}\right) \\
%                   &= \frac{1}{\Vert \Br \Vert}{     \frac{1}{\rcap} \left(\rcap \wedge \dt{\Br}\right)       } \\
                   &= \frac{1}{\Vert \Br \Vert}{     {\rcap} \left(\rcap \wedge \dt{\Br}\right)       } \\
%                   &= \frac{1}{{\Vert \Br \Vert}^3}{     {\Br} \left(\Br \wedge \dt{\Br}\right)       } \\
                   &= \frac{1}{\Vert \Br \Vert}\left({ \dt{\Br} - \rcap (\rcap \cdot \dt{\Br}) }\right) \\
\end{aligned}
\end{equation}

Thus this derivative is the component of
\(\frac{1}{{\Vert \Br \Vert}}\dt{\Br}\)
in the direction perpendicular to
\(\Br\).

\subsection{Another view}

When the objective is not comparing to the cross product, it is also notable that this unit vector derivative can be written

\begin{equation}\label{eqn:gaWikiUnitDerivative:100}
{{ \Br }} \dt{\rcap}
= \rcap \wedge \dt{\Br}
\end{equation}


%
% This was obvious to me at one point but is not now;)  What is the justification for the first statement?
%
%\subsection{A more direct route}
%
%Like a lot of stuff in math, once you know the answer you can get the answer more directly.  There is an unfortunate tendancy
%in some math texts to skip the logical sequence and go straight to the end result by the quickest route.  This is more
%elegant
%
%\begin{align*}
%r \dt{\rcap}
%   &= \dt{\Br} - \rcap \dt{r} \\
%   &= \dt{\Br} - \rcap\left(\rcap \cdot \dt{\Br}\right) \\
%   &= \rcap \left(\rcap \dt{\Br} - \rcap \cdot \dt{\Br}\right) \\
%   &= \rcap \left(\rcap \wedge \dt{\Br}\right) \\
%\end{align*}
%
%and gives the appearance of being clever, but it is easy to be clever when you already know the answer.

