%
% Copyright © 2012 Peeter Joot.  All Rights Reserved.
% Licenced as described in the file LICENSE under the root directory of this GIT repository.
%

Very much like complex numbers we can utilize exponentials to perform rotations.  Rotating in a sense from \(\Be_1\) to \(\Be_2\), can be expressed as
%
\begin{equation}\label{eqn:gaQuickIntro:290}
\begin{aligned}
\Ba e^{i \theta}
&=
(\Be_1 x_1 + \Be_2 x_2) (\cos\theta + \Be_1 \Be_2 \sin\theta) \\
&=
\Be_1 (x_1 \cos\theta - x_2 \sin\theta)
+
\Be_2 (x_2 \cos\theta + x_1 \sin\theta)
\end{aligned}
\end{equation}
%
More generally, even in N dimensional Euclidean spaces, if \(\Ba\) is a vector in a plane, and \(\ucap\) and \(\vcap\) are perpendicular unit vectors in that plane, then the rotation through angle \(\theta\) is given by
%
\begin{equation}\label{eqn:gaQuickIntro:190}
\Ba \rightarrow \Ba e^{\ucap \vcap \theta}.
\end{equation}
%
This is illustrated in \cref{fig:gaQuickIntro:gaQuickIntroFig1}

\pdfTexFigure{../figures/gabook/gaQuickIntroFig1.pdf_tex}{Plane rotation}{fig:gaQuickIntro:gaQuickIntroFig1}{0.6}

Notice that we have expressed the rotation here without utilizing a normal direction for the plane.  The sense of the rotation is encoded by the bivector \(\ucap \vcap\) that describes the plane and the orientation of the rotation (or by duality the direction of the normal in a 3D space).  By avoiding a requirement to encode the rotation using a normal to the plane we have an method of expressing the rotation that works not only in 3D spaces, but also in 2D and greater than 3D spaces, something that is not possible when we restrict ourselves to traditional vector algebra (where quantities like the cross product can not be defined in a 2D or 4D space, despite the fact that things they may represent, like torque are planar phenomena that do not have any intrinsic requirement for a normal that falls out of the plane.).

When \(\Ba\) does not lie in the plane spanned by the vectors \(\ucap\) and \(\vcap\) , as in \cref{fig:gaQuickIntro:gaQuickIntroFig2}, we must express the rotations differently.  A rotation then takes the form
%
\begin{equation}\label{eqn:gaQuickIntro:210}
\Ba \rightarrow
e^{-\ucap \vcap \theta/2}
\Ba
e^{\ucap \vcap \theta/2}.
\end{equation}
%

\pdfTexFigure{../figures/gabook/gaQuickIntroFig2.pdf_tex}{3D rotation}{fig:gaQuickIntro:gaQuickIntroFig2}{0.6}

In the 2D case, and when the vector lies in the plane this reduces to the one sided complex exponential operator used above.  We see these types of paired half angle rotations in QM, and they are also used extensively in computer graphics under the guise of quaternions.
