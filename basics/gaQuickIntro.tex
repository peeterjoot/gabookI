%
% Copyright � 2012 Peeter Joot.  All Rights Reserved.
% Licenced as described in the file LICENSE under the root directory of this GIT repository.
%

%
%
%\input{../peeter_prologue_print.tex}
%\input{../peeter_prologue_widescreen.tex}

\chapter{Geometric Algebra.  The very quickest introduction}
\index{geometric product!introduction}
\label{chap:gaQuickIntro}

%\blogpage{http://sites.google.com/site/peeterjoot2/math2012/gaQuickIntro.pdf}
%\date{Mar 16, 2012}
%\gitRevisionInfo{gaQuickIntro}

\beginArtWithToc
%\beginArtNoToc

\section{Motivation}

An attempt to make a relatively concise introduction to Geometric (or Clifford) Algebra.  Much more complete introductions to the subject can be found in \citep{dorst2007gac}, \citep{doran2003gap}, and \citep{hestenes1999nfc}.

\section{Axioms}

We have a couple basic principles upon which the algebra is based

\begin{itemize}
\item Vectors can be multiplied.
\item The square of a vector is the (squared) length of that vector (with appropriate generalizations for non-Euclidean metrics).
\item Vector products are associative (but not necessarily commutative).
\end{itemize}

That is really all there is to it, and the rest, paraphrasing Feynman, can be figured out by anybody sufficiently clever.

\section{By example.  The 2D case}

Consider a 2D Euclidean space, and the product of two vectors \(\Ba\) and \(\Bb\) in that space.  Utilizing a standard orthonormal basis \(\{\Be_1, \Be_2\}\) we can write

\begin{equation}\label{eqn:gaQuickIntro:10}
\begin{aligned}
\Ba &= \Be_1 x_1 + \Be_2 x_2 \\
\Bb &= \Be_1 y_1 + \Be_2 y_2,
\end{aligned}
\end{equation}

and let us write out the product of these two vectors \(\Ba \Bb\), not yet knowing what we will end up with.  That is

\begin{equation}\label{eqn:gaQuickIntro:230}
\begin{aligned}
\Ba \Bb
&= (\Be_1 x_1 + \Be_2 x_2 )( \Be_1 y_1 + \Be_2 y_2 ) \\
&= \Be_1^2 x_1 y_1 + \Be_2^2 x_2 y_2
+ \Be_1 \Be_2 x_1 y_2 + \Be_2 \Be_1 x_2 y_1
\end{aligned}
\end{equation}

From axiom 2 we have \(\Be_1^2 = \Be_2^2 = 1\), so we have

\begin{equation}\label{eqn:gaQuickIntro:30}
\Ba \Bb = x_1 y_1 + x_2 y_2 + \Be_1 \Be_2 x_1 y_2 + \Be_2 \Be_1 x_2 y_1.
\end{equation}

We have multiplied two vectors and ended up with a scalar component (and recognize that this part of the vector product is the dot product), and a component that is a ``something else''.  We will call this something else a bivector, and see that it is characterized by a product of non-colinear vectors.  These products \(\Be_1 \Be_2\) and \(\Be_2 \Be_1\) are in fact related, and we can see that by looking at the case of \(\Bb = \Ba\).  For that we have

\begin{equation}\label{eqn:gaQuickIntro:250}
\begin{aligned}
\Ba^2
&=
x_1 x_1 + x_2 x_2 + \Be_1 \Be_2 x_1 x_2 + \Be_2 \Be_1 x_2 x_1 \\
&=
\Abs{\Ba}^2 +
x_1 x_2 ( \Be_1 \Be_2 + \Be_2 \Be_1 )
\end{aligned}
\end{equation}

Since axiom (2) requires our vectors square to equal its (squared) length, we must then have

\begin{equation}\label{eqn:gaQuickIntro:50}
\Be_1 \Be_2 + \Be_2 \Be_1 = 0,
\end{equation}

or

\begin{equation}\label{eqn:gaQuickIntro:70}
\Be_2 \Be_1 = -\Be_1 \Be_2.
\end{equation}

We see that Euclidean orthonormal vectors anticommute.  What we can see with some additional study is that any colinear vectors commute, and in Euclidean spaces (of any dimension) vectors that are normal to each other anticommute (this can also be taken as a definition of normal).

We can now return to our product of two vectors \eqnref{eqn:gaQuickIntro:30} and simplify it slightly

\begin{equation}\label{eqn:gaQuickIntro:30b}
\Ba \Bb = x_1 y_1 + x_2 y_2 + \Be_1 \Be_2 (x_1 y_2 - x_2 y_1).
\end{equation}

The product of two vectors in 2D is seen here to have one scalar component, and one bivector component (an irreducible product of two normal vectors).  Observe the symmetric and antisymmetric split of the scalar and bivector components above.  This symmetry and antisymmetry can be made explicit, introducing dot and wedge product notation respectively

\begin{equation}\label{eqn:gaQuickIntro:90}
\begin{aligned}
\Ba \cdot \Bb &= \inv{2}( \Ba \Bb + \Bb \Ba) = x_1 y_1 + x_2 y_2 \\
\Ba \wedge \Bb &= \inv{2}( \Ba \Bb - \Bb \Ba) = \Be_1 \Be_2 (x_1 y_y - x_2 y_1).
\end{aligned}
\end{equation}

so that the vector product can be written as

\begin{equation}\label{eqn:gaQuickIntro:110}
\Ba \Bb = \Ba \cdot \Bb + \Ba \wedge \Bb.
\end{equation}

\section{Pseudoscalar}

In many contexts it is useful to introduce an ordered product of all the unit vectors for the space is called the pseudoscalar.  In our 2D case this is

\begin{equation}\label{eqn:gaQuickIntro:130}
i = \Be_1 \Be_2,
\end{equation}

a quantity that we find behaves like the complex imaginary.  That can be shown by considering its square

\begin{equation}\label{eqn:gaQuickIntro:270}
\begin{aligned}
(\Be_1 \Be_2)^2
&=
(\Be_1 \Be_2)
(\Be_1 \Be_2) \\
&=
\Be_1 (\Be_2 \Be_1) \Be_2 \\
&=
-\Be_1 (\Be_1 \Be_2) \Be_2 \\
&=
-(\Be_1 \Be_1) (\Be_2 \Be_2) \\
&=
-1^2 \\
&= -1
\end{aligned}
\end{equation}

Here the anticommutation of normal vectors property has been used, as well as (for the first time) the associative multiplication axiom.

In a 3D context, you will see the pseudoscalar in many places (expressing the normals to planes for example).  It also shows up in a number of fundamental relationships.  For example, if one writes

\begin{equation}\label{eqn:gaQuickIntro:150}
I = \Be_1 \Be_2 \Be_3
\end{equation}

for the 3D pseudoscalar, then it is also possible to show

\begin{equation}\label{eqn:gaQuickIntro:170}
\Ba \Bb = \Ba \cdot \Bb + I (\Ba \cross \Bb)
\end{equation}

something that will be familiar to the student of QM, where we see this in the context of Pauli matrices.  The Pauli matrices also encode a Clifford algebraic structure, but we do not need an explicit matrix representation to do so.

\section{Rotations}

Very much like complex numbers we can utilize exponentials to perform rotations.  Rotating in a sense from \(\Be_1\) to \(\Be_2\), can be expressed as

\begin{equation}\label{eqn:gaQuickIntro:290}
\begin{aligned}
\Ba e^{i \theta}
&=
(\Be_1 x_1 + \Be_2 x_2) (\cos\theta + \Be_1 \Be_2 \sin\theta) \\
&=
\Be_1 (x_1 \cos\theta - x_2 \sin\theta)
+
\Be_2 (x_2 \cos\theta + x_1 \sin\theta)
\end{aligned}
\end{equation}

More generally, even in N dimensional Euclidean spaces, if \(\Ba\) is a vector in a plane, and \(\ucap\) and \(\vcap\) are perpendicular unit vectors in that plane, then the rotation through angle \(\theta\) is given by

\begin{equation}\label{eqn:gaQuickIntro:190}
\Ba \rightarrow \Ba e^{\ucap \vcap \theta}.
\end{equation}

This is illustrated in \cref{fig:gaQuickIntro:gaQuickIntroFig1}

\pdfTexFigure{../../physicsplay/figures/gabook/gaQuickIntroFig1.pdf_tex}{Plane rotation}{fig:gaQuickIntro:gaQuickIntroFig1}{0.6}

Notice that we have expressed the rotation here without utilizing a normal direction for the plane.  The sense of the rotation is encoded by the bivector \(\ucap \vcap\) that describes the plane and the orientation of the rotation (or by duality the direction of the normal in a 3D space).  By avoiding a requirement to encode the rotation using a normal to the plane we have an method of expressing the rotation that works not only in 3D spaces, but also in 2D and greater than 3D spaces, something that is not possible when we restrict ourselves to traditional vector algebra (where quantities like the cross product can not be defined in a 2D or 4D space, despite the fact that things they may represent, like torque are planar phenomena that do not have any intrinsic requirement for a normal that falls out of the plane.).

When \(\Ba\) does not lie in the plane spanned by the vectors \(\ucap\) and \(\vcap\) , as in \cref{fig:gaQuickIntro:gaQuickIntroFig2}, we must express the rotations differently.  A rotation then takes the form

\begin{equation}\label{eqn:gaQuickIntro:210}
\Ba \rightarrow
e^{-\ucap \vcap \theta/2}
\Ba
e^{\ucap \vcap \theta/2}.
\end{equation}


\pdfTexFigure{../../physicsplay/figures/gabook/gaQuickIntroFig2.pdf_tex}{3D rotation}{fig:gaQuickIntro:gaQuickIntroFig2}{0.6}

In the 2D case, and when the vector lies in the plane this reduces to the one sided complex exponential operator used above.  We see these types of paired half angle rotations in QM, and they are also used extensively in computer graphics under the guise of quaternions.

%\EndArticle
