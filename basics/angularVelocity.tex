%
% Copyright � 2012 Peeter Joot.  All Rights Reserved.
% Licenced as described in the file LICENSE under the root directory of this GIT repository.
%

%
%
\chapter{Rotational dynamics}\label{chap:PJAngVel}
%\date{January 29, 2008.  angularVelocity.tex}

\section{GA introduction of angular velocity}
\index{angular velocity}

By taking the first derivative of a radially expressed vector we have the velocity

\begin{equation}\label{eqn:angularVelocity:20}
\Bv
   = r'\rcap + \rcap(\rcap \wedge \Br')
   = \rcap( v_r + \rcap \wedge \Bv )
\end{equation}

Or,
\begin{equation}\label{eqn:angularVelocity:40}
\rcap \Bv = v_r + \rcap \wedge \Bv
\end{equation}
\begin{equation}\label{eqn:angularVelocity:60}
\rcap \Bv = v_r + (1/r)\Br \wedge \Bv
\end{equation}

Put this way, the earlier calculus exercise to derive this seems a bit silly, since it is probably clear that \(v_r = \rcap \cdot \Bv\).

Anyways, let us work with velocity expressed this way in a few ways.

\subsection{Speed in terms of linear and rotational components}

\begin{equation}\label{eqn:angularVelocity:80}
\Abs{\Bv}^2 = v_r^2 + (\rcap(\rcap \wedge \Bv))^2
\end{equation}

And,
\begin{equation}\label{eqn:angularVelocity:280}
\begin{aligned}
(\rcap(\rcap \wedge \Bv))^2
   &= (\Bv \wedge \rcap)\rcap \rcap(\rcap \wedge \Bv) \\
   &= (\Bv \wedge \rcap)(\rcap \wedge \Bv) \\
   &= -(\rcap \wedge \Bv)^2 \\
   &= \Abs{\rcap \wedge \Bv}^2 \\
\end{aligned}
\end{equation}

\begin{equation}\label{eqn:angularVelocity:300}
\begin{aligned}
\implies
\Abs{\Bv}^2 &= v_r^2 + \Abs{\rcap \wedge \Bv}^2 \\
             &= v_r^2 + \Abs{\rcap \wedge \Bv}^2 \\
\end{aligned}
\end{equation}

So, we can assign a physical significance to the bivector.

\begin{equation}\label{eqn:angularVelocity:100}
\Abs{\rcap \wedge \Bv} = \abs{v_{\perp}}
\end{equation}

The bivector \(\Abs{\rcap \wedge \Bv}\) has the magnitude of the non-radial component of the velocity.  This
equals the magnitude of the component of the velocity perpendicular to its radial component (ie: the angular component of the velocity).

\subsection{angular velocity.  Prep}

Because \(\Abs{\rcap \wedge \Bv}\) is the non-radial velocity component, for small angles
\({v_\perp}/r\) will equal the angle between the vector and its displacement.

This allows for the calculation of the rate of change of that angle with time, what it called the scalar
angular velocity (dimensions are \(1/t\) not \(x/t\)).  This can be done by taking the \(\sin\) as the ratio of the
length of the non-radial component of the delta to the length of the displaced vector.

\begin{equation}\label{eqn:angularVelocity:320}
\begin{aligned}
\sin d\theta &= \frac{\Abs{\rcap(\rcap \wedge d\Br)}}{\Abs{\Br + d\Br}} \\
\end{aligned}
\end{equation}

With \(d\Br = \dt{\Br} dt = \Bv dt\), the angular velocity is

\begin{equation}\label{eqn:angularVelocity:340}
\begin{aligned}
\sin d\theta
   &= \frac{1}{\Abs{\Br + \Bv dt}} \Abs{ \rcap (\rcap \wedge \Bv) dt } \\
   &= \frac{1}{\Abs{\Br + \Bv dt}} \Abs{ (\rcap \wedge \Bv) dt } \\
\frac{\sin d\theta}{\abs{dt}}
   &= \frac{1}{\Abs{\Br + \Bv dt}} \Abs{ \rcap \wedge \Bv } \\
   &= \frac{1}{\Abs{\Br}\Abs{\Br + \Bv dt}} \Abs{ \Br \wedge \Bv } \\
\end{aligned}
\end{equation}

In the limit, taking \(dt > 0\), this is
\begin{equation}\label{eqn:angularVelocity:120}
\omega = \dt{\theta} = \frac{1}{\Br^2} \Abs{ \Br \wedge \Bv }
\end{equation}

\subsection{angular velocity.  Summarizing}

Here is a summary of calculations so far involving the \(\Br \wedge \Bv\) bivector

\begin{equation}\label{eqn:angularVelocity:360}
\begin{aligned}
\Bv &= \rcap v_r + \frac{\rcap}{\Abs{\Br}} (\Br \wedge \Bv) \\
\dt{\rcap} &= \frac{\rcap}{\Br^2} (\Br \wedge \Bv) \\
\abs{v_{\perp}} &= \frac{1}{\Abs{\Br}} \Abs{ \Br \wedge \Bv } \\
\omega = \dt{\theta} &= \frac{1}{\Br^2} \Abs{ \Br \wedge \Bv } \\
\end{aligned}
\end{equation}

It makes sense to give the bivector a name.  Given its magnitude the
angular velocity bivector \(\Bomega\) is designated

\begin{equation}\label{eqn:angularVelocity:140}
\Bomega = \frac{ \Br \wedge \Bv }{\Br^2}
\end{equation}

So the linear and rotational components of the velocity can thus be expressed in terms of this, as can our
unit vector derivative, scalar angular velocity, and perpendicular velocity magnitude:

\begin{equation}\label{eqn:angularVelocity:380}
\begin{aligned}
\omega = \dt{\theta} &= \Abs{ \Bomega } \\
\Bv &= \rcap v_r + \Br \Bomega \\
    &= \rcap( v_r + r \Bomega ) \\
\dt{\rcap} &= \rcap \Bomega \\
\abs{v_{\perp}} &= r \Abs{ \Bomega } \\
\end{aligned}
\end{equation}

This is similar to the vector angular velocity (\(\Bomega = (\Br \times \Bv)/r^2\)), but instead of lying perpendicular to the
plane of rotation, it defines the plane of rotation (for a vector \(\Ba\), \(\Ba \wedge \Bomega\) is zero if the vector is in the plane and non-zero if the vector has a component outside of the plane).

%\begin{align*}
%\Bomega = \frac{1}{\Br^2} (\Br \wedge \Bv)
%\end{align*}
%
%Or,
%\begin{align*}
%\Br \wedge \Bv = \Br^2 \Bomega = r^2\Bomega
%\end{align*}
%
%
%\begin{align*}
%\Bv
%   &= \rcap(v_r + (1/r)\Br \wedge \Bv) \\
%   &= \rcap(v_r + r\Bomega) \\
%   &= v_r\rcap + \Br\Bomega \\
%\end{align*}

\subsection{Explicit perpendicular unit vector}
\index{unit normal}

If one introduces a unit vector \(\thetacap\) in the direction of rejection of \(\Br\) from \(d\Br\), the total velocity takes the symmetrical form
\begin{equation}\label{eqn:angularVelocity:400}
\begin{aligned}
\Bv
   &= v_r\rcap + r\omega\thetacap \\
   &= \dt{r}\rcap + r\dt{\theta}\thetacap \\
\end{aligned}
\end{equation}

\subsection{acceleration in terms of angular velocity bivector}

Taking derivatives of velocity, one can with a bit of work,
express acceleration in terms of
radial and non-radial components

%\Br \wedge \Bv = \Br^2 \Bomega = r^2\Bomega

\begin{equation}\label{eqn:angularVelocity:420}
\begin{aligned}
\Ba
   &= (\rcap v_r + \Br \Bomega)' \\
   &= \rcap' v_r + \rcap v_r' + \Br' \Bomega + \Br \Bomega' \\
   &= \rcap \Bomega v_r + \rcap v_r' + \Br' \Bomega + \Br \Bomega' \\
   &= \rcap \Bomega v_r + \rcap a_r + \Bv \Bomega + \Br \Bomega' \\
\end{aligned}
\end{equation}

But,
\begin{equation}\label{eqn:angularVelocity:440}
\begin{aligned}
\Bomega' &= ((1/r^2) (\Br \wedge \Bv))' \\
         &= (-2/r^3) r' (\Br \wedge \Bv) + (1/r^2) (\Bv \wedge \Bv + \Br \wedge \Ba) \\
         &= -(2/r) v_r \Bomega + (1/r^2) (\Br \wedge \Ba) \\
\end{aligned}
\end{equation}

%\rcap v_r = \Bv - \Br \Bomega
So,

\begin{equation}\label{eqn:angularVelocity:460}
\begin{aligned}
\Ba
   &= \rcap a_r -\rcap \Bomega v_r + \Bv \Bomega + \rcap (\rcap \wedge \Ba) \\
   &= \rcap a_r -( \Bv - \Br \Bomega) \Bomega + \Bv \Bomega + \rcap (\rcap \wedge \Ba) \\
\\
   &= \rcap a_r + \Br \Bomega^2+ \rcap (\rcap \wedge \Ba) \\
   &= \rcap( a_r + r \Bomega^2) + \rcap (\rcap \wedge \Ba) \\
\end{aligned}
\end{equation}

Note that \(\Bomega^2\) is a negative scalar, so as normal writing \(\norm{\Bomega}^2 = -\Bomega^2\), we have acceleration in a fashion similar to the
traditional cross product form:

\begin{equation}\label{eqn:angularVelocity:480}
\begin{aligned}
\Ba
   &= \rcap( a_r - r \norm{\Bomega}^2) + \rcap (\rcap \wedge \Ba) \\
   &= \rcap( a_r - r \norm{\Bomega}^2 + \rcap \wedge \Ba) \\
\end{aligned}
\end{equation}

In the traditional representation, this last term, the non-radial acceleration
component, is often expressed as a derivative.

In terms of the wedge product, this can be done by noting that

\begin{equation}\label{eqn:angularVelocity:160}
(\Br \wedge \Bv)' = \Bv \wedge \Bv + \Br \wedge \Ba = \Br \wedge \Ba
\end{equation}

\begin{equation}\label{eqn:angularVelocity:500}
\begin{aligned}
\Ba
   &= \rcap( a_r - r \norm{\Bomega}^2 ) + \frac{\Br}{r^2}(\Br \wedge \Bv)') \\
   &= \rcap( a_r - r \norm{\Bomega}^2 ) + \frac{1}{\Br}\dt{(\Br^2 \Bomega)} \\
\end{aligned}
\end{equation}

Expressed in terms of force (for constant mass) this is
\begin{equation}\label{eqn:angularVelocity:520}
\begin{aligned}
\BF &= m \Ba \\
    &= \rcap (m a_r) + (m \Br) {\Bomega}^2
       + \frac{1}{\Br}\dt{(m \Br^2 \Bomega)} \\
    &= \BF_r + (m \Br) {\Bomega}^2
             + \frac{1}{\Br}\dt{(m \Br^2 \Bomega)} \\
\end{aligned}
\end{equation}

Alternately, the non-radial term can be expressed in terms of torque

\begin{equation}\label{eqn:angularVelocity:540}
\begin{aligned}
\rcap (\rcap \wedge \Ba)
   &= \rcap (\rcap \wedge m \Ba)  \\
   &= \frac{\Br}{r^2} (\Br \wedge \BF)  \\
   &= \frac{1}{\Br} (\Br \wedge \BF)  \\
   &= \frac{1}{\Br} \Btau \\
\end{aligned}
\end{equation}

Thus the torque bivector, which in magnitude was the angular derivative of
the work
done by the force \(\norm{\Btau} = \tau = \dtheta{W} = \BF \cdot \dtheta{\Br}\)
is also expressible as a time derivative

\begin{equation}\label{eqn:angularVelocity:560}
\begin{aligned}
\Btau
&= \dt{( m \Br^2 \Bomega )}  \\
&= \dt{( m \Br \wedge \Bv)}  \\
&= \dt{( \Br \wedge m \Bv)}  \\
&= \dt{( \Br \wedge \Bp  )}  \\
\end{aligned}
\end{equation}

This bivector \(m \Br^2 \Bomega = \Br \wedge \Bp\) is called the angular
momentum, designated \(\BJ\).  It is related to the total momentum as follows

\begin{equation}\label{eqn:angularVelocity:180}
\Bp = \rcap (\rcap \cdot \Bp) + \frac{1}{\Br} \BJ
\end{equation}

So the total force is

\begin{equation}\label{eqn:angularVelocity:580}
\begin{aligned}
\BF
    &= \BF_r + m \Br {\Bomega}^2 + \frac{1}{\Br}\dt{\BJ} \\
\end{aligned}
\end{equation}

Observe that for a purely radial (ie: central) force, we must have
\(\dt{\BJ} = 0\)
so, the angular
momentum must be constant.

\subsection{Kepler's laws example}
\index{Kepler's laws}

This follows the \citep{salas1990coa} treatment, modified for the GA notation.

Consider the gravitational force

\begin{equation}\label{eqn:angularVelocity:600}
\begin{aligned}
m \Ba &= -G \frac{m M}{r^2} \rcap \\
\Ba &= - G M \frac{\rcap}{r^2} = -\rho \frac{\rcap}{r^2}
\end{aligned}
\end{equation}

Or,
\begin{equation}\label{eqn:angularVelocity:200}
\frac{\rcap}{r^2} = -\frac{1}{\rho} \dt{\Bv}
\end{equation}

The unit vector derivative is

\begin{equation}\label{eqn:angularVelocity:620}
\begin{aligned}
\dt{\rcap} &= \frac{\rcap}{r}(\rcap \wedge \Bv) \\
           &= \frac{\rcap}{r^2}\frac{\BJ}{m} \\
           &= -\frac{1}{m \rho} \dt{\Bv} \BJ \\
           &= \dt{(-\frac{1}{m \rho} \Bv \BJ )} \\
\end{aligned}
\end{equation}

The last because \(\BJ\), \(m\), and \(\rho\) are all constant.

Before continuing, let us examine this funny vector bivector product term.
In general a vector
bivector product will have vector and trivector parts, but
the differential equation implies that this is a vector.  Let us confirm this

\begin{equation}\label{eqn:angularVelocity:640}
\begin{aligned}
\Bv \BJ &= \Bv (\Br \wedge m \Bv) \\
        &= (m \Bv^2) \vcap (\Br \wedge \vcap) \\
        &= - (m \Bv^2) \vcap (\vcap \wedge \Br) \\
\end{aligned}
\end{equation}

So, this is in fact a vector, it is the rejective component of \(\Br\) from
the direction of \(\vcap\) scaled by \(-m\Bv^2\).  We can also calculate
the product \(\BJ \Bv\) from this:

\begin{equation}\label{eqn:angularVelocity:660}
\begin{aligned}
\Bv \BJ
        &= - (m \Bv^2) \vcap (\vcap \wedge \Br) \\
        &= - (m \Bv^2) (\Br \wedge \vcap) \vcap \\
        &= - (\Br \wedge m \Bv) \Bv \\
        &= - \BJ \Bv \\
\end{aligned}
\end{equation}

This antisymetrical result \(\Bv \BJ = - \BJ \Bv\) is actually the defining
property of the vector bivector ``dot product'' (unlike the vector dot product
which is the symmetrical parts).  This vector bivector dot product selects the
vector component, leaving the trivector part.  Since \(\Bv\) lies completely in
the plane of the angular velocity bivector \(\Bv \wedge \BJ = 0\) in this case.

Anyways, back to the problem, integrating
with respect to time, and introducing a vector integration constant \(\Be\)
we have

\begin{equation}\label{eqn:angularVelocity:220}
\rcap + \frac{1}{m \rho} \Bv \BJ = \Be
\end{equation}

Multiplying by \(\Br\)

\begin{equation}\label{eqn:angularVelocity:680}
\begin{aligned}
r + \frac{1}{m \rho} \Br \Bv \BJ &= \Br \Be \\
r + \frac{1}{m^2 \rho} (\Br \cdot \Bp + \BJ) \BJ &= \Br \cdot \Be + \Br \wedge \Be \\
\end{aligned}
\end{equation}

This results in three equations, one for each of the scalar, vector, and bivector parts

\begin{equation}\label{eqn:angularVelocity:700}
\begin{aligned}
r + \frac{\BJ^2}{m^2 \rho} &= \Br \cdot \Be \\
\frac{1}{m \rho} (\Br \cdot \Bv) \BJ &= 0 \\
\Br \wedge \Be &= 0 \\
\end{aligned}
\end{equation}

The first of these equations is the result from Salas and Hille (integration constant differs in sign though).

\begin{equation}\label{eqn:angularVelocity:720}
\begin{aligned}
r - \frac{J^2}{m^2 \rho} &= \Br \cdot \Be \\
%r - \Br \cdot \Be &= \frac{J^2}{m^2 \rho} \\
%\Br \rcap - \Br \cdot \Be &= \frac{J^2}{m^2 \rho} \\
%\Br \rcap - \Br \Be &= \frac{J^2}{m^2 \rho} \\
%\Br (\rcap - \Be) &= \frac{J^2}{m^2 \rho}
\end{aligned}
\end{equation}

%\[
%\frac{\Br_0}{\norm{\Br_0}} + \frac{1}{m \rho} \Bv_0 \BJ = \Be
%\]
%
%\[
%\rcap + \frac{1}{m \rho} \Bv \BJ = \frac{\Br_0}{\norm{\Br_0}} + \frac{1}{m \rho} \Bv_0 \BJ
%\]
%
%\[
%\rcap - \rcap_0 + \frac{1}{m \rho} (\Bv - \Bv_0) \BJ = 0
%\]
%
%J = r ^ m v = r mv - r . mv = r mv
%v = 1/(r m) J

%Diverging from the Salas and Hille treatment, instead of producing a scalar
%equation, lets remove the \(\Bv\) term from the equation:
%
%\[
%\rcap + \frac{1}{m \rho} \Bv \BJ = \Be
%\]
%
%First express \(\Bv\) in terms of \(\Br\) and \(\BJ\).
%
%\begin{align*}
%\BJ
%   &= \Br \wedge (m \Bv) \\
%   &= m \Br \Bv - m \Br \cdot \Bv \\
%   &= m \Br \Bv \\
%\end{align*}
%
%Thus,
%\[
%\Bv = \frac{1}{m \Br}\BJ
%\]
%
%%r_0 - \Br_0 \Be = \frac{1}{m^2 \rho} J^2)
%%-r_0 + \Br_0 \Be = -\frac{1}{m^2 \rho} J^2)
%%\Br_0 \Be = r_0 - \frac{1}{m^2 \rho} J^2)
%%\Be = (1/\Br_0)(r_0 - \frac{1}{m^2 \rho} J^2))
%
%And,
%\begin{align*}
%\rcap + \frac{1}{m^2 \rho \Br} \BJ^2 &= \Be \\
%\rcap(1 - \frac{1}{m^2 \rho r} J^2) &= \Be \\
%r - \frac{1}{m^2 \rho} J^2 &= \Br \Be \\
%r - \Br \Be &= \frac{1}{m^2 \rho} J^2 \\
%r - \Br (1/\Br_0)(r_0 - \frac{1}{m^2 \rho} J^2) &= \frac{1}{m^2 \rho} J^2 \\
%r - \Br \rcap_0 + \Br (1/\Br_0) \frac{1}{m^2 \rho} J^2 &= \frac{1}{m^2 \rho} J^2 \\
%\Br \rcap - \Br \rcap_0 + \Br (1/\Br_0) \frac{1}{m^2 \rho} J^2 &= \frac{1}{m^2 \rho} J^2 \\
%\Br (\rcap - \rcap_0 + (1/\Br_0) \frac{1}{m^2 \rho} J^2) &= \frac{1}{m^2 \rho} J^2 \\
%\end{align*}
%
%Since \(\Br \cdot \Bv = 0\) then:
%
%\begin{align*}
%J^2 &= -(\Br \wedge m \Bv)^2 \\
%    &= m^2 (\Br \wedge \Bv)(\Bv \wedge \Br) \\
%    &= m^2 (\Br \Bv)(\Bv \Br) \\
%    &= m^2 \Br^2 \Bv^2 \\
%    &= m^2 \Br_0^2 \Bv_0^2 \\
%\end{align*}
%
%\begin{align*}
%\Br (\rcap - \rcap_0 + (1/\Br_0) \frac{1}{\rho} \Br_0^2 \Bv_0^2) &= \frac{1}{\rho} \Br_0^2 \Bv_0^2 \\
%\Br (\rcap - \rcap_0 + \frac{1}{\rho} \Br_0 \Bv_0^2) &= \frac{1}{\rho} \Br_0^2 \Bv_0^2 \\
%\end{align*}

\subsection{Circular motion}
\index{circular motion}

For circular motion \(v_r = a_r = 0\), so:

\begin{equation}\label{eqn:angularVelocity:240}
\Bv = \Br \Bomega
\end{equation}
\begin{equation}\label{eqn:angularVelocity:260}
\Ba = \rcap \left(  -\frac{\Bv^2}{r} + \rcap \wedge \Ba \right) \\
\end{equation}

For constant circular motion:
\begin{equation}\label{eqn:angularVelocity:740}
\begin{aligned}
\Ba
   &= \Bv\Bomega + \Br\Bomega' \\
   &= \Bv\Bomega + \Br(\Bzero) \\
   &= \Br(\Bomega)^2 \\
   &= -\Br\Abs{\Bomega}^2 \\
\end{aligned}
\end{equation}

ie: the \(\rcap (\rcap \wedge \Ba )\) term is zero... all acceleration is inwards.

Can also expand this in terms of \(\Br\) and \(\Bv\):
\begin{equation}\label{eqn:angularVelocity:760}
\begin{aligned}
\Ba
   &= \Br\left(\Bomega\right)^2 \\
   &= \Br\left(\frac{1}{\Br}\Bv\right)^2 \\
   &= -\Br\left( \Bv \frac{1}{\Br} \frac{1}{\Br}\Bv \right) \\
   &= -\Br\left( \frac{\Bv^2}{\Br^2}\right) \\
   &= -\frac{1}{\Br}\Bv^2 \\
\end{aligned}
\end{equation}

