%
% Copyright � 2012 Peeter Joot.  All Rights Reserved.
% Licenced as described in the file LICENSE under the root directory of this GIT repository.
%

%
%
\chapter{More details on NFCM plane formulation}
\label{chap:plane}
%\date{Jan 1, 2008.  plane.tex}

\section{Wedge product formula for a plane}
\index{plane!wedge product}

The equation of the plane with bivector \(\BU\) through point \(\Ba\) is given
by

\begin{equation}\label{eqn:plane:20}
(\Bx - \Ba) \wedge \BU = 0
\end{equation}

or

\begin{equation}\label{eqn:plane:40}
\Bx \wedge \BU = \Ba \wedge \BU = \BT
\end{equation}

\subsection{Examining this equation in more details}

Without any loss of generality one can express this plane equation
in terms of a unit bivector \(\Bi\)

\begin{equation}\label{eqn:plane:60}
\Bx \wedge \Bi = \Ba \wedge \Bi
\end{equation}

As with the line equation, to express this in the ``standard'' parametric
form, right multiplication with \(1/\Bi\) is required.

\begin{equation}\label{eqn:plane:80}
(\Bx \wedge \Bi)\frac{1}{\Bi} = (\Ba \wedge \Bi)\frac{1}{\Bi}
\end{equation}

We have a trivector bivector product here, which in general has a vector,
trivector, and 5-vector component.  Since \(\Bi \wedge \Bi = 0\), the
5-vector component is zero:

\begin{equation}\label{eqn:plane:100}
\Bx \wedge \Bi \wedge -\Bi = 0
\end{equation}

and intuition says that the trivector component will also be zero.  However,
as well as providing verification of this, expansion of this product will also
demonstrate how to find the projective and rejective components of a vector
with respect to a plane (ie: components in and out of the plane).

\subsection{Rejection from a plane product expansion}
\index{plane!rejection}

Here is an explicit expansion of the rejective term above

\begin{equation}\label{eqn:plane:320}
\begin{aligned}
(\Bx \wedge \Bi)\frac{1}{\Bi}
&= -(\Bx \wedge \Bi){\Bi} \\
&= -\frac{1}{2}(\Bx\Bi + \Bi\Bx){\Bi} \\
&= \frac{1}{2}(\Bx - \Bi\Bx\Bi) \\
&= \frac{1}{2}(\Bx - (\Bx \Bi + 2 \Bi \cdot \Bx)\Bi) \\
&= \Bx - (\Bi \cdot \Bx)\Bi \\
\end{aligned}
\end{equation}

In this last term the quantity \(\Bi \cdot \Bx\) is a vector in the plane.
This can be demonstrated by writing \(\Bi\) in terms of a pair of orthonormal
vectors \(\Bi = \ucap\vcap = \ucap \wedge \vcap\).

\begin{equation}\label{eqn:plane:340}
\begin{aligned}
\Bi \cdot \Bx &= (\ucap \wedge \vcap) \cdot \Bx \\
              &= \ucap (\vcap \cdot \Bx) - \vcap (\ucap \cdot \Bx) \\
\end{aligned}
\end{equation}

Thus, \((\Bi \cdot \Bx) \wedge \Bi = 0\),
and \((\Bi \cdot \Bx) \Bi = (\Bi \cdot \Bx) \cdot \Bi\).  Inserting this above
we have the end result

\begin{equation}\label{eqn:plane:360}
\begin{aligned}
(\Bx \wedge \Bi)\frac{1}{\Bi}
&= \Bx - (\Bi \cdot \Bx) \cdot \Bi \\
&= \Ba - (\Bi \cdot \Ba) \cdot \Bi \\
\end{aligned}
\end{equation}

Or
\begin{equation}\label{eqn:plane:380}
\begin{aligned}
\Bx  - \Ba
&= (\Bi \cdot (\Bx - \Ba)) \cdot \Bi \\
\end{aligned}
\end{equation}

This is actually the standard parametric equation of a plane, but expressed
in terms of a unit bivector that describes the plane instead of in terms
of a pair of vectors in the plane.

To demonstrate this expansion of the right hand side is required

\begin{equation}\label{eqn:plane:400}
\begin{aligned}
(\Bi \cdot \Bx) \cdot \Bi
&= (\ucap (\vcap \cdot \Bx) - \vcap (\ucap \cdot \Bx)) \ucap \vcap \\
&= \vcap (\vcap \cdot \Bx) + \ucap (\ucap \cdot \Bx) \\
\end{aligned}
\end{equation}

Substituting this back yields:

\begin{equation}\label{eqn:plane:420}
\begin{aligned}
\Bx
&= \Ba + \ucap (\ucap \cdot (\Bx - \Ba)) + \vcap (\vcap \cdot (\Bx - \Ba)) \\
&= \Ba + s \ucap + t \vcap \\
&= \Ba + s' \By + t' \Bw \\
\end{aligned}
\end{equation}

Where \(\By\) and \(\Bw\) are two arbitrary, but non-colinear vectors
in the plane.

In words this says that the plane is specified by a point in the plane,
and the span of any pair of linearly independent vectors directed in that plane.

An expression of this form, or a normal form in terms of the cross product
is often how the plane is defined, and the analysis above demonstrates
that the bivector wedge product formula,

\begin{equation}\label{eqn:plane:120}
\Bx \wedge \BU = \Ba \wedge \BU
\end{equation}

where specific direction vectors in the plane need not be explicitly specified,
also implicitly contains this parametric representation.

\subsection{Orthonormal decomposition of a vector with respect to a plane}
\index{vector!plane projection}

With the expansion above we have a separation of a vector into two
components, and these can be demonstrated to be the components that are
directed entirely within and out of the plane.

Rearranging terms from above we have:

\begin{equation}\label{eqn:plane:440}
\begin{aligned}
\Bx
&=
(\Bx \cdot \Bi) \cdot \frac{1}{\Bi} + (\Bx \wedge \Bi) \cdot \frac{1}{\Bi} \\
&=
(\Bx \cdot \Bi) \frac{1}{\Bi} + (\Bx \wedge \Bi) \frac{1}{\Bi} \\
\end{aligned}
\end{equation}

Writing the vector \(\Bx\) in terms of components parallel and perpendicular
to the plane

\begin{equation}\label{eqn:plane:140}
\Bx = \Bx_{\perp} + \Bx_{\parallel}
\end{equation}

Only the \(\Bx_{\parallel}\) component contributes to the dot product
and only the \(\Bx_{\perp}\) component contributes to the wedge product:

\begin{equation}\label{eqn:plane:460}
\begin{aligned}
\Bx
&=
(\Bx_\parallel \cdot \Bi) \cdot \frac{1}{\Bi} + (\Bx_\perp \wedge \Bi) \cdot \frac{1}{\Bi} \\
\Bx_\parallel &= (\Bx \cdot \Bi) \cdot \frac{1}{\Bi} \\
\Bx_\perp &= (\Bx \wedge \Bi) \cdot \frac{1}{\Bi} \\
\end{aligned}
\end{equation}

So, just as in the orthonormal decomposition of a vector with respect to a
unit vector, this gives us a way to calculate components of a vector
in and rejected from any plane, a very useful result in its own right.

Returning to back to the equation of a plane we have

\begin{equation}\label{eqn:plane:480}
\begin{aligned}
- (\Bx \wedge \Bi)\Bi &= - (\Ba \wedge \Bi)\Bi = \Ba - (\Ba \cdot \Bi) \cdot \frac{1}{\Bi}
\end{aligned}
\end{equation}

Thus, for the fixed point in the plane, the quantity

\begin{equation}\label{eqn:plane:160}
\Bd = (\Ba \wedge \Bi) \cdot \frac{1}{\Bi}
\end{equation}

is the component of that vector perpendicular to the plane or the minimal length directed vector from the origin to the plane (directrix).  In terms
of the unit bivector for the plane and its directrix the equation of a
plane becomes

\begin{equation}\label{eqn:plane:500}
\begin{aligned}
\Bx \wedge \Bi &= \Bd \Bi = \Bd \wedge \Bi
\end{aligned}
\end{equation}

Note that the directrix is a normal to the plane.
%, so we have arrived at something
%similar to the typical cross product normal form plane equation.

\subsection{Alternate derivation of orthonormal planar decomposition}

This could alternately be derived by expanding the vector unit bivector
product directly

\begin{equation}\label{eqn:plane:520}
\begin{aligned}
\Bx \Bi \frac{1}{\Bi}
&= ( \Bx \cdot \Bi + \Bx \wedge \Bi ) \frac{1}{\Bi} \\
&=
- {(\Bx \cdot \Bi) \cdot \Bi} - {(\Bx \cdot \Bi) \wedge \Bi} - {(\Bx \wedge \Bi) \Bi} \\
&=
- {(\Bx \cdot \Bi) \cdot \Bi} - {(\Bx \wedge \Bi) \cdot \Bi } - {\left<(\Bx \wedge \Bi) \Bi\right>_3} - {(\Bx \wedge \Bi) \wedge \Bi} \\
&=
{(\Bx \cdot \Bi) \cdot \frac{1}{\Bi}} + {(\Bx \wedge \Bi) \cdot \frac{1}{\Bi}} - {\left<(\Bx \wedge \Bi) \Bi\right>_3} \\
\end{aligned}
\end{equation}

Since the LHS of this equation is the vector \(\Bx\), the RHS must
also be a vector, which demonstrates that the term

\begin{equation}\label{eqn:plane:180}
\left<(\Bx \wedge \Bi) \Bi\right>_3 = 0
\end{equation}

So, one has

\begin{equation}\label{eqn:plane:540}
\begin{aligned}
\Bx
&=
{(\Bx \cdot \Bi) \cdot \frac{1}{\Bi}} + {(\Bx \wedge \Bi) \cdot \frac{1}{\Bi}} \\
&=
{(\Bx \cdot \Bi) \frac{1}{\Bi}} + {(\Bx \wedge \Bi) \frac{1}{\Bi}} \\
\end{aligned}
\end{equation}

\section{Generalization of orthogonal decomposition to components with respect to a hypervolume}

Having observed how to directly calculate the components of a vector in and out of a plane, we can now do the
same thing for a \(r\)th degree volume element spanned by an \(r\)-blade hypervolume element \(\BU\).

\subsection{Hypervolume element and its inverse written in terms of a spanning orthonormal set}
\index{hypervolume element}
We take \(\BU\) to be a simple element, not an arbitrary multivector of grade \(r\).  Such an element can
always be written in the form

\begin{equation}\label{eqn:plane:200}
\BU = k \Bu_1 \Bu_2 \cdots \Bu_r
\end{equation}

Where \(\Bu_k\) are unit vectors that span the volume element.

The inverse of \(\BU\) is thus

\begin{equation}\label{eqn:plane:560}
\begin{aligned}
\BU^{-1}
&= \frac{\BU^\dagger}{\BU \BU^\dagger} \\
&= \frac{k \Bu_r \cdots \Bu_1}{(k \Bu_1 \cdots \Bu_r)(k \Bu_r \Bu_{r-1} \cdots \Bu_1)} \\
&= \frac{\Bu_r \cdots \Bu_1}{k} \\
\end{aligned}
\end{equation}

\subsection{Expanding the product}

Having gathered the required introductory steps we are now in a position to express the vector \(\Bx\) in terms
of components projected into and rejected from this hypervolume

\begin{equation}\label{eqn:plane:580}
\begin{aligned}
\Bx &= \Bx \BU \frac{1}{\BU} \\
    &= (\Bx \cdot \BU + \Bx \wedge \BU)\frac{1}{\BU} \\
\end{aligned}
\end{equation}

The dot product term can be expanded to
\begin{equation}\label{eqn:plane:600}
\begin{aligned}
&(\Bx \cdot \BU) \frac{1}{\BU}  \\
&= k(
      (\Bx \cdot \Bu_1)\Bu_2\Bu_3\cdots\Bu_r
    - (\Bx \cdot \Bu_2)\Bu_1\Bu_3\Bu_4\cdots\Bu_r
    + \cdots
    ) \frac{1}{\BU} \\
&=
  (\Bx \cdot \Bu_1)(\Bu_2\Bu_3\cdots\Bu_r)(\Bu_r\Bu_{r-1}\cdots\Bu_1)
- (\Bx \cdot \Bu_2)(\Bu_1\Bu_3\Bu_4\cdots\Bu_r)(\Bu_{r-1}\cdots\Bu_1)
+ \cdots \\
&=
  (\Bx \cdot \Bu_1)\Bu_1
+ (\Bx \cdot \Bu_2)\Bu_2
+ \cdots \\
\end{aligned}
\end{equation}

This demonstrates that \((\Bx \cdot \BU) \frac{1}{\BU}\) is a vector.  Because all the potential \(3, 5, ... 2r-1\) grade terms of this product are zero one can write

\begin{equation}\label{eqn:plane:220}
(\Bx \cdot \BU) \frac{1}{\BU} = \left<(\Bx \cdot \BU) \frac{1}{\BU}\right>_1 = (\Bx \cdot \BU) \cdot \frac{1}{\BU}
\end{equation}

In general the product of a \(r-1\)-blade and an \(r\)-blade such as \((\Bx \cdot \BA_r)\BB_r)\) could potentially have any of these higher order terms.

Summarizing the results so far we have
\begin{equation}\label{eqn:plane:620}
\begin{aligned}
\Bx
&= (\Bx \cdot \BU)\frac{1}{\BU} + (\Bx \wedge \BU)\frac{1}{\BU} \\
&= (\Bx \cdot \BU) \cdot \frac{1}{\BU} + (\Bx \wedge \BU)\frac{1}{\BU} \\
\end{aligned}
\end{equation}

Since the RHS of this equation is a vector, this implies that the LHS is also a vector and thus
\begin{equation}\label{eqn:plane:640}
\begin{aligned}
(\Bx \wedge \BU)\frac{1}{\BU}
&= \left<(\Bx \wedge \BU)\frac{1}{\BU}\right>_1 \\
&= (\Bx \wedge \BU) \cdot \frac{1}{\BU} \\
\end{aligned}
\end{equation}

Thus we have an explicit formula for the projective and rejective terms of a vector with respect to a hypervolume element \(\BU\):

\begin{equation}\label{eqn:plane:660}
\begin{aligned}
\Bx
&= (\Bx \cdot \BU)\frac{1}{\BU} + (\Bx \wedge \BU)\frac{1}{\BU} \\
&= (\Bx \cdot \BU) \cdot \frac{1}{\BU} + (\Bx \wedge \BU) \cdot \frac{1}{\BU} \\
&= \frac{-1^{r(r-1)/2}}{\abs{\BU}^2}
\left( (\Bx \cdot \BU) \cdot {\BU} + (\Bx \wedge \BU) \cdot {\BU} \right) \\
\end{aligned}
\end{equation}

\subsection{Special note.  Interpretation for projection and rejective components of a line}

The proof above utilized the general definition of the dot product of two blades, the selection of the lowest grade element of the product:

\begin{equation}\label{eqn:plane:240}
\BA_k \cdot \BB_j = \left<\BA_k \BB_j\right>_{\abs{k-j}}
\end{equation}

Because of this, the scalar-vector dot product is perfectly well defined

\begin{equation}\label{eqn:plane:260}
a \cdot \Bb = \left<a \Bb\right>_{1-0} = a \Bb
\end{equation}

So, when \(\BU\) is a vector, the equations above also hold.

\subsection{Explicit expansion of projective and rejective components}

Having calculated the explicit vector expansion of the projective term to prove that the all the higher grade
product terms were zero, this can be used to explicitly expand the projective and rejective components
in terms of a set of unit vectors that span the hypervolume

\begin{equation}\label{eqn:plane:680}
\begin{aligned}
\Bx_\parallel
&= (\Bx \cdot \BU) \cdot \frac{1}{\BU} \\
&=
  (\Bx \cdot \Bu_1)\Bu_1
+ (\Bx \cdot \Bu_2)\Bu_2
+ \cdots \\
\Bx_\perp
&= (\Bx \wedge \BU) \cdot \frac{1}{\BU} \\
&= \Bx
- (\Bx \cdot \Bu_1)\Bu_1
- (\Bx \cdot \Bu_2)\Bu_2
- \cdots \\
\end{aligned}
\end{equation}

Recall here that the unit vectors \(\Bu_k\) are not the standard basis vectors.
They are instead an arbitrary set of orthonormal vectors that span the hypervolume element \(\BU\).

\section{Verification that projective and rejective components are orthogonal}

In NFCM, for the equation of a line, it is demonstrated that the two vector components (directrix and parametrization) are orthogonal, and that the directrix is the minimal distance to the line from the origin.  That can be done here too for the hypervolume result.

\begin{equation}\label{eqn:plane:700}
\begin{aligned}
\Bx \wedge \BU &= \Ba \wedge \BU \\
(\Bx \wedge \BU){\BU}^{-1} &= (\Ba \wedge \BU){\BU}^{-1} \\
(\Bx\BU - \Bx \cdot \BU){\BU}^{-1} &= (\Ba \wedge \BU){\BU}^{-1} \\
\Bx &= (\Bx \cdot \BU){\BU}^{-1} + (\Ba \wedge \BU){\BU}^{-1} \\
    &= \Balpha{\BU}^{-1} + \Bd \\
\end{aligned}
\end{equation}

This first component, the projective term \(\Balpha{\BU}^{-1} = (\Bx \cdot \BU){\BU}^{-1}\), can be interpreted as a parametrization term.  The
last component, the rejective term \(\Bd = (\Ba \wedge \BU){\BU}^{-1}\) is identified as the directrix.  Calculation of \(\abs{\Bx}\) allows us to verify the physical interpretation of this vector.

Expansion of the projective term has previously shown that given

\begin{equation}\label{eqn:plane:280}
\BU = k \Bu_1 \wedge \Bu_2 \cdots \wedge \Bu_r
\end{equation}

then the expansion of this parametrization term has the form

\begin{equation}\label{eqn:plane:300}
\Balpha = \left(\sum_{i = 1}^{r}{\alpha_i \Bu_i}\right)\BU
\end{equation}

This is a very specific parametrization, a \(r-1\) grade parametrization \(\Balpha\) with \(r\) free variables, producing
a vector directed strictly in hypervolume spanned by \(\BU\).

We can calculate the length of the projective component of \(\Bx\) expressed in terms of this parametrization:

\begin{equation}\label{eqn:plane:720}
\begin{aligned}
{\Bx_\parallel}^2
&= \left((\Bx \cdot \BU) {\BU}^{-1}\right)^2 \\
&= \Balpha \frac{{\BU}^\dagger}{\BU \BU^\dagger}   \left( \frac{{\BU}^\dagger}{\BU \BU^\dagger} \right)^\dagger {\Balpha}^\dagger \\
&= \Balpha \frac{{\BU}^\dagger} {\abs{\BU}^2} \frac{\BU} {\abs{\BU}^2} {\Balpha}^\dagger \\
&= \frac{\abs{\Balpha}^2}{{\abs{\BU}^2}} \\
\end{aligned}
\end{equation}

\begin{equation}\label{eqn:plane:740}
\begin{aligned}
{\Bx}^2
&= (\Balpha {\BU}^{-1})^2 + \Bd^2 + 2 (\Balpha \BU^{-1}) \cdot \Bd \\
&= \frac{\abs{\Balpha}^2}{{\abs{\BU}^2}} + \Bd^2 + 2 (\Balpha \BU^{-1}) \cdot \Bd
\end{aligned}
\end{equation}

Direct computation shows that this last dot product term is zero

\begin{equation}\label{eqn:plane:760}
\begin{aligned}
(\Balpha \BU^{-1}) \cdot \Bd
&= (\Balpha \BU^{-1}) \cdot ((\Ba \wedge \BU){\BU}^{-1}) \\
&= (\Balpha \BU^{-1}) \cdot ( {\BU}^{-1} (\BU \wedge \Ba)) \\
&= \frac{(-1)^{r(r-1)/2}}{\abs{\BU}^2} (\Balpha \BU^{-1}) \cdot ( \BU (\BU \wedge \Ba)) \\
&= \frac{(-1)^{r(r-1)/2}}{\abs{\BU}^2} \left<\Balpha \BU^{-1} \BU (\BU \wedge \Ba)\right>_0 \\
&= \frac{(-1)^{r(r-1)/2}}{\abs{\BU}^2} \left<\Balpha (\BU \wedge \Ba)\right>_0 \\
\end{aligned}
\end{equation}

This last term is a product of an \(r-1\) grade blade and a \(r+1\) grade blade.  The lowest order term of this product has grade \(r+1 -(r-1) = 2\), which
implies that
\(\left<\Balpha (\BU \wedge \Ba)\right>_0 = 0\).  This demonstrates explicitly that the parametrization term is perpendicular to the rejective term as expected.

The length from the origin to the volume is thus

\begin{equation}\label{eqn:plane:780}
\begin{aligned}
{\Bx}^2 &= \frac{\abs{\Balpha}^2}{{\abs{\BU}^2}} + \Bd^2 \\
\end{aligned}
\end{equation}

This is minimized when \(\Balpha = 0\).  Thus \(\Bd\) is a vector directed from the origin to the hypervolume, perpendicular to that hypervolume, and also has the minimal distance to that space.

