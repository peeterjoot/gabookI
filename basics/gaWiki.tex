%
% Copyright � 2012 Peeter Joot.  All Rights Reserved.
% Licenced as described in the file LICENSE under the root directory of this GIT repository.
%

%
%
%\input{../peeter_prologue.tex}

%\chapter{Comparison of many traditional vector and GA identities}
\index{identities}
\label{chap:gaWiki}


% does not work with _ character:
%%\blogpage{http://sites.google.com/site/peeterjoot/geometric-algebra/ga_wiki.pdf}
%%\date{ Oct 13, 2007 }
%\date{Oct 13, 2007.  gaWiki.tex}
%%\revisionInfo{\(RCSfile: gaWiki.tex,v \) Last \(Revision: 1.16 \) \(Date: 2009/10/22 02:07:20 \)}

\beginArtNoToc

\section{Three dimensional vector relationships vs N dimensional equivalents}

Here are some comparisons between standard \({\mathbb R}^3\) vector relations and their corresponding wedge and geometric product equivalents.  All the wedge and geometric product equivalents here are good for more than three dimensions, and some also for two.  In two dimensions the cross product is undefined even if what it describes (like torque) is a perfectly well defined in a plane without introducing an arbitrary normal vector outside of the space.

Many of these relationships only require the introduction of the wedge product to generalize, but since that may not be familiar to somebody with only a traditional background in vector algebra and calculus, some examples are given.

\subsection{wedge and cross products are antisymmetric}
\begin{equation}\label{eqn:gaWiki:20}
\begin{aligned}
\Bv \times \Bu = - (\Bu \times \Bv)
\end{aligned}
\end{equation}
\begin{equation}\label{eqn:gaWiki:40}
\begin{aligned}
\Bv \wedge \Bu = - (\Bu \wedge \Bv)
\end{aligned}
\end{equation}

\subsection{wedge and cross products are zero when identical}
\begin{equation}\label{eqn:gaWiki:60}
\begin{aligned}
\Bu \times \Bu = 0
\end{aligned}
\end{equation}
\begin{equation}\label{eqn:gaWiki:80}
\begin{aligned}
\Bu \wedge \Bu = 0
\end{aligned}
\end{equation}

\subsection{wedge and cross products are linear}

These are both linear in the first variable
\begin{equation}\label{eqn:gaWiki:100}
\begin{aligned}
(\Bv + \Bw) \times \Bw = \Bu \times \Bw + \Bv \times \Bw
\end{aligned}
\end{equation}
\begin{equation}\label{eqn:gaWiki:120}
\begin{aligned}
(\Bv + \Bw) \wedge \Bw = \Bu \wedge \Bw + \Bv \wedge \Bw
\end{aligned}
\end{equation}

and are linear in the second variable
\begin{equation}\label{eqn:gaWiki:140}
\begin{aligned}
\Bu \times (\Bv + \Bw)= \Bu \times \Bv + \Bu \times \Bw
\end{aligned}
\end{equation}
\begin{equation}\label{eqn:gaWiki:160}
\begin{aligned}
\Bu \wedge (\Bv + \Bw)= \Bu \wedge \Bv + \Bu \wedge \Bw
\end{aligned}
\end{equation}

\subsection{In general, cross product is not associative, but the wedge product is}
\begin{equation}\label{eqn:gaWiki:180}
\begin{aligned}
(\Bu \times \Bv) \times \Bw \neq \Bu \times (\Bv \times \Bw)
\end{aligned}
\end{equation}
\begin{equation}\label{eqn:gaWiki:200}
\begin{aligned}
(\Bu \wedge \Bv) \wedge \Bw = \Bu \wedge (\Bv \wedge \Bw)
\end{aligned}
\end{equation}

\subsection{Wedge and cross product relationship to a plane}
\(\Bu \times \Bv\) is perpendicular to plane containing \(\Bu\) and \(\Bv\).
\(\Bu \wedge \Bv\) is an oriented representation of the plane containing \(\Bu\) and \(\Bv\).

\subsection{norm of a vector}

The norm (length) of a vector is defined in terms of the dot product

\begin{equation}\label{eqn:gaWiki:220}
\begin{aligned}
 {\Vert \Bu \Vert}^2 = \Bu \cdot \Bu
\end{aligned}
\end{equation}

Using the geometric product this is also true, but this can be also be expressed more compactly as

\begin{equation}\label{eqn:gaWiki:240}
\begin{aligned}
{\Vert \Bu \Vert}^2 = {\Bu}^2
\end{aligned}
\end{equation}

This follows from the definition of the geometric product and the fact that a vector wedge product with itself is zero

\begin{equation}\label{eqn:gaWiki:260}
\begin{aligned}
 \Bu \, \Bu = \Bu \cdot \Bu + \Bu \wedge \Bu = \Bu \cdot \Bu
\end{aligned}
\end{equation}

\subsection{Lagrange identity}
\index{Lagrange identity}

In three dimensions the product of two vector lengths can be expressed in terms of the dot and cross products

\begin{equation}\label{eqn:gaWiki:280}
\begin{aligned}
{\Vert \Bu  \Vert}^2 {\Vert \Bv  \Vert}^2
=
({\Bu  \cdot \Bv })^2 + {\Vert \Bu  \times \Bv  \Vert}^2
\end{aligned}
\end{equation}

The corresponding generalization expressed using the geometric product is

\begin{equation}\label{eqn:gaWiki:300}
\begin{aligned}
{\Vert \Bu  \Vert}^2 {\Vert \Bv  \Vert}^2
= ({\Bu  \cdot \Bv })^2 - (\Bu  \wedge \Bv )^2
\end{aligned}
\end{equation}

This follows from by expanding the geometric product of a pair of vectors with its reverse

\begin{equation}\label{eqn:gaWiki:320}
\begin{aligned}
(\Bu  \Bv )(\Bv  \Bu )
= ({\Bu  \cdot \Bv } + {\Bu  \wedge \Bv }) ({\Bu  \cdot \Bv } - {\Bu  \wedge \Bv })
\end{aligned}
\end{equation}

\subsection{determinant expansion of cross and wedge products}
\index{wedge!determinant expansion}

\begin{equation}\label{eqn:gaWiki:340}
\begin{aligned}
\Bu \times \Bv = \sum_{i<j}{ \begin{vmatrix}u_i & u_j\\v_i & v_j\end{vmatrix}  {\Be}_i \times {\Be}_j }
\end{aligned}
\end{equation}
\begin{equation}\label{eqn:gaWiki:360}
\begin{aligned}
\Bu \wedge \Bv = \sum_{i<j}{ \begin{vmatrix}u_i & u_j\\v_i & v_j\end{vmatrix}  {\Be}_i \wedge {\Be}_j }
\end{aligned}
\end{equation}

Without justification or historical context, traditional linear algebra texts will often define the determinant as the first step of an elaborate sequence of definitions and theorems leading up to the solution of linear systems, Cramer's rule and matrix inversion.

An alternative treatment is to axiomatically introduce the wedge product, and then demonstrate that this can be used directly to solve linear systems.  This is shown below, and does not require sophisticated math skills to understand.

It is then possible to define determinants as nothing more than the coefficients of the wedge product in terms of "unit k-vectors" (\({\Be}_i \wedge {\Be}_j\) terms) expansions as above.

A one by one determinant is the coefficient of \(\Be _1\) for an \(\mathbb R^1\) 1-vector.

A two-by-two determinant is the coefficient of \(\Be _1 \wedge \Be _2\) for an \(\mathbb R^2\) bivector

A three-by-three determinant is the coefficient of \(\Be _1 \wedge \Be _2 \wedge \Be _3\) for an \(\mathbb R^3\) trivector

When linear system solution is introduced via the wedge product, Cramer's rule follows as a side effect, and there is no need to lead up to the end results with definitions of minors, matrices, matrix invertablity, adjoints, cofactors, Laplace expansions, theorems on determinant multiplication and row column exchanges, and so forth.

\subsection{Equation of a plane}
\index{plane!equation}

For the plane of all points \({\Br}\) through the plane passing through three independent points \({\Br}_0\), \({\Br}_1\), and \({\Br}_2\), the normal form of the equation is

\begin{equation}\label{eqn:gaWiki:380}
\begin{aligned}
(({\Br}_2 - {\Br}_0) \times ({\Br}_1 - {\Br}_0)) \cdot ({\Br} - {\Br}_0) = 0
\end{aligned}
\end{equation}

The equivalent wedge product equation is
\begin{equation}\label{eqn:gaWiki:400}
\begin{aligned}
({\Br}_2 - {\Br}_0) \wedge ({\Br}_1 - {\Br}_0) \wedge ({\Br} - {\Br}_0) = 0
\end{aligned}
\end{equation}

\subsection{Projective and rejective components of a vector}

For three dimensions the projective and rejective components of a vector with respect to an arbitrary non-zero unit vector, can be expressed in terms of the dot and cross product

\begin{equation}\label{eqn:gaWiki:420}
\begin{aligned}
\Bv = (\Bv \cdot \ucap)\ucap + \ucap \times (\Bv \times \ucap)
\end{aligned}
\end{equation}

For the general case the same result can be written in terms of the dot and wedge product and the geometric product of that and the unit vector

\begin{equation}\label{eqn:gaWiki:440}
\begin{aligned}
\Bv = (\Bv \cdot \ucap)\ucap + (\Bv \wedge \ucap) \ucap
\end{aligned}
\end{equation}

It is also worthwhile to point out that this result can also be expressed using right or left vector division as defined by the geometric product

\begin{equation}\label{eqn:gaWiki:460}
\begin{aligned}
\Bv = (\Bv \cdot \Bu)\frac{1}{\Bu} + (\Bv \wedge \Bu) \frac{1}{\Bu}
\end{aligned}
\end{equation}
\begin{equation}\label{eqn:gaWiki:480}
\begin{aligned}
\Bv = \frac{1}{\Bu}(\Bu \cdot \Bv) + \frac{1}{\Bu}(\Bu \wedge \Bv)
\end{aligned}
\end{equation}

\subsection{Area (squared) of a parallelogram is norm of cross product}
\index{parallelogram!area}

\begin{equation}\label{eqn:gaWiki:500}
\begin{aligned}
A^2 = {\Vert \Bu \times \Bv \Vert}^2 = \sum_{i<j}{\begin{vmatrix}u_i & u_j\\v_i & v_j\end{vmatrix}}^2
\end{aligned}
\end{equation}

and is the negated square of a wedge product
\begin{equation}\label{eqn:gaWiki:520}
\begin{aligned}
A^2 = -(\Bu \wedge \Bv)^2 = \sum_{i<j}{\begin{vmatrix}u_i & u_j\\v_i & v_j\end{vmatrix}}^2
\end{aligned}
\end{equation}

Note that this squared bivector is a geometric product.

\subsection{Angle between two vectors}
\index{vectors!angle between}

\begin{equation}\label{eqn:gaWiki:540}
\begin{aligned}
({\sin \theta})^2 = \frac{{\Vert \Bu \times \Bv \Vert}^2}{{\Vert \Bu \Vert}^2 {\Vert \Bv \Vert}^2}
\end{aligned}
\end{equation}
\begin{equation}\label{eqn:gaWiki:560}
\begin{aligned}
({\sin \theta})^2 = -\frac{(\Bu \wedge \Bv)^2}{{ \Bu }^2 { \Bv }^2}
\end{aligned}
\end{equation}

\subsection{Volume of the parallelepiped formed by three vectors}
\index{parallelepiped!volume}

\begin{equation}\label{eqn:gaWiki:580}
\begin{aligned}
V^2 = {\Vert (\Bu \times \Bv) \cdot \Bw \Vert}^2
= {
\begin{vmatrix}
u_1 & u_2 & u_3 \\
v_1 & v_2 & v_3 \\
w_1 & w_2 & w_3 \\
\end{vmatrix}
}^2
\end{aligned}
\end{equation}

\begin{equation}\label{eqn:gaWiki:600}
\begin{aligned}
V^2 = -(\Bu \wedge \Bv \wedge \Bw)^2
= -\left(\sum_{i<j<k}
\begin{vmatrix}
u_i & u_j & u_k \\
v_i & v_j & v_k \\
w_i & w_j & w_k \\
\end{vmatrix}
\ecap_i \wedge \ecap_j \wedge \ecap_k
\right)^2
= \sum_{i<j<k}
{
\begin{vmatrix}
u_i & u_j & u_k \\
v_i & v_j & v_k \\
w_i & w_j & w_k \\
\end{vmatrix}
}^2
\end{aligned}
\end{equation}


\section{Some properties and examples}

Some fundamental geometric algebra manipulations will be provided below, showing how this vector product can be used in calculation of projections, area, and rotations.  How some of these tie together and correlate concepts from other branches of mathematics, such as complex numbers, will also be shown.

In some cases these examples provide details used above in the cross product and geometric product comparisons.

\subsection{Inversion of a vector}
\index{vector!inversion}

One of the powerful properties of the Geometric product is that it provides the capability to express the inverse of a non-zero vector.  This is expressed by:

\begin{equation}\label{eqn:gaWiki:620}
\begin{aligned}
{\Ba}^{-1} = \frac{\Ba}{\Ba \Ba} = \frac{\Ba}{{\Vert \Ba \Vert}^2}.
\end{aligned}
\end{equation}

\subsection{dot and wedge products defined in terms of the geometric product}

Given a definition of the geometric product in terms of the dot and wedge products, adding and subtracting \(\Ba  \Bb \) and \(\Bb  \Ba \) demonstrates that the dot and wedge product of two vectors can also be defined in terms of the geometric product

\subsection{The dot product}
\index{dot product}

\begin{equation}\label{eqn:gaWiki:640}
\begin{aligned}
\Ba \cdot\Bb  = \frac{1}{2}(\Ba \Bb  + \Bb \Ba )
\end{aligned}
\end{equation}

This is the symmetric component of the geometric product.  When two vectors are colinear the geometric and dot products of those vectors are equal.

As a motivation for the dot product it is normal to show that this quantity occurs in the solution of the length of a general triangle where the third side is the vector sum of the first and second sides \(\Bc  = \Ba  + \Bb \).

\begin{equation}\label{eqn:gaWiki:660}
\begin{aligned}
{\Vert \Bc  \Vert}^2 = \sum_{i}(a_i + b_i)^2 = {\Vert \Ba  \Vert}^2 + {\Vert \Bb  \Vert}^2 + 2 \sum_{i}a_i b_i
\end{aligned}
\end{equation}

The last sum is then given the name the dot product and other properties of this quantity are then shown (projection, angle between vectors, ...).

This can also be expressed using the geometric product

\begin{equation}\label{eqn:gaWiki:680}
\begin{aligned}
\Bc ^2 = (\Ba  + \Bb )(\Ba  + \Bb ) = \Ba ^2 + \Bb ^2 + (\Ba \Bb  + \Bb \Ba )
\end{aligned}
\end{equation}

By comparison, the following equality exists

\begin{equation}\label{eqn:gaWiki:700}
\begin{aligned}
\sum_{i}a_i b_i = \frac{1}{2}(\Ba \Bb  + \Bb \Ba )
\end{aligned}
\end{equation}

Without requiring expansion by components one can define the dot product exclusively in terms of the geometric product due to its properties of contraction, distribution and associativity.  This is arguably a more natural way to define the geometric product.  Addition of two similar terms is not immediately required, especially since one of those terms is the wedge product which may also be unfamiliar.

\subsection{The wedge product}
\index{wedge product}

\begin{equation}\label{eqn:gaWiki:720}
\begin{aligned}
\Ba \wedge\Bb  = \frac{1}{2}(\Ba \Bb  - \Bb \Ba )
\end{aligned}
\end{equation}

This is the antisymmetric component of the geometric product.  When two vectors are orthogonal the geometric and wedge products of those vectors are equal.

Switching the order of the vectors negates this antisymmetric geometric product component, and contraction property shows that this is zero if the vectors are equal.  These are the defining properties of the wedge product.

\subsection{Note on symmetric and antisymmetric dot and wedge product formulas}
\index{dot product!symmetric sum}
\index{wedge product!antisymmetric sum}

A generalization of the dot product that allows computation of the component of a vector "in the direction" of a plane (bivector), or other k-vectors can be found below.  Since the signs change depending on the grades of the terms being multiplied, care is required with the formulas above to ensure that they are only used for a pair of vectors.

\subsection{Reversing multiplication order.  Dot and wedge products compared to the real and imaginary parts of a complex number}

Reversing the order of multiplication of two vectors, has the effect of the inverting the sign of just the wedge product term of the product.

It is not a coincidence that this is a similar operation to the conjugate operation of complex numbers.

The reverse of a product is written in the following fashion

\begin{equation}\label{eqn:gaWiki:740}
\begin{aligned}
{\Bb  \Ba } = ({\Ba  \Bb })^\dagger
\end{aligned}
\end{equation}
\begin{equation}\label{eqn:gaWiki:760}
\begin{aligned}
{\Bc  \Bb  \Ba } = ({\Ba  \Bb  \Bc })^\dagger
\end{aligned}
\end{equation}

Expressed this way the dot and wedge products are

\begin{equation}\label{eqn:gaWiki:780}
\begin{aligned}
\Ba \cdot\Bb  = \frac{1}{2}(\Ba \Bb  + ({\Ba  \Bb })^\dagger)
\end{aligned}
\end{equation}

This is the symmetric component of the geometric product.  When two vectors are colinear the geometric and dot products of those vectors are equal.

\begin{equation}\label{eqn:gaWiki:800}
\begin{aligned}
\Ba \wedge\Bb  = \frac{1}{2}(\Ba \Bb  - ({\Ba  \Bb })^\dagger)
\end{aligned}
\end{equation}

These symmetric and antisymmetric pairs, the dot and wedge products extract the scalar and bivector components of a geometric product in the same fashion as the real and imaginary components of a complex number are also extracted by its symmetric and antisymmetric components

\begin{equation}\label{eqn:gaWiki:820}
\begin{aligned}
\mathop{Re}(z) = \frac{1}{2}(z + \overbar{z})
\end{aligned}
\end{equation}
\begin{equation}\label{eqn:gaWiki:840}
\begin{aligned}
\mathop{Im}(z) = \frac{1}{2}(z - \overbar{z})
\end{aligned}
\end{equation}

This extraction of components also applies to higher order geometric product terms.  For example

\begin{equation}\label{eqn:gaWiki:860}
\begin{aligned}
\Ba \wedge\Bb \wedge \Bc
= \frac{1}{2}(\Ba \Bb \Bc  - ({\Ba  \Bb } \Bc )^\dagger)
= \frac{1}{2}(\Bb \Bc \Ba  - ({\Bb  \Bc } \Ba )^\dagger)
= \frac{1}{2}(\Bc \Ba \Bb  - ({\Bc  \Ba } \Bb )^\dagger)
\end{aligned}
\end{equation}

\subsection{Orthogonal decomposition of a vector}

Using the \textAndIndex{Gram-Schmidt} process a single vector can be decomposed into two components with respect to a reference vector, namely the projection onto a unit vector in a reference direction, and the difference between the vector and that projection.

With, \( \ucap = \Bu / {\Vert \Bu \Vert}\), the projection of \(\Bv\) onto \( \ucap\) is

\begin{equation}\label{eqn:gaWiki:880}
\begin{aligned}
 \mathrm{Proj}_{\ucap}\,\Bv  = \ucap (\ucap \cdot \Bv)
\end{aligned}
\end{equation}

Orthogonal to that vector is the difference, designated the rejection,

\begin{equation}\label{eqn:gaWiki:900}
\begin{aligned}
 \Bv - \ucap (\ucap \cdot \Bv) = \frac{1}{{\Vert \Bu \Vert}^2} ( {\Vert \Bu \Vert}^2 \Bv - \Bu (\Bu \cdot \Bv))
\end{aligned}
\end{equation}

The rejection can be expressed as a single geometric algebraic product in a few different ways

\begin{equation}\label{eqn:gaWiki:920}
\begin{aligned}
 \frac{ \Bu }{{\Bu}^2} ( \Bu \Bv - \Bu \cdot \Bv)
= \frac{1}{\Bu} ( \Bu \wedge \Bv )
= \ucap ( \ucap \wedge \Bv )
= ( \Bv \wedge \ucap ) \ucap
\end{aligned}
\end{equation}

The similarity in form between between the projection and the rejection is notable.  The sum of these recovers the original vector

\begin{equation}\label{eqn:gaWiki:940}
\begin{aligned}%\label{eqn:gaWiki:orthoD}
 \Bv = \ucap (\ucap \cdot \Bv) + \ucap ( \ucap \wedge \Bv )
\end{aligned}
\end{equation}

Here the projection is in its customary vector form.  An alternate formulation is possible that puts the projection in a form that differs from the usual vector formulation

\begin{equation}\label{eqn:gaWiki:960}
\begin{aligned}
 \Bv
= \frac{1}{\Bu} (\Bu \cdot \Bv) + \frac{1}{\Bu} ( \Bu \wedge \Bv )
= (\Bv \cdot \Bu) \frac{1}{\Bu}  + ( \Bv \wedge \Bu ) \frac{1}{\Bu}
\end{aligned}
\end{equation}

\subsection{A quicker way to the end result}

Working backwards from the end result, it can be observed that this orthogonal decomposition result can in fact follow more directly from the definition of the geometric product itself.

\begin{equation}\label{eqn:gaWiki:980}
\begin{aligned}
\Bv = \ucap \ucap \Bv
= \ucap (\ucap \cdot \Bv + \ucap \wedge \Bv )
\end{aligned}
\end{equation}

With this approach, the original geometrical consideration is not necessarily obvious, but it is a much quicker way to get at the same algebraic result.

However, the hint that one can work backwards, coupled with the knowledge that the wedge product can be used to solve sets of linear equations,
\footnote{
http://www.grassmannalgebra.info/grassmannalgebra/book/bookpdf/TheExteriorProduct.pdf}
the problem of orthogonal decomposition can be posed directly,

Let \(\Bv = a \Bu + \Bx\), where \(\Bu \cdot \Bx = 0\).  To discard the portions of \(\Bv\) that are colinear with \(\Bu\), take the wedge product

\begin{equation}\label{eqn:gaWiki:1000}
\begin{aligned}
\Bu \wedge \Bv = \Bu \wedge (a \Bu + \Bx) = \Bu \wedge \Bx
\end{aligned}
\end{equation}

Here the geometric product can be employed

\begin{equation}\label{eqn:gaWiki:1020}
\begin{aligned}
\Bu \wedge \Bv = \Bu \wedge \Bx = \Bu \Bx - \Bu \cdot \Bx = \Bu \Bx
\end{aligned}
\end{equation}

Because the geometric product is invertible, this can be solved for x

\begin{equation}\label{eqn:gaWiki:1040}
\begin{aligned}
\Bx = \frac{1}{\Bu}(\Bu \wedge \Bv)
\end{aligned}
\end{equation}

The same techniques can be applied to similar problems, such as calculation of the component of a vector in a plane and perpendicular to the plane.

\subsection{Area of parallelogram spanned by two vectors}
\index{parallelogram!area}

\imageFigure{../gabook-figures/parallelogramArea}{parallelogramArea}{fig:parallelogramArea}{0.4}

As depicted in \cref{fig:parallelogramArea}, one can see that the area of a parallelogram spanned by two vectors is computed from the base times height.  In the figure \(\Bu\) was picked as the base, with length \(\Norm{\Bu}\).  Designating the second vector \(\Bv\), we want the component of \(\Bv\) perpendicular to \(\ucap\) for the height.  An orthogonal decomposition of \(\Bv\) into directions parallel and perpendicular to \(\ucap\) can be performed in two ways.

\begin{equation}\label{eqn:gaWiki:1060}
\begin{aligned}
\Bv &= \Bv \ucap \ucap = (\Bv \cdot \ucap) \ucap + (\Bv \wedge \ucap) \ucap \\
    &= \ucap \ucap \Bv = \ucap (\ucap \cdot \Bv) + \ucap (\ucap \wedge \Bv)
\end{aligned}
\end{equation}

The height is the length of the perpendicular component expressed using the wedge as either \(\ucap (\ucap \wedge \Bv)\) or \((\Bv \wedge \ucap) \ucap\).

Multiplying base times height we have the parallelogram area

\begin{equation}\label{eqn:gaWiki:1080}
\begin{aligned}
A(\Bu,\Bv)
&= \Vert \Bu \Vert \Vert \ucap ( \ucap \wedge \Bv ) \Vert \\
&= \Vert \ucap ( \Bu \wedge \Bv ) \Vert \\
\end{aligned}
\end{equation}

Since the squared length of an Euclidean vector is the geometric square of that vector, we can compute the squared area of this parallogram by squaring this single scaled vector

\begin{equation}\label{eqn:gaWiki:1100}
\begin{aligned}
A^2 &= (\ucap ( \Bu \wedge \Bv ) )^2
\end{aligned}
\end{equation}

Utilizing both encodings of the perpendicular to \(\ucap\) component of \(\Bv\) computed above we have for the squared area

\begin{equation}\label{eqn:gaWiki:1120}
\begin{aligned}
A^2
&= (\ucap( \Bu \wedge {\Bv} ) )^2 \\
&= (( \Bv \wedge {\Bu} ) \ucap) (\ucap ( {\Bu} \wedge \Bv )) \\
&= ( \Bv \wedge \Bu ) ( \Bu \wedge \Bv ) \\
\end{aligned}
\end{equation}

Since \(\Bu \wedge \Bv = -\Bv \wedge \Bu\), we have finally

\begin{equation}\label{eqn:gaWiki:1140}
\begin{aligned}
A^2 = -( \Bu \wedge \Bv )^2
\end{aligned}
\end{equation}

There are a few things of note here.  One is that the parallelogram area can easily be expressed in terms of the square of a bivector.  Another is that the square of a bivector has the same property as a purely imaginary number, a negative square.

It can also be noted that a vector lying completely within a plane anticommutes with the bivector for that plane.  More generally components of vectors that lie within a plane commute with the bivector for that plane while the perpendicular components of that vector commute.  These commutation or anticommutation properties depend both on the vector and the grade of the object that one attempts to commute it with (these properties lie behind the generalized definitions of the dot and wedge product to be seen later).

% SCOTT:
% - Section 3.2.9. This is more of a comment. The commuting of the geometric product in the second line of the equation for A^2 uses the idea that u_hat is in the plane of the bivector, therefore the wedge product is zero. It would be nice if this reasoning were given. In fact, I think it would be very benificial if there were a section near the beginning that pretty much laid out what is possible with the geometric product and when. Maybe a table of sorts with columns signifying that when you fit certain criteria, orthogonal/parallel/in a plane/etc, that various properties like commutation/associativity/etc work.

\subsection{Expansion of a bivector and a vector rejection in terms of the standard basis}
\index{rejection}

If a vector is factored directly into projective and rejective terms using the geometric product \(\Bv = \frac{1}{\Bu}( \Bu \cdot \Bv + \Bu \wedge \Bv)\), then it is not necessarily obvious that the rejection term, a product of vector and bivector is even a vector.  Expansion of the vector bivector product in terms of the standard basis vectors has the following form

Let
\begin{equation}\label{eqn:gaWiki:1160}
\begin{aligned}
\Br
= \frac{1}{\Bu} ( \Bu \wedge \Bv )
= \frac{\Bu}{\Bu^2} ( \Bu \wedge \Bv )
= \frac{1}{{\Vert \Bu \Vert}^2} \Bu ( \Bu \wedge \Bv )
\end{aligned}
\end{equation}

It can be shown that
\begin{equation}\label{eqn:gaWiki:1180}
\begin{aligned}
\Br = \frac{1}{{\Vert{\Bu}\Vert}^2} \sum_{i<j}\begin{vmatrix}u_i & u_j\\v_i & v_j\end{vmatrix}
\begin{vmatrix}u_i & u_j\\ {\Be}_i & {\Be}_j\end{vmatrix}
\end{aligned}
\end{equation}

(a result that can be shown more easily straight from \(\Br = \Bv - \ucap (\ucap \cdot \Bv)\)).

The rejective term is perpendicular to \(\Bu\), since
$\begin{vmatrix}
u_i & u_j\\ u_i & u_j
\end{vmatrix}
 = 0$
implies \(\Br \cdot \Bu = \Bzero\).

The magnitude of \(\Br\), is

\begin{equation}\label{eqn:gaWiki:1200}
\begin{aligned}
{\Vert \Br \Vert}^2 = \Br \cdot \Bv = \frac{1}{{\Vert{\Bu}\Vert}^2} \sum_{i<j}\begin{vmatrix}u_i & u_j\\v_i & v_j\end{vmatrix}^2
\end{aligned}
\end{equation}.

So, the quantity

\begin{equation}\label{eqn:gaWiki:1220}
\begin{aligned}
{\Vert \Br \Vert}^2 {\Vert{\Bu}\Vert}^2 = \sum_{i<j}\begin{vmatrix}u_i & u_j\\v_i & v_j\end{vmatrix}^2
\end{aligned}
\end{equation}

is the squared area of the parallelogram formed by \(\Bu\) and \(\Bv\).

It is also noteworthy that the bivector can be expressed as

\begin{equation}\label{eqn:gaWiki:1240}
\begin{aligned}
\Bu \wedge \Bv = \sum_{i<j}{ \begin{vmatrix}u_i & u_j\\v_i & v_j\end{vmatrix}  {\Be}_i \wedge {\Be}_j }
\end{aligned}
\end{equation}.

Thus is it natural, if one considers each term \({\Be}_i \wedge {\Be}_j\) as a basis vector of the bivector space, to define the (squared) "length" of that bivector as the (squared) area.

Going back to the geometric product expression for the length of the rejection \(\frac{1}{\Bu} ( \Bu \wedge \Bv )\) we see that the length of the quotient, a vector, is in this case is the "length" of the bivector divided by the length of the divisor.

This may not be a general result for the length of the product of two \(k\)-vectors, however it is a result that may help build some intuition about the significance of the algebraic operations.  Namely,

When a vector is divided out of the plane (parallelogram span) formed from it and another vector, what remains is the perpendicular component of the remaining vector, and its length is the planar area divided by the length of the vector that was divided out.

\subsection{Projection and rejection of a vector onto and perpendicular to a plane}
\index{plane!projection}
\index{plane!rejection}

Like vector projection and rejection, higher dimensional analogs of that calculation are also possible using the geometric product.

As an example, one can calculate the component of a vector perpendicular to a plane and the projection of that vector onto the plane.

Let \(\Bw = a \Bu + b \Bv + \Bx\), where \(\Bu \cdot \Bx = \Bv \cdot \Bx = 0\).  As above, to discard the portions of \(\Bw\) that are colinear with \(\Bu\) or \(\Bu\), take the wedge product

\begin{equation}\label{eqn:gaWiki:1260}
\begin{aligned}
\Bw \wedge \Bu \wedge \Bv = (a \Bu + b \Bv + \Bx) \wedge \Bu \wedge \Bv = \Bx \wedge \Bu \wedge \Bv
\end{aligned}
\end{equation}

Having done this calculation with a vector projection, one can guess that this quantity equals \(\Bx (\Bu \wedge \Bv)\).  One can also guess there is a vector and bivector dot product like quantity such that the allows the calculation of the component of a vector that is in the "direction of a plane".  Both of these guesses are correct, and the validating these facts is worthwhile.  However, skipping ahead slightly, this to be proved fact allows for a nice closed form solution of the vector component outside of the plane:

\begin{equation}\label{eqn:gaWiki:1280}
\begin{aligned}
\Bx
= (\Bw \wedge \Bu \wedge \Bv)\frac{1}{\Bu \wedge \Bv}
= \frac{1}{\Bu \wedge \Bv}(\Bu \wedge \Bv  \wedge \Bw)
\end{aligned}
\end{equation}

Notice the similarities between this planar rejection result a the vector rejection result.  To calculation the component of a vector outside of a plane we take the volume spanned by three vectors (trivector) and "divide out" the plane.

Independent of any use of the geometric product it can be shown that this rejection in terms of the standard basis is

\begin{equation}\label{eqn:gaWiki:1300}
\begin{aligned}
\Bx = \frac{1}{(A_{u,v})^2} \sum_{i<j<k}
\begin{vmatrix}w_i & w_j & w_k \\u_i & u_j & u_k \\v_i & v_j & v_k \\\end{vmatrix}
\begin{vmatrix}u_i & u_j & u_k \\v_i & v_j & v_k \\ {\Be}_i & {\Be}_j & {\Be}_k \\ \end{vmatrix}
\end{aligned}
\end{equation}

Where

\begin{equation}\label{eqn:gaWiki:1320}
\begin{aligned}
(A_{u,v})^2
= \sum_{i<j} \begin{vmatrix}u_i & u_j\\v_i & v_j\end{vmatrix}
= -(\Bu \wedge \Bv)^2
\end{aligned}
\end{equation}

is the squared area of the parallelogram formed by \(\Bu\), and \(\Bv\).

The (squared) magnitude of \(\Bx\) is

\begin{equation}\label{eqn:gaWiki:1340}
\begin{aligned}
{\Vert \Bx \Vert}^2 =
\Bx \cdot \Bw =
\frac{1}{(A_{u,v})^2} \sum_{i<j<k}
{\begin{vmatrix}w_i & w_j & w_k \\u_i & u_j & u_k \\v_i & v_j & v_k \\\end{vmatrix}}^2
\end{aligned}
\end{equation}

Thus, the (squared) volume of the parallelepiped (base area times perpendicular height) is

\begin{equation}\label{eqn:gaWiki:1360}
\begin{aligned}
\sum_{i<j<k}
{\begin{vmatrix}w_i & w_j & w_k \\u_i & u_j & u_k \\v_i & v_j & v_k \\\end{vmatrix}}^2
\end{aligned}
\end{equation}

Note the similarity in form to the w,u,v trivector itself

\begin{equation}\label{eqn:gaWiki:1380}
\begin{aligned}
\sum_{i<j<k}
{\begin{vmatrix}w_i & w_j & w_k \\u_i & u_j & u_k \\v_i & v_j & v_k \\\end{vmatrix}} {\Be}_i \wedge {\Be}_j \wedge {\Be}_k
\end{aligned}
\end{equation}

which, if you take the set of \({\Be}_i \wedge {\Be}_j \wedge {\Be}_k\) as a basis for the trivector space, suggests this is the natural way to define the length of a trivector.  Loosely speaking the length of a vector is a length, length of a bivector is area, and the length of a trivector is volume.

\subsection{Product of a vector and bivector.  Defining the "dot product" of a plane and a vector}

In order to justify the normal to a plane result above, a general examination of the product of a vector and bivector is required.  Namely,

\begin{equation}\label{eqn:gaWiki:1400}
\begin{aligned}
\Bw (\Bu \wedge \Bv)
= \sum_{i,j<k}w_i {\Be}_i {\begin{vmatrix}u_j & u_k \\v_j & v_k \\\end{vmatrix}} {\Be}_j \wedge {\Be}_k
\end{aligned}
\end{equation}

This has two parts, the vector part where \(i=j\) or \(i=k\), and the trivector parts where no indices equal.  After some index summation trickery, and grouping terms and so forth, this is


\begin{equation}\label{eqn:gaWiki:1420}
\begin{aligned}
\Bw (\Bu \wedge \Bv) =
\sum_{i<j}(w_i {\Be}_j
- w_j {\Be}_i )
{\begin{vmatrix}u_i & u_j \\v_i & v_j \\\end{vmatrix}}
+
\sum_{i<j<k}
{\begin{vmatrix}w_i & w_j & w_k \\ u_i & u_j & u_k \\v_i & v_j & v_k \\\end{vmatrix}}
{\Be}_i \wedge {\Be}_j \wedge {\Be}_k
\end{aligned}
\end{equation}

The trivector term is \(\Bw \wedge \Bu \wedge \Bv\).  Expansion of \((\Bu \wedge \Bv) \Bw\) yields the same trivector term.  This is the completely symmetric part, and the vector term is negated.
Like the geometric product of two vectors, this geometric product can be grouped into symmetric and antisymmetric parts, one of which is a pure k-vector.  In analogy the antisymmetric part of this product can be called a generalized dot product, and is roughly speaking the dot product of a "plane" (bivector), and a vector.

The properties of this generalized dot product remain to be explored, but first here is a summary of the notation

\begin{equation}\label{eqn:gaWiki:1440}
\begin{aligned}
\Bw (\Bu \wedge \Bv) = \Bw \cdot (\Bu \wedge \Bv) + \Bw \wedge \Bu \wedge \Bv
\end{aligned}
\end{equation}

\begin{equation}\label{eqn:gaWiki:1460}
\begin{aligned}
(\Bu \wedge \Bv) \Bw = - \Bw \cdot (\Bu \wedge \Bv) + \Bw \wedge \Bu \wedge \Bv
\end{aligned}
\end{equation}

\begin{equation}\label{eqn:gaWiki:1480}
\begin{aligned}
\Bw \wedge \Bu \wedge \Bv = \frac{1}{2}(\Bw (\Bu \wedge \Bv) + (\Bu \wedge \Bv) \Bw)
\end{aligned}
\end{equation}

\begin{equation}\label{eqn:gaWiki:1500}
\begin{aligned}
\Bw \cdot (\Bu \wedge \Bv) = \frac{1}{2}(\Bw (\Bu \wedge \Bv) - (\Bu \wedge \Bv) \Bw)
\end{aligned}
\end{equation}

Let \(\Bw = \Bx + \By\), where \(\Bx = a \Bu + b \Bv\), and \(\By \cdot \Bu = \By \cdot \Bv = \Bzero\).  Expressing \(\Bw\) and the \(\Bu \wedge \Bv\), products in terms of these components is

\begin{equation}\label{eqn:gaWiki:1520}
\begin{aligned}
\Bw (\Bu \wedge \Bv) = \Bx (\Bu \wedge \Bv) + \By (\Bu \wedge \Bv)
=
\Bx \cdot (\Bu \wedge \Bv) + \By \cdot (\Bu \wedge \Bv) + \By \wedge \Bu \wedge \Bv
\end{aligned}
\end{equation}

With the conditions and definitions above, and some manipulation, it can be shown that the term \(\By \cdot (\Bu \wedge \Bv) = \Bzero\), which then justifies the previous solution of the normal to a plane problem.  Since the vector term of the vector bivector product the name dot product is zero
when the vector is perpendicular to the plane (bivector), and this vector, bivector "dot product" selects only the components that are in the plane, so in analogy to the vector-vector dot product this name itself is justified by more than the fact this is the non-wedge product term of the geometric vector-bivector product.

\subsection{Complex numbers}
\index{complex numbers}
There is a one to one correspondence between the geometric product of two \(\mathbb{R}^2\) vectors and the field of complex numbers.

Writing, a vector in terms of its components, and left multiplying by the unit vector \({\Be}_1\) yields

\begin{equation}\label{eqn:gaWiki:1540}
\begin{aligned}
 Z = {\Be}_1 \BP = {\Be}_1 ( x {\Be}_1 + y {\Be}_2)
= x (1) + y ({\Be}_1 {\Be}_2)
= x (1) + y ({\Be}_1 \wedge {\Be}_2)
\end{aligned}
\end{equation}

The unit scalar and unit bivector pair \(1, {\Be}_1 \wedge {\Be}_2\) can be considered an alternate basis for a two dimensional vector space.  This alternate vector representation is closed with respect to the geometric product

\begin{equation}\label{eqn:gaWiki:1560}
\begin{aligned}
 Z_1 Z_2
&= {\Be}_1 ( x_1 {\Be}_1 + y_1 {\Be}_2) {\Be}_1 ( x_2 {\Be}_1 + y_2 {\Be}_2) \\
&= ( x_1 + y_1 {\Be}_1 {\Be}_2) ( x_2 + y_2 {\Be}_1 {\Be}_2) \\
&= x_1 x_2 + y_1 y_2 ({\Be}_1 {\Be}_2) {\Be}_1 {\Be}_2) \\
+ (x_1 y_2 + x_2 y_1) {\Be}_1 {\Be}_2 \\
\end{aligned}
\end{equation}

This closure can be observed after calculation of the square of the unit bivector above, a quantity

\begin{equation}\label{eqn:gaWiki:1580}
\begin{aligned}
({\Be}_1 \wedge {\Be}_2)^2 = {\Be}_1 {\Be}_2 {\Be}_1 {\Be}_2 = - {\Be}_1 {\Be}_1 {\Be}_2 {\Be}_2 = -1
\end{aligned}
\end{equation}

that has the characteristics of the complex number \(i^2 = -1\).

This fact allows the simplification of the product above to

\begin{equation}\label{eqn:gaWiki:1600}
\begin{aligned}
Z_1 Z_2
= (x_1 x_2 - y_1 y_2) + (x_1 y_2 + x_2 y_1) ({\Be}_1 \wedge {\Be}_2)
\end{aligned}
\end{equation}

Thus what is traditionally the defining, and arguably arbitrary seeming, rule of complex number multiplication, is found to follow naturally from the higher order structure of the geometric product, once that is applied to a two dimensional vector space.

It is also informative to examine how the length of a vector can be represented in terms of a complex number.  Taking the square of the length

\begin{equation}\label{eqn:gaWiki:1620}
\begin{aligned}
\BP \cdot \BP &= ( x {\Be}_1 + y {\Be}_2) \cdot ( x {\Be}_1 + y {\Be}_2) \\
&= ({\Be}_1 Z) {\Be}_1 Z \\
&= (( x  - y {\Be}_1 {\Be}_2) {\Be}_1) {\Be}_1 Z \\
&= ( x  - y ({\Be}_1 \wedge {\Be}_2)) Z \\
\end{aligned}
\end{equation}

This right multiplication of a vector with \({\Be}_1\), is named the conjugate

\begin{equation}\label{eqn:gaWiki:1640}
\begin{aligned}
\overline{Z} = x  - y ({\Be}_1 \wedge {\Be}_2)
\end{aligned}
\end{equation}

And with that definition, the length of the original vector can be expressed as

\begin{equation}\label{eqn:gaWiki:1660}
\begin{aligned}
\BP \cdot \BP = \overline{Z}Z
\end{aligned}
\end{equation}

This is also a natural definition of the length of a complex number, given the fact that the complex numbers can be considered an isomorphism with the two dimensional Euclidean vector space.

\subsection{Rotation in an arbitrarily oriented plane}
\index{plane!rotation}

A point \(\BP\), of radius \(\Br\), located at an angle \(\theta\) from the vector \(\ucap\) in the direction from \(\Bu\) to \(\Bv\), can be expressed as

\begin{equation}\label{eqn:gaWiki:1680}
\begin{aligned}
\BP = r( \ucap \cos{\theta} +
\frac{\ucap (\ucap \wedge \Bv)}{\Vert \ucap (\ucap \wedge \Bv) \Vert}  \sin{\theta})
=
r \ucap
( \cos{\theta} +
\frac{(\Bu \wedge \Bv)}{\Vert \ucap (\Bu \wedge \Bv) \Vert} \sin{\theta})
\end{aligned}
\end{equation}

Writing \( {\BI}_{\Bu ,\Bv } = \frac{\Bu \wedge \Bv}{\Vert \ucap (\Bu \wedge \Bv) \Vert} \), the square of this bivector has the property \({\BI _{\Bu ,\Bv }}^2 = -1 \) of the imaginary unit complex number.

This allows the point to be specified as a complex exponential

\begin{equation}\label{eqn:gaWiki:1700}
\begin{aligned}
= \ucap r ( \cos\theta + \BI _{\Bu ,\Bv } \sin\theta )
= \ucap r \exp( \BI _{\Bu ,\Bv } \theta )
\end{aligned}
\end{equation}

Complex numbers could be expressed in terms of the \(\mathbb R^2\)unit bivector \({\Be}_1 \wedge {\Be}_2\).  However this isomorphism really only requires a pair of linearly independent vectors in a plane (of arbitrary dimension).

\subsection{Quaternions}
\index{quaternion}

Similar to complex numbers the geometric product of two \(\mathbb{R}^3\) vectors can be used to define quaternions.  Pre and Post multiplication with \({\Be}_1{\Be}_2{\Be}_3\) can be used to express a vector in terms of the quaternion unit numbers \(i, j, k\), as well as describe all the properties of those numbers.

\subsection{Cross product as outer product}

%The cross product of traditional vector algebra (on \(\mathbb{R}^3\)) find its place in geometric algebra \(\calG_3\)

Cross product can be written as a scaled outer product

\begin{equation}\label{eqn:gaWiki:1720}
\begin{aligned}
\Ba \times\Bb  = -i(\Ba \wedge\Bb )
\end{aligned}
\end{equation}

\begin{equation}\label{eqn:gaWiki:1740}
\begin{aligned}
i^2 &= ({\Be}_1{\Be}_2{\Be}_3)^2 \\
&= {\Be}_1{\Be}_2{\Be}_3{\Be}_1{\Be}_2{\Be}_3 \\
&= -{\Be}_1{\Be}_2{\Be}_1{\Be}_3{\Be}_2{\Be}_3 \\
&= {\Be}_1{\Be}_1{\Be}_2{\Be}_3{\Be}_2{\Be}_3 \\
&= -{\Be}_3{\Be}_2{\Be}_2{\Be}_3 \\
&= -1
\end{aligned}
\end{equation}

The equivalence of the \(\mathbb{R}^3\) cross product and the wedge product expression above can be confirmed by direct multiplication of \(-i = -{\Be}_1{\Be}_2{\Be}_3\) with a determinant expansion of the wedge product

\begin{equation}\label{eqn:gaWiki:1760}
\begin{aligned}
\Bu \wedge \Bv = \sum_{1<=i<j<=3}(u_i v_j - v_i u_j) {\Be}_i \wedge {\Be}_j
= \sum_{1<=i<j<=3}(u_i v_j - v_i u_j) {\Be}_i {\Be}_j
\end{aligned}
\end{equation}

%%\EndArticle
%\EndNoBibArticle
