%
% Copyright � 2016 Peeter Joot.  All Rights Reserved.
% Licenced as described in the file LICENSE under the root directory of this GIT repository.
%
%{
%\input{../blogpost.tex}
%\renewcommand{\basename}{hestenesElipseParameterization}
%\renewcommand{\dirname}{notes/phy1520/}
%%\newcommand{\dateintitle}{}
%%\newcommand{\keywords}{}
%
%\input{../peeter_prologue_print2.tex}
%
%\usepackage{peeters_layout_exercise}
%\usepackage{peeters_braket}
%\usepackage{peeters_figures}
%\usepackage{siunitx}
%
%\beginArtNoToc
%
%\generatetitle{Elliptic parameterization}
%%\chapter{Elliptic parameterization}
%%\label{chap:hestenesElipseParameterization}
%% \citep{sakurai2014modern} pr X.Y
%% \citep{pozar2009microwave}
%% \citep{qftLectureNotes}
%% \citep{griffiths1999introduction}
%
\makeoproblem{Elliptic parameterization}{problem:hestenesElipseParameterization:1}{\citep{hestenes1999nfc} ch. 3, pr. 8.6}{
Show that an ellipse can be parameterized by
\index{ellipse}
\begin{equation}\label{eqn:hestenesElipseParameterization:20}
\Br(t) = \Bc \cosh( \mu + i t ).
\end{equation}
Here \( i \) is a unit bivector, and \( i \wedge \Bc \) is zero (i.e. \( \Bc \) must be in the plane of the bivector \( i \)).
} % problem
\makeanswer{problem:hestenesElipseParameterization:1}{
Note that \( \mu, t, i \) all commute since \( \mu, t \) are both scalars.
That allows a complex-like expansion of the hyperbolic cosine to be used
\begin{equation}\label{eqn:hestenesElipseParameterization:40}
\begin{aligned}
\cosh( \mu + i t )
&=
\inv{2} \lr{ e^{\mu + i t} + e^{-\mu -i t} } \\
&=
\inv{2} \lr{ e^{\mu} (\cos t + i \sin t) + e^{-\mu} (\cos t -i \sin t) } \\
&=
\cosh \mu \cos t + i \sinh \mu \sin t.
\end{aligned}
\end{equation}

Since an ellipse can be parameterized as
\begin{equation}\label{eqn:hestenesElipseParameterization:60}
\Br(t) = \Ba \cos t + \Bb \sin t,
\end{equation}
where the vector directions \( \Ba \) and \( \Bb \) are perpendicular, the multivector hyperbolic cosine representation parameterizes the ellipse provided
\begin{equation}\label{eqn:hestenesElipseParameterization:80}
\begin{aligned}
\Ba &= \Bc \cosh \mu \\
\Bb &= \Bc i \sinh \mu.
\end{aligned}
\end{equation}
It is desirable to relate the parameters \( \mu, i \) to the vectors \( \Ba, \Bb \).  Because \( \Bc \wedge i = 0 \), the vector \( \Bc \) anticommutes with \( i \), and therefore \( (\Bc i)^2 = -\Bc i i \Bc = \Bc^2 \), which means
\begin{equation}\label{eqn:hestenesElipseParameterization:100}
\begin{aligned}
\Ba^2 &= \Bc^2 \cosh^2 \mu \\
\Bb^2 &= \Bc^2 \sinh^2 \mu,
\end{aligned}
\end{equation}
or
\begin{equation}\label{eqn:hestenesElipseParameterization:120}
\mu = \tanh^{-1} \frac{\Abs{\Bb}}{\Abs{\Ba}}.
\end{equation}

The bivector \( i \) is just the unit bivector for the plane containing \( \Ba \) and \( \Bb \)
\begin{equation}\label{eqn:hestenesElipseParameterization:140}
\begin{aligned}
\Ba \wedge \Bb
&= \cosh \mu \sinh \mu \Bc \wedge (\Bc i) \\
&= \cosh \mu \sinh \mu \gpgradetwo{ \Bc \Bc i } \\
&= \cosh \mu \sinh \mu i \Bc^2 \\
&= \cosh \mu \sinh \mu i \frac{ \Ba^2 }{ \cosh^2 \mu } \\
&= \Ba^2 \tanh \mu i \\
&= \Ba^2 i \frac{\Abs{\Bb}}{\Abs{\Ba}},
\end{aligned}
\end{equation}
so
\begin{equation}\label{eqn:hestenesElipseParameterization:160}
i = \frac{ \Ba \wedge \Bb }{\Abs{\Ba}\Abs{\Bb}}.
\end{equation}

Observe that \( i \) is a unit bivector provided the vectors \( \Ba, \Bb \) are perpendicular, as required
\begin{equation}\label{eqn:hestenesElipseParameterization:180}
(\Ba \wedge \Bb)^2
&=
(\Ba \wedge \Bb) \cdot (\Ba \wedge \Bb) \\
&=
( (\Ba \wedge \Bb) \cdot \Ba ) \cdot \Bb \\
&=
( \Ba (\Bb \cdot \Ba) - \Bb \Ba^2 ) \cdot \Bb \\
&=
(\Ba \cdot \Bb)^2 - \Bb^2 \Ba^2 \\
&=
- \Bb^2 \Ba^2.
\end{equation}
} % answer
%%}
%\EndArticle
%%\EndNoBibArticle
