%
% Copyright � 2012 Peeter Joot.  All Rights Reserved.
% Licenced as described in the file LICENSE under the root directory of this GIT repository.
%

%
%
%\mychapter{Radial components of vector derivatives}
%\label{chap:PJRadialDer}
\index{vector!radial component}
%\date{Oct 22, 2007.  radialVectorDerivatives.tex}

\section{first derivative of a radially expressed vector}

Having calculated the derivative of a unit vector, the total
derivative of a radially expressed vector can be calculated

\begin{equation}\label{eqn:radialVectorDerivatives:20}
\begin{aligned}
(r\rcap)'
   &= r'\rcap  + r\rcap' \\
   &= r'\rcap  + \BrPrimeRej \\
\end{aligned}
\end{equation}

There are two components.  One is in the \(\rcap\) direction (linear component)
and the other perpendicular to that (a rotational component) in the direction of the rejection
of \(\rcap\) from \(\Br'\).

\section{Second derivative of a vector}

Taking second derivatives of a radially expressed vector, we have

\begin{equation}\label{eqn:radialVectorDerivatives:40}
\begin{aligned}
(r\rcap)''
   &= (r'\rcap + r{\rcap}')' \\
   &= r''\rcap + r'\rcap' + (r\rcap')' \\
   &= r''\rcap + (r'/r)\BrPrimeRej + (r\rcap')' \\
\end{aligned}
\end{equation}

Expanding the last term takes a bit more work
\begin{equation}\label{eqn:radialVectorDerivatives:60}
\begin{aligned}
(r\rcap')'
   &= (\BrPrimeRej)' \\
   &=
\rcap'(\rcap \wedge \Br') +
\rcap(\rcap' \wedge \Br') +
\rcap(\rcap \wedge \Br'') \\
   &=
(1/r)(\BrPrimeRej)(\rcap \wedge \Br') +
\rcap(\rcap' \wedge \Br') +
\rcap(\rcap \wedge \Br'') \\
   &=
(1/r)\rcap(\rcap \wedge \Br')^2 +
\rcap(\rcap' \wedge \Br') +
\rcap(\rcap \wedge \Br'') \\
\end{aligned}
\end{equation}

There are three terms to this.  One a scalar (negative) multiple of \(\rcap\), and another, the rejection of \(\rcap\) from \(\Br''\).  The middle term here remains to be expanded.  In particular,

\begin{equation}\label{eqn:radialVectorDerivatives:80}
\begin{aligned}
\rcap' \wedge \Br'
   &= \rcap' \wedge (r\rcap' + r'\rcap) \\
   &= r' \rcap' \wedge \rcap \\
   &= r'/2 (\rcap'\rcap - \rcap\rcap') \\
   &= r'/2r ((\Br' \wedge \rcap)\rcap\rcap - \rcap\rcap(\rcap \wedge \Br')) \\
   &= r'/2r (\Br' \wedge \rcap - \rcap \wedge \Br') \\
   &= -(r'/r) \rcap \wedge \Br' \\
\end{aligned}
\end{equation}

\begin{equation}\label{eqn:radialVectorDerivatives:100}
\begin{aligned}
\implies
(r\rcap')'
   &=
(1/r)\rcap(\rcap \wedge \Br')^2
-(r'/r)\BrPrimeRej
+\rcap(\rcap \wedge \Br'') \\
\end{aligned}
\end{equation}

\begin{equation}\label{eqn:radialVectorDerivatives:120}
\begin{aligned}
\implies
(r\rcap)''
   &= r''\rcap
+(r'/r)\BrPrimeRej
+(1/r)\rcap(\rcap \wedge \Br')^2
-(r'/r)\BrPrimeRej
+\rcap(\rcap \wedge \Br'') \\
   &= r''\rcap
    +(1/r)\rcap(\rcap \wedge \Br')^2
    +\rcap(\rcap \wedge \Br'') \\
   &=
\rcap \left(  r'' +(1/r)(\rcap \wedge \Br')^2\right) +    \rcap(\rcap \wedge \Br'') \\
\end{aligned}
\end{equation}

There are two terms here that are in the \(\rcap\) direction (the bivector square is a negative scalar), and
one rejective term in the direction of the component perpendicular to \(\rcap\) relative to \(\Br''\).

