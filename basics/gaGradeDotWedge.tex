%
% Copyright � 2012 Peeter Joot.  All Rights Reserved.
% Licenced as described in the file LICENSE under the root directory of this GIT repository.
%

%
%
\chapter{An (earlier) attempt to intuitively introduce the dot, wedge, cross, and geometric products}
\index{dot product!introduction}
\index{wedge product!introduction}
\index{cross product!introduction}
\index{geometric product!introduction}
\label{chap:gaGradeDotWedge}

%\date{March 17, 2008.  gaGradeDotWedge.tex}

\section{Motivation}

Both the NFCM and GAFP books have axiomatic introductions of the
generalized (vector, blade) dot and wedge products, but there are
elements of both that I was unsatisfied with.  Perhaps the biggest
issue with both is that they are not presented in a dumb enough fashion.

NFCM presents but
does not prove the generalized dot and wedge product operations
in terms of symmetric and antisymmetric sums, but it is really the
grade operation that is fundamental.  You need that to define the
dot product of two bivectors for example.

GAFP axiomatic presentation is much clearer, but the definition of
generalized wedge product as the totally antisymmetric sum is a bit
strange when all the differential forms book give such a different
definition.

Here I collect some of my notes on how one starts with the geometric
product action on colinear and perpendicular vectors and gets the
familiar results for two and three vector products.  I may not try to
generalize this, but just want to see things presented in a fashion
that makes sense to me.

\section{Introduction}

The aim of this document is to introduce a ``new'' powerful vector multiplication operation, the geometric product,
to a student with some traditional vector algebra background.

The geometric product, also called the Clifford product
\footnote{After William Clifford (1845-1879).}, has remained a relatively obscure mathematical subject.
This operation actually makes a great deal of vector manipulation simpler than possible with the traditional methods, and
provides a way to naturally expresses many geometric concepts.
There is a great deal of information available on the subject, however most of it is targeted for those with a
university graduate school background in physics or mathematics.  That level of mathematical sophistication
should not required to understand the subject.

It is the author's opinion that this could be dumbed down even further, so that it would be palatable for
somebody without any traditional vector algebra background.

\section{What is multiplication?}

The operations of vector addition, subtraction and numeric multiplication have the usual definitions
(addition defined in terms of addition of coordinates, and numeric multiplication as a scaling of the vector retaining its direction).  Multiplication and division of vectors is often described as ``undefined''.  It is possible however, to define a multiplication, or division operation for vectors, in a natural geometric fashion.

What meaning should be given to multiplication or division of vectors?

\subsection{Rules for multiplication of numbers}

Assuming no prior knowledge of how to multiply two vectors (such as the dot, cross, or wedge products to be introduced later) consider instead the rules for multiplication of numbers.

\begin{enumerate}
\item Product of two positive numbers is positive.  Any consideration of countable sets of objects justifies this rule.

\item Product of a positive and negative number is negative.  Example: multiplying a debt (negative number) increases the amount of the debt.

\item Product of a negative and negative number is positive.

\item Multiplication is distributive.  Product of a sum is the sum of the products.
\footnote{The name of this property is not important and no student should ever be tested on it.  It is a word like
dividand which countless countless school kids are forced to memorize.  Like dividand it is perfectly
acceptable to forget it after the test because nobody has to know it to perform division.
Since most useful sorts of multiplications have this property this is the least important
of the named multiplication properties.  This word exists mostly so that authors of math books can impress themselves writing phrases like ``a mathematical entity that behaves this way is
left and right distributive with respect to addition''.
}

\begin{equation}\label{eqn:gaGradeDotWedge:20}
a (b + c) = a b + a c
\end{equation}
\begin{equation}\label{eqn:gaGradeDotWedge:40}
(a + b) c = a c + b c
\end{equation}

\item Multiplication is associative.  Changing the order that multiplication is grouped by does not change the result.

\begin{equation}\label{eqn:gaGradeDotWedge:60}
(a b) c = a (b c)
\end{equation}

\item Multiplication is commutative.  Switching the order of multiplication does not change the result.

\begin{equation}\label{eqn:gaGradeDotWedge:80}
a b = b a
\end{equation}

\end{enumerate}

Unless the reader had an exceptionally gifted grade three teacher it is likely that rule three was presented without any sort of justification or analogy.  This can be considered as a special case of the previous rule.  Geometrically, a multiplication by -1 results in an inversion on the number line.  If one considers the number line to be a line in space, then this is a 180 degree rotation.  Two negative multiplications results in a 360 degree rotation, and thus takes the number back to its original positive or negative segment on its ``number line''.

\subsection{Rules for multiplication of vectors with the same direction}

Having identified the rules for multiplication of numbers, one can use these to define multiplication rules for a simple case, one dimensional vectors.
Conceptually a one dimensional vector space can be thought of like a number line, or the set of all numbers as the set of all scalar multiples of a unit vector of a particular direction in space.

It is reasonable to expect the rules for multiplication of two vectors with the same direction to have some of the same characteristics as multiplication of numbers.  Lets state this algebraically writing the directed distance from the origin to the points \(a\) and \(b\) in a vector notation

\begin{equation}\label{eqn:gaGradeDotWedge:540}
\begin{aligned}
\Ba &= a\Be \\
\Bb &= b\Be \\
\end{aligned}
\end{equation}

where \(\Be\) is the unit vector alone the line in question.

The product of these two vectors is

\begin{equation}\label{eqn:gaGradeDotWedge:100}
\Ba \Bb = a b \Be \Be
\end{equation}

Although no specific meaning has yet been given to the \(\Be \Be\) term yet, one can make a few observations about a product of this form.
\begin{enumerate}
\item It is commutative, since \(\Ba \Bb = \Bb \Ba = a b \Be \Be\).
\item It is distributive since numeric multiplication is.
\item The product of three such vectors is distributive (no matter the grouping of the multiplications there will be a numeric factor and a \(\Be \Be \Be\) factor.
\end{enumerate}

These properties are consistent with half the properties of numeric multiplication.  If the other half of the numeric multiplication rules are assumed to also apply we have

\begin{enumerate}
\item Product of two vectors in the same direction is positive (rules 1 and 3 above).
\item Product of two vectors pointing in opposite directions is negative (rule 2 above).
\end{enumerate}

This can only be possible by giving the following meaning to the square of a unit vector

\begin{equation}\label{eqn:gaGradeDotWedge:120}
\Be \Be = 1
\end{equation}

Alternately, one can state that the square of a vector is that vectors squared length.

\begin{equation}\label{eqn:gaGradeDotWedge:140}
\Ba \Ba = a^2
\end{equation}

This property, as well as the associative and distributive properties are the defining properties of the geometric product.

It will be shown shortly that in order to retain this squared vector length property for vectors with components in different directions it will be required to drop the commutative property of numeric multiplication:

\begin{equation}\label{eqn:gaGradeDotWedge:160}
\Ba \Bb \neq \Bb \Ba
\end{equation}

This is a choice that will later be observed to have important consequences.
There are many types of multiplications that do not have the commutative property.  Matrix multiplication is not even necessarily defined when the order is switched.  Other multiplication operations (wedge and cross products) change sign when the order is switched.

Another important choice has been made to require the product of two vectors not be a vector itself.  This also breaks
from the number line analogy since the product of two numbers is still a number.  However, it is
notable that
in order to take roots of a negative number one has to introduce a second number line
(the \(i\), or imaginary axis), and so even for numbers, products can be ``different'' than their factors.
Interestingly enough,
it will later be possible to show that the choice to not require a vector product to be a vector
allow complex numbers to be defined directly in terms of the geometric product of two vectors in a plane.

\section{Axioms}

The previous discussion attempts to justify the choice of the following set of axioms for multiplication of vectors

\begin{enumerate}
\item{ linearity }
\item{ associativity }
\item{ contraction }

Square of a vector is its squared length.
\end{enumerate}

This last property is weakened in some circumstances (for example,
an alternate definition of vector length is desirable for relativistic calculations.)

%As justification of the contraction property one could
%consider a set of colinear vectors and the real number line to be
%isomorphic.
%
%The product of two positive numbers is a positive number.  Multiplying
%by \(-1\) (the unit negative) produces a rotatation by 180 degrees.  Two negative multiplications
%produces a rotation of 360.  This can be thought of as a justification
%of the grade school ``rule'' that a negative times a negative is positive.
%
%It seems natural to have the rules for vector multiplication reduce to something
%like the rules for numbers when those vectors are restricted to a linear subspace.
%
%In analogy with numbers, the contraction rule gives us such similar properties.  Namely, the
%product of same facing vectors is positive, and the product of opposite facing
%vectors is negative, both scaled by their magnitudes.

\section{dot product}

One can express the dot product in terms of these axioms.  This follows by calculating the
length of a sum or difference of vectors, starting with the requirement that the vector square is the squared length of that vector.

Given two vectors \(\Ba\) and \(\Bb\), their sum
\(\Bc = \Ba + \Bb\) has squared length:

\begin{equation}\label{eqn:introGaFirst:absquared}
\Bc^2 = (\Ba + \Bb)(\Ba + \Bb) = \Ba^2 + \Bb\Ba + \Ba\Bb + \Bb^2.
\end{equation}

We do not have any specific meaning for the product of vectors, but \eqnref{eqn:introGaFirst:absquared}
shows that the symmetric sum of such a product:

\begin{equation}
\Bb\Ba + \Ba\Bb = \text{scalar}
\end{equation}

since the RHS is also a scalar.

Additionally, if \(\Ba\) and \(\Bb\) are perpendicular, then we must also have:

\begin{equation}\label{eqn:gaGradeDotWedge:180}
\Ba^2 + \Bb^2 = a^2 + b^2.
\end{equation}

This implies a rule for vector multiplication of perpendicular vectors

\begin{equation}\label{eqn:gaGradeDotWedge:200}
\Bb\Ba + \Ba\Bb = 0
\end{equation}

Or,

\begin{equation}\label{eqn:introGaFirst:perpabcommutesign}
\Bb\Ba = -\Ba\Bb.
\end{equation}

Note that \eqnref{eqn:introGaFirst:perpabcommutesign} does not assign any meaning to this product of vectors when they perpendicular.
Whatever that meaning is, the entity such a perpendicular vector product produces changes sign
with commutation.

Performing the same length calculation using standard vector algebra shows that we can identify the symmetric
sum of vector products with the dot product:

\begin{equation}\label{eqn:introGaFirst:standarddot}
\norm{\Bc}^2 = (\Ba + \Bb) \cdot (\Ba + \Bb) = \norm{\Ba}^2 + 2 \Ba \cdot \Bb + \norm{\Bb}^2.
\end{equation}

Thus we can make the identity:

\begin{equation}\label{eqn:introGaFirst:dotprod}
\Ba \cdot \Bb = \inv{2}(\Ba \Bb + \Bb \Ba)
\end{equation}

\section{Coordinate expansion of the geometric product}

A powerful feature of geometric algebra is that it allows for coordinate free results, and the avoidance of basis selection
that coordinates require.  While this is true, explicit coordinate expansion, especially
initially while making the transition from coordinate based vector algebra, is believed to add clarity to the subject.

Writing a pair of vectors in coordinate vector notation:

\begin{equation}\label{eqn:gaGradeDotWedge:220}
\Ba = \sum_i{a_i \Be_i}
\end{equation}
\begin{equation}\label{eqn:gaGradeDotWedge:240}
\Bb = \sum_j{b_j \Be_j}
\end{equation}

Despite not yet
knowing what meaning to give to the geometric product of two general (non-colinear) vectors,
given the axioms above and their consequences we actually have enough information to completely
expand the geometric product of two vectors in terms of these coordinates:

\begin{equation}\label{eqn:gaGradeDotWedge:560}
\begin{aligned}
\Ba\Bb
&= \sum_{ij}{a_i b_j \Be_i \Be_j} \\
&= \sum_{i = j} {a_i b_j \Be_i \Be_j}
 + \sum_{i \ne j} {a_i b_j \Be_i \Be_j} \\
&= \sum_{i} {a_i b_i \Be_i \Be_i}
 + \sum_{i < j} a_i b_j \Be_i \Be_j
 + \sum_{j < i} a_i b_j \Be_i \Be_j \\
&= \sum_{i} {a_i b_i}
 + \sum_{i < j} a_i b_j \Be_i \Be_j + a_j b_i \Be_j \Be_i \\
&= \sum_{i} {a_i b_i}
 + \sum_{i < j} (a_i b_j - b_i a_j)\Be_i \Be_j \\
\end{aligned}
\end{equation}

This can be summarized nicely in terms of determinants:

\begin{equation}\label{eqn:introGaFirst:geocoord}
\Ba\Bb = \sum_{i} {a_i b_i} + \sum_{i < j} \DETuvij{a}{b}{i}{j} \Be_i \Be_j
\end{equation}

This shows,
without requiring the ``triangle law'' expansion of \eqnref{eqn:introGaFirst:standarddot},
that the geometric product has a scalar component that we recognize as the Euclidean vector dot product.  It also shows that the remaining bit
is a ``something else'' component.  This ``something else'' is called a bivector.  We do not yet know what this bivector is or what to do with it,
but will come back to that.

Observe that an interchange of \(\Ba\) and \(\Bb\) leaves the scalar part of equation
\eqnref{eqn:introGaFirst:geocoord} unaltered (ie: it is symmetric), whereas an interchange inverts the bivector (ie: it is the antisymmetric part).

\section{Some specific examples to get a feel for things}

Moving from the abstract, consider a few specific geometric product example.

\begin{itemize}
\item Product of two non-colinear non-orthogonal vectors.
\begin{equation}\label{eqn:gaGradeDotWedge:260}
(\Be_1 + 2\Be_2) (\Be_1 - \Be_2)
= \Be_1\Be_1 -2\Be_2\Be_2 + 2\Be_2\Be_1 - \Be_1\Be_2
= -1 + 3\Be_2\Be_1
\end{equation}

Such a product produces both scalar and bivector parts.

\item Squaring a bivector
\begin{equation}\label{eqn:gaGradeDotWedge:280}
(\Be_1\Be_2)^2
=
(\Be_1\Be_2)(-\Be_2\Be_1)
=
-\Be_1(\Be_2\Be_2)\Be_1
=
-\Be_1\Be_1
=
-1
\end{equation}

This particular bivector squares to minus one very much like the imaginary number \(i\).

\item Product of two perpendicular vectors.

\begin{equation}\label{eqn:gaGradeDotWedge:300}
(\Be_1 + \Be_2) (\Be_1 - \Be_2) = 2\Be_2\Be_1
\end{equation}

Such a product generates just a bivector term.

\item Product of a bivector and a vector in the plane.

\begin{equation}\label{eqn:gaGradeDotWedge:320}
(x\Be_1 + y\Be_2) \Be_1\Be_2
=
x\Be_2 - y\Be_1
\end{equation}

This rotates the vector counterclockwise by 90 degrees.

\item General \R{3} geometric product of two vectors.

\begin{equation}\label{eqn:gaGradeDotWedge:340}
\Bx \By =
(x_1\Be_1
+x_2\Be_2
+x_3\Be_3)
(y_1\Be_1
+y_2\Be_2
+y_3\Be_3)
\end{equation}
\begin{equation}\label{eqn:gaGradeDotWedge:360}
=
\Bx \cdot \By
+\DETuvij{x}{y}{2}{3} \Be_2 \Be_3
+\DETuvij{x}{y}{1}{3} \Be_1 \Be_3
+\DETuvij{x}{y}{1}{2} \Be_1 \Be_2
\end{equation}

Or,
\begin{equation}\label{eqn:gaGradeDotWedge:380}
\Bx \By =
\Bx \cdot \By +
\begin{vmatrix}
\Be_2\Be_3 & \Be_3\Be_1 & \Be_1\Be_2 \\
x_1 & x_2 & x_3 \\
y_1 & y_2 & y_3 \\
\end{vmatrix}
\end{equation}

Observe that if one identifies
\(\Be_2\Be_3\), \(\Be_3\Be_1\), and \(\Be_1\Be_2\) with vectors
\(\Be_1\),
\(\Be_2\),
and \(\Be_3\) respectively, this second term is the cross product.  A precise way to perform this identification will be described later.

The key thing to observe here is
that the structure of the cross product is naturally associated with the geometric product.  One can think of the geometric product
as a complete product including elements of both the dot and cross product.  Unlike the cross product the geometric product is also well defined
in two dimensions and greater than three.

\end{itemize}

These examples are all somewhat random, but give a couple hints of results to come.

\section{Antisymmetric part of the geometric product}

Having identified the symmetric sum of vector products with the dot product we can write the geometric product of two arbitrary vectors
in terms of this and its difference

\begin{equation}\label{eqn:gaGradeDotWedge:580}
\begin{aligned}
\Ba \Bb
&= \inv{2}(\Ba \Bb + \Bb \Ba) + \inv{2}(\Ba \Bb - \Bb \Ba) \\
&= \Ba \cdot \Bb + f(\Ba, \Bb) \\
\end{aligned}
\end{equation}

Let us examine this second term, the bivector, a mapping of a pair of vectors into a different sort of object of yet unknown properties.

\begin{equation}\label{eqn:gaGradeDotWedge:400}
f(\Ba, k\Ba) = \inv{2}(\Ba k\Ba - k\Ba \Ba) = 0
\end{equation}

Property: Zero when the vectors are colinear.

\begin{equation}\label{eqn:gaGradeDotWedge:420}
f(\Ba, k\Ba + \Bb) = \inv{2}(\Ba (k\Ba + \Bb) - (k\Ba + m\Bb)\Ba) = f(\Ba, \Bb)
\end{equation}

Property: colinear contributions are rejected.

\begin{equation}\label{eqn:gaGradeDotWedge:440}
f(\alpha \Ba, \beta \Bb) = \inv{2}(\alpha \Ba \beta \Bb - \beta \Bb \alpha \Ba) = \alpha \beta f(\Ba, \Bb)
\end{equation}

Property: bilinearity.

\begin{equation}\label{eqn:gaGradeDotWedge:460}
f(\Bb, \Ba)
= \inv{2}(\Bb \Ba - \Ba\Bb)
= -\inv{2}(\Ba \Bb - \Bb\Ba)
= -f(\Ba, \Bb)
\end{equation}

Property: Interchange inverts.

Operationally, these are in fact the properties of what in the calculus of differential forms is called the wedge product (uncoincidentally, these are also all properties of the cross product as well.)

Because the properties are identical the notation from differential forms is stolen, and we write

\begin{equation}\label{eqn:introGaFirst:wedge}
\Ba \wedge \Bb = \inv{2}(\Ba \Bb - \Bb \Ba)
\end{equation}

And as mentioned, the object that this
wedge product produces from two vectors is called a bivector.

Strictly speaking the
wedge product of differential calculus is defined as an alternating, associative, multilinear form.  We have here bilinear, not multilinear and associativity is
not meaningful until more than two factors are introduced, however when we get to the product of more than three
vectors, we will find that the geometric vector product produces an entity with all of these properties.

Returning to the product of two vectors we can now write

\begin{equation}\label{eqn:introGaFirst:gaproddotwedge}
\Ba \Bb = \Ba \cdot \Bb + \Ba \wedge \Bb
\end{equation}

This is often used as the initial definition of the geometric product.

\section{Yes, but what is that wedge product thing}

Combination of the symmetric and antisymmetric decomposition in \eqnref{eqn:introGaFirst:gaproddotwedge} shows that the product of two vectors according to the axioms
has a scalar part and a bivector part.  What is this bivector part geometrically?

One can show that the equation of a plane can be written in terms of bivectors.  One can also show that
the area of the parallelogram spanned by two vectors can be expressed in terms of the ``magnitude'' of a bivector.  Both of these
show that a bivector characterizes a plane and can be thought of loosely as a ``plane vector''.

Neither the plane equation or the area result are hard to show, but we will get to those later.  A more direct way to get an
intuitive feel for the geometric properties of the bivector can be obtained by first examining the
square of a bivector.

By subtracting the projection of one vector \(\Ba\) from another \(\Bb\), one can form the rejection of \(\Ba\) from \(\Bb\):

\begin{equation}\label{eqn:gaGradeDotWedge:480}
\Bb' = \Bb - (\Bb \cdot \acap)\acap
\end{equation}

With respect to the dot product, this vector is orthogonal to \(\Ba\).  Since \(\Ba \wedge \acap = 0\), this allows us to
write the wedge product of vectors \(\Ba\) and \(\Bb\) as the direct product of two orthogonal vectors

\begin{equation}\label{eqn:gaGradeDotWedge:600}
\begin{aligned}
\Ba \wedge \Bb
&= \Ba \wedge (\Bb - (\Bb \cdot \acap)\acap)) \\
&= \Ba \wedge \Bb' \\
&= \Ba \Bb' \\
&= -\Bb' \Ba \\
\end{aligned}
\end{equation}

The square of the bivector can then be written

\begin{equation}\label{eqn:gaGradeDotWedge:620}
\begin{aligned}
(\Ba \wedge \Bb)^2
&= (\Ba \Bb')(-\Bb'\Ba) \\
&= -\Ba^2 (\Bb')^2.
\end{aligned}
\end{equation}

Thus the square of a bivector is negative.  It is natural to define a
bivector norm:

\begin{equation}\label{eqn:gaGradeDotWedge:500}
\abs{\Ba \wedge \Bb} = \sqrt{-(\Ba \wedge \Bb)^2} = \sqrt{ (\Ba \wedge \Bb)(\Bb \wedge \Ba) }
\end{equation}

Dividing by this norm we have an entity that acts precisely like the imaginary number \(i\).

Looking back to \eqnref{eqn:introGaFirst:gaproddotwedge} one can now assign additional meaning to the two parts.  The first, the dot product, is a scalar (ie: a real number), and a second part, the wedge product, is a pure imaginary term.  Provided \(\Ba \wedge \Bb \ne 0\), we can write \(i = \frac{\Ba \wedge \Bb}{ \abs{\Ba \wedge \Bb} }\) and express
the geometric product in complex number form:

\begin{equation}\label{eqn:gaGradeDotWedge:520}
\Ba \Bb = \Ba \cdot \Bb + i \abs{\Ba \wedge \Bb}
\end{equation}

The complex number system
is the algebra of the plane, and the geometric product of two vectors can be used to completely characterize the algebra of an arbitrarily oriented plane in a higher
order vector space.

It actually will be very natural to define complex numbers in terms of the geometric product, and we will see later that
the geometric product allows for the ad-hoc definition of ``complex number'' systems according to convenience in many ways.

We will also see that generalizations of complex numbers such as quaternion algebras also find their natural place as specific instances of geometric products.

Concepts familiar from
complex numbers such as conjugation, inversion, exponentials as rotations, and even things like the residue theory of complex contour integration, will
all have a natural geometric algebra analogue.

We will return to this, but first some more detailed initial examination of the wedge product properties is in order, as is a look at the product of greater than
two vectors.
