%
% Copyright � 2012 Peeter Joot.  All Rights Reserved.
% Licenced as described in the file LICENSE under the root directory of this GIT repository.
%

%
%
%\chapter{Cauchy Equations expressed as a gradient}
\index{Cauchy equations}
\index{gradient}
\label{chap:cauchyGradient}
%\date{August 13, 2008.  cauchyGradient.tex}

The complex number derivative, when it exists, is defined as:
%
\begin{equation*}
\frac{\delta f}{\delta z} = \frac{ f(z + \delta z) - f(z)}{\delta z}
\end{equation*}
\begin{equation*}
f'(z) = {\text{limit}}_{\abs{\delta z} \rightarrow 0} \quad \frac{\delta f}{\delta z}
\end{equation*}
%
Like any two variable function, this limit requires that all limiting paths produce the same result, thus it is
minimally necessary that the limits for the particular cases of \(\delta z = \delta x + i \delta y\) exist for both
\(\delta x = 0\), and \(\delta y = 0\) independently.  Of course there are other possible ways for \(\delta z \rightarrow 0\), such as spiraling inwards paths.  Apparently it can be shown that if the specific cases are satisfied, then this limit exists for any path (I am not sure how to show that, nor will try, at least now).

Examining each of these cases separately, we have for \(\delta x = 0\), and \(f(z) = u(x,y) + i v(x,y)\):
%
\begin{equation}\label{eqn:cauchyGradient:20}
\begin{aligned}
\frac{\delta f}{\delta z}
&= \frac{u(x,y + \delta y) + i v(x,y + \delta y)}{i\delta y} \\
&\rightarrow -i \frac{\partial u(x,y)}{\partial y} + \frac{\partial v(x,y)}{\partial y} \\
\end{aligned}
\end{equation}
%
and for \(\delta y = 0\)
\begin{equation}\label{eqn:cauchyGradient:40}
\begin{aligned}
\frac{\delta f}{\delta z}
&= \frac{u(x + \delta x,y) + i v(x + \delta x, y)}{\delta x} \\
&\rightarrow \frac{\partial u(x,y)}{\partial x} + i\frac{\partial v(x,y)}{\partial x} \\
\end{aligned}
\end{equation}
%
If these are equal regardless of the path, then equating real and imaginary parts of these respective equations we have:
%
\begin{equation}\label{eqn:cauchyGradient:60}
\begin{aligned}
\frac{\partial v}{\partial x} + \frac{\partial u}{\partial y} &= 0 \\
\frac{\partial u}{\partial x} - \frac{\partial v}{\partial y} &= 0
\end{aligned}
\end{equation}
%
Now, these are strikingly similar to the gradient, and we make this similarly explicit using the planar
pseudoscalar
\(i=\Be_1 \wedge \Be_2 = \Be_1 \Be_2\)
as the unit imaginary.  For the first equation, pre multiplying by \(1 = \Be_{11}\), and post multiplying by \(\Be_2\) we have:
%
\begin{equation*}
\Be_1 \frac{\partial \Be_{12} v}{\partial x} + \Be_{2}\frac{\partial u}{\partial y} = 0,
\end{equation*}
%
and for the second, pre multiply by \(\Be_1\), and post multiply the \(\partial_y\) term by \(1 = \Be_{22}\), and rearrange:
\begin{equation*}
\Be_1 \frac{\partial u}{\partial x} + \Be_{2} \frac{\partial \Be_{12} v}{\partial y} = 0.
\end{equation*}
%
Adding these we have:
\begin{equation*}
\Be_1 \frac{\partial u + \Be_{12}}{\partial x} + \Be_{2} \frac{\partial u + \Be_{12} v}{\partial y} = 0.
\end{equation*}
%
Since \(f = u + i v\), this is just
%
\begin{equation}
\Be_1 \frac{\partial f}{\partial x} + \Be_{2} \frac{\partial f}{\partial y} = 0.
\end{equation}
%
Or,
\begin{equation}\label{eqn:cauchy_gradient:gradf}
\grad f = 0
\end{equation}
%
By taking second partial derivatives and equating mixed partials we are used to seeing these Cauchy-Riemann equations
take this form as second order equations:
%
\begin{equation}\label{eqn:cauchy_gradient:uxx}
\grad^2 u = u_{xx} + u_{yy} = 0
\end{equation}
\begin{equation}
\grad^2 v = v_{xx} + v_{yy} = 0
\end{equation}
%
Given this, \eqnref{eqn:cauchy_gradient:gradf} is something that we could have perhaps guessed, since the square root of the Laplacian operator, is in fact the gradient (there are an infinite number of such square roots, since any rotation of the coordinate system that expresses the gradient also works).  However, a guess of this is not required since we see this explicitly through some logical composition of relationships.

The end result is that we can make a statement that
in regions where the complex function is analytic (has a derivative), the gradient of that function is zero in that region.

This is a kind of interesting result and I expect that this will relevant when figuring out how the geometric calculus
all fits together.

\section{Verify we still have the Cauchy equations hiding in the gradient}

We have:
%
\begin{equation*}
\grad f \Be_1 = \grad ( \Be_1 u - \Be_2 v) = 0
\end{equation*}
%
If this is to be zero, both the scalar and bivector parts of this equation must also be zero.
%
\begin{equation}\label{eqn:cauchyGradient:80}
\begin{aligned}
(\grad \cdot f) \Be_1
&= \grad \cdot ( \Be_1 u - \Be_2 v) \\
&= (\Be_1 \partial_x + \Be_2 \partial_y) \cdot ( \Be_1 u - \Be_2 v) \\
&= (\partial_x u - \partial_y v) = 0
\end{aligned}
\end{equation}
%
\begin{equation}\label{eqn:cauchyGradient:100}
\begin{aligned}
(\grad \wedge f) \Be_1
&= \grad \wedge ( \Be_1 u - \Be_2 v) \\
&= (\Be_1 \partial_x + \Be_2 \partial_y) \wedge ( \Be_1 u - \Be_2 v) \\
&= -\Be_1 \wedge \Be_2 (\partial_x v + \partial_y u) = 0
\end{aligned}
\end{equation}
%
We therefore see that this recovers the expected pair of Cauchy equations:
%
\begin{equation}\label{eqn:cauchyGradient:120}
\begin{aligned}
\partial_x u - \partial_y v &= 0 \\
\partial_x v + \partial_y u &= 0
\end{aligned}
\end{equation}
