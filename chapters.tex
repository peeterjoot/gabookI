%
% Copyright � 2013 Peeter Joot.  All Rights Reserved.
% Licenced as described in the file LICENSE under the root directory of this GIT repository.
%
\part{Basics and Geometry}
%   % have lots of intros here.  Sort them all out.
%   \input{basics/introGa.tex}
%   %
% Copyright � 2012 Peeter Joot.  All Rights Reserved.
% Licenced as described in the file LICENSE under the root directory of this GIT repository.
%

%
%
%\input{../peeter_prologue_print.tex}
%\input{../peeter_prologue_widescreen.tex}

\chapter{Geometric Algebra.  The very quickest introduction}
\index{geometric product!introduction}
\label{chap:gaQuickIntro}

%\blogpage{http://sites.google.com/site/peeterjoot2/math2012/gaQuickIntro.pdf}
%\date{Mar 16, 2012}
%\gitRevisionInfo{gaQuickIntro}

\beginArtWithToc
%\beginArtNoToc

\section{Motivation}
An attempt to make a relatively concise introduction to Geometric (or Clifford) Algebra.  Much more complete introductions to the subject can be found in \citep{dorst2007gac}, \citep{doran2003gap}, and \citep{hestenes1999nfc}.
\section{Axioms}
We have a couple basic principles upon which the algebra is based
\begin{itemize}
\item Vectors can be multiplied.
\item The square of a vector is the (squared) length of that vector (with appropriate generalizations for non-Euclidean metrics).
\item Vector products are associative (but not necessarily commutative).
\end{itemize}

That is really all there is to it, and the rest, paraphrasing Feynman, can be figured out by anybody sufficiently clever.

\section{By example.  The 2D case}
Consider a 2D Euclidean space, and the product of two vectors \(\Ba\) and \(\Bb\) in that space.  Utilizing a standard orthonormal basis \(\{\Be_1, \Be_2\}\) we can write
\begin{equation}\label{eqn:gaQuickIntro:10}
\begin{aligned}
\Ba &= \Be_1 x_1 + \Be_2 x_2 \\
\Bb &= \Be_1 y_1 + \Be_2 y_2,
\end{aligned}
\end{equation}
and let us write out the product of these two vectors \(\Ba \Bb\), not yet knowing what we will end up with.  That is
\begin{equation}\label{eqn:gaQuickIntro:230}
\begin{aligned}
\Ba \Bb
&= (\Be_1 x_1 + \Be_2 x_2 )( \Be_1 y_1 + \Be_2 y_2 ) \\
&= \Be_1^2 x_1 y_1 + \Be_2^2 x_2 y_2
+ \Be_1 \Be_2 x_1 y_2 + \Be_2 \Be_1 x_2 y_1.
\end{aligned}
\end{equation}

From axiom 2 we have \(\Be_1^2 = \Be_2^2 = 1\), so we have
\begin{equation}\label{eqn:gaQuickIntro:30}
\Ba \Bb = x_1 y_1 + x_2 y_2 + \Be_1 \Be_2 x_1 y_2 + \Be_2 \Be_1 x_2 y_1.
\end{equation}

We have multiplied two vectors and ended up with a scalar component (and recognize that this part of the vector product is the dot product), and a component that is a ``something else''.  We will call this something else a bivector, and see that it is characterized by a product of non-colinear vectors.  These products \(\Be_1 \Be_2\) and \(\Be_2 \Be_1\) are in fact related, and we can see that by looking at the case of \(\Bb = \Ba\).  For that we have
\begin{equation}\label{eqn:gaQuickIntro:250}
\begin{aligned}
\Ba^2
&=
x_1 x_1 + x_2 x_2 + \Be_1 \Be_2 x_1 x_2 + \Be_2 \Be_1 x_2 x_1 \\
&=
\Abs{\Ba}^2 +
x_1 x_2 ( \Be_1 \Be_2 + \Be_2 \Be_1 )
\end{aligned}
\end{equation}

Since axiom (2) requires our vectors square to equal its (squared) length, we must then have
\begin{equation}\label{eqn:gaQuickIntro:50}
\Be_1 \Be_2 + \Be_2 \Be_1 = 0,
\end{equation}
or
\begin{equation}\label{eqn:gaQuickIntro:70}
\Be_2 \Be_1 = -\Be_1 \Be_2.
\end{equation}

We see that Euclidean orthonormal vectors anticommute.  What we can see with some additional study is that any colinear vectors commute, and in Euclidean spaces (of any dimension) vectors that are normal to each other anticommute (this can also be taken as a definition of normal).

We can now return to our product of two vectors \eqnref{eqn:gaQuickIntro:30} and simplify it slightly
\begin{equation}\label{eqn:gaQuickIntro:30b}
\Ba \Bb = x_1 y_1 + x_2 y_2 + \Be_1 \Be_2 (x_1 y_2 - x_2 y_1).
\end{equation}

The product of two vectors in 2D is seen here to have one scalar component, and one bivector component (an irreducible product of two normal vectors).  Observe the symmetric and antisymmetric split of the scalar and bivector components above.  This symmetry and antisymmetry can be made explicit, introducing dot and wedge product notation respectively
\begin{equation}\label{eqn:gaQuickIntro:90}
\begin{aligned}
\Ba \cdot \Bb &= \inv{2}( \Ba \Bb + \Bb \Ba) = x_1 y_1 + x_2 y_2 \\
\Ba \wedge \Bb &= \inv{2}( \Ba \Bb - \Bb \Ba) = \Be_1 \Be_2 (x_1 y_y - x_2 y_1).
\end{aligned}
\end{equation}
so that the vector product can be written as
\begin{equation}\label{eqn:gaQuickIntro:110}
\Ba \Bb = \Ba \cdot \Bb + \Ba \wedge \Bb.
\end{equation}
\section{Pseudoscalar}
In many contexts it is useful to introduce an ordered product of all the unit vectors for the space is called the pseudoscalar.  In our 2D case this is
\begin{equation}\label{eqn:gaQuickIntro:130}
i = \Be_1 \Be_2,
\end{equation}
a quantity that we find behaves like the complex imaginary.  That can be shown by considering its square
\begin{equation}\label{eqn:gaQuickIntro:270}
\begin{aligned}
(\Be_1 \Be_2)^2
&=
(\Be_1 \Be_2)
(\Be_1 \Be_2) \\
&=
\Be_1 (\Be_2 \Be_1) \Be_2 \\
&=
-\Be_1 (\Be_1 \Be_2) \Be_2 \\
&=
-(\Be_1 \Be_1) (\Be_2 \Be_2) \\
&=
-1^2 \\
&= -1
\end{aligned}
\end{equation}

Here the anticommutation of normal vectors property has been used, as well as (for the first time) the associative multiplication axiom.

In a 3D context, you will see the pseudoscalar in many places (expressing the normals to planes for example).  It also shows up in a number of fundamental relationships.  For example, if one writes
\begin{equation}\label{eqn:gaQuickIntro:150}
I = \Be_1 \Be_2 \Be_3
\end{equation}
for the 3D pseudoscalar, then it is also possible to show
\begin{equation}\label{eqn:gaQuickIntro:170}
\Ba \Bb = \Ba \cdot \Bb + I (\Ba \cross \Bb)
\end{equation}
something that will be familiar to the student of QM, where we see this in the context of Pauli matrices.  The Pauli matrices also encode a Clifford algebraic structure, but we do not need an explicit matrix representation to do so.
\section{Rotations}
%
% Copyright © 2012 Peeter Joot.  All Rights Reserved.
% Licenced as described in the file LICENSE under the root directory of this GIT repository.
%

Very much like complex numbers we can utilize exponentials to perform rotations.  Rotating in a sense from \(\Be_1\) to \(\Be_2\), can be expressed as

\begin{equation}\label{eqn:gaQuickIntro:290}
\begin{aligned}
\Ba e^{i \theta}
&=
(\Be_1 x_1 + \Be_2 x_2) (\cos\theta + \Be_1 \Be_2 \sin\theta) \\
&=
\Be_1 (x_1 \cos\theta - x_2 \sin\theta)
+
\Be_2 (x_2 \cos\theta + x_1 \sin\theta)
\end{aligned}
\end{equation}

More generally, even in N dimensional Euclidean spaces, if \(\Ba\) is a vector in a plane, and \(\ucap\) and \(\vcap\) are perpendicular unit vectors in that plane, then the rotation through angle \(\theta\) is given by

\begin{equation}\label{eqn:gaQuickIntro:190}
\Ba \rightarrow \Ba e^{\ucap \vcap \theta}.
\end{equation}

This is illustrated in \cref{fig:gaQuickIntro:gaQuickIntroFig1}

\pdfTexFigure{../figures/gabook/gaQuickIntroFig1.pdf_tex}{Plane rotation}{fig:gaQuickIntro:gaQuickIntroFig1}{0.6}

Notice that we have expressed the rotation here without utilizing a normal direction for the plane.  The sense of the rotation is encoded by the bivector \(\ucap \vcap\) that describes the plane and the orientation of the rotation (or by duality the direction of the normal in a 3D space).  By avoiding a requirement to encode the rotation using a normal to the plane we have an method of expressing the rotation that works not only in 3D spaces, but also in 2D and greater than 3D spaces, something that is not possible when we restrict ourselves to traditional vector algebra (where quantities like the cross product can not be defined in a 2D or 4D space, despite the fact that things they may represent, like torque are planar phenomena that do not have any intrinsic requirement for a normal that falls out of the plane.).

When \(\Ba\) does not lie in the plane spanned by the vectors \(\ucap\) and \(\vcap\) , as in \cref{fig:gaQuickIntro:gaQuickIntroFig2}, we must express the rotations differently.  A rotation then takes the form

\begin{equation}\label{eqn:gaQuickIntro:210}
\Ba \rightarrow
e^{-\ucap \vcap \theta/2}
\Ba
e^{\ucap \vcap \theta/2}.
\end{equation}


\pdfTexFigure{../figures/gabook/gaQuickIntroFig2.pdf_tex}{3D rotation}{fig:gaQuickIntro:gaQuickIntroFig2}{0.6}

In the 2D case, and when the vector lies in the plane this reduces to the one sided complex exponential operator used above.  We see these types of paired half angle rotations in QM, and they are also used extensively in computer graphics under the guise of quaternions.

%\EndArticle

%   \input{basics/gaGradeDotWedge.tex}

   \chapter{Basics}
      \section{Did you ever ask your teacher how to multiply vectors?}
         \input{../GAelectrodynamics/GAmotivation.tex}
         \subsection{Problems}
            \input{../GAelectrodynamics/ComplexInnerProductVsDotAndCrossProduct.tex}
      \section{Vector multiplication}
         \input{../GAelectrodynamics/multiplication.tex}
         \subsection{Problems}
            \input{../GAelectrodynamics/2dvectorsquare.tex}
            \input{../GAelectrodynamics/normalAnticommutation.tex}
      \section{Definitions}
         \input{../GAelectrodynamics/definitions.tex}
         \subsection{Problems}
            \input{../GAelectrodynamics/R3PseudoscalarSquare.tex}
      \section{Grade selection, dot and wedge product operators}
         \input{../GAelectrodynamics/gradeselection.tex}
         \subsection{Problems}
            \input{../GAelectrodynamics/RnDotProduct.tex}
            \input{../GAelectrodynamics/cyclicpermutationtwo.tex}
            \input{../GAelectrodynamics/dotprodSymmetricSum.tex}
            \input{../GAelectrodynamics/planeRotationsExponentials.tex}
            \input{../GAelectrodynamics/complexNumbers.tex}
            \input{../GAelectrodynamics/R3PseudoscalarCommutation.tex}
            \input{../GAelectrodynamics/gradeselVectorWedge.tex}
            \input{../GAelectrodynamics/WedgeRelationshipToCrossProduct.tex}
            \input{../GAelectrodynamics/vectorBivectorDot.tex}
            \input{../GAelectrodynamics/vectorTrivectorDot.tex}
            \input{../GAelectrodynamics/bivectorDot.tex}
            \input{../GAelectrodynamics/r4nonzerobivectorwedgewithself.tex}
            \input{../GAelectrodynamics/scalarPermutation.tex}
      \section{Product of two vectors}
         \input{../GAelectrodynamics/vectorproduct.tex}
         \subsection{Miscellanious theorems}
            \input{../stokesTheorem/bladeDotWedgeSymmetryIdentitiesTheorem.tex}
         %   %
% Copyright © 2016 Peeter Joot.  All Rights Reserved.
% Licenced as described in the file LICENSE under the root directory of this GIT repository.
%
\maketheorem{Distribution of inner products}{thm:stokesTheoremGeometricAlgebra:1420}{
Given two blades \(A_s, B_r\) with grades subject to \(s > r > 0\), and a vector \(b\), the inner product distributes according to
\begin{equation*}
A_s \cdot \lr{ b \wedge B_r } = \lr{ A_s \cdot b } \cdot B_r.
\end{equation*}
}

         \subsection{Problems}
            \input{../GAelectrodynamics/vectorproductCyclicPermutation.tex}
            \input{../GAelectrodynamics/wedgeantisym.tex}
            \input{../GAelectrodynamics/gradethreeselectionWedge.tex}
            \input{../stokesTheorem/bladeDotWedgeSymmetryIdentities.tex}
            %
% Copyright © 2016 Peeter Joot.  All Rights Reserved.
% Licenced as described in the file LICENSE under the root directory of this GIT repository.
%
\makeproblem{Wedge distribution identity.}{problem:wedgeDistributionIdentityProblems:1}{
Prove \cref{thm:stokesTheoremGeometricAlgebra:1420}.
} % problem

%This will allow us, for example, to expand a general inner product of the form \(d^k \Bx \cdot (\boldpartial \wedge F)\).
\makeanswer{problem:wedgeDistributionIdentityProblems:1}{
The proof is straightforward, but also mechanical.  Start by expanding the wedge and dot products within a grade selection operator
\begin{equation}\label{eqn:stokesTheoremGeometricAlgebra:1460}
\begin{aligned}
A_s \cdot \lr{ \Bb \wedge B_r }
&=
\gpgrade{A_s (\Bb \wedge B_r)}{s - (r + 1)} \\
&=
\inv{2} \gpgrade{A_s \lr{\Bb B_r + (-1)^{r} B_r \Bb} }{s - (r + 1)}.
\end{aligned}
\end{equation}

Solving for \(B_r \Bb\) in
\begin{equation}\label{eqn:stokesTheoremGeometricAlgebra:1480}
2 \Bb \cdot B_r = \Bb B_r - (-1)^{r} B_r \Bb,
\end{equation}
we have
\begin{equation}\label{eqn:stokesTheoremGeometricAlgebra:1500}
\begin{aligned}
A_s \cdot \lr{ \Bb \wedge B_r }
&=
\inv{2} \gpgrade{ A_s \Bb B_r + A_s \lr{ \Bb B_r - 2 \Bb \cdot B_r } }{s - (r + 1)} \\
&=
\gpgrade{ A_s \Bb B_r }{s - (r + 1)}
-
\cancel{\gpgrade{ A_s \lr{ \Bb \cdot B_r } }{s - (r + 1)}}.
\end{aligned}
\end{equation}

The last term above is zero since we are selecting the \(s - r - 1\) grade element of a multivector with grades \(s - r + 1\) and \(s + r - 1\), which has no terms for \(r > 0\).  Now we can expand the \(A_s \Bb\) multivector product, for

\begin{dmath}\label{eqn:stokesTheoremGeometricAlgebra:1520}
A_s \cdot \lr{ \Bb \wedge B_r }
=
\gpgrade{ \lr{ A_s \cdot \Bb + A_s \wedge \Bb} B_r }{s - (r + 1)}.
\end{dmath}

The latter multivector (with the wedge product factor) above has grades \(s + 1 - r\) and \(s + 1 + r\), so this selection operator finds nothing.  This leaves

\begin{dmath}\label{eqn:stokesTheoremGeometricAlgebra:1540}
A_s \cdot \lr{ \Bb \wedge B_r }
=
\gpgrade{
\lr{ A_s \cdot \Bb } \cdot B_r
+ \lr{ A_s \cdot \Bb } \wedge B_r
}{s - (r + 1)}.
\end{dmath}

The first dot products term has grade \(s - 1 - r\) and is selected, whereas the wedge term has grade \(s - 1 + r \ne s - r - 1\) (for \(r > 0\)).  \(\qedmarker\)

%Next consider an expansion that we cannot do above, but require
} % answer

%      \section{Problem solutions}
%         \shipoutAnswer

   \chapter{Comparison of many traditional vector and GA identities}
      \input{basics/gaWiki.tex}
   \input{basics/gaWikiCramers.tex}
   \input{basics/gaWikiTorque.tex}
   \input{basics/gaWikiUnitDerivative.tex}
   \input{basics/radialVectorDerivatives.tex}
   \input{basics/angularVelocity.tex}
   \input{basics/bivector.tex}
   \input{basics/trivector.tex}
   \input{basics/scalarCommutes.tex}
   \input{basics/bladegradereduction.tex}
   \input{basics/plane.tex}
   \input{basics/quaternion.tex}
   \input{basics/cauchyGradient.tex}
   \input{basics/legendre.tex}
   \input{basics/levi.tex}
   \input{basics/nfcmCh2.tex}
   \input{basics/outermorphismDet.tex}
   %
% Copyright � 2016 Peeter Joot.  All Rights Reserved.
% Licenced as described in the file LICENSE under the root directory of this GIT repository.
%
%{
%\input{../blogpost.tex}
%\renewcommand{\basename}{hestenesElipseParameterization}
%\renewcommand{\dirname}{notes/phy1520/}
%%\newcommand{\dateintitle}{}
%%\newcommand{\keywords}{}
%
%\input{../peeter_prologue_print2.tex}
%
%\usepackage{peeters_layout_exercise}
%\usepackage{peeters_braket}
%\usepackage{peeters_figures}
%\usepackage{siunitx}
%
%\beginArtNoToc
%
%\generatetitle{Elliptic parameterization}
%%\chapter{Elliptic parameterization}
%%\label{chap:hestenesElipseParameterization}
%% \citep{sakurai2014modern} pr X.Y
%% \citep{pozar2009microwave}
%% \citep{qftLectureNotes}
%% \citep{griffiths1999introduction}
%
\makeoproblem{Elliptic parameterization}{problem:hestenesElipseParameterization:1}{\citep{hestenes1999nfc} ch. 3, pr. 8.6}{
Show that an ellipse can be parameterized by
\index{ellipse}
\begin{equation}\label{eqn:hestenesElipseParameterization:20}
\Br(t) = \Bc \cosh( \mu + i t ).
\end{equation}
Here \( i \) is a unit bivector, and \( i \wedge \Bc \) is zero (i.e. \( \Bc \) must be in the plane of the bivector \( i \)).
} % problem
\makeanswer{problem:hestenesElipseParameterization:1}{
Note that \( \mu, t, i \) all commute since \( \mu, t \) are both scalars.
That allows a complex-like expansion of the hyperbolic cosine to be used
\begin{equation}\label{eqn:hestenesElipseParameterization:40}
\begin{aligned}
\cosh( \mu + i t )
&=
\inv{2} \lr{ e^{\mu + i t} + e^{-\mu -i t} } \\
&=
\inv{2} \lr{ e^{\mu} (\cos t + i \sin t) + e^{-\mu} (\cos t -i \sin t) } \\
&=
\cosh \mu \cos t + i \sinh \mu \sin t.
\end{aligned}
\end{equation}

Since an ellipse can be parameterized as
\begin{equation}\label{eqn:hestenesElipseParameterization:60}
\Br(t) = \Ba \cos t + \Bb \sin t,
\end{equation}
where the vector directions \( \Ba \) and \( \Bb \) are perpendicular, the multivector hyperbolic cosine representation parameterizes the ellipse provided
\begin{equation}\label{eqn:hestenesElipseParameterization:80}
\begin{aligned}
\Ba &= \Bc \cosh \mu \\
\Bb &= \Bc i \sinh \mu.
\end{aligned}
\end{equation}
It is desirable to relate the parameters \( \mu, i \) to the vectors \( \Ba, \Bb \).  Because \( \Bc \wedge i = 0 \), the vector \( \Bc \) anticommutes with \( i \), and therefore \( (\Bc i)^2 = -\Bc i i \Bc = \Bc^2 \), which means
\begin{equation}\label{eqn:hestenesElipseParameterization:100}
\begin{aligned}
\Ba^2 &= \Bc^2 \cosh^2 \mu \\
\Bb^2 &= \Bc^2 \sinh^2 \mu,
\end{aligned}
\end{equation}
or
\begin{equation}\label{eqn:hestenesElipseParameterization:120}
\mu = \tanh^{-1} \frac{\Abs{\Bb}}{\Abs{\Ba}}.
\end{equation}

The bivector \( i \) is just the unit bivector for the plane containing \( \Ba \) and \( \Bb \)
\begin{equation}\label{eqn:hestenesElipseParameterization:140}
\begin{aligned}
\Ba \wedge \Bb
&= \cosh \mu \sinh \mu \Bc \wedge (\Bc i) \\
&= \cosh \mu \sinh \mu \gpgradetwo{ \Bc \Bc i } \\
&= \cosh \mu \sinh \mu i \Bc^2 \\
&= \cosh \mu \sinh \mu i \frac{ \Ba^2 }{ \cosh^2 \mu } \\
&= \Ba^2 \tanh \mu i \\
&= \Ba^2 i \frac{\Abs{\Bb}}{\Abs{\Ba}},
\end{aligned}
\end{equation}
so
\begin{equation}\label{eqn:hestenesElipseParameterization:160}
i = \frac{ \Ba \wedge \Bb }{\Abs{\Ba}\Abs{\Bb}}.
\end{equation}

Observe that \( i \) is a unit bivector provided the vectors \( \Ba, \Bb \) are perpendicular, as required
\begin{equation}\label{eqn:hestenesElipseParameterization:180}
(\Ba \wedge \Bb)^2
&=
(\Ba \wedge \Bb) \cdot (\Ba \wedge \Bb) \\
&=
( (\Ba \wedge \Bb) \cdot \Ba ) \cdot \Bb \\
&=
( \Ba (\Bb \cdot \Ba) - \Bb \Ba^2 ) \cdot \Bb \\
&=
(\Ba \cdot \Bb)^2 - \Bb^2 \Ba^2 \\
&=
- \Bb^2 \Ba^2.
\end{equation}
} % answer
%%}
%\EndArticle
%%\EndNoBibArticle


\part{Projection}
   \input{projection/reciprocalFrame.tex}
   \input{projection/projectionWithMatrixComparison.tex}
   \input{projection/obliqueProj.tex}
   \input{projection/projectionAndMoorePenroseVectorInverse.tex}
   \input{projection/angleBetweenLineAndPlane.tex}
   \input{projection/orthodecomp.tex}
   \input{projection/matrixOfLinearTx.tex}
   \input{projection/juliaVector.tex}

\part{Rotation}
   \input{rotation/rotor.tex}
   \input{rotation/eulerangle.tex}
   \input{rotation/sphericalPolar.tex}
   \input{rotation/slerp.tex}
   \input{rotation/kvectorExponential.tex}
   \input{rotation/rotationGenerator.tex}
   \input{rotation/sphericalPolarUnit.tex}
   \input{rotation/infinitesimalRotation.tex}

\part{Calculus}
   \input{calculus/multivectorTaylors.tex}
   \input{calculus/gradientAndForms.tex}
   \input{calculus/nvolume.tex}
   \input{calculus/vectorDifferentialIdentities.tex}
   \input{calculus/polarGradAndLaplacian.tex}
   \input{calculus/sphericalPolarLaplacian.tex}
   \input{calculus/tangentAndNormalVectors.tex}
   \chapter{Stokes' theorem}
      \input{calculus/stokesTheoremGeometricAlgebra.tex}
      %\section{Problems} % already a problems section above
         \input{calculus/stokesCorollariesGriffiths.tex}
   \chapter{Fundamental Theorem of Geometric Calculus}
      \section{Fundamental Theorem of Geometric Calculus}
         \input{calculus/fundamentalTheoremOfCalculus.tex}
      \section{Green's function for the gradient in Euclidean spaces.}
         %
% Copyright � 2016 Peeter Joot.  All Rights Reserved.
% Licenced as described in the file LICENSE under the root directory of this GIT repository.
%
%{
%\input{../blogpost.tex}
%\renewcommand{\basename}{gradientGreensFunction}
%\renewcommand{\dirname}{notes/ece1228-electromagnetic-theory/}
%%\newcommand{\dateintitle}{}
%%\newcommand{\keywords}{}
%
%\input{../peeter_prologue_print2.tex}
%
%\usepackage{peeters_layout_exercise}
%\usepackage{peeters_braket}
%\usepackage{peeters_figures}
%\usepackage{siunitx}
%
%\beginArtNoToc
%
%\generatetitle{Green's function for the gradient in Euclidean spaces.}
%\label{chap:gradientGreensFunction}
In \citep{doran2003gap} it is stated that the Green's function for the gradient is
\begin{equation}\label{eqn:gradientGreensFunction:20}
   G(x, x') = \inv{S_n} \frac{x - x'}{\Abs{x-x'}^n},
\end{equation}
where \( n \) is the dimension of the space, \( S_n \) is the area of the unit sphere, and
\begin{equation}\label{eqn:gradientGreensFunction:40}
   \grad G = \grad \cdot G = \delta(x - x').
\end{equation}
%
What I'd like to do here is verify that this Green's function operates as asserted.
Here, as in some parts of the text, I am following a convention where vectors are written without boldface.

Let's start with checking that the gradient of the Green's function is zero everywhere that \( x \ne x' \)
\begin{equation}\label{eqn:gradientGreensFunction:100}
\begin{aligned}
\spacegrad \inv{\Abs{x - x'}^n}
&=
-\frac{n}{2} \frac{e^\nu \partial_\nu (x_\mu - x_\mu')(x^\mu - {x^\mu}')}{\Abs{x - x'}^{n+2}} \\
&=
-\frac{n}{2} 2 \frac{e^\nu (x_\mu - x_\mu') \delta_\nu^\mu }{\Abs{x - x'}^{n+2}} \\
&=
-n \frac{ x - x'}{\Abs{x - x'}^{n+2}}.
\end{aligned}
\end{equation}

This means that we have, everywhere that \( x \ne x' \)
\begin{equation}\label{eqn:gradientGreensFunction:120}
\begin{aligned}
\spacegrad \cdot G
&=
\inv{S_n} \lr{ \frac{\spacegrad \cdot \lr{x - x'}}{\Abs{x - x'}^{n}} + \lr{ \spacegrad \inv{\Abs{x - x'}^{n}} } \cdot \lr{ x - x'} } \\
&=
\inv{S_n} \lr{ \frac{n}{\Abs{x - x'}^{n}} + \lr{ -n \frac{x - x'}{\Abs{x - x'}^{n+2} } \cdot \lr{ x - x'} } } \\
&= 0.
\end{aligned}
\end{equation}

Next, consider the curl of the Green's function.
Zero curl will mean that we have \( \grad G = \grad \cdot G = G \lgrad \).
\begin{equation}\label{eqn:gradientGreensFunction:140}
\begin{aligned}
S_n (\grad \wedge G)
&=
\frac{\grad \wedge (x-x')}{\Abs{x - x'}^{n}}
+
\grad \inv{\Abs{x - x'}^{n}} \wedge (x-x') \\
&=
\frac{\grad \wedge (x-x')}{\Abs{x - x'}^{n}}
- \cancel{n
\frac{x - x'}{\Abs{x - x'}^{n}} \wedge (x-x')}.
\end{aligned}
\end{equation}
However,
\begin{equation}\label{eqn:gradientGreensFunction:160}
\begin{aligned}
\grad \wedge (x-x')
&= \grad \wedge x \\
&= e^\mu \wedge e_\nu \partial_\mu x^\nu \\
&= e^\mu \wedge e_\nu \delta_\mu^\nu \\
&= e^\mu \wedge e_\mu.
\end{aligned}
\end{equation}

For any metric where \( e_\mu \propto e^\mu \), which is the case in all the ones with physical interest (i.e. \R{3} and Minkowski space), \( \grad \wedge G \) is zero.

Having shown that the gradient of the (presumed) Green's function is zero everywhere that \( x \ne x' \), the guts of the
demonstration can now proceed.  We wish to evaluate the gradient weighted convolution of the Green's function using the Fundamental Theorem of (Geometric) Calculus.  Here the gradient acts bidirectionally on both the gradient and the test function.  Working in primed coordinates so that the final result is in terms of the unprimed, we have
\begin{equation}\label{eqn:gradientGreensFunction:60}
   \int_V G(x,x') d^n x' \lrgrad' F(x')
   = \int_{\partial V} G(x,x') d^{n-1} x' F(x').
\end{equation}

Let \( d^n x' = dV' I \), \( d^{n-1} x' n = dA' I \), where \( n = n(x') \) is the outward normal to the area element \( d^{n-1} x' \).
From this point on, lets restrict attention to Euclidean spaces, where \( n^2 = 1 \).  In that case
\begin{equation}\label{eqn:gradientGreensFunction:80}
\begin{aligned}
\int_V dV' G(x,x') \lrgrad' F(x')
&= \int_V dV' \lr{G(x,x') \lgrad'} F(x') + \int_V dV' G(x,x') \lr{ \rgrad' F(x') } \\
&= \int_{\partial V} dA' G(x,x') n F(x').
\end{aligned}
\end{equation}

Here, the pseudoscalar \( I \) has been factored out by commuting it with \( G \), using \( G I = (-1)^{n-1} I G \), and then pre-multiplication with \( 1/((-1)^{n-1} I ) \).

Each of these integrals can be considered in sequence.  A convergence bound is required of the multivector test function \( F(x') \) on the infinite surface \( \partial V \).  Since it's true that
\begin{equation}\label{eqn:gradientGreensFunction:180}
\Abs{ \int_{\partial V} dA' G(x,x') n F(x') }
\ge
\int_{\partial V} dA' \Abs{ G(x,x') n F(x') },
\end{equation}
then it is sufficient to require that
\begin{equation}\label{eqn:gradientGreensFunction:200}
\lim_{x' \rightarrow \infty} \Abs{ \frac{x -x'}{\Abs{x - x'}^n} n(x') F(x') } \rightarrow 0,
\end{equation}
in order to kill off the surface integral.  Evaluating the integral on a hypersphere centered on \( x \) where \( x' - x = n \Abs{x - x'} \), that is
\begin{equation}\label{eqn:gradientGreensFunction:260}
\lim_{x' \rightarrow \infty} \frac{ \Abs{F(x')}}{\Abs{x - x'}^{n-1}} \rightarrow 0.
\end{equation}
Given such a constraint, that leaves
\begin{equation}\label{eqn:gradientGreensFunction:220}
\int_V dV' \lr{G(x,x') \lgrad'} F(x')
=
-\int_V dV' G(x,x') \lr{ \rgrad' F(x') }.
\end{equation}
The LHS is zero everywhere that \( x \ne x' \) so it can be restricted to a spherical ball around \( x \), which allows the test function \( F \) to be pulled out of the integral, and a second application of the Fundamental Theorem to be applied.
\begin{equation}\label{eqn:gradientGreensFunction:240}
\begin{aligned}
\int_V dV' \lr{G(x,x') \lgrad'} F(x')
&=
\lim_{\epsilon \rightarrow 0}
\int_{\Abs{x - x'} < \epsilon} dV' \lr{G(x,x') \lgrad'} F(x') \\
&=
\lr{ \lim_{\epsilon \rightarrow 0}
I^{-1} \int_{\Abs{x - x'} < \epsilon} I dV' \lr{G(x,x') \lgrad'}
} F(x) \\
&=
\lr{ \lim_{\epsilon \rightarrow 0}
(-1)^{n-1} I^{-1} \int_{\Abs{x - x'} < \epsilon} G(x,x') d^n x' \lgrad'
} F(x) \\
&=
\lr{ \lim_{\epsilon \rightarrow 0}
(-1)^{n-1} I^{-1} \int_{\Abs{x - x'} = \epsilon} G(x,x') d^{n-1} x'
} F(x) \\
&=
\lr{ \lim_{\epsilon \rightarrow 0}
(-1)^{n-1} I^{-1} \int_{\Abs{x - x'} = \epsilon} G(x,x') dA' I n
} F(x) \\
&=
\lr{ \lim_{\epsilon \rightarrow 0}
\int_{\Abs{x - x'} = \epsilon} dA' G(x,x') n
} F(x) \\
&=
\lr{ \lim_{\epsilon \rightarrow 0}
\int_{\Abs{x - x'} = \epsilon} dA' \frac{\epsilon (-n)}{S_n \epsilon^n} n
} F(x) \\
&=
-\lim_{\epsilon \rightarrow 0}
\frac{F(x)}{S_n \epsilon^{n-1}}
\int_{\Abs{x - x'} = \epsilon} dA' \\
&=
-\lim_{\epsilon \rightarrow 0}
\frac{F(x)}{S_n \epsilon^{n-1}}
S_n \epsilon^{n-1} \\
&=
-F(x).
\end{aligned}
\end{equation}

This essentially calculates the divergence integral around an infinitesimal hypersphere, without assuming that the gradient commutes with the gradient in this infinitesimal region.  So, provided the test function is constrained by \cref{eqn:gradientGreensFunction:260}, we have
\begin{equation}\label{eqn:gradientGreensFunction:280}
F(x) = \int_V dV' G(x,x') \lr{ \grad' F(x') }.
\end{equation}
In particular, should we have a first order gradient equation
\begin{equation}\label{eqn:gradientGreensFunction:300}
\spacegrad' F(x') = M(x'),
\end{equation}
the inverse of this equation is given by
%\begin{equation}\label{eqn:gradientGreensFunction:320}
\boxedEquation{eqn:gradientGreensFunction:320}{
F(x) = \int_V dV' G(x,x') M(x').
}
%\end{equation}

Note that the sign of the Green's function is explicitly tied to the definition of the convolution integral that is used.
%There is a slightly annoying negative sign in this convolution integral.
This is important since since the conventions for the sign of the Green's function or the parameters in the convolution integral often vary.
%For this definition, it seems desirable to eliminate that negation by variable substitution
%
%\begin{equation}\label{eqn:gradientGreensFunction:340}
%F(y)
%= -\int_V dV' G(y,x') M(x')
%= -\int_V dV G(y,x) M(x),
%\end{equation}
%
%so the inverse of \( \grad F = M \) can be written as
%
%%\begin{equation}\label{eqn:gradientGreensFunction:360}
%\boxedEquation{eqn:gradientGreensFunction:360}{
%F(y)
%= \int_V dV G(x, y) M(x).
%}
%\end{equation}

What's cool about this result is that it applies not only to gradient equations in Euclidean spaces, but also to multivector (or even just vector) fields \( F \), instead of the usual scalar functions that we usually apply Green's functions to.
%}
%\EndArticle

      % example:
         %
% Copyright � 2016 Peeter Joot.  All Rights Reserved.
% Licenced as described in the file LICENSE under the root directory of this GIT repository.
%
%{
%\input{../blogpost.tex}
%\renewcommand{\basename}{biotSavartGreens}
%%\renewcommand{\dirname}{notes/phy1520/}
%\renewcommand{\dirname}{notes/ece1228-electromagnetic-theory/}
%%\newcommand{\dateintitle}{}
%%\newcommand{\keywords}{}
%
%\input{../peeter_prologue_print2.tex}
%
%\usepackage{peeters_layout_exercise}
%\usepackage{peeters_braket}
%\usepackage{peeters_figures}
%\usepackage{siunitx}
%
%\beginArtNoToc
%
%\generatetitle{Green's function inversion of magnetostatic equation}
%\chapter{Green's function inversion of magnetostatic equation}
%\label{chap:biotSavartGreens}
% \citep{sakurai2014modern} pr X.Y
% \citep{pozar2009microwave}
% \citep{qftLectureNotes}
% \citep{doran2003gap}
% \citep{jackson1975cew}
% \citep{griffiths1999introduction}
\makeexample{Magnetostatics.}{example:biotSavartGreens:1}{
The magnetostatics equation in linear media has the Geometric Algebra form
%A previous example of inverting a gradient equation was the electrostatics equation.  We can do the same for the magnetostatics equation, which has the following Geometric Algebra form in linear media
\begin{equation}\label{eqn:biotSavartGreens:20}
\spacegrad I \BB = - \mu \BJ.
\end{equation}

The Green's inversion of this is
\begin{equation}\label{eqn:biotSavartGreens:40}
\begin{aligned}
I \BB(\Bx)
&= \int_V dV' G(\Bx, \Bx') \spacegrad' I \BB(\Bx') \\
&= \gpgradeone{ \int_V dV' G(\Bx, \Bx') \spacegrad' I \BB(\Bx') } \\
&= \int_V dV' \gpgradeone{ G(\Bx, \Bx') (-\mu \BJ(\Bx')) } \\
&= \inv{4\pi} \int_V dV' \frac{\Bx - \Bx'}{ \Abs{\Bx - \Bx'}^3 } \wedge (-\mu \BJ(\Bx')) \\
&= \frac{\mu}{4\pi} \int_V dV' \BJ(\Bx') \wedge \frac{\Bx - \Bx'}{ \Abs{\Bx - \Bx'}^3 }.
\end{aligned}
\end{equation}
A duality transformation can be used to obtain the usual cross product form of the Biot-Savart law if desired.

Note that freedom to insert a no-op vector grade selection was utilized to simplify the calculation above.
It can be demonstrated that the scalar component of this integral is explicitly zero with some of the usual trickery
\begin{equation}\label{eqn:biotSavartGreens:60}
\begin{aligned}
-\frac{\mu}{4\pi} \int_V dV' \frac{\Bx - \Bx'}{ \Abs{\Bx - \Bx'}^3 } \cdot \BJ(\Bx')
&= \frac{\mu}{4\pi} \int_V dV' \lr{ \spacegrad \inv{ \Abs{\Bx - \Bx'} }} \cdot \BJ(\Bx') \\
&= -\frac{\mu}{4\pi} \int_V dV' \lr{ \spacegrad' \inv{ \Abs{\Bx - \Bx'} }} \cdot \BJ(\Bx') \\
&= -\frac{\mu}{4\pi} \int_V dV' \lr{
\spacegrad' \cdot \frac{\BJ(\Bx')}{ \Abs{\Bx - \Bx'} }
-
\frac{\spacegrad' \cdot \BJ(\Bx')}{ \Abs{\Bx - \Bx'} }
}.
\end{aligned}
\end{equation}

By premultiplying \cref{eqn:biotSavartGreens:20} by the gradient, we have
\begin{equation}\label{eqn:biotSavartGreens:80}
\spacegrad^2 I \BB = -\mu \spacegrad \BJ,
\end{equation}
showing that the current \( \BJ \) is not unconstrained.  In particular, since
the LHS is a bivector, the gradient of the current must also be a bivector
\( \spacegrad \BJ = \spacegrad \wedge \BJ \),
or equivalently the divergence of the current must be zero 
\( \spacegrad \cdot \BJ = 0 \).  This kills the \( \spacegrad' \cdot \BJ(\Bx') \) integrand numerator in \cref{eqn:biotSavartGreens:60}, leaving
\begin{equation}\label{eqn:biotSavartGreens:100}
\begin{aligned}
-\frac{\mu}{4\pi} \int_V dV' \frac{\Bx - \Bx'}{ \Abs{\Bx - \Bx'}^3 } \cdot \BJ(\Bx')
&= -\frac{\mu}{4\pi} \int_V dV' \spacegrad' \cdot \frac{\BJ(\Bx')}{ \Abs{\Bx - \Bx'} } \\
&= -\frac{\mu}{4\pi} \int_{\partial V} dA' \ncap \cdot \frac{\BJ(\Bx')}{ \Abs{\Bx - \Bx'} }.
\end{aligned}
\end{equation}

Provided the normal component of \( \BJ/\Abs{\Bx - \Bx'} \) vanishes on the boundary of the infinite sphere, we see that the 
the scalar selection of the convolution integral is zero, justifying the vector selection operation.
%Observe that the traditional vector form of the Biot-Savart law can be obtained by premultiplying both sides with \( -I \), leaving
%
%\begin{equation}\label{eqn:biotSavartGreens:140}
%\BB(\Bx)
%= \frac{\mu}{4\pi} \int_V dV' \BJ(\Bx') \cross \frac{\Bx - \Bx'}{ \Abs{\Bx - \Bx'}^3 }.
%\end{equation}
%
%This checks against a trusted source such as \citep{griffiths1999introduction} (eq. 5.39).
} % example
%}
%\EndArticle

      \section{Problems}
         % convert to example?
         %
% Copyright � 2016 Peeter Joot.  All Rights Reserved.
% Licenced as described in the file LICENSE under the root directory of this GIT repository.
%
%{
%\input{../blogpost.tex}
%\renewcommand{\basename}{helmholtzDerviationMultivector}
%%\renewcommand{\dirname}{notes/phy1520/}
%\renewcommand{\dirname}{notes/ece1228-electromagnetic-theory/}
%%\newcommand{\dateintitle}{}
%%\newcommand{\keywords}{}
%
%\input{../peeter_prologue_print2.tex}
%
%\usepackage{peeters_layout_exercise}
%\usepackage{peeters_braket}
%\usepackage{peeters_figures}
%\usepackage{siunitx}
%
%\beginArtNoToc
%
%\generatetitle{Helmholtz theorem}
%\chapter{Helmholtz theorem}
%\label{chap:helmholtzDerviationMultivector}
% \citep{sakurai2014modern} pr X.Y
% \citep{pozar2009microwave}
% \citep{qftLectureNotes}
% \citep{doran2003gap}
% \citep{jackson1975cew}
% \citep{griffiths1999introduction}

%\section{Appendix IV.  2nd Geometric Algebra solution to problem 5.}

%This is a problem from ece1228.  I attempted solutions in a number of ways.  One using Geometric Algebra, one devoid of that algebra, and then this method, which combined aspects of both.  Of the three methods I tried to obtain this result, this is the most compact and elegant.  It does however, require a fair bit of Geometric Algebra knowledge, including the Fundamental Theorem of Geometric Calculus, as detailed in \citep{doran2003gap}, \citep{sobczyk2011fundamental} and \citep{aMacdonaldVAGC}.

\makeproblem{Helmholtz theorem}{problem:helmholtzTakeIII}{
Prove the first Helmholtz's theorem.
\input{calculus/helmholtzDerviationMultivectorStatement.tex}
%Note: Assume there is a vector \( \BN \) with its divergence and curl equal to \( s \) and \( \BC \) respectively, then show that \( \BM = \BN \) .
} % makeproblem

\makeanswer{problem:helmholtzTakeIII}{
%%I attempted this problem in two different ways.  My first approach assembled the divergence and curl relations above into a single (Geometric Algebra) multivector gradient equation and applied the vector valued Green's function for the gradient to invert that equation.  That approach logically led from the differential equation for \( \BM \) to the solution for \( \BM \) in terms of \( s \) and \( \BC \).  However, this strategy introduced some complexities
%%that make me doubt the correctness of the associated boundary analysis.
%%
%%Even if the details of the boundary handling in my multivector approach is not correct, I thought that approach was interesting enough to share, and have placed it in an appendix to this problem set.  It is accompanied with a primer on Geometric Algebra that is hopefully enough to allow the reader to grasp the basic ideas of the approach, but is probably not sufficient to understand all the details without further study.
%%
%%The answer obtained in that first attempt at this problem, when \( \Abs{\BM}/r^2 \), and \( \Abs{\BC}/r \) both vanish on an infinite sphere, is that the field has a unique value determined by \( s \) and \( \BC \), namely
%
%\begin{dmath}\label{eqn:helmholtzDerviationMultivector:60}
%\BM =
%-\spacegrad \int_V dV' \frac{ s(\Bx')}{ 4 \pi \Abs{\Bx - \Bx'} }
%+\spacegrad \cross \int_V dV' \frac{ \BC(\Bx') }{ 4 \pi \Abs{\Bx - \Bx'} }.
%\end{dmath}
%
%It's possible to work backwards from this result to obtain second order gradient terms applied to \( \BM(\Bx')/\Abs{\Bx - \Bx'} \) .  This suggests that a Laplacian (i.e. scalar) representation of the delta function may be a superior way to tackle this problem, perhaps also yielding a simpler result for the boundary term.  This is in fact the case, and the logical starting point is a convolution representation of the vector function \( \BM \)
%

\input{calculus/helmholtzDerviationMultivectorSolution.tex}

%we recover the non-boundary integral of \cref{eqn:helmholtzDerviationMultivector:200}.  The boundary term is seen to have a particularly simple form using this technique.  Note that the dot and double cross product expression obtained with the vector algebra approach can be recovered from this directly if desired using an expansion of the following form

%Using this expansion in \cref{eqn:helmholtzDerviationMultivector:720} recovers \cref{eqn:helmholtzDerviationMultivector:200}.
%Of the three methods I tried to obtain this result, this is the most compact and elegant of all three solution attempts.
%, but also requires full knowledge of the Geometric Algebra toolbox to understand.
}

%\EndArticle

%\paragraph{Future thought.}
%
%Does this Green's function also work for mixed metric spaces?  If so, in such a metric, what does it mean to
%calculate the surface area of a unit sphere in a mixed signature space?

\part{General Physics}
   \input{physics/angularAcc.tex}
   \input{physics/angularAccCross.tex}
   \input{physics/keRotation.tex}
   \input{physics/radial.tex}
   \input{physics/tensor.tex}
   \input{physics/inertialTensor.tex}
   \input{physics/locateSatellite.tex}
   \input{physics/laplace.tex}
   \input{physics/complex.tex}
   \input{physics/intersectionNewton.tex}
   \input{physics/torusCenterOfMass.tex}
%\input{whatsTheDifference.tex}
