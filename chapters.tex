%
% Copyright � 2013 Peeter Joot.  All Rights Reserved.
% Licenced as described in the file LICENSE under the root directory of this GIT repository.
%
\part{Basics and Geometry}
%   % have lots of intros here.  Sort them all out.
%   %
% Copyright � 2012 Peeter Joot.  All Rights Reserved.
% Licenced as described in the file LICENSE under the root directory of this GIT repository.
%

%
%
\mychapter{Introductory concepts}
\label{chap:introGa}

%%%%%\section{My search for Geometric Algebra.}
%%%%%
%%%%%When you learned vector algebra initially, you learned how to add vectors and scale them, and you probably asked your teacher
%%%%%
%%%%%``How do we multiply vectors?''
%%%%%
%%%%%My teacher's response was something like:
%%%%%
%%%%%``It is impossible to multiply vectors, but some multiplication like operations can be defined.''
%%%%%
%%%%%This may have been followed with a lesson on the dot and cross product operators, or at least a mention that this topic would be covered later.
%%%%%
%%%%%You'll also learn how to generalize vectors from two and three dimensions to higher dimensions, and may learn to generalize vectors from real valued to complex valued.  The dot product generalizes nicely to higher dimensions and even complex valued vectors.  However, given the usefulness of the cross product, you probably find your self asking ``How does the cross product generalize?''
%%%%%
%%%%%The cross product is an explicitly three dimensional beast, and isn't even well defined in two.  You have to introduce a 3D normal direction to describe quanities like torque that are perfectly well defined in a plane.  When I was confronted with this oddity, my conclusion was that there must be a way to generalize the cross product to two dimensions or to greater than 3 dimenions.  A search for that generalization eventually led me to discover Geometric Algebra (with a stop over at differential forms on the way).
%%%%%
%%%%%Geometric Algebra answers the ``How do I multiply vectors?'' question and supplies the generalization of the cross product, among many other things.
%%%%%It also provides a mathematical toolbox that incorporates and extends many not-obviously related fields within mathematics.
%%%%%
%%%%%%Geometric Algebra is not the only answer to some of these questions.  The student of differential forms will know that the cross product can be found generalized using the wedge product of one-forms.  When I first found that, it was not obvious how to apply that generalization to many problems of geometry.  Vector objects have to be promoted to differential forms to apply the wedge product, even if a differential version of such an object does not make any sense.
%%%%%
%%%%%%Eventually, I blundered through an attempt of my own to generalize the cross product, and found a way that worked well for the generalized ``cross product'' of \( n -1 \) n-dimensional vectors.  Such a product worked nicely to define a normal to the \( n -1 \) dimensional subspace that was spanned by that set of vectors.  I was left wondering how to apply this generalization in other obvious contexts, such as a generalization of Stokes' Theorem, which is expressed in terms of the cross product in \R{3}.  I wasn't smart enough at the time to just go looking for existing generalizations of Stokes' theorem.  A friend, much smarter than I, did that search for me, and pointed me towards the field of differential forms which had a wedge product that generalized the cross product.
%%%%%
%%%%%%, and perhaps apply linear operators (i.e. matrices) to them.  One of your
%%%%%%one of your questions to the instructor was probably

Here is an attempt to provide a naturally sequenced introduction to Geometric Algebra.

\section{The Axioms}

Two basic axioms are required, contraction and associative multiplication respectively

\begin{equation}\label{eqn:introGa:40}
\begin{aligned}
a^2 &= \text{scalar} \\
a (b c) &= (a b) c = a b c
\end{aligned}
\end{equation}

Linearity and scalar multiplication should probably also be included for completeness, but even with those this is a surprisingly small set of rules.  The choice to impose these as the rules for vector multiplication will be seen to have a rich set of consequences once explored.  It will take a fair amount of work to extract all the consequences of this decision, and some of that will be done here.

\section{Contraction and the metric}

Defining \(a^2\) itself requires introduction of a metric, the specification of the multiplication rules for a particular basis for the vector space.  For Euclidean spaces, a requirement that

\begin{equation}\label{eqn:introGa:60}
\begin{aligned}
a^2 = \Abs{a}^2
\end{aligned}
\end{equation}

is sufficient to implicitly define this metric.  However, for the Minkowski spaces of special relativity one wants the squares of time and spatial basis vectors to be opposing in sign.  Deferring the discussion of metric temporarily one can work with the axioms above to discover their implications, and in particular how these relate to the coordinate vector space constructions that are so familiar.

\section{Symmetric sum of vector products}

Squaring a vector sum provides the first interesting feature of the general vector product

\begin{equation}\label{sumSquared}
\begin{aligned}
(a + b)^2 %&= (a + b)(a + b) \\
&= a^2 + b^2 + a b + b a
\end{aligned}
\end{equation}

Observe that the LHS is a scalar by the contraction identity, and on the RHS we have scalars \(a^2\) and \(b^2\) by the same.  This implies that the symmetric sum of products

\begin{equation}\label{eqn:introGa:80}
\begin{aligned}
a b + b a
\end{aligned}
\end{equation}

is also a scalar, independent of any choice of metric.  Symmetric sums of this form have a place in physics over the space of operators, often instantiated in matrix form.  There one writes this as the commutator and denotes it as

\begin{equation}\label{eqn:intro_ga:anticommutator}
\begin{aligned}
\symmetric{a}{b} \equiv a b + b a
\end{aligned}
\end{equation}

In an Euclidean space one can observe that equation \ref{sumSquared} has the same structure as the law of cosines so it should not be surprising that this symmetric sum is also related to the dot product.  For a Euclidean space where one the notion of perpendicularity can be expressed as

\begin{equation}\label{eqn:introGa:100}
\begin{aligned}
\Abs{ a + b }^2 = \Abs{a}^2 + \Abs{b}^2
\end{aligned}
\end{equation}

we can then see that an implication of the vector product is the fact that perpendicular vectors have the property

\begin{equation}\label{eqn:introGa:120}
\begin{aligned}
a b + ba = 0
\end{aligned}
\end{equation}

or

\begin{equation}\label{eqn:introGa:140}
\begin{aligned}
b a = - a b
\end{aligned}
\end{equation}

This notion of perpendicularity will also be seen to make sense for non-Euclidean spaces.

Although it retracts from a purist Geometric Algebra approach where things can be done in a coordinate free fashion, the connection between the symmetric product and the standard vector dot product can be most easily shown by considering an expansion with respect to an orthonormal basis.

Lets write two vectors in an orthonormal basis as

\begin{equation}\label{eqn:introGa:160}
\begin{aligned}
a &= \sum_\mu a^\mu e_\mu \\
b &= \sum_\mu b^\mu e_\mu
\end{aligned}
\end{equation}

Here the choice to utilize raised indices rather than lower for the coordinates is taken from physics where summation is typically implied when upper and lower indices are matched as above.

Forming the symmetric product we have

\begin{equation}\label{eqn:introGa:180}
\begin{aligned}
a b + b a
&=
\sum_{\mu,\nu} a^\mu e_\mu b^\nu e_\nu + b^\mu e_\mu a^\nu e_\nu \\
&=
\sum_{\mu,\nu} a^\mu b^\nu \left( e_\mu e_\nu + e_\nu e_\mu \right) \\
&=
2 \sum_{\mu} a^\mu b^\mu {e_\mu}^2 + \sum_{\mu \ne \nu} a^\mu b^\nu \left( e_\mu e_\nu + e_\nu e_\mu \right) \\
\end{aligned}
\end{equation}

For an Euclidean space we have \({e_\mu}^2 = 1\), and \(e_\nu e_\mu = -e_\mu e_\nu\), so we are left with

\begin{equation}\label{eqn:introGa:200}
\begin{aligned}
\sum_{\mu} a^\mu b^\mu = \inv{2} ( a b + b a)
\end{aligned}
\end{equation}

This shows that we can make an identification between the symmetric product, and the anticommutator of physics with the dot product, and then define

\begin{equation}\label{eqn:intro_ga:dotDefined}
\begin{aligned}
a \cdot b \equiv \inv{2} \symmetric{a}{b} = \inv{2} (a b + ba)
\end{aligned}
\end{equation}

\section{Antisymmetric product of two vectors (wedge product)}

Having identified or defined the symmetric product with the dot product we are now prepared to examine a general product of two vectors.  Employing a symmetric + antisymmetric decomposition we can write such a general product as

\begin{equation}\label{eqn:introGa:220}
\begin{aligned}
a b = \mathLabelBox{\inv{2}(a b + b a)}{\(a \cdot b\)} + \mathLabelBox{ \inv{2} ( a b - b a ) }{\(a\) something \(b\)}
\end{aligned}
\end{equation}

What is this remaining vector operation between the two vectors

\begin{equation}\label{eqn:introGa:240}
\begin{aligned}
a \something b = \inv{2} ( a b - b a )
\end{aligned}
\end{equation}

One can continue the comparison with the quantum mechanics, and like the
anticommutator operator that expressed our symmetric sum in equation
\eqnref{eqn:intro_ga:anticommutator} one can introduce a commutator operator

\begin{equation}\label{eqn:intro_ga:commutator}
\begin{aligned}
\antisymmetric{a}{b} \equiv a b - b a
\end{aligned}
\end{equation}

The commutator however, does not naturally extend to more than two vectors, so
as with the scalar part of the vector product (the dot product part),
it is desirable to make a different identification for this part of the vector
product.

One observation that we can make is that this vector operation changes sign when the operations are reversed.  We have

\begin{equation}\label{eqn:introGa:260}
\begin{aligned}
b \something a = \inv{2} ( b a - a b) = - a \something b
\end{aligned}
\end{equation}

Similarly, if \(a\) and \(b\) are colinear, say \(b = \alpha a\), this product is zero

\begin{equation}\label{eqn:introGa:280}
\begin{aligned}
a \something (\alpha a)
&= \inv{2} ( a  (\alpha a) - (\alpha a) a ) \\
&= 0
\end{aligned}
\end{equation}

This complete antisymmetry, aside from a potential difference in sign, are precisely the properties of the wedge product used in the mathematics of differential forms.  In this differential geometry the wedge product of \(m\) one-forms (vectors in this context) can be defined as

\begin{equation}\label{eqn:intro_ga:wedge}
\begin{aligned}
a_1 \wedge a_2 \cdots \wedge a_m
&= \inv{m!} \sum a_{i_1} a_{i_2} \cdots a_{i_m} \sgn(\pi(i_1 i_2 \cdots i_m))
\end{aligned}
\end{equation}

Here \(\sgn(\pi(\cdots))\) is the sign of the permutation of the indices.  While we have not gotten yet to products of more than two vectors it is helpful to know that the wedge product will have a place in such a general product.   An equation like \eqnref{eqn:intro_ga:wedge} makes a lot more sense after writing it out in full for a few specific cases.  For two vectors \(a_1\) and \(a_2\) this is

\begin{equation}\label{eqn:intro_ga:wedgeTwo}
\begin{aligned}
a_1 \wedge a_2 = \inv{2}
\left( a_1 a_2 (1) + a_2 a_1 (-1) \right)
\end{aligned}
\end{equation}

and for three vectors this is

\begin{equation}\label{eqn:introGa:300}
\begin{aligned}
a_1 \wedge a_2 \wedge a_3 = \inv{6}
(
&a_1 a_2 a_3 (1) + a_1 a_3 a_2 (-1) \\
+&a_2 a_1 a_3 (-1) + a_3 a_1 a_2 (1) \\
+&a_2 a_3 a_1 (1) + a_3 a_2 a_1 (-1) )
\end{aligned}
\end{equation}

We will see later that this completely antisymmetrized sum, the wedge product of differential forms will have an important place in this algebra, but like the dot product it is a specific construction of the more general vector product.  The choice to identify the antisymmetric sum with the wedge product is an action that amounts to a definition of the wedge product.  Explicitly, and complementing
the dot product definition of \eqnref{eqn:intro_ga:dotDefined} for the dot product
of two vectors, we say

\begin{equation}\label{eqn:intro_ga:wedgeDefined}
\begin{aligned}
a \wedge b \equiv \inv{2} \antisymmetric{a}{b} = \inv{2} ( a b - b a )
\end{aligned}
\end{equation}

Having made this definition, the symmetric and antisymmetric decomposition of two vectors leaves us with a peculiar looking hybrid construction:

\begin{equation}\label{eqn:intro_ga:dotPlusWedge}
\begin{aligned}
a b %&= \inv{2} (a b + b a) + \inv{2} ( a b - b a ) \\
&= a \cdot b + a \wedge b
\end{aligned}
\end{equation}

We had already seen that part of this vector product was not a vector, but was in fact a scalar.  We now see that the remainder is also not a vector but is instead something that resides in a different space.  In differential geometry this object is called a two form, or a simple element in \(\bigwedge^2\).  Various labels are available for this object are available in Geometric (or Clifford) algebra, one of which is a 2-blade.  2-vector or bivector is also used in some circumstances, but in dimensions greater than three there are reasons to reserve these labels for a slightly more general construction.

The definition of \eqnref{eqn:intro_ga:dotPlusWedge} is often used as the starting point in Geometric Algebra introductions.  While there is value to this approach I have personally found that the non-axiomatic approach becomes confusing if one attempts to sort out which of the many identities in the algebra are the fundamental ones.  That is why my preference is to treat this as a consequence rather than the starting point.

\section{Expansion of the wedge product of two vectors}

Many introductory geometric algebra treatments try very hard to avoid explicit coordinate treatment.  It is true that GA provides infrastructure for coordinate free treatment, however, this avoidance perhaps contributes to making the subject less accessible.  Since we are so used to coordinate geometry in vector and tensor algebra, let us take advantage of this comfort, and express the wedge product explicitly in coordinate form to help get some comfort for it.

Employing the definition of \eqnref{eqn:intro_ga:wedgeDefined}, and an orthonormal basis expansion in coordinates for two vectors \(a\), and \(b\), we have

\begin{equation}\label{eqn:introGa:320}
\begin{aligned}
2 (a \wedge b)
&= ( a b - b a ) \\
&=
\sum_{\mu,\nu} a^\mu b^\nu e_\mu e_\nu
-\sum_{\alpha,\beta} a^\alpha b^\beta e_\alpha e_\beta \\
&=
\mathLabelBox{\sum_{\mu} a^\mu b^\mu - \sum_{\alpha} a^\alpha b^\alpha }{\(=0\)}
+ \sum_{\mu \ne \nu} a^\mu b^\nu e_\mu e_\nu
- \sum_{\alpha \ne \beta} a^\alpha b^\beta e_\alpha e_\beta \\
&=
\sum_{\mu < \nu} (a^\mu b^\nu e_\mu e_\nu + a^\nu b^\mu e_\nu e_\mu)
- \sum_{\alpha < \beta} (a^\alpha b^\beta e_\alpha e_\beta + a^\beta b^\alpha e_\beta e_\alpha )
\\
&=
2 \sum_{\mu < \nu} ( a^\mu b^\nu - a^\nu b^\mu ) e_\mu e_\nu
\end{aligned}
\end{equation}

So we have
\begin{equation}\label{eqn:introGa:340}
\begin{aligned}
a \wedge b
&= \sum_{\mu < \nu} \uDETuvij{a}{b}{\mu}{\nu} e_\mu e_\nu
\end{aligned}
\end{equation}

The similarity to the \R{3} vector cross product is not accidental.  This similarity can be made explicit by observing the following

\begin{equation}\label{eqn:introGa:360}
\begin{aligned}
e_1 e_2 &= e_1 e_2 (e_3 e_3) = (e_1 e_2 e_3) e_3 \\
e_2 e_3 &= e_2 e_3 (e_1 e_1) = (e_1 e_2 e_3) e_1 \\
e_1 e_3 &= e_1 e_3 (e_2 e_2) = -(e_1 e_2 e_3) e_2 \\
\end{aligned}
\end{equation}

This common factor, a product of three normal vectors, or grade three blade, is called the pseudoscalar for \R{3}.  We write
\(i = e_1 e_2 e_3\), and can then express the \R{3} wedge product in terms of the cross product

\begin{equation}\label{eqn:introGa:380}
\begin{aligned}
a \wedge b
&=
\uDETuvij{a}{b}{2}{3} e_2 e_3
+\uDETuvij{a}{b}{1}{3} e_1 e_3
+\uDETuvij{a}{b}{1}{2} e_1 e_2  \\
&=
(e_1 e_2 e_3) \left( \uDETuvij{a}{b}{2}{3} e_1
-\uDETuvij{a}{b}{1}{3} e_2
+\uDETuvij{a}{b}{1}{2} e_3 \right) \\
\end{aligned}
\end{equation}

This is

\begin{equation}\label{eqn:intro_ga:wedgeAsCross}
\begin{aligned}
a \wedge b &= i (a \cross b)
\end{aligned}
\end{equation}

With this identification we now also have a curious integrated relation where the dot and cross products are united into
a single structure

\begin{equation}\label{eqn:intro_ga:scalarPlusIcross}
\begin{aligned}
a b = a \cdot b + i (a \cross b)
\end{aligned}
\end{equation}

\section{Vector product in exponential form}

One naturally expects there is an inherent connection between the dot and cross products, especially when expressed in terms of
the angle between the vectors, as in

\begin{equation}\label{eqn:introGa:400}
\begin{aligned}
a \cdot b &= \Abs{a}\Abs{b} \cos\theta_{a,b} \\
a \cross b &= \Abs{a}\Abs{b} \sin\theta_{a,b} \ncap_{a,b}
\end{aligned}
\end{equation}

However, without the structure of the geometric product the specifics of what
connection is is not obvious.  In particular the use of \eqnref{eqn:intro_ga:scalarPlusIcross} and the angle relations, one can easily
blunder upon the natural complex structure of the geometric product

\begin{equation}\label{eqn:introGa:420}
\begin{aligned}
a b
&= a \cdot b + i (a \cross b) \\
&=
\Abs{a}\Abs{b} \left( \cos\theta_{a,b} + i\ncap_{a,b} \sin\theta_{a,b} \right) \\
\end{aligned}
\end{equation}

As we have seen pseudoscalar multiplication in \R{3} provides a mapping between a grade 2 blade and a vector, so
this \(i\ncap\) product is a 2-blade.

In \R{3} we also have \(i \ncap = \ncap i\) (exercise for reader) and also \(i^2 = -1\) (again for the reader), so this
2-blade \(i\ncap\) has all the properties of the \(i\) of complex arithmetic.  We can, in fact, write

\begin{equation}\label{eqn:introGa:440}
\begin{aligned}
a b
&= a \cdot b + i (a \cross b) \\
&=
\Abs{a}\Abs{b} \exp( i\ncap_{a,b} \theta_{a,b} )
\end{aligned}
\end{equation}

In particular, for unit vectors \(a\), \(b\) one is able to quaternion exponentials of this form to rotate from one vector to the other

\begin{equation}\label{eqn:introGa:460}
\begin{aligned}
b &= a \exp( i\ncap_{a,b} \theta_{a,b} )
\end{aligned}
\end{equation}

This natural GA use
of multivector exponentials to implement rotations is not restricted to \R{3} or even Euclidean space, and is one of the most
powerful features of the algebra.

\section{Pseudoscalar}

In general the
pseudoscalar for \R{N} is a product of \(N\) normal vectors and multiplication by such an object maps m-blades to (N-m) blades.

For \R{2} the unit pseudoscalar has a negative square

\begin{equation}\label{eqn:introGa:480}
\begin{aligned}
(e_1 e_2) (e_1 e_2)
&=
- (e_2 e_1) (e_1 e_2) \\
&=
- e_2 (e_1 e_1) e_2 \\
&=
- e_2 e_2 \\
&=
-1
\end{aligned}
\end{equation}

and we have seen an example of such a planar pseudoscalar in the subspace of the span of two vectors above (where \(\ncap i\) was a pseudoscalar
for that subspace).  In general the sign of the square of the pseudoscalar depends on both the dimension and the metric of the space,
so the ``complex'' exponentials that rotate one vector into another may represent hyperbolic rotations.

For example we have for a four dimensional space the pseudoscalar square is

\begin{equation}\label{eqn:introGa:500}
\begin{aligned}
i^2 &=
(e_0 e_1 e_2 e_3) (e_0 e_1 e_2 e_3) \\
&=
- e_0 e_0 e_1 e_2 e_3 e_1 e_2 e_3 \\
&=
- e_0 e_0 e_1 e_2 e_3 e_1 e_2 e_3 \\
&=
- e_0 e_0 e_1 e_1 e_2 e_3 e_2 e_3 \\
&=
e_0 e_0 e_1 e_1 e_2 e_2 e_3 e_3 \\
\end{aligned}
\end{equation}

For a Euclidean space where each of the \({e_k}^2 = 1\), we have \(i^2 = 1\), but for a Minkowski space where one would have for \(k\ne0\), \({e_0}^2 {e_k}^2 = -1\), we have \(i^2 = -1\)

Such a mixed signature metric will allow for implementation of Lorentz transformations as exponentials (hyperbolic) rotations
in a fashion very much like the quaternionic spatial rotations for Euclidean spaces.

It is also worth pointing out that the pseudoscalar multiplication naturally provides a mapping operator into a dual space, as we have seen
in the cross product example, mapping vectors to bivectors, or bivectors to vectors.  Pseudoscalar multiplication in fact provides an
implementation of the Hodge duality operation of differential geometry.

In higher than three dimensions, such as four, this duality operation can in fact map 2-blades to orthogonal 2-blades (orthogonal in the sense
of having no common factors).  Take for example the typical example of a non-simple element from differential geometry

\begin{equation}\label{eqn:introGa:520}
\begin{aligned}
\omega = e_1 \wedge e_2 + e_3 \wedge e_4
\end{aligned}
\end{equation}

The two blades that compose this sum have no common factors and thus cannot be formed as the wedge product of two vectors.  These two blades
are orthogonal in a sense that can be made more exact later.   As this time we just wish to make the observation that
the pseudoscalar provides a natural duality operation between these two subspaces of \(\bigwedge^2\).  Take for example

\begin{equation}\label{eqn:introGa:540}
\begin{aligned}
i e_1 \wedge e_2
&=
 e_1 e_2 e_3 e_4 e_1 e_2  \\
&=
- e_1 e_1 e_2 e_3 e_4 e_2  \\
&=
- e_1 e_1 e_2 e_2 e_3 e_4 \\
&\propto
e_3 e_4 \\
\end{aligned}
\end{equation}

\section{FIXME: orphaned}
As an exercise work out axiomatically some of the key vector identities of Geometric Algebra.

Want to at least derive the vector bivector dot product distribution
identity

\begin{equation}\label{eqn:introGa:20}
\begin{aligned}
a \cdot ( b \wedge c) = (a \cdot b) c - (a \cdot c) b
\end{aligned}
\end{equation}

%\section{Higher order products}

%   %
% Copyright � 2012 Peeter Joot.  All Rights Reserved.
% Licenced as described in the file LICENSE under the root directory of this GIT repository.
%

%
%
%\input{../peeter_prologue_print.tex}
%\input{../peeter_prologue_widescreen.tex}

\mychapter{Geometric Algebra.  The very quickest introduction}
\index{geometric product!introduction}
\label{chap:gaQuickIntro}

%\blogpage{http://sites.google.com/site/peeterjoot2/math2012/gaQuickIntro.pdf}
%\date{Mar 16, 2012}
%\gitRevisionInfo{gaQuickIntro}

\beginArtWithToc
%\beginArtNoToc

\section{Motivation}
An attempt to make a relatively concise introduction to Geometric (or Clifford) Algebra.  Much more complete introductions to the subject can be found in \citep{dorst2007gac}, \citep{doran2003gap}, and \citep{hestenes1999nfc}.
\section{Axioms}
We have a couple basic principles upon which the algebra is based
\begin{itemize}
\item Vectors can be multiplied.
\item The square of a vector is the (squared) length of that vector (with appropriate generalizations for non-Euclidean metrics).
\item Vector products are associative (but not necessarily commutative).
\end{itemize}

That is really all there is to it, and the rest, paraphrasing Feynman, can be figured out by anybody sufficiently clever.

\section{By example.  The 2D case}
Consider a 2D Euclidean space, and the product of two vectors \(\Ba\) and \(\Bb\) in that space.  Utilizing a standard orthonormal basis \(\{\Be_1, \Be_2\}\) we can write
\begin{equation}\label{eqn:gaQuickIntro:10}
\begin{aligned}
\Ba &= \Be_1 x_1 + \Be_2 x_2 \\
\Bb &= \Be_1 y_1 + \Be_2 y_2,
\end{aligned}
\end{equation}
and let us write out the product of these two vectors \(\Ba \Bb\), not yet knowing what we will end up with.  That is
\begin{equation}\label{eqn:gaQuickIntro:230}
\begin{aligned}
\Ba \Bb
&= (\Be_1 x_1 + \Be_2 x_2 )( \Be_1 y_1 + \Be_2 y_2 ) \\
&= \Be_1^2 x_1 y_1 + \Be_2^2 x_2 y_2
+ \Be_1 \Be_2 x_1 y_2 + \Be_2 \Be_1 x_2 y_1.
\end{aligned}
\end{equation}
%
From axiom 2 we have \(\Be_1^2 = \Be_2^2 = 1\), so we have
\begin{equation}\label{eqn:gaQuickIntro:30}
\Ba \Bb = x_1 y_1 + x_2 y_2 + \Be_1 \Be_2 x_1 y_2 + \Be_2 \Be_1 x_2 y_1.
\end{equation}
%
We have multiplied two vectors and ended up with a scalar component (and recognize that this part of the vector product is the dot product), and a component that is a ``something else''.  We will call this something else a bivector, and see that it is characterized by a product of non-colinear vectors.  These products \(\Be_1 \Be_2\) and \(\Be_2 \Be_1\) are in fact related, and we can see that by looking at the case of \(\Bb = \Ba\).  For that we have
\begin{equation}\label{eqn:gaQuickIntro:250}
\begin{aligned}
\Ba^2
&=
x_1 x_1 + x_2 x_2 + \Be_1 \Be_2 x_1 x_2 + \Be_2 \Be_1 x_2 x_1 \\
&=
\Abs{\Ba}^2 +
x_1 x_2 ( \Be_1 \Be_2 + \Be_2 \Be_1 )
\end{aligned}
\end{equation}
%
Since axiom (2) requires our vectors square to equal its (squared) length, we must then have
\begin{equation}\label{eqn:gaQuickIntro:50}
\Be_1 \Be_2 + \Be_2 \Be_1 = 0,
\end{equation}
or
\begin{equation}\label{eqn:gaQuickIntro:70}
\Be_2 \Be_1 = -\Be_1 \Be_2.
\end{equation}
%
We see that Euclidean orthonormal vectors anticommute.  What we can see with some additional study is that any colinear vectors commute, and in Euclidean spaces (of any dimension) vectors that are normal to each other anticommute (this can also be taken as a definition of normal).

We can now return to our product of two vectors \eqnref{eqn:gaQuickIntro:30} and simplify it slightly
\begin{equation}\label{eqn:gaQuickIntro:30b}
\Ba \Bb = x_1 y_1 + x_2 y_2 + \Be_1 \Be_2 (x_1 y_2 - x_2 y_1).
\end{equation}
%
The product of two vectors in 2D is seen here to have one scalar component, and one bivector component (an irreducible product of two normal vectors).  Observe the symmetric and antisymmetric split of the scalar and bivector components above.  This symmetry and antisymmetry can be made explicit, introducing dot and wedge product notation respectively
\begin{equation}\label{eqn:gaQuickIntro:90}
\begin{aligned}
\Ba \cdot \Bb &= \inv{2}( \Ba \Bb + \Bb \Ba) = x_1 y_1 + x_2 y_2 \\
\Ba \wedge \Bb &= \inv{2}( \Ba \Bb - \Bb \Ba) = \Be_1 \Be_2 (x_1 y_y - x_2 y_1).
\end{aligned}
\end{equation}
so that the vector product can be written as
\begin{equation}\label{eqn:gaQuickIntro:110}
\Ba \Bb = \Ba \cdot \Bb + \Ba \wedge \Bb.
\end{equation}
\section{Pseudoscalar}
In many contexts it is useful to introduce an ordered product of all the unit vectors for the space is called the pseudoscalar.  In our 2D case this is
\begin{equation}\label{eqn:gaQuickIntro:130}
i = \Be_1 \Be_2,
\end{equation}
a quantity that we find behaves like the complex imaginary.  That can be shown by considering its square
\begin{equation}\label{eqn:gaQuickIntro:270}
\begin{aligned}
(\Be_1 \Be_2)^2
&=
(\Be_1 \Be_2)
(\Be_1 \Be_2) \\
&=
\Be_1 (\Be_2 \Be_1) \Be_2 \\
&=
-\Be_1 (\Be_1 \Be_2) \Be_2 \\
&=
-(\Be_1 \Be_1) (\Be_2 \Be_2) \\
&=
-1^2 \\
&= -1
\end{aligned}
\end{equation}
%
Here the anticommutation of normal vectors property has been used, as well as (for the first time) the associative multiplication axiom.

In a 3D context, you will see the pseudoscalar in many places (expressing the normals to planes for example).  It also shows up in a number of fundamental relationships.  For example, if one writes
\begin{equation}\label{eqn:gaQuickIntro:150}
I = \Be_1 \Be_2 \Be_3
\end{equation}
for the 3D pseudoscalar, then it is also possible to show
\begin{equation}\label{eqn:gaQuickIntro:170}
\Ba \Bb = \Ba \cdot \Bb + I (\Ba \cross \Bb)
\end{equation}
something that will be familiar to the student of QM, where we see this in the context of Pauli matrices.  The Pauli matrices also encode a Clifford algebraic structure, but we do not need an explicit matrix representation to do so.
\section{Rotations}
%
% Copyright © 2012 Peeter Joot.  All Rights Reserved.
% Licenced as described in the file LICENSE under the root directory of this GIT repository.
%

Very much like complex numbers we can utilize exponentials to perform rotations.  Rotating in a sense from \(\Be_1\) to \(\Be_2\), can be expressed as
%
\begin{equation}\label{eqn:gaQuickIntro:290}
\begin{aligned}
\Ba e^{i \theta}
&=
(\Be_1 x_1 + \Be_2 x_2) (\cos\theta + \Be_1 \Be_2 \sin\theta) \\
&=
\Be_1 (x_1 \cos\theta - x_2 \sin\theta)
+
\Be_2 (x_2 \cos\theta + x_1 \sin\theta)
\end{aligned}
\end{equation}
%
More generally, even in N dimensional Euclidean spaces, if \(\Ba\) is a vector in a plane, and \(\ucap\) and \(\vcap\) are perpendicular unit vectors in that plane, then the rotation through angle \(\theta\) is given by
%
\begin{equation}\label{eqn:gaQuickIntro:190}
\Ba \rightarrow \Ba e^{\ucap \vcap \theta}.
\end{equation}
%
This is illustrated in \cref{fig:gaQuickIntro:gaQuickIntroFig1}

\pdfTexFigure{../figures/gabook/gaQuickIntroFig1.pdf_tex}{Plane rotation}{fig:gaQuickIntro:gaQuickIntroFig1}{0.6}

Notice that we have expressed the rotation here without utilizing a normal direction for the plane.  The sense of the rotation is encoded by the bivector \(\ucap \vcap\) that describes the plane and the orientation of the rotation (or by duality the direction of the normal in a 3D space).  By avoiding a requirement to encode the rotation using a normal to the plane we have an method of expressing the rotation that works not only in 3D spaces, but also in 2D and greater than 3D spaces, something that is not possible when we restrict ourselves to traditional vector algebra (where quantities like the cross product can not be defined in a 2D or 4D space, despite the fact that things they may represent, like torque are planar phenomena that do not have any intrinsic requirement for a normal that falls out of the plane.).

When \(\Ba\) does not lie in the plane spanned by the vectors \(\ucap\) and \(\vcap\) , as in \cref{fig:gaQuickIntro:gaQuickIntroFig2}, we must express the rotations differently.  A rotation then takes the form
%
\begin{equation}\label{eqn:gaQuickIntro:210}
\Ba \rightarrow
e^{-\ucap \vcap \theta/2}
\Ba
e^{\ucap \vcap \theta/2}.
\end{equation}
%

\pdfTexFigure{../figures/gabook/gaQuickIntroFig2.pdf_tex}{3D rotation}{fig:gaQuickIntro:gaQuickIntroFig2}{0.6}

In the 2D case, and when the vector lies in the plane this reduces to the one sided complex exponential operator used above.  We see these types of paired half angle rotations in QM, and they are also used extensively in computer graphics under the guise of quaternions.

%\EndArticle

%   %
% Copyright � 2012 Peeter Joot.  All Rights Reserved.
% Licenced as described in the file LICENSE under the root directory of this GIT repository.
%

%
%
\mychapter{An (earlier) attempt to intuitively introduce the dot, wedge, cross, and geometric products}
\index{dot product!introduction}
\index{wedge product!introduction}
\index{cross product!introduction}
\index{geometric product!introduction}
\label{chap:gaGradeDotWedge}

%\date{March 17, 2008.  gaGradeDotWedge.tex}

\section{Motivation}

Both the NFCM and GAFP books have axiomatic introductions of the
generalized (vector, blade) dot and wedge products, but there are
elements of both that I was unsatisfied with.  Perhaps the biggest
issue with both is that they are not presented in a dumb enough fashion.

NFCM presents but
does not prove the generalized dot and wedge product operations
in terms of symmetric and antisymmetric sums, but it is really the
grade operation that is fundamental.  You need that to define the
dot product of two bivectors for example.

GAFP axiomatic presentation is much clearer, but the definition of
generalized wedge product as the totally antisymmetric sum is a bit
strange when all the differential forms book give such a different
definition.

Here I collect some of my notes on how one starts with the geometric
product action on colinear and perpendicular vectors and gets the
familiar results for two and three vector products.  I may not try to
generalize this, but just want to see things presented in a fashion
that makes sense to me.

\section{Introduction}

The aim of this document is to introduce a ``new'' powerful vector multiplication operation, the geometric product,
to a student with some traditional vector algebra background.

The geometric product, also called the Clifford product
\footnote{After William Clifford (1845-1879).}, has remained a relatively obscure mathematical subject.
This operation actually makes a great deal of vector manipulation simpler than possible with the traditional methods, and
provides a way to naturally expresses many geometric concepts.
There is a great deal of information available on the subject, however most of it is targeted for those with a
university graduate school background in physics or mathematics.  That level of mathematical sophistication
should not required to understand the subject.

It is the author's opinion that this could be dumbed down even further, so that it would be palatable for
somebody without any traditional vector algebra background.

\section{What is multiplication?}

The operations of vector addition, subtraction and numeric multiplication have the usual definitions
(addition defined in terms of addition of coordinates, and numeric multiplication as a scaling of the vector retaining its direction).  Multiplication and division of vectors is often described as ``undefined''.  It is possible however, to define a multiplication, or division operation for vectors, in a natural geometric fashion.

What meaning should be given to multiplication or division of vectors?

\subsection{Rules for multiplication of numbers}

Assuming no prior knowledge of how to multiply two vectors (such as the dot, cross, or wedge products to be introduced later) consider instead the rules for multiplication of numbers.

\begin{enumerate}
\item Product of two positive numbers is positive.  Any consideration of countable sets of objects justifies this rule.

\item Product of a positive and negative number is negative.  Example: multiplying a debt (negative number) increases the amount of the debt.

\item Product of a negative and negative number is positive.

\item Multiplication is distributive.  Product of a sum is the sum of the products.
\footnote{The name of this property is not important and no student should ever be tested on it.  It is a word like
dividand which countless countless school kids are forced to memorize.  Like dividand it is perfectly
acceptable to forget it after the test because nobody has to know it to perform division.
Since most useful sorts of multiplications have this property this is the least important
of the named multiplication properties.  This word exists mostly so that authors of math books can impress themselves writing phrases like ``a mathematical entity that behaves this way is
left and right distributive with respect to addition''.
}

\begin{equation}\label{eqn:gaGradeDotWedge:20}
a (b + c) = a b + a c
\end{equation}
\begin{equation}\label{eqn:gaGradeDotWedge:40}
(a + b) c = a c + b c
\end{equation}

\item Multiplication is associative.  Changing the order that multiplication is grouped by does not change the result.

\begin{equation}\label{eqn:gaGradeDotWedge:60}
(a b) c = a (b c)
\end{equation}

\item Multiplication is commutative.  Switching the order of multiplication does not change the result.

\begin{equation}\label{eqn:gaGradeDotWedge:80}
a b = b a
\end{equation}

\end{enumerate}

Unless the reader had an exceptionally gifted grade three teacher it is likely that rule three was presented without any sort of justification or analogy.  This can be considered as a special case of the previous rule.  Geometrically, a multiplication by -1 results in an inversion on the number line.  If one considers the number line to be a line in space, then this is a 180 degree rotation.  Two negative multiplications results in a 360 degree rotation, and thus takes the number back to its original positive or negative segment on its ``number line''.

\subsection{Rules for multiplication of vectors with the same direction}

Having identified the rules for multiplication of numbers, one can use these to define multiplication rules for a simple case, one dimensional vectors.
Conceptually a one dimensional vector space can be thought of like a number line, or the set of all numbers as the set of all scalar multiples of a unit vector of a particular direction in space.

It is reasonable to expect the rules for multiplication of two vectors with the same direction to have some of the same characteristics as multiplication of numbers.  Lets state this algebraically writing the directed distance from the origin to the points \(a\) and \(b\) in a vector notation

\begin{equation}\label{eqn:gaGradeDotWedge:540}
\begin{aligned}
\Ba &= a\Be \\
\Bb &= b\Be \\
\end{aligned}
\end{equation}

where \(\Be\) is the unit vector alone the line in question.

The product of these two vectors is

\begin{equation}\label{eqn:gaGradeDotWedge:100}
\Ba \Bb = a b \Be \Be
\end{equation}

Although no specific meaning has yet been given to the \(\Be \Be\) term yet, one can make a few observations about a product of this form.
\begin{enumerate}
\item It is commutative, since \(\Ba \Bb = \Bb \Ba = a b \Be \Be\).
\item It is distributive since numeric multiplication is.
\item The product of three such vectors is distributive (no matter the grouping of the multiplications there will be a numeric factor and a \(\Be \Be \Be\) factor.
\end{enumerate}

These properties are consistent with half the properties of numeric multiplication.  If the other half of the numeric multiplication rules are assumed to also apply we have

\begin{enumerate}
\item Product of two vectors in the same direction is positive (rules 1 and 3 above).
\item Product of two vectors pointing in opposite directions is negative (rule 2 above).
\end{enumerate}

This can only be possible by giving the following meaning to the square of a unit vector

\begin{equation}\label{eqn:gaGradeDotWedge:120}
\Be \Be = 1
\end{equation}

Alternately, one can state that the square of a vector is that vectors squared length.

\begin{equation}\label{eqn:gaGradeDotWedge:140}
\Ba \Ba = a^2
\end{equation}

This property, as well as the associative and distributive properties are the defining properties of the geometric product.

It will be shown shortly that in order to retain this squared vector length property for vectors with components in different directions it will be required to drop the commutative property of numeric multiplication:

\begin{equation}\label{eqn:gaGradeDotWedge:160}
\Ba \Bb \neq \Bb \Ba
\end{equation}

This is a choice that will later be observed to have important consequences.
There are many types of multiplications that do not have the commutative property.  Matrix multiplication is not even necessarily defined when the order is switched.  Other multiplication operations (wedge and cross products) change sign when the order is switched.

Another important choice has been made to require the product of two vectors not be a vector itself.  This also breaks
from the number line analogy since the product of two numbers is still a number.  However, it is
notable that
in order to take roots of a negative number one has to introduce a second number line
(the \(i\), or imaginary axis), and so even for numbers, products can be ``different'' than their factors.
Interestingly enough,
it will later be possible to show that the choice to not require a vector product to be a vector
allow complex numbers to be defined directly in terms of the geometric product of two vectors in a plane.

\section{Axioms}

The previous discussion attempts to justify the choice of the following set of axioms for multiplication of vectors

\begin{enumerate}
\item{ linearity }
\item{ associativity }
\item{ contraction }

Square of a vector is its squared length.
\end{enumerate}

This last property is weakened in some circumstances (for example,
an alternate definition of vector length is desirable for relativistic calculations.)

%As justification of the contraction property one could
%consider a set of colinear vectors and the real number line to be
%isomorphic.
%
%The product of two positive numbers is a positive number.  Multiplying
%by \(-1\) (the unit negative) produces a rotatation by 180 degrees.  Two negative multiplications
%produces a rotation of 360.  This can be thought of as a justification
%of the grade school ``rule'' that a negative times a negative is positive.
%
%It seems natural to have the rules for vector multiplication reduce to something
%like the rules for numbers when those vectors are restricted to a linear subspace.
%
%In analogy with numbers, the contraction rule gives us such similar properties.  Namely, the
%product of same facing vectors is positive, and the product of opposite facing
%vectors is negative, both scaled by their magnitudes.

\section{dot product}

One can express the dot product in terms of these axioms.  This follows by calculating the
length of a sum or difference of vectors, starting with the requirement that the vector square is the squared length of that vector.

Given two vectors \(\Ba\) and \(\Bb\), their sum
\(\Bc = \Ba + \Bb\) has squared length:

\begin{equation}\label{eqn:introGaFirst:absquared}
\Bc^2 = (\Ba + \Bb)(\Ba + \Bb) = \Ba^2 + \Bb\Ba + \Ba\Bb + \Bb^2.
\end{equation}

We do not have any specific meaning for the product of vectors, but \eqnref{eqn:introGaFirst:absquared}
shows that the symmetric sum of such a product:

\begin{equation}
\Bb\Ba + \Ba\Bb = \text{scalar}
\end{equation}

since the RHS is also a scalar.

Additionally, if \(\Ba\) and \(\Bb\) are perpendicular, then we must also have:

\begin{equation}\label{eqn:gaGradeDotWedge:180}
\Ba^2 + \Bb^2 = a^2 + b^2.
\end{equation}

This implies a rule for vector multiplication of perpendicular vectors

\begin{equation}\label{eqn:gaGradeDotWedge:200}
\Bb\Ba + \Ba\Bb = 0
\end{equation}

Or,

\begin{equation}\label{eqn:introGaFirst:perpabcommutesign}
\Bb\Ba = -\Ba\Bb.
\end{equation}

Note that \eqnref{eqn:introGaFirst:perpabcommutesign} does not assign any meaning to this product of vectors when they perpendicular.
Whatever that meaning is, the entity such a perpendicular vector product produces changes sign
with commutation.

Performing the same length calculation using standard vector algebra shows that we can identify the symmetric
sum of vector products with the dot product:

\begin{equation}\label{eqn:introGaFirst:standarddot}
\norm{\Bc}^2 = (\Ba + \Bb) \cdot (\Ba + \Bb) = \norm{\Ba}^2 + 2 \Ba \cdot \Bb + \norm{\Bb}^2.
\end{equation}

Thus we can make the identity:

\begin{equation}\label{eqn:introGaFirst:dotprod}
\Ba \cdot \Bb = \inv{2}(\Ba \Bb + \Bb \Ba)
\end{equation}

\section{Coordinate expansion of the geometric product}

A powerful feature of geometric algebra is that it allows for coordinate free results, and the avoidance of basis selection
that coordinates require.  While this is true, explicit coordinate expansion, especially
initially while making the transition from coordinate based vector algebra, is believed to add clarity to the subject.

Writing a pair of vectors in coordinate vector notation:

\begin{equation}\label{eqn:gaGradeDotWedge:220}
\Ba = \sum_i{a_i \Be_i}
\end{equation}
\begin{equation}\label{eqn:gaGradeDotWedge:240}
\Bb = \sum_j{b_j \Be_j}
\end{equation}

Despite not yet
knowing what meaning to give to the geometric product of two general (non-colinear) vectors,
given the axioms above and their consequences we actually have enough information to completely
expand the geometric product of two vectors in terms of these coordinates:

\begin{equation}\label{eqn:gaGradeDotWedge:560}
\begin{aligned}
\Ba\Bb
&= \sum_{ij}{a_i b_j \Be_i \Be_j} \\
&= \sum_{i = j} {a_i b_j \Be_i \Be_j}
 + \sum_{i \ne j} {a_i b_j \Be_i \Be_j} \\
&= \sum_{i} {a_i b_i \Be_i \Be_i}
 + \sum_{i < j} a_i b_j \Be_i \Be_j
 + \sum_{j < i} a_i b_j \Be_i \Be_j \\
&= \sum_{i} {a_i b_i}
 + \sum_{i < j} a_i b_j \Be_i \Be_j + a_j b_i \Be_j \Be_i \\
&= \sum_{i} {a_i b_i}
 + \sum_{i < j} (a_i b_j - b_i a_j)\Be_i \Be_j \\
\end{aligned}
\end{equation}

This can be summarized nicely in terms of determinants:

\begin{equation}\label{eqn:introGaFirst:geocoord}
\Ba\Bb = \sum_{i} {a_i b_i} + \sum_{i < j} \DETuvij{a}{b}{i}{j} \Be_i \Be_j
\end{equation}

This shows,
without requiring the ``triangle law'' expansion of \eqnref{eqn:introGaFirst:standarddot},
that the geometric product has a scalar component that we recognize as the Euclidean vector dot product.  It also shows that the remaining bit
is a ``something else'' component.  This ``something else'' is called a bivector.  We do not yet know what this bivector is or what to do with it,
but will come back to that.

Observe that an interchange of \(\Ba\) and \(\Bb\) leaves the scalar part of equation
\eqnref{eqn:introGaFirst:geocoord} unaltered (ie: it is symmetric), whereas an interchange inverts the bivector (ie: it is the antisymmetric part).

\section{Some specific examples to get a feel for things}

Moving from the abstract, consider a few specific geometric product example.

\begin{itemize}
\item Product of two non-colinear non-orthogonal vectors.
\begin{equation}\label{eqn:gaGradeDotWedge:260}
(\Be_1 + 2\Be_2) (\Be_1 - \Be_2)
= \Be_1\Be_1 -2\Be_2\Be_2 + 2\Be_2\Be_1 - \Be_1\Be_2
= -1 + 3\Be_2\Be_1
\end{equation}

Such a product produces both scalar and bivector parts.

\item Squaring a bivector
\begin{equation}\label{eqn:gaGradeDotWedge:280}
(\Be_1\Be_2)^2
=
(\Be_1\Be_2)(-\Be_2\Be_1)
=
-\Be_1(\Be_2\Be_2)\Be_1
=
-\Be_1\Be_1
=
-1
\end{equation}

This particular bivector squares to minus one very much like the imaginary number \(i\).

\item Product of two perpendicular vectors.

\begin{equation}\label{eqn:gaGradeDotWedge:300}
(\Be_1 + \Be_2) (\Be_1 - \Be_2) = 2\Be_2\Be_1
\end{equation}

Such a product generates just a bivector term.

\item Product of a bivector and a vector in the plane.

\begin{equation}\label{eqn:gaGradeDotWedge:320}
(x\Be_1 + y\Be_2) \Be_1\Be_2
=
x\Be_2 - y\Be_1
\end{equation}

This rotates the vector counterclockwise by 90 degrees.

\item General \R{3} geometric product of two vectors.

\begin{equation}\label{eqn:gaGradeDotWedge:340}
\Bx \By =
(x_1\Be_1
+x_2\Be_2
+x_3\Be_3)
(y_1\Be_1
+y_2\Be_2
+y_3\Be_3)
\end{equation}
\begin{equation}\label{eqn:gaGradeDotWedge:360}
=
\Bx \cdot \By
+\DETuvij{x}{y}{2}{3} \Be_2 \Be_3
+\DETuvij{x}{y}{1}{3} \Be_1 \Be_3
+\DETuvij{x}{y}{1}{2} \Be_1 \Be_2
\end{equation}

Or,
\begin{equation}\label{eqn:gaGradeDotWedge:380}
\Bx \By =
\Bx \cdot \By +
\begin{vmatrix}
\Be_2\Be_3 & \Be_3\Be_1 & \Be_1\Be_2 \\
x_1 & x_2 & x_3 \\
y_1 & y_2 & y_3 \\
\end{vmatrix}
\end{equation}

Observe that if one identifies
\(\Be_2\Be_3\), \(\Be_3\Be_1\), and \(\Be_1\Be_2\) with vectors
\(\Be_1\),
\(\Be_2\),
and \(\Be_3\) respectively, this second term is the cross product.  A precise way to perform this identification will be described later.

The key thing to observe here is
that the structure of the cross product is naturally associated with the geometric product.  One can think of the geometric product
as a complete product including elements of both the dot and cross product.  Unlike the cross product the geometric product is also well defined
in two dimensions and greater than three.

\end{itemize}

These examples are all somewhat random, but give a couple hints of results to come.

\section{Antisymmetric part of the geometric product}

Having identified the symmetric sum of vector products with the dot product we can write the geometric product of two arbitrary vectors
in terms of this and its difference

\begin{equation}\label{eqn:gaGradeDotWedge:580}
\begin{aligned}
\Ba \Bb
&= \inv{2}(\Ba \Bb + \Bb \Ba) + \inv{2}(\Ba \Bb - \Bb \Ba) \\
&= \Ba \cdot \Bb + f(\Ba, \Bb) \\
\end{aligned}
\end{equation}

Let us examine this second term, the bivector, a mapping of a pair of vectors into a different sort of object of yet unknown properties.

\begin{equation}\label{eqn:gaGradeDotWedge:400}
f(\Ba, k\Ba) = \inv{2}(\Ba k\Ba - k\Ba \Ba) = 0
\end{equation}

Property: Zero when the vectors are colinear.

\begin{equation}\label{eqn:gaGradeDotWedge:420}
f(\Ba, k\Ba + \Bb) = \inv{2}(\Ba (k\Ba + \Bb) - (k\Ba + m\Bb)\Ba) = f(\Ba, \Bb)
\end{equation}

Property: colinear contributions are rejected.

\begin{equation}\label{eqn:gaGradeDotWedge:440}
f(\alpha \Ba, \beta \Bb) = \inv{2}(\alpha \Ba \beta \Bb - \beta \Bb \alpha \Ba) = \alpha \beta f(\Ba, \Bb)
\end{equation}

Property: bilinearity.

\begin{equation}\label{eqn:gaGradeDotWedge:460}
f(\Bb, \Ba)
= \inv{2}(\Bb \Ba - \Ba\Bb)
= -\inv{2}(\Ba \Bb - \Bb\Ba)
= -f(\Ba, \Bb)
\end{equation}

Property: Interchange inverts.

Operationally, these are in fact the properties of what in the calculus of differential forms is called the wedge product (uncoincidentally, these are also all properties of the cross product as well.)

Because the properties are identical the notation from differential forms is stolen, and we write

\begin{equation}\label{eqn:introGaFirst:wedge}
\Ba \wedge \Bb = \inv{2}(\Ba \Bb - \Bb \Ba)
\end{equation}

And as mentioned, the object that this
wedge product produces from two vectors is called a bivector.

Strictly speaking the
wedge product of differential calculus is defined as an alternating, associative, multilinear form.  We have here bilinear, not multilinear and associativity is
not meaningful until more than two factors are introduced, however when we get to the product of more than three
vectors, we will find that the geometric vector product produces an entity with all of these properties.

Returning to the product of two vectors we can now write

\begin{equation}\label{eqn:introGaFirst:gaproddotwedge}
\Ba \Bb = \Ba \cdot \Bb + \Ba \wedge \Bb
\end{equation}

This is often used as the initial definition of the geometric product.

\section{Yes, but what is that wedge product thing}

Combination of the symmetric and antisymmetric decomposition in \eqnref{eqn:introGaFirst:gaproddotwedge} shows that the product of two vectors according to the axioms
has a scalar part and a bivector part.  What is this bivector part geometrically?

One can show that the equation of a plane can be written in terms of bivectors.  One can also show that
the area of the parallelogram spanned by two vectors can be expressed in terms of the ``magnitude'' of a bivector.  Both of these
show that a bivector characterizes a plane and can be thought of loosely as a ``plane vector''.

Neither the plane equation or the area result are hard to show, but we will get to those later.  A more direct way to get an
intuitive feel for the geometric properties of the bivector can be obtained by first examining the
square of a bivector.

By subtracting the projection of one vector \(\Ba\) from another \(\Bb\), one can form the rejection of \(\Ba\) from \(\Bb\):

\begin{equation}\label{eqn:gaGradeDotWedge:480}
\Bb' = \Bb - (\Bb \cdot \acap)\acap
\end{equation}

With respect to the dot product, this vector is orthogonal to \(\Ba\).  Since \(\Ba \wedge \acap = 0\), this allows us to
write the wedge product of vectors \(\Ba\) and \(\Bb\) as the direct product of two orthogonal vectors

\begin{equation}\label{eqn:gaGradeDotWedge:600}
\begin{aligned}
\Ba \wedge \Bb
&= \Ba \wedge (\Bb - (\Bb \cdot \acap)\acap)) \\
&= \Ba \wedge \Bb' \\
&= \Ba \Bb' \\
&= -\Bb' \Ba \\
\end{aligned}
\end{equation}

The square of the bivector can then be written

\begin{equation}\label{eqn:gaGradeDotWedge:620}
\begin{aligned}
(\Ba \wedge \Bb)^2
&= (\Ba \Bb')(-\Bb'\Ba) \\
&= -\Ba^2 (\Bb')^2.
\end{aligned}
\end{equation}

Thus the square of a bivector is negative.  It is natural to define a
bivector norm:

\begin{equation}\label{eqn:gaGradeDotWedge:500}
\abs{\Ba \wedge \Bb} = \sqrt{-(\Ba \wedge \Bb)^2} = \sqrt{ (\Ba \wedge \Bb)(\Bb \wedge \Ba) }
\end{equation}

Dividing by this norm we have an entity that acts precisely like the imaginary number \(i\).

Looking back to \eqnref{eqn:introGaFirst:gaproddotwedge} one can now assign additional meaning to the two parts.  The first, the dot product, is a scalar (ie: a real number), and a second part, the wedge product, is a pure imaginary term.  Provided \(\Ba \wedge \Bb \ne 0\), we can write \(i = \frac{\Ba \wedge \Bb}{ \abs{\Ba \wedge \Bb} }\) and express
the geometric product in complex number form:

\begin{equation}\label{eqn:gaGradeDotWedge:520}
\Ba \Bb = \Ba \cdot \Bb + i \abs{\Ba \wedge \Bb}
\end{equation}

The complex number system
is the algebra of the plane, and the geometric product of two vectors can be used to completely characterize the algebra of an arbitrarily oriented plane in a higher
order vector space.

It actually will be very natural to define complex numbers in terms of the geometric product, and we will see later that
the geometric product allows for the ad-hoc definition of ``complex number'' systems according to convenience in many ways.

We will also see that generalizations of complex numbers such as quaternion algebras also find their natural place as specific instances of geometric products.

Concepts familiar from
complex numbers such as conjugation, inversion, exponentials as rotations, and even things like the residue theory of complex contour integration, will
all have a natural geometric algebra analogue.

We will return to this, but first some more detailed initial examination of the wedge product properties is in order, as is a look at the product of greater than
two vectors.


   \chapter{Basics}
      \section{Did you ever ask your teacher how to multiply vectors?}
         \input{../GAelectrodynamics/.junk/GAmotivation.tex}
         \subsection{Problems}
            \input{../GAelectrodynamics/ComplexInnerProductVsDotAndCrossProduct.tex}
      \section{Vector multiplication}
         \input{../GAelectrodynamics/multiplication.tex}
         \subsection{Problems}
            \input{../GAelectrodynamics/2dvectorsquare.tex}
            \input{../GAelectrodynamics/normalAnticommutation.tex}
      \section{Definitions}
         \input{../GAelectrodynamics/definitions.tex}
         \subsection{Problems}
            \input{../GAelectrodynamics/R3PseudoscalarSquare.tex}
      \section{Grade selection, dot and wedge product operators}
         \input{../GAelectrodynamics/gradeselection.tex}
         \subsection{Problems}
            \input{../GAelectrodynamics/RnDotProduct.tex}
            \input{../GAelectrodynamics/cyclicpermutationtwo.tex}
            \input{../GAelectrodynamics/dotprodSymmetricSum.tex}
            \input{../GAelectrodynamics/planeRotationsExponentials.tex}
            \input{../GAelectrodynamics/complexNumbers.tex}
            \input{../GAelectrodynamics/R3PseudoscalarCommutation.tex}
            \input{../GAelectrodynamics/gradeselVectorWedge.tex}
            \input{../GAelectrodynamics/WedgeRelationshipToCrossProduct.tex}
            \input{../GAelectrodynamics/vectorBivectorDot.tex}
            \input{../GAelectrodynamics/vectorTrivectorDot.tex}
            \input{../GAelectrodynamics/bivectorDot.tex}
            \input{../GAelectrodynamics/r4nonzerobivectorwedgewithself.tex}
            \input{../GAelectrodynamics/scalarPermutation.tex}
      \section{Product of two vectors}
         \input{../GAelectrodynamics/vectorproduct.tex}
         \subsection{Miscellanious theorems}
            \input{../stokesTheorem/bladeDotWedgeSymmetryIdentitiesTheorem.tex}
         %   %
% Copyright © 2016 Peeter Joot.  All Rights Reserved.
% Licenced as described in the file LICENSE under the root directory of this GIT repository.
%
\maketheorem{Distribution of inner products}{thm:stokesTheoremGeometricAlgebra:1420}{
Given two blades \(A_s, B_r\) with grades subject to \(s > r > 0\), and a vector \(b\), the inner product distributes according to
\begin{equation*}
A_s \cdot \lr{ b \wedge B_r } = \lr{ A_s \cdot b } \cdot B_r.
\end{equation*}
}

This will allow us, for example, to expand a general inner product of the form \(d^k \Bx \cdot (\boldpartial \wedge F)\).

The proof is straightforward, but also mechanical.  Start by expanding the wedge and dot products within a grade selection operator

\begin{dmath}\label{eqn:stokesTheoremGeometricAlgebra:1460}
A_s \cdot \lr{ b \wedge B_r }
=
\gpgrade{A_s (b \wedge B_r)}{s - (r + 1)}
=
\inv{2} \gpgrade{A_s \lr{b B_r + (-1)^{r} B_r b} }{s - (r + 1)}
\end{dmath}

Solving for \(B_r b\) in

\begin{dmath}\label{eqn:stokesTheoremGeometricAlgebra:1480}
2 b \cdot B_r = b B_r - (-1)^{r} B_r b,
\end{dmath}

we have

\begin{dmath}\label{eqn:stokesTheoremGeometricAlgebra:1500}
A_s \cdot \lr{ b \wedge B_r }
=
\inv{2} \gpgrade{ A_s b B_r + A_s \lr{ b B_r - 2 b \cdot B_r } }{s - (r + 1)}
=
\gpgrade{ A_s b B_r }{s - (r + 1)}
-
\cancel{\gpgrade{ A_s \lr{ b \cdot B_r } }{s - (r + 1)}}.
\end{dmath}

The last term above is zero since we are selecting the \(s - r - 1\) grade element of a multivector with grades \(s - r + 1\) and \(s + r - 1\), which has no terms for \(r > 0\).  Now we can expand the \(A_s b\) multivector product, for

\begin{dmath}\label{eqn:stokesTheoremGeometricAlgebra:1520}
A_s \cdot \lr{ b \wedge B_r }
=
\gpgrade{ \lr{ A_s \cdot b + A_s \wedge b} B_r }{s - (r + 1)}.
\end{dmath}

The latter multivector (with the wedge product factor) above has grades \(s + 1 - r\) and \(s + 1 + r\), so this selection operator finds nothing.  This leaves

\begin{dmath}\label{eqn:stokesTheoremGeometricAlgebra:1540}
A_s \cdot \lr{ b \wedge B_r }
=
\gpgrade{
\lr{ A_s \cdot b } \cdot B_r
+ \lr{ A_s \cdot b } \wedge B_r
}{s - (r + 1)}.
\end{dmath}

The first dot products term has grade \(s - 1 - r\) and is selected, whereas the wedge term has grade \(s - 1 + r \ne s - r - 1\) (for \(r > 0\)).  \(\qedmarker\)

%Next consider an expansion that we cannot do above, but require


         \subsection{Problems}
            \input{../GAelectrodynamics/vectorproductCyclicPermutation.tex}
            \input{../GAelectrodynamics/wedgeantisym.tex}
            \input{../GAelectrodynamics/gradethreeselectionWedge.tex}
            \input{../stokesTheorem/bladeDotWedgeSymmetryIdentities.tex}
            %
% Copyright © 2016 Peeter Joot.  All Rights Reserved.
% Licenced as described in the file LICENSE under the root directory of this GIT repository.
%
\makeproblem{Wedge distribution identity.}{problem:wedgeDistributionIdentityProblems:1}{
Prove \cref{thm:stokesTheoremGeometricAlgebra:1420}.
} % problem

%This will allow us, for example, to expand a general inner product of the form \(d^k \Bx \cdot (\boldpartial \wedge F)\).
\makeanswer{problem:wedgeDistributionIdentityProblems:1}{
The proof is straightforward, but also mechanical.  Start by expanding the wedge and dot products within a grade selection operator
\begin{equation}\label{eqn:stokesTheoremGeometricAlgebra:1460}
\begin{aligned}
A_s \cdot \lr{ \Bb \wedge B_r }
&=
\gpgrade{A_s (\Bb \wedge B_r)}{s - (r + 1)} \\
&=
\inv{2} \gpgrade{A_s \lr{\Bb B_r + (-1)^{r} B_r \Bb} }{s - (r + 1)}.
\end{aligned}
\end{equation}

Solving for \(B_r \Bb\) in
\begin{equation}\label{eqn:stokesTheoremGeometricAlgebra:1480}
2 \Bb \cdot B_r = \Bb B_r - (-1)^{r} B_r \Bb,
\end{equation}
we have
\begin{equation}\label{eqn:stokesTheoremGeometricAlgebra:1500}
\begin{aligned}
A_s \cdot \lr{ \Bb \wedge B_r }
&=
\inv{2} \gpgrade{ A_s \Bb B_r + A_s \lr{ \Bb B_r - 2 \Bb \cdot B_r } }{s - (r + 1)} \\
&=
\gpgrade{ A_s \Bb B_r }{s - (r + 1)}
-
\cancel{\gpgrade{ A_s \lr{ \Bb \cdot B_r } }{s - (r + 1)}}.
\end{aligned}
\end{equation}

The last term above is zero since we are selecting the \(s - r - 1\) grade element of a multivector with grades \(s - r + 1\) and \(s + r - 1\), which has no terms for \(r > 0\).  Now we can expand the \(A_s \Bb\) multivector product, for

\begin{dmath}\label{eqn:stokesTheoremGeometricAlgebra:1520}
A_s \cdot \lr{ \Bb \wedge B_r }
=
\gpgrade{ \lr{ A_s \cdot \Bb + A_s \wedge \Bb} B_r }{s - (r + 1)}.
\end{dmath}

The latter multivector (with the wedge product factor) above has grades \(s + 1 - r\) and \(s + 1 + r\), so this selection operator finds nothing.  This leaves

\begin{dmath}\label{eqn:stokesTheoremGeometricAlgebra:1540}
A_s \cdot \lr{ \Bb \wedge B_r }
=
\gpgrade{
\lr{ A_s \cdot \Bb } \cdot B_r
+ \lr{ A_s \cdot \Bb } \wedge B_r
}{s - (r + 1)}.
\end{dmath}

The first dot products term has grade \(s - 1 - r\) and is selected, whereas the wedge term has grade \(s - 1 + r \ne s - r - 1\) (for \(r > 0\)).  \(\qedmarker\)

%Next consider an expansion that we cannot do above, but require
} % answer

%      \section{Problem solutions}
%         \shipoutAnswer

   \chapter{Comparison of many traditional vector and GA identities}
      %
% Copyright � 2012 Peeter Joot.  All Rights Reserved.
% Licenced as described in the file LICENSE under the root directory of this GIT repository.
%

%
%
%\input{../peeter_prologue.tex}

\chapter{Comparison of many traditional vector and GA identities}
\index{identities}
\label{chap:gaWiki}


% does not work with _ character:
%%\blogpage{http://sites.google.com/site/peeterjoot/geometric-algebra/ga_wiki.pdf}
%%\date{ Oct 13, 2007 }
%\date{Oct 13, 2007.  gaWiki.tex}
%%\revisionInfo{\(RCSfile: gaWiki.tex,v \) Last \(Revision: 1.16 \) \(Date: 2009/10/22 02:07:20 \)}

\beginArtNoToc

\section{Three dimensional vector relationships vs N dimensional equivalents}

Here are some comparisons between standard \({\mathbb R}^3\) vector relations and their corresponding wedge and geometric product equivalents.  All the wedge and geometric product equivalents here are good for more than three dimensions, and some also for two.  In two dimensions the cross product is undefined even if what it describes (like torque) is a perfectly well defined in a plane without introducing an arbitrary normal vector outside of the space.

Many of these relationships only require the introduction of the wedge product to generalize, but since that may not be familiar to somebody with only a traditional background in vector algebra and calculus, some examples are given.

\subsection{wedge and cross products are antisymmetric}
\begin{equation}\label{eqn:gaWiki:20}
\begin{aligned}
\Bv \times \Bu = - (\Bu \times \Bv)
\end{aligned}
\end{equation}
\begin{equation}\label{eqn:gaWiki:40}
\begin{aligned}
\Bv \wedge \Bu = - (\Bu \wedge \Bv)
\end{aligned}
\end{equation}

\subsection{wedge and cross products are zero when identical}
\begin{equation}\label{eqn:gaWiki:60}
\begin{aligned}
\Bu \times \Bu = 0
\end{aligned}
\end{equation}
\begin{equation}\label{eqn:gaWiki:80}
\begin{aligned}
\Bu \wedge \Bu = 0
\end{aligned}
\end{equation}

\subsection{wedge and cross products are linear}

These are both linear in the first variable
\begin{equation}\label{eqn:gaWiki:100}
\begin{aligned}
(\Bv + \Bw) \times \Bw = \Bu \times \Bw + \Bv \times \Bw
\end{aligned}
\end{equation}
\begin{equation}\label{eqn:gaWiki:120}
\begin{aligned}
(\Bv + \Bw) \wedge \Bw = \Bu \wedge \Bw + \Bv \wedge \Bw
\end{aligned}
\end{equation}

and are linear in the second variable
\begin{equation}\label{eqn:gaWiki:140}
\begin{aligned}
\Bu \times (\Bv + \Bw)= \Bu \times \Bv + \Bu \times \Bw
\end{aligned}
\end{equation}
\begin{equation}\label{eqn:gaWiki:160}
\begin{aligned}
\Bu \wedge (\Bv + \Bw)= \Bu \wedge \Bv + \Bu \wedge \Bw
\end{aligned}
\end{equation}

\subsection{In general, cross product is not associative, but the wedge product is}
\begin{equation}\label{eqn:gaWiki:180}
\begin{aligned}
(\Bu \times \Bv) \times \Bw \neq \Bu \times (\Bv \times \Bw)
\end{aligned}
\end{equation}
\begin{equation}\label{eqn:gaWiki:200}
\begin{aligned}
(\Bu \wedge \Bv) \wedge \Bw = \Bu \wedge (\Bv \wedge \Bw)
\end{aligned}
\end{equation}

\subsection{Wedge and cross product relationship to a plane}
\(\Bu \times \Bv\) is perpendicular to plane containing \(\Bu\) and \(\Bv\).
\(\Bu \wedge \Bv\) is an oriented representation of the plane containing \(\Bu\) and \(\Bv\).

\subsection{norm of a vector}

The norm (length) of a vector is defined in terms of the dot product

\begin{equation}\label{eqn:gaWiki:220}
\begin{aligned}
 {\Vert \Bu \Vert}^2 = \Bu \cdot \Bu
\end{aligned}
\end{equation}

Using the geometric product this is also true, but this can be also be expressed more compactly as

\begin{equation}\label{eqn:gaWiki:240}
\begin{aligned}
{\Vert \Bu \Vert}^2 = {\Bu}^2
\end{aligned}
\end{equation}

This follows from the definition of the geometric product and the fact that a vector wedge product with itself is zero

\begin{equation}\label{eqn:gaWiki:260}
\begin{aligned}
 \Bu \, \Bu = \Bu \cdot \Bu + \Bu \wedge \Bu = \Bu \cdot \Bu
\end{aligned}
\end{equation}

\subsection{Lagrange identity}
\index{Lagrange identity}

In three dimensions the product of two vector lengths can be expressed in terms of the dot and cross products

\begin{equation}\label{eqn:gaWiki:280}
\begin{aligned}
{\Vert \Bu  \Vert}^2 {\Vert \Bv  \Vert}^2
=
({\Bu  \cdot \Bv })^2 + {\Vert \Bu  \times \Bv  \Vert}^2
\end{aligned}
\end{equation}

The corresponding generalization expressed using the geometric product is

\begin{equation}\label{eqn:gaWiki:300}
\begin{aligned}
{\Vert \Bu  \Vert}^2 {\Vert \Bv  \Vert}^2
= ({\Bu  \cdot \Bv })^2 - (\Bu  \wedge \Bv )^2
\end{aligned}
\end{equation}

This follows from by expanding the geometric product of a pair of vectors with its reverse

\begin{equation}\label{eqn:gaWiki:320}
\begin{aligned}
(\Bu  \Bv )(\Bv  \Bu )
= ({\Bu  \cdot \Bv } + {\Bu  \wedge \Bv }) ({\Bu  \cdot \Bv } - {\Bu  \wedge \Bv })
\end{aligned}
\end{equation}

\subsection{determinant expansion of cross and wedge products}
\index{wedge!determinant expansion}

\begin{equation}\label{eqn:gaWiki:340}
\begin{aligned}
\Bu \times \Bv = \sum_{i<j}{ \begin{vmatrix}u_i & u_j\\v_i & v_j\end{vmatrix}  {\Be}_i \times {\Be}_j }
\end{aligned}
\end{equation}
\begin{equation}\label{eqn:gaWiki:360}
\begin{aligned}
\Bu \wedge \Bv = \sum_{i<j}{ \begin{vmatrix}u_i & u_j\\v_i & v_j\end{vmatrix}  {\Be}_i \wedge {\Be}_j }
\end{aligned}
\end{equation}

Without justification or historical context, traditional linear algebra texts will often define the determinant as the first step of an elaborate sequence of definitions and theorems leading up to the solution of linear systems, Cramer's rule and matrix inversion.

An alternative treatment is to axiomatically introduce the wedge product, and then demonstrate that this can be used directly to solve linear systems.  This is shown below, and does not require sophisticated math skills to understand.

It is then possible to define determinants as nothing more than the coefficients of the wedge product in terms of "unit k-vectors" (\({\Be}_i \wedge {\Be}_j\) terms) expansions as above.

A one by one determinant is the coefficient of \(\Be _1\) for an \(\mathbb R^1\) 1-vector.

A two-by-two determinant is the coefficient of \(\Be _1 \wedge \Be _2\) for an \(\mathbb R^2\) bivector

A three-by-three determinant is the coefficient of \(\Be _1 \wedge \Be _2 \wedge \Be _3\) for an \(\mathbb R^3\) trivector

When linear system solution is introduced via the wedge product, Cramer's rule follows as a side effect, and there is no need to lead up to the end results with definitions of minors, matrices, matrix invertablity, adjoints, cofactors, Laplace expansions, theorems on determinant multiplication and row column exchanges, and so forth.

\subsection{Equation of a plane}
\index{plane!equation}

For the plane of all points \({\Br}\) through the plane passing through three independent points \({\Br}_0\), \({\Br}_1\), and \({\Br}_2\), the normal form of the equation is

\begin{equation}\label{eqn:gaWiki:380}
\begin{aligned}
(({\Br}_2 - {\Br}_0) \times ({\Br}_1 - {\Br}_0)) \cdot ({\Br} - {\Br}_0) = 0
\end{aligned}
\end{equation}

The equivalent wedge product equation is
\begin{equation}\label{eqn:gaWiki:400}
\begin{aligned}
({\Br}_2 - {\Br}_0) \wedge ({\Br}_1 - {\Br}_0) \wedge ({\Br} - {\Br}_0) = 0
\end{aligned}
\end{equation}

\subsection{Projective and rejective components of a vector}

For three dimensions the projective and rejective components of a vector with respect to an arbitrary non-zero unit vector, can be expressed in terms of the dot and cross product

\begin{equation}\label{eqn:gaWiki:420}
\begin{aligned}
\Bv = (\Bv \cdot \ucap)\ucap + \ucap \times (\Bv \times \ucap)
\end{aligned}
\end{equation}

For the general case the same result can be written in terms of the dot and wedge product and the geometric product of that and the unit vector

\begin{equation}\label{eqn:gaWiki:440}
\begin{aligned}
\Bv = (\Bv \cdot \ucap)\ucap + (\Bv \wedge \ucap) \ucap
\end{aligned}
\end{equation}

It is also worthwhile to point out that this result can also be expressed using right or left vector division as defined by the geometric product

\begin{equation}\label{eqn:gaWiki:460}
\begin{aligned}
\Bv = (\Bv \cdot \Bu)\frac{1}{\Bu} + (\Bv \wedge \Bu) \frac{1}{\Bu}
\end{aligned}
\end{equation}
\begin{equation}\label{eqn:gaWiki:480}
\begin{aligned}
\Bv = \frac{1}{\Bu}(\Bu \cdot \Bv) + \frac{1}{\Bu}(\Bu \wedge \Bv)
\end{aligned}
\end{equation}

\subsection{Area (squared) of a parallelogram is norm of cross product}
\index{parallelogram!area}

\begin{equation}\label{eqn:gaWiki:500}
\begin{aligned}
A^2 = {\Vert \Bu \times \Bv \Vert}^2 = \sum_{i<j}{\begin{vmatrix}u_i & u_j\\v_i & v_j\end{vmatrix}}^2
\end{aligned}
\end{equation}

and is the negated square of a wedge product
\begin{equation}\label{eqn:gaWiki:520}
\begin{aligned}
A^2 = -(\Bu \wedge \Bv)^2 = \sum_{i<j}{\begin{vmatrix}u_i & u_j\\v_i & v_j\end{vmatrix}}^2
\end{aligned}
\end{equation}

Note that this squared bivector is a geometric product.

\subsection{Angle between two vectors}
\index{vectors!angle between}

\begin{equation}\label{eqn:gaWiki:540}
\begin{aligned}
({\sin \theta})^2 = \frac{{\Vert \Bu \times \Bv \Vert}^2}{{\Vert \Bu \Vert}^2 {\Vert \Bv \Vert}^2}
\end{aligned}
\end{equation}
\begin{equation}\label{eqn:gaWiki:560}
\begin{aligned}
({\sin \theta})^2 = -\frac{(\Bu \wedge \Bv)^2}{{ \Bu }^2 { \Bv }^2}
\end{aligned}
\end{equation}

\subsection{Volume of the parallelepiped formed by three vectors}
\index{parallelepiped!volume}

\begin{equation}\label{eqn:gaWiki:580}
\begin{aligned}
V^2 = {\Vert (\Bu \times \Bv) \cdot \Bw \Vert}^2
= {
\begin{vmatrix}
u_1 & u_2 & u_3 \\
v_1 & v_2 & v_3 \\
w_1 & w_2 & w_3 \\
\end{vmatrix}
}^2
\end{aligned}
\end{equation}

\begin{equation}\label{eqn:gaWiki:600}
\begin{aligned}
V^2 = -(\Bu \wedge \Bv \wedge \Bw)^2
= -\left(\sum_{i<j<k}
\begin{vmatrix}
u_i & u_j & u_k \\
v_i & v_j & v_k \\
w_i & w_j & w_k \\
\end{vmatrix}
\ecap_i \wedge \ecap_j \wedge \ecap_k
\right)^2
= \sum_{i<j<k}
{
\begin{vmatrix}
u_i & u_j & u_k \\
v_i & v_j & v_k \\
w_i & w_j & w_k \\
\end{vmatrix}
}^2
\end{aligned}
\end{equation}


\section{Some properties and examples}

Some fundamental geometric algebra manipulations will be provided below, showing how this vector product can be used in calculation of projections, area, and rotations.  How some of these tie together and correlate concepts from other branches of mathematics, such as complex numbers, will also be shown.

In some cases these examples provide details used above in the cross product and geometric product comparisons.

\subsection{Inversion of a vector}
\index{vector!inversion}

One of the powerful properties of the Geometric product is that it provides the capability to express the inverse of a non-zero vector.  This is expressed by:

\begin{equation}\label{eqn:gaWiki:620}
\begin{aligned}
{\Ba}^{-1} = \frac{\Ba}{\Ba \Ba} = \frac{\Ba}{{\Vert \Ba \Vert}^2}.
\end{aligned}
\end{equation}

\subsection{dot and wedge products defined in terms of the geometric product}

Given a definition of the geometric product in terms of the dot and wedge products, adding and subtracting \(\Ba  \Bb \) and \(\Bb  \Ba \) demonstrates that the dot and wedge product of two vectors can also be defined in terms of the geometric product

\subsection{The dot product}
\index{dot product}

\begin{equation}\label{eqn:gaWiki:640}
\begin{aligned}
\Ba \cdot\Bb  = \frac{1}{2}(\Ba \Bb  + \Bb \Ba )
\end{aligned}
\end{equation}

This is the symmetric component of the geometric product.  When two vectors are colinear the geometric and dot products of those vectors are equal.

As a motivation for the dot product it is normal to show that this quantity occurs in the solution of the length of a general triangle where the third side is the vector sum of the first and second sides \(\Bc  = \Ba  + \Bb \).

\begin{equation}\label{eqn:gaWiki:660}
\begin{aligned}
{\Vert \Bc  \Vert}^2 = \sum_{i}(a_i + b_i)^2 = {\Vert \Ba  \Vert}^2 + {\Vert \Bb  \Vert}^2 + 2 \sum_{i}a_i b_i
\end{aligned}
\end{equation}

The last sum is then given the name the dot product and other properties of this quantity are then shown (projection, angle between vectors, ...).

This can also be expressed using the geometric product

\begin{equation}\label{eqn:gaWiki:680}
\begin{aligned}
\Bc ^2 = (\Ba  + \Bb )(\Ba  + \Bb ) = \Ba ^2 + \Bb ^2 + (\Ba \Bb  + \Bb \Ba )
\end{aligned}
\end{equation}

By comparison, the following equality exists

\begin{equation}\label{eqn:gaWiki:700}
\begin{aligned}
\sum_{i}a_i b_i = \frac{1}{2}(\Ba \Bb  + \Bb \Ba )
\end{aligned}
\end{equation}

Without requiring expansion by components one can define the dot product exclusively in terms of the geometric product due to its properties of contraction, distribution and associativity.  This is arguably a more natural way to define the geometric product.  Addition of two similar terms is not immediately required, especially since one of those terms is the wedge product which may also be unfamiliar.

\subsection{The wedge product}
\index{wedge product}

\begin{equation}\label{eqn:gaWiki:720}
\begin{aligned}
\Ba \wedge\Bb  = \frac{1}{2}(\Ba \Bb  - \Bb \Ba )
\end{aligned}
\end{equation}

This is the antisymmetric component of the geometric product.  When two vectors are orthogonal the geometric and wedge products of those vectors are equal.

Switching the order of the vectors negates this antisymmetric geometric product component, and contraction property shows that this is zero if the vectors are equal.  These are the defining properties of the wedge product.

\subsection{Note on symmetric and antisymmetric dot and wedge product formulas}
\index{dot product!symmetric sum}
\index{wedge product!antisymmetric sum}

A generalization of the dot product that allows computation of the component of a vector "in the direction" of a plane (bivector), or other k-vectors can be found below.  Since the signs change depending on the grades of the terms being multiplied, care is required with the formulas above to ensure that they are only used for a pair of vectors.

\subsection{Reversing multiplication order.  Dot and wedge products compared to the real and imaginary parts of a complex number}

Reversing the order of multiplication of two vectors, has the effect of the inverting the sign of just the wedge product term of the product.

It is not a coincidence that this is a similar operation to the conjugate operation of complex numbers.

The reverse of a product is written in the following fashion

\begin{equation}\label{eqn:gaWiki:740}
\begin{aligned}
{\Bb  \Ba } = ({\Ba  \Bb })^\dagger
\end{aligned}
\end{equation}
\begin{equation}\label{eqn:gaWiki:760}
\begin{aligned}
{\Bc  \Bb  \Ba } = ({\Ba  \Bb  \Bc })^\dagger
\end{aligned}
\end{equation}

Expressed this way the dot and wedge products are

\begin{equation}\label{eqn:gaWiki:780}
\begin{aligned}
\Ba \cdot\Bb  = \frac{1}{2}(\Ba \Bb  + ({\Ba  \Bb })^\dagger)
\end{aligned}
\end{equation}

This is the symmetric component of the geometric product.  When two vectors are colinear the geometric and dot products of those vectors are equal.

\begin{equation}\label{eqn:gaWiki:800}
\begin{aligned}
\Ba \wedge\Bb  = \frac{1}{2}(\Ba \Bb  - ({\Ba  \Bb })^\dagger)
\end{aligned}
\end{equation}

These symmetric and antisymmetric pairs, the dot and wedge products extract the scalar and bivector components of a geometric product in the same fashion as the real and imaginary components of a complex number are also extracted by its symmetric and antisymmetric components

\begin{equation}\label{eqn:gaWiki:820}
\begin{aligned}
\mathop{Re}(z) = \frac{1}{2}(z + \overbar{z})
\end{aligned}
\end{equation}
\begin{equation}\label{eqn:gaWiki:840}
\begin{aligned}
\mathop{Im}(z) = \frac{1}{2}(z - \overbar{z})
\end{aligned}
\end{equation}

This extraction of components also applies to higher order geometric product terms.  For example

\begin{equation}\label{eqn:gaWiki:860}
\begin{aligned}
\Ba \wedge\Bb \wedge \Bc
= \frac{1}{2}(\Ba \Bb \Bc  - ({\Ba  \Bb } \Bc )^\dagger)
= \frac{1}{2}(\Bb \Bc \Ba  - ({\Bb  \Bc } \Ba )^\dagger)
= \frac{1}{2}(\Bc \Ba \Bb  - ({\Bc  \Ba } \Bb )^\dagger)
\end{aligned}
\end{equation}

\subsection{Orthogonal decomposition of a vector}

Using the \textAndIndex{Gram-Schmidt} process a single vector can be decomposed into two components with respect to a reference vector, namely the projection onto a unit vector in a reference direction, and the difference between the vector and that projection.

With, \( \ucap = \Bu / {\Vert \Bu \Vert}\), the projection of \(\Bv\) onto \( \ucap\) is

\begin{equation}\label{eqn:gaWiki:880}
\begin{aligned}
 \mathrm{Proj}_{\ucap}\,\Bv  = \ucap (\ucap \cdot \Bv)
\end{aligned}
\end{equation}

Orthogonal to that vector is the difference, designated the rejection,

\begin{equation}\label{eqn:gaWiki:900}
\begin{aligned}
 \Bv - \ucap (\ucap \cdot \Bv) = \frac{1}{{\Vert \Bu \Vert}^2} ( {\Vert \Bu \Vert}^2 \Bv - \Bu (\Bu \cdot \Bv))
\end{aligned}
\end{equation}

The rejection can be expressed as a single geometric algebraic product in a few different ways

\begin{equation}\label{eqn:gaWiki:920}
\begin{aligned}
 \frac{ \Bu }{{\Bu}^2} ( \Bu \Bv - \Bu \cdot \Bv)
= \frac{1}{\Bu} ( \Bu \wedge \Bv )
= \ucap ( \ucap \wedge \Bv )
= ( \Bv \wedge \ucap ) \ucap
\end{aligned}
\end{equation}

The similarity in form between between the projection and the rejection is notable.  The sum of these recovers the original vector

\begin{equation}\label{eqn:gaWiki:940}
\begin{aligned}%\label{eqn:gaWiki:orthoD}
 \Bv = \ucap (\ucap \cdot \Bv) + \ucap ( \ucap \wedge \Bv )
\end{aligned}
\end{equation}

Here the projection is in its customary vector form.  An alternate formulation is possible that puts the projection in a form that differs from the usual vector formulation

\begin{equation}\label{eqn:gaWiki:960}
\begin{aligned}
 \Bv
= \frac{1}{\Bu} (\Bu \cdot \Bv) + \frac{1}{\Bu} ( \Bu \wedge \Bv )
= (\Bv \cdot \Bu) \frac{1}{\Bu}  + ( \Bv \wedge \Bu ) \frac{1}{\Bu}
\end{aligned}
\end{equation}

\subsection{A quicker way to the end result}

Working backwards from the end result, it can be observed that this orthogonal decomposition result can in fact follow more directly from the definition of the geometric product itself.

\begin{equation}\label{eqn:gaWiki:980}
\begin{aligned}
\Bv = \ucap \ucap \Bv
= \ucap (\ucap \cdot \Bv + \ucap \wedge \Bv )
\end{aligned}
\end{equation}

With this approach, the original geometrical consideration is not necessarily obvious, but it is a much quicker way to get at the same algebraic result.

However, the hint that one can work backwards, coupled with the knowledge that the wedge product can be used to solve sets of linear equations,
\footnote{
http://www.grassmannalgebra.info/grassmannalgebra/book/bookpdf/TheExteriorProduct.pdf}
the problem of orthogonal decomposition can be posed directly,

Let \(\Bv = a \Bu + \Bx\), where \(\Bu \cdot \Bx = 0\).  To discard the portions of \(\Bv\) that are colinear with \(\Bu\), take the wedge product

\begin{equation}\label{eqn:gaWiki:1000}
\begin{aligned}
\Bu \wedge \Bv = \Bu \wedge (a \Bu + \Bx) = \Bu \wedge \Bx
\end{aligned}
\end{equation}

Here the geometric product can be employed

\begin{equation}\label{eqn:gaWiki:1020}
\begin{aligned}
\Bu \wedge \Bv = \Bu \wedge \Bx = \Bu \Bx - \Bu \cdot \Bx = \Bu \Bx
\end{aligned}
\end{equation}

Because the geometric product is invertible, this can be solved for x

\begin{equation}\label{eqn:gaWiki:1040}
\begin{aligned}
\Bx = \frac{1}{\Bu}(\Bu \wedge \Bv)
\end{aligned}
\end{equation}

The same techniques can be applied to similar problems, such as calculation of the component of a vector in a plane and perpendicular to the plane.

\subsection{Area of parallelogram spanned by two vectors}
\index{parallelogram!area}

\imageFigure{../../physicsplay/figures/gabook/parallelogramArea}{parallelogramArea}{fig:parallelogramArea}{0.4}

As depicted in \cref{fig:parallelogramArea}, one can see that the area of a parallelogram spanned by two vectors is computed from the base times height.  In the figure \(\Bu\) was picked as the base, with length \(\Norm{\Bu}\).  Designating the second vector \(\Bv\), we want the component of \(\Bv\) perpendicular to \(\ucap\) for the height.  An orthogonal decomposition of \(\Bv\) into directions parallel and perpendicular to \(\ucap\) can be performed in two ways.

\begin{equation}\label{eqn:gaWiki:1060}
\begin{aligned}
\Bv &= \Bv \ucap \ucap = (\Bv \cdot \ucap) \ucap + (\Bv \wedge \ucap) \ucap \\
    &= \ucap \ucap \Bv = \ucap (\ucap \cdot \Bv) + \ucap (\ucap \wedge \Bv)
\end{aligned}
\end{equation}

The height is the length of the perpendicular component expressed using the wedge as either \(\ucap (\ucap \wedge \Bv)\) or \((\Bv \wedge \ucap) \ucap\).

Multiplying base times height we have the parallelogram area

\begin{equation}\label{eqn:gaWiki:1080}
\begin{aligned}
A(\Bu,\Bv)
&= \Vert \Bu \Vert \Vert \ucap ( \ucap \wedge \Bv ) \Vert \\
&= \Vert \ucap ( \Bu \wedge \Bv ) \Vert \\
\end{aligned}
\end{equation}

Since the squared length of an Euclidean vector is the geometric square of that vector, we can compute the squared area of this parallogram by squaring this single scaled vector

\begin{equation}\label{eqn:gaWiki:1100}
\begin{aligned}
A^2 &= (\ucap ( \Bu \wedge \Bv ) )^2
\end{aligned}
\end{equation}

Utilizing both encodings of the perpendicular to \(\ucap\) component of \(\Bv\) computed above we have for the squared area

\begin{equation}\label{eqn:gaWiki:1120}
\begin{aligned}
A^2
&= (\ucap( \Bu \wedge {\Bv} ) )^2 \\
&= (( \Bv \wedge {\Bu} ) \ucap) (\ucap ( {\Bu} \wedge \Bv )) \\
&= ( \Bv \wedge \Bu ) ( \Bu \wedge \Bv ) \\
\end{aligned}
\end{equation}

Since \(\Bu \wedge \Bv = -\Bv \wedge \Bu\), we have finally

\begin{equation}\label{eqn:gaWiki:1140}
\begin{aligned}
A^2 = -( \Bu \wedge \Bv )^2
\end{aligned}
\end{equation}

There are a few things of note here.  One is that the parallelogram area can easily be expressed in terms of the square of a bivector.  Another is that the square of a bivector has the same property as a purely imaginary number, a negative square.

It can also be noted that a vector lying completely within a plane anticommutes with the bivector for that plane.  More generally components of vectors that lie within a plane commute with the bivector for that plane while the perpendicular components of that vector commute.  These commutation or anticommutation properties depend both on the vector and the grade of the object that one attempts to commute it with (these properties lie behind the generalized definitions of the dot and wedge product to be seen later).

% SCOTT:
% - Section 3.2.9. This is more of a comment. The commuting of the geometric product in the second line of the equation for A^2 uses the idea that u_hat is in the plane of the bivector, therefore the wedge product is zero. It would be nice if this reasoning were given. In fact, I think it would be very benificial if there were a section near the beginning that pretty much laid out what is possible with the geometric product and when. Maybe a table of sorts with columns signifying that when you fit certain criteria, orthogonal/parallel/in a plane/etc, that various properties like commutation/associativity/etc work.

\subsection{Expansion of a bivector and a vector rejection in terms of the standard basis}
\index{rejection}

If a vector is factored directly into projective and rejective terms using the geometric product \(\Bv = \frac{1}{\Bu}( \Bu \cdot \Bv + \Bu \wedge \Bv)\), then it is not necessarily obvious that the rejection term, a product of vector and bivector is even a vector.  Expansion of the vector bivector product in terms of the standard basis vectors has the following form

Let
\begin{equation}\label{eqn:gaWiki:1160}
\begin{aligned}
\Br
= \frac{1}{\Bu} ( \Bu \wedge \Bv )
= \frac{\Bu}{\Bu^2} ( \Bu \wedge \Bv )
= \frac{1}{{\Vert \Bu \Vert}^2} \Bu ( \Bu \wedge \Bv )
\end{aligned}
\end{equation}

It can be shown that
\begin{equation}\label{eqn:gaWiki:1180}
\begin{aligned}
\Br = \frac{1}{{\Vert{\Bu}\Vert}^2} \sum_{i<j}\begin{vmatrix}u_i & u_j\\v_i & v_j\end{vmatrix}
\begin{vmatrix}u_i & u_j\\ {\Be}_i & {\Be}_j\end{vmatrix}
\end{aligned}
\end{equation}

(a result that can be shown more easily straight from \(\Br = \Bv - \ucap (\ucap \cdot \Bv)\)).

The rejective term is perpendicular to \(\Bu\), since
$\begin{vmatrix}
u_i & u_j\\ u_i & u_j
\end{vmatrix}
 = 0$
implies \(\Br \cdot \Bu = \Bzero\).

The magnitude of \(\Br\), is

\begin{equation}\label{eqn:gaWiki:1200}
\begin{aligned}
{\Vert \Br \Vert}^2 = \Br \cdot \Bv = \frac{1}{{\Vert{\Bu}\Vert}^2} \sum_{i<j}\begin{vmatrix}u_i & u_j\\v_i & v_j\end{vmatrix}^2
\end{aligned}
\end{equation}.

So, the quantity

\begin{equation}\label{eqn:gaWiki:1220}
\begin{aligned}
{\Vert \Br \Vert}^2 {\Vert{\Bu}\Vert}^2 = \sum_{i<j}\begin{vmatrix}u_i & u_j\\v_i & v_j\end{vmatrix}^2
\end{aligned}
\end{equation}

is the squared area of the parallelogram formed by \(\Bu\) and \(\Bv\).

It is also noteworthy that the bivector can be expressed as

\begin{equation}\label{eqn:gaWiki:1240}
\begin{aligned}
\Bu \wedge \Bv = \sum_{i<j}{ \begin{vmatrix}u_i & u_j\\v_i & v_j\end{vmatrix}  {\Be}_i \wedge {\Be}_j }
\end{aligned}
\end{equation}.

Thus is it natural, if one considers each term \({\Be}_i \wedge {\Be}_j\) as a basis vector of the bivector space, to define the (squared) "length" of that bivector as the (squared) area.

Going back to the geometric product expression for the length of the rejection \(\frac{1}{\Bu} ( \Bu \wedge \Bv )\) we see that the length of the quotient, a vector, is in this case is the "length" of the bivector divided by the length of the divisor.

This may not be a general result for the length of the product of two \(k\)-vectors, however it is a result that may help build some intuition about the significance of the algebraic operations.  Namely,

When a vector is divided out of the plane (parallelogram span) formed from it and another vector, what remains is the perpendicular component of the remaining vector, and its length is the planar area divided by the length of the vector that was divided out.

\subsection{Projection and rejection of a vector onto and perpendicular to a plane}
\index{plane!projection}
\index{plane!rejection}

Like vector projection and rejection, higher dimensional analogs of that calculation are also possible using the geometric product.

As an example, one can calculate the component of a vector perpendicular to a plane and the projection of that vector onto the plane.

Let \(\Bw = a \Bu + b \Bv + \Bx\), where \(\Bu \cdot \Bx = \Bv \cdot \Bx = 0\).  As above, to discard the portions of \(\Bw\) that are colinear with \(\Bu\) or \(\Bu\), take the wedge product

\begin{equation}\label{eqn:gaWiki:1260}
\begin{aligned}
\Bw \wedge \Bu \wedge \Bv = (a \Bu + b \Bv + \Bx) \wedge \Bu \wedge \Bv = \Bx \wedge \Bu \wedge \Bv
\end{aligned}
\end{equation}

Having done this calculation with a vector projection, one can guess that this quantity equals \(\Bx (\Bu \wedge \Bv)\).  One can also guess there is a vector and bivector dot product like quantity such that the allows the calculation of the component of a vector that is in the "direction of a plane".  Both of these guesses are correct, and the validating these facts is worthwhile.  However, skipping ahead slightly, this to be proved fact allows for a nice closed form solution of the vector component outside of the plane:

\begin{equation}\label{eqn:gaWiki:1280}
\begin{aligned}
\Bx
= (\Bw \wedge \Bu \wedge \Bv)\frac{1}{\Bu \wedge \Bv}
= \frac{1}{\Bu \wedge \Bv}(\Bu \wedge \Bv  \wedge \Bw)
\end{aligned}
\end{equation}

Notice the similarities between this planar rejection result a the vector rejection result.  To calculation the component of a vector outside of a plane we take the volume spanned by three vectors (trivector) and "divide out" the plane.

Independent of any use of the geometric product it can be shown that this rejection in terms of the standard basis is

\begin{equation}\label{eqn:gaWiki:1300}
\begin{aligned}
\Bx = \frac{1}{(A_{u,v})^2} \sum_{i<j<k}
\begin{vmatrix}w_i & w_j & w_k \\u_i & u_j & u_k \\v_i & v_j & v_k \\\end{vmatrix}
\begin{vmatrix}u_i & u_j & u_k \\v_i & v_j & v_k \\ {\Be}_i & {\Be}_j & {\Be}_k \\ \end{vmatrix}
\end{aligned}
\end{equation}

Where

\begin{equation}\label{eqn:gaWiki:1320}
\begin{aligned}
(A_{u,v})^2
= \sum_{i<j} \begin{vmatrix}u_i & u_j\\v_i & v_j\end{vmatrix}
= -(\Bu \wedge \Bv)^2
\end{aligned}
\end{equation}

is the squared area of the parallelogram formed by \(\Bu\), and \(\Bv\).

The (squared) magnitude of \(\Bx\) is

\begin{equation}\label{eqn:gaWiki:1340}
\begin{aligned}
{\Vert \Bx \Vert}^2 =
\Bx \cdot \Bw =
\frac{1}{(A_{u,v})^2} \sum_{i<j<k}
{\begin{vmatrix}w_i & w_j & w_k \\u_i & u_j & u_k \\v_i & v_j & v_k \\\end{vmatrix}}^2
\end{aligned}
\end{equation}

Thus, the (squared) volume of the parallelepiped (base area times perpendicular height) is

\begin{equation}\label{eqn:gaWiki:1360}
\begin{aligned}
\sum_{i<j<k}
{\begin{vmatrix}w_i & w_j & w_k \\u_i & u_j & u_k \\v_i & v_j & v_k \\\end{vmatrix}}^2
\end{aligned}
\end{equation}

Note the similarity in form to the w,u,v trivector itself

\begin{equation}\label{eqn:gaWiki:1380}
\begin{aligned}
\sum_{i<j<k}
{\begin{vmatrix}w_i & w_j & w_k \\u_i & u_j & u_k \\v_i & v_j & v_k \\\end{vmatrix}} {\Be}_i \wedge {\Be}_j \wedge {\Be}_k
\end{aligned}
\end{equation}

which, if you take the set of \({\Be}_i \wedge {\Be}_j \wedge {\Be}_k\) as a basis for the trivector space, suggests this is the natural way to define the length of a trivector.  Loosely speaking the length of a vector is a length, length of a bivector is area, and the length of a trivector is volume.

\subsection{Product of a vector and bivector.  Defining the "dot product" of a plane and a vector}

In order to justify the normal to a plane result above, a general examination of the product of a vector and bivector is required.  Namely,

\begin{equation}\label{eqn:gaWiki:1400}
\begin{aligned}
\Bw (\Bu \wedge \Bv)
= \sum_{i,j<k}w_i {\Be}_i {\begin{vmatrix}u_j & u_k \\v_j & v_k \\\end{vmatrix}} {\Be}_j \wedge {\Be}_k
\end{aligned}
\end{equation}

This has two parts, the vector part where \(i=j\) or \(i=k\), and the trivector parts where no indices equal.  After some index summation trickery, and grouping terms and so forth, this is


\begin{equation}\label{eqn:gaWiki:1420}
\begin{aligned}
\Bw (\Bu \wedge \Bv) =
\sum_{i<j}(w_i {\Be}_j
- w_j {\Be}_i )
{\begin{vmatrix}u_i & u_j \\v_i & v_j \\\end{vmatrix}}
+
\sum_{i<j<k}
{\begin{vmatrix}w_i & w_j & w_k \\ u_i & u_j & u_k \\v_i & v_j & v_k \\\end{vmatrix}}
{\Be}_i \wedge {\Be}_j \wedge {\Be}_k
\end{aligned}
\end{equation}

The trivector term is \(\Bw \wedge \Bu \wedge \Bv\).  Expansion of \((\Bu \wedge \Bv) \Bw\) yields the same trivector term.  This is the completely symmetric part, and the vector term is negated.
Like the geometric product of two vectors, this geometric product can be grouped into symmetric and antisymmetric parts, one of which is a pure k-vector.  In analogy the antisymmetric part of this product can be called a generalized dot product, and is roughly speaking the dot product of a "plane" (bivector), and a vector.

The properties of this generalized dot product remain to be explored, but first here is a summary of the notation

\begin{equation}\label{eqn:gaWiki:1440}
\begin{aligned}
\Bw (\Bu \wedge \Bv) = \Bw \cdot (\Bu \wedge \Bv) + \Bw \wedge \Bu \wedge \Bv
\end{aligned}
\end{equation}

\begin{equation}\label{eqn:gaWiki:1460}
\begin{aligned}
(\Bu \wedge \Bv) \Bw = - \Bw \cdot (\Bu \wedge \Bv) + \Bw \wedge \Bu \wedge \Bv
\end{aligned}
\end{equation}

\begin{equation}\label{eqn:gaWiki:1480}
\begin{aligned}
\Bw \wedge \Bu \wedge \Bv = \frac{1}{2}(\Bw (\Bu \wedge \Bv) + (\Bu \wedge \Bv) \Bw)
\end{aligned}
\end{equation}

\begin{equation}\label{eqn:gaWiki:1500}
\begin{aligned}
\Bw \cdot (\Bu \wedge \Bv) = \frac{1}{2}(\Bw (\Bu \wedge \Bv) - (\Bu \wedge \Bv) \Bw)
\end{aligned}
\end{equation}

Let \(\Bw = \Bx + \By\), where \(\Bx = a \Bu + b \Bv\), and \(\By \cdot \Bu = \By \cdot \Bv = \Bzero\).  Expressing \(\Bw\) and the \(\Bu \wedge \Bv\), products in terms of these components is

\begin{equation}\label{eqn:gaWiki:1520}
\begin{aligned}
\Bw (\Bu \wedge \Bv) = \Bx (\Bu \wedge \Bv) + \By (\Bu \wedge \Bv)
=
\Bx \cdot (\Bu \wedge \Bv) + \By \cdot (\Bu \wedge \Bv) + \By \wedge \Bu \wedge \Bv
\end{aligned}
\end{equation}

With the conditions and definitions above, and some manipulation, it can be shown that the term \(\By \cdot (\Bu \wedge \Bv) = \Bzero\), which then justifies the previous solution of the normal to a plane problem.  Since the vector term of the vector bivector product the name dot product is zero
when the vector is perpendicular to the plane (bivector), and this vector, bivector "dot product" selects only the components that are in the plane, so in analogy to the vector-vector dot product this name itself is justified by more than the fact this is the non-wedge product term of the geometric vector-bivector product.

\subsection{Complex numbers}
\index{complex numbers}
There is a one to one correspondence between the geometric product of two \(\mathbb{R}^2\) vectors and the field of complex numbers.

Writing, a vector in terms of its components, and left multiplying by the unit vector \({\Be}_1\) yields

\begin{equation}\label{eqn:gaWiki:1540}
\begin{aligned}
 Z = {\Be}_1 \BP = {\Be}_1 ( x {\Be}_1 + y {\Be}_2)
= x (1) + y ({\Be}_1 {\Be}_2)
= x (1) + y ({\Be}_1 \wedge {\Be}_2)
\end{aligned}
\end{equation}

The unit scalar and unit bivector pair \(1, {\Be}_1 \wedge {\Be}_2\) can be considered an alternate basis for a two dimensional vector space.  This alternate vector representation is closed with respect to the geometric product

\begin{equation}\label{eqn:gaWiki:1560}
\begin{aligned}
 Z_1 Z_2
&= {\Be}_1 ( x_1 {\Be}_1 + y_1 {\Be}_2) {\Be}_1 ( x_2 {\Be}_1 + y_2 {\Be}_2) \\
&= ( x_1 + y_1 {\Be}_1 {\Be}_2) ( x_2 + y_2 {\Be}_1 {\Be}_2) \\
&= x_1 x_2 + y_1 y_2 ({\Be}_1 {\Be}_2) {\Be}_1 {\Be}_2) \\
+ (x_1 y_2 + x_2 y_1) {\Be}_1 {\Be}_2 \\
\end{aligned}
\end{equation}

This closure can be observed after calculation of the square of the unit bivector above, a quantity

\begin{equation}\label{eqn:gaWiki:1580}
\begin{aligned}
({\Be}_1 \wedge {\Be}_2)^2 = {\Be}_1 {\Be}_2 {\Be}_1 {\Be}_2 = - {\Be}_1 {\Be}_1 {\Be}_2 {\Be}_2 = -1
\end{aligned}
\end{equation}

that has the characteristics of the complex number \(i^2 = -1\).

This fact allows the simplification of the product above to

\begin{equation}\label{eqn:gaWiki:1600}
\begin{aligned}
Z_1 Z_2
= (x_1 x_2 - y_1 y_2) + (x_1 y_2 + x_2 y_1) ({\Be}_1 \wedge {\Be}_2)
\end{aligned}
\end{equation}

Thus what is traditionally the defining, and arguably arbitrary seeming, rule of complex number multiplication, is found to follow naturally from the higher order structure of the geometric product, once that is applied to a two dimensional vector space.

It is also informative to examine how the length of a vector can be represented in terms of a complex number.  Taking the square of the length

\begin{equation}\label{eqn:gaWiki:1620}
\begin{aligned}
\BP \cdot \BP &= ( x {\Be}_1 + y {\Be}_2) \cdot ( x {\Be}_1 + y {\Be}_2) \\
&= ({\Be}_1 Z) {\Be}_1 Z \\
&= (( x  - y {\Be}_1 {\Be}_2) {\Be}_1) {\Be}_1 Z \\
&= ( x  - y ({\Be}_1 \wedge {\Be}_2)) Z \\
\end{aligned}
\end{equation}

This right multiplication of a vector with \({\Be}_1\), is named the conjugate

\begin{equation}\label{eqn:gaWiki:1640}
\begin{aligned}
\overline{Z} = x  - y ({\Be}_1 \wedge {\Be}_2)
\end{aligned}
\end{equation}

And with that definition, the length of the original vector can be expressed as

\begin{equation}\label{eqn:gaWiki:1660}
\begin{aligned}
\BP \cdot \BP = \overline{Z}Z
\end{aligned}
\end{equation}

This is also a natural definition of the length of a complex number, given the fact that the complex numbers can be considered an isomorphism with the two dimensional Euclidean vector space.

\subsection{Rotation in an arbitrarily oriented plane}
\index{plane!rotation}

A point \(\BP\), of radius \(\Br\), located at an angle \(\theta\) from the vector \(\ucap\) in the direction from \(\Bu\) to \(\Bv\), can be expressed as

\begin{equation}\label{eqn:gaWiki:1680}
\begin{aligned}
\BP = r( \ucap \cos{\theta} +
\frac{\ucap (\ucap \wedge \Bv)}{\Vert \ucap (\ucap \wedge \Bv) \Vert}  \sin{\theta})
=
r \ucap
( \cos{\theta} +
\frac{(\Bu \wedge \Bv)}{\Vert \ucap (\Bu \wedge \Bv) \Vert} \sin{\theta})
\end{aligned}
\end{equation}

Writing \( {\BI}_{\Bu ,\Bv } = \frac{\Bu \wedge \Bv}{\Vert \ucap (\Bu \wedge \Bv) \Vert} \), the square of this bivector has the property \({\BI _{\Bu ,\Bv }}^2 = -1 \) of the imaginary unit complex number.

This allows the point to be specified as a complex exponential

\begin{equation}\label{eqn:gaWiki:1700}
\begin{aligned}
= \ucap r ( \cos\theta + \BI _{\Bu ,\Bv } \sin\theta )
= \ucap r \exp( \BI _{\Bu ,\Bv } \theta )
\end{aligned}
\end{equation}

Complex numbers could be expressed in terms of the \(\mathbb R^2\)unit bivector \({\Be}_1 \wedge {\Be}_2\).  However this isomorphism really only requires a pair of linearly independent vectors in a plane (of arbitrary dimension).

\subsection{Quaternions}
\index{quaternion}

Similar to complex numbers the geometric product of two \(\mathbb{R}^3\) vectors can be used to define quaternions.  Pre and Post multiplication with \({\Be}_1{\Be}_2{\Be}_3\) can be used to express a vector in terms of the quaternion unit numbers \(i, j, k\), as well as describe all the properties of those numbers.

\subsection{Cross product as outer product}

%The cross product of traditional vector algebra (on \(\mathbb{R}^3\)) find its place in geometric algebra \(\calG_3\)

Cross product can be written as a scaled outer product

\begin{equation}\label{eqn:gaWiki:1720}
\begin{aligned}
\Ba \times\Bb  = -i(\Ba \wedge\Bb )
\end{aligned}
\end{equation}

\begin{equation}\label{eqn:gaWiki:1740}
\begin{aligned}
i^2 &= ({\Be}_1{\Be}_2{\Be}_3)^2 \\
&= {\Be}_1{\Be}_2{\Be}_3{\Be}_1{\Be}_2{\Be}_3 \\
&= -{\Be}_1{\Be}_2{\Be}_1{\Be}_3{\Be}_2{\Be}_3 \\
&= {\Be}_1{\Be}_1{\Be}_2{\Be}_3{\Be}_2{\Be}_3 \\
&= -{\Be}_3{\Be}_2{\Be}_2{\Be}_3 \\
&= -1
\end{aligned}
\end{equation}

The equivalence of the \(\mathbb{R}^3\) cross product and the wedge product expression above can be confirmed by direct multiplication of \(-i = -{\Be}_1{\Be}_2{\Be}_3\) with a determinant expansion of the wedge product

\begin{equation}\label{eqn:gaWiki:1760}
\begin{aligned}
\Bu \wedge \Bv = \sum_{1<=i<j<=3}(u_i v_j - v_i u_j) {\Be}_i \wedge {\Be}_j
= \sum_{1<=i<j<=3}(u_i v_j - v_i u_j) {\Be}_i {\Be}_j
\end{aligned}
\end{equation}

%%\EndArticle
%\EndNoBibArticle

   \chapter{Cramer's rule}
      %
% Copyright � 2012 Peeter Joot.  All Rights Reserved.
% Licenced as described in the file LICENSE under the root directory of this GIT repository.
%

%
%
%\chapter{Cramer's rule}
\index{Cramer's rule}
\label{chap:gaWikiCramers}
%\date{October 16, 2007.  gaWikiCramers.tex}

\section{Cramer's rule, determinants, and matrix inversion can be naturally expressed in terms of the wedge product}

The use of the wedge product in the solution of linear equations can be quite useful.

This does not require any notion of geometric algebra, only an exterior product and the concept of similar elements, and a nice example of such a treatment can be found in Solution of Linear equations section of \citep{grassmanbookExteriorProduct}.

Traditionally, instead of using the wedge product, Cramer's rule is usually presented as a generic algorithm that can be used to solve linear equations of the form \(\BA \Bx = \Bb\) (or equivalently to invert a matrix).  Namely

\begin{equation}\label{eqn:gaWikiCramers:20}
\Bx = \frac{1}{|\BA|}\operatorname{adj}(\BA)
\end{equation}

This is a useful theoretic result.  For numerical problems row reduction with pivots and other methods are more stable and efficient.

When the wedge product is coupled with the Clifford product and put into a natural geometric context, the fact that the determinants are used in the expression of \({\mathbb R}^N\) parallelogram area and parallelepiped volumes (and higher dimensional generalizations of these) also comes as a nice side effect.

As is also shown below, results such as Cramer's rule also follow directly from the property of the wedge product that it selects non identical elements.  The end result is then simple enough that it could be derived easily if required instead of having to remember or look up a rule.

\subsection{Two variables example}

\begin{equation}\label{eqn:gaWikiCramers:40}
\begin{bmatrix}
\Ba & \Bb
\end{bmatrix}
\begin{bmatrix}
x \\ y
\end{bmatrix}
= \Ba x + \Bb y = \Bc
\end{equation}

Pre and post multiplying by \(\Ba\) and \(\Bb\).

\begin{equation}\label{eqn:gaWikiCramers:60}
      ( \Ba x + \Bb y ) \wedge \Bb = (\Ba \wedge \Bb) x =       \Bc \wedge \Bb
\end{equation}
\begin{equation}\label{eqn:gaWikiCramers:80}
\Ba \wedge ( \Ba x + \Bb y )       = (\Ba \wedge \Bb) y = \Ba \wedge \Bc
\end{equation}

Provided \(\Ba \wedge \Bb \neq 0\) the solution is

\begin{equation}\label{eqn:gaWikiCramers:100}
\begin{bmatrix}x \\ y\end{bmatrix}
= \frac{1}{\Ba \wedge \Bb}
\begin{bmatrix}
\Bc \wedge \Bb \\ \Ba \wedge \Bc
\end{bmatrix}
\end{equation}

For \(\Ba, \Bb \in {\mathbb R}^2\), this is Cramer's rule since the \(\Be _1 \wedge \Be _2\) factors of the wedge products

\begin{equation}\label{eqn:gaWikiCramers:120}
\Bu \wedge \Bv = \begin{vmatrix}u_1 & u_2 \\ v_1 & v_2 \end{vmatrix} \Be _1 \wedge \Be _2
\end{equation}

divide out.

Similarly, for three, or N variables, the same ideas hold

\begin{equation}\label{eqn:gaWikiCramers:140}
\begin{bmatrix}
\Ba & \Bb & \Bc
\end{bmatrix}
\begin{bmatrix}
x \\ y \\ z
\end{bmatrix}
= \Bd
\end{equation}

\begin{equation}\label{eqn:gaWikiCramers:160}
\begin{bmatrix}
x \\ y \\ z
\end{bmatrix}
= \frac{1}{\Ba \wedge \Bb \wedge \Bc}
\begin{bmatrix}
\Bd \wedge \Bb \wedge \Bc \\
\Ba \wedge \Bd \wedge \Bc \\
\Ba \wedge \Bb \wedge \Bd
\end{bmatrix}
\end{equation}

Again, for the three variable three equation case this is Cramer's rule since the \(\Be _1 \wedge \Be _2 \wedge \Be _3\) factors of all the wedge products divide out, leaving the familiar determinants.

\subsection{A numeric example}

When there are more equations than variables case, if the equations have a solution, each of the k-vector quotients will be scalars

To illustrate here is the solution of a simple example with three equations and two unknowns.

\begin{equation}\label{eqn:gaWikiCramers:180}
\begin{bmatrix}
1 \\ 1 \\ 0
\end{bmatrix}
x
+
\begin{bmatrix}
1 \\ 1 \\ 1
\end{bmatrix}
y
=
\begin{bmatrix}
1 \\ 1 \\ 2
\end{bmatrix}
\end{equation}

The right wedge product with \((1, 1, 1)\) solves for \(x\)

\begin{equation}\label{eqn:gaWikiCramers:200}
\begin{bmatrix}
1 \\ 1 \\ 0
\end{bmatrix}
\wedge
\begin{bmatrix}
1 \\ 1 \\ 1
\end{bmatrix}
x
=
\begin{bmatrix}
1 \\ 1 \\ 2
\end{bmatrix}
\wedge
\begin{bmatrix}
1 \\ 1 \\ 1
\end{bmatrix}
\end{equation}

and a left wedge product with \((1, 1, 0)\) solves for \(y\)

\begin{equation}\label{eqn:gaWikiCramers:220}
\begin{bmatrix}
1 \\ 1 \\ 0
\end{bmatrix}
\wedge
\begin{bmatrix}
1 \\ 1 \\ 1
\end{bmatrix}
y
=
\begin{bmatrix}
1 \\ 1 \\ 0
\end{bmatrix}
\wedge
\begin{bmatrix}
1 \\ 1 \\ 2
\end{bmatrix}
\end{equation}

Observe that both of these equations have the same factor, so
one can compute this only once (if this was zero it would
indicate the system of equations has no solution).

Collection of results for
\(x\) and \(y\) yields a Cramer's rule like form
(writing \(\Be _i \wedge \Be _j = \Be _{ij}\)):

\begin{equation}\label{eqn:gaWikiCramers:240}
\begin{bmatrix}
x \\ y\end
{bmatrix}
=
\frac{1}{(1, 1, 0) \wedge (1, 1, 1)}
\begin{bmatrix}
(1, 1, 2) \wedge (1, 1, 1) \\
(1, 1, 0) \wedge (1, 1, 2)
\end{bmatrix}
=
\frac{1}{\Be_{13} + \Be_{23}}
\begin{bmatrix}
{-\Be_{13} - \Be_{23}} \\
{2\Be_{13} +2\Be_{23}} \\
\end{bmatrix}
=
\begin{bmatrix}
-1 \\ 2
\end{bmatrix}
\end{equation}


   \chapter{Torque}
      %
% Copyright � 2012 Peeter Joot.  All Rights Reserved.
% Licenced as described in the file LICENSE under the root directory of this GIT repository.
%

%
%
%\chapter{Torque}
\index{torque}
\label{chap:gaWikiTorque}
%\date{Oct 13, 2007.  gaWikiTorque.tex}

Torque is generally defined as the magnitude of the perpendicular force component times distance, or work per unit angle.

Suppose a circular path in an arbitrary plane containing orthonormal vectors \(\ucap\) and \(\vcap\) is parametrized by angle.
%
\begin{equation}\label{eqn:gaWikiTorque:20}
\Br = r(\ucap \cos \theta + \vcap \sin \theta) = r \ucap(\cos \theta + \ucap \vcap \sin \theta)
\end{equation}
%
By designating the unit bivector of this plane as the imaginary number
%
\begin{equation}\label{eqn:gaWikiTorque:40}
\Bi  = \ucap \vcap = \ucap \wedge \vcap
\end{equation}
\begin{equation}\label{eqn:gaWikiTorque:60}
\Bi ^2 = -1
\end{equation}
%
this path vector can be conveniently written in complex exponential form
%
\begin{equation}\label{eqn:gaWikiTorque:80}
\Br = r \ucap e^{\Bi \theta}
\end{equation}
%
and the derivative with respect to angle is
%
\begin{equation}\label{eqn:gaWikiTorque:100}
\dtheta{\Br} = r \ucap \Bi  e^{\Bi  \theta} = \Br  \Bi
\end{equation}
%
So the torque, the rate of change of work \(W\), due to a force \(F\), is
%
\begin{equation}\label{eqn:gaWikiTorque:120}
\tau = \dtheta{W} = \BF \cdot \dtheta{\Br} = \BF \cdot (\Br  \Bi )
\end{equation}
%
Unlike the cross product description of torque, \(\Btau = \Br \times \BF\) no vector in a normal direction had to be introduced, a normal that does not exist in two dimensions or in greater than three dimensions.  The unit bivector describes the plane and the orientation of the rotation, and the sense of the rotation is relative to the angle between the vectors \(\ucap\) and \(\vcap\).

\section{Expanding the result in terms of components}

At a glance this does not appear much like the familiar torque as a determinant or cross product, but this can be expanded to demonstrate its equivalence.  Note that the cross product is hiding there in the bivector \(\Bi = \ucap \wedge \vcap\).  Expanding the position vector in terms of the planar unit vectors
%
\begin{equation}\label{eqn:gaWikiTorque:140}
\Br \Bi =
\left(
r_u \ucap + r_v \vcap
\right)
\ucap \vcap
=
r_u \vcap
- r_v \ucap
\end{equation}
%
and expanding the force by components in the same direction plus the possible perpendicular remainder term
%
\begin{equation}\label{eqn:gaWikiTorque:160}
\BF  = F_u \ucap + F_v \vcap + \BF _{\perp \ucap,\vcap}
\end{equation}
%
and then taking dot products yields is the torque
%
\begin{equation}\label{eqn:gaWikiTorque:180}
\tau = \BF \cdot (\Br  \Bi ) = r_u F_v - r_v F_u
\end{equation}
%
This determinant may be familiar from derivations with \(\ucap = \Be _1\), and \(\vcap = \Be _2\) (See the Feynman lectures Volume I for example).

\section{Geometrical description}

When the magnitude of the "rotational arm" is factored out, the torque can be written as
%
\begin{equation}\label{eqn:gaWikiTorque:200}
\tau = \BF \cdot (\Br  \Bi ) = |\Br |  (\BF \cdot (\rcap \Bi ))
\end{equation}
%
The vector \(\rcap \Bi \) is the unit vector perpendicular to the \(\Br\).  Thus the torque can also be described as the product of the magnitude of the rotational arm times the component of the force that is in the direction of the rotation (ie: the work done rotating something depends on length of the lever, and the size of the useful part of the force pushing on it).

\section{Slight generalization.  Application of the force to a lever not in the plane}

If the rotational arm that the force is applied to is not in the plane of rotation then only the components of the lever arm direction and the component of the force that are in the plane will contribute to the work done.  The calculation above was general with respect to the direction of the force, so to generalize
it for an arbitrarily oriented lever arm, the quantity \(\Br\) needs to be replaced by the projection of \(\Br\) onto the plane of rotation.

That component in the plane (bivector) \(\Bi\) can be described with the geometric product nicely
%
\begin{equation}\label{eqn:gaWikiTorque:220}
\Br _{\Bi } =  (\Br  \cdot \Bi ) \frac{1}{\Bi } =  -(\Br  \cdot \Bi ) \Bi
\end{equation}
%
Thus, the vector with this magnitude that is perpendicular to this in the plane of the rotation  is
%
\begin{equation}\label{eqn:gaWikiTorque:240}
\Br _{\Bi } \Bi
=  -(\Br  \cdot \Bi ) \Bi ^2
=  (\Br  \cdot \Bi )
\end{equation}
%
So, the most general for torque for rotation constrained to the plane \(i\) is:
%
\begin{equation}\label{eqn:gaWikiTorque:260}
\tau
=  \BF  \cdot (\Br  \cdot \Bi )
\end{equation}
%
This makes sense when once considers that only the dot product part of \(\Br  \Bi  = \Br  \cdot \Bi  + \Br  \wedge \Bi \) contributes to the component of \(\Br \) in the plane, and when the lever is in the rotational plane this wedge product component of
\(\Br \Bi \) is zero.

\section{expressing torque as a bivector}

The general expression for torque for a rotation constrained to a plane has been found to be:
%
\begin{equation}\label{eqn:gaWikiTorque:280}
\tau
=  \BF  \cdot (\Br  \cdot \Bi )
\end{equation}
%
We have an expectation that torque should have a form similar to the traditional vector torque
\begin{equation}\label{eqn:gaWikiTorque:300}
\Btau = \Br \times \BF = -\bithree (\Br \wedge \BF)
\end{equation}
%
Note that here \(\bithree = \Be_1 \Be_2 \Be_3 \) is the unit pseudoscalar for \(\mathbb{R}^3\), not the unit bivector for the rotational plane.
We should be able to express torque in a form related to \(\Br \wedge \BF\), but modified
in a fashion that results in a scalar value.

When the rotation is not constrained to a specific plane the motion will be in
%
\begin{equation}\label{eqn:gaWikiTorque:320}
\Bi = \frac{\rcap \wedge \Br'}{\nrrp}
\end{equation}
%
The lever arm component in this plane is
%
\begin{equation}\label{eqn:gaWikiTorque:420}
\begin{aligned}
\Br \cdot \Bi
   &= \frac{1}{2}           (\Br \Bi - \Bi \Br) \\
   &= \frac{1}{2\nrrp} (\Br (\rcap \wedge \Br') - (\rcap \wedge \Br') \Br) \\
   &= \frac{1}{\nrrp}   \Br (\rcap \wedge \Br') \\
\end{aligned}
\end{equation}
%
So the torque in this natural plane of rotation is
%
\begin{equation}\label{eqn:gaWikiTorque:440}
\begin{aligned}
\tau
   &=  \BF  \cdot (\Br  \cdot \Bi )  \\
   &=  \frac{1}{\nrrp}     \BF \cdot ( \Br (\rcap \wedge \Br') ) \\
   &=  \frac{1}{2 \nrrp} \left(    \BF \Br (\rcap \wedge \Br') + (\Br' \wedge \rcap) \Br \BF    \right)\\
   &=  \frac{1}{2} ( \BF \Br \Bi + (\BF \Br \Bi)^\dagger ) = \frac{1}{2} ( \Bi \Br \BF + (\Bi \Br \BF)^\dagger ) \\
   &=  {\langle \Bi \Br \BF \rangle}_0
\end{aligned}
\end{equation}
%
The torque is the scalar part of \(\Bi (\Br \BF)\).
%
\begin{equation}\label{eqn:gaWikiTorque:340}
\tau
   =  {\langle \Bi (\Br \cdot \BF + \Br \wedge \BF) \rangle}_0 \\
\end{equation}
%
Since the bivector scalar product \(\Bi (\Br \cdot \BF)\) here contributes only a bivector part the scalar part comes only from the \(\Bi (\Br \wedge \BF)\) component,
and one can write the torque in a fashion that is very similar to the vector cross product torque.  Here is both for comparison
%
\begin{equation}\label{eqn:gaWikiTorque:460}
\begin{aligned}
\tau &=  {\langle \Bi (\Br \wedge \BF) \rangle}_0 \\
\Btau &= -\bithree (\Br \wedge \BF)
\end{aligned}
\end{equation}
%
Note again that \(\Bi\) here is the unit bivector for the plane of rotation and not the unit 3D pseudoscalar \(\bithree\).

\section{Plane common to force and vector}
Physical intuition provides one further way to express this.  Namely, the unit bivector for the rotational plane should also be in the plane common to \(\BF\) and \(\Br\)
%
\begin{equation}\label{eqn:gaWikiTorque:360}
\Bi = \frac{\BF \wedge \Br}{\sqrt{-(\BF \wedge \Br)^2}}
\end{equation}
%
So the torque is
\begin{equation}\label{eqn:gaWikiTorque:480}
\begin{aligned}
\tau
   &=  \frac{1}{\sqrt{-(\BF \wedge \Br)^2}} {\langle (\BF \wedge \Br)(\Br \wedge \BF) \rangle}_0 \\
   &=  \frac{1}{\sqrt{-(\BF \wedge \Br)^2}} {(\BF \wedge \Br)(\Br \wedge \BF) } \\
   &=  \frac {-(\Br \wedge \BF)^2 } {\sqrt{-(\Br \wedge \BF)^2}} \\
   &=  \sqrt {-(\Br \wedge \BF)^2 } \\
   &=  \norm {\Br \wedge \BF}
\end{aligned}
\end{equation}
%
Above the \({\langle \cdots \rangle}_0\) could be dropped because the quantity has only a scalar part.
The fact that the sign of the square root can be either plus or minus follows from the fact that the orientation of the unit bivector in the \(\Br\), \(\BF\) plane has two possibilities.  The positive root selection here is due to the orientation picked for \(\Bi\).

For comparison, this can also be expressed with the cross product:
\begin{equation}\label{eqn:gaWikiTorque:500}
\begin{aligned}
\tau
   &=  \sqrt {-(\Br \wedge \BF)^2 } \\
   &=  \sqrt {-(\Br \wedge \BF)(\Br \wedge \BF) } \\
   &=  \sqrt {-((\Br \times \BF)\bithree)(\bithree(\Br \times \BF)) } \\
   &=  \sqrt {(\Br \times \BF)^2} \\
   &=  \norm{\Br \times \BF} \\
   &=  \norm{\Btau} \\
\end{aligned}
\end{equation}
%
\section{Torque as a bivector}

It is natural to drop the magnitude in the torque expression and name the
bivector quantity
%
\begin{equation}\label{eqn:gaWikiTorque:380}
   \Br \wedge \BF
\end{equation}
%
This defines both the plane of rotation (when that rotation is unconstrained) and the orientation of the rotation, since inverting either the force or the arm position will invert the rotational direction.

When examining the general equations for motion of a particle of fixed mass we will see this quantity again related to the non-radial component of that particles acceleration.  Thus we define a torque bivector
%
\begin{equation}\label{eqn:gaWikiTorque:400}
\Btau = \Br \wedge \BF
\end{equation}
%
The magnitude of this bivector is our scalar torque, the rate of change of work on the object with respect to the angle of rotation.

   \chapter{Derivatives of a unit vector}\label{chap:PJUnitDer}
      %
% Copyright � 2012 Peeter Joot.  All Rights Reserved.
% Licenced as described in the file LICENSE under the root directory of this GIT repository.
%

%
%
%\mychapter{Derivatives of a unit vector}
%\label{chap:PJUnitDer}
\index{unit vector!derivative}
%\date{Oct 16, 2007.  gaWikiUnitDerivative.tex}

\section{First derivative of a unit vector}

\subsection{Expressed with the cross product}

It can be shown that a unit vector derivative can be expressed using the cross product.  Two cross product operations are required to get the result back into the plane of the rotation, since a unit vector is constrained to circular (really perpendicular to itself) motion.
%
\begin{equation}\label{eqn:gaWikiUnitDerivative:20}
\dt{}\left(\frac{\Br}{\Vert \Br \Vert}\right)
= \frac{1}{{\Vert \Br \Vert}^3}\left(\Br \times \dt{\Br}\right) \times \Br
= \left(\rcap \times \frac{1}{{\Vert \Br \Vert}} \dt{\Br}\right) \times \rcap
\end{equation}
%
This derivative is the rejective component of \(\dt{\Br}\) with respect to \(\rcap\), but is scaled by \(1/\Vert \Br \Vert\).

How to calculate this result can be found in other places, such as
\citep{salas1990coa}.

\section{Equivalent result utilizing the geometric product}

The equivalent geometric product result can be obtained by calculating the derivative of a vector \(\Br = r \rcap\).
%
\begin{equation}\label{eqn:gaWikiUnitDerivative:40}
\dt{\Br} = r \dt{\rcap} + \rcap \dt{r}
\end{equation}
%
\subsection{Taking dot products}
One trick is required first (as was also the case in the Salus and Hille derivation), which is expressing \(\dt{r}\) via the dot product.
%
\begin{equation}\label{eqn:gaWikiUnitDerivative:120}
\begin{aligned}
\dt{(r^2)} &= 2r \dt{r} \\
\dt{(\Br \cdot \Br)} &= 2 \Br \cdot \dt{\Br} \\
\end{aligned}
\end{equation}
%
Thus,
\begin{equation}\label{eqn:gaWikiUnitDerivative:60}
\dt{r} = \rcap \cdot \dt{\Br}
\end{equation}
%
Taking dot products of the derivative above yields
%
\begin{equation}\label{eqn:gaWikiUnitDerivative:140}
\begin{aligned}
\rcap \cdot \dt{\Br} &= \rcap \cdot r \dt{\rcap} + \rcap \cdot \rcap \dt{r} \\
                            &= \Br \cdot \dt{\rcap} + \dt{r} \\
                            &= \Br \cdot \dt{\rcap} + \rcap \cdot \dt{\Br}
\end{aligned}
\end{equation}
%
\begin{equation}\label{eqn:gaWikiUnitDerivative:80}
\implies
\Br \cdot \dt{\rcap} = \Bzero
\end{equation}
%
One could alternatively prove this with a diagram.


\subsection{Taking wedge products}

As in linear equation solution, the \(\rcap\) component can be eliminated by taking a wedge product
%
\begin{equation}\label{eqn:gaWikiUnitDerivative:160}
\begin{aligned}
\rcap \wedge \dt{\Br} &= \rcap \wedge r \dt{\rcap} + \rcap \wedge \rcap \dt{r} \\
                             &= r \rcap \wedge \dt{\rcap} \\
                             &= \Br \wedge \dt{\rcap}  \\
                             &= \Br \wedge \dt{\rcap} + \Br \cdot \dt{\rcap} \\
                             &= \Br \dt{\rcap}
\end{aligned}
\end{equation}
%
This allows expression of \(\dt{\rcap}\) in terms of \(\dt{\Br}\) in various ways (compare to the cross product results above)
%
\begin{equation}\label{eqn:gaWikiUnitDerivative:180}
\begin{aligned}
\dt{\rcap} &= \frac{1}{{ \Br }}\left(\rcap \wedge \dt{\Br}\right) \\
%                   &= \frac{1}{\Vert \Br \Vert}{     \frac{1}{\rcap} \left(\rcap \wedge \dt{\Br}\right)       } \\
                   &= \frac{1}{\Vert \Br \Vert}{     {\rcap} \left(\rcap \wedge \dt{\Br}\right)       } \\
%                   &= \frac{1}{{\Vert \Br \Vert}^3}{     {\Br} \left(\Br \wedge \dt{\Br}\right)       } \\
                   &= \frac{1}{\Vert \Br \Vert}\left({ \dt{\Br} - \rcap (\rcap \cdot \dt{\Br}) }\right) \\
\end{aligned}
\end{equation}
%
Thus this derivative is the component of
\(\frac{1}{{\Vert \Br \Vert}}\dt{\Br}\)
in the direction perpendicular to
\(\Br\).

\subsection{Another view}

When the objective is not comparing to the cross product, it is also notable that this unit vector derivative can be written
%
\begin{equation}\label{eqn:gaWikiUnitDerivative:100}
{{ \Br }} \dt{\rcap}
= \rcap \wedge \dt{\Br}
\end{equation}
%

%
% This was obvious to me at one point but is not now;)  What is the justification for the first statement?
%
%\subsection{A more direct route}
%
%Like a lot of stuff in math, once you know the answer you can get the answer more directly.  There is an unfortunate tendancy
%in some math texts to skip the logical sequence and go straight to the end result by the quickest route.  This is more
%elegant
%
%\begin{align*}
%r \dt{\rcap}
%   &= \dt{\Br} - \rcap \dt{r} \\
%   &= \dt{\Br} - \rcap\left(\rcap \cdot \dt{\Br}\right) \\
%   &= \rcap \left(\rcap \dt{\Br} - \rcap \cdot \dt{\Br}\right) \\
%   &= \rcap \left(\rcap \wedge \dt{\Br}\right) \\
%\end{align*}
%
%and gives the appearance of being clever, but it is easy to be clever when you already know the answer.


   \chapter{Radial components of vector derivatives}\label{chap:PJRadialDer}
      %
% Copyright � 2012 Peeter Joot.  All Rights Reserved.
% Licenced as described in the file LICENSE under the root directory of this GIT repository.
%

%
%
%\chapter{Radial components of vector derivatives}\label{chap:PJRadialDer}
\index{vector!radial component}
%\date{Oct 22, 2007.  radialVectorDerivatives.tex}

\section{first derivative of a radially expressed vector}

Having calculated the derivative of a unit vector, the total
derivative of a radially expressed vector can be calculated

\begin{equation}\label{eqn:radialVectorDerivatives:20}
\begin{aligned}
(r\rcap)'
   &= r'\rcap  + r\rcap' \\
   &= r'\rcap  + \BrPrimeRej \\
\end{aligned}
\end{equation}

There are two components.  One is in the \(\rcap\) direction (linear component)
and the other perpendicular to that (a rotational component) in the direction of the rejection
of \(\rcap\) from \(\Br'\).

\section{Second derivative of a vector}

Taking second derivatives of a radially expressed vector, we have

\begin{equation}\label{eqn:radialVectorDerivatives:40}
\begin{aligned}
(r\rcap)''
   &= (r'\rcap + r{\rcap}')' \\
   &= r''\rcap + r'\rcap' + (r\rcap')' \\
   &= r''\rcap + (r'/r)\BrPrimeRej + (r\rcap')' \\
\end{aligned}
\end{equation}

Expanding the last term takes a bit more work
\begin{equation}\label{eqn:radialVectorDerivatives:60}
\begin{aligned}
(r\rcap')'
   &= (\BrPrimeRej)' \\
   &=
\rcap'(\rcap \wedge \Br') +
\rcap(\rcap' \wedge \Br') +
\rcap(\rcap \wedge \Br'') \\
   &=
(1/r)(\BrPrimeRej)(\rcap \wedge \Br') +
\rcap(\rcap' \wedge \Br') +
\rcap(\rcap \wedge \Br'') \\
   &=
(1/r)\rcap(\rcap \wedge \Br')^2 +
\rcap(\rcap' \wedge \Br') +
\rcap(\rcap \wedge \Br'') \\
\end{aligned}
\end{equation}

There are three terms to this.  One a scalar (negative) multiple of \(\rcap\), and another, the rejection of \(\rcap\) from \(\Br''\).  The middle term here remains to be expanded.  In particular,

\begin{equation}\label{eqn:radialVectorDerivatives:80}
\begin{aligned}
\rcap' \wedge \Br'
   &= \rcap' \wedge (r\rcap' + r'\rcap) \\
   &= r' \rcap' \wedge \rcap \\
   &= r'/2 (\rcap'\rcap - \rcap\rcap') \\
   &= r'/2r ((\Br' \wedge \rcap)\rcap\rcap - \rcap\rcap(\rcap \wedge \Br')) \\
   &= r'/2r (\Br' \wedge \rcap - \rcap \wedge \Br') \\
   &= -(r'/r) \rcap \wedge \Br' \\
\end{aligned}
\end{equation}

\begin{equation}\label{eqn:radialVectorDerivatives:100}
\begin{aligned}
\implies
(r\rcap')'
   &=
(1/r)\rcap(\rcap \wedge \Br')^2
-(r'/r)\BrPrimeRej
+\rcap(\rcap \wedge \Br'') \\
\end{aligned}
\end{equation}

\begin{equation}\label{eqn:radialVectorDerivatives:120}
\begin{aligned}
\implies
(r\rcap)''
   &= r''\rcap
+(r'/r)\BrPrimeRej
+(1/r)\rcap(\rcap \wedge \Br')^2
-(r'/r)\BrPrimeRej
+\rcap(\rcap \wedge \Br'') \\
   &= r''\rcap
    +(1/r)\rcap(\rcap \wedge \Br')^2
    +\rcap(\rcap \wedge \Br'') \\
   &=
\rcap \left(  r'' +(1/r)(\rcap \wedge \Br')^2\right) +    \rcap(\rcap \wedge \Br'') \\
\end{aligned}
\end{equation}

There are two terms here that are in the \(\rcap\) direction (the bivector square is a negative scalar), and
one rejective term in the direction of the component perpendicular to \(\rcap\) relative to \(\Br''\).


   \chapter{Rotational dynamics}\label{chap:PJAngVel}
      %
% Copyright � 2012 Peeter Joot.  All Rights Reserved.
% Licenced as described in the file LICENSE under the root directory of this GIT repository.
%

%
%
%\chapter{Rotational dynamics}\label{chap:PJAngVel}
%\date{January 29, 2008.  angularVelocity.tex}

\section{GA introduction of angular velocity}
\index{angular velocity}

By taking the first derivative of a radially expressed vector we have the velocity

\begin{equation}\label{eqn:angularVelocity:20}
\Bv
   = r'\rcap + \rcap(\rcap \wedge \Br')
   = \rcap( v_r + \rcap \wedge \Bv )
\end{equation}

Or,
\begin{equation}\label{eqn:angularVelocity:40}
\rcap \Bv = v_r + \rcap \wedge \Bv
\end{equation}
\begin{equation}\label{eqn:angularVelocity:60}
\rcap \Bv = v_r + (1/r)\Br \wedge \Bv
\end{equation}

Put this way, the earlier calculus exercise to derive this seems a bit silly, since it is probably clear that \(v_r = \rcap \cdot \Bv\).

Anyways, let us work with velocity expressed this way in a few ways.

\subsection{Speed in terms of linear and rotational components}

\begin{equation}\label{eqn:angularVelocity:80}
\Abs{\Bv}^2 = v_r^2 + (\rcap(\rcap \wedge \Bv))^2
\end{equation}

And,
\begin{equation}\label{eqn:angularVelocity:280}
\begin{aligned}
(\rcap(\rcap \wedge \Bv))^2
   &= (\Bv \wedge \rcap)\rcap \rcap(\rcap \wedge \Bv) \\
   &= (\Bv \wedge \rcap)(\rcap \wedge \Bv) \\
   &= -(\rcap \wedge \Bv)^2 \\
   &= \Abs{\rcap \wedge \Bv}^2 \\
\end{aligned}
\end{equation}

\begin{equation}\label{eqn:angularVelocity:300}
\begin{aligned}
\implies
\Abs{\Bv}^2 &= v_r^2 + \Abs{\rcap \wedge \Bv}^2 \\
             &= v_r^2 + \Abs{\rcap \wedge \Bv}^2 \\
\end{aligned}
\end{equation}

So, we can assign a physical significance to the bivector.

\begin{equation}\label{eqn:angularVelocity:100}
\Abs{\rcap \wedge \Bv} = \abs{v_{\perp}}
\end{equation}

The bivector \(\Abs{\rcap \wedge \Bv}\) has the magnitude of the non-radial component of the velocity.  This
equals the magnitude of the component of the velocity perpendicular to its radial component (ie: the angular component of the velocity).

\subsection{angular velocity.  Prep}

Because \(\Abs{\rcap \wedge \Bv}\) is the non-radial velocity component, for small angles
\({v_\perp}/r\) will equal the angle between the vector and its displacement.

This allows for the calculation of the rate of change of that angle with time, what it called the scalar
angular velocity (dimensions are \(1/t\) not \(x/t\)).  This can be done by taking the \(\sin\) as the ratio of the
length of the non-radial component of the delta to the length of the displaced vector.

\begin{equation}\label{eqn:angularVelocity:320}
\begin{aligned}
\sin d\theta &= \frac{\Abs{\rcap(\rcap \wedge d\Br)}}{\Abs{\Br + d\Br}} \\
\end{aligned}
\end{equation}

With \(d\Br = \dt{\Br} dt = \Bv dt\), the angular velocity is

\begin{equation}\label{eqn:angularVelocity:340}
\begin{aligned}
\sin d\theta
   &= \frac{1}{\Abs{\Br + \Bv dt}} \Abs{ \rcap (\rcap \wedge \Bv) dt } \\
   &= \frac{1}{\Abs{\Br + \Bv dt}} \Abs{ (\rcap \wedge \Bv) dt } \\
\frac{\sin d\theta}{\abs{dt}}
   &= \frac{1}{\Abs{\Br + \Bv dt}} \Abs{ \rcap \wedge \Bv } \\
   &= \frac{1}{\Abs{\Br}\Abs{\Br + \Bv dt}} \Abs{ \Br \wedge \Bv } \\
\end{aligned}
\end{equation}

In the limit, taking \(dt > 0\), this is
\begin{equation}\label{eqn:angularVelocity:120}
\omega = \dt{\theta} = \frac{1}{\Br^2} \Abs{ \Br \wedge \Bv }
\end{equation}

\subsection{angular velocity.  Summarizing}

Here is a summary of calculations so far involving the \(\Br \wedge \Bv\) bivector

\begin{equation}\label{eqn:angularVelocity:360}
\begin{aligned}
\Bv &= \rcap v_r + \frac{\rcap}{\Abs{\Br}} (\Br \wedge \Bv) \\
\dt{\rcap} &= \frac{\rcap}{\Br^2} (\Br \wedge \Bv) \\
\abs{v_{\perp}} &= \frac{1}{\Abs{\Br}} \Abs{ \Br \wedge \Bv } \\
\omega = \dt{\theta} &= \frac{1}{\Br^2} \Abs{ \Br \wedge \Bv } \\
\end{aligned}
\end{equation}

It makes sense to give the bivector a name.  Given its magnitude the
angular velocity bivector \(\Bomega\) is designated

\begin{equation}\label{eqn:angularVelocity:140}
\Bomega = \frac{ \Br \wedge \Bv }{\Br^2}
\end{equation}

So the linear and rotational components of the velocity can thus be expressed in terms of this, as can our
unit vector derivative, scalar angular velocity, and perpendicular velocity magnitude:

\begin{equation}\label{eqn:angularVelocity:380}
\begin{aligned}
\omega = \dt{\theta} &= \Abs{ \Bomega } \\
\Bv &= \rcap v_r + \Br \Bomega \\
    &= \rcap( v_r + r \Bomega ) \\
\dt{\rcap} &= \rcap \Bomega \\
\abs{v_{\perp}} &= r \Abs{ \Bomega } \\
\end{aligned}
\end{equation}

This is similar to the vector angular velocity (\(\Bomega = (\Br \times \Bv)/r^2\)), but instead of lying perpendicular to the
plane of rotation, it defines the plane of rotation (for a vector \(\Ba\), \(\Ba \wedge \Bomega\) is zero if the vector is in the plane and non-zero if the vector has a component outside of the plane).

%\begin{align*}
%\Bomega = \frac{1}{\Br^2} (\Br \wedge \Bv)
%\end{align*}
%
%Or,
%\begin{align*}
%\Br \wedge \Bv = \Br^2 \Bomega = r^2\Bomega
%\end{align*}
%
%
%\begin{align*}
%\Bv
%   &= \rcap(v_r + (1/r)\Br \wedge \Bv) \\
%   &= \rcap(v_r + r\Bomega) \\
%   &= v_r\rcap + \Br\Bomega \\
%\end{align*}

\subsection{Explicit perpendicular unit vector}
\index{unit normal}

If one introduces a unit vector \(\thetacap\) in the direction of rejection of \(\Br\) from \(d\Br\), the total velocity takes the symmetrical form
\begin{equation}\label{eqn:angularVelocity:400}
\begin{aligned}
\Bv
   &= v_r\rcap + r\omega\thetacap \\
   &= \dt{r}\rcap + r\dt{\theta}\thetacap \\
\end{aligned}
\end{equation}

\subsection{acceleration in terms of angular velocity bivector}

Taking derivatives of velocity, one can with a bit of work,
express acceleration in terms of
radial and non-radial components

%\Br \wedge \Bv = \Br^2 \Bomega = r^2\Bomega

\begin{equation}\label{eqn:angularVelocity:420}
\begin{aligned}
\Ba
   &= (\rcap v_r + \Br \Bomega)' \\
   &= \rcap' v_r + \rcap v_r' + \Br' \Bomega + \Br \Bomega' \\
   &= \rcap \Bomega v_r + \rcap v_r' + \Br' \Bomega + \Br \Bomega' \\
   &= \rcap \Bomega v_r + \rcap a_r + \Bv \Bomega + \Br \Bomega' \\
\end{aligned}
\end{equation}

But,
\begin{equation}\label{eqn:angularVelocity:440}
\begin{aligned}
\Bomega' &= ((1/r^2) (\Br \wedge \Bv))' \\
         &= (-2/r^3) r' (\Br \wedge \Bv) + (1/r^2) (\Bv \wedge \Bv + \Br \wedge \Ba) \\
         &= -(2/r) v_r \Bomega + (1/r^2) (\Br \wedge \Ba) \\
\end{aligned}
\end{equation}

%\rcap v_r = \Bv - \Br \Bomega
So,

\begin{equation}\label{eqn:angularVelocity:460}
\begin{aligned}
\Ba
   &= \rcap a_r -\rcap \Bomega v_r + \Bv \Bomega + \rcap (\rcap \wedge \Ba) \\
   &= \rcap a_r -( \Bv - \Br \Bomega) \Bomega + \Bv \Bomega + \rcap (\rcap \wedge \Ba) \\
\\
   &= \rcap a_r + \Br \Bomega^2+ \rcap (\rcap \wedge \Ba) \\
   &= \rcap( a_r + r \Bomega^2) + \rcap (\rcap \wedge \Ba) \\
\end{aligned}
\end{equation}

Note that \(\Bomega^2\) is a negative scalar, so as normal writing \(\norm{\Bomega}^2 = -\Bomega^2\), we have acceleration in a fashion similar to the
traditional cross product form:

\begin{equation}\label{eqn:angularVelocity:480}
\begin{aligned}
\Ba
   &= \rcap( a_r - r \norm{\Bomega}^2) + \rcap (\rcap \wedge \Ba) \\
   &= \rcap( a_r - r \norm{\Bomega}^2 + \rcap \wedge \Ba) \\
\end{aligned}
\end{equation}

In the traditional representation, this last term, the non-radial acceleration
component, is often expressed as a derivative.

In terms of the wedge product, this can be done by noting that

\begin{equation}\label{eqn:angularVelocity:160}
(\Br \wedge \Bv)' = \Bv \wedge \Bv + \Br \wedge \Ba = \Br \wedge \Ba
\end{equation}

\begin{equation}\label{eqn:angularVelocity:500}
\begin{aligned}
\Ba
   &= \rcap( a_r - r \norm{\Bomega}^2 ) + \frac{\Br}{r^2}(\Br \wedge \Bv)') \\
   &= \rcap( a_r - r \norm{\Bomega}^2 ) + \frac{1}{\Br}\dt{(\Br^2 \Bomega)} \\
\end{aligned}
\end{equation}

Expressed in terms of force (for constant mass) this is
\begin{equation}\label{eqn:angularVelocity:520}
\begin{aligned}
\BF &= m \Ba \\
    &= \rcap (m a_r) + (m \Br) {\Bomega}^2
       + \frac{1}{\Br}\dt{(m \Br^2 \Bomega)} \\
    &= \BF_r + (m \Br) {\Bomega}^2
             + \frac{1}{\Br}\dt{(m \Br^2 \Bomega)} \\
\end{aligned}
\end{equation}

Alternately, the non-radial term can be expressed in terms of torque

\begin{equation}\label{eqn:angularVelocity:540}
\begin{aligned}
\rcap (\rcap \wedge \Ba)
   &= \rcap (\rcap \wedge m \Ba)  \\
   &= \frac{\Br}{r^2} (\Br \wedge \BF)  \\
   &= \frac{1}{\Br} (\Br \wedge \BF)  \\
   &= \frac{1}{\Br} \Btau \\
\end{aligned}
\end{equation}

Thus the torque bivector, which in magnitude was the angular derivative of
the work
done by the force \(\norm{\Btau} = \tau = \dtheta{W} = \BF \cdot \dtheta{\Br}\)
is also expressible as a time derivative

\begin{equation}\label{eqn:angularVelocity:560}
\begin{aligned}
\Btau
&= \dt{( m \Br^2 \Bomega )}  \\
&= \dt{( m \Br \wedge \Bv)}  \\
&= \dt{( \Br \wedge m \Bv)}  \\
&= \dt{( \Br \wedge \Bp  )}  \\
\end{aligned}
\end{equation}

This bivector \(m \Br^2 \Bomega = \Br \wedge \Bp\) is called the angular
momentum, designated \(\BJ\).  It is related to the total momentum as follows

\begin{equation}\label{eqn:angularVelocity:180}
\Bp = \rcap (\rcap \cdot \Bp) + \frac{1}{\Br} \BJ
\end{equation}

So the total force is

\begin{equation}\label{eqn:angularVelocity:580}
\begin{aligned}
\BF
    &= \BF_r + m \Br {\Bomega}^2 + \frac{1}{\Br}\dt{\BJ} \\
\end{aligned}
\end{equation}

Observe that for a purely radial (ie: central) force, we must have
\(\dt{\BJ} = 0\)
so, the angular
momentum must be constant.

\subsection{Kepler's laws example}
\index{Kepler's laws}

This follows the \citep{salas1990coa} treatment, modified for the GA notation.

Consider the gravitational force

\begin{equation}\label{eqn:angularVelocity:600}
\begin{aligned}
m \Ba &= -G \frac{m M}{r^2} \rcap \\
\Ba &= - G M \frac{\rcap}{r^2} = -\rho \frac{\rcap}{r^2}
\end{aligned}
\end{equation}

Or,
\begin{equation}\label{eqn:angularVelocity:200}
\frac{\rcap}{r^2} = -\frac{1}{\rho} \dt{\Bv}
\end{equation}

The unit vector derivative is

\begin{equation}\label{eqn:angularVelocity:620}
\begin{aligned}
\dt{\rcap} &= \frac{\rcap}{r}(\rcap \wedge \Bv) \\
           &= \frac{\rcap}{r^2}\frac{\BJ}{m} \\
           &= -\frac{1}{m \rho} \dt{\Bv} \BJ \\
           &= \dt{(-\frac{1}{m \rho} \Bv \BJ )} \\
\end{aligned}
\end{equation}

The last because \(\BJ\), \(m\), and \(\rho\) are all constant.

Before continuing, let us examine this funny vector bivector product term.
In general a vector
bivector product will have vector and trivector parts, but
the differential equation implies that this is a vector.  Let us confirm this

\begin{equation}\label{eqn:angularVelocity:640}
\begin{aligned}
\Bv \BJ &= \Bv (\Br \wedge m \Bv) \\
        &= (m \Bv^2) \vcap (\Br \wedge \vcap) \\
        &= - (m \Bv^2) \vcap (\vcap \wedge \Br) \\
\end{aligned}
\end{equation}

So, this is in fact a vector, it is the rejective component of \(\Br\) from
the direction of \(\vcap\) scaled by \(-m\Bv^2\).  We can also calculate
the product \(\BJ \Bv\) from this:

\begin{equation}\label{eqn:angularVelocity:660}
\begin{aligned}
\Bv \BJ
        &= - (m \Bv^2) \vcap (\vcap \wedge \Br) \\
        &= - (m \Bv^2) (\Br \wedge \vcap) \vcap \\
        &= - (\Br \wedge m \Bv) \Bv \\
        &= - \BJ \Bv \\
\end{aligned}
\end{equation}

This antisymetrical result \(\Bv \BJ = - \BJ \Bv\) is actually the defining
property of the vector bivector ``dot product'' (unlike the vector dot product
which is the symmetrical parts).  This vector bivector dot product selects the
vector component, leaving the trivector part.  Since \(\Bv\) lies completely in
the plane of the angular velocity bivector \(\Bv \wedge \BJ = 0\) in this case.

Anyways, back to the problem, integrating
with respect to time, and introducing a vector integration constant \(\Be\)
we have

\begin{equation}\label{eqn:angularVelocity:220}
\rcap + \frac{1}{m \rho} \Bv \BJ = \Be
\end{equation}

Multiplying by \(\Br\)

\begin{equation}\label{eqn:angularVelocity:680}
\begin{aligned}
r + \frac{1}{m \rho} \Br \Bv \BJ &= \Br \Be \\
r + \frac{1}{m^2 \rho} (\Br \cdot \Bp + \BJ) \BJ &= \Br \cdot \Be + \Br \wedge \Be \\
\end{aligned}
\end{equation}

This results in three equations, one for each of the scalar, vector, and bivector parts

\begin{equation}\label{eqn:angularVelocity:700}
\begin{aligned}
r + \frac{\BJ^2}{m^2 \rho} &= \Br \cdot \Be \\
\frac{1}{m \rho} (\Br \cdot \Bv) \BJ &= 0 \\
\Br \wedge \Be &= 0 \\
\end{aligned}
\end{equation}

The first of these equations is the result from Salas and Hille (integration constant differs in sign though).

\begin{equation}\label{eqn:angularVelocity:720}
\begin{aligned}
r - \frac{J^2}{m^2 \rho} &= \Br \cdot \Be \\
%r - \Br \cdot \Be &= \frac{J^2}{m^2 \rho} \\
%\Br \rcap - \Br \cdot \Be &= \frac{J^2}{m^2 \rho} \\
%\Br \rcap - \Br \Be &= \frac{J^2}{m^2 \rho} \\
%\Br (\rcap - \Be) &= \frac{J^2}{m^2 \rho}
\end{aligned}
\end{equation}

%\[
%\frac{\Br_0}{\norm{\Br_0}} + \frac{1}{m \rho} \Bv_0 \BJ = \Be
%\]
%
%\[
%\rcap + \frac{1}{m \rho} \Bv \BJ = \frac{\Br_0}{\norm{\Br_0}} + \frac{1}{m \rho} \Bv_0 \BJ
%\]
%
%\[
%\rcap - \rcap_0 + \frac{1}{m \rho} (\Bv - \Bv_0) \BJ = 0
%\]
%
%J = r ^ m v = r mv - r . mv = r mv
%v = 1/(r m) J

%Diverging from the Salas and Hille treatment, instead of producing a scalar
%equation, lets remove the \(\Bv\) term from the equation:
%
%\[
%\rcap + \frac{1}{m \rho} \Bv \BJ = \Be
%\]
%
%First express \(\Bv\) in terms of \(\Br\) and \(\BJ\).
%
%\begin{align*}
%\BJ
%   &= \Br \wedge (m \Bv) \\
%   &= m \Br \Bv - m \Br \cdot \Bv \\
%   &= m \Br \Bv \\
%\end{align*}
%
%Thus,
%\[
%\Bv = \frac{1}{m \Br}\BJ
%\]
%
%%r_0 - \Br_0 \Be = \frac{1}{m^2 \rho} J^2)
%%-r_0 + \Br_0 \Be = -\frac{1}{m^2 \rho} J^2)
%%\Br_0 \Be = r_0 - \frac{1}{m^2 \rho} J^2)
%%\Be = (1/\Br_0)(r_0 - \frac{1}{m^2 \rho} J^2))
%
%And,
%\begin{align*}
%\rcap + \frac{1}{m^2 \rho \Br} \BJ^2 &= \Be \\
%\rcap(1 - \frac{1}{m^2 \rho r} J^2) &= \Be \\
%r - \frac{1}{m^2 \rho} J^2 &= \Br \Be \\
%r - \Br \Be &= \frac{1}{m^2 \rho} J^2 \\
%r - \Br (1/\Br_0)(r_0 - \frac{1}{m^2 \rho} J^2) &= \frac{1}{m^2 \rho} J^2 \\
%r - \Br \rcap_0 + \Br (1/\Br_0) \frac{1}{m^2 \rho} J^2 &= \frac{1}{m^2 \rho} J^2 \\
%\Br \rcap - \Br \rcap_0 + \Br (1/\Br_0) \frac{1}{m^2 \rho} J^2 &= \frac{1}{m^2 \rho} J^2 \\
%\Br (\rcap - \rcap_0 + (1/\Br_0) \frac{1}{m^2 \rho} J^2) &= \frac{1}{m^2 \rho} J^2 \\
%\end{align*}
%
%Since \(\Br \cdot \Bv = 0\) then:
%
%\begin{align*}
%J^2 &= -(\Br \wedge m \Bv)^2 \\
%    &= m^2 (\Br \wedge \Bv)(\Bv \wedge \Br) \\
%    &= m^2 (\Br \Bv)(\Bv \Br) \\
%    &= m^2 \Br^2 \Bv^2 \\
%    &= m^2 \Br_0^2 \Bv_0^2 \\
%\end{align*}
%
%\begin{align*}
%\Br (\rcap - \rcap_0 + (1/\Br_0) \frac{1}{\rho} \Br_0^2 \Bv_0^2) &= \frac{1}{\rho} \Br_0^2 \Bv_0^2 \\
%\Br (\rcap - \rcap_0 + \frac{1}{\rho} \Br_0 \Bv_0^2) &= \frac{1}{\rho} \Br_0^2 \Bv_0^2 \\
%\end{align*}

\subsection{Circular motion}
\index{circular motion}

For circular motion \(v_r = a_r = 0\), so:

\begin{equation}\label{eqn:angularVelocity:240}
\Bv = \Br \Bomega
\end{equation}
\begin{equation}\label{eqn:angularVelocity:260}
\Ba = \rcap \left(  -\frac{\Bv^2}{r} + \rcap \wedge \Ba \right) \\
\end{equation}

For constant circular motion:
\begin{equation}\label{eqn:angularVelocity:740}
\begin{aligned}
\Ba
   &= \Bv\Bomega + \Br\Bomega' \\
   &= \Bv\Bomega + \Br(\Bzero) \\
   &= \Br(\Bomega)^2 \\
   &= -\Br\Abs{\Bomega}^2 \\
\end{aligned}
\end{equation}

ie: the \(\rcap (\rcap \wedge \Ba )\) term is zero... all acceleration is inwards.

Can also expand this in terms of \(\Br\) and \(\Bv\):
\begin{equation}\label{eqn:angularVelocity:760}
\begin{aligned}
\Ba
   &= \Br\left(\Bomega\right)^2 \\
   &= \Br\left(\frac{1}{\Br}\Bv\right)^2 \\
   &= -\Br\left( \Bv \frac{1}{\Br} \frac{1}{\Br}\Bv \right) \\
   &= -\Br\left( \frac{\Bv^2}{\Br^2}\right) \\
   &= -\frac{1}{\Br}\Bv^2 \\
\end{aligned}
\end{equation}


   \chapter{Bivector Geometry}
      %
% Copyright � 2012 Peeter Joot.  All Rights Reserved.
% Licenced as described in the file LICENSE under the root directory of this GIT repository.
%

%
%
%\input{../latex/peeter_prologue.tex}

%\chapter{Bivector Geometry}
\index{bivector}
\label{chap:bivector}

%\blogpage{http://sites.google.com/site/peeterjoot/geometric-algebra/bivector.pdf?revision=2}
%\revisionInfo{ March 9, 2008.  \(RCSfile: bivector.tex,v \) Last \(Revision: 1.16 \) \(Date: 2009/10/22 02:07:20 \) }

%\beginArtWithToc

\section{Motivation}

Consider the derivative of a vector parametrized bivector square such as

\begin{equation}\label{eqn:bivector:23}
\begin{aligned}
\frac{d}{d\lambda} {(\Bx \wedge \Bk)^2} =
\left(\frac{d\Bx}{d\lambda} \wedge \Bk\right) \left(\Bx \wedge \Bk\right)
+\left(\Bx \wedge \Bk\right) \left(\frac{d \Bx}{d\lambda} \wedge \Bk\right)
\end{aligned}
\end{equation}

where \(\Bk\) is constant.  In this case, the left hand side is a scalar so the right hand side, this symmetric product of bivectors must also be a scalar.  In the more general case, do we have any reason to assume a symmetric bivector product is a scalar as is the case for the symmetric vector product?

Here this question is considered, and examination of products of intersecting bivectors is examined.  We take intersecting bivectors to mean that there a common vector (\(\Bk\) above) can be factored from both of the two bivectors, leaving a vector remainder.  Since all non coplanar bivectors in \R{3} intersect this examination will cover the important special case of three dimensional plane geometry.

A result of this examination is that many of the concepts familiar from vector geometry such as orthogonality, projection, and rejection will have direct bivector equivalents.

General bivector geometry, in spaces where non-coplanar bivectors do not necessarily intersect (such as in \R{4}) is also considered.  Some of the results require plane intersection, or become simpler in such circumstances.  This will be pointed out when appropriate.

\section{Components of grade two multivector product}

The geometric product of two bivectors can be written:

\begin{equation}
\label{eqn:bivector:ABprod}
\BA \BB =
\gpgrade{\BA \BB}{0}
+\gpgrade{\BA \BB}{2}
+\gpgrade{\BA \BB}{4}
=
{\BA \cdot \BB}
+\gpgrade{\BA \BB}{2}
+{\BA \wedge \BB}
\end{equation}
\begin{equation}
\label{eqn:bivector:BAprod}
\BB \BA =
\gpgrade{\BB \BA}{0}
+\gpgrade{\BB \BA}{2}
+\gpgrade{\BB \BA}{4}
=
{\BB \cdot \BA}
+\gpgrade{\BB \BA}{2}
+{\BB \wedge \BA}
\end{equation}

Unlike the vector dot and wedge products we cannot generally separate the grades of a bivector product using symmetric and antisymmetric sums.
One of the symmetric or antisymmetric sums must contain more than one grade.
We will see that the antisymmetric sum of bivectors is a bivector, whereas the symmetric sum of bivectors may contain both grades 0 and 4.

To understand all the possible grades that can be found in a product of bivectors, consider the following coordinate expansion of a pair of bivectors and their product.
%However forming those sums will still worthwhile, especially for the case of intersecting bivectors since the last term will be zero in that case.
Let

\begin{dmath}\label{eqn:bivector:1323}
\begin{aligned}
\BA &= \sum_{ij \in [1,N]} a_{ij} \Be_{ij} \lr{ 1 - \delta_{ij} } \\
\BB &= \sum_{ij \in [1,N]} b_{ij} \Be_{ij} \lr{ 1 - \delta_{ij} }.
\end{aligned}
\end{dmath}

The product \( \BA \BB \) is

\begin{dmath}\label{eqn:bivector:1343}
\BA \BB
=
\sum_{qrst \in [1,N]}
a_{qr}
b_{st}
\lr{ 1 - \delta_{qr} }
\lr{ 1 - \delta_{st} }
\Be_{qrst}
.
\end{dmath}

The grades of this multivector are determined by the grades of \( \Be_q \Be_r \Be_s \Be_t \).
Whenever any two adjacent indices are not equal, those unit vectors may be anticommuted, but whenever they are equal, the vector products result in a scalar.
This means that only even grades are possible, so a bivector product may only have grades 0, 2, or 4.

\subsection{Simple case.  When the bivectors are blades.}

If a bivector is a 2-blade (i.e. a wedge of two vectors), then it can be written as the product of two normal vectors, such as

\begin{dmath}\label{eqn:bivector:1123}
\BA = \Bm \Bn.
\end{dmath}

In this case the square of a bivector

\begin{dmath}\label{eqn:bivector:1143}
\BA^2
= \Bm \Bn \Bm \Bn
= - \Bm^2 \Bn^2,
\end{dmath}

is always a scalar.  For products of two 2-blades, there are two cases of interest.  The first is when there is a line of intersection between the two blades, and the second when all factors of the blades are normal.

\begin{enumerate}[(a)]
\item Intersecting case.  Let

\begin{equation}\label{eqn:bivector:1163}
\begin{aligned}
\BA &= \Bm \Bn \\
\BB &= \Bm \Bp,
\end{aligned}
\end{equation}

where \( \Bm \cdot \Bn = \Bm \cdot \Bp = 0 \).
Here we have

\begin{dmath}\label{eqn:bivector:1183}
\begin{aligned}
\BA \BB &= \Bm \Bn \Bm \Bp = -\Bm^2 \Bn \Bp \\
\BB \BA &= \Bm \Bp \Bm \Bn = -\Bm^2 \Bp \Bn,
\end{aligned}
\end{dmath}

so
\begin{dmath}\label{eqn:bivector:1203}
\begin{aligned}
\inv{2} \lr{ \BA \BB + \BB \BA }
&= -\Bm^2 \inv{2} \lr{ \Bn \Bp + \Bp \Bn }
= - \Bm^2 \lr{ \Bn \cdot \Bp } \\
\inv{2} \lr{ \BA \BB - \BB \BA } &= -\Bm^2 \inv{2} \lr{ \Bn \Bp - \Bp \Bn } = - \Bm^2 \lr{ \Bn \wedge \Bp } \\
\end{aligned}
\end{dmath}

The symmetric sum of two intersecting 2-blades is a scalar, whereas the antisymmetric sum is a 2-blade.

By applying the scalar grade selection operator to \cref{eqn:bivector:1183}, and from the definition \( \BA \cdot \BB = \gpgradezero{ \BA \BB } \), for intersecting 2-blades, we also have

\begin{dmath}\label{eqn:bivector:1403}
\begin{aligned}
\BA \cdot \BB &= \gpgradezero{\Bm \Bn \Bm \Bp} = -\Bm^2 \Bn \cdot \Bp = \inv{2} \lr{ \BA \BB + \BB \BA } \\
\BB \cdot \BA &= \gpgradezero{\Bm \Bp \Bm \Bn} = -\Bm^2 \Bp \cdot \Bn = \inv{2} \lr{ \BA \BB + \BB \BA }.
\end{aligned}
\end{dmath}

\item Non-intersecting case.  This case is only possible in spaces with dimension greater than 3.  Let

\begin{equation}\label{eqn:bivector:1223}
\begin{aligned}
\BA &= \Bm \Bn \\
\BB &= \Bs \Bt,
\end{aligned}
\end{equation}

where all of \( \Bm, \Bn, \Bs, \Bt \) are normal.  The products of these 2-blades are
\begin{dmath}\label{eqn:bivector:1243}
\BA \BB = \Bm \Bn \Bs \Bt,
\end{dmath}

and
\begin{dmath}\label{eqn:bivector:1263}
\BB \BA
= \Bs \Bt \Bm \Bn
= \Bm \Bs \Bt \Bn
= \Bm \Bn \Bs \Bt
= \BA \BB.
\end{dmath}

So, we have
\begin{dmath}\label{eqn:bivector:1283}
\begin{aligned}
\inv{2} \lr{ \BA \BB + \BB \BA } &= \gpgrade{ \BA \BB }{4} \\
\inv{2} \lr{ \BA \BB - \BB \BA } &= 0.
\end{aligned}
\end{dmath}

The symmetric sum of non-intersecting 2-blades is a 4-blade, but the antisymetric sum is zero.
\end{enumerate}

From this, we can conclude that for 2-blades \( \BA, \BB \)

\index{grade selection}
\begin{equation}
\label{eqn:bivector:AdotB}
\BA \cdot \BB = \gpgrade{\frac{\BA \BB + \BB\BA}{2}}{0}
\end{equation}

\begin{equation}
\label{eqn:bivector:AtwoB}
\gpgradetwo{\BA \BB} = \frac{\BA \BB - \BB\BA}{2}
\end{equation}

\begin{equation}
\label{eqn:bivector:AwedgeB}
\BA \wedge \BB = \gpgrade{\frac{\BA \BB + \BB\BA}{2}}{4}
\end{equation}

When these intersect in a line the wedge term is zero, so for that special case we can write:

\begin{equation*}
\BA \cdot \BB = \frac{\BA \BB + \BB\BA}{2}
\end{equation*}

\begin{equation*}
\gpgradetwo{\BA \BB} = \frac{\BA \BB - \BB\BA}{2}
\end{equation*}

\begin{equation*}
\BA \wedge \BB = 0
\end{equation*}

(note that this is always the case for \R{3}).

\subsection{General bivector case.}

A general bivector can not neccessarily be written as a wedge product.  One such example is

\begin{dmath}\label{eqn:bivector:1303}
\BA = \Be_{12} + \Be_{34}.
\end{dmath}

There is no common factor in this bivector, so it cannot be expressed as product of normal vectors, or equivalently, as a wedge product.  To understand the effects of commutation with a pair of general bivectors consider the coordinate representation of the two general bivectors defined in
\cref{eqn:bivector:1323}, and their product \cref{eqn:bivector:1343}.
Considering the commutation behaviour of the possible grades of the multivector \( \Be_{qrst} \) is sufficient.

\begin{itemize}
\item 0.  Whenever \( qr = st \), or \( qr = ts \), then
\( \Be_{qrst} \) is a scalar.
In this case
\( \Be_{qrst} = \Be_{stqr} \).
\item 2.  Whenever one but not both of \( q \) or \( r \) equals \( s \) or \( t \), then
\( \Be_{qrst} \) is a 2-blade.
In this case
\( \Be_{qrst} = -\Be_{stqr} \).
\item 4.  Whenever
\( qrst \) are all distinct,
then \( \Be_{qrst} \) is a 4-blade.
In this case
\( \Be_{qrst} = \Be_{stqr} \).
\end{itemize}

We can conclude that the grades 0 and 4 components of the product \( \BA \BB \) commute, whereas the grade 2 component anticommutes.
This means that

\begin{dmath}\label{eqn:bivector:1363}
\begin{aligned}
\inv{2} \lr{ \BA \BB + \BB \BA } &=
\gpgradezero{ \BA \BB }
+
\gpgrade{ \BA \BB }{4}
=
\BA \cdot \BB + \BA \wedge \BB \\
\inv{2} \lr{ \BA \BB - \BB \BA } &=
\gpgrade{ \BA \BB }{2},
\end{aligned}
\end{dmath}

and that

\begin{dmath}\label{eqn:bivector:1383}
\begin{aligned}
\BA \cdot \BB          &= \BB \cdot \BA  \\
\BA \wedge \BB         &= \BB \wedge \BA \\
\gpgrade{ \BA \BB }{2} &= -\gpgrade{ \BB \BA }{2}.
\end{aligned}
\end{dmath}

%  Mohamed Karimullah <moh_imt_kar@hotmail.com> :
%
% "What this is saying is that A�B + B�A must commute with interchange of A and B; which clearly it is.  Therefore, some steps are missing in order to make the final conclusion, namely that A�B = B�A. "
%
% (rewrote this subsection above in response.)
%
%\subsection{ORIGINAL TEXT.  REWRITE.}
%\subsubsection{Sign change of each grade term with commutation}
%
%Starting with the last term we can first observe that
%
%\begin{equation}
%\label{eqn:bivector:wedgesign}
%\BA \wedge \BB = \BB \wedge \BA
%\end{equation}
%
%To show this let \(\BA = \Ba \wedge \Bb\), and \(\BB = \Bc \wedge \Bd\).  When
%
%\(\BA \wedge \BB \ne 0\), one can write:
%
%\begin{equation}\label{eqn:bivector:43}
%\begin{aligned}
%\BA \wedge \BB
%&= \Ba \wedge \Bb \wedge \Bc \wedge \Bd \\
%&= - \Bb \wedge \Bc \wedge \Bd \wedge \Ba \\
%&= \Bc \wedge \Bd \wedge \Ba \wedge \Bb \\
%&= \BB \wedge \BA \\
%\end{aligned}
%\end{equation}
%
%To see how the signs of the remaining two terms vary with commutation form:
%
%\begin{equation}\label{eqn:bivector:63}
%\begin{aligned}
%(\BA + \BB)^2
%&= (\BA + \BB)(\BA + \BB) \\
%&= \BA^2 + \BB^2 + \BA \BB + \BB \BA \\
%\end{aligned}
%\end{equation}
%
%When \(\BA\) and \(\BB\) intersect we can write \(\BA = \Ba \wedge \Bx\), and \(\BB = \Bb \wedge \Bx\), thus the sum is a bivector
%
%\begin{equation}\label{eqn:bivector:83}
%\begin{aligned}
%(\BA + \BB)
%= (\Ba + \Bb) \wedge \Bx
%\end{aligned}
%\end{equation}
%
%And so, the square of the two is a scalar.  When \(\BA\) and \(\BB\) have only non intersecting components, such as the grade two \R{4} multivector \(\Be_{12} + \Be_{34}\), the square of this sum will have both grade four and scalar parts.
%
%Since the LHS = RHS, and the grades of the two also must be the same.  This implies that the quantity
%
%\begin{equation}\label{eqn:bivector:103}
%\begin{aligned}
%\BA \BB + \BB \BA =
%\BA \cdot \BB + \BB \cdot \BA
%+\gpgradetwo{\BA \BB} + \gpgradetwo{\BB \BA}
%+\BA \wedge \BB + \BB \wedge \BA
%\end{aligned}
%\end{equation}
%
%is a scalar \(\iff\) \(\BA + \BB\) is a bivector, and in general has scalar and grade four terms.  Because this symmetric sum has no grade two terms, regardless of whether \(\BA\), and \(\BB\) intersect, we have:
%
%\begin{equation}\label{eqn:bivector:123}
%\begin{aligned}
%\gpgradetwo{\BA \BB} + \gpgradetwo{\BB \BA} = 0
%\end{aligned}
%\end{equation}
%\begin{equation}
%\label{eqn:bivector:signgradetwo}
%\implies
%\gpgradetwo{\BA \BB} = -\gpgradetwo{\BB \BA}
%\end{equation}
%
%One would intuitively expect \(\BA \cdot \BB = \BB \cdot \BA\).  This can be demonstrated by forming the complete symmetric sum
%
%\begin{equation}\label{eqn:bivector:143}
%\begin{aligned}
%\BA \BB + \BB \BA
%&=
%{\BA \cdot \BB} +{\BB \cdot \BA}
%+\gpgrade{\BA \BB}{2} +\gpgrade{\BB \BA}{2}
%+{\BA \wedge \BB} + {\BB \wedge \BA} \\
%&=
%{\BA \cdot \BB} +{\BB \cdot \BA}
%+\gpgrade{\BA \BB}{2} -\gpgrade{\BA \BB}{2}
%+{\BA \wedge \BB} + {\BA \wedge \BB} \\
%&=
%{\BA \cdot \BB} +{\BB \cdot \BA}
%+2{\BA \wedge \BB} \\
%\end{aligned}
%\end{equation}
%
%The LHS commutes with interchange of \(\BA\) and \(\BB\), as does \({\BA \wedge \BB}\).  So for the RHS to also commute, the remaining grade 0 term must also:
%
%\begin{equation}
%\label{eqn:bivector:dotsign}
%\BA \cdot \BB = \BB \cdot \BA
%\end{equation}
%
\section{Intersection of planes}

Starting with two planes specified parametrically, each in terms of two direction vectors and a point on the plane:

\begin{equation}
\label{eqn:bivector:twoplanes}
\begin{aligned}
\Bx &= \Bp + \alpha \Bu + \beta \Bv \\
\By &= \Bq + a \Bw + b \Bz
\end{aligned}
\end{equation}

If these intersect then all points on the line must satisfy \(\Bx = \By\), so the solution requires:

\begin{equation}\label{eqn:bivector:183}
\begin{aligned}
\Bp + \alpha \Bu + \beta \Bv = \Bq + a \Bw + b \Bz
\end{aligned}
\end{equation}
\begin{equation}\label{eqn:bivector:203}
\begin{aligned}
\implies
(\Bp + \alpha \Bu + \beta \Bv) \wedge \Bw \wedge \Bz = (\Bq + a \Bw + b \Bz) \wedge \Bw \wedge \Bz = \Bq \wedge \Bw \wedge \Bz
\end{aligned}
\end{equation}

Rearranging for \(\beta\), and writing \(\BB = \Bw \wedge \Bz\):

\begin{equation}\label{eqn:bivector:223}
\begin{aligned}
\beta = \frac{\Bq \wedge \BB - (\Bp + \alpha \Bu) \wedge \BB}{\Bv \wedge \BB}
\end{aligned}
\end{equation}

Note that when the solution exists the left vs right order of the division by \(\Bv \wedge \BB\) should not matter since the numerator will be proportional to this bivector (or else the \(\beta\) would not be a scalar).

Substitution of \(\beta\) back into \(\Bx = \Bp + \alpha \Bu + \beta \Bv\) (all points in the first plane) gives you a parametric equation for a line:

\begin{equation}\label{eqn:bivector:243}
\begin{aligned}
\Bx = \Bp + \frac{(\Bq-\Bp)\wedge \BB}{\Bv \wedge \BB}\Bv + \alpha\frac{1}{\Bv \wedge \BB}((\Bv \wedge \BB) \Bu - (\Bu \wedge \BB)\Bv)
\end{aligned}
\end{equation}

Where a point on the line is:

\begin{equation}\label{eqn:bivector:263}
\begin{aligned}
\Bp + \frac{(\Bq-\Bp)\wedge \BB}{\Bv \wedge \BB}\Bv
%= \frac{1}{\Bv \wedge \BB}((\Bv \wedge \BB)\Bp + ((\Bq-\Bp)\wedge \BB)\Bv)
\end{aligned}
\end{equation}

And a direction vector for the line is:

\begin{equation}\label{eqn:bivector:283}
\begin{aligned}
\frac{1}{\Bv \wedge \BB}((\Bv \wedge \BB) \Bu - (\Bu \wedge \BB)\Bv)
\end{aligned}
\end{equation}
\begin{equation}\label{eqn:bivector:303}
\begin{aligned}
\propto
(\Bv \wedge \BB)^2 \Bu - (\Bv \wedge \BB)(\Bu \wedge \BB)\Bv
\end{aligned}
\end{equation}

Now, this result is only valid if \(\Bv \wedge \BB \ne 0\) (ie: line of intersection is not directed along \(\Bv\)), but if that is the case the second form will be zero.  Thus we can add the results (or any non-zero linear combination of) allowing for either of \(\Bu\), or \(\Bv\) to be directed along the line of intersection:

\begin{equation}
\label{eqn:bivector:dirvecintersection}
a\left( (\Bv \wedge \BB)^2 \Bu
- (\Bv \wedge \BB)(\Bu \wedge \BB)\Bv \right)
+ b\left((\Bu \wedge \BB)^2 \Bv
- (\Bu \wedge \BB)(\Bv \wedge \BB)\Bu\right)
\end{equation}

Alternately, one could formulate this in terms of \(\BA = \Bu \wedge \Bv\), \(\Bw\), and \(\Bz\).  Is there a more symmetrical form for this direction vector?

\subsection{Vector along line of intersection in \texorpdfstring{\R{3}}{3D Euclidean spaces}}

For \R{3} one can solve the intersection problem using the normals to the planes.  For simplicity put the origin on the line of intersection (and all planes through a common point in \R{3} have at least a line of intersection).  In this case, for bivectors \(\BA\) and \(\BB\), normals to those planes are \(i\BA\), and \(i\BB\) respectively.  The plane through both of those normals is:

\begin{equation}\label{eqn:bivector:323}
\begin{aligned}
(i\BA) \wedge (i\BB)
= \frac{(i\BA)(i\BB) - (i\BB)(i\BA)}{2}
= \frac{\BB\BA - \BA\BB}{2}
= \gpgradetwo{\BB\BA}
\end{aligned}
\end{equation}

The normal to this plane

\begin{equation}
\label{eqn:bivector:r3planeintersect}
i\gpgradetwo{\BB\BA}
\end{equation}

is directed along the line of intersection.  This result is more appealing than the general \R{N} result of \eqnref{eqn:bivector:dirvecintersection}, not just because it is simpler, but also because it is a function of only the bivectors for the planes, without a requirement to find or calculate two specific independent direction vectors in one of the planes.

\subsection{Applying this result to \texorpdfstring{\R{N}}{ND Euclidean spaces}}

If you reject the component of \(\BA\) from \(\BB\) for two intersecting bivectors:

\begin{equation}\label{eqn:bivector:343}
\begin{aligned}
\RejName_{\BA}(\BB) = \frac{1}{\BA}\gpgradetwo{\BA\BB}
\end{aligned}
\end{equation}

the line of intersection remains the same ... that operation rotates \(\BB\) so that the two are mutually perpendicular.  This essentially reduces the problem to that of the three dimensional case, so the solution has to be of the same form... you just need to calculate a ``pseudoscalar'' (what you are calling the join), for the subspace spanned by the two bivectors.

That can be computed by taking any direction vector that is on one plane, but is not in the second.  For example, pick a vector \(\Bu\) in the plane \(\BA\) that is not on the intersection of \(\BA\) and \(\BB\).  In mathese that is \(\Bu = \inv{\BA}(\BA\cdot \Bu)\) (or \(\Bu \wedge \BA = 0\)), where \(\Bu \wedge \BB \ne 0\).  Thus a pseudoscalar for this subspace is:

\begin{equation}\label{eqn:bivector:363}
\begin{aligned}
\Bi = \frac{\Bu \wedge \BB}{\Abs{\Bu \wedge \BB}}
\end{aligned}
\end{equation}

To calculate the direction vector along the intersection we do not care about the scaling above.  Also note that provided \(\Bu\) has a component in the plane \(\BA\), \(\Bu \cdot \BA\) is also in the plane (it is rotated \(\pi/2\) from \(\inv{\BA}(\BA \cdot \Bu)\).

Thus, provided that \(\Bu \cdot \BA\) is not on the intersection, a scaled ``pseudoscalar''
for the subspace can be calculated by taking from any vector \(\Bu\) with a component in the plane \(\BA\):

\begin{equation}\label{eqn:bivector:383}
\begin{aligned}
\Bi \propto (\Bu \cdot \BA) \wedge \BB
\end{aligned}
\end{equation}

Thus a vector along the intersection is:

\begin{equation}
\label{eqn:bivector:pseudoscalarinter}
\Bd = ((\Bu \cdot \BA) \wedge \BB) \gpgradetwo{\BA\BB}
\end{equation}

Interchange of \(\BA\) and \(\BB\) in either the trivector or bivector terms above would also work.

Without showing the steps one can write the complete parametric solution of the line through the planes of equations \eqnref{eqn:bivector:twoplanes} in terms of this direction vector:

\begin{equation}
\label{eqn:bivector:finalsolnofRNplaneintersection}
\Bx = \Bp + \left(\frac{(\Bq - \Bp)\wedge \BB}{(\Bd \cdot \BA) \wedge \BB}\right) (\Bd \cdot \BA) + \alpha \Bd
\end{equation}

Since \((\Bd \cdot \BA) \ne 0\) and \((\Bd \cdot \BA) \wedge \BB \ne 0\) (unless \(\BA\) and \(\BB\) are coplanar), observe that this is a natural generator of the pseudoscalar for the subspace, and as such shows up in the expression above.

Also observe the non-coincidental similarity of the \(\Bq-\Bp\) term to Cramer's rule (a ration of determinants).

\section{Components of a grade two multivector}

The procedure to calculate projections and rejections of planes onto planes is similar to a vector projection onto a space.

To arrive at that result we can consider the product of a grade two multivector \(\BA\) with a bivector \(\BB\) and its inverse (
the restriction that \(\BB\) be a bivector, a grade two multivector that can be written as a wedge product of two vectors, is required for general invertability).

\begin{equation}\label{eqn:bivector:403}
\begin{aligned}
\BA\inv{\BB}\BB
&= \left(\BA \cdot \inv{\BB} + \gpgradetwo{ \BA \inv{\BB} } + \BA \wedge \inv{\BB}\right) \BB \\
&=
\BA \cdot \inv{\BB} \BB \\
&
+\gpgradetwo{ \BA \inv{\BB} } \cdot \BB
+\gpgradetwo{ \gpgradetwo{ \BA \inv{\BB} } \BB }
+\gpgradetwo{ \BA \inv{\BB} } \wedge \BB \\
&
+\left(\BA \wedge \inv{\BB}\right) \cdot \BB
+\gpgradefour{\BA \wedge \inv{\BB} \BB}
+\BA \wedge \inv{\BB} \wedge \BB \\
\end{aligned}
\end{equation}

Since \(\inv{\BB} = -\frac{\BB}{{\Abs{\BB}}^2}\), this implies that the 6-grade term \(\BA \wedge \inv{\BB} \wedge \BB\) is zero.  Since the LHS has grade 2, this implies that the 0-grade and 4-grade terms are zero (also independently implies that the 6-grade term is zero).  This leaves:

\begin{equation}
\label{eqn:bivector:bivectorprojbivector}
\BA
=
\BA \cdot \inv{\BB} \BB \\
+\gpgradetwo{\gpgradetwo{\BA\inv{\BB}} \BB}
+\left(\BA \wedge \inv{\BB}\right) \cdot \BB
\end{equation}

This could be written somewhat more symmetrically as
\begin{equation}\label{eqn:bivector:423}
\begin{aligned}
\BA
&=\sum_{i=0,2,4}\gpgradetwo{\gpgrade{\BA \inv{\BB}}{i} \BB} \\
&= \gpgradetwo{ \gpgradezero{\BA \inv{\BB}} \BB +\gpgradetwo{\BA \inv{\BB}} \BB +\gpgradefour{\BA \inv{\BB}} \BB } \\
\end{aligned}
\end{equation}

This is also a more direct way to derive the result in retrospect.

Looking at \eqnref{eqn:bivector:bivectorprojbivector} we have three terms.  The first is

\begin{equation}
\label{eqn:bivector:foo1}
\begin{aligned}
\BA \cdot \inv{\BB} \BB
\end{aligned}
\end{equation}

This is the component of \(\BA\) that lies in the plane \(\BB\) (the projection of \(\BA\) onto \(\BB\)).

The next is
\begin{equation}
\label{eqn:bivector:gpgradetwo}
\gpgradetwo{\gpgradetwo{\BA\inv{\BB}} \BB}
\end{equation}

If \(\BB\) and \(\BA\) have any intersecting components, this is the components of \(\BA\) from the intersection that are perpendicular to \(\BB\) with respect to the bivector dot product.  ie: This is the rejective term.

And finally,

\begin{equation}
\label{eqn:bivector:foo2}
\begin{aligned}
\left(\BA \wedge \inv{\BB}\right) \cdot \BB
\end{aligned}
\end{equation}

This is the remainder, the non-projective and non-coplanar terms.  Greater than three dimensions is required to generate such a term.  Example:

\begin{equation}\label{eqn:bivector:483}
\begin{aligned}
\BA &= \Be_{12} + \Be_{23} + \Be_{43} \\
\BB &= \Be_{34} \\
\end{aligned}
\end{equation}

Product terms for these are:

\begin{equation}\label{eqn:bivector:503}
\begin{aligned}
\BA \cdot \BB &= 1 \\
\gpgradetwo{\BA \BB} &= \Be_{24} \\
\BA \wedge \BB &= \Be_{1234} \\
\end{aligned}
\end{equation}

The decomposition is thus:
\begin{equation}\label{eqn:bivector:523}
\begin{aligned}
\BA = \left(\BA \cdot \BB + \gpgradetwo{\BA \BB} + \BA \wedge \BB\right) \inv{\BB} = (1 + \Be_{24} + \Be_{1234}) \Be_{43}
\end{aligned}
\end{equation}

\subsection{Closer look at the grade two term}

The grade two term of \eqnref{eqn:bivector:gpgradetwo} can be expanded using its antisymmetric bivector product representation

\begin{equation}\label{eqn:bivector:543}
\begin{aligned}
\gpgradetwo{\BA\inv{\BB}} \BB
&= \inv{2}\left(\BA\inv{\BB} - \inv{\BB}\BA\right) \BB \\
&= \inv{2}\left(\BA - \inv{\BB}\BA \BB\right) \\
&= \inv{2}\left(\BA - \inv{\hat{\BB}}\BA \hat{\BB}\right) \\
\end{aligned}
\end{equation}

Observe here one can restrict the examination to the case where \(\BB\) is a unit bivector without loss of generality.

\begin{equation}\label{eqn:bivector:563}
\begin{aligned}
\gpgradetwo{\BA\inv{\Bi}} \Bi
&= \inv{2}\left(\BA + \Bi\BA\Bi\right) \\
&= \inv{2}\left(\BA - \Bi^\dagger\BA\Bi\right) \\
\end{aligned}
\end{equation}

The second term is a rotation in the plane \(\Bi\), by 180 degrees:

\begin{equation}\label{eqn:bivector:583}
\begin{aligned}
\Bi^\dagger\BA\Bi = e^{-\Bi \pi/2}\BA e^{\Bi \pi/2}
\end{aligned}
\end{equation}

So, any components of \(\BA\) that are completely in the plane cancel out (ie: the \(\BA \cdot \inv{\Bi}\Bi\) component).

Also, if \(\gpgradefour{\BA \Bi} \ne 0\) then those components of \(\BA \Bi\) commute so

\begin{equation}\label{eqn:bivector:603}
\begin{aligned}
\gpgradefour{\BA - \Bi^\dagger\BA\Bi}
&= \gpgradefour{\BA} - \gpgradefour{\Bi^\dagger\BA\Bi} \\
&= \gpgradefour{\BA} - \gpgradefour{\Bi^\dagger\Bi\BA} \\
&= \gpgradefour{\BA} - \gpgradefour{\BA} \\
&= 0 \\
\end{aligned}
\end{equation}

This implies that we have only grade two terms, and the final grade selection in \eqnref{eqn:bivector:gpgradetwo} can be dropped:

\begin{equation}
\label{eqn:bivector:simplergpgradetwo}
\gpgradetwo{\gpgradetwo{\BA\inv{\BB}} \BB} = \gpgradetwo{\BA\inv{\BB}} \BB
\end{equation}

It is also possible to write this in a few alternate variations which are useful to list explicitly so that one can recognize them in other contexts:

\begin{equation}\label{eqn:bivector:623}
\begin{aligned}
\gpgradetwo{\BA\inv{\BB}} \BB
&= \inv{2}\left(\BA - \inv{\BB}\BA\BB\right)  \\
&= \inv{2}\left(\BA + \hat{\BB}\BA\hat{\BB}\right)  \\
&= \inv{2}\left( \hat{\BB}\BA -\BA\hat{\BB} \right)\hat{\BB} \\
&= \gpgradetwo{\hat{\BB}\BA}\hat{\BB} \\
&= \hat{\BB}\gpgradetwo{{\BA}\hat{\BB}} \\
\end{aligned}
\end{equation}

\subsection{Projection and Rejection}
\index{projection!bivector}
\index{rejection!bivector}

\Eqnref{eqn:bivector:simplergpgradetwo} can be substituted back into \eqnref{eqn:bivector:bivectorprojbivector} yielding:

\begin{equation}
\label{eqn:bivector:simplerbivectorprojbivector}
\BA =
\BA \cdot \inv{\BB} \BB \\
+\gpgradetwo{\BA\inv{\BB}} \BB
+\left(\BA \wedge \inv{\BB}\right) \cdot \BB
\end{equation}

Now, for the special case where \(\BA \wedge \BB = 0\) (all bivector components of the grade two multivector \(\BA\) have a common vector with bivector \(\BB\)) we can write

\begin{equation}\label{eqn:bivector:643}
\begin{aligned}
\BA
&= \BA \cdot \inv{\BB} \BB +\gpgradetwo{\BA\inv{\BB}} \BB \\
&= \BB \inv{\BB} \cdot {\BA} + \BB \gpgradetwo{\inv{\BB}\BA} \\
\end{aligned}
\end{equation}

This is
\begin{equation}
\label{eqn:bivector:projrejbivectorbivector}
\begin{aligned}
\BA = \Proj_{\BB}(\BA) + \RejName_{\BB}(\BA)
\end{aligned}
\end{equation}

It is worth verifying that these two terms are orthogonal (with respect to the grade two vector dot product)
\begin{equation}\label{eqn:bivector:683}
\begin{aligned}
\Proj_{\BB}(\BA) \cdot \RejName_{\BB}(\BA)
&= \gpgradezero{ \Proj_{\BB}(\BA) \RejName_{\BB}(\BA) } \\
&= \gpgradezero{ \BA \cdot \inv{\BB} \BB \BB \gpgradetwo{\inv{\BB}\BA} } \\
&= \inv{4\BB^2}\gpgradezero{ (\BA\BB + \BB\BA)(\BB\BA - \BA\BB) } \\
&= \inv{4\BB^2}\gpgradezero{ \BA\BB\BB\BA -\BA\BB\BA\BB +\BB\BA\BB\BA -\BB\BA\BA\BB } \\
&= \inv{4\BB^2}\gpgradezero{ -\BA\BB\BA\BB +\BB\BA\BB\BA } \\
\end{aligned}
\end{equation}

Since we have introduced the restriction \(\BA \wedge \BB \ne 0\), we can use the dot product to reorder product terms:

\begin{equation}\label{eqn:bivector:703}
\begin{aligned}
\BA\BB = -\BB\BA + 2 \BA \cdot \BB
\end{aligned}
\end{equation}

This can be used to reduce the grade zero term above:

\begin{equation}\label{eqn:bivector:723}
\begin{aligned}
\gpgradezero{ \BB\BA\BB\BA -\BA\BB\BA\BB }
&= \gpgradezero{ \BB\BA(-\BA\BB + 2 \BA \cdot \BB) -(-\BB\BA + 2 \BA \cdot \BB)\BA\BB } \\
&= + 2 (\BA \cdot \BB)\gpgradezero{\BB\BA - \BA\BB } \\
&= + 4 (\BA \cdot \BB)\gpgradezero{\gpgradetwo{\BB\BA}} \\
&= 0 \\
\end{aligned}
\end{equation}

This proves orthogonality as expected.

\subsection{Grade two term as a generator of rotations}
\index{rotation!generator}

\imageFigure{../figures/gabook/planerejection}{Bivector rejection.  Perpendicular component of plane}{fig:planerejection}{0.4}

\Cref{fig:planerejection} illustrates how the grade 2 component of the bivector product acts as a rotation in the rejection operation.

Provided that \(\BA\) and \(\BB\) are not coplanar, \(\gpgradetwo{\BA\BB}\) is a plane mutually perpendicular to both.

Given two mutually perpendicular unit bivectors \({\BA}\) and \({\BB}\), we can in fact write:

\begin{equation}\label{eqn:bivector:743}
\begin{aligned}
{\BB} = {\BA}\gpgradetwo{{\BB}{\BA}}
\end{aligned}
\end{equation}
\begin{equation}\label{eqn:bivector:763}
\begin{aligned}
{\BB} = \gpgradetwo{{\BA}{\BB}}{\BA}
\end{aligned}
\end{equation}

Compare this to a unit bivector for two mutually perpendicular vectors:

\begin{equation}\label{eqn:bivector:783}
\begin{aligned}
\Bb = \Ba (\Ba \wedge \Bb)
\end{aligned}
\end{equation}
\begin{equation}\label{eqn:bivector:803}
\begin{aligned}
\Bb = (\Bb \wedge \Ba) \Ba
\end{aligned}
\end{equation}

In both cases, the unit bivector functions as an imaginary number, applying a rotation of \(\pi/2\) rotating one of the perpendicular entities onto the other.

As with vectors one can split the rotation of the unit bivector into half angle left and right rotations.  For example, for the same mutually perpendicular pair of bivectors one can write

\begin{equation}\label{eqn:bivector:823}
\begin{aligned}
\BB
&= \BA\gpgradetwo{\BB \BA} \\
&= \BA e^{\gpgradetwo{\BB \BA}\pi/2} \\
&= e^{-\gpgradetwo{\BB \BA}\pi/4} \BA e^{\gpgradetwo{\BB \BA}\pi/4} \\
&= \left(\inv{\sqrt{2}}(1 - \BB \BA)\right) \BA \left(\inv{\sqrt{2}}(1 + \BB \BA) \right) \\
\end{aligned}
\end{equation}
%&= \inv{2}(\BA + \BB )(1 + \BB \BA)
%&= \inv{2}(\BA + \BB + \BA\BB\BA + \BB\BB\BA)
%&= \inv{2}(\BA + \BB - \BA\BA\BB + \BB\BB\BA)
%&= \inv{2}(\BA + \BB + \BB - \BA)

Direct multiplication can be used to verify that this does in fact produce the desired result.

In general, writing

\begin{equation}\label{eqn:bivector:843}
\begin{aligned}
\Bi = \frac{\gpgradetwo{\BB \BA}}{\Abs{\gpgradetwo{\BB \BA}}}
\end{aligned}
\end{equation}

the rotation of plane \(\BB\) towards \(\BA\) by angle \(\theta\) can be expressed with either a single sided full angle

\begin{equation}\label{eqn:bivector:863}
\begin{aligned}
\Rot_{\theta: \BA \rightarrow \BB}(\BA)
&= \BA e^{\Bi \theta} \\
&= e^{-\Bi \theta} \BA \\
\end{aligned}
\end{equation}

or double sided the half angle rotor formulas:

\begin{equation}
\label{eqn:bivector:rotor}
\Rot_{\theta: \BA \rightarrow \BB}(\BA) = e^{-\Bi \theta/2} \BA e^{\Bi \theta/2} = \BR^\dagger \BA \BR
\end{equation}

Where:
\begin{equation}\label{eqn:bivector:883}
\begin{aligned}
\BR
&= e^{\Bi\theta/2} \\
&= \cos(\theta/2) + \frac{\gpgradetwo{\BB \BA}}{\Abs{\gpgradetwo{\BB \BA}}}\sin(\theta/2) \\
\end{aligned}
\end{equation}

As with half angle rotors applied to vectors, there are two possible orientations to rotate.  Here the orientation of the rotation is such that the angle is measured along the minimal arc between the two, where the angle between the two is in the range \((0,\pi)\) as opposed to the \((\pi,2\pi)\) rotational direction.

\subsection{Angle between two intersecting planes}
\index{bivector!intersecting planes}

Worth pointing out for comparison to the vector result, one can use the bivector dot product to calculate the angle between two intersecting planes.  This angle of separation \(\theta\) between the two can be expressed using the exponential:

\begin{equation}\label{eqn:bivector:903}
\begin{aligned}
\hat{\BB} = \hat{\BA} e^{ \frac{\gpgradetwo{\BB \BA}}{\Abs{\gpgradetwo{\BB \BA}}} \theta}
\end{aligned}
\end{equation}
\begin{equation}\label{eqn:bivector:923}
\begin{aligned}
\implies
-\hat{\BA} \hat{\BB} = e^{ \frac{\gpgradetwo{\BB \BA}}{\Abs{\gpgradetwo{\BB \BA}}} \theta}
\end{aligned}
\end{equation}

Taking the grade zero terms of both sides we have:
\begin{equation}\label{eqn:bivector:943}
\begin{aligned}
-\gpgradezero{\hat{\BA} \hat{\BB}} = \gpgradezero{ e^{ \frac{\gpgradetwo{\BB \BA}}{\Abs{\gpgradetwo{\BB \BA}}} \theta} }
\end{aligned}
\end{equation}
\begin{equation}\label{eqn:bivector:963}
\begin{aligned}
\implies
\cos(\theta) = - \frac{\BA \cdot \BB}{\Abs{\BA}\Abs{\BB}}
\end{aligned}
\end{equation}

The sine can be obtained by selecting the grade two terms

\begin{equation}\label{eqn:bivector:983}
\begin{aligned}
-\gpgradetwo{\hat{\BA} \hat{\BB}} = \frac{\gpgradetwo{\BB \BA}}{\Abs{\gpgradetwo{\BB \BA}}} \sin(\theta)
\end{aligned}
\end{equation}
\begin{equation}\label{eqn:bivector:1003}
\begin{aligned}
%\inv{\Abs{\BA}\Abs{\BB}} = \frac{1}{\Abs{\gpgradetwo{\BB \BA}}} \sin(\theta)
\implies
\sin(\theta) = \frac{\Abs{\gpgradetwo{\BB \BA}}}{ \Abs{\BA}\Abs{\BB} }
\end{aligned}
\end{equation}

Note that the strictly positive sine result here is consistent with the fact that the angle is being measured such that it is in the
\((0,\pi)\) range.

\subsection{Rotation of an arbitrarily oriented plane}
\index{bivector!rotation}

As stated in a few of the GA books the rotor equation is a rotation representation that works for all grade vectors.  Let us verify this for the bivector case.  Given a plane through the origin spanned by two direction vectors and rotated about the origin in a plane specified by unit magnitude rotor \(\BR\), the rotated plane will be specified by the wedge of the rotations applied to the two direction vectors.  Let

\begin{equation}\label{eqn:bivector:1023}
\begin{aligned}
\BA = \Bu \wedge \Bv
\end{aligned}
\end{equation}

Then,

\begin{equation}\label{eqn:bivector:1043}
\begin{aligned}
R(\BA)
&= R(\Bu) \wedge R(\Bv) \\
&= (\BR^\dagger \Bu \BR) \wedge (\BR^\dagger \Bv \BR) \\
&= \inv{2}( \BR^\dagger \Bu \BR \BR^\dagger \Bv \BR - \BR^\dagger \Bv \BR \BR^\dagger \Bu \BR) \\
&= \inv{2}( \BR^\dagger \Bu \Bv \BR - \BR^\dagger \Bv \Bu \BR) \\
&= \BR^\dagger \frac{\Bu \Bv - \Bv \Bu}{2} \BR \\
&= \BR^\dagger \Bu \wedge \Bv \BR \\
&= \BR^\dagger \BA \BR \\
\end{aligned}
\end{equation}

Observe that with this half angle double sided rotation equation, any component of \(\BA\) in the plane of rotation, or any component that does not intersect the plane of rotation, will be unchanged by the rotor since it will commute with it.  In those cases the opposing sign half angle rotations will cancel out.  Only the components of the plane that are perpendicular to the rotational plane will be changed by this rotation operation.

\section{A couple of reduction formula equivalents from \texorpdfstring{\R{3}}{3D} vector geometry}

The reduction of the \R{3} dot of cross products to dot products can be naturally derived using GA arguments.  Writing \(i\) as the \R{3} pseudoscalar we have:

\begin{equation}\label{eqn:bivector:1063}
\begin{aligned}
( \Ba \cross \Bb ) \cdot ( \Bc \cross \Bd )
&= \frac{\Ba \wedge \Bb}{i} \cdot \frac{\Bc \wedge \Bd}{i} \\
&= \inv{2}\left( \frac{\Ba \wedge \Bb}{i} \frac{\Bc \wedge \Bd}{i} + \frac{\Bc \wedge \Bd}{i} \frac{\Ba \wedge \Bb}{i} \right) \\
&= -\inv{2}\left( (\Ba \wedge \Bb) (\Bc \wedge \Bd) + (\Bc \wedge \Bd) (\Ba \wedge \Bb) \right) \\
&= - (\Ba \wedge \Bb) \cdot (\Bc \wedge \Bd) - (\Ba \wedge \Bb) \wedge (\Bc \wedge \Bd)
\end{aligned}
\end{equation}

In \R{3} this last term must be zero, thus one can write

\begin{equation}
\label{eqn:bivector:boo1}
%\begin{aligned}
( \Ba \cross \Bb ) \cdot ( \Bc \cross \Bd ) = -(\Ba \wedge \Bb) \cdot (\Bc \wedge \Bd)
%\end{aligned}
\end{equation}

This is now in a form where it can be reduced to products of vector dot products.

\begin{equation}\label{eqn:bivector:1103}
\begin{aligned}
(\Ba \wedge \Bb) \cdot (\Bc \wedge \Bd)
&= \inv{2}\gpgradezero{ (\Ba \wedge \Bb) (\Bc \wedge \Bd) + (\Bc \wedge \Bd) (\Ba \wedge \Bb) } \\
&= \inv{2}\gpgradezero{ (\Ba \wedge \Bb) (\Bc \wedge \Bd) + (\Bd \wedge \Bc) (\Bb \wedge \Ba) } \\
&= \inv{2}\gpgradezero{ (\Ba\Bb - \Ba \cdot \Bb ) (\Bc \wedge \Bd) + (\Bd \wedge \Bc) (\Bb \Ba - \Bb \cdot \Ba ) } \\
&= \inv{2}\gpgradezero{ \Ba\Bb (\Bc \wedge \Bd) + (\Bd \wedge \Bc) \Bb \Ba } \\
&= \inv{2}\gpgradezero{ \Ba (\Bb \cdot (\Bc \wedge \Bd) + \Bb \wedge (\Bc \wedge \Bd)) ( (\Bd \wedge \Bc) \cdot \Bb + (\Bd \wedge \Bc) \wedge \Bb) \Ba } \\
&= \inv{2}\gpgradezero{ \Ba (\Bb \cdot (\Bc \wedge \Bd)) + ( (\Bd \wedge \Bc) \cdot \Bb ) \Ba } \\
&= \inv{2}\gpgradezero{ \Ba ( (\Bb \cdot \Bc) \Bd - (\Bb \cdot \Bd) \Bc ) + ( \Bd (\Bc \cdot \Bb) - \Bc (\Bd \cdot \Bb) ) \Ba } \\
&= \inv{2}( ( \Ba \cdot \Bd ) ( \Bb \cdot \Bc ) - ( \Bb \cdot \Bd ) ( \Ba \cdot \Bc ) + ( \Bd \cdot \Ba ) ( \Bc \cdot \Bb ) - ( \Bc \cdot \Ba ) ( \Bd \cdot \Bb ) ) \\
&= ( \Ba \cdot \Bd ) ( \Bb \cdot \Bc ) - ( \Ba \cdot \Bc ) ( \Bb \cdot \Bd ) \\
\end{aligned}
\end{equation}

Summarizing with a comparison to the \R{3} relations we have:

\begin{equation}
\label{eqn:bivector:boo2}
(\Ba \wedge \Bb) \cdot (\Bc \wedge \Bd) = -(\Ba \cross \Bb) \cdot (\Bc \cross \Bd) = ( \Ba \cdot \Bd ) ( \Bb \cdot \Bc ) - ( \Ba \cdot \Bc ) ( \Bb \cdot \Bd )
\end{equation}
\begin{equation}
\label{eqn:bivector:boo3}
(\Ba \wedge \Bc) \cdot (\Bb \wedge \Bc) = -(\Ba \cross \Bc) \cdot (\Bb \cross \Bc) = ( \Ba \cdot \Bc ) ( \Bb \cdot \Bc ) - \Bc^2 ( \Ba \cdot \Bb )
\end{equation}

The bivector relations hold for all of \R{N}.

%\EndNoBibArticle

   \chapter{Trivector geometry}
      %
% Copyright � 2012 Peeter Joot.  All Rights Reserved.
% Licenced as described in the file LICENSE under the root directory of this GIT repository.
%

%
%
%\chapter{Trivector geometry}
\index{trivector}
\label{chap:trivector}
%\date{Mar 9, 2008.  trivector.tex}

\section{Motivation}

The direction vector for two intersecting planes can be found to have the
form:
%
\begin{equation}\label{eqn:trivector:dirvecintersection}
a\left( (\Bv \wedge \BB)^2 \Bu
- (\Bv \wedge \BB)(\Bu \wedge \BB)\Bv \right)
+ b\left((\Bu \wedge \BB)^2 \Bv
- (\Bu \wedge \BB)(\Bv \wedge \BB)\Bu\right)
\end{equation}
%
While trying to put \eqnref{eqn:trivector:dirvecintersection} into a form
that eliminated \(\Bu\), and \(\Bv\) in favor of \(\BA = \Bu \wedge \Bv\)
symmetric and antisymmetric formulations for the various grade terms
of a trivector product looked like they could be handy.  Here is a summary
of those results.

\section{Grade components of a trivector product}

\subsection{Grade 6 term}

Writing two trivectors in terms
of mutually orthogonal components
%
\begin{equation}\label{eqn:trivector:26}
\BA = \Bx \wedge \By \wedge \Bz = \Bx\By\Bz
\end{equation}
%
and
%
\begin{equation}\label{eqn:trivector:46}
\BB = \Bu \wedge \Bv \wedge \Bw =\Bu\Bv\Bw
\end{equation}
%
Assuming that there is no common vector between the two, the
wedge of these is
%
\begin{equation}\label{eqn:trivector:186}
\begin{aligned}
\BA \wedge \BB
&= \gpgrade{\BA\BB}{6} \\
&= \gpgrade{\Bx\By\Bz\Bu\Bv\Bw}{6} \\
&= \gpgrade{\By\Bz(\Bx\Bu)\Bv\Bw}{6} \\
&= \gpgrade{\By\Bz(-\Bu\Bx + 2\Bu \cdot \Bx)\Bv\Bw}{6} \\
&= -\gpgrade{\By\Bz\Bu(\Bx\Bv)\Bw}{6} \\
&= -\gpgrade{\By\Bz\Bu(-\Bv\Bx + 2\Bv \cdot \Bx)\Bw}{6} \\
&= \gpgrade{\By\Bz\Bu\Bv(\Bx\Bw)}{6} \\
&= \cdots \\
&= -\gpgrade{\Bu\Bv\Bw\Bx\By\Bz}{6} \\
&= -\gpgrade{\BB\BA}{6} \\
&= -\BB \wedge \BA
\end{aligned}
\end{equation}
%
Note above that any interchange of terms inverts the sign (demonstrated
explicitly for all the \(\Bx\) interchanges).

As an aside, this
sign change on interchange is taken as the defining property of the
wedge product in differential forms.  That property also
implies also that the wedge product is
zero when a vector is wedged with itself since zero is the only
value that is the negation of itself.  Thus we see explicitly
how the notation of using the wedge for the highest grade term
of two blades is consistent with the traditional
wedge product definition.

The end result here is that the grade 6 term of a trivector trivector product
changes sign on interchange of the trivectors:
%
\begin{equation}\label{eqn:trivector:trivecgpgrade6}
\gpgrade{\BA\BB}{6} = -\gpgrade{\BB\BA}{6}
\end{equation}
%
\subsection{Grade 4 term}

For a trivector product to have a grade 4 term there must be a common
vector between the two
%
\begin{equation}\label{eqn:trivector:66}
\BA = \Bx \wedge \By \wedge \Bz = \Bx\By\Bz
\end{equation}
%
and
%
\begin{equation}\label{eqn:trivector:86}
\BB = \Bu \wedge \Bv \wedge \Bz =\Bu\Bv\Bz
\end{equation}
%
The grade four term of the product is
%
\begin{equation}\label{eqn:trivector:206}
\begin{aligned}
\gpgrade{\BB \BA}{4}
&= \gpgrade{ \Bu\Bv\Bz \Bx\By\Bz }{4} \\
&= \gpgrade{ \Bu\Bv\Bz \Bz\Bx\By }{4} \\
&= \Bz^2\gpgrade{ \Bu\Bv\Bx\By }{4} \\
&= \Bz^2\gpgrade{ \Bu(\Bv\Bx)\By }{4} \\
&= \Bz^2\gpgrade{ \Bu(-\Bx\Bv + 2 \Bx \cdot \Bv)\By }{4} \\
&= -\Bz^2\gpgrade{ \Bu\Bx\Bv\By }{4} \\
&= \cdots \\
&= \Bz^2\gpgrade{ \Bx\By\Bu\Bv }{4} \\
&= \gpgrade{ \Bx\By\Bz\Bz\Bu\Bv }{4} \\
&= \gpgrade{ \Bx\By\Bz\Bu\Bv\Bz }{4} \\
&= \gpgrade{ \Bx\By\Bz\Bu\Bv\Bz }{4} \\
&= \gpgrade{\BA \BB}{4}
\end{aligned}
\end{equation}
%
Thus the grade 4 term commutes on interchange:
%
\begin{equation}\label{eqn:trivector:trivecgpgrade4}
\gpgrade{\BA\BB}{4} = \gpgrade{\BB\BA}{4}
\end{equation}
%
\subsection{Grade 2 term}

Similar to above,
for a trivector product to have a grade 2 term there must be two common
vectors between the two
%
\begin{equation}\label{eqn:trivector:106}
\BA = \Bx \wedge \By \wedge \Bz = \Bx\By\Bz
\end{equation}
%
and
%
\begin{equation}\label{eqn:trivector:126}
\BB = \Bu \wedge \By \wedge \Bz =\Bu\By\Bz
\end{equation}
%
The grade two term of the product is
%
\begin{equation}\label{eqn:trivector:226}
\begin{aligned}
\gpgrade{\BA \BB}{2}
&= \gpgrade{ \Bx\By\Bz \Bu\By\Bz }{2} \\
&= \gpgrade{ \Bx\By\Bz \By\Bz \Bu}{2} \\
&= (\By\Bz)^2\gpgrade{ \Bx \Bu}{2} \\
&= -(\By\Bz)^2\gpgrade{ \Bu \Bx}{2} \\
&= -\gpgrade{ \BB \BA }{2} \\
\end{aligned}
\end{equation}
%
The grade 2 term anticommutes on interchange:
%
\begin{equation}\label{eqn:trivector:trivecgpgrade2}
\gpgrade{\BA\BB}{2} = -\gpgrade{\BB\BA}{2}
\end{equation}
%
\subsection{Grade 0 term}

Any grade 0 terms are due to products of the form \(\BA = k\BB\)
%
\begin{equation}\label{eqn:trivector:246}
\begin{aligned}
\gpgrade{\BA \BB}{0}
&= \gpgrade{k\BB \BB}{0} \\
&= \gpgrade{\BB k\BB}{0} \\
&= \gpgrade{\BB \BA}{0} \\
\end{aligned}
\end{equation}
%
The grade 2 term commutes on interchange:
%
\begin{equation}\label{eqn:trivector:trivecgpgrade0}
\gpgrade{\BA\BB}{0} = \gpgrade{\BB\BA}{0}
\end{equation}
%
\subsection{combining results}
%
\begin{equation*}
\BA \BB
=\gpgrade{\BA\BB}{0}
+\gpgrade{\BA\BB}{2}
+\gpgrade{\BA\BB}{4}
+\gpgrade{\BA\BB}{6}
\end{equation*}
%
\begin{equation}\label{eqn:trivector:266}
\begin{aligned}
\BB\BA
&=\gpgrade{\BB\BA}{0}
+\gpgrade{\BB\BA}{2}
+\gpgrade{\BB\BA}{4}
+\gpgrade{\BB\BA}{6} \\
&=\gpgrade{\BA\BB}{0}
-\gpgrade{\BA\BB}{2}
+\gpgrade{\BA\BB}{4}
-\gpgrade{\BA\BB}{6} \\
\end{aligned}
\end{equation}
%
These can be combined to express each of the grade terms as subsets
of the symmetric and antisymmetric parts:
%
\begin{equation}\label{eqn:trivector:286}
\begin{aligned}
\BA \cdot \BB = \gpgrade{\BA\BB}{0} &= \gpgrade{\frac{\BA\BB + \BB\BA}{2}}{0} \\
\gpgrade{\BA\BB}{2} &= \gpgrade{\frac{\BA\BB - \BB\BA}{2}}{2} \\
\gpgrade{\BA\BB}{4} &= \gpgrade{\frac{\BA\BB + \BB\BA}{2}}{4} \\
\BA \wedge \BB = \gpgrade{\BA\BB}{6} &= \gpgrade{\frac{\BA\BB - \BB\BA}{2}}{6} \\
\end{aligned}
\end{equation}
%
Note that above I have been somewhat loose with the argument above.  A grade three vector
will have the following form:
%
\begin{equation}\label{eqn:trivector:146}
\sum_{i<j<k} D_{ijk} \Be_{ijk}
\end{equation}
%
Where \(D_{ijk}\) is the determinant of \(ijk\) components of the vectors being wedged.  Thus the product
of two trivectors will be of the following form:
%
\begin{equation}\label{eqn:trivector:166}
\sum_{i<j<k} \sum_{i'<j'<k'} D_{ijk} D'_{i'j'k'} (\Be_{ijk} \Be_{i'j'k'})
\end{equation}
%
It is really each of these \(\Be_{ijk} \Be_{i'j'k'}\) products that have to be considered in the grade
and sign arguments above.  The end result will be the same though... one would just have to present
it a bit more carefully for a true proof.

\subsection{Intersecting trivector cases}

As with the intersecting bivector case, when there is a line of intersection between the two volumes one can
write:
%
\begin{equation}\label{eqn:trivector:306}
\begin{aligned}
\BA \cdot \BB = \gpgrade{\BA\BB}{0} &= \gpgrade{\frac{\BA\BB + \BB\BA}{2}}{0} \\
\gpgrade{\BA\BB}{2} &= \frac{\BA\BB - \BB\BA}{2} \\
\gpgrade{\BA\BB}{4} &= \gpgrade{\frac{\BA\BB + \BB\BA}{2}}{4} \\
\BA \wedge \BB = \gpgrade{\BA\BB}{6} &= 0 \\
\end{aligned}
\end{equation}
%
And if these volumes intersect in a plane a further simplification is possible:
\begin{equation}\label{eqn:trivector:326}
\begin{aligned}
\BA \cdot \BB = \gpgrade{\BA\BB}{0} &= \frac{\BA\BB + \BB\BA}{2} \\
\gpgrade{\BA\BB}{2} &= \frac{\BA\BB - \BB\BA}{2} \\
\gpgrade{\BA\BB}{4} &= 0 \\
\BA \wedge \BB = \gpgrade{\BA\BB}{6} &= 0 \\
\end{aligned}
\end{equation}
%

   \chapter{Multivector product grade zero terms}
      %
% Copyright � 2012 Peeter Joot.  All Rights Reserved.
% Licenced as described in the file LICENSE under the root directory of this GIT repository.
%

%
%
\chapter{Multivector product grade zero terms}
\label{chap:scalarCommutes}
%\date{Mar 16, 2008.  scalarCommutes.tex}

One can show that the grade zero component of a multivector product
is independent of the order of the terms:

\begin{equation}
\gpgradezero{\BA \BB} = \gpgradezero{\BB \BA}
\end{equation}

Doran/Lasenby has an elegant proof of this, but a dumber proof using an
explicit expansion by basis also works and highlights the similarities
with the standard component definition of the vector dot product.

Writing:

\begin{equation}\label{eqn:scalarCommutes:20}
\BA = \sum_i \gpgrade{\BA}{i}
\end{equation}
\begin{equation}\label{eqn:scalarCommutes:40}
\BB = \sum_i \gpgrade{\BB}{i}
\end{equation}

The product of \(\BA\) and \(\BB\) is:

\begin{equation}\label{eqn:scalarCommutes:180}
\begin{aligned}
\BA \BB
&= \sum_{ij} \gpgrade{\BA}{i} \gpgrade{\BB}{j} \\
&= \sum_{ij} \sum_{k=0}^{\min(i,j)}\gpgrade{\gpgrade{\BA}{i} \gpgrade{\BB}{j}}{2k + \abs{i-j}} \\
\end{aligned}
\end{equation}

\begin{equation}\label{eqn:scalar_commutes:product}
\BA \BB
= \sum_{ij} \sum_{k=0}^{\min(i,j)}\gpgrade{\gpgrade{\BA}{i} \gpgrade{\BB}{j}}{2k + \abs{i-j}}
\end{equation}

To get a better feel for this, consider an example

\begin{equation}\label{eqn:scalarCommutes:60}
\BA = \Be_1 + \Be_2 + \Be_{12} + \Be_{13} + \Be_{34} + \Be_{345}
\end{equation}
\begin{equation}\label{eqn:scalarCommutes:80}
\BB = \Be_2 + \Be_{21} + \Be_{23}
\end{equation}
\begin{equation}\label{eqn:scalarCommutes:100}
\BA \BB = ( \Be_1 + \Be_2 + \Be_{12} + \Be_{13} + \Be_{34} + \Be_{345})(\Be_2 + \Be_{21} + \Be_{23})
\end{equation}

Here are multivectors with grades ranging from zero to three.  This multiplication will include vector/vector, vector/bivector, vector/trivector, bivector/bivector, and bivector/trivector.  Some of these will be grade lowering, some grade preserving and some grade raising.

Only the like grade terms can potentially generate grade zero terms, so the grade zero terms of the product in \eqnref{eqn:scalar_commutes:product} are:

\begin{equation}\label{eqn:scalar_commutes:scalarproduct}
\BA \BB
= \sum_{i=j} \gpgradezero{\gpgrade{\BA}{i} \gpgrade{\BB}{j}}
\end{equation}

Using the example above we have

\begin{equation}\label{eqn:scalarCommutes:120}
\gpgradezero{\BA \BB}
= \gpgradezero{ (\Be_1 + \Be_2)\Be_2 }
+ \gpgradezero{ (\Be_{12} + \Be_{13} + \Be_{34})\Be_{21} }
\end{equation}

In general one can introduce an orthonormal basis
\(\sigma^k = \{\Bsigma_i^k\}_i\) for each of the \(\gpgrade{}{k}\) spaces.
Here orthonormal is with respect to the k-vector dot product

\begin{equation}\label{eqn:scalar_commutes:orthonormal}
\Bsigma_i^k \cdot \Bsigma_j^k = (-1)^{k(k-1)/2}\delta_{ij}
\end{equation}

then one can decompose each of the k-vectors with respect to that
basis:

\begin{equation}\label{eqn:scalarCommutes:140}
\gpgrade{\BA}{k} = \sum_i \left(\gpgrade{\BA}{k} \cdot \Bsigma_i^k\right) \inv{\Bsigma_i^k}
\end{equation}

\begin{equation}\label{eqn:scalarCommutes:160}
\gpgrade{\BB}{k} = \sum_{j} \left(\gpgrade{\BB}{k} \cdot \Bsigma_{j}^k\right) \inv{\Bsigma_{j}^k}
\end{equation}

Thus the scalar part of the product is

\begin{equation}\label{eqn:scalarCommutes:200}
\begin{aligned}
\gpgradezero{\BA \BB}
&= \sum_{k, i, j} \gpgradezero {
\left(\gpgrade{\BA}{k} \cdot \Bsigma_{i}^k\right) \inv{\Bsigma_{i}^k}
\left(\gpgrade{\BB}{k} \cdot \Bsigma_{j}^k\right) \inv{\Bsigma_{j}^k}
} \\
&= \sum_{k, i, j}
\gpgradezero { \Bsigma_{i}^k \Bsigma_{j}^k }
\left(\gpgrade{\BA}{k} \cdot \Bsigma_{i}^k\right)
\left(\gpgrade{\BB}{k} \cdot \Bsigma_{j}^k\right) \\
&= \sum_{k, i, j}
\left(-1\right)^{k\left(k-1\right)/2} \delta_{ij}
\left(\gpgrade{\BA}{k} \cdot \Bsigma_{i}^k\right)
\left(\gpgrade{\BB}{k} \cdot \Bsigma_{j}^k\right)
\end{aligned}
\end{equation}

Thus the complete scalar product can be written

\begin{equation}
\gpgradezero{\BA \BB} = \sum_{k, i}
\left(-1\right)^{k\left(k-1\right)/2}
\left(\gpgrade{\BA}{k} \cdot \Bsigma_{i}^k\right)
\left(\gpgrade{\BB}{k} \cdot \Bsigma_{i}^k\right)
\end{equation}

Note, compared to the vector dot product, the alternation in sign, which is
dependent on the grades involved.

Also note that this now trivially proves that the scalar product is commutative.

Perhaps more importantly we see how similar this generalized dot product is to the
standard component formulation of the vector dot product we are so used to.
At a glance the component-less geometric algebra formulation seems
so much different than the standard vector dot product expressed in terms of components, but
we see here that this is in fact not the case.


   \chapter{Blade grade reduction}
      %
% Copyright � 2012 Peeter Joot.  All Rights Reserved.
% Licenced as described in the file LICENSE under the root directory of this GIT repository.
%

%
%
\chapter{Blade grade reduction}
\index{grade reduction}
\label{chap:bladegradereduction}
%\date{Mar 25, 2008.  bladegradereduction.tex}

\section{General triple product reduction formula}

Consideration of the reciprocal frame bivector decomposition required the following identity

\begin{equation}
(\BA_a \wedge \BA_b) \cdot \BA_c =
\BA_a \cdot (\BA_b \cdot \BA_c)
\end{equation}

This holds when \(a + b \le c\), and \(a <= b\).  Similar equations for vector wedge blade dot blade reduction can be found in NFCM, but intuition let me to believe the above generalization was valid.

To prove this use the definition of the generalized dot product of two blades:

\begin{equation}\label{eqn:bladegradereduction:282}
\begin{aligned}
(\BA_a \wedge \BA_b) \cdot \BA_c
&= \gpgrade{ (\BA_a \wedge \BA_b) \BA_c }{\abs{c-(a+b)}} \\
\end{aligned}
\end{equation}

The subsequent discussion
is restricted to the \(b \ge a\) case.  Would have to think whether this restriction is required.

\begin{equation}
\label{eqn:bladegradereduction:bladewedge}
\begin{aligned}
\BA_a \wedge \BA_b
&= \BA_a \BA_b - \sum_{i=\abs{b-a},i+=2}^{a+b}\gpgrade{\BA_a\BA_b}{i} \\
&= \BA_a \BA_b - \sum_{k=0}^{a-1}\gpgrade{\BA_a\BA_b}{2k + b - a} \\
\end{aligned}
\end{equation}

Back substitution gives:

\begin{equation}\label{eqn:bladegradereduction:322}
\begin{aligned}
\gpgrade{ (\BA_a \wedge \BA_b) \BA_c }{\abs{c-(a+b)}}
&=
\gpgrade{ \BA_a \BA_b \BA_c }{\abs{c-(a+b)}}
-
\sum_{k=0}^{a-1}
\gpgrade{ \gpgrade{\BA_a\BA_b}{2k + b - a} \BA_c }{c-a-b}
\end{aligned}
\end{equation}

Temporarily writing \(\gpgrade{\BA_a\BA_b}{2k + b - a} = \BC_i\),
\begin{equation}\label{eqn:bladegradereduction:342}
\begin{aligned}
\gpgrade{\BA_a\BA_b}{2k + b - a} \BA_c
&= \sum_{j=c-i,j+=2}^{c+i} \gpgrade{ \BC_i \BA_c }{j} \\
&= \sum_{r=0}^{i} \gpgrade{ \BC_i \BA_c }{c-i+2r} \\
&= \sum_{r=0}^{2k+b-a} \gpgrade{ \BC_i \BA_c }{c-2k-b+a+2r} \\
&= \sum_{r=0}^{2k+b-a} \gpgrade{ \BC_i \BA_c }{c-b+a +2(r-k)} \\
\end{aligned}
\end{equation}

We want the only the following grade terms:

\begin{equation}\label{eqn:bladegradereduction:42}
c-b+a+2(r-k) = c - b - a
\implies
r=k-a
\end{equation}

There are many such \(k,r\) combinations, but we have a \(k \in [0,a-1]\) constraint, which implies \(r \in [-a,-1]\).  This contradicts with \(r\) strictly
positive,
so there are no such grade elements.

This gives an intermediate result, the reduction of the triple product to a direct product, removing the explicit wedge:

\begin{equation}
(\BA_a \wedge \BA_b) \cdot \BA_c =
\gpgrade{\BA_a \BA_b \BA_c}{c-a-b}
\end{equation}

\begin{equation}\label{eqn:bladegradereduction:362}
\begin{aligned}
\gpgrade{\BA_a \BA_b \BA_c}{c-a-b}
&= \gpgrade{\BA_a (\BA_b \BA_c)}{c-a-b} \\
&= \gpgrade{\BA_a \sum_{i}\gpgrade{\BA_b \BA_c}{i}}{c-a-b} \\
&= \gpgrade{\sum_{j}\gpgrade{\BA_a \sum_{i}\gpgrade{\BA_b \BA_c}{i}}{j}}{c-a-b} \\
\end{aligned}
\end{equation}

Explicitly specifying the grades here is omitted for simplicity.  The lowest grade of these is \((c-b)-a\), and all others are higher,
so grade selection excludes them.

By definition

\begin{equation}\label{eqn:bladegradereduction:62}
\gpgrade{\BA_b \BA_c}{c-b} = \BA_b \cdot \BA_c
\end{equation}

so that lowest grade term is thus

\begin{equation}\label{eqn:bladegradereduction:82}
\gpgrade{\BA_a \gpgrade{\BA_b \BA_c}{c-b}}{c-a-b}
= \gpgrade{\BA_a (\BA_b \cdot \BA_c)}{c-a-b}
= \BA_a \cdot (\BA_b \cdot \BA_c)
\end{equation}

This completes the proof.

\section{reduction of grade of dot product of two blades}

The result above can be applied to reducing the dot product of two blades.  For \(k<=s\):

\begin{equation}\label{eqn:bladegradereduction:102}
(\Ba_1 \wedge \Ba_2 \wedge \Ba_3 \cdots \wedge \Ba_k) \cdot (\Bb_1 \wedge \Bb_2 \cdots \wedge \Bb_s)
\end{equation}
\begin{equation}\label{eqn:bladegradereduction:382}
\begin{aligned}
&= (\Ba_1 \wedge (\Ba_2 \wedge \Ba_3 \cdots \wedge \Ba_k)) \cdot (\Bb_1 \wedge \Bb_2 \cdots \wedge \Bb_s) \\
&= (\Ba_1 \cdot ((\Ba_2 \wedge \Ba_3 \cdots \wedge \Ba_k)) \cdot (\Bb_1 \wedge \Bb_2 \cdots \wedge \Bb_s)) \\
&= (\Ba_1 \cdot (\Ba_2 \cdot (\Ba_3 \cdots \wedge \Ba_k)) \cdot (\Bb_1 \wedge \Bb_2 \cdots \wedge \Bb_s)) \\
&= \cdots \\
&= \Ba_1 \cdot (\Ba_2 \cdot (\Ba_3 \cdot (\cdots \cdot (\Ba_k \cdot (\Bb_1 \wedge \Bb_2 \cdots \wedge \Bb_s))))) \\
\end{aligned}
\end{equation}

This can be reduced to a single determinant, as is done in
the Flanders' differential forms book definition of the
\({\bigwedge}^k\) inner product (which is then used to define the Hodge dual).

The first such product is:

\begin{equation}\label{eqn:bladegradereduction:122}
\Ba_k \cdot (\Bb_1 \wedge \Bb_2 \cdots \wedge \Bb_k)
= \sum (-1)^{u-1} (\Ba_k \cdot \Bb_u) \Bb_1 \wedge \cdots \check{\Bb_u} \cdots \wedge \Bb_k
\end{equation}

Next, take dot product with \(\Ba_{k-1}\):

\begin{enumerate}
\item \(k = 2\)

\begin{equation}\label{eqn:bladegradereduction:402}
\begin{aligned}
&\Ba_{k-1} \cdot (\Ba_k \cdot (\Bb_1 \wedge \Bb_2 \cdots \wedge \Bb_k)) \\
&= \sum_{v \ne u} (-1)^{u-1} (\Ba_k \cdot \Bb_u) (\Ba_1 \cdot \Bb_v) \\
&=
 \sum_{u < v} (-1)^{v-1} (\Ba_k \cdot \Bb_v) (\Ba_1 \cdot \Bb_u)
+\sum_{u < v} (-1)^{u-1} (\Ba_k \cdot \Bb_u) (\Ba_1 \cdot \Bb_v) \\
&=
+\sum_{u < v} (\Ba_k \cdot \Bb_u) (\Ba_1 \cdot \Bb_v)
-\sum_{u < v} (\Ba_k \cdot \Bb_v) (\Ba_1 \cdot \Bb_u) \\
&=
+\sum_{u< v} (\Ba_k \cdot \Bb_u) (\Ba_1 \cdot \Bb_v)
- (\Ba_k \cdot \Bb_v) (\Ba_1 \cdot \Bb_u) \\
\end{aligned}
\end{equation}
\begin{equation}\label{eqn:bladegradereduction:k2dot}
-\sum_{u< v}
\begin{vmatrix}
\Ba_{k-1} \cdot \Bb_u & \Ba_{k-1} \cdot \Bb_v \\
\Ba_k \cdot \Bb_u & \Ba_k \cdot \Bb_v \\
\end{vmatrix}
\end{equation}

\item \(k>2\)
\end{enumerate}

\begin{equation}\label{eqn:bladegradereduction:142}
\Ba_{k-1} \cdot (\Ba_k \cdot (\Bb_1 \wedge \Bb_2 \cdots \wedge \Bb_k))
\end{equation}
\begin{equation}\label{eqn:bladegradereduction:422}
\begin{aligned}
&= \sum (-1)^{u-1} (\Ba_k \cdot \Bb_u) \Ba_{k-1} \cdot (\Bb_1 \wedge \cdots \check{\Bb_u} \cdots \wedge \Bb_k) \\
&= \sum_{v<u} (-1)^{u-1} (\Ba_k \cdot \Bb_u) (-1)^{v-1} (\Ba_{k-1} \cdot \Bb_v) (\Bb_1 \wedge \cdots \check{\Bb_v} \cdots \check{\Bb_u} \cdots \wedge \Bb_k) \\
&+ \sum_{v>u} (-1)^{u-1} (\Ba_k \cdot \Bb_u) (-1)^{v} (\Ba_{k-1} \cdot \Bb_v) (\Bb_1 \wedge \cdots \check{\Bb_u} \cdots \check{\Bb_v} \cdots \wedge \Bb_k) \\
\end{aligned}
\end{equation}

Add negation exponents, and use a change of variables for the first sum
\begin{equation}\label{eqn:bladegradereduction:442}
\begin{aligned}
&= \sum_{u<v} (-1)^{v+u} (\Ba_k \cdot \Bb_v) (\Ba_{k-1} \cdot \Bb_u) (\Bb_1 \wedge \cdots \check{\Bb_u} \cdots \check{\Bb_v} \cdots \wedge \Bb_k) \\
&- \sum_{u<v} (-1)^{u+v} (\Ba_k \cdot \Bb_u) (\Ba_{k-1} \cdot \Bb_v) (\Bb_1 \wedge \cdots \check{\Bb_u} \cdots \check{\Bb_v} \cdots \wedge \Bb_k) \\
\end{aligned}
\end{equation}

Merge sums:
\begin{equation}\label{eqn:bladegradereduction:462}
\begin{aligned}
&= \sum_{u<v} (-1)^{u+v}
\left(
(\Ba_k \cdot \Bb_v) (\Ba_{k-1} \cdot \Bb_u)
-(\Ba_k \cdot \Bb_u) (\Ba_{k-1} \cdot \Bb_v)
\right) \\
& \; (\Bb_1 \wedge \cdots \check{\Bb_u} \cdots \check{\Bb_v} \cdots \wedge \Bb_k)
\end{aligned}
\end{equation}

\begin{equation}\label{eqn:bladegradereduction:bivectordotkvector}
\Ba_{k-1} \cdot (\Ba_k \cdot (\Bb_1 \wedge \Bb_2 \cdots \wedge \Bb_k))
=
\end{equation}
\begin{equation*}
\sum_{u<v} (-1)^{u+v}
\begin{vmatrix}
\Ba_{k-1} \cdot \Bb_u & \Ba_{k-1} \cdot \Bb_v \\
\Ba_k \cdot \Bb_u & \Ba_k \cdot \Bb_v \\
\end{vmatrix}
(\Bb_1 \wedge \cdots \check{\Bb_u} \cdots \check{\Bb_v} \cdots \wedge \Bb_k) \\
\end{equation*}

Note that special casing \(k=2\) does not seem to be required because in that
case \(-1^{u+v} = -1^{1+2}=-1\), so this is identical to \eqnref{eqn:bladegradereduction:k2dot} after all.

\subsection{Pause to reflect}

Although my initial aim was to show that \(\BA_k \cdot \BB_k\) could be
expressed as a determinant as in the differential forms book (different
sign though), and to determine exactly what that determinant is, there
are some useful identities that fall out of this even just for this
bivector kvector dot product expansion.

Here is a summary of some of the things figured out so far

\begin{enumerate}
\item Dot product of grade one blades.

Here we have a result that can be expressed as a one by one determinant.  Worth mentioning to explicitly show the sign.

\begin{equation}\label{eqn:bladegradereduction:dotoneblades}
\Ba \cdot \Bb = \det[\Ba \cdot \Bb]
\end{equation}

%(Used \(\det{}\) here instead of \(\Det{}\) to avoid confusing with absolute value).
\item Dot product of grade two blades.

\begin{equation}\label{eqn:bladegradereduction:k2k2dot}
(\Ba_1 \wedge \Ba_2) \cdot (\Bb_1 \wedge \Bb_2)
=
-
\begin{vmatrix}
\Ba_1 \cdot \Bb_1 & \Ba_1 \cdot \Bb_2 \\
\Ba_2 \cdot \Bb_1 & \Ba_2 \cdot \Bb_2 \\
\end{vmatrix}
=
-\det[\Ba_i \cdot \Bb_j]
\end{equation}

\item Dot product of grade two blade with grade \(>2\) blade.

\begin{equation*}
(\Ba_{1} \wedge \Ba_2) \cdot (\Bb_1 \wedge \Bb_2 \cdots \wedge \Bb_k)
\end{equation*}
\begin{equation}\label{eqn:bladegradereduction:bivectordot}
=
\sum_{u<v} (-1)^{u+v-1}
(\Ba_1 \wedge \Ba_2) \cdot (\Bb_u \wedge \Bb_v)
(\Bb_1 \wedge \cdots \check{\Bb_u} \cdots \check{\Bb_v} \cdots \wedge \Bb_k)
\end{equation}
\end{enumerate}

Observe how similar this is to the vector blade dot product expansion:

\begin{equation}\label{eqn:bladegradereduction:vectordot}
\Ba \cdot (\Bb_1 \wedge \Bb_2 \cdots \wedge \Bb_k)
=
\sum (-1)^{i-1}
(\Ba \cdot \Bb_i) (\Bb_1 \wedge \cdots \check{\Bb_i} \cdots \wedge \Bb_k)
\end{equation}

\subsubsection{Expand it for \texorpdfstring{\(k=3\)}{k equal 3}}

Explicit expansion of \eqnref{eqn:bladegradereduction:bivectordot} for the \(k=3\) case, is also helpful to get a feel for
the equation:

\begin{equation}\label{eqn:bladegradereduction:482}
\begin{aligned}
(\Ba_{1} \wedge \Ba_2) \cdot (\Bb_1 \wedge \Bb_2 \wedge \Bb_3)
&=
(\Ba_1 \wedge \Ba_2) \cdot (\Bb_1 \wedge \Bb_2) \Bb_3 \\
&+(\Ba_1 \wedge \Ba_2) \cdot (\Bb_3 \wedge \Bb_1) \Bb_2 \\
&+(\Ba_1 \wedge \Ba_2) \cdot (\Bb_2 \wedge \Bb_3) \Bb_1
\end{aligned}
\end{equation}

Observe the cross product like alternation in sign and indices.
This suggests that a more natural way to express the sign coefficient may be via a \(\Sgn(\pi)\) expression for the sign of the
permutation of indices.

\section{trivector dot product}

With the result of \eqnref{eqn:bladegradereduction:bivectordot}, or the earlier equivalent determinant expression in equation
\eqnref{eqn:bladegradereduction:bivectordotkvector} we are now in a position to evaluate the dot product of a trivector and a greater or equal grade blade.

\begin{equation*}
\Ba_1 \cdot ((\Ba_{2} \wedge \Ba_3) \cdot (\Bb_1 \wedge \Bb_2 \cdots \wedge \Bb_k))
\end{equation*}
\begin{equation}\label{eqn:bladegradereduction:502}
\begin{aligned}
&=
\sum_{u<v} (-1)^{u+v-1}
(\Ba_2 \wedge \Ba_3) \cdot (\Bb_u \wedge \Bb_v)
\Ba_1 \cdot (\Bb_1 \wedge \cdots \check{\Bb_u} \cdots \check{\Bb_v} \cdots \wedge \Bb_k)  \\
&=
\sum_{w<u<v} (-1)^{u+v+w}
(\Ba_2 \wedge \Ba_3) \cdot (\Bb_u \wedge \Bb_v)
(\Ba_1 \cdot \Bb_w) (\Bb_1 \wedge \cdots \check{\Bb_w} \cdots \check{\Bb_u} \cdots \check{\Bb_v} \cdots \wedge \Bb_k)  \\
&+\sum_{u<w<v} (-1)^{u+v+w-1}
(\Ba_2 \wedge \Ba_3) \cdot (\Bb_u \wedge \Bb_v)
(\Ba_1 \cdot \Bb_w) (\Bb_1 \wedge \cdots \check \Bb_u \cdots \check{\Bb_w} \cdots \check{\Bb_v} \cdots \wedge \Bb_k)  \\
&+\sum_{u<v<w} (-1)^{u+v+w}
(\Ba_2 \wedge \Ba_3) \cdot (\Bb_u \wedge \Bb_v)
(\Ba_1 \cdot \Bb_w) (\Bb_1 \wedge \cdots \check \Bb_u \cdots \check{\Bb_v} \cdots \check{\Bb_w} \cdots \wedge \Bb_k)  \\
\end{aligned}
\end{equation}

Change the indices of summation and grouping like terms we have:
\begin{equation}\label{eqn:bladegradereduction:522}
\begin{aligned}
\sum_{u<v<w} (-1)^{u+v+w}
(
&(\Ba_2 \wedge \Ba_3) \cdot (\Bb_v \wedge \Bb_w) (\Ba_1 \cdot \Bb_u)  \\
&-(\Ba_2 \wedge \Ba_3) \cdot (\Bb_u \wedge \Bb_w) (\Ba_1 \cdot \Bb_v)  \\
&+(\Ba_2 \wedge \Ba_3) \cdot (\Bb_u \wedge \Bb_v) (\Ba_1 \cdot \Bb_w)  \\
)
(\Bb_1 \wedge \cdots \check \Bb_u \cdots \check{\Bb_v} \cdots \check{\Bb_w} \cdots \wedge \Bb_k)  \\
\end{aligned}
\end{equation}

Now, each of the embedded dot products were in fact determinants:
\begin{equation}\label{eqn:bladegradereduction:162}
(\Ba_2 \wedge \Ba_3) \cdot (\Bb_x \wedge \Bb_y)
=
-
\begin{vmatrix}
\Ba_2 \cdot \Bb_x & \Ba_2 \cdot \Bb_y \\
\Ba_3 \cdot \Bb_x & \Ba_3 \cdot \Bb_y \\
\end{vmatrix}
\end{equation}

Thus, we can expand these triple dot products like so (factor of \(-1\) omitted):
\begin{equation}\label{eqn:bladegradereduction:542}
\begin{aligned}
&(\Ba_2 \wedge \Ba_3) \cdot (\Bb_v \wedge \Bb_w) (\Ba_1 \cdot \Bb_u) \\
&-(\Ba_2 \wedge \Ba_3) \cdot (\Bb_u \wedge \Bb_w) (\Ba_1 \cdot \Bb_v) \\
&+(\Ba_2 \wedge \Ba_3) \cdot (\Bb_u \wedge \Bb_v) (\Ba_1 \cdot \Bb_w)  \\
&=
(\Ba_1 \cdot \Bb_u)
\begin{vmatrix}
\Ba_2 \cdot \Bb_v & \Ba_2 \cdot \Bb_w \\
\Ba_3 \cdot \Bb_v & \Ba_3 \cdot \Bb_w \\
\end{vmatrix} \\
&-
(\Ba_1 \cdot \Bb_v)
\begin{vmatrix}
\Ba_2 \cdot \Bb_u & \Ba_2 \cdot \Bb_w \\
\Ba_3 \cdot \Bb_u & \Ba_3 \cdot \Bb_w \\
\end{vmatrix} \\
&+
(\Ba_1 \cdot \Bb_w)
\begin{vmatrix}
\Ba_2 \cdot \Bb_u & \Ba_2 \cdot \Bb_v \\
\Ba_3 \cdot \Bb_u & \Ba_3 \cdot \Bb_v \\
\end{vmatrix} \\
%&=
%\begin{vmatrix}
%\Ba_1 \cdot \Bb_u & 0 & 0 \\
%0 & \Ba_2 \cdot \Bb_v & \Ba_2 \cdot \Bb_w \\
%0 & \Ba_3 \cdot \Bb_v & \Ba_3 \cdot \Bb_w \\
%\end{vmatrix} \\
%&+
%\begin{vmatrix}
%0 & \Ba_1 \cdot \Bb_v & 0 \\
%\Ba_2 \cdot \Bb_u & 0 & \Ba_2 \cdot \Bb_w \\
%\Ba_3 \cdot \Bb_u & 0 & \Ba_3 \cdot \Bb_w \\
%\end{vmatrix} \\
%&+
%\begin{vmatrix}
%0 & 0 & \Ba_1 \cdot \Bb_w \\
%\Ba_2 \cdot \Bb_u & \Ba_2 \cdot \Bb_v & 0 \\
%\Ba_3 \cdot \Bb_u & \Ba_3 \cdot \Bb_v & 0 \\
%\end{vmatrix} \\
&=
\begin{vmatrix}
\Ba_1 \cdot \Bb_u & \Ba_1 \cdot \Bb_v & \Ba_1 \cdot \Bb_w \\
\Ba_2 \cdot \Bb_u & \Ba_2 \cdot \Bb_v & \Ba_2 \cdot \Bb_w \\
\Ba_3 \cdot \Bb_u & \Ba_3 \cdot \Bb_v & \Ba_3 \cdot \Bb_w \\
\end{vmatrix} \\
\end{aligned}
\end{equation}

Final back substitution gives:

\begin{equation*}
(\Ba_1 \wedge \Ba_{2} \wedge \Ba_3) \cdot (\Bb_1 \wedge \Bb_2 \cdots \wedge \Bb_k)
\end{equation*}
\begin{equation}\label{eqn:bladegradereduction:trivectordotdet}
=
\sum_{u<v<w} (-1)^{u+v+w-1}
\begin{vmatrix}
\Ba_1 \cdot \Bb_u & \Ba_1 \cdot \Bb_v & \Ba_1 \cdot \Bb_w \\
\Ba_2 \cdot \Bb_u & \Ba_2 \cdot \Bb_v & \Ba_2 \cdot \Bb_w \\
\Ba_3 \cdot \Bb_u & \Ba_3 \cdot \Bb_v & \Ba_3 \cdot \Bb_w \\
\end{vmatrix}
(\Bb_1 \wedge \cdots \check \Bb_u \cdots \check{\Bb_v} \cdots \check{\Bb_w} \cdots \wedge \Bb_k)  \\
\end{equation}

In particular for \(k=3\) we have
\begin{equation*}
(\Ba_1 \wedge \Ba_{2} \wedge \Ba_3) \cdot (\Bb_1 \wedge \Bb_2 \wedge \Bb_3)
\end{equation*}
\begin{equation}\label{eqn:bladegradereduction:trivectordotdettri}
=
-\begin{vmatrix}
\Ba_1 \cdot \Bb_1 & \Ba_1 \cdot \Bb_2 & \Ba_1 \cdot \Bb_3 \\
\Ba_2 \cdot \Bb_1 & \Ba_2 \cdot \Bb_2 & \Ba_2 \cdot \Bb_3 \\
\Ba_3 \cdot \Bb_1 & \Ba_3 \cdot \Bb_2 & \Ba_3 \cdot \Bb_3 \\
\end{vmatrix}
=
-\det[\Ba_i \cdot \Bb_j]
\end{equation}

This can be substituted back into \eqnref{eqn:bladegradereduction:trivectordotdet} to put it in a non determinant form.

\begin{equation*}
(\Ba_1 \wedge \Ba_{2} \wedge \Ba_3) \cdot (\Bb_1 \wedge \Bb_2 \cdots \wedge \Bb_k)
\end{equation*}
\begin{equation}\label{eqn:bladegradereduction:trivectordotnondet}
=
\sum_{u<v<w} (-1)^{u+v+w}
(\Ba_1 \wedge \Ba_{2} \wedge \Ba_3) \cdot (\Bb_u \wedge \Bb_v \wedge \Bb_w)
(\Bb_1 \wedge \cdots \check \Bb_u \cdots \check{\Bb_v} \cdots \check{\Bb_w} \cdots \wedge \Bb_k)  \\
\end{equation}

\section{Induction on the result}

It is pretty clear that recursively performing these calculations will yield similar determinant and inner dot product reduction
results.

\subsection{dot product of like grade terms as determinant}

Let us consider the equal grade case first, summarizing the results so far

\begin{equation}\label{eqn:bladegradereduction:562}
\begin{aligned}
\Ba \cdot \Bb &= \det[\Ba \cdot \Bb] \\
(\Ba_1 \wedge \Ba_2) \cdot (\Bb_1 \wedge \Bb_2) &= -\det[\Ba_i \cdot \Bb_j] \\
(\Ba_1 \wedge \Ba_2 \wedge \Ba_3) \cdot (\Bb_1 \wedge \Bb_2 \wedge \Bb_3) &= -\det[\Ba_i \cdot \Bb_j] \\
\end{aligned}
\end{equation}

What will the sign be for the higher grade equivalents?  It has the appearance of being related to the sign associated with blade
reversion.  To verify this calculate the dot product of a blade formed from a set of perpendicular unit vectors with itself.

\begin{equation}\label{eqn:bladegradereduction:582}
\begin{aligned}
&(\Be_1 \wedge \cdots \wedge \Be_k) \cdot (\Be_1 \wedge \Be_2 \wedge \cdots \wedge \Be_k) \\
&= (-1)^{k(k-1)/2}(\Be_1 \wedge \cdots \wedge \Be_k) \cdot (\Be_k \wedge \cdots \wedge \Be_2 \wedge \Be_1) \\
&= (-1)^{k(k-1)/2}\Be_1 \cdot (\Be_2 \cdots (\Be_k \cdot (\Be_k \wedge \cdots \wedge \Be_2 \wedge \Be_1))) \\
&= (-1)^{k(k-1)/2}\Be_1 \cdot (\Be_2 \cdots (\Be_{k-1} \cdot (\Be_{k-1} \wedge \cdots \wedge \Be_2 \wedge \Be_1))) \\
&= \cdots \\
&= (-1)^{k(k-1)/2}
\end{aligned}
\end{equation}

This fixes the sign, and provides the induction hypothesis for the general case:

\begin{equation}\label{eqn:bladegradereduction:bladedothyp}
(\Ba_1 \wedge \cdots \wedge \Ba_k) \cdot (\Bb_1 \wedge \Bb_2 \wedge \cdots \wedge \Bb_k) = (-1)^{k(k-1)/2}\det[\Ba_i \cdot \Bb_j]
\end{equation}

Alternately, one can remove the sign change coefficient with reversion of one of the blades:

\begin{equation}\label{eqn:bladegradereduction:bladedothyprev}
(\Ba_1 \wedge \cdots \wedge \Ba_k) \cdot (\Bb_k \wedge \Bb_{k-1} \wedge \cdots \wedge \Bb_1) = \det[\Ba_i \cdot \Bb_j]
\end{equation}

\subsection{Unlike grades}

Let us summarize the results for unlike grades at the same time reformulating the previous results in terms of index
permutation, also writing for brevity \(\BA_s = \Ba_1 \wedge \cdots \wedge \Ba_s\), and \(\BB_k = \Bb_1 \wedge \cdots \wedge \Bb_k\):

\begin{equation}\label{eqn:bladegradereduction:182}
\BA_1 \cdot \BB_k =
\sum_i \Sgn(\pi(i,1,2,\cdots\check{i}\cdots,k)) (\BA_1 \cdot \Bb_i) (\Bb_1 \wedge \cdots \check{\Bb_i} \cdots \wedge \Bb_k)
\end{equation}

\begin{equation}\label{eqn:bladegradereduction:202}
\BA_2 \cdot \BB_k =
\sum_{i_1<i_2} \Sgn(\pi(i_1,i_2,1,2,\cdots\check{i_1}\cdots\check{i_2}\cdots,k))
\end{equation}
\begin{equation}\label{eqn:bladegradereduction:222}
   \BA_2 \cdot (\Bb_{i_1} \wedge \Bb_{i_2})
   (\Bb_1 \wedge \cdots \check{\Bb_{i_1}} \cdots \check{\Bb_{i_2}} \cdots \wedge \Bb_k)
\end{equation}

\begin{equation}\label{eqn:bladegradereduction:242}
\BA_3 \cdot \BB_k =
\sum_{i_1<i_2<i_3} \Sgn(\pi(i_1,i_2,i_3,1,2,\cdots\check{i_1}\cdots\check{i_2}\cdots\check{i_3}\cdots,k))
\end{equation}
\begin{equation}\label{eqn:bladegradereduction:262}
\BA_3 \cdot (\Bb_{i_1} \wedge \Bb_{i_2} \wedge \Bb_{i_3})
(\Bb_1 \wedge \cdots \check{\Bb_{i_1}} \cdots \check{\Bb_{i_2}} \cdots \check{\Bb_{i_3}} \cdots \wedge \Bb_k)
\end{equation}

We see that the dot product consumes any of the excess sign variation not described by the sign of the permutation of indices.

The induction hypothesis is basically described above (change \(3\) to \(s\), and add extra dots):

\begin{equation*}
\BA_s \cdot \BB_k =
\sum_{i_1<i_2\cdots<i_s} \Sgn(\pi(i_1,i_2\cdots,i_s,1,2,\cdots\check{i_1}\cdots\check{i_2}\cdots\check{i_s}\cdots,k))
\end{equation*}
\begin{equation}\label{eqn:bladegradereduction:inductionbigdotblade}
\BA_s \cdot (\Bb_{i_1} \wedge \Bb_{i_2} \cdots \wedge \Bb_{i_s})
(\Bb_1 \wedge \cdots \check{\Bb_{i_1}} \cdots \check{\Bb_{i_2}} \cdots \check{\Bb_{i_s}} \cdots \wedge \Bb_k)
\end{equation}

\subsection{Perform the induction}

In a sense this has already been done.  The steps will be pretty much the same as the logic that produced the bivector and trivector
results.  Thinking about typing this up in latex is not fun, so this will be left for a paper proof.

   \chapter{More details on NFCM plane formulation}
      %
% Copyright � 2012 Peeter Joot.  All Rights Reserved.
% Licenced as described in the file LICENSE under the root directory of this GIT repository.
%

%
%
%\chapter{More details on NFCM plane formulation}
\label{chap:plane}
%\date{Jan 1, 2008.  plane.tex}

\section{Wedge product formula for a plane}
\index{plane!wedge product}

The equation of the plane with bivector \(\BU\) through point \(\Ba\) is given
by

\begin{equation}\label{eqn:plane:20}
(\Bx - \Ba) \wedge \BU = 0
\end{equation}

or

\begin{equation}\label{eqn:plane:40}
\Bx \wedge \BU = \Ba \wedge \BU = \BT
\end{equation}

\subsection{Examining this equation in more details}

Without any loss of generality one can express this plane equation
in terms of a unit bivector \(\Bi\)

\begin{equation}\label{eqn:plane:60}
\Bx \wedge \Bi = \Ba \wedge \Bi
\end{equation}

As with the line equation, to express this in the ``standard'' parametric
form, right multiplication with \(1/\Bi\) is required.

\begin{equation}\label{eqn:plane:80}
(\Bx \wedge \Bi)\frac{1}{\Bi} = (\Ba \wedge \Bi)\frac{1}{\Bi}
\end{equation}

We have a trivector bivector product here, which in general has a vector,
trivector, and 5-vector component.  Since \(\Bi \wedge \Bi = 0\), the
5-vector component is zero:

\begin{equation}\label{eqn:plane:100}
\Bx \wedge \Bi \wedge -\Bi = 0
\end{equation}

and intuition says that the trivector component will also be zero.  However,
as well as providing verification of this, expansion of this product will also
demonstrate how to find the projective and rejective components of a vector
with respect to a plane (ie: components in and out of the plane).

\subsection{Rejection from a plane product expansion}
\index{plane!rejection}

Here is an explicit expansion of the rejective term above

\begin{equation}\label{eqn:plane:320}
\begin{aligned}
(\Bx \wedge \Bi)\frac{1}{\Bi}
&= -(\Bx \wedge \Bi){\Bi} \\
&= -\frac{1}{2}(\Bx\Bi + \Bi\Bx){\Bi} \\
&= \frac{1}{2}(\Bx - \Bi\Bx\Bi) \\
&= \frac{1}{2}(\Bx - (\Bx \Bi + 2 \Bi \cdot \Bx)\Bi) \\
&= \Bx - (\Bi \cdot \Bx)\Bi \\
\end{aligned}
\end{equation}

In this last term the quantity \(\Bi \cdot \Bx\) is a vector in the plane.
This can be demonstrated by writing \(\Bi\) in terms of a pair of orthonormal
vectors \(\Bi = \ucap\vcap = \ucap \wedge \vcap\).

\begin{equation}\label{eqn:plane:340}
\begin{aligned}
\Bi \cdot \Bx &= (\ucap \wedge \vcap) \cdot \Bx \\
              &= \ucap (\vcap \cdot \Bx) - \vcap (\ucap \cdot \Bx) \\
\end{aligned}
\end{equation}

Thus, \((\Bi \cdot \Bx) \wedge \Bi = 0\),
and \((\Bi \cdot \Bx) \Bi = (\Bi \cdot \Bx) \cdot \Bi\).  Inserting this above
we have the end result

\begin{equation}\label{eqn:plane:360}
\begin{aligned}
(\Bx \wedge \Bi)\frac{1}{\Bi}
&= \Bx - (\Bi \cdot \Bx) \cdot \Bi \\
&= \Ba - (\Bi \cdot \Ba) \cdot \Bi \\
\end{aligned}
\end{equation}

Or
\begin{equation}\label{eqn:plane:380}
\begin{aligned}
\Bx  - \Ba
&= (\Bi \cdot (\Bx - \Ba)) \cdot \Bi \\
\end{aligned}
\end{equation}

This is actually the standard parametric equation of a plane, but expressed
in terms of a unit bivector that describes the plane instead of in terms
of a pair of vectors in the plane.

To demonstrate this expansion of the right hand side is required

\begin{equation}\label{eqn:plane:400}
\begin{aligned}
(\Bi \cdot \Bx) \cdot \Bi
&= (\ucap (\vcap \cdot \Bx) - \vcap (\ucap \cdot \Bx)) \ucap \vcap \\
&= \vcap (\vcap \cdot \Bx) + \ucap (\ucap \cdot \Bx) \\
\end{aligned}
\end{equation}

Substituting this back yields:

\begin{equation}\label{eqn:plane:420}
\begin{aligned}
\Bx
&= \Ba + \ucap (\ucap \cdot (\Bx - \Ba)) + \vcap (\vcap \cdot (\Bx - \Ba)) \\
&= \Ba + s \ucap + t \vcap \\
&= \Ba + s' \By + t' \Bw \\
\end{aligned}
\end{equation}

Where \(\By\) and \(\Bw\) are two arbitrary, but non-colinear vectors
in the plane.

In words this says that the plane is specified by a point in the plane,
and the span of any pair of linearly independent vectors directed in that plane.

An expression of this form, or a normal form in terms of the cross product
is often how the plane is defined, and the analysis above demonstrates
that the bivector wedge product formula,

\begin{equation}\label{eqn:plane:120}
\Bx \wedge \BU = \Ba \wedge \BU
\end{equation}

where specific direction vectors in the plane need not be explicitly specified,
also implicitly contains this parametric representation.

\subsection{Orthonormal decomposition of a vector with respect to a plane}
\index{vector!plane projection}

With the expansion above we have a separation of a vector into two
components, and these can be demonstrated to be the components that are
directed entirely within and out of the plane.

Rearranging terms from above we have:

\begin{equation}\label{eqn:plane:440}
\begin{aligned}
\Bx
&=
(\Bx \cdot \Bi) \cdot \frac{1}{\Bi} + (\Bx \wedge \Bi) \cdot \frac{1}{\Bi} \\
&=
(\Bx \cdot \Bi) \frac{1}{\Bi} + (\Bx \wedge \Bi) \frac{1}{\Bi} \\
\end{aligned}
\end{equation}

Writing the vector \(\Bx\) in terms of components parallel and perpendicular
to the plane

\begin{equation}\label{eqn:plane:140}
\Bx = \Bx_{\perp} + \Bx_{\parallel}
\end{equation}

Only the \(\Bx_{\parallel}\) component contributes to the dot product
and only the \(\Bx_{\perp}\) component contributes to the wedge product:

\begin{equation}\label{eqn:plane:460}
\begin{aligned}
\Bx
&=
(\Bx_\parallel \cdot \Bi) \cdot \frac{1}{\Bi} + (\Bx_\perp \wedge \Bi) \cdot \frac{1}{\Bi} \\
\Bx_\parallel &= (\Bx \cdot \Bi) \cdot \frac{1}{\Bi} \\
\Bx_\perp &= (\Bx \wedge \Bi) \cdot \frac{1}{\Bi} \\
\end{aligned}
\end{equation}

So, just as in the orthonormal decomposition of a vector with respect to a
unit vector, this gives us a way to calculate components of a vector
in and rejected from any plane, a very useful result in its own right.

Returning to back to the equation of a plane we have

\begin{equation}\label{eqn:plane:480}
\begin{aligned}
- (\Bx \wedge \Bi)\Bi &= - (\Ba \wedge \Bi)\Bi = \Ba - (\Ba \cdot \Bi) \cdot \frac{1}{\Bi}
\end{aligned}
\end{equation}

Thus, for the fixed point in the plane, the quantity

\begin{equation}\label{eqn:plane:160}
\Bd = (\Ba \wedge \Bi) \cdot \frac{1}{\Bi}
\end{equation}

is the component of that vector perpendicular to the plane or the minimal length directed vector from the origin to the plane (directrix).  In terms
of the unit bivector for the plane and its directrix the equation of a
plane becomes

\begin{equation}\label{eqn:plane:500}
\begin{aligned}
\Bx \wedge \Bi &= \Bd \Bi = \Bd \wedge \Bi
\end{aligned}
\end{equation}

Note that the directrix is a normal to the plane.
%, so we have arrived at something
%similar to the typical cross product normal form plane equation.

\subsection{Alternate derivation of orthonormal planar decomposition}

This could alternately be derived by expanding the vector unit bivector
product directly

\begin{equation}\label{eqn:plane:520}
\begin{aligned}
\Bx \Bi \frac{1}{\Bi}
&= ( \Bx \cdot \Bi + \Bx \wedge \Bi ) \frac{1}{\Bi} \\
&=
- {(\Bx \cdot \Bi) \cdot \Bi} - {(\Bx \cdot \Bi) \wedge \Bi} - {(\Bx \wedge \Bi) \Bi} \\
&=
- {(\Bx \cdot \Bi) \cdot \Bi} - {(\Bx \wedge \Bi) \cdot \Bi } - {\left<(\Bx \wedge \Bi) \Bi\right>_3} - {(\Bx \wedge \Bi) \wedge \Bi} \\
&=
{(\Bx \cdot \Bi) \cdot \frac{1}{\Bi}} + {(\Bx \wedge \Bi) \cdot \frac{1}{\Bi}} - {\left<(\Bx \wedge \Bi) \Bi\right>_3} \\
\end{aligned}
\end{equation}

Since the LHS of this equation is the vector \(\Bx\), the RHS must
also be a vector, which demonstrates that the term

\begin{equation}\label{eqn:plane:180}
\left<(\Bx \wedge \Bi) \Bi\right>_3 = 0
\end{equation}

So, one has

\begin{equation}\label{eqn:plane:540}
\begin{aligned}
\Bx
&=
{(\Bx \cdot \Bi) \cdot \frac{1}{\Bi}} + {(\Bx \wedge \Bi) \cdot \frac{1}{\Bi}} \\
&=
{(\Bx \cdot \Bi) \frac{1}{\Bi}} + {(\Bx \wedge \Bi) \frac{1}{\Bi}} \\
\end{aligned}
\end{equation}

\section{Generalization of orthogonal decomposition to components with respect to a hypervolume}

Having observed how to directly calculate the components of a vector in and out of a plane, we can now do the
same thing for a \(r\)th degree volume element spanned by an \(r\)-blade hypervolume element \(\BU\).

\subsection{Hypervolume element and its inverse written in terms of a spanning orthonormal set}
\index{hypervolume element}
We take \(\BU\) to be a simple element, not an arbitrary multivector of grade \(r\).  Such an element can
always be written in the form

\begin{equation}\label{eqn:plane:200}
\BU = k \Bu_1 \Bu_2 \cdots \Bu_r
\end{equation}

Where \(\Bu_k\) are unit vectors that span the volume element.

The inverse of \(\BU\) is thus

\begin{equation}\label{eqn:plane:560}
\begin{aligned}
\BU^{-1}
&= \frac{\BU^\dagger}{\BU \BU^\dagger} \\
&= \frac{k \Bu_r \cdots \Bu_1}{(k \Bu_1 \cdots \Bu_r)(k \Bu_r \Bu_{r-1} \cdots \Bu_1)} \\
&= \frac{\Bu_r \cdots \Bu_1}{k} \\
\end{aligned}
\end{equation}

\subsection{Expanding the product}

Having gathered the required introductory steps we are now in a position to express the vector \(\Bx\) in terms
of components projected into and rejected from this hypervolume

\begin{equation}\label{eqn:plane:580}
\begin{aligned}
\Bx &= \Bx \BU \frac{1}{\BU} \\
    &= (\Bx \cdot \BU + \Bx \wedge \BU)\frac{1}{\BU} \\
\end{aligned}
\end{equation}

The dot product term can be expanded to
\begin{equation}\label{eqn:plane:600}
\begin{aligned}
&(\Bx \cdot \BU) \frac{1}{\BU}  \\
&= k(
      (\Bx \cdot \Bu_1)\Bu_2\Bu_3\cdots\Bu_r
    - (\Bx \cdot \Bu_2)\Bu_1\Bu_3\Bu_4\cdots\Bu_r
    + \cdots
    ) \frac{1}{\BU} \\
&=
  (\Bx \cdot \Bu_1)(\Bu_2\Bu_3\cdots\Bu_r)(\Bu_r\Bu_{r-1}\cdots\Bu_1)
- (\Bx \cdot \Bu_2)(\Bu_1\Bu_3\Bu_4\cdots\Bu_r)(\Bu_{r-1}\cdots\Bu_1)
+ \cdots \\
&=
  (\Bx \cdot \Bu_1)\Bu_1
+ (\Bx \cdot \Bu_2)\Bu_2
+ \cdots \\
\end{aligned}
\end{equation}

This demonstrates that \((\Bx \cdot \BU) \frac{1}{\BU}\) is a vector.  Because all the potential \(3, 5, ... 2r-1\) grade terms of this product are zero one can write

\begin{equation}\label{eqn:plane:220}
(\Bx \cdot \BU) \frac{1}{\BU} = \left<(\Bx \cdot \BU) \frac{1}{\BU}\right>_1 = (\Bx \cdot \BU) \cdot \frac{1}{\BU}
\end{equation}

In general the product of a \(r-1\)-blade and an \(r\)-blade such as \((\Bx \cdot \BA_r)\BB_r)\) could potentially have any of these higher order terms.

Summarizing the results so far we have
\begin{equation}\label{eqn:plane:620}
\begin{aligned}
\Bx
&= (\Bx \cdot \BU)\frac{1}{\BU} + (\Bx \wedge \BU)\frac{1}{\BU} \\
&= (\Bx \cdot \BU) \cdot \frac{1}{\BU} + (\Bx \wedge \BU)\frac{1}{\BU} \\
\end{aligned}
\end{equation}

Since the RHS of this equation is a vector, this implies that the LHS is also a vector and thus
\begin{equation}\label{eqn:plane:640}
\begin{aligned}
(\Bx \wedge \BU)\frac{1}{\BU}
&= \left<(\Bx \wedge \BU)\frac{1}{\BU}\right>_1 \\
&= (\Bx \wedge \BU) \cdot \frac{1}{\BU} \\
\end{aligned}
\end{equation}

Thus we have an explicit formula for the projective and rejective terms of a vector with respect to a hypervolume element \(\BU\):

\begin{equation}\label{eqn:plane:660}
\begin{aligned}
\Bx
&= (\Bx \cdot \BU)\frac{1}{\BU} + (\Bx \wedge \BU)\frac{1}{\BU} \\
&= (\Bx \cdot \BU) \cdot \frac{1}{\BU} + (\Bx \wedge \BU) \cdot \frac{1}{\BU} \\
&= \frac{-1^{r(r-1)/2}}{\abs{\BU}^2}
\left( (\Bx \cdot \BU) \cdot {\BU} + (\Bx \wedge \BU) \cdot {\BU} \right) \\
\end{aligned}
\end{equation}

\subsection{Special note.  Interpretation for projection and rejective components of a line}

The proof above utilized the general definition of the dot product of two blades, the selection of the lowest grade element of the product:

\begin{equation}\label{eqn:plane:240}
\BA_k \cdot \BB_j = \left<\BA_k \BB_j\right>_{\abs{k-j}}
\end{equation}

Because of this, the scalar-vector dot product is perfectly well defined

\begin{equation}\label{eqn:plane:260}
a \cdot \Bb = \left<a \Bb\right>_{1-0} = a \Bb
\end{equation}

So, when \(\BU\) is a vector, the equations above also hold.

\subsection{Explicit expansion of projective and rejective components}

Having calculated the explicit vector expansion of the projective term to prove that the all the higher grade
product terms were zero, this can be used to explicitly expand the projective and rejective components
in terms of a set of unit vectors that span the hypervolume

\begin{equation}\label{eqn:plane:680}
\begin{aligned}
\Bx_\parallel
&= (\Bx \cdot \BU) \cdot \frac{1}{\BU} \\
&=
  (\Bx \cdot \Bu_1)\Bu_1
+ (\Bx \cdot \Bu_2)\Bu_2
+ \cdots \\
\Bx_\perp
&= (\Bx \wedge \BU) \cdot \frac{1}{\BU} \\
&= \Bx
- (\Bx \cdot \Bu_1)\Bu_1
- (\Bx \cdot \Bu_2)\Bu_2
- \cdots \\
\end{aligned}
\end{equation}

Recall here that the unit vectors \(\Bu_k\) are not the standard basis vectors.
They are instead an arbitrary set of orthonormal vectors that span the hypervolume element \(\BU\).

\section{Verification that projective and rejective components are orthogonal}

In NFCM, for the equation of a line, it is demonstrated that the two vector components (directrix and parametrization) are orthogonal, and that the directrix is the minimal distance to the line from the origin.  That can be done here too for the hypervolume result.

\begin{equation}\label{eqn:plane:700}
\begin{aligned}
\Bx \wedge \BU &= \Ba \wedge \BU \\
(\Bx \wedge \BU){\BU}^{-1} &= (\Ba \wedge \BU){\BU}^{-1} \\
(\Bx\BU - \Bx \cdot \BU){\BU}^{-1} &= (\Ba \wedge \BU){\BU}^{-1} \\
\Bx &= (\Bx \cdot \BU){\BU}^{-1} + (\Ba \wedge \BU){\BU}^{-1} \\
    &= \Balpha{\BU}^{-1} + \Bd \\
\end{aligned}
\end{equation}

This first component, the projective term \(\Balpha{\BU}^{-1} = (\Bx \cdot \BU){\BU}^{-1}\), can be interpreted as a parametrization term.  The
last component, the rejective term \(\Bd = (\Ba \wedge \BU){\BU}^{-1}\) is identified as the directrix.  Calculation of \(\abs{\Bx}\) allows us to verify the physical interpretation of this vector.

Expansion of the projective term has previously shown that given

\begin{equation}\label{eqn:plane:280}
\BU = k \Bu_1 \wedge \Bu_2 \cdots \wedge \Bu_r
\end{equation}

then the expansion of this parametrization term has the form

\begin{equation}\label{eqn:plane:300}
\Balpha = \left(\sum_{i = 1}^{r}{\alpha_i \Bu_i}\right)\BU
\end{equation}

This is a very specific parametrization, a \(r-1\) grade parametrization \(\Balpha\) with \(r\) free variables, producing
a vector directed strictly in hypervolume spanned by \(\BU\).

We can calculate the length of the projective component of \(\Bx\) expressed in terms of this parametrization:

\begin{equation}\label{eqn:plane:720}
\begin{aligned}
{\Bx_\parallel}^2
&= \left((\Bx \cdot \BU) {\BU}^{-1}\right)^2 \\
&= \Balpha \frac{{\BU}^\dagger}{\BU \BU^\dagger}   \left( \frac{{\BU}^\dagger}{\BU \BU^\dagger} \right)^\dagger {\Balpha}^\dagger \\
&= \Balpha \frac{{\BU}^\dagger} {\abs{\BU}^2} \frac{\BU} {\abs{\BU}^2} {\Balpha}^\dagger \\
&= \frac{\abs{\Balpha}^2}{{\abs{\BU}^2}} \\
\end{aligned}
\end{equation}

\begin{equation}\label{eqn:plane:740}
\begin{aligned}
{\Bx}^2
&= (\Balpha {\BU}^{-1})^2 + \Bd^2 + 2 (\Balpha \BU^{-1}) \cdot \Bd \\
&= \frac{\abs{\Balpha}^2}{{\abs{\BU}^2}} + \Bd^2 + 2 (\Balpha \BU^{-1}) \cdot \Bd
\end{aligned}
\end{equation}

Direct computation shows that this last dot product term is zero

\begin{equation}\label{eqn:plane:760}
\begin{aligned}
(\Balpha \BU^{-1}) \cdot \Bd
&= (\Balpha \BU^{-1}) \cdot ((\Ba \wedge \BU){\BU}^{-1}) \\
&= (\Balpha \BU^{-1}) \cdot ( {\BU}^{-1} (\BU \wedge \Ba)) \\
&= \frac{(-1)^{r(r-1)/2}}{\abs{\BU}^2} (\Balpha \BU^{-1}) \cdot ( \BU (\BU \wedge \Ba)) \\
&= \frac{(-1)^{r(r-1)/2}}{\abs{\BU}^2} \left<\Balpha \BU^{-1} \BU (\BU \wedge \Ba)\right>_0 \\
&= \frac{(-1)^{r(r-1)/2}}{\abs{\BU}^2} \left<\Balpha (\BU \wedge \Ba)\right>_0 \\
\end{aligned}
\end{equation}

This last term is a product of an \(r-1\) grade blade and a \(r+1\) grade blade.  The lowest order term of this product has grade \(r+1 -(r-1) = 2\), which
implies that
\(\left<\Balpha (\BU \wedge \Ba)\right>_0 = 0\).  This demonstrates explicitly that the parametrization term is perpendicular to the rejective term as expected.

The length from the origin to the volume is thus

\begin{equation}\label{eqn:plane:780}
\begin{aligned}
{\Bx}^2 &= \frac{\abs{\Balpha}^2}{{\abs{\BU}^2}} + \Bd^2 \\
\end{aligned}
\end{equation}

This is minimized when \(\Balpha = 0\).  Thus \(\Bd\) is a vector directed from the origin to the hypervolume, perpendicular to that hypervolume, and also has the minimal distance to that space.


   \chapter{Quaternions}
      %
% Copyright � 2012 Peeter Joot.  All Rights Reserved.
% Licenced as described in the file LICENSE under the root directory of this GIT repository.
%

%
%
%\chapter{Quaternions}
\index{quaternion}
\label{chap:quaternion}
%\date{Feb 2, 2008.  quaternion.tex}

Like complex numbers, quaternions may be written as a multivector with scalar and bivector components (a 0,2-multivector).

\begin{equation}\label{eqn:quaternion:20}
q = \alpha + \mathbf{B}
\end{equation}

Where the complex number has one bivector component, and the quaternions have three.

One can describe quaternions as 0,2-multivectors where the basis for the bivector part is left handed.  There is not really anything special about quaternion multiplication, or complex number multiplication, for that matter.  Both are just a specific examples of a 0,2-multivector multiplication.  Other quaternion operations can also be found to have natural multivector equivalents.  The most important of which is likely the quaternion conjugate, since it implies the norm and the inverse.  As a multivector, like complex numbers, the conjugate operation is reversal:

\begin{equation}\label{eqn:quaternion:40}
\overline{q} = q^\dagger = \alpha - \mathbf{B}
\end{equation}

Thus \(\abs{q}^2 = q\overline{q} = \alpha^2 - \mathbf{B}^2\).  Note that this norm is a positive definite as expected since a bivector square is negative.

To be more specific about the left handed basis property of quaternions one can note that the quaternion bivector basis is usually defined in terms of the following properties

\begin{equation}\label{eqn:quaternion:60}
\mathbf{i}^2 = \mathbf{j}^2 = \mathbf{k}^2 = -1
\end{equation}
\begin{equation}\label{eqn:quaternion:80}
\mathbf{i}\mathbf{j} = -\mathbf{j}\mathbf{i}, \mathbf{i}\mathbf{k} = -\mathbf{k}\mathbf{i}, \mathbf{j}\mathbf{k} = -\mathbf{k}\mathbf{j}
\end{equation}
\begin{equation}\label{eqn:quaternion:100}
\mathbf{i}\mathbf{j} = \mathbf{k}
\end{equation}

The first two properties are satisfied by any set of orthogonal unit bivectors for the space.  The last property, which could also be written \(\mathbf{i}\mathbf{j}\mathbf{k} = -1\), amounts to a choice for the orientation of this bivector basis of the 2-vector part of the quaternion.

As an example suppose one picks

\begin{equation}\label{eqn:quaternion:120}
\mathbf{i} = \mathbf{e}_2\mathbf{e}_3
\end{equation}
\begin{equation}\label{eqn:quaternion:140}
\mathbf{j} = \mathbf{e}_3\mathbf{e}_1
\end{equation}

Then the third bivector required to complete the basis set subject to the properties above is

\begin{equation}\label{eqn:quaternion:160}
\mathbf{i}\mathbf{j} = \mathbf{e}_2\mathbf{e}_1 = \mathbf{k}
\end{equation}.

Suppose that, instead of the above, one picked a slightly more natural bivector basis, the duals of the unit vectors obtained by multiplication with the pseudoscalar (\(\mathbf{e}_1\mathbf{e}_2\mathbf{e}_3\mathbf{e}_i\)).  These bivectors are

\begin{equation}\label{eqn:quaternion:180}
\mathbf{i}=\mathbf{e}_2\mathbf{e}_3, \mathbf{j}=\mathbf{e}_3\mathbf{e}_1, \mathbf{k}=\mathbf{e}_1\mathbf{e}_2
\end{equation}.

A 0,2-multivector with this as the basis for the bivector part would have properties similar to the standard quaternions (anti-commutative unit quaternions, negation for unit quaternion square, same conjugate, norm and inversion operations, ...), however the triple product would have the value \(\mathbf{i}\mathbf{j}\mathbf{k} = 1\), instead of \(-1\).

\section{quaternion as generator of dot and cross product}

The product of pure quaternions is noted as being a generator of dot and cross products.  This is also true
of a vector bivector product.

Writing a vector \(\Bx\) as

\begin{equation}\label{eqn:quaternion:200}
\Bx = \sum_i x_i \Be_i = x_1 \Be_1 + x_2 \Be_2 + x_3 \Be_3
\end{equation}

And a bivector \(\BB\) (where for short, \(\Be_{ij} = \Be_i \Be_j = \Be_i \wedge \Be_j\)) as:

\begin{equation}\label{eqn:quaternion:220}
\BB = \sum_i b_i \Be_i I = b_1 \Be_{23} + b_2 \Be_{31} + b_3 \Be_{12}
\end{equation}

The product of these two is
\begin{equation}\label{eqn:quaternion:280}
\begin{aligned}
\Bx \BB
&= (x_1 \Be_1 + x_2 \Be_2 + x_3 \Be_3)(b_1 \Be_{23} + b_2 \Be_{31} + b_3 \Be_{12}) \\
&= (x_3 b_2 - x_2 b_3) \Be_1 + (x_1 b_3 - x_3 b_1) \Be_2 + (x_2 b_1 - x_1 b_2) \Be_3 \\
&+ (x_1 b_1 + x_2 b_2 + x_3 b_3) \Be_{123} \\
\end{aligned}
\end{equation}

Looking at the vector and trivector components of this we recognize the dot product and negated cross product
immediately (as with multiplication of pure quaternions).

Those products are, in fact, \(\Bx \cdot \BB\) and \(\Bx \wedge \BB\) respectively.

Introducing a vector and bivector basis \(\alpha = \{ \Be_i \}\), and \(\beta = \{ \Be_i I \}\), we can
express the dot product and cross product of the associated coordinate vectors
in terms of vector bivectors products as follows:

\begin{equation}\label{eqn:quaternion:240}
[\Bx]_\alpha \cdot [\BB]_\beta = \frac{\BB \wedge \Bx}{I}
\end{equation}
\begin{equation}\label{eqn:quaternion:260}
[\Bx]_\alpha \cross [\BB]_\beta = [\BB \cdot \Bx]_\alpha
\end{equation}


   \chapter{Cauchy Equations expressed as a gradient}
      %
% Copyright � 2012 Peeter Joot.  All Rights Reserved.
% Licenced as described in the file LICENSE under the root directory of this GIT repository.
%

%
%
\chapter{Cauchy Equations expressed as a gradient}
\index{Cauchy equations}
\index{gradient}
\label{chap:cauchyGradient}
%\date{August 13, 2008.  cauchyGradient.tex}

The complex number derivative, when it exists, is defined as:

\begin{equation*}
\frac{\delta f}{\delta z} = \frac{ f(z + \delta z) - f(z)}{\delta z}
\end{equation*}
\begin{equation*}
f'(z) = {\text{limit}}_{\abs{\delta z} \rightarrow 0} \quad \frac{\delta f}{\delta z}
\end{equation*}

Like any two variable function, this limit requires that all limiting paths produce the same result, thus it is
minimally necessary that the limits for the particular cases of \(\delta z = \delta x + i \delta y\) exist for both
\(\delta x = 0\), and \(\delta y = 0\) independently.  Of course there are other possible ways for \(\delta z \rightarrow 0\), such as spiraling inwards paths.  Apparently it can be shown that if the specific cases are satisfied, then this limit exists for any path (I am not sure how to show that, nor will try, at least now).

Examining each of these cases separately, we have for \(\delta x = 0\), and \(f(z) = u(x,y) + i v(x,y)\):

\begin{equation}\label{eqn:cauchyGradient:20}
\begin{aligned}
\frac{\delta f}{\delta z}
&= \frac{u(x,y + \delta y) + i v(x,y + \delta y)}{i\delta y} \\
&\rightarrow -i \frac{\partial u(x,y)}{\partial y} + \frac{\partial v(x,y)}{\partial y} \\
\end{aligned}
\end{equation}

and for \(\delta y = 0\)
\begin{equation}\label{eqn:cauchyGradient:40}
\begin{aligned}
\frac{\delta f}{\delta z}
&= \frac{u(x + \delta x,y) + i v(x + \delta x, y)}{\delta x} \\
&\rightarrow \frac{\partial u(x,y)}{\partial x} + i\frac{\partial v(x,y)}{\partial x} \\
\end{aligned}
\end{equation}

If these are equal regardless of the path, then equating real and imaginary parts of these respective equations we have:

\begin{equation}\label{eqn:cauchyGradient:60}
\begin{aligned}
\frac{\partial v}{\partial x} + \frac{\partial u}{\partial y} &= 0 \\
\frac{\partial u}{\partial x} - \frac{\partial v}{\partial y} &= 0
\end{aligned}
\end{equation}

Now, these are strikingly similar to the gradient, and we make this similarly explicit using the planar
pseudoscalar
\(i=\Be_1 \wedge \Be_2 = \Be_1 \Be_2\)
as the unit imaginary.  For the first equation, pre multiplying by \(1 = \Be_{11}\), and post multiplying by \(\Be_2\) we have:

\begin{equation*}
\Be_1 \frac{\partial \Be_{12} v}{\partial x} + \Be_{2}\frac{\partial u}{\partial y} = 0,
\end{equation*}

and for the second, pre multiply by \(\Be_1\), and post multiply the \(\partial_y\) term by \(1 = \Be_{22}\), and rearrange:
\begin{equation*}
\Be_1 \frac{\partial u}{\partial x} + \Be_{2} \frac{\partial \Be_{12} v}{\partial y} = 0.
\end{equation*}

Adding these we have:
\begin{equation*}
\Be_1 \frac{\partial u + \Be_{12}}{\partial x} + \Be_{2} \frac{\partial u + \Be_{12} v}{\partial y} = 0.
\end{equation*}

Since \(f = u + i v\), this is just

\begin{equation}
\Be_1 \frac{\partial f}{\partial x} + \Be_{2} \frac{\partial f}{\partial y} = 0.
\end{equation}

Or,
\begin{equation}\label{eqn:cauchy_gradient:gradf}
\grad f = 0
\end{equation}

By taking second partial derivatives and equating mixed partials we are used to seeing these Cauchy-Riemann equations
take this form as second order equations:

\begin{equation}\label{eqn:cauchy_gradient:uxx}
\grad^2 u = u_{xx} + u_{yy} = 0
\end{equation}
\begin{equation}
\grad^2 v = v_{xx} + v_{yy} = 0
\end{equation}

Given this, \eqnref{eqn:cauchy_gradient:gradf} is something that we could have perhaps guessed, since the square root of the Laplacian operator, is in fact the gradient (there are an infinite number of such square roots, since any rotation of the coordinate system that expresses the gradient also works).  However, a guess of this is not required since we see this explicitly through some logical composition of relationships.

The end result is that we can make a statement that
in regions where the complex function is analytic (has a derivative), the gradient of that function is zero in that region.

This is a kind of interesting result and I expect that this will relevant when figuring out how the geometric calculus
all fits together.

\section{Verify we still have the Cauchy equations hiding in the gradient}

We have:

\begin{equation*}
\grad f \Be_1 = \grad ( \Be_1 u - \Be_2 v) = 0
\end{equation*}

If this is to be zero, both the scalar and bivector parts of this equation must also be zero.

\begin{equation}\label{eqn:cauchyGradient:80}
\begin{aligned}
(\grad \cdot f) \Be_1
&= \grad \cdot ( \Be_1 u - \Be_2 v) \\
&= (\Be_1 \partial_x + \Be_2 \partial_y) \cdot ( \Be_1 u - \Be_2 v) \\
&= (\partial_x u - \partial_y v) = 0
\end{aligned}
\end{equation}

\begin{equation}\label{eqn:cauchyGradient:100}
\begin{aligned}
(\grad \wedge f) \Be_1
&= \grad \wedge ( \Be_1 u - \Be_2 v) \\
&= (\Be_1 \partial_x + \Be_2 \partial_y) \wedge ( \Be_1 u - \Be_2 v) \\
&= -\Be_1 \wedge \Be_2 (\partial_x v + \partial_y u) = 0
\end{aligned}
\end{equation}

We therefore see that this recovers the expected pair of Cauchy equations:

\begin{equation}\label{eqn:cauchyGradient:120}
\begin{aligned}
\partial_x u - \partial_y v &= 0 \\
\partial_x v + \partial_y u &= 0
\end{aligned}
\end{equation}

   \chapter{Legendre Polynomials}
      %
% Copyright � 2012 Peeter Joot.  All Rights Reserved.
% Licenced as described in the file LICENSE under the root directory of this GIT repository.
%

%
%
%\chapter{Legendre Polynomials}
\index{Legendre polynomial}
\label{chap:legendre}
%\date{Feb 4, 2008.  legendre.tex}

Exercise 8.4, from \citep{hestenes1999nfc}.

Find the first couple terms of the Legendre polynomial expansion of

\begin{equation}\label{eqn:legendre:20}
\inv{\abs{\Bx - \Ba}}
\end{equation}

Write

\begin{equation}\label{eqn:legendre:40}
f(x) = \inv{\abs{\Bx}}
\end{equation}

Expanding \(f(\Bx - \Ba)\) about \(\Bx\) we have

\begin{equation}\label{eqn:legendre:60}
\inv{\abs{\Bx - \Ba}} =
\sum_{k=0}{ \inv{k!} (-\agrad)^k} \inv{\abs{\Bx}}
\end{equation}

Expanding the first term we have

\begin{equation}\label{eqn:legendre:200}
\begin{aligned}
-\agrad \inv{\abs{\Bx}}
&=
\inv{{\abs{\Bx}}^2} \agrad {\abs{\Bx}} \\
&=
\inv{{\abs{\Bx}}^2} \agrad (\Bx^2)^{1/2} \\
&=
\inv{{\abs{\Bx}}^2} \frac{(1/2)}{({\abs{\Bx}}^2)^{1/2}}\agrad \Bx^2 \\
&=
\frac{\Ba \cdot \Bx}{{\abs{\Bx}}^3}
\end{aligned}
\end{equation}

Expansion of the second derivative term is
\begin{equation}\label{eqn:legendre:220}
\begin{aligned}
\frac{(-\agrad)}{2}\frac{(-\agrad)}{1}\inv{\abs{\Bx}}
&=
\frac{\agrad}{2} \left(\frac{-\Ba \cdot \Bx}{{\abs{\Bx}}^3}\right) \\
&=
\frac{-1}{2}
\left(
\frac{\agrad {(\Ba \cdot \Bx)}}{{\abs{\Bx}}^3} + {(\Ba \cdot \Bx)}\agrad \inv{{\abs{\Bx}}^3} \right) \\
\end{aligned}
\end{equation}

For this we need
\begin{equation}\label{eqn:legendre:80}
\agrad {(\Ba \cdot \Bx)} =
\Ba \cdot (\agrad {\Bx}) = \Ba^2
\end{equation}

And
\begin{equation}\label{eqn:legendre:240}
\begin{aligned}
\agrad \inv{{\abs{\Bx}}^k}
&=
k \inv{{\abs{\Bx}}^{k-1}} \agrad \inv{{\abs{\Bx}}} \\
&=
k \inv{{\abs{\Bx}}^{k-1}} \frac{- \Ba \cdot \Bx }{{\abs{\Bx}}^3} \\
&=
-k \frac{\Ba \cdot \Bx }{{\abs{\Bx}}^{k+2}} \\
\end{aligned}
\end{equation}

Thus the second derivative term is
\begin{equation}\label{eqn:legendre:260}
\begin{aligned}
\frac{-1}{2}
\left(
\frac{\Ba^2}{{\abs{\Bx}}^3} -3 \frac{(\Ba \cdot \Bx)^2} {{\abs{\Bx}}^5} \right)
=
\frac{ (1/2)\left( 3 (\Ba \cdot \Bx)^2 - \Ba^2 \Bx^2 \right) }
{ {{\abs{\Bx}}^5} }
\end{aligned}
\end{equation}

Summing these terms we have

\begin{equation}\label{eqn:legendre:100}
\inv{\abs{\Bx -\Ba}} =
\inv{\abs{\Bx}} +
\frac{ \Ba \cdot \Bx } { {\abs{\Bx}}^3 } +
\frac{ (1/2)\left( 3 (\Ba \cdot \Bx)^2 - \Ba^2 \Bx^2 \right) } { {{\abs{\Bx}}^5} } + \cdots
\end{equation}

NFCM writes this as
\begin{equation}\label{eqn:legendre:120}
\inv{\abs{\Bx -\Ba}} =
\frac{ P_0(\bxa) } {  \abs{\Bx}} +
\frac{ P_1(\bxa) } { {\abs{\Bx}}^3 } +
\frac{ P_2(\bxa) } { {\abs{\Bx}}^5 } + \cdots
\end{equation}

And calls \(P_i = P_i(\bxa)\) terms the Legendre polynomials.  This is not terribly clear since one expects a different form for the Legendre polynomials.

Using the Taylor formula one can derive a recurrence relation for these that makes the calculation a bit
simpler

\begin{equation}\label{eqn:legendre:280}
\begin{aligned}
\frac{P_{k+1}}{\abs{\Bx}^{2(k+1)+1}}
&= \frac{-\agrad}{k+1}\left(\frac{P_k}{\abs{\Bx}^{2k+1}}\right) \\
&=
\frac{-1}{k+1}
\left(
\frac{\agrad({P_k}}
{\abs{\Bx}^{2k+1}}
+
{P_k}\frac{\agrad}
{\abs{\Bx}^{2k+1}}
\right) \\
&=
\inv{k+1}
\left(
{P_k}(2k+1) \frac{\Ba \cdot \Bx}
{\abs{\Bx}^{2k+3}}
-\Bx^2 \frac{\agrad{P_k}}
{\abs{\Bx}^{2k+3}}
\right) \\
\end{aligned}
\end{equation}

Or
\begin{equation}\label{eqn:legendre:300}
\begin{aligned}
(k+1){P_{k+1}}
=
{P_k}(2k+1) {\Ba \cdot \Bx}
-\Bx^2 {\agrad{P_k}}
\end{aligned}
\end{equation}

Some of these have been calculated

\begin{equation}\label{eqn:legendre:320}
\begin{aligned}
P_0 &= 1 \\
P_1 &= \Ba \cdot \Bx \\
P_2 &= \half(3(\Ba \cdot \Bx)^2 -\Ba^2\Bx^2) \\
\end{aligned}
\end{equation}

And for the derivatives

\begin{equation}\label{eqn:legendre:340}
\begin{aligned}
\agrad P_0 &= 0 \\
\agrad P_1 &= \Ba^2 \\
\agrad P_2 &= \half((3)(2)(\Ba \cdot \Bx)\Ba^2 - 2\Ba^2\Bx \cdot \Ba) \\
           &= 2\Ba^2(\Bx \cdot \Ba) \\
\end{aligned}
\end{equation}

Using the recurrence relation one can calculate \(P_3\) for example.

\begin{equation}\label{eqn:legendre:360}
\begin{aligned}
P_3
%(k+1){P_{k+1}} ; k=2
&=
(1/3)\left(
\frac{5}{2}(3(\Ba \cdot \Bx)^2 -\Ba^2\Bx^2)({\Ba \cdot \Bx})
- 2 \Bx^2 \Ba^2(\Bx \cdot \Ba) \right) \\
&=
(1/3) ({\Ba \cdot \Bx}) \left(
\frac{5}{2}(3(\Ba \cdot \Bx)^2 -\Ba^2\Bx^2)
- 2 \Bx^2 \Ba^2 \right) \\
&=
({\Ba \cdot \Bx}) \left( \frac{5}{2}((\Ba \cdot \Bx)^2 ) - 3/2 \Bx^2 \Ba^2 \right) \\
&=
\half({\Ba \cdot \Bx}) ( {5}(\Ba \cdot \Bx)^2 - 3 \Bx^2 \Ba^2 ) \\
\end{aligned}
\end{equation}

\section{ Putting things in standard Legendre polynomial form}

This is still pretty laborious to calculate, especially because of not having a closed form recurrence
relation for \(\agrad P_k\).  Let us relate these to the standard Legendre polynomial form.

Observe that we can write

\begin{equation}\label{eqn:legendre:380}
\begin{aligned}
P_0(\bxa) &= 1 \\
\frac{P_1(\bxa)}{\abs{\Bx} \abs{\Ba}} &= \costheta \\
\frac{P_2(\bxa)}{\abs{\Bx}^2 \abs{\Ba}^2} &= \half(3(\costheta)^2 - 1) \\
\frac{P_3(\bxa)}{\abs{\Bx}^3 \abs{\Ba}^3} &= \half ( {5}(\costheta)^3 - 3 {(\costheta)} ) \\
\end{aligned}
\end{equation}

With this scaling, we have the standard form for the Legendre polynomials, and can write

\begin{equation}\label{eqn:legendre:140}
\inv{\Bx-\Ba} = \inv{\abs{\Bx}}\left(
P_0
+ \frac{\abs{\Ba}}{\abs{\Bx}} P_1(\costheta)
+ \left(\frac{\abs{\Ba}}{\abs{\Bx}}\right)^2 P_2(\costheta)
+ \left(\frac{\abs{\Ba}}{\abs{\Bx}}\right)^3 P_3(\costheta)
+ \cdots \right)
\end{equation}

\section{ Scaling standard form Legendre polynomials}

Since the odd Legendre polynomials have only odd terms and even have only even terms this allows for
the scaled form that NFCM uses.

\begin{equation}\label{eqn:legendre:400}
\begin{aligned}
P_0(\bxa) &= P_0(\costheta) \\
P_1(\bxa) &= \abs{\Bx}\abs{\Ba} P_1(\costheta) = \Ba \cdot \Bx \\
P_2(\bxa) &= \abs{\Bx}^2\abs{\Ba}^2 P_2(\costheta) = \half(3(\Ba \cdot \Bx)^2 - \Bx^2\Ba^2) \\
P_3(\bxa) &= \abs{\Bx}^3\abs{\Ba}^3 P_3(\costheta) = \half(5(\Ba \cdot \Bx)^3 - 3(\Ba \cdot \Bx) \Bx^2\Ba^2) \\
\end{aligned}
\end{equation}

Every term for the \(k^{th}\) polynomial is a permutation of the geometric product \(\Bx^k\Ba^k\).

This allows for writing some of these terms using the wedge product.  Using the product expansion:

\begin{equation}\label{eqn:legendre:160}
%\Ba \Bx \Bx \Ba = \Ba^2 \Bx^2 = (\Ba \cdot \Bx + \Ba \wedge \Bx)(\Bx \cdot \Ba + \Bx \wedge \Ba) = (\Ba \cdot \Bx)^2 - ( \Ba \wedge \Bx )^2
%\Ba^2 \Bx^2 = (\Ba \cdot \Bx)^2 - ( \Ba \wedge \Bx )^2
(\Ba \cdot \Bx)^2 = ( \Ba \wedge \Bx )^2 + \Ba^2 \Bx^2
\end{equation}

Thus we have:
\begin{equation}\label{eqn:legendre:420}
\begin{aligned}
P_2(\bxa)
&= (\Ba \cdot \Bx)^2 + \half(\Ba \wedge \Bx)^2 \\
&= (\Ba \cdot \Bx)^2 - \half\abs{\Ba \wedge \Bx}^2 \\
\end{aligned}
\end{equation}

This is nice geometrically since the directional dependence of this term on the co-linearity and
perpendicularity of the vectors \(\Ba\) and \(\Bx\) is clear.

Doing the same for the \(P_3\):

\begin{equation}\label{eqn:legendre:440}
\begin{aligned}
P_3(\bxa) &= (\Ba \cdot \Bx)\half(5(\Ba \cdot \Bx)^2 - 3\Bx^2\Ba^2) \\
          &= (\Ba \cdot \Bx)\half(2(\Ba \cdot \Bx)^2 + 3(\Ba \wedge \Bx)^2) \\
          &= (\Ba \cdot \Bx)((\Ba \cdot \Bx)^2 - \frac{3}{2}\abs{\Ba \wedge \Bx}^2) \\
\end{aligned}
\end{equation}

I suppose that one could get the same geometrical interpretation with a standard Legendre expansion in terms of \(\costheta = cos(\theta)\) terms, by collect both \(sin(\theta)\) and \(cos(\theta)\) powers, but one
can see the power of writing things explicitly in terms of the original vectors.

\section{ Note on NFCM Legendre polynomial notation}

In NFCM's slightly abusive notation \(P_k\) was used with various meanings.  He wrote \(P_k(\costheta) = \frac{P_k(\bxa)}{\abs{\Bx}^k \abs{\Ba}^k}\).

Note for example that the standard first degree Legendre polynomial \(P_1(x) = x\) evaluated with a \(\bxa\) value:

\begin{equation}\label{eqn:legendre:460}
\begin{aligned}
\inv {\abs{\Bx}\abs{\Ba}} {P_1(x) \vert_{x=\bxa}} &= \xcap \acap \\
&= \xcap \cdot \acap + \xcap \wedge \acap \\
\end{aligned}
\end{equation}

This has a bivector component in addition to the component identical to the standard Legendre polynomial
term (the first part).

By luck it happens that the scalar part of this equals \(P_1(\costheta)\), but this
is not the case for other terms.  Example, \(P_2(\bxa)\):

\begin{equation}\label{eqn:legendre:480}
\begin{aligned}
{P_2(x) \vert_{x=\bxa}}
&= \half( 3(\Bx \Ba)^2 - 1 ) \\
&= \half( 3(-\Ba \Bx + 2 \Ba \cdot \Bx )(\Bx \Ba) - 1 ) \\
&= \half( 3(-\Ba^2 \Bx^2 + 2(\Ba \cdot \Bx)^2 + 2(\Ba \cdot \Bx)(\Bx \wedge \Ba)) - 1 ) \\
&=  -(3/2)\Ba^2 \Bx^2 + 3(\Ba \cdot \Bx)^2 + 3(\Ba \cdot \Bx)(\Bx \wedge \Ba) - 1/2  \\
\end{aligned}
\end{equation}

Scaling this by \(1/(\Ba^2\Bx^2)\) is
\begin{equation}\label{eqn:legendre:180}
-\frac{3}{2} + 3(\costheta)^2 + 3(\costheta)(\xcap \wedge \acap) - \inv{\Ba^2\Bx^2} \\
\end{equation}

The scalar part of this is not anything recognizable.

   \chapter{Levi-Civitica summation identity}
      %
% Copyright � 2012 Peeter Joot.  All Rights Reserved.
% Licenced as described in the file LICENSE under the root directory of this GIT repository.
%

%
%
\chapter{Levi-Civitica summation identity}
\index{Levi-Civitica tensor}
\label{chap:levi}
%\date{March 13, 2009.  levi.tex}

\section{Motivation}

In \citep{byron1992mca} it is left to the reader to show

\index{contraction!Levi-Civitica tensor}
\begin{equation}\label{eqn:levi:20}
\begin{aligned}
\sum_k \epsilon_{ijk} \epsilon_{klm} = \delta_{il}\delta_{jm} - \delta_{jl}\delta_{im}
\end{aligned}
\end{equation}

\section{A mechanical proof}

Although it is not mathematical, this is easy to prove, at least for 3D.  The
following perl code does the trick

\lstinputlisting{listings/levi.pl}

The output produced has all the variations of indices, such as

\begin{equation}\label{eqn:levi:40}
\begin{aligned}
0 &= \sum_{k=1}^{3} \epsilon_{11k} \epsilon_{k11} = \delta_{11}\delta_{11} - \delta_{11}\delta_{11} \\
0 &= \sum_{k=1}^{3} \epsilon_{11k} \epsilon_{k12} = \delta_{11}\delta_{12} - \delta_{11}\delta_{12} \\
\vdots \\
0 &= \sum_{k=1}^{3} \epsilon_{11k} \epsilon_{k33} = \delta_{13}\delta_{13} - \delta_{13}\delta_{13} \\
0 &= \sum_{k=1}^{3} \epsilon_{12k} \epsilon_{k11} = \delta_{11}\delta_{21} - \delta_{21}\delta_{11} \\
1 &= \sum_{k=1}^{3} \epsilon_{12k} \epsilon_{k12} = \delta_{11}\delta_{22} - \delta_{21}\delta_{12} \\
0 &= \sum_{k=1}^{3} \epsilon_{12k} \epsilon_{k13} = \delta_{11}\delta_{23} - \delta_{21}\delta_{13} \\
-1 &= \sum_{k=1}^{3} \epsilon_{12k} \epsilon_{k21} = \delta_{12}\delta_{21} - \delta_{22}\delta_{11} \\
\vdots \\
\end{aligned}
\end{equation}

\section{Proof using bivector dot product}

This identity can also be derived from an expansion of the bivector
dot product in two different ways.

\begin{equation}\label{eqn:levi:60}
\begin{aligned}
( \Be_i \wedge \Be_j ) \cdot ( \Be_m \wedge \Be_n )
&=
( ( \Be_i \wedge \Be_j ) \cdot \Be_m ) \cdot \Be_n  \\
&=
(
\Be_i ( \Be_j \cdot \Be_m )
-\Be_j ( \Be_i \cdot \Be_m )
) \cdot \Be_n  \\
&=
( \Be_i \delta_{jm} -\Be_j \delta_{im} ) \cdot \Be_n  \\
&=
\delta_{in} \delta_{jm} -\delta_{jn} \delta_{im}
\end{aligned}
\end{equation}

Expressing the wedge product in terms duality, using the pseudoscalar
\(I = \Be_1 \Be_2 \Be_3\), we have

\begin{equation}\label{eqn:levi:80}
\begin{aligned}
(\Be_i \wedge \Be_j ) \Be_k = I \epsilon_{ijk}
\end{aligned}
\end{equation}

Or
\begin{equation}\label{eqn:levi:100}
\begin{aligned}
\Be_i \wedge \Be_j = I \sum_k \epsilon_{ijk} \Be_k
\end{aligned}
\end{equation}

Then the bivector dot product is
\begin{equation}\label{eqn:levi:120}
\begin{aligned}
( \Be_i \wedge \Be_j ) \cdot ( \Be_m \wedge \Be_n )
&=
\gpgradezero{
I \sum_k \epsilon_{ijk} \Be_k I \sum_p \epsilon_{mnp} \Be_p
} \\
&=
I^2 \sum_{k,p} \epsilon_{ijk} \epsilon_{mnp} \gpgradezero{ \Be_k \Be_p } \\
&=
- \sum_{k,p} \epsilon_{ijk} \epsilon_{mnp} \delta_{kp} \\
&=
- \sum_{k} \epsilon_{ijk} \epsilon_{mnk} \\
\end{aligned}
\end{equation}

Comparing the two expansions we have

\begin{equation}\label{eqn:levi:140}
\begin{aligned}
\sum_{k} \epsilon_{ijk} \epsilon_{mnk} &= \delta_{jn} \delta_{im} - \delta_{in} \delta_{jm}
\end{aligned}
\end{equation}

Which is equivalent to the original identity (after an index switcheroo).
Note both the dimension and metric dependencies in this proof.

   \chapter{Some NFCM exercise solutions and notes}
      %
% Copyright � 2012 Peeter Joot.  All Rights Reserved.
% Licenced as described in the file LICENSE under the root directory of this GIT repository.
%

%
%
%\chapter{Some NFCM exercise solutions and notes}
\label{chap:nfcmCh2}
%\date{Nov 27, 2008.  nfcmCh2.tex}

\paragraph{Solutions for problems in chapter 2}

I recall that some of the problems from this chapter of
\citep{hestenes1999nfc}
were fairly tricky.  Did I end up doing them all?  I intended to
revisit these and make sure I understood it all.  As I do so, write up
solutions, starting with \(1.3\), a question on the Geometric Algebra group.

Another thing I recall from the text is that I was fairly confused about
all the mass of identities by the time I got through it, and it was not clear
to me which were the fundamental ones.
Eventually I figured out that it is
really grade selection that is the fundamental operation, and
found better presentations of axiomatic treatment in \citep{doran2003gap}.

For reference the GA axioms are

\begin{itemize}
\item vector product is linear
%
\begin{equation}\label{eqn:nfcmCh2:20}
\begin{aligned}
a ( \alpha b + \beta c) &= \alpha a b + \beta a c \\
( \alpha a + \beta b) c &= \alpha a c + \beta b c
\end{aligned}
\end{equation}
%
\item distribution of vector product
%
\begin{equation}\label{eqn:nfcmCh2:40}
\begin{aligned}
(a b) c = a (b c) = a b c
\end{aligned}
\end{equation}
%
\item vector contraction
%
\begin{equation}\label{eqn:nfcm_ch2:contractionAxiom}
\begin{aligned}
a^2 \in \mathbb{R}
\end{aligned}
\end{equation}
%
For a Euclidean space, this provides the length \(a^2 = \Abs{a}^2\), but for relativity and conformal geometry this specific meaning is not required.

\end{itemize}

The definition of the generalized dot between two blades is
%
\begin{equation}\label{eqn:nfcm_ch2:generalDot}
\begin{aligned}
A_r \cdot B_s = \gpgrade{A B}{{\Abs{r -s}}}
\end{aligned}
\end{equation}
%
and the generalized wedge product definition for two blades is
%
\begin{equation}\label{eqn:nfcm_ch2:generalWedge}
\begin{aligned}
A_r \wedge B_s = \gpgrade{A B}{r + s}.
\end{aligned}
\end{equation}
%
With these definitions and the GA axioms everything else should logically follow.

I personally found it was really easy to go around in circles attempting the various proofs, and intended to revisit all of these
and prove them all for myself making sure I did not invoke any circular arguments and used only things already proven.

\subsection{Exercise 1.3}

Solve for \(x\)
%
\begin{equation}\label{eqn:nfcmCh2:60}
\begin{aligned}
\alpha x + a x \cdot b = c
\end{aligned}
\end{equation}
%
where \(\alpha\) is a scalar and all the rest are vectors.

\subsubsection{Solution}

Can dot or wedge the entire equation with the constant vectors.  In particular
%
\begin{equation}\label{eqn:nfcmCh2:80}
\begin{aligned}
c \cdot b &= (\alpha x + a x \cdot b) \cdot b \\
&= (\alpha + a \cdot b) x \cdot b
\end{aligned}
\end{equation}
\begin{equation}\label{eqn:nfcmCh2:100}
\begin{aligned}
\implies
x \cdot b &= \frac{c \cdot b}{\alpha + a \cdot b} \\
\end{aligned}
\end{equation}
%
and
\begin{equation}\label{eqn:nfcmCh2:120}
\begin{aligned}
c \wedge a &= (\alpha x + a x \cdot b) \wedge a \\
&= \alpha (x \wedge a) + \mathLabelBox
[
   labelstyle={xshift=2cm},
   linestyle={out=270,in=90, latex-}
]
{(a \wedge a)}{\(=0\)} (x \cdot b) \wedge a \\
\end{aligned}
\end{equation}
\begin{equation}\label{eqn:nfcmCh2:140}
\begin{aligned}
\implies
x \wedge a &= \inv{\alpha} (c \wedge a) \\
\end{aligned}
\end{equation}
%
This last can be reduced by dotting with \(b\), and then substitute the
result for \(x \cdot b\) from above
%
\begin{equation}\label{eqn:nfcmCh2:160}
\begin{aligned}
(x \wedge a) \cdot b
&= x (a \cdot b) - (x \cdot b) a \\
&= x (a \cdot b) - \frac{c \cdot b}{\alpha + a \cdot b} a \\
\end{aligned}
\end{equation}
%
Thus the final solution is
%
\begin{equation}\label{eqn:nfcmCh2:180}
\begin{aligned}
x = \inv{a \cdot b}\left(
\frac{c \cdot b}{\alpha + a \cdot b} a
+ \inv{\alpha} (c \wedge a) \cdot b
\right)
\end{aligned}
\end{equation}
%
Question: was there a geometric or physical motivation for this question.  I can not recall one?

\section{Sequential proofs of required identities}

\subsection{Split of symmetric and antisymmetric parts of the vector product}

NFCM defines the vector dot and wedge products in terms of the symmetric and antisymmetric parts, and not in terms of grade
selection.

The symmetric and antisymmetric split of a vector product takes the form
%
\begin{equation}\label{eqn:nfcmCh2:200}
\begin{aligned}
a b &= \inv{2}(a b + b a) + \inv{2}(a b - b a)
\end{aligned}
\end{equation}
%
Observe that if the two vectors are colinear, say \(b = \alpha a\), then this is
%
\begin{equation}\label{eqn:nfcmCh2:220}
\begin{aligned}
a b &= \frac{\alpha}{2} (a^2 + a^2) + \frac{\alpha}{2}(a^2 - a^2 )
\end{aligned}
\end{equation}
%
The antisymmetric part is zero for any colinear vectors, while the symmetric part is a scalar by the contraction axiom \eqnref{eqn:nfcm_ch2:contractionAxiom}.

Now, suppose that one splits the vector \(b\) into a part that is explicit
colinear with \(a\), as in \(b = \alpha a + c\).

Here one can observe that none of the colinear component of this vector
contributes to the antisymmetric part of the split
%
\begin{equation}\label{eqn:nfcmCh2:240}
\begin{aligned}
\inv{2}(a b - b a) &= \inv{2}(a (\alpha a + c) - (\alpha a + c) a) \\
&= \inv{2}(a c - c a)
\end{aligned}
\end{equation}
%
So, in a very loose fashion the symmetric part can be observed to be
due to only colinear parts of the vectors whereas colinear components of the
vectors do not contribute at all to the antisymmetric part of the product split.
One can see that there is a notion of parallelism and perpendicularity built
into this construction.

What is of interest here is to show that this symmetric and antisymmetric split
also provides the scalar and bivector parts of the product, and thus matches the
definitions of generalized dot and wedge products.

While it has been observed that the symmetric product is a scalar for colinear vectors
it has not been
demonstrated that this is necessarily a scalar in the general case.

Consideration of the square of \(a + b\) is enough to do so.
%
\begin{equation}\label{eqn:nfcmCh2:260}
\begin{aligned}
(a + b)^2 &= a^2 + b^2 + ab + ba \\
\implies \\
\end{aligned}
\end{equation}
\begin{equation}\label{eqn:nfcm_ch2:pythagorus}
\begin{aligned}
\inv{2}\left((a + b)^2 - a^2 - b^2\right) &= \inv{2}(ab + ba)
\end{aligned}
\end{equation}
%
We have only scalar terms on the LHS, which demonstrates that the symmetric product is necessarily a scalar.
This is despite the fact that the exact definition of \(a^2\) (ie: the metric for the space) has not been specified, nor even
a requirement that this vector square is even satisfies \(a^2 >= 0\).  Such an omission is valuable since it allows
for a natural formulation of relativistic four-vector algebra where both signs are allowed for the vector square.

Observe that \eqnref{eqn:nfcm_ch2:pythagorus} provides a generalization of the Pythagorean theorem.  If one defines, as in
Euclidean space, that two vectors are perpendicular by
%
\begin{equation}\label{eqn:nfcmCh2:280}
\begin{aligned}
(a + b)^2 = a^2 + b^2
\end{aligned}
\end{equation}
%
Then one necessarily has
%
\begin{equation}\label{eqn:nfcmCh2:300}
\begin{aligned}
\inv{2}(ab + ba) = 0
\end{aligned}
\end{equation}
%
So, that we have as a consequence of this perpendicularity definition a sign inversion on reversal
\begin{equation}\label{eqn:nfcmCh2:320}
\begin{aligned}
ba = -ab
\end{aligned}
\end{equation}
%
This equation contains the essence of the concept of grade.  The product of a pair of vectors is grade two
if reversal of the factors changes the sign, which in turn implies the two factors must be perpendicular.

Given a set of vectors that, according to the symmetric vector product (dot product) are all either mutually perpendicular or colinear, grouping by colinear sets determines the grade
%
\begin{equation}\label{eqn:nfcmCh2:340}
\begin{aligned}
a_1 a_2 a_3 ... a_m = (b_{j_1} b_{j_2} ... ) (b_{k_1} b_{k_2} ... ) ...  (b_{l_1} b_{l_2} ... )
\end{aligned}
\end{equation}
%
after grouping in pairs of colinear vectors (for which the squares are scalars) the count of the remaining elements is the grade.  By
example, suppose that \({e_i}\) is a normal basis for \R{N} \(e_i \cdot e_j \propto \delta_{ij}\), and one wishes to determine the grade
of a product.  Permuting this product so that it is ordered by index leaves it in a form that the grade can be observed by inspection
%
\begin{equation}\label{eqn:nfcmCh2:360}
\begin{aligned}
e_3 e_7 e_1 e_2 e_1 e_7 e_6 e_7
&= - e_3 e_1 e_7 e_2 e_1 e_7 e_6 e_7 \\
&= e_1 e_3 e_7 e_2 e_1 e_7 e_6 e_7 \\
&= ... \\
&\propto e_1 e_1 e_2 e_3 e_6 e_7 e_7 e_7 \\
&= (e_1 e_1) e_2 e_3 e_6 (e_7 e_7) e_7 \\
&\propto e_2 e_3 e_6 e_7 \\
\end{aligned}
\end{equation}
%
This is an example of a grade four product.  Given this implicit definition of grade, one can then see that the antisymmetric product of
two vectors is necessarily grade two.  An explicit enumeration of a vector product in terms of an explicit normal basis and associated
coordinates is helpful here to demonstrate this.

Let
%
\begin{equation}\label{eqn:nfcmCh2:380}
\begin{aligned}
a &= \sum_i a_i e_i \\
b &= \sum_j b_j e_j
\end{aligned}
\end{equation}
%
now, form the product
\begin{equation}\label{eqn:nfcmCh2:400}
\begin{aligned}
a b
&= \sum_i \sum_j a_i b_j e_i e_j \\
&=
 \sum_{i < j} a_i b_j e_i e_j
+\sum_{i = j} a_i b_j e_i e_j
+\sum_{i > j} a_i b_j e_i e_j \\
&=
 \sum_{i < j} a_i b_j e_i e_j
+\sum_{i = j} a_i b_j e_i e_j
+\sum_{j > i} a_j b_i e_j e_i \\
&=
 \sum_{i < j} a_i b_j e_i e_j
+\sum_{i = j} a_i b_j e_i e_j
-\sum_{i < j} a_j b_i e_i e_j \\
&= \sum_{i} a_i b_i (e_i)^2
+ \sum_{i < j} (a_i b_j - a_j b_i) e_i e_j  \\
\end{aligned}
\end{equation}
%
similarly
%
\begin{equation}\label{eqn:nfcmCh2:420}
\begin{aligned}
b a &= \sum_{i} a_i b_i (e_i)^2 - \sum_{i < j} (a_i b_j - a_j b_i) e_i e_j  \\
\end{aligned}
\end{equation}
%
Thus the symmetric and antisymmetric products are respectively
%
\begin{equation}\label{eqn:nfcmCh2:440}
\begin{aligned}
\inv{2}(a b + b a) &= \sum_{i} a_i b_i (e_i)^2 \\
\inv{2}(a b - b a) &= \sum_{i < j} (a_i b_j - a_j b_i) e_i e_j  \\
\end{aligned}
\end{equation}
%
The first part as shown above with non-coordinate arguments is a scalar.  Each term in the antisymmetric product has a grade two term, which
as a product of perpendicular vectors cannot be reduced any further, so it is therefore grade two in its entirety.

following the definitions of \eqnref{eqn:nfcm_ch2:generalDot} and \eqnref{eqn:nfcm_ch2:generalWedge} respectively, one can then write
%
\begin{equation}\label{eqn:nfcmCh2:460}
\begin{aligned}
a \cdot b &= \inv{2}(a b + b a) \\
a \wedge b &= \inv{2}(a b - b a)
\end{aligned}
\end{equation}
%
These can therefore be seen to be a consequence of the definitions and axioms rather than a required a-priori definition in their own right.  Establishing
these as derived results is important to avoid confusion when one moves on to general higher grade products.  The vector dot and wedge products are
not sufficient by themselves if taken as a fundamental definition to establish the required results for such higher grade products (in particular the useful
formulas for vector times blade dot and wedge products should be observed to be derived results as opposed to definitions).

\subsection{bivector dot with vector reduction}

In the \(1.3\) solution above the identity
%
\begin{equation}\label{eqn:nfcmCh2:480}
\begin{aligned}
(a \wedge b) \cdot c &= a (b \cdot c) - (a \cdot c) b \\
\end{aligned}
\end{equation}
%
was used.  Let us prove this.
%
\begin{equation}\label{eqn:nfcmCh2:500}
\begin{aligned}
(a \wedge b) \cdot c
&= \gpgradeone{(a \wedge b) c} \\
\implies \\
2 (a \wedge b) \cdot c
&= \gpgradeone{a b c - b a c} \\
&= \gpgradeone{a b c - b (- c a + 2 a \cdot c )} \\
&= \gpgradeone{a b c + b c a} - 2 b (a \cdot c ) \\
&= \gpgradeone{a (b \cdot c + b \wedge c) + (b \cdot c + b \wedge c) a} - 2 b (a \cdot c ) \\
&= 2 a (b \cdot c) + a \cdot (b \wedge c) + (b \wedge c) \cdot a - 2 b (a \cdot c ) \\
\end{aligned}
\end{equation}
%
To complete the proof we need \(a \cdot B = -B \cdot a\), but once that is demonstrated, one is left with the desired identity after dividing through
by \(2\).

\subsection{vector bivector dot product reversion}

Prove \(a \cdot B = -B \cdot a\).

   \chapter{Outermorphism Question}
      %
% Copyright � 2012 Peeter Joot.  All Rights Reserved.
% Licenced as described in the file LICENSE under the root directory of this GIT repository.
%

%
%
%\chapter{Outermorphism Question}
\index{outermorphism}
\label{chap:outermorphismDet}
%\date{Sept. 2, 2008.  outermorphismDet.tex}

\section{}

\citep{doran2003gap}
has an example of a linear operator.

\begin{equation}\label{eqn:outermorphism_det:F}
F(a) = a + \alpha(a \cdot f_1) f_2.
\end{equation}

This is used to compute the determinant without putting the operator
in matrix form.

\subsection{bivector outermorphism}

Their first step is to compute the wedge of this function applied to two vectors.  Doing this myself (not omitting steps), I get:

\begin{equation}\label{eqn:outermorphismDet:21}
\begin{aligned}
F(a \wedge b)
&= F(a) \wedge F(b) \\
&= (a + \alpha(a \cdot f_1) f_2 ) \wedge (b + \alpha(b \cdot f_1) f_2 ) \\
&= a \wedge b + \alpha(a \cdot f_1) f_2 \wedge b
+ \alpha (b \cdot f_1) a \wedge f_2
+ \alpha^2 (a \cdot f_1) (b \cdot f_1) \mathLabelBox{f_2 \wedge f_2}{\(=0\)} \\
&= a \wedge b
+ \alpha \left( (b \cdot f_1) a - (a \cdot f_1) b \right) \wedge f_2
\\
&= a \wedge b
+ \alpha \left( (a \wedge b ) \cdot f_1 \right) \wedge f_2
\end{aligned}
\end{equation}

This has a very similar form to the original function \(F\).  In particular
one can write

\begin{equation}\label{eqn:outermorphismDet:41}
\begin{aligned}
F(a)
&= a + \alpha(a \cdot f_1) f_2 \\
&= a + \gpgradeone{\alpha(a \cdot f_1) f_2} \\
&= a + \gpgrade{\alpha(a \cdot f_1) f_2}{0+1} \\
&= a + \alpha(a \cdot f_1) \wedge f_2 \\
\end{aligned}
\end{equation}

Here the fundamental definition of the wedge product as the
highest grade part of a product of blades has been used to show that the new
bivector function defined via outermorphism has the same form as the original, once we put the original in the new form that applies to bivector and vector:

\begin{equation}
F(A) = A + \alpha(A \cdot f_1) \wedge f_2
\end{equation}

\subsection{Induction}

Now, proceeding inductively, assuming that this is true for some grade \(k\) blade A, one can calculate \(F(A) \wedge F(b)\) for a vector \(b\):

\begin{equation}\label{eqn:outermorphismDet:61}
\begin{aligned}
&F(A) \wedge F(b) \\
&= (A + \alpha(A \cdot f_1) \wedge f_2) \wedge (b + \alpha(b \cdot f_1) f_2 ) \\
&= A \wedge b
+ \alpha( b \cdot f_1 ) A \wedge f_2
+ \alpha (( A \cdot f_1) \wedge f_2) \wedge b
+ \alpha^2 (b \cdot f_1) ((A \cdot f_1) \wedge f_2) \wedge f_2 \\
&= A \wedge b + \alpha \left( ( b \cdot f_1 ) A - ( A \cdot f_1) \wedge b \right) \wedge f_2 \\
&= A \wedge b + \alpha \gpgrade{ ( b \cdot f_1 ) A - ( A \cdot f_1) b}{k} \wedge f_2 \\
\end{aligned}
\end{equation}

Now, similar to the bivector case, this inner quantity can be reduced, but it is messier to do so:

\begin{equation}\label{eqn:outermorphismDet:81}
\begin{aligned}
\gpgrade{ ( b \cdot f_1 ) A - ( A \cdot f_1) b}{k}
&= \inv{2} \gpgrade{ b f_1 A - A f_1 b + f_1 (b A + (-1)^{k} A b) }{k} \\
\end{aligned}
\end{equation}
\begin{equation} \label{eqn:outermorphism_det:r1}
\implies
\gpgrade{ ( b \cdot f_1 ) A - ( A \cdot f_1) b}{k} = \inv{2} \gpgrade{ b f_1 A - A f_1 b}{k} + \gpgrade{ f_1 (b \wedge A) }{k}
\end{equation}

Consider first the right hand expression:
\begin{equation}\label{eqn:outermorphismDet:101}
\begin{aligned}
\gpgrade{ f_1 (b \wedge A) }{k}
&= f_1 \cdot (b \wedge A) \\
&= (-1)^{k} f_1 \cdot (A \wedge b) \\
&= (-1)^{k} (-1)^k (A \wedge b) \cdot f_1 \\
&= (A \wedge b) \cdot f_1 \\
\end{aligned}
\end{equation}

The right hand expression in \eqnref{eqn:outermorphism_det:r1} can be shown to equal zero.  That is messier still and the calculation can be found
at the end.

Using that equals zero result we now have:
\begin{equation}\label{eqn:outermorphismDet:121}
\begin{aligned}
F(A) \wedge F(b)
&= A \wedge b + \alpha ((A \wedge b) \cdot f_1) \wedge f_2 \\
\end{aligned}
\end{equation}

This completes the induction.

\subsection{Can the induction be avoided?}

Now, GAFP did not do this induction, nor even claim it was required.  The statement is "It follows that", after only calculating the bivector
case.  Is there a reason that they would be able to make such a statement without proof that is obvious to them perhaps but not to me?

It has been pointed out that this question is answered, ``yes, the induction can be avoided'', in \citep{aMacdonaldLAGC} page 148.

%I am guessing this would be related to the matrix concept of rank in
%some way too, but it is not clear to me exactly how.

\section{Appendix. Messy reduction for induction}

Q: Is there an easier way to do this?

Here we want to show that

\begin{equation*}
\inv{2} \gpgrade{ b f_1 A - A f_1 b}{k} = 0
\end{equation*}

Expanding the innards of this expression to group \(A\) and \(b\) parts together:

\begin{equation}\label{eqn:outermorphismDet:141}
\begin{aligned}
b f_1 A - A f_1 b
&= (f_1 b - 2 b \wedge f_1 ) A - A (b f_1 - 2 f_1 \wedge b) \\
&=
f_1 b A - A b f_1
- 2 (b \wedge f_1) A + 2 A (f_1 \wedge b) \\
&=
f_1 (b \cdot A + b \wedge A) - (A \cdot b + A \wedge b) f_1 \\
&- 2 \left( (b \wedge f_1) \cdot A + \gpgrade{(b \wedge f_1) A}{k} + (b \wedge f_1) \wedge A \right) \\
&+ 2 \left( A \cdot (f_1 \wedge b) + \gpgrade{A (f_1 \wedge b)}{k} + A \wedge (f_1 \wedge b) \right)
\end{aligned}
\end{equation}

the grade \(k-2\), and grade \(k+2\) terms of the bivector product
cancel (we are also only interested in the grade-\(k\) parts so can discard them).  This leaves:
\begin{equation*}
f_1 \wedge (b \cdot A) - (A \cdot b) \wedge f_1
+ f_1 \cdot (b \wedge A) - (A \wedge b) \cdot f_1
- 2 \gpgrade{(b \wedge f_1) A}{k}
+ 2 \gpgrade{A (f_1 \wedge b)}{k}
\end{equation*}

The bivector, blade product part of this is the antisymmetric part of that product so those two last terms can be expressed with the
commutator relationship for a bivector with blade: \(\gpgrade{B_2 A}{k} = \inv{2}(B_2 A - A B_2)\):

\begin{equation}\label{eqn:outermorphismDet:161}
\begin{aligned}
2 \gpgrade{A (f_1 \wedge b)}{k}
- 2 \gpgrade{(b \wedge f_1) A}{k}
&= A (f_1 \wedge b) - (f_1 \wedge b) A - (b \wedge f_1) A + A (b \wedge f_1) \\
&= A (f_1 \wedge b) - (f_1 \wedge b) A + (f_1 \wedge b) A - A (f_1 \wedge b) \\
&= 0
\end{aligned}
\end{equation}

So, we now have to show that we have zero for the remainder:
\begin{equation}\label{eqn:outermorphismDet:181}
\begin{aligned}
2 \gpgrade{ b f_1 A - A f_1 b}{k}
&= f_1 \wedge (b \cdot A) - (A \cdot b) \wedge f_1 \\
&\quad + f_1 \cdot (b \wedge A) - (A \wedge b) \cdot f_1 \\
&= (-1)^{k-1}f_1 \wedge (A \cdot b) - (-1)^{k-1}f_1 \wedge (A \cdot b) \\
&\quad + (-1)^{k}f_1 \cdot (A \wedge b) - (-1)^{k} f_1 \cdot (A \wedge b) \\
&= 0
\end{aligned}
\end{equation}

\section{New observation}

Looking again, I think I see one thing that I missed.  The text said they were
constructing the action on a general multivector.  So, perhaps they meant
\(b\) to be a blade.  This is a typesetting subtlety if that is the case.  Let us
assume that is what they meant, and that \(b\) is a grade \(k\) blade.  This
makes the coefficient of the scalar \(\alpha\) in equation 4.147 :

\begin{equation}\label{eqn:outermorphismDet:201}
\begin{aligned}
a \cdot f_1 f_2 \wedge b + b \cdot f_1 a \wedge f_2
&= \left( (b \cdot f_1) a + (-1)^{k} (a \cdot f_1) b \right) \wedge f_2 \\
\end{aligned}
\end{equation}

whereas they have:
\begin{equation*}
\left( (b \cdot f_1) a - (a \cdot f_1) b \right) \wedge f_2
\end{equation*}

So, no, I think they must have intended \(b\) to be a vector, not an
arbitrary grade blade.

Now, indirectly, it has been
proven here that for a vectors \(x\), \(y\), and a grade-\(k\) blade \(B\):

\begin{equation}\label{eqn:outermorphism_det:distrib}
(A \wedge x) \cdot y = A ( x \cdot y ) - ( A \cdot y ) \wedge x
\end{equation}

Or,
\begin{equation}
(A \wedge x) \cdot y = ( y \cdot x ) A + (-1)^{k}( y \cdot A ) \wedge x
\end{equation}

(changed variable names to disassociate this from the specifics of this
particular example), which is a generalization of the wedge product with
dot product distribution identity for vectors:

\begin{equation}
(a \wedge b) \cdot c = a ( b \cdot c ) - ( a \cdot c ) \wedge b
\end{equation}

I believe I have seen a still more general form of \eqnref{eqn:outermorphism_det:distrib}
in a
Hestenes paper, but did not think about using it a-priori.  Regardless, it
does not really appear the the GAFP text was treating \(b\) as anything but a
vector, since there would have to be a \((-1)^k\) factor on equation 4.147 for
it to be general.

   %%\chapter{Elliptic parameterization}
      %
% Copyright � 2016 Peeter Joot.  All Rights Reserved.
% Licenced as described in the file LICENSE under the root directory of this GIT repository.
%
%{
%\input{../blogpost.tex}
%\renewcommand{\basename}{hestenesElipseParameterization}
%\renewcommand{\dirname}{notes/phy1520/}
%%\newcommand{\dateintitle}{}
%%\newcommand{\keywords}{}
%
%\input{../peeter_prologue_print2.tex}
%
%\usepackage{peeters_layout_exercise}
%\usepackage{peeters_braket}
%\usepackage{peeters_figures}
%\usepackage{siunitx}
%
%\beginArtNoToc
%
%\generatetitle{Elliptic parameterization}
%%\chapter{Elliptic parameterization}
%%\label{chap:hestenesElipseParameterization}
%% \citep{sakurai2014modern} pr X.Y
%% \citep{pozar2009microwave}
%% \citep{qftLectureNotes}
%% \citep{griffiths1999introduction}
%
\makeoproblem{Elliptic parameterization}{problem:hestenesElipseParameterization:1}{\citep{hestenes1999nfc} ch. 3, pr. 8.6}{
Show that an ellipse can be parameterized by
\index{ellipse}
\begin{equation}\label{eqn:hestenesElipseParameterization:20}
\Br(t) = \Bc \cosh( \mu + i t ).
\end{equation}
Here \( i \) is a unit bivector, and \( i \wedge \Bc \) is zero (i.e. \( \Bc \) must be in the plane of the bivector \( i \)).
} % problem
\makeanswer{problem:hestenesElipseParameterization:1}{
Note that \( \mu, t, i \) all commute since \( \mu, t \) are both scalars.
That allows a complex-like expansion of the hyperbolic cosine to be used
\begin{equation}\label{eqn:hestenesElipseParameterization:40}
\begin{aligned}
\cosh( \mu + i t )
&=
\inv{2} \lr{ e^{\mu + i t} + e^{-\mu -i t} } \\
&=
\inv{2} \lr{ e^{\mu} (\cos t + i \sin t) + e^{-\mu} (\cos t -i \sin t) } \\
&=
\cosh \mu \cos t + i \sinh \mu \sin t.
\end{aligned}
\end{equation}

Since an ellipse can be parameterized as
\begin{equation}\label{eqn:hestenesElipseParameterization:60}
\Br(t) = \Ba \cos t + \Bb \sin t,
\end{equation}
where the vector directions \( \Ba \) and \( \Bb \) are perpendicular, the multivector hyperbolic cosine representation parameterizes the ellipse provided
\begin{equation}\label{eqn:hestenesElipseParameterization:80}
\begin{aligned}
\Ba &= \Bc \cosh \mu \\
\Bb &= \Bc i \sinh \mu.
\end{aligned}
\end{equation}
It is desirable to relate the parameters \( \mu, i \) to the vectors \( \Ba, \Bb \).  Because \( \Bc \wedge i = 0 \), the vector \( \Bc \) anticommutes with \( i \), and therefore \( (\Bc i)^2 = -\Bc i i \Bc = \Bc^2 \), which means
\begin{equation}\label{eqn:hestenesElipseParameterization:100}
\begin{aligned}
\Ba^2 &= \Bc^2 \cosh^2 \mu \\
\Bb^2 &= \Bc^2 \sinh^2 \mu,
\end{aligned}
\end{equation}
or
\begin{equation}\label{eqn:hestenesElipseParameterization:120}
\mu = \tanh^{-1} \frac{\Abs{\Bb}}{\Abs{\Ba}}.
\end{equation}

The bivector \( i \) is just the unit bivector for the plane containing \( \Ba \) and \( \Bb \)
\begin{equation}\label{eqn:hestenesElipseParameterization:140}
\begin{aligned}
\Ba \wedge \Bb
&= \cosh \mu \sinh \mu \Bc \wedge (\Bc i) \\
&= \cosh \mu \sinh \mu \gpgradetwo{ \Bc \Bc i } \\
&= \cosh \mu \sinh \mu i \Bc^2 \\
&= \cosh \mu \sinh \mu i \frac{ \Ba^2 }{ \cosh^2 \mu } \\
&= \Ba^2 \tanh \mu i \\
&= \Ba^2 i \frac{\Abs{\Bb}}{\Abs{\Ba}},
\end{aligned}
\end{equation}
so
\begin{equation}\label{eqn:hestenesElipseParameterization:160}
i = \frac{ \Ba \wedge \Bb }{\Abs{\Ba}\Abs{\Bb}}.
\end{equation}

Observe that \( i \) is a unit bivector provided the vectors \( \Ba, \Bb \) are perpendicular, as required
\begin{equation}\label{eqn:hestenesElipseParameterization:180}
\begin{aligned}
(\Ba \wedge \Bb)^2
&=
(\Ba \wedge \Bb) \cdot (\Ba \wedge \Bb) \\
&=
( (\Ba \wedge \Bb) \cdot \Ba ) \cdot \Bb \\
&=
( \Ba (\Bb \cdot \Ba) - \Bb \Ba^2 ) \cdot \Bb \\
&=
(\Ba \cdot \Bb)^2 - \Bb^2 \Ba^2 \\
&=
- \Bb^2 \Ba^2.
\end{aligned}
\end{equation}
} % answer
%%}
%\EndArticle
%%\EndNoBibArticle


\part{Projection}
   \chapter{Reciprocal Frame Vectors}
      %
% Copyright � 2012 Peeter Joot.  All Rights Reserved.
% Licenced as described in the file LICENSE under the root directory of this GIT repository.
%

%
%
%\chapter{Reciprocal Frame Vectors}
\index{reciprocal frame}
\label{chap:reciprocalFrame}
%\date{March 29, 2008.  reciprocalFrame.tex}

\section{Approach without Geometric Algebra}

Without employing geometric algebra, one can use the projection
operation expressed as a dot product and calculate the a vector
orthogonal to a set of other vectors, in the direction of a reference
vector.

Such a calculation also yields \R{N} results in terms of determinants, and as a side
effect produces equations for
parallelogram area, parallelepiped volume and higher dimensional analogues as a side effect
(without having to employ change of basis diagonalization arguments that do not work well
for higher dimensional subspaces).

\subsection{Orthogonal to one vector}

The simplest case is the vector perpendicular to another.  In anything
but \R{2} there are a whole set of such vectors, so to express this as a
non-set result a reference vector is required.

Calculation of the coordinate vector for this case follows directly from
the dot product.  Borrowing the GA term, we subtract the projection
to calculate the rejection.
%
\begin{equation}\label{eqn:reciprocalFrame:363}
\begin{aligned}
\Rej{\ucap}{\Bv}
&= \Bv - \Bv \cdot \ucap \ucap \\
&= \inv{\Bu^2}(\Bv\Bu^2 - \Bv \cdot \Bu \Bu) \\
&= \inv{\Bu^2}\sum{v_i\Be_i u_j u_j - v_j u_j u_i \Be_i} \\
&= \inv{\Bu^2}\sum{u_j\Be_i\DETuvij{v}{u}{i}{j}} \\
&= \inv{\Bu^2}\sum_{i<j}{(u_i \Be_j -u_j\Be_i)\DETuvij{u}{v}{i}{j}} \\
\end{aligned}
\end{equation}
%
Thus we can write the rejection of \(\Bv\) from \(\ucap\) as:
%
\begin{equation}\label{eqn:reciprocal_frame:rejonevector}
\Rej{\ucap}{\Bv} = \inv{\Bu^2}\sum_{i<j}\DETuvij{u}{v}{i}{j}\DETuvij{u}{\Be}{i}{j}
\end{equation}
%
Or introducing some shorthand:
%
\begin{equation}\label{eqn:reciprocalFrame:383}
\begin{aligned}
D_{ij}^{\Bu \Bv} &= \DETuvij{u}{v}{i}{j} \\
D_{ij}^{\Bu \Be} &= \DETuvij{u}{\Be}{i}{j} \\
\end{aligned}
\end{equation}
%
\eqnref{eqn:reciprocal_frame:rejonevector} can be expressed in a form that will be slightly more convenient for larger sets of
vectors:
%
\begin{equation}\label{eqn:reciprocal_frame:rejonevectorD}
\Rej{\ucap}{\Bv} = \inv{\Bu^2}\sum_{i<j} D_{ij}^{\Bu \Bv} D_{ij}^{\Bu \Be}
\end{equation}
%
Note that although the GA axiom \(\Bu^2 = \Bu \cdot \Bu\) has been used
in equations \eqnref{eqn:reciprocal_frame:rejonevector} and \eqnref{eqn:reciprocal_frame:rejonevectorD} above and the derivation, that was
not necessary to prove them.
This can, for now, be thought of as a notational convenience, to avoid having to write \(\Bu \cdot \Bu\), or
\(\norm{\Bu}^2\).

This result can be used to express the \R{N} area of a parallelogram since we just have to multiply the length
of \(\Rej{\ucap}{\Bv}\):
%
\begin{equation}\label{eqn:reciprocalFrame:23}
\norm{\Rej{\ucap}{\Bv}}^2 =
\Rej{\ucap}{\Bv} \cdot \Bv =
\inv{\Bu^2}\sum_{i<j} {\left(D_{ij}^{\Bu \Bv}\right)}^2
\end{equation}
%
with the length of the base \(\norm{\Bu}\). [FIXME: insert figure.]

Thus the area (squared) is:
%
\begin{equation}\label{eqn:reciprocal_frame:parallogramarea}
\AreaOp{\Bu,\Bv}^2 = \sum_{i<j} {\left(D_{ij}^{\Bu \Bv}\right)}^2
\end{equation}
%
For the special case of a vector in \R{2} this is
\begin{equation}\label{eqn:reciprocal_frame:parallogramarear2}
\AreaOp{\Bu,\Bv} = \abs{D_{12}^{\Bu \Bv}} = \AbsName\left(\DETuvij{u}{v}{i}{j}\right)
\end{equation}
%
\subsection{Vector orthogonal to two vectors in direction of a third}

The same procedure can be followed for three vectors, but the algebra gets messier.  Given three vectors \(\Bu\), \(\Bv\), and \(\Bw\)
we can calculate the component \(\Bw'\) of \(\Bw\) perpendicular to \(\Bu\) and \(\Bv\).  That is:
%
\begin{equation}\label{eqn:reciprocalFrame:403}
\begin{aligned}
\Bv' &= \Bv - \Bv \cdot \ucap \ucap \\
\implies & \\
\Bw' &= \Bw - \Bw \cdot \ucap \ucap - \Bw \cdot \hat{\Bv'} \hat{\Bv'}
\end{aligned}
\end{equation}
%
After expanding this out, a number of the terms magically cancel out and one is left with
%
\begin{equation}\label{eqn:reciprocalFrame:423}
\begin{aligned}
\Bw'' = \Bw' (\Bu^2\Bv^2 - (\Bu \cdot \Bv)^2)
&= \Bu \left(-\Bu \cdot \Bw \Bv^2 + (\Bu \cdot \Bv)(\Bv \cdot \Bw)\right)  \\
&+ \Bv \left(-\Bu^2(\Bv \cdot \Bw) - (\Bu \cdot \Bv)(\Bu \cdot \Bw)\right)  \\
&+ \Bw \left(\Bu^2\Bv^2 - (\Bu \cdot \Bv)^2\right) \\
\end{aligned}
\end{equation}
%
And this in turn can be expanded in terms of coordinates and the results collected yielding
%
\begin{equation}\label{eqn:reciprocalFrame:443}
\begin{aligned}
\Bw'' &= \sum \Be_i u_j v_k \left(
u_i \DETuvij{v}{w}{j}{k}
-v_i \DETuvij{u}{w}{j}{k}
w_i \DETuvij{u}{v}{j}{k}
\right) \\
&= \sum \Be_i u_j v_k \DETuvwijk{u}{v}{w}{i}{j}{k} \\
&= \sum_{i,j<k} \Be_i \DETuvij{u}{v}{j}{k} \DETuvwijk{u}{v}{w}{i}{j}{k} \\
&=
\left(\sum_{i<j<k} + \sum_{j<i<k} + \sum_{j<k<i} \right) \Be_i \DETuvij{u}{v}{j}{k} \DETuvwijk{u}{v}{w}{i}{j}{k}.
\end{aligned}
\end{equation}
%
Expanding the sum of the denominator in terms of coordinates:
\begin{equation}\label{eqn:reciprocalFrame:43}
\Bu^2\Bv^2 - (\Bu \cdot \Bv)^2 = \sum_{i<j} \DETuvij{u}{v}{i}{j}^2
\end{equation}
%
and using a change of summation indices, our final result for the vector perpendicular to two others in the direction of a third is:
%
\begin{equation}\label{eqn:reciprocal_frame:orthotwovectors}
\Rej{\ucap,\vcap}{\Bw} =
\frac{\sum_{i<j<k} \DETuvwijk{u}{v}{w}{i}{j}{k} \DETuvwijk{u}{v}{\Be}{i}{j}{k}}
{\sum_{i<j} \DETuvij{u}{v}{i}{j}^2}
\end{equation}
%
As a small aside, it is notable here to observe that
\(\Span\left\{\DETuvij{u}{\Be}{i}{j}\right\}\) is the null space for the vector \(\Bu\), and
the set \(\Span\left\{\DETuvwijk{u}{v}{\Be}{i}{j}{k}\right\}\) is the null space for the two vectors \(\Bu\) and \(\Bv\) respectively.

Since the rejection is a normal to the set of vectors it must necessarily include these cross product like determinant terms.

As in \eqnref{eqn:reciprocal_frame:rejonevectorD}, use of a \(D_{ijk}^{\Bu\Bv\Bw}\) notation allows for a more compact
result:
%
\begin{equation}\label{eqn:reciprocal_frame:rejtwovectorsD}
\Rej{\ucap\vcap}{\Bw} =
{\left(\sum_{i<j} \left(D_{ij}^{\Bu\Bv}\right)^2\right)}^{-1}
\sum_{i<j<k} D_{ijk}^{\Bu\Bv\Bw} D_{ijk}^{\Bu\Bv\Be}
\end{equation}
%
And, as before this yields the Volume of the parallelepiped by multiplying perpendicular height:
%
\begin{equation}\label{eqn:reciprocalFrame:63}
\norm{\Rej{\ucap\vcap}{\Bw}} =
\Rej{\ucap\vcap}{\Bw} \cdot \Bw =
{\left(\sum_{i<j} \left(D_{ij}^{\Bu\Bv}\right)^2\right)}^{-1}
\sum_{i<j<k} \left(D_{ijk}^{\Bu\Bv\Bw} \right)^2
\end{equation}
%
by the base area.

Thus the squared volume of a parallelepiped spanned by the three vectors is:
%
\begin{equation}\label{eqn:reciprocal_frame:parallopipedvolume}
\VolumeOp{\Bu,\Bv,\Bw}^2 = \sum_{i<j<k} {\left(D_{ijk}^{\Bu \Bv \Bw}\right)}^2.
\end{equation}
%
The simplest case is for \R{3} where we have only one summand:
%
\begin{equation}\label{eqn:reciprocal_frame:parallopipedvolumer3}
\VolumeOp{\Bu,\Bv,\Bw}
= \abs{D_{ijk}^{\Bu \Bv \Bw}}
= \AbsName\left(
\DETuvwijk{u}{v}{w}{1}{2}{3}
\right).
\end{equation}
%
\subsection{Generalization.  Inductive Hypothesis}

There are two things to prove

\begin{enumerate}
\item hypervolume of parallelepiped spanned by vectors \(\Bu_1, \Bu_2, \dots, \Bu_k\)
%
\begin{equation} \label{eqn:reciprocal_frame:hypervolume}
\VolumeOp{\Bu_1, \Bu_2, \cdots, \Bu_k}^2
=
\sum_{i_1 < i_2 < \cdots < i_k } \left(
D_{i_1 i_2 \cdots i_k}^{\Bu_{i_1} \Bu_{i_2} \cdots \Bu_{i_k}}
\right)^2
\end{equation}
%
\item Orthogonal rejection of a set of vectors in direction of another.
%
\begin{equation} \label{eqn:reciprocal_frame:hyperrejection}
\Rej{\ucap_1\cdots\ucap_{k-1}}{\Bu_k} =
\frac{
\sum_{i_1 < \cdots < i_{k} }
D_{i_1 \cdots i_{k}}^{\Bu_{i_1} \cdots \Bu_{i_{k}}}
D_{i_1 \cdots i_{k}}^{\Bu_{i_1} \cdots \Bu_{i_{k-1}} \Be }}
{
\sum_{i_1 < \cdots < i_{k-1} } \left(D_{i_1 \cdots i_{k-1}}^{\Bu_{i_1} \cdots \Bu_{i_{k-1}}}\right)^2
}
\end{equation}
\end{enumerate}

I cannot recall if I ever did the inductive proof for this.
Proving for the initial case is done (since it is proved for both the
two and three vector cases).  For the limiting case where \(k=n\) it can be observed that this is normal to all the others, so the
only thing to prove for that case is if the scaling provided by hypervolume \eqnref{eqn:reciprocal_frame:hypervolume} is correct.

\subsection{Scaling required for reciprocal frame vector}

Presuming an inductive proof of the general result of \eqnref{eqn:reciprocal_frame:hyperrejection} is possible, this rejection
has the property
%
\begin{equation*}
\Rej{\ucap_1\cdots\ucap_{k-1}}{\Bu_k} \cdot \Bu_i \propto \delta_{ki}
\end{equation*}
%
With the scaling factor picked so that this equals \(\delta_{ki}\), the resulting ``reciprocal frame vector'' is
%
\begin{equation} \label{eqn:reciprocal_frame:framevec}
\Bu^k =
\frac{
\sum_{i_1 < \cdots < i_{k} }
D_{i_1 \cdots i_{k}}^{\Bu_{i_1} \cdots \Bu_{i_{k}}}
D_{i_1 \cdots i_{k}}^{\Bu_{i_1} \cdots \Bu_{i_{k-1}} \Be }}
{
\sum_{i_1 < \cdots < i_{k} } \left(D_{i_1 \cdots i_{k}}^{\Bu_{i_1} \cdots \Bu_{i_{k}}}\right)^2
}
\end{equation}
%
The superscript notation is borrowed from Doran/Lasenby, and denotes not a vector raised to a power, but this
this special vector satisfying the following orthogonality and scaling criteria:
%
\begin{equation}\label{eqn:reciprocal_frame:reciportho}
\Bu^k \cdot \Bu_i = \delta_{ki}.
\end{equation}
%
Note that for \(k=n-1\), \eqnref{eqn:reciprocal_frame:framevec} reduces to
%
\begin{equation} \label{eqn:reciprocal_frame:framevecnminus}
\Bu^n =
\frac{ D_{1 \cdots (n-1)}^{\Bu_1 \cdots \Bu_{n-1} \Be} } { D_{1 \cdots n}^{\Bu_1 \cdots \Bu_n} }.
\end{equation}
%
This or some other scaled version of this is likely as close as we can come to generalizing the cross product
as an operation that takes vectors to vectors.

\subsection{Example.  \texorpdfstring{\R{3}}{3D} case.  Perpendicular to two vectors}

Observe that for \R{3}, writing \(\Bu = \Bu_1\), \(\Bv = \Bu_2\), \(\Bw = \Bu_3\), and \(\Bw' = {\Bu_3}^3\) this is:
%
\begin{equation}
\Bw' =
\frac{\DETuvwijk{u}{v}{\Be}{1}{2}{3}}{\DETuvwijk{u}{v}{w}{1}{2}{3}}
=
\frac{\Bu \cross \Bv}{(\Bu \cross \Bv) \cdot \Bw}
\end{equation}
%
This is the cross product scaled by the (signed) volume for the parallelepiped spanned by the three vectors.

\section{Derivation with GA}

Regression with respect to a set of vectors can be expressed directly.  For vectors \({\Bu_i}\) write \(\BB = \Bu_1 \wedge \Bu_2 \cdots \Bu_k\).  Then for any vector we have:
%
\begin{equation}\label{eqn:reciprocalFrame:463}
\begin{aligned}
\Bx
&= \Bx \BB \inv{\BB}  \\
&= \gpgradeone{ \Bx \BB \inv{\BB} } \\
&= \gpgradeone{ (\Bx \cdot \BB + \Bx \wedge \BB) \inv{\BB} }
\end{aligned}
\end{equation}
%
All the grade three and grade five terms are selected out by the grade one operation, leaving just
%
\begin{equation}
\Bx = (\Bx \cdot \BB) \cdot \inv{\BB} + (\Bx \wedge \BB) \cdot \inv{\BB}.
\end{equation}
%
This last term is the rejective component.
%
\begin{equation}\label{eqn:reciprocal_frame:bladerejection}
\Rej{\BB}{\Bx} =
(\Bx \wedge \BB) \cdot \inv{\BB}
=
\frac{
(\Bx \wedge \BB) \cdot {\BB}^\dagger
}
{
\BB \BB^\dagger
}
\end{equation}
%
Here we see in the denominator the squared sum of determinants in the denominator of \eqnref{eqn:reciprocal_frame:hyperrejection}:
%
\begin{equation}\label{eqn:reciprocalFrame:83}
\BB \BB^\dagger =
\sum_{i_1 < \cdots < i_{k} } \left(D_{i_1 \cdots i_{k}}^{\Bu_{i_1} \cdots \Bu_{i_{k}}}\right)^2
\end{equation}
%
In the numerator we have the dot product of two wedge products, each expressible as sums of determinants:
%
\begin{equation}\label{eqn:reciprocalFrame:103}
\BB^\dagger = (-1)^{k(k-1)/2}
\sum_{i_1 < \cdots < i_{k} }
D_{i_1 \cdots i_{k}}^{\Bu_{i_1} \cdots \Bu_{i_{k}}} \Be_{i_1} \Be_{i_2} \cdots \Be_{i_{k}}
\end{equation}
%
And
\begin{equation}\label{eqn:reciprocalFrame:123}
\Bx \wedge \BB =
\sum_{i_1 < \cdots < i_{k+1} }
D_{i_1 \cdots i_{k+1}}^{\Bx \Bu_{i_1} \cdots \Bu_{i_{k}}} \Be_{i_1} \Be_{i_2} \cdots \Be_{i_{k+1}}
\end{equation}
%
Dotting these is all the grade one components of the product.
Performing that calculation would likely provide an explicit confirmation of the inductive hypothesis of
\eqnref{eqn:reciprocal_frame:hyperrejection}.  This can be observed directly for the \(k+1=n\) case.  That product produces a Laplace
expansion sum.
%
\begin{equation}\label{eqn:reciprocalFrame:483}
\begin{aligned}
(\Bx \wedge \BB) \cdot \BB^\dagger
&=
%((-1)^{k(k-1)/2})^2
D_{1 2 \cdots n}^{\Bx \Bu_{1} \cdots \Bu_{n-1}}
\left(
\Be_{1} D_{2 3 4 \cdots n}^{\Bu_{1} \cdots \Bu_{n-1}}
-\Be_{2} D_{1 3 4 \cdots n}^{\Bu_{1} \cdots \Bu_{n-1}}
+\Be_{3} D_{1 2 4 \cdots n}^{\Bu_{1} \cdots \Bu_{n-1}}
\right)
\end{aligned}
\end{equation}
%
\begin{equation}\label{eqn:reciprocal_frame:hyperrejectionganminus}
(\Bx \wedge \BB) \cdot \inv{\BB}
=
\frac{
D_{1 2 \cdots n}^{\Bx \Bu_{1} \cdots \Bu_{n-1}}
D_{1 2 \cdots n}^{\Be \Bu_{1} \cdots \Bu_{n-1}}
}
{
\sum_{i_1 < \cdots < i_{k} } \left(D_{i_1 \cdots i_{k}}^{\Bu_{i_1} \cdots \Bu_{i_{k}}}\right)^2
}
\end{equation}
%
Thus \eqnref{eqn:reciprocal_frame:hyperrejection} for the \(k = n-1\) case is proved without induction.  A proof for the \(k+1<n\) case would be harder.
No proof is required if one picks the set of basis vectors \({\Be_i}\) such that \(\Be_i \wedge \BB = 0\) (then the \(k+1=n\) result applies).
I believe that proves the general case too if one observes that a rotation to any other basis in the span of the set of vectors only
changes the sign of the each of the determinants, and the product of the two sign changes will then have value one.

Follow through of the details for a proof of original non GA induction hypothesis is probably not worthwhile since this
reciprocal frame vector problem can be
tackled with a different approach using a subspace pseudovector.

It is notable that although this had no induction in the argument above, note that it is fundamentally required.
That is because there is an inductive proof
required to prove that the general wedge and dot product vector formulas:
%
\begin{equation}\label{eqn:reciprocalFrame:143}
\Bx \cdot \BB = \inv{2}(\Bx \BB - (-1)^k\BB \Bx)
\end{equation}
\begin{equation}\label{eqn:reciprocalFrame:163}
\Bx \wedge \BB = \inv{2}(\Bx \BB + (-1)^k\BB \Bx)
\end{equation}
%
from the GA axioms (that is an easier proof without the mass of indices and determinant products.)

\section{Pseudovector from rejection}

As noted in the previous section the reciprocal frame vector \(\Bu^k\) is the vector in the direction of \(\Bu_k\) that has no component
in \(\Span{ \Bu_1, \cdots, \Bu_{k-1}}\), normalized such that \(\Bu_k \cdot \Bu^k = 1\).  Explicitly, with
\(\BB = \Bu_1 \wedge \Bu_2 \cdots \wedge \Bu_{k-1}\) this is:
%
\begin{equation}\label{eqn:reciprocal_frame:reciprej}
\Bu^k =
\frac
{
(\Bu_k \wedge \BB) \cdot \BB
}
{
\Bu_k \cdot \left((\Bu_k \wedge \BB) \cdot \BB\right)
}
\end{equation}
%
This is derived from \eqnref{eqn:reciprocal_frame:bladerejection}, after noting that
\(\frac{\BB^\dagger}{\BB\BB^\dagger} \propto \BB\), and further
scaling to produce the desired orthonormal property of equation
\eqnref{eqn:reciprocal_frame:reciportho}
that defines the reciprocal frame vector.

\subsection{back to reciprocal result}

Now,
\eqnref{eqn:reciprocal_frame:reciprej}
looks considerably different from the Doran/Lasenby result.
Reduction to a direct pseudovector/blade product is possible since the
dot product here can be converted to a direct product.
%
\begin{equation}\label{eqn:reciprocalFrame:503}
\begin{aligned}
(\Bu_k \wedge \BB) \cdot \BB
&=
\mathLabelBox
[
   labelstyle={xshift=2cm},
   linestyle={out=270,in=90, latex-}
]
{(\Bx \BB)}{\(\Bx = \Bu_k - (\Bu_k \cdot \BB) \cdot \inv{\BB}\)}
\cdot \BB \\
&= \gpgradeone{
\Bx\BB \BB
} \\
&= \Bx \BB^2 \\
&= \left(\left(\Bu_k - (\Bu_k \cdot \BB) \cdot \inv{\BB}\right) \wedge \BB\right) \BB \\
&= (\Bu_k \wedge \BB) \BB \\
\end{aligned}
\end{equation}
%
Thus \eqnref{eqn:reciprocal_frame:reciprej} is a scaled pseudovector for the subspace
defined by \(\Span {\Bu_i}\), multiplied by a k-1 blade.

\section{Components of a vector}

The delta property of
\eqnref{eqn:reciprocal_frame:reciportho} allows one to use the reciprocal frame
vectors and the basis that generated them to calculate the coordinates
of the a vector with respect to this (not necessarily orthonormal) basis.

That is a pretty powerful result, but somewhat obscured by the Doran/Lasenby
super/sub script notation.

Suppose one writes a vector in \(\Span{\Bu_i}\) in terms of unknown coefficients
%
\begin{equation}\label{eqn:reciprocalFrame:183}
\Ba = \sum a_i \Bu_i
\end{equation}
%
Dotting with \(\Bu^j\) gives:
%
\begin{equation}\label{eqn:reciprocalFrame:203}
\Ba \cdot \Bu^j
= \sum a_i \Bu_i \cdot \Bu^j
= \sum a_i \delta_{ij}
= a_j
\end{equation}
%
Thus
\begin{equation}\label{eqn:reciprocal_frame:nonrecipcomponents}
\Ba = \sum (\Ba \cdot \Bu^i) \Bu_i
\end{equation}
%
Similarly, writing this vectors in terms of \(\Bu^i\) we have
%
\begin{equation}\label{eqn:reciprocalFrame:223}
\Ba = \sum b_i \Bu^i
\end{equation}
%
Dotting with \(\Bu_j\) gives:
%
\begin{equation}\label{eqn:reciprocalFrame:243}
\Ba \cdot \Bu_j
= \sum b_i \Bu^i \cdot \Bu_j
= \sum b_i \delta_{ij}
= b_j
\end{equation}
%
Thus
\begin{equation}\label{eqn:reciprocal_frame:recipcomponents}
\Ba = \sum (\Ba \cdot \Bu_i) \Bu^i
\end{equation}
%
We are used to seeing the equation for components of a vector in terms of a
basis in the following form:
%
\begin{equation}
\Ba = \sum (\Ba \cdot \Bu_i) \Bu_i
\end{equation}
%
This is true only when the basis vectors are orthonormal.
Equations
\eqnref{eqn:reciprocal_frame:nonrecipcomponents} and \eqnref{eqn:reciprocal_frame:recipcomponents} provide the
general decomposition of a vector in terms of a general linearly independent
set.

\subsection{Reciprocal frame vectors by solving coordinate equation}

A more natural way to these results are to take repeated wedge products.
Given a vector decomposition in terms of a basis \({\Bu_i}\), we want to solve for \(a_i\):
%
\begin{equation}\label{eqn:reciprocalFrame:263}
\Ba = \sum_{i=1}^k a_i \Bu_i
\end{equation}
%
The solution, from the wedge is:
%
\begin{equation}\label{eqn:reciprocalFrame:283}
\Ba \wedge (\Bu_1 \wedge \Bu_2 \cdots \check{\Bu_i} \cdots \wedge \Bu_k  = a_i (-1)^{i-1} \Bu_1 \wedge \cdots \wedge \Bu_k
\end{equation}
\begin{equation}\label{eqn:reciprocalFrame:303}
\implies
a_i =
(-1)^{i-1}
\frac{\Ba \wedge (\Bu_1 \wedge \Bu_2 \cdots \check{\Bu_i} \cdots \wedge \Bu_k}{ \Bu_1 \wedge \cdots \wedge \Bu_k })
\end{equation}
%
The complete vector in terms of components is thus:
%
\begin{equation}\label{eqn:reciprocal_frame:coordinateswedge}
\Ba =
\sum(-1)^{i-1}
\frac{\Ba \wedge \left(\Bu_1 \wedge \Bu_2 \cdots \check{\Bu_i} \cdots \wedge \Bu_k\right)}{ \Bu_1 \wedge \cdots \wedge \Bu_k } \Bu_i
\end{equation}
%

We are used to seeing the coordinates expressed in terms of dot products instead of wedge products.  As in \R{3} where
the pseudovector allows wedge products to be expressed in terms of the dot product we can do the same for the general case.

Writing \(\BB \in \bigwedge^{k-1}\) and \(\BI \in \bigwedge^k\) we want to reduce an equation of the following form
%
\begin{equation}\label{eqn:reciprocal_frame:wedgereduction}
\frac{\Ba \wedge \BB}{\BI} = \inv{\BI} \frac{\Ba\BB + (-1)^{k-1}\BB\Ba}{2}
\end{equation}
%
The pseudovector either commutes or anticommutes with a vector in the subspace depending on the grade
%
\begin{equation}\label{eqn:reciprocalFrame:523}
\begin{aligned}
\BI \Ba
&= \BI \cdot \Ba + \mathLabelBox{\BI \wedge \Ba}{\(=0\)} \\
&= (-1)^{k-1}\Ba \cdot \BI \\
&= (-1)^{k-1}\Ba \BI \\
\end{aligned}
\end{equation}
%
Substituting back into \eqnref{eqn:reciprocal_frame:wedgereduction} we have
%
\begin{equation}\label{eqn:reciprocalFrame:543}
\begin{aligned}
\frac{\Ba \wedge \BB}{\BI}
&= (-1)^{k-1} \frac{\Ba \left(\inv{\BI}\BB\right) + \left(\inv{\BI}\BB\right)\Ba}{2} \\
&= (-1)^{k-1} \Ba \cdot \left(\inv{\BI}\BB\right) \\
&= \Ba \cdot \left(\BB \inv{\BI}\right) \\
\end{aligned}
\end{equation}
%
With \(\BI = \Bu_1 \wedge \cdots \Bu_k\), and \(\BB = \Bu_1 \wedge \Bu_2 \cdots \check{\Bu_i} \cdots \wedge \Bu_k\),
back substitution back into \eqnref{eqn:reciprocal_frame:coordinateswedge} is thus
%
\begin{equation}\label{eqn:reciprocalFrame:563}
\begin{aligned}
\Ba
%&= \sum(-1)^{i-1} (-1)^{k-1} \Ba \cdot \left(\inv{\BI}\BB\right) \Bu_i \\
&= \sum
\Ba \cdot \left(
(-1)^{i-1}
\BB \inv{\BI}\right) \Bu_i
\end{aligned}
\end{equation}
%
The final result yields the reciprocal frame vector \(\Bu^k\), and we see how to arrive at this result naturally attempting
to answer the question of how to find the coordinates of a vector with respect to a (not necessarily orthonormal) basis.
%
\begin{equation}\label{eqn:reciprocal_frame:recipdot}
\Ba =
\sum
\Ba \cdot
\mathLabelBox{
\left((\Bu_1 \wedge \Bu_2 \cdots \check{\Bu_i} \cdots \wedge \Bu_k) \frac{ (-1)^{i-1} }{ \Bu_1 \wedge \cdots \wedge \Bu_k }\right)
}{\(\Bu^k\)}
\Bu_i
\end{equation}
%
\section{Components of a bivector}

To find the coordinates of a bivector with respect to an arbitrary basis we have a similar problem.
For a vector basis \({\Ba_i}\), introduce a bivector basis \({\Ba_i \wedge \Ba_j}\), and write
%
\begin{equation}\label{eqn:reciprocal_frame:bivectorcoord}
\BB = \sum_{u<v} b_{uv} \Ba_u \wedge \Ba_v
\end{equation}
%
Wedging with \(\Ba_i \wedge \Ba_j\) will select all but the \(ij\) component.  Specifically
%
\begin{equation}\label{eqn:reciprocalFrame:583}
\begin{aligned}
\BB \wedge
(\Ba_1 \wedge \cdots \check{\Ba_i} \cdots \check{\Ba_j} \cdots \wedge \Ba_k)
&= b_{ij} \Ba_i \wedge \Ba_j \wedge (\Ba_1 \wedge \cdots \check{\Ba_i} \cdots \check{\Ba_j} \cdots \wedge \Ba_k)  \\
&= b_{ij} (-1)^{j-2 + i-1}(\Ba_1 \wedge \cdots \wedge \Ba_k) \\
\end{aligned}
\end{equation}
%
Thus
\begin{equation}\label{eqn:reciprocal_frame:bivectorrecip}
b_{ij} = (-1)^{i+j-3}\BB \wedge
\frac{ (\Ba_1 \wedge \cdots \check{\Ba_i} \cdots \check{\Ba_j} \cdots \wedge \Ba_k) }
{\Ba_1 \wedge \cdots \wedge \Ba_k}
\end{equation}
%
We want to put this in dot product form like \eqnref{eqn:reciprocal_frame:recipdot}.  To do so we need a generalized grade reduction formula
%
\begin{equation}\label{eqn:reciprocal_frame:gradereduction}
(\BA_a \wedge \BA_b) \cdot \BA_c = \BA_a \cdot (\BA_b \cdot \BA_c)
\end{equation}
%
This holds when \(a + b \le c\).  Writing
\(\BA = \Ba_1 \wedge \cdots \check{\Ba_i} \cdots \check{\Ba_j} \cdots \wedge \Ba_k\), and
\(\BI = \Ba_1 \wedge \cdots \wedge \Ba_k\), we have
%
\begin{equation}\label{eqn:reciprocalFrame:603}
\begin{aligned}
(\BB \wedge \BA) \inv{\BI}
&= (\BB \wedge \BA) \cdot \inv{\BI} \\
&= \BB \cdot \left( \BA \cdot \inv{\BI} \right) \\
&= \BB \cdot \left( \BA \inv{\BI} \right) \\
\end{aligned}
\end{equation}
%
Thus the bivector in terms of its coordinates for this basis is:
%
\begin{equation}\label{eqn:reciprocal_frame:bivectordecomp}
\sum_{u<v}
\BB \cdot
\left(
(\Ba_1 \wedge \cdots \check{\Ba_u} \cdots \check{\Ba_v} \cdots \wedge \Ba_k)
\frac{(-1)^{u+v-2-1}}
{\Ba_1 \wedge \cdots \wedge \Ba_k}
\right)
\Ba_u \wedge \Ba_v
\end{equation}
%
It is easy to see how this generalizes to higher order blades since
\eqnref{eqn:reciprocal_frame:gradereduction} is good for all required grades.  In all cases, the form is going to be the same, with only differences
in sign and the number of omitted vectors in the \(\BA\) blade.

For example for a trivector
%
\begin{equation}\label{eqn:reciprocalFrame:323}
\BT = \sum_{u<v<w}t_{uvw} \Ba_u \wedge \Ba_v \wedge \Ba_w
\end{equation}
%
It is pretty straightforward to show that this can be decomposed as follows
%
\begin{equation}\label{eqn:reciprocal_frame:trivectordecomp}
\BT = \sum_{u<v<w} \BT \cdot
\left(
(\Ba_1 \wedge \cdots \check{\Ba_u} \cdots \check{\Ba_v} \cdots \check{\Ba_w} \cdots \wedge \Ba_k)
\frac{(-1)^{u+v+w-3-2-1}}
{\Ba_1 \wedge \cdots \wedge \Ba_k}
\right)
\Ba_u \wedge \Ba_v \wedge \Ba_w
\end{equation}
%
\subsection{Compare to GAFP}

Doran/Lasenby's GAFP
demonstrates \eqnref{eqn:reciprocal_frame:recipdot}, and with some incomprehensible steps skips to a generalized
result of the form
\footnote{ In retrospect I do not think that the in between steps had anything to do with logical sequence.  The authors wanted some of the results for subsequent stuff (like: rotor recovery) and sandwiched it between the vector and reciprocal frame multivector results somewhat out of sequence.}
%
\begin{equation}\label{eqn:reciprocal_frame:bivectordecompwithrecipbivector}
\BB = \sum_{i<j} \BB \cdot \left(\Ba^j \wedge \Ba^i\right) \Ba_i \wedge \Ba_j
\end{equation}
%
GAFP states this for general multivectors instead of bivectors, but the idea is the same.

This makes intuitive sense based on the very similar vector result.  This does not show that
the generalized reciprocal frame k-vectors calculated in
\eqnref{eqn:reciprocal_frame:bivectordecomp} or \eqnref{eqn:reciprocal_frame:trivectordecomp} can be produced simply
by wedging the corresponding individual reciprocal frame vectors.

To show that either takes algebraic identities that I do not know, or am not thinking of as applicable.
Alternately perhaps it would just take simple brute force.

Easier is to demonstrate the validity of the final result directly.  Then assuming my direct calculations
are correct implicitly demonstrates equivalence.

Starting with \(\BB\) as defined in \eqnref{eqn:reciprocal_frame:bivectorcoord}, take dot products with
\(\Ba^j \wedge \Ba^i\).
%
\begin{equation}\label{eqn:reciprocalFrame:623}
\begin{aligned}
\BB \cdot (\Ba^j \wedge \Ba^i)
 &= \sum_{u<v} b_{uv} (\Ba_u \wedge \Ba_v) \cdot (\Ba^j \wedge \Ba^i) \\
 &= \sum_{u<v} b_{uv}
\begin{vmatrix}
\Ba_u \cdot \Ba^i & \Ba_u \cdot \Ba^j \\
\Ba_v \cdot \Ba^i & \Ba_v \cdot \Ba^j \\
\end{vmatrix} \\
 &= \sum_{u<v} b_{uv}
\begin{vmatrix}
\delta_{ui} & \delta_{uj} \\
\delta_{vi} & \delta_{vj} \\
\end{vmatrix} \\
\end{aligned}
\end{equation}
%
Consider this determinant when \(u=i\) for example
\begin{equation}\label{eqn:reciprocalFrame:343}
\begin{vmatrix}
\delta_{ui} & \delta_{uj} \\
\delta_{vi} & \delta_{vj} \\
\end{vmatrix}
=
\begin{vmatrix}
1 & \delta_{ij} \\
\delta_{vi} & \delta_{vj} \\
\end{vmatrix}
=
\begin{vmatrix}
1 & 0 \\
\delta_{vi} & \delta_{vj} \\
\end{vmatrix}
= \delta_{vj}
\end{equation}
%
If any one index is common, then both must be common (\(ij=uv\)) for this determinant to have a non-zero (ie: one) value.  On the other hand, if no index is common then all the \(\delta\)'s are zero.

Like
\eqnref{eqn:reciprocal_frame:reciportho}
this demonstrates an orthonormal selection behavior like the reciprocal frame vector.  It has the action:
%
\begin{equation}
(\Ba_i \wedge \Ba_j) \cdot (\Ba^v \wedge \Ba^u) = \delta_{ij,uv}
\end{equation}
%
This means that we can write \(b_{uv}\) directly in terms of a bivector dot product
%
\begin{equation}\label{eqn:reciprocalFrame:643}
\begin{aligned}
b_{uv} = \BB \cdot (\Ba^v \wedge \Ba^u)
\end{aligned}
\end{equation}
%
and thus proves \eqnref{eqn:reciprocal_frame:bivectordecompwithrecipbivector}.  Proof of the general result
also follows from the determinant expansion of the respective blade dot products.

\subsection{Direct expansion of bivector in terms of reciprocal frame vectors}

Looking at linear operators I realized that the result for bivectors above can follow more easily from direct expansion of a bivector written in terms of vector factors:
%
\begin{equation}\label{eqn:reciprocalFrame:663}
\begin{aligned}
\Ba \wedge \Bb
&= \sum (\Ba \cdot \Bu_i \Bu^i) \wedge (\Bb \cdot \Bu_j \Bu^j) \\
&= \sum_{i<j} \left(\Ba \cdot \Bu_i \Bb \cdot \Bu_j - \Ba \cdot \Bu_j \Bb \cdot \Bu_i \right) \Bu^i \wedge \Bu^j \\
&= \sum_{i<j}
\begin{vmatrix}
\Ba \cdot \Bu_i  & \Ba \cdot \Bu_j  \\
\Bb \cdot \Bu_i & \Bb \cdot \Bu_j  \\
\end{vmatrix}
\Bu^i \wedge \Bu^j \\
\end{aligned}
\end{equation}
%
When the set of vectors \(\Bu_i = \Bu^i\) are orthonormal we have already
calculated this result when looking at the wedge product in a differential
forms context:
%
\begin{equation}
\Ba \wedge \Bb = \sum_{i<j} \DETuvij{a}{b}{i}{j} \Bu_i \wedge \Bu_j
\end{equation}
%
For this general case for possibly non-orthonormal frames, this
determinant of dot products can be recognized as the dot product of two blades
%
\begin{equation}\label{eqn:reciprocalFrame:683}
\begin{aligned}
( \Ba \wedge \Bb ) \cdot (\Bu_j \wedge \Bu_i)
&= \Ba \cdot (\Bb \cdot (\Bu_j \wedge \Bu_i)) \\
&= \Ba \cdot (\Bb \cdot \Bu_j \Bu_i - \Bb \cdot \Bu_i \Bu_j) \\
&= \Bb \cdot \Bu_j \Ba \cdot \Bu_i - \Bb \cdot \Bu_i \Ba \cdot \Bu_j \\
\end{aligned}
\end{equation}
%
Thus we have a decomposition of the bivector directly into a sum of components
for the reciprocal frame bivectors:
%
\begin{equation}
\Ba \wedge \Bb
= \sum_{i<j} \left((\Ba \wedge \Bb) \cdot (\Bu_j \wedge \Bu_i) \right) \Bu^i \wedge \Bu^j
\end{equation}

   \chapter{Matrix review}\label{chap:PJMatrixReview}
      %
% Copyright � 2012 Peeter Joot.  All Rights Reserved.
% Licenced as described in the file LICENSE under the root directory of this GIT repository.
%

%
%
%\chapter{Matrix review}\label{chap:PJMatrixReview}
\index{matrix}
%\date{April 11, 2008.  projectionWithMatrixComparison.tex}

\section{Motivation}

My initial intention for subset of notes was to get a feel for the similarities and differences between GA and matrix approaches to solution of projection.  Attempting to write up that comparison I found gaps in my understanding of the matrix algebra.  In particular the topic of projection as well as the related ideas of pseudoinverses and SVD were not adequately covered in my university courses, nor my texts from those courses.
Here is my attempt to write up what I understand of these subjects and explore the gaps in my knowledge.

Particularly helpful was
Gilbert Strang's excellent MIT lecture on subspace projection (available on the MIT opencourseware website).
Much of the notes below are probably detailed in his course textbook.
%, which I do not have.
%However, if I can not explain the ideas to myself, or write them up in a way that I feel would explain to others, then I obviously do not understand them sufficiently.

There is some GA content here, but the focus in this chapter is not neccessarily GA.

\section{Subspace projection in matrix notation}
\index{projection!matrix}

\subsection{Projection onto line}

\imageFigure{../figures.gabook/Projection_line}{Projection onto line}{fig:Projection_line}{0.4}

The simplest sort of projection to compute is projection onto a line.  Given a direction vector \(b\), and a line with direction vector \(u\) as in \cref{fig:Projection_line}.

The projection onto \(u\) is some value:

\begin{equation}\label{eqn:projectionWithMatrixComparison:20}
p = \alpha u
\end{equation}

and we can write

\begin{equation}\label{eqn:projectionWithMatrixComparison:40}
b = p + e
\end{equation}

where e is the component perpendicular to the line \(u\).

Expressed in terms of the dot product this relationship is described by:

\begin{equation}\label{eqn:projectionWithMatrixComparison:60}
(b - p) \cdot a = 0
\end{equation}

Or,

\begin{equation}\label{eqn:projectionWithMatrixComparison:80}
b \cdot a = \alpha a \cdot a
\end{equation}

and solving for \(\alpha\) and substituting we have:

\begin{equation}
p = a \frac{a \cdot b}{a \cdot a}
\end{equation}

In matrix notation that is:

\begin{equation}
p = a \left(\frac{a^\T b}{a^\T a} \right)
\end{equation}

Following Gilbert Strang's MIT lecture on subspace projection, the parenthesis can be moved to directly express this as a projection matrix operating on b.

\begin{equation}\label{eqn:projMatrixCompare:projectmatrixline}
p = \left(\frac{a a^\T}{a^\T a}\right) b = P b
\end{equation}


\subsection{Projection onto plane (or subspace)}
\index{plane projection!matrix}



\imageFigure{../figures.gabook/Projection_plane}{Projection onto plane}{fig:Projection_plane}{0.4}

Calculation of the projection matrix to project onto a plane is similar.  The variables to solve for are \(p\), and \(e\) in as \cref{fig:Projection_plane}.

For projection onto a plane (or hyperplane) the idea is the same, splitting the vector into a component in the plane and an perpendicular component.
Since the
idea is the same for any dimensional subspace, explicit specification of the summation range is omitted here so the result is good for higher dimensional
subspaces as well as the plane:

\begin{equation}\label{eqn:projectionWithMatrixComparison:100}
b - p = e
\end{equation}
\begin{equation}\label{eqn:projectionWithMatrixComparison:120}
p = \sum \alpha_i u_i
\end{equation}

however, we get a set of equations, one for each direction vector in the plane

\begin{equation}\label{eqn:projectionWithMatrixComparison:140}
(b - p) \cdot u_i = 0
\end{equation}

Expanding \(p\) explicitly and rearranging we have the following set of equations:

\begin{equation}\label{eqn:projectionWithMatrixComparison:160}
b \cdot u_i = (\sum_s \alpha_s u_s) \cdot u_i
\end{equation}

putting this in matrix form

\begin{equation}\label{eqn:projectionWithMatrixComparison:560}
\begin{aligned}
[b \cdot u_i]_i
&=
{
\begin{bmatrix}
(\sum_s \alpha u_s) \cdot u_i
\end{bmatrix}
}_i \\
\end{aligned}
\end{equation}

Writing $U =
\begin{bmatrix}
u_1 & u_2 & \cdots
\end{bmatrix}$
\begin{equation}\label{eqn:projectionWithMatrixComparison:580}
\begin{aligned}
\begin{bmatrix}
u_1^\T \\
u_2^\T \\
\vdots
\end{bmatrix}
b
&=
\begin{bmatrix}
(\sum_s \alpha_s u_s) \cdot u_i
\end{bmatrix} \\
&=
{
\begin{bmatrix}
u_i \cdot u_j
\end{bmatrix}
}_{ij}
\begin{bmatrix}
\alpha_1 \\
\alpha_2 \\
\vdots \\
\end{bmatrix} \\
\end{aligned}
\end{equation}

Solving for the vector of unknown coefficients \(\alpha = [\alpha_i]_i\) we have

\begin{equation}\label{eqn:projectionWithMatrixComparison:180}
\alpha
=
{{
\begin{bmatrix}
u_i \cdot u_j
\end{bmatrix}
}_{ij}}^{-1}
U^\T b
\end{equation}

And

\begin{equation}\label{eqn:projMatrixCompare:projectionwitheucliandot}
p = U \alpha = U
{{
\begin{bmatrix}
u_i \cdot u_j

\end{bmatrix}
}_{ij}}^{-1}
U^\T b
\end{equation}

However, this matrix in the middle is just \(U^\T U\):

\begin{equation}\label{eqn:projectionWithMatrixComparison:600}
\begin{aligned}
\begin{bmatrix}
u_1^\T \\
u_2^\T \\
\vdots \\
\end{bmatrix}
\begin{bmatrix}
{u_1} & {u_2} & \hdots \\
\end{bmatrix}
&=
\begin{bmatrix}
u_1^\T {u_1} & u_1^\T {u_2} & \hdots \\
u_2^\T {u_1} & u_2^\T {u_2} & \hdots \\
\vdots & & \\
\end{bmatrix} \\
&=
{
\begin{bmatrix}
u_i^\T {u_j}
\end{bmatrix}
}_{ij} \\
&=
{
\begin{bmatrix}
{u_i} \cdot {u_j}
\end{bmatrix}
}_{ij} \\
\end{aligned}
\end{equation}

This provides the final result:

\begin{equation}\label{eqn:projMatrixCompare:projectiongeneralmatrix}
\Proj_{U}\left(b\right) = U (U^\T U)^{-1} U^\T b
\end{equation}

\subsection{Simplifying case.  Orthonormal basis for column space}

To evaluate \eqnref{eqn:projMatrixCompare:projectiongeneralmatrix} we need only full column rank for \(U\), but this will be messy in general due to the matrix inversion required for the center product.  That can be avoided by picking an orthonormal basis for the vector space that we are projecting on.  With an orthonormal column basis that
central product term to invert is:

\begin{equation}\label{eqn:projectionWithMatrixComparison:200}
U^\T U = [ u_i^\T u_j ]_{ij} = [ \delta_{ij} ]_{ij} = I_{r,r}
\end{equation}

Therefore, the projection matrix can be expressed using the two exterior terms alone:

\begin{equation}\label{eqn:projMatrixCompare:projOrthonormal}
\Proj_U = U (U^\T U)^{-1} U^\T = U U^\T
\end{equation}

\subsection{Numerical expansion of left pseudoscalar matrix with matrix}

Numerically expanding the projection matrix \(A (A^\T A)^{-1}A^\T\) is not something
that we want to do, but the simpler projection matrix of equation
\eqnref{eqn:projMatrixCompare:projOrthonormal} that we get with an orthonormal basis makes this not so daunting.

Let us do this to get a feel for things.

\subsubsection{\texorpdfstring{\R{4}}{4D} plane projection example}

Take a simple projection onto the plane spanned by the following two orthonormal vectors

\begin{equation}\label{eqn:projectionWithMatrixComparison:220}
u_1 =
\frac{\sqrt{2}}{4}
\begin{bmatrix}
1 \\
2 \\
\sqrt{3} \\
0 \\
\end{bmatrix}
\end{equation}

\begin{equation}\label{eqn:projectionWithMatrixComparison:240}
u_2 =
\frac{\sqrt{2}}{4}
\begin{bmatrix}
-\sqrt{3} \\
0 \\
1 \\
2 \\
\end{bmatrix}
\end{equation}

Thus the projection matrix is:

\begin{equation}\label{eqn:projectionWithMatrixComparison:260}
P =
\begin{bmatrix}
u_1 & u_2 \\
\end{bmatrix}
\begin{bmatrix}
u_1^\T \\
u_2^\T \\
\end{bmatrix}
=
\inv{8}
\begin{bmatrix}
1 & -\sqrt{3} \\
2 & 0 \\
\sqrt{3} & 1 \\
0 & 2 \\
\end{bmatrix}
\begin{bmatrix}
1 & 2 & \sqrt{3} & 0 \\
-\sqrt{3} & 0 & 1 & 2 \\
\end{bmatrix}
\end{equation}
\begin{equation}\label{eqn:projectionWithMatrixComparison:280}
\implies
P =
\inv{8}
\begin{bmatrix}
4 & 2 & 0 & -2\sqrt{3} \\
2 & 4 & 2\sqrt{3} & 0 \\
0 & 2\sqrt{3} & 4 & 2 \\
-2\sqrt{3} & 0 & 2 & 4 \\
\end{bmatrix}
=
\begin{bmatrix}
1/2 & 1/4 & 0 & -\sqrt{3}/4 \\
1/4 & 1/2 & \sqrt{3}/4 & 0 \\
0 & \sqrt{3}/4 & 1/2 & 1/4 \\
-\sqrt{3}/4 & 0 & 1/4 & 1/2 \\
\end{bmatrix}
\end{equation}

What can be said about this just by looking at the matrix itself?

\begin{enumerate}
\item
One can verify by inspection that \(P u_1 = u_1\) and \(P u_2 = u_2\).  This is what we expected
so this validates all the math performed so far.  Good!

%, so this
%is has at least one of the expected properties of a projection matrix.
%We also expect that \(P x = 0\) for any \(x \in N(V^\T)\), so if
%it has that property too we can call it the projection matrix for the subspace
%spanned by \(u_1\), and \(u_2\).  We also assume that this is the projection
%matrix for \(A\) (we have not yet shown any explicit
%relationship between this first
%\(r\) column vectors in the matrix \(V\) and the original matrix \(A\)).

\item
It is symmetric.  Analytically, we know to expect this, since for a
a full column rank matrix \(A\) the transpose of the projection matrix is:

\begin{equation}\label{eqn:projectionWithMatrixComparison:300}
P^\T = {\left(A \inv{A^\T A} A^\T \right)}^\T = P.
\end{equation}

\item
In this particular case columns 2,4 and columns 1,3 are each pairs of
perpendicular vectors.  Is something like this to be expected in general for
projection matrices?

\item
We expect this to be a rank two matrix, so the null space has dimension two.  This can be verified.

\end{enumerate}

\subsection{Return to analytic treatment}

Let us look at the matrix for projection onto an orthonormal basis in a bit more detail.  This simpler form allows for
some observations that are a bit harder in the general form.

Suppose we have a vector \(n\) that is perpendicular to all the orthonormal vectors \(u_i\) that span the subspace.  We can then write:

\begin{equation}\label{eqn:projectionWithMatrixComparison:320}
u_i \cdot n = 0
\end{equation}

Or,
\begin{equation}\label{eqn:projectionWithMatrixComparison:340}
u_i^\T n = 0
\end{equation}

In block matrix form for all \(u_i\) that is:

\begin{equation}\label{eqn:projectionWithMatrixComparison:360}
[u_i^\T n]_i = [u_i^\T]_i n = U^\T n = 0
\end{equation}

This is all we need to verify that our projection matrix indeed produces a zero for any vector completely outside of the subspace:

\begin{equation}\label{eqn:projectionWithMatrixComparison:380}
\Proj_U(n) = U (U^\T n) = U 0 = 0
\end{equation}

Now we have seen numerically that \(U U^\T\) is not an identity matrix despite
operating as one on any vector that lies completely in the subspace.
%Although this matrix may
%have blocks of identity matrixes along the diagonal in some cases.

Having seen the action of this matrix on vectors in the null space, we can now
directly examine the action of this matrix on any vector that lies in the
span of the set \(\{u_i\}\).  By linearity it is sufficient to do this calculation
for a particular \(u_i\):

\begin{equation}\label{eqn:projectionWithMatrixComparison:620}
\begin{aligned}
U U^\T u_i
&=
U
\begin{bmatrix}
u_1^\T u_i \\
u_2^\T u_i \\
\vdots \\
u_r^\T u_i \\
\end{bmatrix}
\\
&=
\begin{bmatrix}
{u_1} & {u_2} & \cdots & {u_r} \\
\end{bmatrix}
{
\begin{bmatrix}
\delta_{si}
\end{bmatrix}
}_s \\
&= \sum_{k=1}^{r} u_k \delta_{ki} \\
&= u_i \\
\end{aligned}
\end{equation}

This now completes the validation of the properties of this matrix (in its simpler form with an orthonormal basis for the subspace).

\section{Matrix projection vs. Geometric Algebra}
\index{projection!matrix}

I found it pretty interesting how similar the projection product is to the projection matrix above from traditional matrix algebra.  It is worthwhile
to write this out and point out the similarities and differences.

\subsection{General projection matrix}

We have shown above, provided a matrix A is of full column rank, a projection onto its columns can be written:

\begin{equation}\label{eqn:projectionWithMatrixComparison:400}
\Proj_A(x) = \left( A \frac{1}{A^\T A} A^\T \right) x
\end{equation}

Now contrast this with the projection written in terms of a vector dot product with a non-null blade

\begin{equation}\label{eqn:projectionWithMatrixComparison:420}
\Proj_A(x) = A \left(\inv{A} \cdot x\right) = A \inv{A^\dagger A} \left(A^\dagger \cdot x\right)
\end{equation}

This is a curious correspondence and naive comparison could lead one to think that perhaps there the concepts of matrix
transposition and blade reversal are equivalent.

This is not actually the case since the matrix transposition actually corresponds to
the adjoint operation of a linear transformation for blades.  Be that as it may, there appears to be other situations
other than projections where matrix operations of the form \(B^\T A\) end up with GA equivalents in the form \(B^\dagger A\).  An example
is the rigid body equations where the body angular velocity bivector corresponding to a rotor \(R\) is of the form \(\BOmega = R' R^\dagger\), whereas the matrix
form for a rotation matrix \(R\) is of the form \(\BOmega = R' R^\dagger\).

\subsection{Projection matrix full column rank requirement}

The projection matrix derivations above required full column rank.  A reformulation in terms
of a generalized matrix (Moore-Penrose) inverse, or SVD can eliminate this full column rank requirement for
the formulation of the projection matrix.

We will get to this later, but we never really proved that
full column rank implies \(A^\T A\) invertability.

If one writes out the matrix \(A^\T A\) in full

Now, if \(A = [a_i]_i\), the matrix

\begin{equation}\label{eqn:projMatrixCompare:AtA}
A^\T A
=
\begin{bmatrix}
{a_1} \cdot {a_1} & {a_1} \cdot {a_2} & \hdots \\
{a_2} \cdot {a_1} & {a_2} \cdot {a_2} & \hdots \\
\vdots & & \\
\end{bmatrix}.
\end{equation}

This is an invertible matrix provided \(\{a_i\}_i\) is a linearly independent set of vectors.
For full column rank to imply invertability, it would be sufficient to prove that the
determinant of this matrix was non-zero.

I am not sure how to show that this is true with just matrix algebra, however
one can identify the determinant of the matrix of \eqnref{eqn:projMatrixCompare:AtA}, after an adjustment
of sign for reversion, as the GA dot product of a k-blade:

\begin{equation}\label{eqn:projMatrixCompare:AdagdotA}
(-1)^{k(k-1)/2} (a_1 \wedge \cdots \wedge a_k) \cdot (a_1 \wedge \cdots \wedge a_k).
\end{equation}

Linear independence means that this wedge product is non-zero, and therefore the dot product, and thus original determinant is also non-zero.

When the \(A^\T A\) matrix of \eqnref{eqn:projMatrixCompare:AtA} is invertible that inverse can be written using a cofactor matrix (adjoint expansion):

Let us write this out, where \(C_{ij}\) are the cofactor matrices of \(A^\T A\), we have:

\begin{equation}\label{eqn:projectionWithMatrixComparison:440}
% \Abs for Det:
\inv{A^\T A} = \inv{ \Abs{ A^\T A } } {[ C_{ij} ]}^\T
\end{equation}

Observe that the denominator here is exactly the determinant of \eqnref{eqn:projMatrixCompare:AdagdotA}.  This illustrates
the motivation of Hestenes to label the explicit alternating vector-vector dot product expansion of a
blade-vector dot product the ``generalized Laplace expansion''.

%  This allows us to give additional meaning to the
%\(\inv{A^\T A}A^\T\) factor of the general projection matrix.
%  Operationally this appears to
%provide a way to compute the dot product of a
%blade with vector (just scaled by the determinant and by the sign of the reversion).

\subsection{Projection onto orthonormal columns}

When the
columns of the matrix \(A\) are orthonormal, the projection matrix is reduced to:

\begin{equation}\label{eqn:projectionWithMatrixComparison:460}
\Proj_A(x) = \left( A A^\T \right) x.
\end{equation}

The corresponding GA entity is a projection onto a unit magnitude blade.  With that scaling the inverse term also drops out leaving:

\begin{equation}\label{eqn:projectionWithMatrixComparison:480}
\Proj_A(x) = A \left(A^\dagger \cdot x\right)
\end{equation}

This helps point out the similarity between the matrix inverse \(\inv{A^\T A}\) and the blade product inverse \(\inv{{A^\dagger}A}\) is only on the surface,
since this blade product is only a scalar.

\section{Proof of omitted details and auxiliary stuff}

\subsection{That we can remove parenthesis to form projection matrix in line projection equation}

Remove the parenthesis in some of these expressions may not always be correct, so it is worth demonstrating that this is okay as
done to calculate the projection matrix \(P\) in
\eqnref{eqn:projMatrixCompare:projectmatrixline}.
We only need to look at the numerator since the denominator is a scalar in this case.

\begin{equation}\label{eqn:projectionWithMatrixComparison:640}
\begin{aligned}
(a a^\T) b
&= [ a_i a_j ]_{ij} [b_i]_i \\
&=
{\begin{bmatrix}
\sum_k a_i a_k b_k
\end{bmatrix}
}_i \\
&=
{\begin{bmatrix}
a_i \sum_k a_k b_k
\end{bmatrix}
}_i \\
&= [ a_i ]_i a^\T b \\
&= a (a^\T b) \\
\end{aligned}
\end{equation}



\subsection{Any invertible scaling of column space basis vectors does not change the projection}


Suppose that one introduces an alternate basis for the column space


\begin{equation}\label{eqn:projectionWithMatrixComparison:500}
v_i = \sum \alpha_{ik} u_k
\end{equation}

This can be expressed in matrix form as:

\begin{equation}\label{eqn:projectionWithMatrixComparison:520}
V = U E
\end{equation}

or

\begin{equation}\label{eqn:projectionWithMatrixComparison:540}
U E^{-1} = V
\end{equation}

We should expect that the projection onto the plane expressed with this alternate basis should be identical to the original.  Verification
is straightforward:

\begin{equation}\label{eqn:projectionWithMatrixComparison:660}
\begin{aligned}
\Proj_V
&= V \left(V^\T V\right)^{-1} V^\T \\
&= \left(U E^{-1}\right) \left({\left(U E^{-1}\right)}^\T \left(U E^{-1}\right)\right)^{-1} {\left(U E^{-1}\right)}^\T \\
&= \left(U E^{-1}\right) \left( {E^{-1}}^T U^\T U E^{-1}\right)^{-1} {\left(U E^{-1}\right)}^\T \\
&= \left(U E^{-1}\right) E \left( U^\T U\right)^{-1} E^\T {\left(U E^{-1}\right)}^\T \\
&= U \left( U^\T U\right)^{-1} E^\T {E^{-1}}^\T U^\T \\
&= U \left( U^\T U\right)^{-1} U^\T \\
&= \Proj_U
\end{aligned}
\end{equation}

   \chapter{Oblique projection and reciprocal frame vectors}
      %
% Copyright � 2012 Peeter Joot.  All Rights Reserved.
% Licenced as described in the file LICENSE under the root directory of this GIT repository.
%

%
%
%\chapter{Oblique projection and reciprocal frame vectors}
\index{reciprocal frame}
\index{oblique projection}
\label{chap:obliqueProj}
%\date{May 16, 2008.  obliqueProj.tex}

\section{Motivation}

Followup on wikipedia projection article's description of an oblique
projection.  Calculate this myself.

\section{Using GA.  Oblique projection onto a line}

INSERT DIAGRAM.

Problem is to project a vector \(\Bx\) onto a line with direction \(\pcap\), along a direction vector \(\dcap\).

Write:
%
\begin{equation}\label{eqn:oblique_proj:solveForprojLineUnit}
\Bx + \alpha \dcap = \beta \pcap
\end{equation}
%
and solve for \(\Bp = \beta \pcap\).  Wedging with \(\dcap\) provides the solution:
%
\begin{equation}\label{eqn:obliqueProj:20}
\Bx \wedge \dcap + \alpha \mathLabelBox{\dcap \wedge \dcap}{\(=0\)} = \beta \pcap \wedge \dcap
\end{equation}
\begin{equation}\label{eqn:obliqueProj:40}
\implies
\beta = \frac{\Bx \wedge \dcap}{\pcap \wedge \dcap}
\end{equation}
%
So the ``oblique'' projection onto this line (using direction \(\dcap\)) is:
%
\begin{equation}
\Proj_{\dcap \rightarrow \pcap}(\Bx) =
\frac{\Bx \wedge \dcap}{\pcap \wedge \dcap} \pcap
\end{equation}
%
This also shows that we do not need unit vectors for this sort of projection
operation, since we can scale these two vectors by any quantity since they are
in both the numerator and denominator.

Let \(\BD\), and \(\BP\) be vectors in the directions of \(\dcap\), and \(\pcap\) respectively.  Then the projection can also be written:
%
\begin{equation}\label{eqn:oblique_proj:obliqueGAproj}
\Proj_{\BD \rightarrow \BP}(\Bx) =
\frac{\Bx \wedge \BD}{\BP \wedge \BD} \BP
\end{equation}
%
It is interesting to see projection expressed here without any sort of dot
product when all our previous projection calculations had intrinsic
requirements for a metric.

Now, let us compare this to the matrix forms of projection that we have become familiar with.  For the matrix result we need a metric, but because
this result is intrinsically non-metric, we can introduce one if convenient and express this result with that too.  Such an expansion is:
%
\begin{equation}\label{eqn:obliqueProj:400}
\begin{aligned}
\frac{\Bx \wedge \BD}{\BP \wedge \BD} \BP
&=
\Bx \wedge \BD
\frac{\BD \wedge \BP}{\BD \wedge \BP}
\inv{\BP \wedge \BD}
\BP \\
&=
(\Bx \wedge \BD) \cdot (\BD \wedge \BP)
\inv{\abs{\BP \wedge \BD}^2}
\BP \\
&=
((\Bx \wedge \BD) \cdot \BD) \cdot \BP
\inv{\abs{\BP \wedge \BD}^2}
\BP \\
&=
(
\Bx \BD^2 - \Bx \cdot \BD \BD
) \cdot \BP
\inv{\abs{\BP \wedge \BD}^2}
\BP \\
&=
\frac{\Bx \cdot \BP \BD^2 - \Bx \cdot \BD \BD \cdot \BP}
{\BP^2 \BD^2 - (\BP \cdot \BD)^2
%p ^ d . d ^ p =http://antwrp.gsfc.nasa.gov/apod/ap080424.html p ^ d . d . p = (p d^2 - d.p d) . p = p^2 d^2 - (d.p)^2
}
\BP \\
\end{aligned}
\end{equation}
%
This gives us the projection explicitly:
%
\begin{equation}\label{eqn:oblique_proj:obliqueProjNearMatrixForm}
\Proj_{\BD \rightarrow \BP}(\Bx)
=
\left(\Bx \cdot \frac{\BP \BD^2 - \BD \BD \cdot \BP}{\BP^2 \BD^2 - (\BP \cdot \BD)^2}\right)
\BP
\end{equation}
%
It sure does not simplify things to expand things out, but we now have things prepared to express in matrix form.

Assuming a euclidean metric, and a bit of playing shows that the denominator can be written more simply as:
%
\begin{equation}\label{eqn:obliqueProj:60}
\BP^2 \BD^2 - (\BP \cdot \BD)^2 =
\begin{vmatrix}
U^\T U
\end{vmatrix}
\end{equation}
%
where:
%
\begin{equation}\label{eqn:obliqueProj:80}
U =
\begin{bmatrix}
\BP & \BD \\
\end{bmatrix}
\end{equation}
%
Similarly the numerator can be written:
%
\begin{equation}\label{eqn:obliqueProj:100}
\Bx \cdot \BP \BD^2 - \Bx \cdot \BD \BD \cdot \BP =
D^\T U
\begin{bmatrix}
0 & -1 \\
1 & 0 \\
\end{bmatrix}
U^\T \Bx.
\end{equation}
%
Combining these yields a projection matrix:
%
\begin{equation}\label{eqn:oblique_proj:matrixprojfromwedge}
\Proj_{\BD \rightarrow \BP}(\Bx) =
\left(
\BP
\inv{
\begin{vmatrix}
U^\T U
\end{vmatrix}
}
\BD^\T U
\begin{bmatrix}
0 & -1 \\
1 & 0 \\
\end{bmatrix}
U^\T\right) \Bx.
\end{equation}
%
The alternation above suggests that this is related to the matrix inverse of something.  Let us try to calculate this directly instead.

\section{Oblique projection onto a line using matrices}

Let us start at the same place as in \eqnref{eqn:oblique_proj:solveForprojLineUnit}, except that we know we can discard the unit vectors and work with any vectors in the projection directions:
%
\begin{equation}\label{eqn:oblique_proj:solveForprojLineNonUnit}
\Bx + \alpha \BD = \beta \BP
\end{equation}
%
Assuming an inner product, we have two sets of results:
%
\begin{equation}\label{eqn:obliqueProj:420}
\begin{aligned}
\innerprod{\BP}{\Bx} + \alpha \innerprod{\BP}{\BD} &= \beta \innerprod{\BP}{\BP} \\
\innerprod{\BD}{\Bx} + \alpha \innerprod{\BD}{\BD} &= \beta \innerprod{\BD}{\BP} \\
\end{aligned}
\end{equation}
%
and can solve this for \(\alpha\), and \(\beta\).
%
\begin{equation}\label{eqn:oblique_proj:matrixtosolve}
\begin{bmatrix}
\innerprod{\BP}{\BD} & \innerprod{\BP}{\BP} \\
\innerprod{\BD}{\BD} & \innerprod{\BD}{\BP} \\
\end{bmatrix}
\begin{bmatrix}
-\alpha \\
\beta \\
\end{bmatrix}
=
\begin{bmatrix}
\innerprod{\BP}{\Bx} \\
\innerprod{\BD}{\Bx} \\
\end{bmatrix}
\end{equation}
%
%This can be solved with matrix inversion, but we are only

If our inner product is defined by \(\innerprod{\Bu}{\Bv} = \Bu^* A \Bv\), we have:
%
\begin{equation}\label{eqn:obliqueProj:440}
\begin{aligned}
\begin{bmatrix}
\innerprod{\BP}{\BD} & \innerprod{\BP}{\BP} \\
\innerprod{\BD}{\BD} & \innerprod{\BD}{\BP} \\
\end{bmatrix}
&=
\begin{bmatrix}
{\BP}^* A {\BD} & {\BP}^* A {\BP} \\
{\BD}^* A {\BD} & {\BD}^* A {\BP} \\
\end{bmatrix} \\
&=
{
\begin{bmatrix}
\BP & \BD \\
\end{bmatrix}
}^*
A
\begin{bmatrix}
{\BD} & {\BP} \\
\end{bmatrix} \\
\end{aligned}
\end{equation}
%
Thus the solution to \eqnref{eqn:oblique_proj:matrixtosolve} is
%
\begin{equation}
\begin{bmatrix}
-\alpha \\
\beta \\
\end{bmatrix}
=
\left(
\inv{
{
\begin{bmatrix}
\BP & \BD \\
\end{bmatrix}
}^*
A
\begin{bmatrix}
{\BD} & {\BP} \\
\end{bmatrix}
}
{
\begin{bmatrix}
\BP & \BD \\
\end{bmatrix}
}^*
A
\right)
\Bx
\end{equation}
%
Again writing $U =
\begin{bmatrix}
\BP & \BD \\
\end{bmatrix}
$, this is:
%
\begin{equation}\label{eqn:obliqueProj:460}
\begin{aligned}
\begin{bmatrix}
-\alpha \\
\beta \\
\end{bmatrix}
&=
\left(
\inv{U^* A U
\begin{bmatrix}
0 & 1 \\
1 & 0 \\
\end{bmatrix}
}
U^*
A
\right)
\Bx \\
&=
\left(
\begin{bmatrix}
0 & 1 \\
1 & 0 \\
\end{bmatrix}
\inv{U^* A U
}
U^*
A
\right)
\Bx \\
\end{aligned}
\end{equation}
%
Since we only care about solution for \(\beta\) to find the projection, we have to discard half the inversion work, and just select
that part of the solution (suggests that a Cramer's rule method is more efficient than matrix inversion in this case) :
%
\begin{equation}\label{eqn:obliqueProj:120}
\beta =
\begin{bmatrix}
0 & 1 \\
\end{bmatrix}
\begin{bmatrix}
-\alpha \\
\beta \\
\end{bmatrix}
\end{equation}
%
Thus the solution of this oblique projection problem in terms of matrices is:
%
\begin{equation}\label{eqn:obliqueProj:480}
\begin{aligned}
\Proj_{\BD \rightarrow \BP}(\Bx)
&=
\left(
\BP
\begin{bmatrix}
0 & 1 \\
\end{bmatrix}
\begin{bmatrix}
0 & 1 \\
1 & 0 \\
\end{bmatrix}
\inv{U^* A U
}
U^*
A
\right)
\Bx
\end{aligned}
\end{equation}
%
Which is:
%
\begin{equation}
\Proj_{\BD \rightarrow \BP}(\Bx) =
\left(
\BP
\begin{bmatrix}
1 & 0 \\
\end{bmatrix}
\inv{U^* A U
}
U^*
A
\right)
\Bx
\end{equation}
%
Explicit expansion can be done easily enough to show that this is identical to \eqnref{eqn:oblique_proj:obliqueProjNearMatrixForm}, so
the question of what we were implicitly inverting in \eqnref{eqn:oblique_proj:matrixprojfromwedge} is answered.

\section{Oblique projection onto hyperplane}

Now that we have got this directed projection problem solved for a line in both GA and matrix form, the next logical step is a \(k\)-dimensional hyperplane projection.  The equation to solve is now:
%
\begin{equation}\label{eqn:oblique_proj:solveForprojPlane}
\Bx + \alpha \BD = \sum \beta_i \BP_i
\end{equation}
%
\subsection{Non metric solution using wedge products}

For \(\Bx\) with some component not in the hyperplane, we can wedge with \(P = \BP_1 \wedge \BP_2 \wedge \cdots \wedge \BP_k\)
%
\begin{equation*}
\Bx \wedge P + \alpha \BD \wedge P = \sum_{i=1}^k \beta_i \mathLabelBox{\BP_i \wedge P}{\(=0\)}
\end{equation*}
%- \Bx \wedge P = \alpha \BD \wedge P

Thus the projection onto the hyperplane spanned by \(P\) is going from \(\Bx\) along \(\BD\) is \(\Bx + \alpha \BD\):
%
\begin{equation}\label{eqn:oblique_proj:GAhyperplaneProjection}
\Proj_{\BD \rightarrow P}(\Bx) = \Bx - \frac{\Bx \wedge P}{\BD \wedge P} \BD
\end{equation}
%
\subsubsection{Q: reduction of this}

When P is a single vector we can reduce this to our previous result:
%
\begin{equation}\label{eqn:obliqueProj:500}
\begin{aligned}
\Proj_{\BD \rightarrow \BP}(\Bx)
&= \Bx - \frac{\Bx \wedge \BP}{\BD \wedge \BP} \BD \\
&= \inv{\BD \wedge \BP} \left((\BD \wedge \BP)\Bx - (\Bx \wedge \BP) \BD \right) \\
&= \inv{\BD \wedge \BP} \left( (\BD \wedge \BP) \cdot \Bx - (\Bx \wedge \BP)  \cdot \BD \right) \\
&= \inv{\BD \wedge \BP} \left( \BD \BP \cdot \Bx -\BP \BD \cdot \Bx -\Bx \BP \cdot \BD +\BP \Bx \cdot \BD \right) \\
&= \inv{\BD \wedge \BP} \left( \BD \BP \cdot \Bx -\Bx \BP \cdot \BD \right) \\
\end{aligned}
\end{equation}
%
Which is:
\begin{equation}\label{eqn:oblique_proj:followup}
\Proj_{\BD \rightarrow \BP}(\Bx)
= \inv{\BP \wedge \BD} \BP \cdot ( \BD \wedge \Bx ).
\end{equation}
%
A result that is equivalent to our original \eqnref{eqn:oblique_proj:obliqueGAproj}.  Can we similarly reduce the general result to something of this form.  Initially I wrote:
%
\begin{equation}\label{eqn:obliqueProj:520}
\begin{aligned}
\Proj_{\BD \rightarrow P}(\Bx)
&= \Bx - \frac{\Bx \wedge P}{\BD \wedge P} \BD \\
&= \frac{\BD \wedge P}{\BD \wedge P} \Bx - \frac{\Bx \wedge P}{\BD \wedge P} \BD \\
&= \inv{\BD \wedge P} \left( (\BD \wedge P) \Bx - (\Bx \wedge P) \BD \right) \\
&= \inv{\BD \wedge P} \left( (\BD \wedge P) \cdot \Bx - (\Bx \wedge P) \cdot \BD \right) \\
&= \inv{\BD \wedge P} \left( \BD P \cdot \Bx -P \BD \cdot \Bx - \Bx P \cdot \BD + P \Bx \cdot \BD \right) \\
&= \inv{\BD \wedge P} \left( \BD P \cdot \Bx - \Bx P \cdot \BD \right) \\
&= -\inv{\BD \wedge P} P \cdot (\BD \wedge \Bx ) \\
\end{aligned}
\end{equation}
%
However, I am not sure that about the manipulations done on the last few lines where P has grade greater than 1 (ie: the triple product expansion and recollection later).

\subsection{hyperplane directed projection using matrices}

To solve \eqnref{eqn:oblique_proj:solveForprojPlane} using matrices, we can take a set of inner products:
%
\begin{equation}\label{eqn:obliqueProj:540}
\begin{aligned}
\innerprod{\BD}{\Bx} + \alpha \innerprod{\BD}{\BD} &= \sum_{u=1}^k \beta_u \innerprod{\BD}{\BP_u} \\
\innerprod{\BP_i}{\Bx} + \alpha \innerprod{\BP_i}{\BD} &= \sum_{u=1}^k \beta_u \innerprod{\BP_i}{\BP_u}
\end{aligned}
\end{equation}
%
Write \(\BD = \BP_{k+1}\), and \(\alpha = -\beta_{k+1}\) for symmetry, which reduces this to:
%
\begin{equation}\label{eqn:obliqueProj:560}
\begin{aligned}
\innerprod{\BP_{k+1}}{\Bx} &= \sum_{u=1}^k \beta_u \innerprod{\BP_{k+1}}{\BP_u} + \beta_{k+1} \innerprod{\BP_{k+1}}{\BP_{k+1}}  \\
\innerprod{\BP_i}{\Bx} &= \sum_{u=1}^k \beta_u \innerprod{\BP_i}{\BP_u} + \beta_{k+1} \innerprod{\BP_i}{\BP_{k+1}}
\end{aligned}
\end{equation}
%
That is the following set of equations:
%
\begin{equation}\label{eqn:obliqueProj:140}
\innerprod{\BP_i}{\Bx} = \sum_{u=1}^{k+1} \beta_u \innerprod{\BP_i}{\BP_u}
\end{equation}
%
Which we can now express as a single matrix equation (for \(i,j \in [1,k+1]\)) :
%
\begin{equation}
{
\begin{bmatrix}
\innerprod{\BP_i}{\Bx}
\end{bmatrix}
}_i
=
{
\begin{bmatrix}
\innerprod{\BP_i}{\BP_j}
\end{bmatrix}
}_{ij}
{
\begin{bmatrix}
\beta_i
\end{bmatrix}
}_i
\end{equation}
%
Solving for $\Bbeta =
{
\begin{bmatrix}
\beta_i
\end{bmatrix}
}_i
$, gives:
%
\begin{equation}\label{eqn:obliqueProj:160}
\Bbeta =
\inv{
\begin{bmatrix}
\innerprod{\BP_i}{\BP_j}
\end{bmatrix}
}_{ij}
{
\begin{bmatrix}
\innerprod{\BP_i}{\Bx}
\end{bmatrix}
}_i
\end{equation}
%
The projective components of interest are \(\sum_{i=1}^k \beta_i \BP_i\).  In matrix form that is:
%
\begin{equation}\label{eqn:obliqueProj:580}
\begin{aligned}
\begin{bmatrix}
\BP_1 & \BP_2 & \cdots & \BP_k
\end{bmatrix}
\begin{bmatrix}
\beta_1 \\
\beta_2 \\
\vdots \\
\beta_k
\end{bmatrix}
=
\begin{bmatrix}
\BP_1 & \BP_2 & \cdots & \BP_k
\end{bmatrix}
\begin{bmatrix}
I_{k,k} & 0_{k,1}
\end{bmatrix}
\Bbeta
\end{aligned}
\end{equation}
%
Therefore the directed projection is:
%
\begin{equation}
\Proj_{\BD \rightarrow P}(\Bx)
=
\begin{bmatrix}
\BP_1 & \BP_2 & \cdots & \BP_k
\end{bmatrix}
\begin{bmatrix}
I_{k,k} & 0_{k,1}
\end{bmatrix}
\inv{
\begin{bmatrix}
\innerprod{\BP_i}{\BP_j}
\end{bmatrix}
}_{ij}
{
\begin{bmatrix}
\innerprod{\BP_i}{\Bx}
\end{bmatrix}
}_i
\end{equation}
%
As before writing \( U =
\begin{bmatrix}
\BP_1 & \BP_2 & \cdots & \BP_k & \BD
\end{bmatrix}
\), and write \(\innerprod{\Bu}{\Bv} = \Bu^* A \Bv \).  The directed projection is now:
%
\begin{equation*}
\Proj_{\BD \rightarrow P}(\Bx)
=
\left(
U
\begin{bmatrix}
I_{k,k} \\
0_{1,k} \\
\end{bmatrix}
\begin{bmatrix}
I_{k,k} & 0_{k,1}
\end{bmatrix}
\inv{
U^* A U
}
U^* A
\right)
\Bx
\end{equation*}
\begin{equation}
=
\left(
U
\begin{bmatrix}
I_{k,k} & 0_{k,1} \\
0_{1,k} & 0_{1,1} \\
\end{bmatrix}
\inv{
U^* A U
}
U^* A
\right)
\Bx
\end{equation}
%
\section{Projection using reciprocal frame vectors}

%The calculations above amount to finding components of a vector along the projection direction vector \(\BD\), and components in the hyperplane.
%We can generalize this by taking advantage of the fact that
%componentization with respect to an arbitrary basis can be naturally expressed using the reciprocal frame for that basis.  One of more of these basis vectors can define the ``direction'' to project towards the hyperplane spanned by
%the remaining vectors.

In a sense the projection operation is essentially a calculation of components of vectors that span a given subspace.  We can also calculate these components using a reciprocal frame.  To start with consider
just orthogonal projection, where the equation to solve is:
%
\begin{equation}
\Bx = \Be + \sum \beta_j \BP_j
\end{equation}
%
and \(\Be \cdot \BP_i = 0\).

Introduce a reciprocal frame \(\{\BP^j\}\) that also spans the space of \(\{\BP_j\}\), and is defined by:
%
\begin{equation}\label{eqn:obliqueProj:180}
\BP_i \cdot \BP^j = \delta_{ij}
\end{equation}
%
With this we have:
%
\begin{equation}\label{eqn:obliqueProj:600}
\begin{aligned}
\Bx \cdot \BP^i
&= \mathLabelBox{\Be \cdot \BP^i}{\(=0\)} + \sum \beta_j \BP_j \cdot \BP^i \\
&= \sum \beta_j \delta_{ij} \\
&= \beta_i \\
\end{aligned}
\end{equation}
%
\begin{equation*}
\Bx = \Be + \sum \BP_j (\BP^j \cdot \Bx)
\end{equation*}
%
For a Euclidean metric the projection part of this is:
%
\begin{equation}\label{eqn:oblique_proj:projmatrix}
\Proj_P(\Bx) = \left( \sum \BP_j (\BP^j)^\T \right) \Bx
\end{equation}
%
Note that there is a freedom to remove the dot product that was employed
to form the matrix representation of \eqnref{eqn:oblique_proj:projmatrix} that may not be obvious.
I did not find that this
was obvious, when seen in Prof. Gilbert Strang's MIT OCW lectures, and
had to prove it for myself.  That proof is available at the end of \chapcite{PJMatrixReview} comparing the geometric and matrix projection operations
, in the 'That we can remove parenthesis to form projection matrix in line projection equation.' subsection.

Writing
$P =
\begin{bmatrix}
\BP_1 & \BP_2 & \cdots & \BP_k
\end{bmatrix}
$
and for the reciprocal frame vectors:
$Q =
\begin{bmatrix}
\BP^1 & \BP^2 & \cdots & \BP^k
\end{bmatrix}
$

We now have the following simple calculation for the projection matrix onto a set of linearly independent vectors (columns of \(P\)):
%
\begin{equation}
\Proj_P(\Bx) = P Q^\T \Bx.
\end{equation}
%
Compare to the general projection matrix previously calculated when the columns of \(P\) weare not orthonormal:
%
\begin{equation}
\Proj_P(\Bx) = P \inv{P^\T P} P^\T \Bx
\end{equation}
%
With orthonormal columns the \(P^T P\) becomes identity and the inverse term drops out, and we get something similar with reciprocal frames.  As a side effect this shows us how to calculate without GA the reciprocal frame vectors.  Those vectors
are thus the columns of
%
\begin{equation}
Q = P \inv{P^\T P}
\end{equation}
%
We are thus able to get a specific understanding of some of the interior terms of the general orthogonal projection matrix.

Also note that the orthonormality of these columns is confirmed by observing that \(Q^\T P = \inv{P^\T P} P^\T P = I\).

\subsection{example/verification}

As an example to see that this works write:
%
\begin{equation}\label{eqn:obliqueProj:200}
P =
\begin{bmatrix}
1 & 1 \\
1 & 0 \\
0 & 1 \\
\end{bmatrix}
\end{equation}
%
\begin{equation}\label{eqn:obliqueProj:220}
P^\T P =
\begin{bmatrix}
1 & 1 & 0 \\
1 & 0 & 1 \\
\end{bmatrix}
\begin{bmatrix}
1 & 1 \\
1 & 0 \\
0 & 1 \\
\end{bmatrix}
=
\begin{bmatrix}
2 & 1 \\
1 & 2 \\
\end{bmatrix}
\end{equation}
%
\begin{equation}\label{eqn:obliqueProj:240}
\inv{P^\T P} =
\inv{3}
\begin{bmatrix}
2 & -1 \\
-1 & 2 \\
\end{bmatrix}
\end{equation}
%
\begin{equation}\label{eqn:obliqueProj:260}
Q = P \inv{P^\T P} =
\inv{3}
\begin{bmatrix}
1 & 1 \\
1 & 0 \\
0 & 1 \\
\end{bmatrix}
\begin{bmatrix}
2 & -1 \\
-1 & 2 \\
\end{bmatrix}
=
\inv{3}
\begin{bmatrix}
1 & 1 \\
2 & 1 \\
-1 & 2 \\
\end{bmatrix}
\end{equation}
%
By inspection these columns have the desired properties.

\section{Directed projection in terms of reciprocal frames}

Suppose that one has a set of vectors \(\{\BP_i\}\) that span the subspace that contains the vector
to be projected \(\Bx\).   If one wants to project onto a subset of those \(P_k\),
say, the first \(k\) of \(l\) of these vectors, and wants the projection directed along components
of the remaining \(l-k\) of these vectors, then solution of the following is required:
%
\begin{equation*}
\Bx = \sum_{j=1}^{l} \beta_j \BP_j
\end{equation*}
%
This (affine?) projection is then just the \(\sum_{j=1}^k \beta_j \BP_j\) components of this vector.

Given a reciprocal frame for the space, the solution follows immediately.
%
\begin{equation}\label{eqn:obliqueProj:280}
\BP^i \cdot \Bx = \sum \beta_j \BP^i \cdot \BP_j = \beta_i
\end{equation}
\begin{equation}\label{eqn:obliqueProj:300}
\beta_i = \BP^i \cdot \Bx
\end{equation}
%
Or,
%
\begin{equation}\label{eqn:obliqueProj:320}
\Proj_{P_k}(\Bx) = \sum_{j=1}^k \BP_j \BP^j \cdot \Bx
\end{equation}
%
In matrix form, with inner product \(\Bu \cdot \Bv = \Bu^* A \Bv\), and writing
\(P =
\begin{bmatrix}
\BP_1 & \BP_2 & \cdots & \BP_l
\end{bmatrix}\), and \(Q =
\begin{bmatrix}
\BP^1 & \BP^2 & \cdots & \BP^l
\end{bmatrix}\),
this is:
%
\begin{equation}
\Proj_{P_k}(\Bx) =
\left(P
\begin{bmatrix}
I_k & 0 \\
0   & 0 \\
\end{bmatrix}
Q^* A\right) \Bx
\end{equation}
%
Observe that the reciprocal frame vectors can be expressed as the rows of the matrix
%
\begin{equation*}
Q^* A = \inv{P^* A P} P^* A
\end{equation*}
%
Assuming the matrix of dot product is invertible, the reciprocal frame vectors are the columns of:
\begin{equation}
Q = P \inv{P^* A^* P}
\end{equation}
%
I had expect that the matrix of most dot product forms would also have

\(A = A^*\) (ie: Hermitian symmetric).

That is certainly true for all the vector dot products
I am interested in utilizing. ie: the standard euclidean dot product, Minkowski space time metrics,
and complex field vector space inner products (all of those are not only real symmetric, but are also
all diagonal).  For completeness, exploring this form for a more generalized
form of inner product was also explored in \chapcite{PJprojGen}.

\subsection{Calculation efficiency}

It would be interesting to compare the computational complexity for a set of reciprocal frame vectors calculated with the GA method (where overhat indicates omission) :
%
\begin{equation}\label{eqn:obliqueProj:340}
P^i = (-1)^{i-1} P_1 \wedge \cdots \hat{P_i} \cdots \wedge P_k \inv { P_1 \wedge P_2 \wedge \cdots \wedge P_k }
\end{equation}
%
The wedge in the denominator can be done just once for all the frame vectors.  Is there a way to use that for the numerator too (dividing out the wedge product with the vector in question)?

Calculation of the \(\inv{P^\T P}\) term could be avoided by using SVD.

Writing \(P = U \Sigma V^\T\), the reciprocal frame vectors will be \(Q = U \Sigma \inv{\Sigma^\T \Sigma} V^\T\).

Would that be any more efficient, or perhaps more importantly, for larger degree vectors is that a more numerically stable calculation?

\section{Followup}

Review: Have questionable GA algebra reduction earlier for grade \(>1\) (following \eqnref{eqn:oblique_proj:followup}).

Q: Can a directed frame vector projection be defined in terms of an ``oblique'' dot product.

Q: What applications would a non-diagonal bilinear form have?

Editorial: I have defined the inner product in matrix form with:
%
\begin{equation}\label{eqn:obliqueProj:360}
\innerprod{\Bu}{\Bv} = \Bu^* A \Bv
\end{equation}
%
This is slightly irregular since it is the conjugate of the normal complex
inner product, so in retrospect I would have been better to express things as:
%
\begin{equation}\label{eqn:obliqueProj:380}
\innerprod{\Bu}{\Bv} = \Bu^\T A \overbar{\Bv}
\end{equation}
%
Editorial: I have used the term oblique projection.  In retrospect I think I have really been describing what is closer to an affine (non-metric) projection so that would probably have been better to use.

   \chapter{Projection and Moore-Penrose vector inverse}
      %
% Copyright � 2012 Peeter Joot.  All Rights Reserved.
% Licenced as described in the file LICENSE under the root directory of this GIT repository.
%

%
%
\chapter{Projection and Moore-Penrose vector inverse}
\index{projection}
\index{Moore-Penrose inverse}
\label{chap:projectionAndMoorePenroseVectorInverse}
%\date{May 16, 2008.  projectionAndMoorePenroseVectorInverse.tex}

\section{Projection and Moore-Penrose vector inverse}

One can observe that the Moore Penrose left vector inverse \(\Bv^+\) shows up in the projection matrix for a projection onto a line with a direction vector \(\Bv\):

\begin{equation}
\Proj_\Bv(\Bx) = \Bv \mathLabelBox{\inv{ \Bv^\T \Bv} \Bv^\T}{\(\Bv^+\)} \Bx
\end{equation}

I do not know of any other ``application'' of this Moore-Penrose vector inverse in traditional matrix algebra.  As stated it is an interesting mathematical curiosity that yes one can define a vector inverse, however what would you do with it?

In geometric algebra we also have a vector inverse, but it plays a much more fundamental role, and does not have the restriction of only acting from the left and
producing a scalar result.  As an example consider the projection, and rejection decomposition of a vector:

\begin{equation}\label{eqn:projectionAndMoorePenroseVectorInverse:120}
\begin{aligned}
\Bx
&= \Bv \inv{\Bv} \Bx \\
&= \Bv \left(\inv{\Bv} \cdot \Bx\right) + \Bv \left(\inv{\Bv} \wedge \Bx\right) \\
&= \Bv
\left(
\frac{\Bv}{\Bv^2} \cdot
 \Bx\right)
 + \Bv \left(\frac{\Bv}{\Bv^2} \wedge \Bx\right) \\
\end{aligned}
\end{equation}

In the above, \(\frac{\Bv}{\Bv^2} \cdot = \frac{\Bv^\T}{\Bv^\T \Bv} = \Bv^+\).  We can therefore describe the Moore Penrose vector left inverse as the matrix of the GA linear transformation \(\inv{\Bv} \cdot\).

Unlike the GA vector inverse, whos associativity allowed for the projection/rejection derivation above, this Moore-Penrose vector inverse has only left action, so in the above, you can not further write:

\begin{equation}\label{eqn:projectionAndMoorePenroseVectorInverse:20}
\Bv \Bv^{+} = 1
\end{equation}

(ie: \(\Bv \Bv^{+}\) is a projection matrix not scalar or matrix unity).

\subsection{matrix of wedge project transformation?}

Q: What is the matrix of the linear transformation \(\inv{\Bv} \wedge\)?

In rigid body dynamics we see the matrix of the linear transformation \(T_\Bv(\Bx) = (\Bv \cross)(\Bx)\).  This is the completely antisymmetric matrix as follows:

\begin{equation}
\Bv \times \Bx =
\begin{bmatrix}
0 & -v_3 & v_2 \\
v_3 & 0 & -v_1 \\
-v_2 & v_1 & 0 \\
\end{bmatrix}
\begin{bmatrix}
x_1 \\
x_2 \\
x_3 \\
\end{bmatrix}
\end{equation}

In order to specify the matrix of a vector-vector wedge product linear transformation we must introduce bivector coordinate vectors.  For the matrix of the cross product linear transformation the standard vector basis was the obvious choice.

Let us pick the following orthonormal basis:

\begin{equation}\label{eqn:projectionAndMoorePenroseVectorInverse:40}
\sigma = \{ \sigma_{ij} = \Be_i \wedge \Be_j \}_{i<j}
\end{equation}

and construct the matrix of the wedge project \(T_\Bv : \mathbb{R}^N \rightarrow {\bigwedge}^2\)

\begin{equation}\label{eqn:projectionAndMoorePenroseVectorInverse:60}
T_\Bv(\Bx) = \Bv \wedge \Bx = \sum_{\mu = ij, i<j} \DETuvij{v}{x}{i}{j} \sigma_{\mu}
\end{equation}
\begin{equation}\label{eqn:projectionAndMoorePenroseVectorInverse:80}
\implies
T_\Bv(\Be_k) \cdot {\sigma_{ij}}^\dagger =
\sum_{k \in ij, i<j} \DETuvij{v}{x}{i}{j}
= %\sum_{k \in ij, i<j}
v_i \delta_{kj} - v_j \delta_{ki}
\end{equation}

Since \(k\) cannot be simultaneously equal to both \(i\), and \(j\), this is:

\begin{equation}\label{eqn:projectionAndMoorePenroseVectorInverse:100}
T_\Bv(\Be_k) \cdot {\sigma_{ij}}^\dagger =
\left\{
\begin{array}{rl}
v_i & k=j \\
-v_j & k=i \\
0 & k \ne i,j \\
\end{array}
\right\}
\end{equation}

Unlike the left Moore-Penrose vector inverse that we find as the matrix of the linear transformation \(v \cdot ( \cdot )\), except for \R{3} where we have the cross product, I do not recognize this as the matrix of any common linear transformation.

   \chapter{Angle between geometric elements}
      %
% Copyright � 2012 Peeter Joot.  All Rights Reserved.
% Licenced as described in the file LICENSE under the root directory of this GIT repository.
%

%
%
\chapter{Angle between geometric elements}
\index{vectors!angle between}
\label{chap:angleBetweenLineAndPlane}
%\date{Mar 17, 2008.  angleBetweenLineAndPlane.tex}

Have the calculation for the angle between bivectors done elsewhere

\begin{equation}\label{eqn:angleLinePlane:bivectorangle}
\cos\theta = - \frac{\BA \cdot \BB}{\abs{\BA} \abs{\BB} }
\end{equation}

For \(\theta \in [0,\pi]\).

The vector/vector result is well known and also works fine in \R{N}

\begin{equation}\label{eqn:angleLinePlane:vectorangle}
\cos\theta = \frac{\Bu \cdot \Bv}{\abs{\Bu} \abs{\Bv} }
\end{equation}

\section{Calculation for a line and a plane}

Given a line with unit direction vector \(\Bu\), and plane with unit direction bivector \(\BA\), the component of that
vector in the plane is:

\begin{equation}\label{eqn:angleBetweenLineAndPlane:20}
-\Bu \cdot \BA \BA.
\end{equation}

So the direction cosine is available immediately

\begin{equation}\label{eqn:angleBetweenLineAndPlane:40}
\cos\theta = \Bu \cdot \frac{-\Bu \cdot \BA \BA}{\abs{\Bu \cdot \BA \BA}}
\end{equation}

However, this can be reduced significantly.  Start with the denominator

\begin{equation}\label{eqn:angleBetweenLineAndPlane:60}
\begin{aligned}
\abs{\Bu \cdot \BA \BA}^2
&= (\Bu \cdot \BA \BA)(\BA \BA \cdot \Bu) \\
&= (\Bu \cdot \BA )^2. \\
\end{aligned}
\end{equation}

And in the numerator we have:

\begin{equation}\label{eqn:angleBetweenLineAndPlane:80}
\begin{aligned}
\Bu \cdot (\Bu \cdot \BA \BA)
&= \inv{2}(
  \Bu (\Bu \cdot \BA \BA)
+ (\Bu \cdot \BA \BA) \Bu
) \\
&= \inv{2}(
  (\Bu \Bu \cdot \BA) \BA
+ (\Bu \cdot \BA) \BA \Bu
) \\
&= \inv{2}(
  (\BA \cdot \Bu \Bu) \BA
- (\BA \cdot \Bu) \BA \Bu
) \\
&= (\BA \cdot \Bu) \inv{2}( \Bu \BA - \BA \Bu ) \\
&= -(\BA \cdot \Bu)^2.
\end{aligned}
\end{equation}

Putting things back together

\begin{equation*}
\cos\theta
= \frac{(\BA \cdot \Bu)^2}{\abs{\Bu \cdot \BA}} = \abs{\Bu \cdot \BA}
\end{equation*}

The strictly positive value here is consistent with the fact that theta as calculated is in the \([0,\pi/2]\) range.

Restated for consistency with equations \eqnref{eqn:angleLinePlane:vectorangle} and \eqnref{eqn:angleLinePlane:bivectorangle} in terms of not necessarily
unit vector and bivectors \(\Bu\) and \(\BA\), we have

\begin{equation}
\cos\theta =
\frac{\abs{\Bu \cdot \BA}}{ \abs{\Bu} \abs{\BA} }
\end{equation}

   \chapter{Orthogonal decomposition take II}
      %
% Copyright � 2012 Peeter Joot.  All Rights Reserved.
% Licenced as described in the file LICENSE under the root directory of this GIT repository.
%

%
%
\chapter{Orthogonal decomposition take II}
\label{chap:orthodecomp}
%\date{April 1, 2008.  orthodecomp.tex}
\section{Lemma.  Orthogonal decomposition}
To do so we first need to be able to express a vector \(\Bx\) in terms
of components parallel and perpendicular to the blade \(\BA \in \wedge^k\).

\begin{equation}\label{eqn:orthodecomp:20}
\begin{aligned}
\Bx
&= \Bx \BA \inv{\BA} \\
&= (\Bx \cdot \BA + \Bx \wedge \BA) \inv{\BA} \\
&=
(\Bx \cdot \BA) \cdot \inv{\BA}
+ \sum_{i=3,5,\cdots,2k-1} \gpgrade{(\Bx \cdot \BA) \inv{\BA}}{i} \\
&+
(\Bx \wedge \BA) \cdot \inv{\BA}
+ \sum_{i=3,5,\cdots,2k-1} \gpgrade{(\Bx \wedge \BA) \inv{\BA}}{i}
+ \mathLabelBox{(\Bx \wedge \BA) \wedge \inv{\BA}}{\(=0\)}
\end{aligned}
\end{equation}

Since the LHS and RHS must both be vectors all the non-grade one terms
are either zero or cancel out.  This can be observed directly since:

\begin{equation}\label{eqn:orthodecomp:40}
\begin{aligned}
\gpgrade{\Bx \cdot \BA \inv{\BA}}{i}
&= \gpgrade{ \frac{\Bx \BA - (-1)^{k}\BA\Bx}{2}\inv{\BA} }{i}  \\
&= -\frac{(-1)^{k}}{2} \gpgrade{ \BA\Bx \inv{\BA} }{i}  \\
\end{aligned}
\end{equation}

and

\begin{equation}\label{eqn:orthodecomp:60}
\begin{aligned}
\gpgrade{\Bx \wedge \BA \inv{\BA}}{i}
&= \gpgrade{ \frac{\Bx \BA + (-1)^{k}\BA\Bx}{2}\inv{\BA} }{i}  \\
&= +\frac{(-1)^{k}}{2} \gpgrade{ \BA\Bx \inv{\BA} }{i}  \\
\end{aligned}
\end{equation}

Thus all of the grade \(3, \cdots ,2k-1\) terms cancel each other out.  Some terms
like \((\Bx \cdot \BA) \wedge \inv{\BA}\) are also independently zero.

(This is a result I have got in other places, but I thought it is worth
 writing down since I thought the direct cancellation is elegant).

   \chapter{Matrix of grade k multivector linear transformations}
      %
% Copyright � 2012 Peeter Joot.  All Rights Reserved.
% Licenced as described in the file LICENSE under the root directory of this GIT repository.
%

%
%
%\chapter{Matrix of grade k multivector linear transformations}
\label{chap:matrixOfLinearTx}

%\date{May 16, 2008.  matrixOfLinearTx.tex}

\section{Motivation}

The following shows explicitly the calculation required to form the matrix of a linear transformation between two grade k multivector subspaces (is there a name for a multivector of fixed grade that is not neccessarily a blade?).
This is nothing fancy or original, but just helpful
to have written out showing naturally how one generates this matrix from a consideration of the two sets of basis vectors.  After so much exposure to linear transformations only in matrix
form it is good to write this out in a way so that it is clear exactly how the coordinate matrices come in to the picture when and if they are introduced.

\section{Content}

Given \(T\), a linear transformation between two grade k multivector subspaces,
let \(\sigma = \{\sigma_i\}_{i=1}^m\) be a basis for a grade k multivector subspace.
For \(T(x) \in span\{ \beta_i \}\) (ie: image of T contained in this span).  Let \(\beta = \{\beta_i\}_{i=1}^n\) be a basis for this (possibly different) grade k multivector subspace.

Additionally introduce a set of respective reciprocal frames \(\{\sigma^i\}\), and \(\{\beta^i\}\).
Define the reciprocal frame with respect to the dot product for the space.  For a linearly independent, but not necessary orthogonal (or orthonormal), set of vectors \(\{u_i\}\) this set
has as its defining property:
%
\begin{equation}\label{eqn:matrixOfLinearTx:20}
u_i \cdot u^j = \delta_{ij}
\end{equation}
%
I have chosen to use this covariant, contravariant coordinate notation since that works well for both vectors (not necessarily orthogonal or orthonormal), as well as higher grade vectors.  When the basis is orthonormal these reciprocal frame grade k multivectors can be computed with just reversion.  For example, suppose that \(\{\beta_i\}\) is an orthonormal bivector basis for the image of \(T\), then the reciprocal frame bivectors are just \(\beta^i = {\beta_i}^\dagger\).

With this we can decompose the linear transformation into components generated by each of the \(\sigma_i\) grade k multivectors:
%
\begin{equation}
T(x) = T(\sum x \cdot \sigma_j \sigma^j) = \sum x_j T(\sigma^j)
\end{equation}
%
This we can write as a matrix equation:
%
\begin{equation}
T(x) =
\begin{bmatrix}
T(\sigma^1) & T(\sigma^1) & \cdots & T(\sigma^n)
\end{bmatrix}
\begin{bmatrix}
x_1 \\
x_2 \\
\vdots \\
x_n \\
\end{bmatrix}
\end{equation}
%
Now, decompose the \(T(\sigma^j)\) in terms of the basis \(\beta\):
%
\begin{equation}
T(\sigma^j) = \sum T(\sigma^j) \cdot \beta^i \beta_i
\end{equation}
%
This we can also write as a matrix equation
%
\begin{equation}
\begin{aligned}
T(\sigma^j) &=
\begin{bmatrix}
\beta_1 & \beta_2 & \cdots & \beta_m
\end{bmatrix}
\begin{bmatrix}
T(\sigma^j) \cdot \beta^1 \\
T(\sigma^j) \cdot \beta^2 \\
\vdots \\
T(\sigma^j) \cdot \beta^m \\
\end{bmatrix} \\
&=
\begin{bmatrix}
\beta_1 & \beta_2 & \cdots & \beta_m
\end{bmatrix}
\begin{bmatrix}
\beta^1 \cdot T(\sigma^j) \\
\beta^2 \cdot T(\sigma^j) \\
\vdots \\
\beta^m \cdot T(\sigma^j) \\
\end{bmatrix}
\end{aligned}
\end{equation}
%
These two sets of matrix equations, can be combined into a single equation:
%
\begin{equation}\label{eqn:matOfLinTx:matrixexpansion}
T(x) =
{
\begin{bmatrix}
\beta_1 \\
\beta_2 \\
\vdots \\
\beta_m
\end{bmatrix}
}^{\text{T}}
\begin{bmatrix}
\beta^1 \cdot T(\sigma^1) & \beta^1 \cdot T(\sigma^2) & \cdots & \beta^1 \cdot T(\sigma^n) \\
\beta^2 \cdot T(\sigma^1) & \beta^2 \cdot T(\sigma^2) & \cdots & \beta^2 \cdot T(\sigma^n) \\
\vdots & \cdots & \ddots & \vdots \\
\beta^m \cdot T(\sigma^1) & \beta^m \cdot T(\sigma^2) & \cdots & \beta^m \cdot T(\sigma^n) \\
\end{bmatrix}
\begin{bmatrix}
x_1 \\
x_2 \\
\vdots \\
x_n \\
\end{bmatrix}
\end{equation}
%
Here the matrix \(( x_1, x_2, \cdots, x_n )\) is a coordinate vector with respect to basis \(\sigma\), but the vector
\(( \beta_1, \beta_2, \cdots, \beta_n )\) is matrix of the basis vectors \(\beta_i \in \beta\).  This makes sense since the end result has not been defined in terms of
a coordinate vector space, but the space of T itself.

This can all be written more compactly as
%
\begin{equation}
T(x)
=
{
\begin{bmatrix}
\beta_i \\
\end{bmatrix}
}^{\text{T}}
\begin{bmatrix}
\beta^i \cdot T(\sigma^j)
\end{bmatrix}
\begin{bmatrix}
x_i \\
\end{bmatrix}
\end{equation}
%
We can also recover the original result from this by direct expansion and then regrouping:
%
\begin{equation}\label{eqn:matrixOfLinearTx:40}
\begin{aligned}
{
\begin{bmatrix}
\beta_i \\
\end{bmatrix}
}^{\text{T}}
\begin{bmatrix}
\sum_j \beta^i \cdot T(\sigma^j) x_j
\end{bmatrix}
&=
\begin{bmatrix}
\sum_{kj} \beta_k \beta^k \cdot T(\sigma^j) x_j
\end{bmatrix} \\
&=
\sum_{kj} \beta_k \beta^k \cdot T(\sigma^j) x_j \\
&=
\sum_{k} \beta_k \beta^k \cdot T(\sum_j \sigma^j x_j) \\
&=
\sum_{k} \beta_k \beta^k \cdot T(x) \\
&=
T(x) \\
\end{aligned}
\end{equation}
%
Observe that this demonstrates that we can write the coordinate vector \([T]_\beta\) as the two left most matrices above
%
\begin{equation}\label{eqn:matrixOfLinearTx:60}
\begin{aligned}
[T(x)]_\beta
&=
{
\begin{bmatrix}
\beta^i \cdot T(x)
\end{bmatrix}
}_i \\
&=
{
\begin{bmatrix}
\sum_{j} \beta^i \cdot T(\sigma^j) x_j
\end{bmatrix}
}_i \\
&=
\begin{bmatrix}
\beta^i \cdot T(\sigma^j)
\end{bmatrix}
\begin{bmatrix}
x_i \\
\end{bmatrix}
\end{aligned}
\end{equation}
%
Looking at the above I found it interesting that \eqnref{eqn:matOfLinTx:matrixexpansion} which embeds the coordinate vector of \(T(x)\) has the structure of a bilinear form, so in a sense one can view the
matrix of a linear transformation:
%
\begin{equation}
[T]_\sigma^\beta =
\begin{bmatrix}
\beta^i \cdot T(\sigma^j)
\end{bmatrix}
\end{equation}
%
as a bilinear form that can act as a mapping from the generating basis to the image basis.

   \chapter{Vector form of Julia fractal}
      %
% Copyright � 2012 Peeter Joot.  All Rights Reserved.
% Licenced as described in the file LICENSE under the root directory of this GIT repository.
%

%
%
%\input{../peeter_prologue_print.tex}
%\input{../peeter_prologue_widescreen.tex}

\chapter{Vector form of Julia fractal}
\index{Julia fractal}
\label{chap:juliaVector}

%\blogpage{http://sites.google.com/site/peeterjoot/math2010/juliaVector.pdf}
%\date{Dec 27, 2010}
%\revisionInfo{juliaVector.tex}

%\beginArtWithToc
\beginArtNoToc

\section{Motivation}

As outlined in \citep{dorst2007gac}, 2-D and N-D Julia fractals can be computed using the geometric product, instead of complex numbers.  Explore a couple of details related to that here.

\section{Guts}

Fractal patterns like the Mandelbrot and Julia sets are typically using iterative computations in the complex plane.  For the Julia set, our iteration has the form

\begin{equation}\label{eqn:juliaFractal:10}
Z \rightarrow Z^p + C
\end{equation}

where \(p\) is an integer constant, and \(Z\), and \(C\) are complex numbers.  For \(p=2\) I believe we obtain the Mandelbrot set.  Given the isomorphism between complex numbers and vectors using the geometric product, we can use write

\begin{equation}\label{eqn:juliaFractal:20}
\begin{aligned}
Z &= \Bx \ncap \\
C &= \Bc \ncap,
\end{aligned}
\end{equation}

and re-express the Julia iterator as

\begin{equation}\label{eqn:juliaFractal:30}
\Bx \rightarrow (\Bx \ncap)^p \ncap + \Bc
\end{equation}

It is not obvious that the RHS of this equation is a vector and not a multivector, especially when the vector \(\Bx\) lies in \R{3} or higher dimensional space.  To get a feel for this, let us start by write this out in components for \(\ncap = \Be_1\) and \(p=2\).  We obtain for the product term

\begin{equation}\label{eqn:juliaVector:80}
\begin{aligned}
(\Bx \ncap)^p \ncap
&= \Bx \ncap \Bx \ncap \ncap \\
&= \Bx \ncap \Bx \\
&=
(
x_1 \Be_1
+ x_2 \Be_2  )
\Be_1
(
x_1 \Be_1
+ x_2 \Be_2  ) \\
&=
(
x_1
+ x_2 \Be_2 \Be_1 )
(
x_1 \Be_1
+ x_2 \Be_2  ) \\
&=
(
x_1^2 - x_2^2 ) \Be_1 + 2 x_1 x_2 \Be_2
\end{aligned}
\end{equation}

Looking at the same square in coordinate representation for the \R{n} case (using summation notation unless otherwise specified), we have

\begin{equation}\label{eqn:juliaVector:100}
\begin{aligned}
\Bx \ncap \Bx
&=
x_k \Be_k
\Be_1
x_m \Be_m  \\
&=
\left(x_1 + \sum_{k>1} x_k \Be_k \Be_1\right)
x_m \Be_m  \\
&=
x_1 x_m \Be_m
+
\sum_{k>1} x_k x_m \Be_k \Be_1 \Be_m \\
&=
x_1 x_m \Be_m
+
\sum_{k>1} x_k x_1 \Be_k
+\sum_{k>1,m>1} x_k x_m \Be_k \Be_1 \Be_m \\
&=
\left(x_1^2
-\sum_{k>1} x_k^2\right) \Be_1
+
2 \sum_{k>1} x_1 x_k \Be_k
+\sum_{1 < k < m, 1 < m < k} x_k x_m \Be_k \Be_1 \Be_m \\
\end{aligned}
\end{equation}

This last term is zero since \(\Be_k \Be_1 \Be_m = -\Be_m \Be_1 \Be_k\), and we are left with

\begin{equation}\label{eqn:juliaFractal:50}
\Bx \ncap \Bx =
\left(x_1^2
-\sum_{k>1} x_k^2\right) \Be_1
+
2 \sum_{k>1} x_1 x_k \Be_k,
\end{equation}

a vector, even for non-planar vectors.  How about for an arbitrary orientation of the unit vector in \R{n}?  For that we get

\begin{equation}\label{eqn:juliaVector:120}
\begin{aligned}
\Bx \ncap \Bx
&=
(\Bx \cdot \ncap \ncap + \Bx \wedge \ncap \ncap ) \ncap \Bx  \\
&=
(\Bx \cdot \ncap + \Bx \wedge \ncap ) (\Bx \cdot \ncap \ncap + \Bx \wedge \ncap \ncap )
  \\
&=
((\Bx \cdot \ncap)^2 + (\Bx \wedge \ncap)^2) \ncap
+ 2 (\Bx \cdot \ncap) (\Bx \wedge \ncap) \ncap
\end{aligned}
\end{equation}

We can read \eqnref{eqn:juliaFractal:50} off of this result by inspection for the \(\ncap = \Be_1\) case.

It is now straightforward to show that the product \((\Bx \ncap)^p \ncap\) is a vector for integer \(p \ge 2\).  We have covered the \(p=2\) case, justifying an assumption that this product has the following form

\begin{equation}\label{eqn:juliaFractal:60}
(\Bx \ncap)^{p-1} \ncap = a \ncap + b (\Bx \wedge \ncap) \ncap,
\end{equation}

for scalars \(a\) and \(b\).  The induction test becomes

\begin{equation}\label{eqn:juliaVector:140}
\begin{aligned}
(\Bx \ncap)^{p} \ncap
&= (\Bx \ncap)^{p-1} (\Bx \ncap) \ncap \\
&= (\Bx \ncap)^{p-1} \Bx \\
&= (a + b (\Bx \wedge \ncap) ) ((\Bx \cdot \ncap )\ncap + (\Bx \wedge \ncap) \ncap) \\
&=
( a(\Bx \cdot \ncap )^2 - b (\Bx \wedge \ncap)^2 ) \ncap
+ ( a + b(\Bx \cdot \ncap ) ) (\Bx \wedge \ncap) \ncap.
\end{aligned}
\end{equation}

Again we have a vector split nicely into projective and rejective components, so for any integer power of \(p\) our iterator \eqnref{eqn:juliaFractal:30} employing the geometric product is a mapping from vectors to vectors.

There is a striking image in the text of such a Julia set for such a 3D iterator, and an exercise left for the adventurous reader to attempt to code that based on the 2D \(p=2\) sample code they provide.

%\EndArticle


\part{Rotation}
   \chapter{Rotor Notes}\label{chap:rotor}
      %
% Copyright � 2012 Peeter Joot.  All Rights Reserved.
% Licenced as described in the file LICENSE under the root directory of this GIT repository.
%

%
%
\chapter{Rotor Notes}\label{chap:rotor}
\index{rotor}
%\date{Feb 19, 2008.  rotor.tex}

\section{Rotations strictly in a plane}

For a plane rotation, a rotation does not have to
be expressed in terms of left and right half angle rotations, as is the case
with complex numbers.  Starting with this ``natural'' one sided rotation
we will see why the half angle double sided Rotor formula works.

\subsection{Identifying a plane with a bivector.  Justification}
Given a bivector \(\BB\), we can say this defines the orientation of a plane
(through the origin)
since for any vector in the plane we have \(\BB \wedge \Bx = 0\), or any vector
strictly normal to the plane \(\BB \cdot \Bx = 0\).

Note that this naturally compares
to the equation of a line (through the origin) expressed in terms of a
direction vector \(\Bb\),
where \(\Bb \wedge \Bx=0\) if \(\Bx\) lies on the line, and \(\Bb \cdot \Bx = 0\)
if \(\Bx\) is normal to the line.

Given this it is not unreasonable to identify the plane with its bivector.  This
will be done below, and it should be clear that
loose language such as ``the plane \(\BB\)'', should really be interpreted
as ``the plane with direction bivector \(\BB\)'', where the direction bivector
has the wedge and dot product properties noted above.

\subsection{Components of a vector in and out of a plane}

To calculate the components of a vector in and out of a plane, we can form
the product

\begin{equation}\label{eqn:rotor:20}
\Bx = \Bx \BB \inv{\BB} = \Bx \cdot \BB \inv{\BB} + \Bx \wedge \BB \inv{\BB}
\end{equation}

This is an orthogonal decomposition of the vector \(\Bx\) where the first
part is the projective term onto the plane \(\BB\), and the second is the rejective
term, the component not in the plane.  Let us verify this.

Write \(\Bx = \Bx_\parallel + \Bx_\perp\), where \(\Bx_\parallel\), and \(\Bx_\perp\) are the components of \(\Bx\) parallel and perpendicular to the plane.  Also write
\(\BB = \Bb_1 \wedge \Bb_2\), where \(\Bb_i\) are non-colinear vectors in the plane \(\BB\).

If \(\Bx = \Bx_\parallel\), a vector entirely in the plane \(\BB\), then one can
write

\begin{equation}\label{eqn:rotor:40}
\Bx = a_1\Bb_1 + a_2\Bb_2
\end{equation}

and the wedge product term is zero

\begin{equation}\label{eqn:rotor:740}
\begin{aligned}
\Bx \wedge \BB
&= \left( a_1\Bb_1 + a_2\Bb_2 \right) \wedge \Bb_1 \wedge \Bb_2 \\
&= a_1 ( \Bb_1 \wedge \Bb_1 ) \wedge \Bb_2
 - a_2 ( \Bb_2 \wedge \Bb_2 ) \wedge \Bb_1 \\
&= 0
\end{aligned}
\end{equation}

Thus the component parallel to the plane is composed strictly of the dot
product term

\begin{equation}
\Bx_\parallel = \Bx \cdot \BB \inv{\BB}
\end{equation}

Or for a general vector not necessarily in the plane the component
of that vector in the plane, its projection onto the plane is,

\begin{equation}\label{eqn:rotor:60}
\Proj_{\BB}(\Bx) = \Bx \cdot \BB \inv{\BB}
= \inv{\abs{\BB}^2}(\BB \cdot \Bx)\BB
= (\hat{\BB} \cdot \Bx)\hat{\BB}
\end{equation}

Now, for a vector that lies completely perpendicular to the plane \(\Bx = \Bx_\perp\), the dot product term with the plane is zero.  To verify this observe

\begin{equation}\label{eqn:rotor:760}
\begin{aligned}
\Bx_\perp \cdot \BB
&= \Bx_\perp \cdot (\Bb_1 \wedge \Bb_2) \\
&= (\Bx_\perp \cdot \Bb_1) \Bb_2 - (\Bx_\perp \cdot \Bb_2) \Bb_1 \\
\end{aligned}
\end{equation}

Each of these dot products are zero since \(\Bx\) has no components that lie
in the plane (those components if they existed could be expressed as linear
combinations of \(\Bb_i\)).

Thus only the component perpendicular to the plane is composed strictly of the
wedge product term

\begin{equation}
\Bx_\perp = \Bx \wedge \BB \inv{\BB}
\end{equation}

And again for a general vector the component that lies out
of the plane as, the rejection of the plane from the vector is

\begin{equation}\label{eqn:rotor:80}
\RejName_{\BB}(\Bx)
= \Bx \wedge \BB \inv{\BB}
= -\inv{\abs{\BB}^2} \Bx \wedge \BB {\BB}
= -\Bx \wedge \hat{\BB} \hat{\BB}
\end{equation}

\section{Rotation around normal to arbitrarily oriented plane through origin}

Having established the preliminaries, we can now express a rotation around
the normal to a plane (with the plane and that normal through the origin).

\imageFigure{../../figures/gabook/rotor}{Rotation of Vector}{fig:rotor}{0.4}

Such a rotation is illustrated in \cref{fig:rotor}
preserves all components of the vector that are perpendicular
to the plane, and operates only on the components parallel to the plane.

Expressed in terms of exponentials and the projective and rejective decompositions above, this is

\begin{equation}\label{eqn:rotor:780}
\begin{aligned}
R_\theta(\Bx)
&= \Bx \wedge \BB \inv{\BB} + \left(\Bx \cdot \BB \inv{\BB}\right)e^{\hat{\BB}\theta} \\
&= \Bx \wedge \BB \inv{\BB} + e^{-\hat{\BB}\theta}\left(\Bx \cdot \BB \inv{\BB}\right) \\
\end{aligned}
\end{equation}

Where we have made explicit note that a plane rotation does not commute with a vector in a plane (its reverse is required).

To demonstrate this write \(i = \Be_2 \Be_1\), a unit bivector in some plane with unit vectors \(\Be_i\) also in the plane.  If a vector
lies in that plane we can write the rotation

\begin{equation}\label{eqn:rotor:800}
\begin{aligned}
\Bx e^{i\theta}
&= \left(a_1\Be_1 + a_2\Be_2\right)\left(\cos\theta + i\sin\theta\right) \\
&= \cos\theta\left(a_1\Be_1 + a_2\Be_2\right) + \left(a_1\Be_1 + a_2\Be_2\right)\left(\Be_2 \Be_1\sin\theta\right) \\
&= \cos\theta\left(a_1\Be_1 + a_2\Be_2\right) + \sin\theta \left(-a_1\Be_2 + a_2\Be_1\right) \\
&= \cos\theta\left(a_1\Be_1 + a_2\Be_2\right) -\Be_2 \Be_1\sin\theta \left(a_1\Be_1 + a_2\Be_2\right) \\
&= e^{-i\theta}\Bx \\
\end{aligned}
\end{equation}

Similarly for a vector that lies outside of the plane we can write

\begin{equation}\label{eqn:rotor:820}
\begin{aligned}
\Bx e^{i\theta}
&= (\sum_{j \ne 1,2} a_j \Be_j)(\cos\theta + \Be_2 \Be_1\sin\theta) \\
&= (\cos\theta + \Be_2 \Be_1\sin\theta) (\sum_{j \ne 1,2} a_j \Be_j) \\
&= e^{i\theta}\Bx
\end{aligned}
\end{equation}

The multivector for a rotation in a plane perpendicular to a vector commutes with that vector.  The properties of the
exponential allow us to factor a rotation

\begin{equation}\label{eqn:rotor:100}
R(\theta) = R(\alpha\theta) R((1-\alpha)\theta)
\end{equation}

where \(\alpha <= 1\), and in particular we can set \(\alpha = 1/2\), and write

\begin{equation}\label{eqn:rotor:840}
\begin{aligned}
R_\theta(\Bx)
&= \Bx \wedge \BB \inv{\BB} + \left(\Bx \cdot \BB \inv{\BB}\right)e^{\hat{\BB}\theta} \\
&= \left(\Bx \wedge \BB \inv{\BB}\right) e^{-\hat{\BB}\theta/2} e^{\hat{\BB}\theta/2}
 + \left(\Bx \cdot \BB \inv{\BB} \right) e^{\hat{\BB}\theta/2} e^{\hat{\BB}\theta/2} \\
&= e^{-\hat{\BB}\theta/2} \left(\Bx \wedge \BB \inv{\BB}\right) e^{\hat{\BB}\theta/2}
+ e^{-\hat{\BB}\theta/2} \left(\Bx \cdot \BB \inv{\BB}\right)e^{\hat{\BB}\theta/2} \\
&= e^{-\hat{\BB}\theta/2} \left(\Bx \wedge \BB + \Bx \cdot \BB\right) \inv{\BB} e^{\hat{\BB}\theta/2} \\
&= e^{-\hat{\BB}\theta/2} \left(\Bx \BB \inv{\BB} \right) e^{\hat{\BB}\theta/2}
\end{aligned}
\end{equation}

This takes us full circle from dot and wedge products back to \(\Bx\), and allows us to express the rotated vector as:

\begin{equation}\label{eqn:rotor:rotor}
R_\theta(\Bx)
= e^{-\hat{\BB}\theta/2} \Bx e^{\hat{\BB}\theta/2}
\end{equation}

Only when the vector lies in the plane (\(\Bx = \Bx_\parallel\), or \(\Bx \wedge \BB = 0\)) can be written using the familiar left or right ``full angle'' rotation exponential that we are used to from complex arithmetic:

\begin{equation}\label{eqn:rotor:120}
R_\theta(\Bx) = e^{-\hat{\BB}\theta} \Bx = \Bx e^{\hat{\BB}\theta}
\end{equation}

\section{Rotor equation in terms of normal to plane}

The rotor equation above is valid for any number of dimensions.  For \R{3} we can alternatively parametrize the plane in terms of
a unit normal \(\Bn\):

\begin{equation}\label{eqn:rotor:140}
\BB = k i\Bn
\end{equation}

Here \(i\) is the \R{3} pseudoscalar \(\Be_1 \Be_2 \Be_3\).

Thus we can write

\begin{equation}\label{eqn:rotor:160}
\hat{\BB} = i\Bn
\end{equation}

and expressing \eqnref{eqn:rotor:rotor} in terms of the unit normal becomes trivial

\begin{equation}
R_\theta(\Bx)
= e^{- i {\Bn}\theta/2} \Bx e^{i{\Bn}\theta/2}
\end{equation}

Expressing this in terms of components and the unit normal is a bit harder

\begin{equation}\label{eqn:rotor:860}
\begin{aligned}
R_\theta(\Bx)
&= \Bx \wedge \BB \inv{\BB} + \left(\Bx \cdot \BB \inv{\BB}\right)e^{\hat{\BB}\theta} \\
&= \Bx \wedge (i\Bn) \inv{i\Bn} + \left(\Bx \cdot (i\Bn) \inv{i\Bn}\right)e^{{i\Bn}\theta} \\
\end{aligned}
\end{equation}

Now,

\begin{equation}\label{eqn:rotor:880}
\begin{aligned}
\Bx \wedge (i\Bn)
&= \inv{2}(\Bx i \Bn + i \Bn \Bx) \\
&= \frac{i}{2}(\Bx \Bn + \Bn \Bx) \\
&= (\Bx \cdot \Bn) i
\end{aligned}
\end{equation}

And

\begin{equation}\label{eqn:rotor:900}
\begin{aligned}
\inv{i\Bn}
&= \inv{i\Bn} \inv{\Bn i} \Bn i \\
&= - i \Bn \\
\end{aligned}
\end{equation}

So the rejective term becomes
\begin{equation}\label{eqn:rotor:920}
\begin{aligned}
\Bx \wedge \BB \inv{\BB}
&= \Bx \wedge (i\Bn) \inv{i\Bn} \\
&= \Bx \wedge (i\Bn) \inv{i\Bn} \\
&= (\Bx \cdot \Bn) i (-i) \Bn \\
&= (\Bx \cdot \Bn) \Bn \\
&= \Proj_{\Bn}(\Bx) \\
\end{aligned}
\end{equation}

Now, for the dot product with the plane term, we have

\begin{equation}\label{eqn:rotor:940}
\begin{aligned}
\Bx \cdot \BB
&= \Bx \cdot (i \Bn) \\
&= \inv{2}(\Bx i \Bn - i \Bn \Bx) \\
&= (\Bx \wedge \Bn)i \\
\end{aligned}
\end{equation}

Putting it all together we have

\begin{equation}\label{eqn:rotor:rotexp}
R_\theta(\Bx)
= (\Bx \cdot \Bn) \Bn + (\Bx \wedge \Bn)\Bn e^{{i\Bn}\theta}
\end{equation}

In terms of explicit sine and cosine terms this is (observe that \((i\Bn)^2 = -1\)),

\begin{equation}\label{eqn:rotor:960}
\begin{aligned}
R_\theta(\Bx)
&= \left(\Bx \cdot \Bn\right) \Bn + \left(\Bx \wedge \Bn\right)\Bn \left(\cos\theta + i\Bn \sin\theta\right) \\
\end{aligned}
\end{equation}

\begin{equation}\label{eqn:rotor:rotnorm}
R_\theta(\Bx) =
\left(\Bx \cdot \Bn\right) \Bn + \left(\Bx \wedge \Bn\right)\Bn \cos\theta + (\Bx \wedge \Bn) i \sin\theta
\end{equation}

\imageFigure{../../figures/gabook/normalRot}{Direction vectors associated with rotation}{fig:normalRot}{0.4}

This triplet of mutually orthogonal direction vectors,
\(\Bn\), \((\Bx \wedge \Bn)\Bn\), and \((\Bx \wedge \Bn) i\)
are illustrated in \cref{fig:normalRot}.  The component of the vector in the direction of the normal
\(\Proj_\Bn(\Bx) = \Bx \cdot \Bn \Bn\) is unaltered by the rotation.
The rotation is applied to the remaining component of \(\Bx\), \(\RejName_{\Bn}(\Bx) = (\Bx \wedge \Bn)\Bn\), and we rotate
in the direction \((\Bx \wedge \Bn) i\)

\subsection{Vector rotation in terms of dot and cross products only}

Expression of this rotation formula \eqnref{eqn:rotor:rotnorm} in terms of ``vector'' relations is also possible, by removing the wedge
products and the pseudoscalar references.

First the rejective term

\begin{equation}\label{eqn:rotor:980}
\begin{aligned}
(\Bx \wedge \Bn) \Bn
&= ((\Bx \cross \Bn) i) \Bn \\
&= ((\Bx \cross \Bn) i) \cdot \Bn \\
&= \inv{2} ( ((\Bx \cross \Bn) i) \Bn - \Bn ((\Bx \cross \Bn) i)) \\
&= \frac{i}{2} ( (\Bx \cross \Bn) \Bn - \Bn (\Bx \cross \Bn) ) \\
&= i ( (\Bx \cross \Bn) \wedge \Bn ) \\
&= i^2 ( (\Bx \cross \Bn) \cross \Bn ) \\
&= \Bn \cross (\Bx \cross \Bn) \\
\end{aligned}
\end{equation}

The next term expressed in terms of the cross product is

\begin{equation}\label{eqn:rotor:1000}
\begin{aligned}
(\Bx \wedge \Bn) i
&=
(\Bx \cross \Bn) i^2 \\
&= \Bn \cross \Bx \\
\end{aligned}
\end{equation}

And putting it all together we have

\begin{equation}\label{eqn:rotor:rotcross}
R_\theta(\Bx) =
\left(\Bx \cdot \Bn\right) \Bn
 + \left(\Bn \cross \Bx\right) \cross \Bn \cos\theta
 + \Bn \cross \Bx \sin\theta
\end{equation}

Compare \eqnref{eqn:rotor:rotcross} to \eqnref{eqn:rotor:rotnorm} and \eqnref{eqn:rotor:rotexp}, and then back to \eqnref{eqn:rotor:rotor}.

\section{Giving a meaning to the sign of the bivector}

For a rotation between two vectors in the plane containing those vectors, we can write the rotation
in terms of the exponential as either a left or right rotation operator:

\begin{equation}\label{eqn:rotor:180}
\Bb = \Ba e^{\Bi\theta} = e^{-\Bi\theta}\Ba
\end{equation}
\begin{equation}\label{eqn:rotor:200}
\Bb = e^{\Bj\theta}\Ba = \Ba e^{-\Bj\theta/2}
\end{equation}

Here both \(\Bi\) and \(\Bj=-\Bi\) are unit bivectors with the property \(\Bi^2 = \Bj^2 = -1\).
Thus in order to write a rotation in exponential form a meaning must be assigned to the sign of the unit bivector that describes the
plane and the orientation of the rotation.

Consider for example the case of a rotation by \(\pi/2\).  For this is the exponential is:

\begin{equation}\label{eqn:rotor:220}
e^{\Bi\pi/2} = \cos(\pi/2) + \Bi \sin(\pi/2) = \Bi
\end{equation}

Thus for perpendicular unit vectors \(\Bu\) and \(\Bv\), if we wish \(\Bi\) to act as a \(\pi/2\) rotation left acting operator on \(\Bu\)
towards \(\Bv\) its value must be:

\begin{equation}\label{eqn:rotor:240}
\Bi = \Bu \wedge \Bv
\end{equation}
\begin{equation}\label{eqn:rotor:260}
\Bu\Bi = \Bu \Bu \wedge \Bv = \Bu\Bu\Bv = \Bv
\end{equation}

For that same rotation if the bivector is employed as a right acting operator, the reverse is required:

\begin{equation}\label{eqn:rotor:280}
\Bj = \Bv \wedge \Bu
\end{equation}
\begin{equation}\label{eqn:rotor:300}
\Bj\Bu = \Bv \wedge \Bu = \Bv\Bu\Bu = \Bv
\end{equation}

\imageFigure{../../figures/gabook/imaginaryorientation}{Orientation of unit imaginary}{fig:imaginaryorientation}{0.4}

In general, for any two vectors, one can find an angle \(\theta\) in the range \(0 \le \theta \le \pi\) between those vectors.
If one lets that angle define the orientation of the rotation between the vectors, and implicitly
define a sort of ``imaginary axis'' for that plane, that imaginary axis will have direction

\begin{equation}\label{eqn:rotor:320}
\inv{\Ba} \Ba \wedge \Bb = \Bb \wedge \Ba \inv {\Ba}.
\end{equation}

This is illustrated in \cref{fig:imaginaryorientation}.

Thus the bivector

\begin{equation}\label{eqn:rotor:340}
\Bi = \frac{\Ba \wedge \Bb}{\abs{\Ba \wedge \Bb}}
\end{equation}

When acting as an operator to the left (\(\Ba \Bi\)) with a vector in the plane can be interpreted as acting as a rotation by \(\pi/2\) towards \(\Bb\).

Similarly the bivector

\begin{equation}\label{eqn:rotor:360}
\Bj = \Bi^\dagger = -\Bi = \frac{\Bb \wedge \Ba}{\abs{\Bb \wedge \Ba}}
\end{equation}

also applied to a vector in the plane produces the same rotation when
acting as an operator to the right.  Thus, in general we can write
a rotation by theta in the plane containing non-colinear vectors \(\Ba\) and \(\Bb\) in the direction of minimal angle
from \(\Ba\) towards \(\Bb\) in one of the three forms:

\begin{equation}\label{eqn:rotor:380}
R_{\theta : \Ba \rightarrow \Bb}(\Ba)
= \Ba e^{ \frac{\Ba \wedge \Bb}{\abs{\Ba \wedge \Bb}} \theta }
= e^{ \frac{\Bb \wedge \Ba}{\abs{\Bb \wedge \Ba}} \theta } \Ba
\end{equation}

Or,
\begin{equation}\label{eqn:rotor:400}
R_{\theta : \Ba \rightarrow \Bb}(\Bx)
= e^{ \frac{\Bb \wedge \Ba}{\abs{\Bb \wedge \Ba}} \theta/2 } \Bx e^{ \frac{\Ba \wedge \Bb}{\abs{\Ba \wedge \Bb}} \theta/2 }
\end{equation}

This last (writing \(\Bx\) instead of \(\Ba\) since it also applies to vectors that lie outside of the \(\Ba \wedge \Bb\) plane),
is our rotor formula \eqnref{eqn:rotor:rotor}, reexpressed in a way that removes the sign ambiguity of the bivector \(\Bi\) in that equation.

\section{Rotation between two unit vectors}

\imageFigure{../../figures/gabook/parallelogramvec}{Sum of unit vectors bisects angle between}{fig:parallelogramvec}{0.4}

As illustrated in \cref{fig:parallelogramvec}, when the angle between two vectors is less than \(\pi\)
the fact that the sum of two arbitrarily oriented unit vectors bisects those vectors provides a convenient
way to compute the half angle rotation exponential.

Thus we can write

\begin{equation*}
\frac{\Ba + \Bb}{\abs{\Ba + \Bb}} = \Ba e^{\Bi\theta/2} = e^{\Bj\theta/2} \Ba
\end{equation*}

Where \(\Bi = \Bj^\dagger\) are unit bivectors of appropriate sign.  Multiplication through by \(\Ba\) gives

\begin{equation*}
e^{\Bi\theta/2} =
\frac{1 + \Ba\Bb}{\abs{\Ba + \Bb}}
\end{equation*}

Or,
\begin{equation*}
e^{\Bj\theta/2} =
\frac{1 + \Bb\Ba}{\abs{\Ba + \Bb}}
\end{equation*}

Thus we can write the total rotation from \(\Ba\) to \(\Bb\) as

\begin{equation*}
\Bb
= e^{-\Bi\theta/2} \Ba e^{\Bi\theta/2}
= e^{\Bj\theta/2} \Ba e^{-\Bj\theta/2}
= \left(\frac{1 + \Bb\Ba}{\abs{\Ba + \Bb}}\right) \Ba \left(\frac{1 + \Ba\Bb}{\abs{\Ba + \Bb}}\right)
\end{equation*}

For the case where the rotation is through an angle \(\theta\) where \(\pi < \theta < 2\pi\), again employing a left acting
exponential operator we have

\begin{equation}\label{eqn:rotor:1020}
\begin{aligned}
\frac{\Ba + \Bb}{\abs{\Ba + \Bb}}
&= \Bb e^{\Bi(2\pi - \theta)/2} \\
&= \Bb e^{\Bi \pi} e^{- \Bi\theta/2} \\
&= -\Bb e^{- \Bi\theta/2} \\
\end{aligned}
\end{equation}

Or,
\begin{equation}\label{eqn:rotor:420}
e^{- \Bi\theta/2} = -\frac{\Bb\Ba + 1}{\abs{\Ba + \Bb}}
\end{equation}

Thus

\begin{equation}\label{eqn:rotor:rotunit}
\Bb = e^{- \Bi\theta/2} \Ba e^{ \Bi\theta/2} =
\left(-\frac{1 + \Bb\Ba}{\abs{\Ba + \Bb}}\right) \Ba \left(-\frac{1 + \Ba\Bb}{\abs{\Ba + \Bb}}\right)
\end{equation}

Note that the two negatives cancel, giving the same result as in the \(\theta < \pi\) case.  Thus \eqnref{eqn:rotor:rotunit} is valid for all vectors \(\Ba \ne -\Bb\) (this can be verified by direct multiplication.)

These
half angle exponentials are called rotors, writing the rotor as

\begin{equation}\label{eqn:rotor:440}
R = \frac{1 + \Ba\Bb}{\abs{\Ba + \Bb}}
\end{equation}

and the rotation in terms of rotors is:

\begin{equation}\label{eqn:rotor:460}
\Bb = R^\dagger \Ba R
\end{equation}

The angle associated with this rotor \(R\) is the minimal angle between the two vectors (\(0 < \theta < \pi\)), and is directed from \(\Ba\) to \(\Bb\).  Inverting the rotor will not change the net effect of the rotation, but has the geometric meaning that the rotation from \(\Ba\) to \(\Bb\)
rotates in the opposite direction through the larger angle (\(\pi < \theta < 2\pi\)) between the vectors.

\section{Eigenvalues, vectors and coordinate vector and matrix of the rotation linear transformation}

Given the plane containing two orthogonal vectors \(\Bu\) and \(\Bv\), we can form a unit bivector for the plane

\begin{equation}\label{eqn:rotor:480}
\BB = \Bu\Bv
\end{equation}

A normal to this plane is \(\Bn = \Bv\Bu I\).

The rotation operator for a rotation around \(\Bn\) in that plane (directed from \(\Bu\) towards \(\Bv\)) is

\begin{equation}\label{eqn:rotor:500}
R_\theta(\Bx) = e^{\Bv\Bu \theta/2} \Bx e^{\Bu\Bv \theta/2}
\end{equation}

To form the matrix of this linear transformation assume an orthonormal basis \(\sigma = \{ \Be_i \}\).

In terms of these basis vectors we can write

\begin{equation}\label{eqn:rotor:520}
R_\theta(\Be_j) =
e^{-\Bv\Bu \theta/2} \Be_j e^{\Bu\Bv \theta/2}
=
\sum_i \left(e^{-\Bv\Bu \theta/2} \Be_j e^{\Bu\Bv \theta/2}\right) \cdot \Be_i \Be_i
\end{equation}

Thus the coordinate vector for this basis is

\begin{equation}\label{eqn:rotor:540}
{
\begin{bmatrix}
R_\theta(\Be_j)
\end{bmatrix}
}_\sigma
=
\begin{bmatrix}
\left(e^{-\Bv\Bu \theta/2} \Be_j e^{\Bu\Bv \theta/2}\right) \cdot \Be_1 \\
\vdots \\
\left(e^{-\Bv\Bu \theta/2} \Be_j e^{\Bu\Bv \theta/2}\right) \cdot \Be_n \\
\end{bmatrix}
\end{equation}

We can use this to form the matrix for the linear operator that takes coordinate vectors from
the basis \(\sigma\) to \(\sigma\):

\begin{equation}\label{eqn:rotor:560}
{
\begin{bmatrix}
R_\theta(\Bx)
\end{bmatrix}
}_\sigma
=
{
\begin{bmatrix}
R_\theta
\end{bmatrix}
}_\sigma^\sigma
{
\begin{bmatrix}
\Bx
\end{bmatrix}
}_\sigma
\end{equation}

Where
\begin{equation}\label{eqn:rotor:rotcoords}
{
\begin{bmatrix}
R_\theta
\end{bmatrix}
}_\sigma^\sigma
=
\begin{bmatrix}
{
\begin{bmatrix}
R_\theta(\Be_1)
\end{bmatrix}
}_\sigma
\hdots
{
\begin{bmatrix}
R_\theta(\Be_n)
\end{bmatrix}
}_\sigma
\end{bmatrix}
=
{
\begin{bmatrix}
\left(e^{-\Bv\Bu \theta/2} \Be_j e^{\Bu\Bv \theta/2}\right) \cdot \Be_i \\
\end{bmatrix}
}_{ij}
\end{equation}

If one uses the plane and its normal to form an alternate orthonormal basis
\(\alpha = \{\Bu, \Bv, \Bn\}\).

The transformation matrix for coordinate vectors in this basis is

\begin{equation}\label{eqn:rotor:580}
{
\begin{bmatrix}
R_\theta
\end{bmatrix}
}_\alpha^\alpha
=
\begin{bmatrix}
\left(\Bu e^{\Bu\Bv \theta}\right) \cdot \Bu & \left(\Bv e^{\Bu\Bv \theta}\right) \cdot \Bu & 0 \\
\left(\Bu e^{\Bu\Bv \theta}\right) \cdot \Bv & \left(\Bv e^{\Bu\Bv \theta}\right) \cdot \Bv & 0 \\
0 & 0 & \Bn\cdot\Bn \\
\end{bmatrix}
=
\begin{bmatrix}
\cos\theta & -\sin\theta & 0 \\
\sin\theta & \cos\theta & 0 \\
0 & 0 & 1 \\
\end{bmatrix}
\end{equation}

This matrix has eigenvalues \(e^{i\theta}, e^{-i\theta}, 1\), with (coordinate) eigenvectors

\begin{equation}\label{eqn:rotor:600}
\inv{\sqrt{2}}
\begin{bmatrix}
1 \\
-i \\
0 \\
\end{bmatrix},
\inv{\sqrt{2}}
\begin{bmatrix}
1 \\
i \\
0 \\
\end{bmatrix},
\begin{bmatrix}
0 \\
0 \\
1 \\
\end{bmatrix}
\end{equation}

Its interesting to observe that without introducing coordinate vectors an eigensolution is possible directly from
the linear transformation itself.

The rotation linear operator has right and left eigenvalues \(e^{\Bu\Bv \theta}\), \(e^{\Bv\Bu \theta}\) (respectively), where the eigenvectors for these are any vectors in the plane.  There is also a scalar eigenvalue \(1\) (both left and right eigenvalue), for the eigenvector \(\Bn\):

\begin{equation}\label{eqn:rotor:1040}
\begin{aligned}
R_\theta(\Bu) &= e^{\Bv \Bu \theta} \Bx = \Bx e^{\Bu \Bv \theta} \\
R_\theta(\Bu) &= e^{\Bv \Bu \theta} \Bx = \Bx e^{\Bu \Bv \theta} \\
R_\theta(\Bn) &= \Bn (1) \\
\end{aligned}
\end{equation}

Observe that the eigenvalues here are not all scalars, which is likely related
to the fact that the coordinate matrix was not diagonalizable with real vectors.

the matrix of the linear transformation.
Given this, one can write:

\begin{equation}\label{eqn:rotor:1060}
\begin{aligned}
\begin{bmatrix}
R_\theta(\Bu) & R_\theta(\Bv) & R_\theta(\Bn) \\
\end{bmatrix}
&=
\begin{bmatrix}
\Bu & \Bv & \Bn \\
\end{bmatrix}
\begin{bmatrix}
e^{\Bu \Bv \theta} & 0 & 0 \\
0 & e^{\Bu \Bv \theta} & 0 \\
0 & 0 & 1 \\
\end{bmatrix} \\
&=
\begin{bmatrix}
e^{\Bv \Bu \theta} & 0 & 0 \\
0 & e^{\Bv \Bu \theta} & 0 \\
0 & 0 & 1 \\
\end{bmatrix}
\begin{bmatrix}
\Bu & \Bv & \Bn \\
\end{bmatrix}
\end{aligned}
\end{equation}

But neither of these can be used to diagonalize the matrix of the transformation.  To do that
we require dot products that span the matrix product to form the coordinate vector columns.

Observe that interestingly
enough the left and right eigenvalues of the operator in the plane are of complex exponential form (\(e^{\pm \Bn I \theta}\)) just as the eigenvalues for
coordinate vectors restricted to the plane are complex exponentials (\(e^{\pm i\theta}\)).
%This suggests that a basis for a quaternion
%like space (0-2 multivectors) will be required to diagonalize a rotation operator.

\section{matrix for rotation linear transformation}

Let us expand the terms in \eqnref{eqn:rotor:rotcoords} to calculate explicitly the rotation matrix for an arbitrary
rotation.  Also, as before, write \(\Bn = \Bv\Bu I\), and parametrize the Rotor as follows:

\begin{equation}\label{eqn:rotor:620}
R = e^{\Bn I \theta/2} = \cos{\theta/2} + \Bn I \sin{\theta/2} = \alpha + I\Bbeta
\end{equation}

Thus the \(ij\) terms in the matrix are:

\begin{equation}\label{eqn:rotor:1080}
\begin{aligned}
\Be_i \cdot \left(e^{-\Bn I \theta/2} \Be_j e^{\Bn I \theta/2}\right)
&= \langle{ \Be_i (\alpha -I\Bbeta) \Be_j (\alpha +I\Bbeta) } \rangle \\
&= \langle{ \Be_i (\Be_j \alpha -I\Bbeta\Be_j) (\alpha +I\Bbeta) } \rangle \\
&= \langle{ \Be_i \left( \Be_j \alpha^2 -I\alpha(\Bbeta\Be_j - \Be_j\Bbeta) + \Bbeta\Be_j\Bbeta \right) } \rangle \\
&= \delta_{ij}\alpha^2 + \langle{ \Be_i \left( -2I\alpha(\Bbeta \wedge \Be_j) + \Bbeta\Be_j\Bbeta \right) } \rangle \\
&= \delta_{ij}\alpha^2 + 2\alpha \Be_i \cdot (\Bbeta \cross \Be_j) + \langle{ \Be_i \Bbeta \Be_j \Bbeta } \rangle \\
\end{aligned}
\end{equation}

Lets expand the last term separately:
\begin{equation}\label{eqn:rotor:1100}
\begin{aligned}
\langle{ \Be_i \Bbeta \Be_j \Bbeta } \rangle
&= \langle{ ( \Be_i \cdot \Bbeta + \Be_i \wedge \Bbeta) ( \Be_j \cdot \Bbeta + \Be_j \wedge \Bbeta) } \rangle  \\
&= (\Be_i \cdot \Bbeta)(\Be_j \cdot \Bbeta) + \langle{ (\Be_i \wedge \Bbeta) ( \Be_j \wedge \Bbeta) } \rangle  \\
\end{aligned}
\end{equation}

And once more considering first the \(i=j\) case (writing \(s \ne i \ne t\)).

\begin{equation}\label{eqn:rotor:1120}
\begin{aligned}
\langle{ (\Be_i \wedge \Bbeta)^2 }\rangle
&= \lr{ \sum_{k \ne i}{ \Be_{ik} \beta_k} }^2 \\
&= ( \Be_{is} \beta_s + \Be_{it} \beta_t ) ( \Be_{is} \beta_s + \Be_{it} \beta_t ) \\
&= -\beta_s^2 -\beta_t^2 -  \Be_{st} \beta_s \beta_t + \Be_{ts} \beta_t \beta_s  \\
&= -\beta_s^2 -\beta_t^2 \\
&= -\Bbeta^2 + \beta_i^2 \\
\end{aligned}
\end{equation}

For the \(i \ne j\) term, writing \(i \ne j \ne k\)
\begin{equation}\label{eqn:rotor:1140}
\begin{aligned}
\langle{(\Be_i \wedge \Bbeta) (\Be_j \wedge \Bbeta)}\rangle
&= \langle{\sum_{s \ne i} \Be_{is} \beta_s\sum_{t \ne i} \Be_{it} \beta_t}\rangle \\
&= \langle{( \Be_{ij} \beta_j + \Be_{ik} \beta_k) ( \Be_{ji} \beta_i + \Be_{jk} \beta_k)}\rangle \\
&= \beta_i\beta_j + \langle{ \Be_{ji} \beta_k^2 +\Be_{ik} \beta_j \beta_k +\Be_{kj} \beta_k \beta_i }\rangle \\
&= \beta_i\beta_j \\
\end{aligned}
\end{equation}

Thus
\begin{equation}\label{eqn:rotor:640}
\langle{ (\Be_i \wedge \Bbeta) ( \Be_j \wedge \Bbeta) } \rangle
= \delta_{ij}(-\Bbeta^2 + \beta_i^2) + (1-\delta_{ij})\beta_i\beta_j
= \beta_i\beta_j -\delta_{ij}\Bbeta^2
\end{equation}

And putting it all back together
\begin{equation}\label{eqn:rotor:rotmgreek}
\Be_i \cdot \left(e^{-\Bn I \theta/2} \Be_j e^{\Bn I \theta/2}\right)
= \delta_{ij}(\alpha^2 -\Bbeta^2) + 2\alpha \Be_i \cdot (\Bbeta \cross \Be_j) + 2\beta_i\beta_j
\end{equation}


The \(\alpha\) and \(\beta\) terms can be expanded in terms of \(\theta\).
we see that The \(\delta_{ij}\) coefficient is

\begin{equation}\label{eqn:rotor:660}
\alpha^2 -\Bbeta^2 = 2{\cos}^2{\theta} -1 = \cos\theta.
\end{equation}

The triple product \(\Be_i \cdot (\Bbeta \cross \Be_j)\) is zero along the diagonal where \(i=j\) since an \(\Be_j=\Be_i\) cross has no \(\Be_i\) component, so
for \(k \ne i \ne j\), the triple product term is

\begin{equation}\label{eqn:rotor:1160}
\begin{aligned}
2\alpha \Be_i \cdot (\Bbeta \cross \Be_j)
&= 2\alpha \beta_k \Be_i \cdot (\Be_k \cross \Be_j) \\
&= 2\alpha \beta_k \Sgn{\pi_{ikj}} \\
&= 2 n_k \cos({\theta/2})\sin({\theta/2}) \Sgn{\pi_{ikj}} \\
&= n_k \sin{\theta} \Sgn{\pi_{ikj}} \\
\end{aligned}
\end{equation}

The last term is:
\begin{equation}\label{eqn:rotor:680}
2\beta_i\beta_j
= 2 n_i n_j {\sin}^2({\theta/2})
= n_i n_j (1-\cos\theta)
\end{equation}

Thus we can alternatively write \eqnref{eqn:rotor:rotmgreek}

\begin{equation}\label{eqn:rotor:rotmn}
\Be_i \cdot \left(e^{-\Bn I \theta/2} \Be_j e^{\Bn I \theta/2}\right)
= \delta_{ij}\cos\theta
+ n_k \sin{\theta} \epsilon_{ikj} + n_i n_j (1-\cos\theta)
\end{equation}

This is enough to easily and explicitly write out the complete rotation matrix for a rotation about unit vector \(\Bn = (n_1, n_2, n_3)\):
(with basis \(\sigma = \{\Be_i\}\)):

\begin{equation}\label{eqn:rotor:700}
[
R_\theta
]_\sigma^\sigma
=
\begin{bmatrix}
\cos\theta(1 -n_1^2) + n_1^2 & n_1 n_2 (1-\cos\theta) - n_3 \sin\theta & n_1 n_3 (1-\cos\theta) + n_2 \sin\theta \\
n_1 n_2 (1-\cos\theta) + n_3 \sin\theta & \cos\theta(1 -n_2^2) + n_2^2 & n_2 n_3 (1-\cos\theta) - n_1 \sin\theta \\
n_1 n_3 (1-\cos\theta) - n_2 \sin\theta & n_2 n_3 (1-\cos\theta) + n_1 \sin\theta & \cos\theta(1 -n_3^2) + n_3^2 \\
\end{bmatrix}
\end{equation}

Note also that the \(n_i\) terms are the direction cosines of the unit normal for the rotation, so all the terms above
are really strictly sums of sine and cosine products, so we have the rotation matrix completely described in terms of four
angles.  Also observe how much additional complexity we have to express a rotation in terms of the matrix.  This representation also
does not work for plane rotations, just vectors (whereas that is not the case for the rotor form).

It is actually somewhat simpler looking to leave things in terms of the \(\alpha\), and \(\beta\) parameters.  We can rewrite
\eqnref{eqn:rotor:rotmgreek} as:

\begin{equation}
\Be_i \cdot \left(e^{-\Bn I \theta/2} \Be_j e^{\Bn I \theta/2}\right)
= \delta_{ij}(2\alpha^2 -1)
+2\alpha \beta_k \epsilon_{ikj} + 2\beta_i\beta_j
\end{equation}

and the rotation matrix:

\begin{equation}\label{eqn:rotor:720}
[
R_\theta
]_\sigma^\sigma
=
2
\begin{bmatrix}
\alpha^2 -\frac{1}{2} + \beta_1^2 & \beta_1 \beta_2  - \beta_3 \alpha & \beta_1 \beta_3  + \beta_2 \alpha \\
\beta_1 \beta_2  + \beta_3 \alpha & \alpha^2 -\frac{1}{2} + \beta_2^2 & \beta_2 \beta_3  - \beta_1 \alpha \\
\beta_1 \beta_3  - \beta_2 \alpha & \beta_2 \beta_3  + \beta_1 \alpha & \alpha^2 -\frac{1}{2} + \beta_3^2 \\
\end{bmatrix}
\end{equation}

Not really that much simpler, but a bit.  The trade off is that the similarity to the standard \(2x2\) rotation matrix is not obvious.


   \chapter{Euler Angle Notes}\label{chap:PJEulerAngle}
      %
% Copyright � 2012 Peeter Joot.  All Rights Reserved.
% Licenced as described in the file LICENSE under the root directory of this GIT repository.
%

%
%
%\chapter{Euler Angle Notes}\label{chap:PJEulerAngle}
\index{Euler angle}
%\date{November 1, 2008.  eulerangle.tex}

\section{Removing the rotors from the exponentials}

In \citep{doran2003gap} section 2.7.5 the Euler angle formula is
developed for \(\{z,x',z''\}\) axis rotations by \(\{\phi, \theta, \psi\}\)
respectively.

Other than a few details the derivation is pretty straightforward.  Equation
2.153 would be clearer with a series expansion hint like

\begin{equation}\label{eqn:eulerangle:20}
\begin{aligned}
\exp(R \alpha i R^\dagger)
&= \sum_k \inv{k!} (R \alpha i R^\dagger)^k \\
&= \sum_k \inv{k!} R (\alpha i)^k R^\dagger \\
&= R \exp(\alpha i) R^\dagger
\end{aligned}
\end{equation}

where \(i\) is a bivector, and \(R\) is a rotor (\(RR^\dagger = 1\)).

The first rotation is straightforward, by an angle \(\phi\) around the z axis

\begin{equation}\label{eqn:eulerangle:40}
\begin{aligned}
R_\phi(x) = \exp(-Ie_3 \phi/2) x \exp(Ie_3 \phi/2) = R_\phi x R_\phi^\dagger
\end{aligned}
\end{equation}

The next rotation is around the transformed x axis, for which the rotational plane is
\begin{equation}\label{eqn:eulerangle:60}
\begin{aligned}
I R_\phi e_1 R_\phi^\dagger
&= R_\phi I e_1 R_\phi^\dagger  \\
\end{aligned}
\end{equation}

So the rotor for this plane by angle \(\theta\) is

\begin{equation}\label{eqn:eulerangle:80}
\begin{aligned}
R_\theta
&= \exp( R_\phi I e_1 R_\phi^\dagger ) \\
&= R_\phi \exp( I e_1 \theta/2) R_\phi^\dagger,
\end{aligned}
\end{equation}

resulting in the composite rotor
\begin{equation}\label{eqn:eulerangle:100}
\begin{aligned}
R_{\theta\phi}
&=
R_\theta
R_\phi \\
&= R_\phi \exp( I e_1 \theta/2) R_\phi^\dagger R_\phi \\
&= R_\phi \exp( I e_1 \theta/2) \\
&=
\exp{( -I e_3 \phi/2 )}
\exp( -I e_1 \theta/2 ) \\
\end{aligned}
\end{equation}

The composition of the simple rotation around the \(e_3\) axis, followed by the rotation around the \(e_1'\) axis ends up
as a product of rotors around the original \(e_1\) and \(e_3\) axis, but curiously enough in inverted order.

\section{Expanding the rotor product}

Completing the calculation outlined above follows in the same fashion.  The
end result is that the composite Euler rotations has the following rotor form

\begin{equation}\label{eqn:eulerangle:120}
\begin{aligned}
R(x) &= R x R^\dagger \\
R &= \exp(-e_{12}\phi/2) \exp(-e_{23}\theta/2) \exp(-e_{12}\psi/2)
\end{aligned}
\end{equation}

Then there are notes saying this is easier to visualize and work with than
the equivalent matrix formula.  Let us see what the equivalent matrix formula
is.  First calculate the rotor action on \(\Be_1\)

\begin{equation}\label{eqn:eulerangle:140}
\begin{aligned}
R \Be_1 R^\dagger
&=
e^{-e_{12}\phi/2} e^{-e_{23}\theta/2} e^{-e_{12}\psi/2}
\Be_1
e^{e_{12}\psi/2}
e^{e_{23}\theta/2}
e^{e_{12}\phi/2} \\
&=
e^{-e_{12}\phi/2} e^{-e_{23}\theta/2}
\Be_1 e^{e_{12}\psi}
e^{e_{23}\theta/2}
e^{e_{12}\phi/2} \\
&=
e^{-e_{12}\phi/2} e^{-e_{23}\theta/2}
(\Be_1 C_\psi + \Be_2 S_\psi)
e^{e_{23}\theta/2}
e^{e_{12}\phi/2} \\
&=
e^{-e_{12}\phi/2}
(\Be_1 C_\psi + \Be_2 S_\psi e^{e_{23}\theta} )
e^{e_{12}\phi/2} \\
&=
e^{-e_{12}\phi/2}
(\Be_1 C_\psi + \Be_2 S_\psi C_\theta + \Be_3 S_\psi S_\theta )
e^{e_{12}\phi/2} \\
&=
(\Be_1 C_\psi + \Be_2 S_\psi C_\theta ) e^{e_{12}\phi}
+ \Be_3 S_\psi S_\theta  \\
&=
  \Be_1 (C_\psi C_\phi - S_\psi C_\theta S_\phi)
+ \Be_2 (C_\psi S_\phi + S_\psi C_\theta C_\phi)
+ \Be_3 S_\psi S_\theta  \\
\end{aligned}
\end{equation}

Now \(\Be_2\):

\begin{equation}\label{eqn:eulerangle:160}
\begin{aligned}
R \Be_2 R^\dagger
&=
e^{-e_{12}\phi/2} e^{-e_{23}\theta/2} e^{-e_{12}\psi/2}
\Be_2
e^{e_{12}\psi/2}
e^{e_{23}\theta/2}
e^{e_{12}\phi/2} \\
&=
e^{-e_{12}\phi/2} e^{-e_{23}\theta/2}
(\Be_2 C_\psi - \Be_1 S_\psi)
e^{e_{23}\theta/2}
e^{e_{12}\phi/2} \\
&=
(\Be_2 C_\psi C_\theta
- \Be_1 S_\psi)
e^{e_{12}\phi}
+\Be_3 C_\psi S_\theta \\
&=
\Be_1 (-C_\psi C_\theta S_\phi - S_\psi C_\phi)
+ \Be_2 (-S_\psi S_\phi + C_\psi C_\theta C_\phi)
+ \Be_3 C_\psi S_\theta \\
\end{aligned}
\end{equation}

And finally \(\Be_3\)

\begin{equation}\label{eqn:eulerangle:180}
\begin{aligned}
R \Be_3 R^\dagger
&=
e^{-e_{12}\phi/2} e^{-e_{23}\theta/2} e^{-e_{12}\psi/2}
\Be_3
e^{e_{12}\psi/2}
e^{e_{23}\theta/2}
e^{e_{12}\phi/2} \\
&=
e^{-e_{12}\phi/2}
\Be_3 e^{e_{23}\theta}
e^{e_{12}\phi/2} \\
&=
e^{-e_{12}\phi/2}
(
\Be_3 C_\theta
-\Be_2 S_\theta
)
e^{e_{12}\phi/2} \\
&= -\Be_2 S_\theta e^{e_{12}\phi} +\Be_3 C_\theta \\
&=
 \Be_1 S_\theta S_\phi
-\Be_2 S_\theta C_\phi
+\Be_3 C_\theta \\
\end{aligned}
\end{equation}

This can now be assembled into matrix form
\begin{equation}\label{eqn:eulerangle:200}
\begin{aligned}
\begin{bmatrix}
R(x) \cdot \Be_1 \\
R(x) \cdot \Be_2 \\
R(x) \cdot \Be_3
\end{bmatrix}
=
\begin{bmatrix}
(R \Be_1 R^\dagger) \cdot \Be_1 & (R \Be_2 R^\dagger) \cdot \Be_1 & (R \Be_3 R^\dagger) \cdot \Be_1 \\
(R \Be_1 R^\dagger) \cdot \Be_2 & (R \Be_2 R^\dagger) \cdot \Be_2 & (R \Be_3 R^\dagger) \cdot \Be_2 \\
(R \Be_1 R^\dagger) \cdot \Be_3 & (R \Be_2 R^\dagger) \cdot \Be_3 & (R \Be_3 R^\dagger) \cdot \Be_3 \\
\end{bmatrix}
\begin{bmatrix}
x^1 \\
x^2 \\
x^3 \\
\end{bmatrix} = \BR \Bx
\end{aligned}
\end{equation}

Therefore we have the composite matrix form for the Euler angle rotations
\begin{equation}\label{eqn:eulerangle:220}
\begin{aligned}
\BR &=
\begin{bmatrix}
C_\psi C_\phi - S_\psi C_\theta S_\phi & -S_\psi C_\phi - C_\psi C_\theta S_\phi & S_\theta S_\phi \\
C_\psi S_\phi + S_\psi C_\theta C_\phi & -S_\psi S_\phi + C_\psi C_\theta C_\phi & -S_\theta C_\phi \\
S_\psi S_\theta                        &                  C_\psi S_\theta        & C_\theta \\
\end{bmatrix}
\end{aligned}
\end{equation}

Lots of opportunity to make sign errors here.  Let us check with matrix multiplication, which should give the same result

\section{With composition of rotation matrices (done wrong, but with discussion and required correction)}

\begin{equation}\label{eqn:eulerangle:wrong}
\begin{aligned}
R(x) &= \BR \Bx \\
&=
\BR_{\psi \Be_3}
\BR_{\theta \Be_1}
\BR_{\phi \Be_3} \Bx
\end{aligned}
\end{equation}

Now, that first rotation is

\begin{equation}\label{eqn:eulerangle:240}
\begin{aligned}
R_\phi(x)
&= e^{-e_{12}\phi/2} (\Be_i x^i) e^{e_{12}\phi/2} \\
&= ( \Be_1 x^1 +\Be_2 x^2) e^{e_{12}\phi} + \Be_3 x^3 \\
&=
 x^1 ( \Be_1 \cos\phi +\Be_2 \sin\phi )
+x^2 ( \Be_2 \cos\phi -\Be_1 \sin\phi )
+x^3 \Be_3 \\
&=
 \Be_1 ( x^1 \cos\phi - x^2 \sin\phi )
+\Be_2 ( x^1 \sin\phi + x^2 \cos\phi )
+\Be_3 x^3 \\
\end{aligned}
\end{equation}

Which has the expected matrix form

\begin{equation}\label{eqn:eulerangle:260}
\begin{aligned}
\BR_{\phi \Be_3} \Bx =
\begin{bmatrix}
\cos\phi & - \sin\phi & 0 \\
\sin\phi & \cos\phi & 0 \\
0 & 0 & 1 \\
\end{bmatrix}
\begin{bmatrix}
x^1 \\
x^2 \\
x^3 \\
\end{bmatrix}
\end{aligned}
\end{equation}

Using \(C_x = \cos(x)\), and \(S_x = \sin(x)\) for brevity, the composite rotation is
\begin{equation}\label{eqn:eulerangle:280}
\begin{aligned}
\BR_{\psi \Be_3} \BR_{\theta \Be_1} \BR_{\phi \Be_3}
&=
\begin{bmatrix}
C_\psi & - S_\psi & 0 \\
S_\psi & C_\psi & 0 \\
0 & 0 & 1 \\
\end{bmatrix}
\begin{bmatrix}
1 & 0 & 0 \\
0 & C_\theta & - S_\theta \\
0 & S_\theta & C_\theta \\
\end{bmatrix}
\begin{bmatrix}
C_\phi & - S_\phi & 0 \\
S_\phi & C_\phi & 0 \\
0 & 0 & 1 \\
\end{bmatrix} \\
&=
\begin{bmatrix}
C_\psi & - S_\psi C_\theta & S_\psi S_\theta \\
S_\psi & C_\psi C_\theta & -C_\psi S_\theta \\
0 & S_\theta & C_\theta \\
\end{bmatrix}
\begin{bmatrix}
C_\phi & - S_\phi & 0 \\
S_\phi & C_\phi & 0 \\
0 & 0 & 1 \\
\end{bmatrix} \\
&=
\begin{bmatrix}
C_\psi C_\phi - S_\psi C_\theta S_\phi & -C_\psi S_\phi - S_\psi C_\theta C_\phi &   S_\psi S_\theta \\
S_\psi C_\phi + C_\psi C_\theta S_\phi & -S_\psi S_\phi + C_\psi C_\theta C_\phi & - C_\psi S_\theta \\
                       S_\theta S_\phi &                         S_\theta C_\phi &          C_\theta \\
\end{bmatrix} \\
\end{aligned}
\end{equation}

This is different from the rotor generated result above, although with a \(\phi\), and \(\psi\) interchange things appear to match?

\begin{equation}\label{eqn:eulerangle:300}
\begin{aligned}
\begin{bmatrix}
C_\phi C_\psi - S_\phi C_\theta S_\psi & -S_\phi C_\psi - C_\phi C_\theta S_\psi & S_\theta S_\psi \\
C_\phi S_\psi + S_\phi C_\theta C_\psi & -S_\phi S_\psi + C_\phi C_\theta C_\psi & -S_\theta C_\psi \\
S_\phi S_\theta                        &                  C_\phi S_\theta        & C_\theta \\
\end{bmatrix}
\end{aligned}
\end{equation}

Where is the mistake?  I suspect it is in the matrix formulation, where the plain old rotations for the axis were multiplied.  Because the rotations need to be along the transformed
axis I bet there is a reversion of matrix products as there was an reversion of rotors?  How would one show if this is the case?

What is needed is more careful treatment of the rotation about the transformed axis.  Considering the first, for a rotation
around the \(\Be_1' = (C_\phi, S_\phi, 0)\) axis.  From \chapcite{rotor} we have the rotation matrix for a \(\theta\) rotation about
an arbitrary normal \(\Bn = (n_1, n_2, n_3)\)

\begin{equation}\label{eqn:eulerangle:320}
\begin{aligned}
R_\theta
&=
\begin{bmatrix}
C_\theta(1 -n_1^2) + n_1^2 & n_1 n_2 (1-C_\theta) - n_3 S_\theta & n_1 n_3 (1-C_\theta) + n_2 S_\theta \\
n_1 n_2 (1-C_\theta) + n_3 S_\theta & C_\theta(1 -n_2^2) + n_2^2 & n_2 n_3 (1-C_\theta) - n_1 S_\theta \\
n_1 n_3 (1-C_\theta) - n_2 S_\theta & n_2 n_3 (1-C_\theta) + n_1 S_\theta & C_\theta(1 -n_3^2) + n_3^2 \\
\end{bmatrix} \\
&=
\begin{bmatrix}
C_\theta(1 -{C_\phi}^2) + {C_\phi}^2 & {C_\phi} {S_\phi} (1-C_\theta) & {S_\phi} S_\theta \\
{C_\phi} {S_\phi} (1-C_\theta) & C_\theta(1 -{S_\phi}^2) + {S_\phi}^2 & - {C_\phi} S_\theta \\
- {S_\phi} S_\theta & {C_\phi} S_\theta & C_\theta \\
\end{bmatrix} \\
&=
\begin{bmatrix}
C_\theta({S_\phi}^2) + {C_\phi}^2 & {C_\phi} {S_\phi} (1-C_\theta) & {S_\phi} S_\theta \\
{C_\phi} {S_\phi} (1-C_\theta) & C_\theta({C_\phi}^2) + {S_\phi}^2 & - {C_\phi} S_\theta \\
- {S_\phi} S_\theta & {C_\phi} S_\theta & C_\theta \\
\end{bmatrix} \\
\end{aligned}
\end{equation}

Now the composite rotation for the sequence of \(\phi\), and \(\theta\) rotations about the \(z\), and \(x'\) axis is

\begin{equation}\label{eqn:eulerangle:340}
\begin{aligned}
\BR_{\phi \Be_3, \theta \Be_1'}
&=
\begin{bmatrix}
C_\theta({S_\phi}^2) + {C_\phi}^2 & {C_\phi} {S_\phi} (1-C_\theta) & {S_\phi} S_\theta \\
{C_\phi} {S_\phi} (1-C_\theta) & C_\theta({C_\phi}^2) + {S_\phi}^2 & - {C_\phi} S_\theta \\
- {S_\phi} S_\theta & {C_\phi} S_\theta & C_\theta \\
\end{bmatrix}
\begin{bmatrix}
C_\phi & - S_\phi & 0 \\
S_\phi & C_\phi & 0 \\
0 & 0 & 1 \\
\end{bmatrix} \\
&=
\begin{bmatrix}
C_\phi & -C_\theta S_\phi & {S_\phi} S_\theta \\
S_\phi & C_\theta C_\phi & - {C_\phi} S_\theta \\
0 & S_\theta & C_\theta \\
\end{bmatrix} \\
&=
\begin{bmatrix}
C_\phi & - S_\phi & 0 \\
S_\phi & C_\phi & 0 \\
0 & 0 & 1 \\
\end{bmatrix}
\begin{bmatrix}
1 & 0 & 0 \\
0 & C_\theta & - S_\theta \\
0 & S_\theta & C_\theta \\
\end{bmatrix} \\
&=
\BR_{\phi \Be_3}
\BR_{\theta \Be_1} \\
\end{aligned}
\end{equation}

Wow, sure enough the composite rotation matrix is the result of the inverted order product of the two elementary rotation matrices.  The
algebra here is fairly messy, so it would not be terribly fun to go one step further using just matrices that the final triple rotation
is not the product of \eqnref{eqn:eulerangle:wrong}, but instead
requires
the matrix product

\begin{equation}\label{eqn:eulerangle:fixed}
\begin{aligned}
R(x) &= \BR \Bx \\
&=
\BR_{\phi \Be_3}
\BR_{\theta \Be_1}
\BR_{\psi \Be_3}
\Bx
\end{aligned}
\end{equation}

Assuming that this is true, the swapping of angles to match the rotor expression is fully accounted for, and it is understood how to
correctly do the same calculation in matrix form.

\section{Relation to Cayley-Klein parameters}

Exercise 2.9 from \citep{doran2003gap} is to relate the rotation matrix expressed in terms of Cayley-Klein parameters back to the rotor
formulation.  That matrix has the look of something that involves the half angles.  Use of software to expressing the rotation in terms of the
half angle signs and cosines, plus some manually factoring (which could be carried
further), produces the following mess

\begin{equation}\label{eqn:eulerangle:360}
\begin{aligned}
R_{11} &= S_\phi^2 S_\psi^2 - C_\psi^2 S_\phi^2 - C_\phi^2 S_\psi^2 + C_\phi^2 C_\psi^2 - 4 C_\phi S_\phi C_\psi S_\psi (C_\theta^2 - S_\theta^2) \\
R_{21} &= 2 C_\psi S_\psi (C_\phi^2 -S_\phi^2 )(C_\theta^2 -S_\theta^2)
+2 C_\phi S_\phi (C_\psi^2 -S_\psi^2) \\
R_{31} &= 4 C_\psi C_\theta S_\psi S_\theta  \\
R_{12} &= -2 C_\phi S_\phi ( C_\psi^2 -S_\psi^2 )(C_\theta^2 - S_\theta^2) - 2 C_\psi S_\psi ( C_\phi^2 -S_\phi^2 ) \\
R_{22} &= (C_\theta^2 -S_\theta^2)(-S_\phi^2 + C_\phi^2)(C_\psi^2 -S_\psi^2) -4 C_\phi S_\phi C_\psi S_\psi  \\
R_{32} &= 2 C_\theta S_\theta ( C_\psi^2 - S_\psi^2 ) \\
R_{13} &= 4 C_\phi C_\theta S_\phi S_\theta  \\
R_{23} &= -2 C_\theta S_\theta ( C_\phi^2 - S_\phi^2 ) \\
R_{33} &= + C_\theta^2 S_\phi^2 - S_\psi^2 S_\theta^2 - C_\psi^2 S_\theta^2 + C_\phi^2 C_\theta^2
\end{aligned}
\end{equation}

It kind of looks like the terms \(C_\theta S_\phi\) may be related to these parameters.
error prone.  A Google search for Cayley Klein also verifies that those parameters are expressed in terms of half angle relations, but even
with the hint, I was not successful getting something tidy out of all this.

An alternate approach is to just expand the rotor, so terms may be grouped before that rotor and its reverse is applied to the object to be
rotated.  Again in terms of half angle signs and cosines this is

\begin{equation}\label{eqn:eulerangle:380}
\begin{aligned}
R &= \exp(-e_{12}\phi/2) \exp(-e_{23}\theta/2) \exp(-e_{12}\psi/2) \\
&=
( C_\phi C_\psi - S_\phi S_\psi ) C_\theta
- ( C_\psi S_\phi + C_\phi S_\psi ) C_\theta \Be_{12} \\
&+ ( C_\phi S_\psi - C_\psi S_\phi ) S_\theta \Be_{31}
- ( S_\phi S_\psi + C_\phi C_\psi ) S_\theta \Be_{23}  \\
\end{aligned}
\end{equation}

Grouping terms produces
\begin{equation}\label{eqn:eulerangle:cayleyhalf}
\begin{aligned}
R
=
\cos\left(\frac{\theta}{2}\right)
\exp\left(
\frac{- \Be_{12} }{2}\left(\psi+\phi\right)
\right)
+
\sin\left(\frac{\theta}{2}\right)
\exp\left(
\frac{\Be_{12} }{2}\left(\psi-\phi\right)
\right)
\Be_{32}
\end{aligned}
\end{equation}

Okay... now I see how you naturally get four parameters out of this.  Also see why it was hard to get there
from the fully expanded rotation product ... it would first be required to group all the \(\phi\) and \(\psi\) terms just right
in terms of sums and differences.

With
\begin{equation}\label{eqn:eulerangle:400}
\begin{aligned}
R &= \alpha + \delta\Be_{12} + \beta \Be_{23} + \gamma \Be_{31} \\
\alpha &= \cos\left(\frac{\theta}{2}\right) \cos\left( \inv{2}\left(\psi+\phi\right) \right) \\
\delta &= -\cos\left(\frac{\theta}{2}\right) \sin\left( \inv{2}\left(\psi+\phi\right) \right) \\
\beta &= -\sin\left(\frac{\theta}{2}\right) \cos\left( \inv{2}\left(\psi-\phi\right) \right) \\
\gamma &= \sin\left(\frac{\theta}{2}\right) \sin\left( \inv{2}\left(\psi-\phi\right) \right) \\
\end{aligned}
\end{equation}


By inspection, these have the required property \(\alpha^2 + \beta^2 + \gamma^2 + \delta^2 = 1\), and
multiplying out the rotors yields the rotation matrix

\begin{equation}\label{eqn:eulerangle:420}
\begin{aligned}
\BU =
\begin{bmatrix}
 - \gamma^2 + \beta^2 - \delta^2 + \alpha^2 & + 2 \beta \gamma + 2 \alpha \delta & + 2 \delta \beta - 2 \alpha \gamma \\
+ 2 \beta \gamma - 2 \alpha \delta & + \gamma^2 - \beta^2 - \delta^2 + \alpha^2 & + 2 \delta \gamma + 2 \alpha \beta \\
+ 2 \delta \beta + 2 \alpha \gamma & + 2 \delta \gamma - 2 \alpha \beta & - \gamma^2 - \beta^2 + \delta^2 + \alpha^2 \\
\end{bmatrix}
\end{aligned}
\end{equation}

Now that particular choice of sign and permutation of the \(\alpha\), \(\beta\), \(\gamma\), and \(\delta\) is not at all obvious, and is also arbitrary.  Of the
120 different sign and permutation variations that can be tried, this one results in the
particular desired matrix \(\BU\) from the problem.  The process of performing
the multiplications was well suited to a symbolic GA calculator and one
was written with this and other problems in mind.

\citep{goldstein1951cm} also treats these Cayley-Klein parametrizations, but
does so considerably differently.
He has complex parametrizations and quaternion matrix representations, and
it will probably be worthwhile to reconcile all of these.

\section{Omitted details}
\subsection{Cayley Klein details}
\index{Cayley Klein angle}

\Eqnref{eqn:eulerangle:cayleyhalf} was obtained with the following manipulations

\begin{equation}\label{eqn:eulerangle:440}
\begin{aligned}
R
&=
( C_\phi C_\psi - S_\phi S_\psi ) C_\theta
- ( C_\psi S_\phi + C_\phi S_\psi ) C_\theta \Be_{12} \\
&+ ( C_\phi S_\psi - C_\psi S_\phi ) S_\theta \Be_{31}
- ( S_\phi S_\psi + C_\phi C_\psi ) S_\theta \Be_{23}  \\
&=
\cos\left(\inv{2}(\phi+\psi)\right) C_\theta
-\sin\left(\inv{2}(\phi + \psi)\right) C_\theta \Be_{12} \\
&- \sin\left(\inv{2}(\phi - \psi)\right) S_\theta \Be_{31}
- \cos\left(\inv{2}(\phi - \psi)\right) S_\theta \Be_{23} \\
&=
\cos\left(\inv{2}(\psi+\phi)\right) C_\theta
-\sin\left(\inv{2}(\psi + \phi)\right) C_\theta \Be_{12} \\
&+ \sin\left(\inv{2}(\psi - \phi)\right) S_\theta \Be_{31}
- \cos\left(\inv{2}(\psi - \phi)\right) S_\theta \Be_{23} \\
&=
C_\theta \left(
\cos\left(\inv{2}(\psi+\phi)\right)
-\sin\left(\inv{2}(\psi + \phi)\right) \Be_{12}
\right) \\
&+
S_\theta \left(
\sin\left(\inv{2}(\psi - \phi)\right)
\Be_{31}
- \cos\left(\inv{2}(\psi - \phi)\right) \Be_{23} \right) \\
&=
C_\theta
\exp\left(
\frac{- \Be_{12} }{2}\left(\psi+\phi\right)
\right)
\\
&+
S_\theta \Be_{3}
\left(
\sin\left(\inv{2}(\psi - \phi)\right)
\Be_{1}
+ \cos\left(\inv{2}(\psi - \phi)\right) \Be_{2} \right) \\
&=
C_\theta
\exp\left(
\frac{- \Be_{12} }{2}\left(\psi+\phi\right)
\right)
+
S_\theta
\exp\left(
\frac{\Be_{12} }{2}\left(\psi-\phi\right)
\right)
\Be_{32} \\
\end{aligned}
\end{equation}

In that last step an arbitrary but convenient decision to write the complex number \(i\) as \(\Be_{12}\) was employed.


%\subsection{Scripted output.  All parameterization variations}
%
%
%% 19 (factor out sign).
%
%
%
% VERSION 1
%
%
%
%
%Let us write
%\begin{align*}
%R &= \alpha + \beta\Be_{21} + \gamma \Be_{32} + \delta \Be_{31} \\
%\alpha &= \cos\left(\frac{\theta}{2}\right) \cos\left( \inv{2}\left(\psi+\phi\right) \right) \\
%\beta &= \cos\left(\frac{\theta}{2}\right) \sin\left( \inv{2}\left(\psi+\phi\right) \right) \\
%\gamma &= \sin\left(\frac{\theta}{2}\right) \cos\left( \inv{2}\left(\psi-\phi\right) \right) \\
%\delta &= \sin\left(\frac{\theta}{2}\right) \sin\left( \inv{2}\left(\psi-\phi\right) \right) \\
%\end{align*}
%
%By inspection, these have the required property \(\alpha^2 + \gamma^2 + \delta^2 + \beta^2 = 1\), and
%multiplying out the rotors yields the rotation matrix
%\begin{align*}
%R =
%\begin{bmatrix}
%- \delta^2 + \gamma^2 - \beta^2 + \alpha^2 & -2 \gamma \delta -2 \alpha \beta & +2 \beta \gamma -2 \alpha \delta \\
%- 2 \gamma \delta + 2 \alpha \beta & +\delta^2 - \gamma^2 - \beta^2 + \alpha^2 & -2 \beta \delta -2 \alpha \gamma \\
%+ 2 \beta \gamma + 2 \alpha \delta & -2 \beta \delta + 2 \alpha \gamma & - \delta^2 - \gamma^2 + \beta^2 + \alpha^2 \\
%\end{bmatrix}
%\end{align*}
%
%
%
%
% VERSION 2
%
%
%
%
%Let us write
%\begin{align*}
%R &= \alpha + \beta\Be_{21} + \delta \Be_{32} + \gamma \Be_{31} \\
%\alpha &= \cos\left(\frac{\theta}{2}\right) \cos\left( \inv{2}\left(\psi+\phi\right) \right) \\
%\beta &= \cos\left(\frac{\theta}{2}\right) \sin\left( \inv{2}\left(\psi+\phi\right) \right) \\
%\delta &= \sin\left(\frac{\theta}{2}\right) \cos\left( \inv{2}\left(\psi-\phi\right) \right) \\
%\gamma &= \sin\left(\frac{\theta}{2}\right) \sin\left( \inv{2}\left(\psi-\phi\right) \right) \\
%\end{align*}
%
%By inspection, these have the required property \(\alpha^2 + \delta^2 + \gamma^2 + \beta^2 = 1\), and
%multiplying out the rotors yields the rotation matrix
%\begin{align*}
%R =
%\begin{bmatrix}
%- \gamma^2 + \delta^2 - \beta^2 + \alpha^2 & -2 \delta \gamma -2 \alpha \beta & +2 \beta \delta -2 \alpha \gamma \\
%- 2 \delta \gamma + 2 \alpha \beta & +\gamma^2 - \delta^2 - \beta^2 + \alpha^2 & -2 \beta \gamma -2 \alpha \delta \\
%+ 2 \beta \delta + 2 \alpha \gamma & -2 \beta \gamma + 2 \alpha \delta & - \gamma^2 - \delta^2 + \beta^2 + \alpha^2 \\
%\end{bmatrix}
%\end{align*}
%
%
%
%
% VERSION 3
%
%
%
%
%Let us write
%\begin{align*}
%R &= \alpha + \gamma\Be_{21} + \delta \Be_{32} + \beta \Be_{31} \\
%\alpha &= \cos\left(\frac{\theta}{2}\right) \cos\left( \inv{2}\left(\psi+\phi\right) \right) \\
%\gamma &= \cos\left(\frac{\theta}{2}\right) \sin\left( \inv{2}\left(\psi+\phi\right) \right) \\
%\delta &= \sin\left(\frac{\theta}{2}\right) \cos\left( \inv{2}\left(\psi-\phi\right) \right) \\
%\beta &= \sin\left(\frac{\theta}{2}\right) \sin\left( \inv{2}\left(\psi-\phi\right) \right) \\
%\end{align*}
%
%By inspection, these have the required property \(\alpha^2 + \delta^2 + \beta^2 + \gamma^2 = 1\), and
%multiplying out the rotors yields the rotation matrix
%\begin{align*}
%R =
%\begin{bmatrix}
%- \beta^2 + \delta^2 - \gamma^2 + \alpha^2 & -2 \delta \beta -2 \alpha \gamma & +2 \gamma \delta -2 \alpha \beta \\
%- 2 \delta \beta + 2 \alpha \gamma & +\beta^2 - \delta^2 - \gamma^2 + \alpha^2 & -2 \gamma \beta -2 \alpha \delta \\
%+ 2 \gamma \delta + 2 \alpha \beta & -2 \gamma \beta + 2 \alpha \delta & - \beta^2 - \delta^2 + \gamma^2 + \alpha^2 \\
%\end{bmatrix}
%\end{align*}
%
%
%
%
% VERSION 4
%
%
%
%
%Let us write
%\begin{align*}
%R &= \alpha + \gamma\Be_{21} + \beta \Be_{32} + \delta \Be_{31} \\
%\alpha &= \cos\left(\frac{\theta}{2}\right) \cos\left( \inv{2}\left(\psi+\phi\right) \right) \\
%\gamma &= \cos\left(\frac{\theta}{2}\right) \sin\left( \inv{2}\left(\psi+\phi\right) \right) \\
%\beta &= \sin\left(\frac{\theta}{2}\right) \cos\left( \inv{2}\left(\psi-\phi\right) \right) \\
%\delta &= \sin\left(\frac{\theta}{2}\right) \sin\left( \inv{2}\left(\psi-\phi\right) \right) \\
%\end{align*}
%
%By inspection, these have the required property \(\alpha^2 + \beta^2 + \delta^2 + \gamma^2 = 1\), and
%multiplying out the rotors yields the rotation matrix
%\begin{align*}
%R =
%\begin{bmatrix}
%- \delta^2 + \beta^2 - \gamma^2 + \alpha^2 & -2 \beta \delta -2 \alpha \gamma & +2 \gamma \beta -2 \alpha \delta \\
%- 2 \beta \delta + 2 \alpha \gamma & +\delta^2 - \beta^2 - \gamma^2 + \alpha^2 & -2 \gamma \delta -2 \alpha \beta \\
%+ 2 \gamma \beta + 2 \alpha \delta & -2 \gamma \delta + 2 \alpha \beta & - \delta^2 - \beta^2 + \gamma^2 + \alpha^2 \\
%\end{bmatrix}
%\end{align*}
%
%
%
%
% VERSION 5
%
%
%
%
%Let us write
%\begin{align*}
%R &= \alpha + \delta\Be_{21} + \beta \Be_{32} + \gamma \Be_{31} \\
%\alpha &= \cos\left(\frac{\theta}{2}\right) \cos\left( \inv{2}\left(\psi+\phi\right) \right) \\
%\delta &= \cos\left(\frac{\theta}{2}\right) \sin\left( \inv{2}\left(\psi+\phi\right) \right) \\
%\beta &= \sin\left(\frac{\theta}{2}\right) \cos\left( \inv{2}\left(\psi-\phi\right) \right) \\
%\gamma &= \sin\left(\frac{\theta}{2}\right) \sin\left( \inv{2}\left(\psi-\phi\right) \right) \\
%\end{align*}
%
%By inspection, these have the required property \(\alpha^2 + \beta^2 + \gamma^2 + \delta^2 = 1\), and
%multiplying out the rotors yields the rotation matrix
%\begin{align*}
%R =
%\begin{bmatrix}
%- \gamma^2 + \beta^2 - \delta^2 + \alpha^2 & -2 \beta \gamma -2 \alpha \delta & +2 \delta \beta -2 \alpha \gamma \\
%- 2 \beta \gamma + 2 \alpha \delta & +\gamma^2 - \beta^2 - \delta^2 + \alpha^2 & -2 \delta \gamma -2 \alpha \beta \\
%+ 2 \delta \beta + 2 \alpha \gamma & -2 \delta \gamma + 2 \alpha \beta & - \gamma^2 - \beta^2 + \delta^2 + \alpha^2 \\
%\end{bmatrix}
%\end{align*}
%
%
%
%
% VERSION 6
%
%
%
%
%Let us write
%\begin{align*}
%R &= \alpha + \delta\Be_{21} + \gamma \Be_{32} + \beta \Be_{31} \\
%\alpha &= \cos\left(\frac{\theta}{2}\right) \cos\left( \inv{2}\left(\psi+\phi\right) \right) \\
%\delta &= \cos\left(\frac{\theta}{2}\right) \sin\left( \inv{2}\left(\psi+\phi\right) \right) \\
%\gamma &= \sin\left(\frac{\theta}{2}\right) \cos\left( \inv{2}\left(\psi-\phi\right) \right) \\
%\beta &= \sin\left(\frac{\theta}{2}\right) \sin\left( \inv{2}\left(\psi-\phi\right) \right) \\
%\end{align*}
%
%By inspection, these have the required property \(\alpha^2 + \gamma^2 + \beta^2 + \delta^2 = 1\), and
%multiplying out the rotors yields the rotation matrix
%\begin{align*}
%R =
%\begin{bmatrix}
%- \beta^2 + \gamma^2 - \delta^2 + \alpha^2 & -2 \gamma \beta -2 \alpha \delta & +2 \delta \gamma -2 \alpha \beta \\
%- 2 \gamma \beta + 2 \alpha \delta & +\beta^2 - \gamma^2 - \delta^2 + \alpha^2 & -2 \delta \beta -2 \alpha \gamma \\
%+ 2 \delta \gamma + 2 \alpha \beta & -2 \delta \beta + 2 \alpha \gamma & - \beta^2 - \gamma^2 + \delta^2 + \alpha^2 \\
%\end{bmatrix}
%\end{align*}
%
%
%
%
% VERSION 7
%
%
%
%
%Let us write
%\begin{align*}
%R &= \beta + \alpha\Be_{21} + \gamma \Be_{32} + \delta \Be_{31} \\
%\beta &= \cos\left(\frac{\theta}{2}\right) \cos\left( \inv{2}\left(\psi+\phi\right) \right) \\
%\alpha &= \cos\left(\frac{\theta}{2}\right) \sin\left( \inv{2}\left(\psi+\phi\right) \right) \\
%\gamma &= \sin\left(\frac{\theta}{2}\right) \cos\left( \inv{2}\left(\psi-\phi\right) \right) \\
%\delta &= \sin\left(\frac{\theta}{2}\right) \sin\left( \inv{2}\left(\psi-\phi\right) \right) \\
%\end{align*}
%
%By inspection, these have the required property \(\beta^2 + \gamma^2 + \delta^2 + \alpha^2 = 1\), and
%multiplying out the rotors yields the rotation matrix
%\begin{align*}
%R =
%\begin{bmatrix}
%- \delta^2 + \gamma^2 - \alpha^2 + \beta^2 & -2 \gamma \delta -2 \beta \alpha & +2 \alpha \gamma -2 \beta \delta \\
%- 2 \gamma \delta + 2 \beta \alpha & +\delta^2 - \gamma^2 - \alpha^2 + \beta^2 & -2 \alpha \delta -2 \beta \gamma \\
%+ 2 \alpha \gamma + 2 \beta \delta & -2 \alpha \delta + 2 \beta \gamma & - \delta^2 - \gamma^2 + \alpha^2 + \beta^2 \\
%\end{bmatrix}
%\end{align*}
%
%
%
%
% VERSION 8
%
%
%
%
%Let us write
%\begin{align*}
%R &= \beta + \alpha\Be_{21} + \delta \Be_{32} + \gamma \Be_{31} \\
%\beta &= \cos\left(\frac{\theta}{2}\right) \cos\left( \inv{2}\left(\psi+\phi\right) \right) \\
%\alpha &= \cos\left(\frac{\theta}{2}\right) \sin\left( \inv{2}\left(\psi+\phi\right) \right) \\
%\delta &= \sin\left(\frac{\theta}{2}\right) \cos\left( \inv{2}\left(\psi-\phi\right) \right) \\
%\gamma &= \sin\left(\frac{\theta}{2}\right) \sin\left( \inv{2}\left(\psi-\phi\right) \right) \\
%\end{align*}
%
%By inspection, these have the required property \(\beta^2 + \delta^2 + \gamma^2 + \alpha^2 = 1\), and
%multiplying out the rotors yields the rotation matrix
%\begin{align*}
%R =
%\begin{bmatrix}
%- \gamma^2 + \delta^2 - \alpha^2 + \beta^2 & -2 \delta \gamma -2 \beta \alpha & +2 \alpha \delta -2 \beta \gamma \\
%- 2 \delta \gamma + 2 \beta \alpha & +\gamma^2 - \delta^2 - \alpha^2 + \beta^2 & -2 \alpha \gamma -2 \beta \delta \\
%+ 2 \alpha \delta + 2 \beta \gamma & -2 \alpha \gamma + 2 \beta \delta & - \gamma^2 - \delta^2 + \alpha^2 + \beta^2 \\
%\end{bmatrix}
%\end{align*}
%
%
%
%
% VERSION 9
%
%
%
%
%Let us write
%\begin{align*}
%R &= \beta + \gamma\Be_{21} + \delta \Be_{32} + \alpha \Be_{31} \\
%\beta &= \cos\left(\frac{\theta}{2}\right) \cos\left( \inv{2}\left(\psi+\phi\right) \right) \\
%\gamma &= \cos\left(\frac{\theta}{2}\right) \sin\left( \inv{2}\left(\psi+\phi\right) \right) \\
%\delta &= \sin\left(\frac{\theta}{2}\right) \cos\left( \inv{2}\left(\psi-\phi\right) \right) \\
%\alpha &= \sin\left(\frac{\theta}{2}\right) \sin\left( \inv{2}\left(\psi-\phi\right) \right) \\
%\end{align*}
%
%By inspection, these have the required property \(\beta^2 + \delta^2 + \alpha^2 + \gamma^2 = 1\), and
%multiplying out the rotors yields the rotation matrix
%\begin{align*}
%R =
%\begin{bmatrix}
%- \alpha^2 + \delta^2 - \gamma^2 + \beta^2 & -2 \delta \alpha -2 \beta \gamma & +2 \gamma \delta -2 \beta \alpha \\
%- 2 \delta \alpha + 2 \beta \gamma & +\alpha^2 - \delta^2 - \gamma^2 + \beta^2 & -2 \gamma \alpha -2 \beta \delta \\
%+ 2 \gamma \delta + 2 \beta \alpha & -2 \gamma \alpha + 2 \beta \delta & - \alpha^2 - \delta^2 + \gamma^2 + \beta^2 \\
%\end{bmatrix}
%\end{align*}
%
%
%
%
% VERSION 10
%
%
%
%
%Let us write
%\begin{align*}
%R &= \beta + \gamma\Be_{21} + \alpha \Be_{32} + \delta \Be_{31} \\
%\beta &= \cos\left(\frac{\theta}{2}\right) \cos\left( \inv{2}\left(\psi+\phi\right) \right) \\
%\gamma &= \cos\left(\frac{\theta}{2}\right) \sin\left( \inv{2}\left(\psi+\phi\right) \right) \\
%\alpha &= \sin\left(\frac{\theta}{2}\right) \cos\left( \inv{2}\left(\psi-\phi\right) \right) \\
%\delta &= \sin\left(\frac{\theta}{2}\right) \sin\left( \inv{2}\left(\psi-\phi\right) \right) \\
%\end{align*}
%
%By inspection, these have the required property \(\beta^2 + \alpha^2 + \delta^2 + \gamma^2 = 1\), and
%multiplying out the rotors yields the rotation matrix
%\begin{align*}
%R =
%\begin{bmatrix}
%- \delta^2 + \alpha^2 - \gamma^2 + \beta^2 & -2 \alpha \delta -2 \beta \gamma & +2 \gamma \alpha -2 \beta \delta \\
%- 2 \alpha \delta + 2 \beta \gamma & +\delta^2 - \alpha^2 - \gamma^2 + \beta^2 & -2 \gamma \delta -2 \beta \alpha \\
%+ 2 \gamma \alpha + 2 \beta \delta & -2 \gamma \delta + 2 \beta \alpha & - \delta^2 - \alpha^2 + \gamma^2 + \beta^2 \\
%\end{bmatrix}
%\end{align*}
%
%
%
%
% VERSION 11
%
%
%
%
%Let us write
%\begin{align*}
%R &= \beta + \delta\Be_{21} + \alpha \Be_{32} + \gamma \Be_{31} \\
%\beta &= \cos\left(\frac{\theta}{2}\right) \cos\left( \inv{2}\left(\psi+\phi\right) \right) \\
%\delta &= \cos\left(\frac{\theta}{2}\right) \sin\left( \inv{2}\left(\psi+\phi\right) \right) \\
%\alpha &= \sin\left(\frac{\theta}{2}\right) \cos\left( \inv{2}\left(\psi-\phi\right) \right) \\
%\gamma &= \sin\left(\frac{\theta}{2}\right) \sin\left( \inv{2}\left(\psi-\phi\right) \right) \\
%\end{align*}
%
%By inspection, these have the required property \(\beta^2 + \alpha^2 + \gamma^2 + \delta^2 = 1\), and
%multiplying out the rotors yields the rotation matrix
%\begin{align*}
%R =
%\begin{bmatrix}
%- \gamma^2 + \alpha^2 - \delta^2 + \beta^2 & -2 \alpha \gamma -2 \beta \delta & +2 \delta \alpha -2 \beta \gamma \\
%- 2 \alpha \gamma + 2 \beta \delta & +\gamma^2 - \alpha^2 - \delta^2 + \beta^2 & -2 \delta \gamma -2 \beta \alpha \\
%+ 2 \delta \alpha + 2 \beta \gamma & -2 \delta \gamma + 2 \beta \alpha & - \gamma^2 - \alpha^2 + \delta^2 + \beta^2 \\
%\end{bmatrix}
%\end{align*}
%
%
%
%
% VERSION 12
%
%
%
%
%Let us write
%\begin{align*}
%R &= \beta + \delta\Be_{21} + \gamma \Be_{32} + \alpha \Be_{31} \\
%\beta &= \cos\left(\frac{\theta}{2}\right) \cos\left( \inv{2}\left(\psi+\phi\right) \right) \\
%\delta &= \cos\left(\frac{\theta}{2}\right) \sin\left( \inv{2}\left(\psi+\phi\right) \right) \\
%\gamma &= \sin\left(\frac{\theta}{2}\right) \cos\left( \inv{2}\left(\psi-\phi\right) \right) \\
%\alpha &= \sin\left(\frac{\theta}{2}\right) \sin\left( \inv{2}\left(\psi-\phi\right) \right) \\
%\end{align*}
%
%By inspection, these have the required property \(\beta^2 + \gamma^2 + \alpha^2 + \delta^2 = 1\), and
%multiplying out the rotors yields the rotation matrix
%\begin{align*}
%R =
%\begin{bmatrix}
%- \alpha^2 + \gamma^2 - \delta^2 + \beta^2 & -2 \gamma \alpha -2 \beta \delta & +2 \delta \gamma -2 \beta \alpha \\
%- 2 \gamma \alpha + 2 \beta \delta & +\alpha^2 - \gamma^2 - \delta^2 + \beta^2 & -2 \delta \alpha -2 \beta \gamma \\
%+ 2 \delta \gamma + 2 \beta \alpha & -2 \delta \alpha + 2 \beta \gamma & - \alpha^2 - \gamma^2 + \delta^2 + \beta^2 \\
%\end{bmatrix}
%\end{align*}
%
%
%
%
% VERSION 13
%
%
%
%
%Let us write
%\begin{align*}
%R &= \gamma + \beta\Be_{21} + \alpha \Be_{32} + \delta \Be_{31} \\
%\gamma &= \cos\left(\frac{\theta}{2}\right) \cos\left( \inv{2}\left(\psi+\phi\right) \right) \\
%\beta &= \cos\left(\frac{\theta}{2}\right) \sin\left( \inv{2}\left(\psi+\phi\right) \right) \\
%\alpha &= \sin\left(\frac{\theta}{2}\right) \cos\left( \inv{2}\left(\psi-\phi\right) \right) \\
%\delta &= \sin\left(\frac{\theta}{2}\right) \sin\left( \inv{2}\left(\psi-\phi\right) \right) \\
%\end{align*}
%
%By inspection, these have the required property \(\gamma^2 + \alpha^2 + \delta^2 + \beta^2 = 1\), and
%multiplying out the rotors yields the rotation matrix
%\begin{align*}
%R =
%\begin{bmatrix}
%- \delta^2 + \alpha^2 - \beta^2 + \gamma^2 & -2 \alpha \delta -2 \gamma \beta & +2 \beta \alpha -2 \gamma \delta \\
%- 2 \alpha \delta + 2 \gamma \beta & +\delta^2 - \alpha^2 - \beta^2 + \gamma^2 & -2 \beta \delta -2 \gamma \alpha \\
%+ 2 \beta \alpha + 2 \gamma \delta & -2 \beta \delta + 2 \gamma \alpha & - \delta^2 - \alpha^2 + \beta^2 + \gamma^2 \\
%\end{bmatrix}
%\end{align*}
%
%
%
%
% VERSION 14
%
%
%
%
%Let us write
%\begin{align*}
%R &= \gamma + \beta\Be_{21} + \delta \Be_{32} + \alpha \Be_{31} \\
%\gamma &= \cos\left(\frac{\theta}{2}\right) \cos\left( \inv{2}\left(\psi+\phi\right) \right) \\
%\beta &= \cos\left(\frac{\theta}{2}\right) \sin\left( \inv{2}\left(\psi+\phi\right) \right) \\
%\delta &= \sin\left(\frac{\theta}{2}\right) \cos\left( \inv{2}\left(\psi-\phi\right) \right) \\
%\alpha &= \sin\left(\frac{\theta}{2}\right) \sin\left( \inv{2}\left(\psi-\phi\right) \right) \\
%\end{align*}
%
%By inspection, these have the required property \(\gamma^2 + \delta^2 + \alpha^2 + \beta^2 = 1\), and
%multiplying out the rotors yields the rotation matrix
%\begin{align*}
%R =
%-
%\begin{bmatrix}
%+ \alpha^2 - \delta^2 + \beta^2 - \gamma^2 & +2 \delta \alpha +2 \gamma \beta & -2 \beta \delta +2 \gamma \alpha \\
%+ 2 \delta \alpha - 2 \gamma \beta & -\alpha^2 + \delta^2 + \beta^2 - \gamma^2 & +2 \beta \alpha +2 \gamma \delta \\
%- 2 \beta \delta - 2 \gamma \alpha & +2 \beta \alpha - 2 \gamma \delta & + \alpha^2 + \delta^2 - \beta^2 - \gamma^2 \\
%\end{bmatrix}
%\end{align*}
%
%
%
%
% VERSION 15
%
%
%
%
%Let us write
%\begin{align*}
%R &= \gamma + \alpha\Be_{21} + \delta \Be_{32} + \beta \Be_{31} \\
%\gamma &= \cos\left(\frac{\theta}{2}\right) \cos\left( \inv{2}\left(\psi+\phi\right) \right) \\
%\alpha &= \cos\left(\frac{\theta}{2}\right) \sin\left( \inv{2}\left(\psi+\phi\right) \right) \\
%\delta &= \sin\left(\frac{\theta}{2}\right) \cos\left( \inv{2}\left(\psi-\phi\right) \right) \\
%\beta &= \sin\left(\frac{\theta}{2}\right) \sin\left( \inv{2}\left(\psi-\phi\right) \right) \\
%\end{align*}
%
%By inspection, these have the required property \(\gamma^2 + \delta^2 + \beta^2 + \alpha^2 = 1\), and
%multiplying out the rotors yields the rotation matrix
%\begin{align*}
%R =
%\begin{bmatrix}
%- \beta^2 + \delta^2 - \alpha^2 + \gamma^2 & -2 \delta \beta -2 \gamma \alpha & +2 \alpha \delta -2 \gamma \beta \\
%- 2 \delta \beta + 2 \gamma \alpha & +\beta^2 - \delta^2 - \alpha^2 + \gamma^2 & -2 \alpha \beta -2 \gamma \delta \\
%+ 2 \alpha \delta + 2 \gamma \beta & -2 \alpha \beta + 2 \gamma \delta & - \beta^2 - \delta^2 + \alpha^2 + \gamma^2 \\
%\end{bmatrix}
%\end{align*}
%
%
%
%
% VERSION 16
%
%
%
%
%Let us write
%\begin{align*}
%R &= \gamma + \alpha\Be_{21} + \beta \Be_{32} + \delta \Be_{31} \\
%\gamma &= \cos\left(\frac{\theta}{2}\right) \cos\left( \inv{2}\left(\psi+\phi\right) \right) \\
%\alpha &= \cos\left(\frac{\theta}{2}\right) \sin\left( \inv{2}\left(\psi+\phi\right) \right) \\
%\beta &= \sin\left(\frac{\theta}{2}\right) \cos\left( \inv{2}\left(\psi-\phi\right) \right) \\
%\delta &= \sin\left(\frac{\theta}{2}\right) \sin\left( \inv{2}\left(\psi-\phi\right) \right) \\
%\end{align*}
%
%By inspection, these have the required property \(\gamma^2 + \beta^2 + \delta^2 + \alpha^2 = 1\), and
%multiplying out the rotors yields the rotation matrix
%\begin{align*}
%R =
%\begin{bmatrix}
%- \delta^2 + \beta^2 - \alpha^2 + \gamma^2 & -2 \beta \delta -2 \gamma \alpha & +2 \alpha \beta -2 \gamma \delta \\
%- 2 \beta \delta + 2 \gamma \alpha & +\delta^2 - \beta^2 - \alpha^2 + \gamma^2 & -2 \alpha \delta -2 \gamma \beta \\
%+ 2 \alpha \beta + 2 \gamma \delta & -2 \alpha \delta + 2 \gamma \beta & - \delta^2 - \beta^2 + \alpha^2 + \gamma^2 \\
%\end{bmatrix}
%\end{align*}
%
%
%
%
% VERSION 17
%
%
%
%
%Let us write
%\begin{align*}
%R &= \gamma + \delta\Be_{21} + \beta \Be_{32} + \alpha \Be_{31} \\
%\gamma &= \cos\left(\frac{\theta}{2}\right) \cos\left( \inv{2}\left(\psi+\phi\right) \right) \\
%\delta &= \cos\left(\frac{\theta}{2}\right) \sin\left( \inv{2}\left(\psi+\phi\right) \right) \\
%\beta &= \sin\left(\frac{\theta}{2}\right) \cos\left( \inv{2}\left(\psi-\phi\right) \right) \\
%\alpha &= \sin\left(\frac{\theta}{2}\right) \sin\left( \inv{2}\left(\psi-\phi\right) \right) \\
%\end{align*}
%
%By inspection, these have the required property \(\gamma^2 + \beta^2 + \alpha^2 + \delta^2 = 1\), and
%multiplying out the rotors yields the rotation matrix
%\begin{align*}
%R =
%\begin{bmatrix}
%- \alpha^2 + \beta^2 - \delta^2 + \gamma^2 & -2 \beta \alpha -2 \gamma \delta & +2 \delta \beta -2 \gamma \alpha \\
%- 2 \beta \alpha + 2 \gamma \delta & +\alpha^2 - \beta^2 - \delta^2 + \gamma^2 & -2 \delta \alpha -2 \gamma \beta \\
%+ 2 \delta \beta + 2 \gamma \alpha & -2 \delta \alpha + 2 \gamma \beta & - \alpha^2 - \beta^2 + \delta^2 + \gamma^2 \\
%\end{bmatrix}
%\end{align*}
%
%
%
%
% VERSION 18
%
%
%
%
%Let us write
%\begin{align*}
%R &= \gamma + \delta\Be_{21} + \alpha \Be_{32} + \beta \Be_{31} \\
%\gamma &= \cos\left(\frac{\theta}{2}\right) \cos\left( \inv{2}\left(\psi+\phi\right) \right) \\
%\delta &= \cos\left(\frac{\theta}{2}\right) \sin\left( \inv{2}\left(\psi+\phi\right) \right) \\
%\alpha &= \sin\left(\frac{\theta}{2}\right) \cos\left( \inv{2}\left(\psi-\phi\right) \right) \\
%\beta &= \sin\left(\frac{\theta}{2}\right) \sin\left( \inv{2}\left(\psi-\phi\right) \right) \\
%\end{align*}
%
%By inspection, these have the required property \(\gamma^2 + \alpha^2 + \beta^2 + \delta^2 = 1\), and
%multiplying out the rotors yields the rotation matrix
%\begin{align*}
%R =
%\begin{bmatrix}
%- \beta^2 + \alpha^2 - \delta^2 + \gamma^2 & -2 \alpha \beta -2 \gamma \delta & +2 \delta \alpha -2 \gamma \beta \\
%- 2 \alpha \beta + 2 \gamma \delta & +\beta^2 - \alpha^2 - \delta^2 + \gamma^2 & -2 \delta \beta -2 \gamma \alpha \\
%+ 2 \delta \alpha + 2 \gamma \beta & -2 \delta \beta + 2 \gamma \alpha & - \beta^2 - \alpha^2 + \delta^2 + \gamma^2 \\
%\end{bmatrix}
%\end{align*}
%
%
%
%
% VERSION 19
%
%
%
%
%Let us write
%\begin{align*}
%R &= \delta + \beta\Be_{21} + \gamma \Be_{32} + \alpha \Be_{31} \\
%\delta &= \cos\left(\frac{\theta}{2}\right) \cos\left( \inv{2}\left(\psi+\phi\right) \right) \\
%\beta &= \cos\left(\frac{\theta}{2}\right) \sin\left( \inv{2}\left(\psi+\phi\right) \right) \\
%\gamma &= \sin\left(\frac{\theta}{2}\right) \cos\left( \inv{2}\left(\psi-\phi\right) \right) \\
%\alpha &= \sin\left(\frac{\theta}{2}\right) \sin\left( \inv{2}\left(\psi-\phi\right) \right) \\
%\end{align*}
%
%By inspection, these have the required property \(\delta^2 + \gamma^2 + \alpha^2 + \beta^2 = 1\), and
%multiplying out the rotors yields the rotation matrix
%\begin{align*}
%R =
%-
%\begin{bmatrix}
%+ \alpha^2 - \gamma^2 + \beta^2 - \delta^2 & +2 \gamma \alpha +2 \delta \beta & -2 \beta \gamma +2 \delta \alpha \\
%+ 2 \gamma \alpha - 2 \delta \beta & -\alpha^2 + \gamma^2 + \beta^2 - \delta^2 & +2 \beta \alpha +2 \delta \gamma \\
%- 2 \beta \gamma - 2 \delta \alpha & +2 \beta \alpha - 2 \delta \gamma & + \alpha^2 + \gamma^2 - \beta^2 - \delta^2 \\
%\end{bmatrix}
%\end{align*}
%
%
%
%
% VERSION 20
%
%
%
%
%Let us write
%\begin{align*}
%R &= \delta + \beta\Be_{21} + \alpha \Be_{32} + \gamma \Be_{31} \\
%\delta &= \cos\left(\frac{\theta}{2}\right) \cos\left( \inv{2}\left(\psi+\phi\right) \right) \\
%\beta &= \cos\left(\frac{\theta}{2}\right) \sin\left( \inv{2}\left(\psi+\phi\right) \right) \\
%\alpha &= \sin\left(\frac{\theta}{2}\right) \cos\left( \inv{2}\left(\psi-\phi\right) \right) \\
%\gamma &= \sin\left(\frac{\theta}{2}\right) \sin\left( \inv{2}\left(\psi-\phi\right) \right) \\
%\end{align*}
%
%By inspection, these have the required property \(\delta^2 + \alpha^2 + \gamma^2 + \beta^2 = 1\), and
%multiplying out the rotors yields the rotation matrix
%\begin{align*}
%R =
%\begin{bmatrix}
%- \gamma^2 + \alpha^2 - \beta^2 + \delta^2 & -2 \alpha \gamma -2 \delta \beta & +2 \beta \alpha -2 \delta \gamma \\
%- 2 \alpha \gamma + 2 \delta \beta & +\gamma^2 - \alpha^2 - \beta^2 + \delta^2 & -2 \beta \gamma -2 \delta \alpha \\
%+ 2 \beta \alpha + 2 \delta \gamma & -2 \beta \gamma + 2 \delta \alpha & - \gamma^2 - \alpha^2 + \beta^2 + \delta^2 \\
%\end{bmatrix}
%\end{align*}
%
%
%
%
% VERSION 21
%
%
%
%
%Let us write
%\begin{align*}
%R &= \delta + \gamma\Be_{21} + \alpha \Be_{32} + \beta \Be_{31} \\
%\delta &= \cos\left(\frac{\theta}{2}\right) \cos\left( \inv{2}\left(\psi+\phi\right) \right) \\
%\gamma &= \cos\left(\frac{\theta}{2}\right) \sin\left( \inv{2}\left(\psi+\phi\right) \right) \\
%\alpha &= \sin\left(\frac{\theta}{2}\right) \cos\left( \inv{2}\left(\psi-\phi\right) \right) \\
%\beta &= \sin\left(\frac{\theta}{2}\right) \sin\left( \inv{2}\left(\psi-\phi\right) \right) \\
%\end{align*}
%
%By inspection, these have the required property \(\delta^2 + \alpha^2 + \beta^2 + \gamma^2 = 1\), and
%multiplying out the rotors yields the rotation matrix
%\begin{align*}
%R =
%\begin{bmatrix}
%- \beta^2 + \alpha^2 - \gamma^2 + \delta^2 & -2 \alpha \beta -2 \delta \gamma & +2 \gamma \alpha -2 \delta \beta \\
%- 2 \alpha \beta + 2 \delta \gamma & +\beta^2 - \alpha^2 - \gamma^2 + \delta^2 & -2 \gamma \beta -2 \delta \alpha \\
%+ 2 \gamma \alpha + 2 \delta \beta & -2 \gamma \beta + 2 \delta \alpha & - \beta^2 - \alpha^2 + \gamma^2 + \delta^2 \\
%\end{bmatrix}
%\end{align*}
%
%
%
%
% VERSION 22
%
%
%
%
%Let us write
%\begin{align*}
%R &= \delta + \gamma\Be_{21} + \beta \Be_{32} + \alpha \Be_{31} \\
%\delta &= \cos\left(\frac{\theta}{2}\right) \cos\left( \inv{2}\left(\psi+\phi\right) \right) \\
%\gamma &= \cos\left(\frac{\theta}{2}\right) \sin\left( \inv{2}\left(\psi+\phi\right) \right) \\
%\beta &= \sin\left(\frac{\theta}{2}\right) \cos\left( \inv{2}\left(\psi-\phi\right) \right) \\
%\alpha &= \sin\left(\frac{\theta}{2}\right) \sin\left( \inv{2}\left(\psi-\phi\right) \right) \\
%\end{align*}
%
%By inspection, these have the required property \(\delta^2 + \beta^2 + \alpha^2 + \gamma^2 = 1\), and
%multiplying out the rotors yields the rotation matrix
%\begin{align*}
%R =
%\begin{bmatrix}
%- \alpha^2 + \beta^2 - \gamma^2 + \delta^2 & -2 \beta \alpha -2 \delta \gamma & +2 \gamma \beta -2 \delta \alpha \\
%- 2 \beta \alpha + 2 \delta \gamma & +\alpha^2 - \beta^2 - \gamma^2 + \delta^2 & -2 \gamma \alpha -2 \delta \beta \\
%+ 2 \gamma \beta + 2 \delta \alpha & -2 \gamma \alpha + 2 \delta \beta & - \alpha^2 - \beta^2 + \gamma^2 + \delta^2 \\
%\end{bmatrix}
%\end{align*}
%
%
%
%
% VERSION 23
%
%
%
%
%Let us write
%\begin{align*}
%R &= \delta + \alpha\Be_{21} + \beta \Be_{32} + \gamma \Be_{31} \\
%\delta &= \cos\left(\frac{\theta}{2}\right) \cos\left( \inv{2}\left(\psi+\phi\right) \right) \\
%\alpha &= \cos\left(\frac{\theta}{2}\right) \sin\left( \inv{2}\left(\psi+\phi\right) \right) \\
%\beta &= \sin\left(\frac{\theta}{2}\right) \cos\left( \inv{2}\left(\psi-\phi\right) \right) \\
%\gamma &= \sin\left(\frac{\theta}{2}\right) \sin\left( \inv{2}\left(\psi-\phi\right) \right) \\
%\end{align*}
%
%By inspection, these have the required property \(\delta^2 + \beta^2 + \gamma^2 + \alpha^2 = 1\), and
%multiplying out the rotors yields the rotation matrix
%\begin{align*}
%R =
%\begin{bmatrix}
%- \gamma^2 + \beta^2 - \alpha^2 + \delta^2 & -2 \beta \gamma -2 \delta \alpha & +2 \alpha \beta -2 \delta \gamma \\
%- 2 \beta \gamma + 2 \delta \alpha & +\gamma^2 - \beta^2 - \alpha^2 + \delta^2 & -2 \alpha \gamma -2 \delta \beta \\
%+ 2 \alpha \beta + 2 \delta \gamma & -2 \alpha \gamma + 2 \delta \beta & - \gamma^2 - \beta^2 + \alpha^2 + \delta^2 \\
%\end{bmatrix}
%\end{align*}
%
%
%
%
% VERSION 24
%
%
%
%
%Let us write
%\begin{align*}
%R &= \delta + \alpha\Be_{21} + \gamma \Be_{32} + \beta \Be_{31} \\
%\delta &= \cos\left(\frac{\theta}{2}\right) \cos\left( \inv{2}\left(\psi+\phi\right) \right) \\
%\alpha &= \cos\left(\frac{\theta}{2}\right) \sin\left( \inv{2}\left(\psi+\phi\right) \right) \\
%\gamma &= \sin\left(\frac{\theta}{2}\right) \cos\left( \inv{2}\left(\psi-\phi\right) \right) \\
%\beta &= \sin\left(\frac{\theta}{2}\right) \sin\left( \inv{2}\left(\psi-\phi\right) \right) \\
%\end{align*}
%
%By inspection, these have the required property \(\delta^2 + \gamma^2 + \beta^2 + \alpha^2 = 1\), and
%multiplying out the rotors yields the rotation matrix
%\begin{align*}
%R =
%\begin{bmatrix}
%- \beta^2 + \gamma^2 - \alpha^2 + \delta^2 & -2 \gamma \beta -2 \delta \alpha & +2 \alpha \gamma -2 \delta \beta \\
%- 2 \gamma \beta + 2 \delta \alpha & +\beta^2 - \gamma^2 - \alpha^2 + \delta^2 & -2 \alpha \beta -2 \delta \gamma \\
%+ 2 \alpha \gamma + 2 \delta \beta & -2 \alpha \beta + 2 \delta \gamma & - \beta^2 - \gamma^2 + \alpha^2 + \delta^2 \\
%\end{bmatrix}
%\end{align*}
%

%\subsection{Scripted output.  All parameterization variations}
%
%
%
%
%
%
% VERSION 1
%
%
%
%
%\begin{align*}
%R &= \alpha - \beta\Be_{21} - \gamma \Be_{32} - \delta \Be_{31} \\
%\alpha &= \cos\left(\frac{\theta}{2}\right) \cos\left( \inv{2}\left(\psi+\phi\right) \right) \\
%\beta &= -\cos\left(\frac{\theta}{2}\right) \sin\left( \inv{2}\left(\psi+\phi\right) \right) \\
%\gamma &= -\sin\left(\frac{\theta}{2}\right) \cos\left( \inv{2}\left(\psi-\phi\right) \right) \\
%\delta &= -\sin\left(\frac{\theta}{2}\right) \sin\left( \inv{2}\left(\psi-\phi\right) \right) \\
%\end{align*}
%
%\begin{align*}
%\begin{bmatrix}
% - \delta^2 + \gamma^2 - \beta^2 + \alpha^2 & - 2  \gamma \delta + 2 \alpha \beta & + 2 \beta \gamma + 2 \alpha \delta \\
% - 2 \gamma \delta - 2 \alpha \beta & + \delta^2 - \gamma^2 - \beta^2 + \alpha^2 & - 2 \beta \delta + 2 \alpha \gamma \\
% + 2 \beta \gamma - 2 \alpha \delta & - 2 \beta \delta - 2 \alpha \gamma & - \delta^2 - \gamma^2 + \beta^2 + \alpha^2 \\
%\end{bmatrix}
%\end{align*}
%
%
%
%
% VERSION 2
%
%
%
%
%\begin{align*}
%R &= \alpha - \beta\Be_{21} - \delta \Be_{32} - \gamma \Be_{31} \\
%\alpha &= \cos\left(\frac{\theta}{2}\right) \cos\left( \inv{2}\left(\psi+\phi\right) \right) \\
%\beta &= -\cos\left(\frac{\theta}{2}\right) \sin\left( \inv{2}\left(\psi+\phi\right) \right) \\
%\delta &= -\sin\left(\frac{\theta}{2}\right) \cos\left( \inv{2}\left(\psi-\phi\right) \right) \\
%\gamma &= -\sin\left(\frac{\theta}{2}\right) \sin\left( \inv{2}\left(\psi-\phi\right) \right) \\
%\end{align*}
%
%\begin{align*}
%\begin{bmatrix}
% - \gamma^2 + \delta^2 - \beta^2 + \alpha^2 & - 2  \delta \gamma + 2 \alpha \beta & + 2 \beta \delta + 2 \alpha \gamma \\
% - 2 \delta \gamma - 2 \alpha \beta & + \gamma^2 - \delta^2 - \beta^2 + \alpha^2 & - 2 \beta \gamma + 2 \alpha \delta \\
% + 2 \beta \delta - 2 \alpha \gamma & - 2 \beta \gamma - 2 \alpha \delta & - \gamma^2 - \delta^2 + \beta^2 + \alpha^2 \\
%\end{bmatrix}
%\end{align*}
%
%
%
%
% VERSION 3
%
%
%
%
%\begin{align*}
%R &= \alpha - \gamma\Be_{21} - \delta \Be_{32} - \beta \Be_{31} \\
%\alpha &= \cos\left(\frac{\theta}{2}\right) \cos\left( \inv{2}\left(\psi+\phi\right) \right) \\
%\gamma &= -\cos\left(\frac{\theta}{2}\right) \sin\left( \inv{2}\left(\psi+\phi\right) \right) \\
%\delta &= -\sin\left(\frac{\theta}{2}\right) \cos\left( \inv{2}\left(\psi-\phi\right) \right) \\
%\beta &= -\sin\left(\frac{\theta}{2}\right) \sin\left( \inv{2}\left(\psi-\phi\right) \right) \\
%\end{align*}
%
%\begin{align*}
%\begin{bmatrix}
% - \beta^2 + \delta^2 - \gamma^2 + \alpha^2 & - 2  \delta \beta + 2 \alpha \gamma & + 2 \gamma \delta + 2 \alpha \beta \\
% - 2 \delta \beta - 2 \alpha \gamma & + \beta^2 - \delta^2 - \gamma^2 + \alpha^2 & - 2 \gamma \beta + 2 \alpha \delta \\
% + 2 \gamma \delta - 2 \alpha \beta & - 2 \gamma \beta - 2 \alpha \delta & - \beta^2 - \delta^2 + \gamma^2 + \alpha^2 \\
%\end{bmatrix}
%\end{align*}
%
%
%
%
% VERSION 4
%
%
%
%
%\begin{align*}
%R &= \alpha - \gamma\Be_{21} - \beta \Be_{32} - \delta \Be_{31} \\
%\alpha &= \cos\left(\frac{\theta}{2}\right) \cos\left( \inv{2}\left(\psi+\phi\right) \right) \\
%\gamma &= -\cos\left(\frac{\theta}{2}\right) \sin\left( \inv{2}\left(\psi+\phi\right) \right) \\
%\beta &= -\sin\left(\frac{\theta}{2}\right) \cos\left( \inv{2}\left(\psi-\phi\right) \right) \\
%\delta &= -\sin\left(\frac{\theta}{2}\right) \sin\left( \inv{2}\left(\psi-\phi\right) \right) \\
%\end{align*}
%
%\begin{align*}
%\begin{bmatrix}
% - \delta^2 + \beta^2 - \gamma^2 + \alpha^2 & - 2  \beta \delta + 2 \alpha \gamma & + 2 \gamma \beta + 2 \alpha \delta \\
% - 2 \beta \delta - 2 \alpha \gamma & + \delta^2 - \beta^2 - \gamma^2 + \alpha^2 & - 2 \gamma \delta + 2 \alpha \beta \\
% + 2 \gamma \beta - 2 \alpha \delta & - 2 \gamma \delta - 2 \alpha \beta & - \delta^2 - \beta^2 + \gamma^2 + \alpha^2 \\
%\end{bmatrix}
%\end{align*}
%
%
%
%
% VERSION 5
%
%
%
%
%\begin{align*}
%R &= \alpha - \delta\Be_{21} - \beta \Be_{32} - \gamma \Be_{31} \\
%\alpha &= \cos\left(\frac{\theta}{2}\right) \cos\left( \inv{2}\left(\psi+\phi\right) \right) \\
%\delta &= -\cos\left(\frac{\theta}{2}\right) \sin\left( \inv{2}\left(\psi+\phi\right) \right) \\
%\beta &= -\sin\left(\frac{\theta}{2}\right) \cos\left( \inv{2}\left(\psi-\phi\right) \right) \\
%\gamma &= -\sin\left(\frac{\theta}{2}\right) \sin\left( \inv{2}\left(\psi-\phi\right) \right) \\
%\end{align*}
%
%\begin{align*}
%\begin{bmatrix}
% - \gamma^2 + \beta^2 - \delta^2 + \alpha^2 & - 2  \beta \gamma + 2 \alpha \delta & + 2 \delta \beta + 2 \alpha \gamma \\
% - 2 \beta \gamma - 2 \alpha \delta & + \gamma^2 - \beta^2 - \delta^2 + \alpha^2 & - 2 \delta \gamma + 2 \alpha \beta \\
% + 2 \delta \beta - 2 \alpha \gamma & - 2 \delta \gamma - 2 \alpha \beta & - \gamma^2 - \beta^2 + \delta^2 + \alpha^2 \\
%\end{bmatrix}
%\end{align*}
%
%
%
%
% VERSION 6
%
%
%
%
%\begin{align*}
%R &= \alpha - \delta\Be_{21} - \gamma \Be_{32} - \beta \Be_{31} \\
%\alpha &= \cos\left(\frac{\theta}{2}\right) \cos\left( \inv{2}\left(\psi+\phi\right) \right) \\
%\delta &= -\cos\left(\frac{\theta}{2}\right) \sin\left( \inv{2}\left(\psi+\phi\right) \right) \\
%\gamma &= -\sin\left(\frac{\theta}{2}\right) \cos\left( \inv{2}\left(\psi-\phi\right) \right) \\
%\beta &= -\sin\left(\frac{\theta}{2}\right) \sin\left( \inv{2}\left(\psi-\phi\right) \right) \\
%\end{align*}
%
%\begin{align*}
%\begin{bmatrix}
% - \beta^2 + \gamma^2 - \delta^2 + \alpha^2 & - 2  \gamma \beta + 2 \alpha \delta & + 2 \delta \gamma + 2 \alpha \beta \\
% - 2 \gamma \beta - 2 \alpha \delta & + \beta^2 - \gamma^2 - \delta^2 + \alpha^2 & - 2 \delta \beta + 2 \alpha \gamma \\
% + 2 \delta \gamma - 2 \alpha \beta & - 2 \delta \beta - 2 \alpha \gamma & - \beta^2 - \gamma^2 + \delta^2 + \alpha^2 \\
%\end{bmatrix}
%\end{align*}
%
%
%
%
% VERSION 7
%
%
%
%
%\begin{align*}
%R &= \beta - \alpha\Be_{21} - \gamma \Be_{32} - \delta \Be_{31} \\
%\beta &= \cos\left(\frac{\theta}{2}\right) \cos\left( \inv{2}\left(\psi+\phi\right) \right) \\
%\alpha &= -\cos\left(\frac{\theta}{2}\right) \sin\left( \inv{2}\left(\psi+\phi\right) \right) \\
%\gamma &= -\sin\left(\frac{\theta}{2}\right) \cos\left( \inv{2}\left(\psi-\phi\right) \right) \\
%\delta &= -\sin\left(\frac{\theta}{2}\right) \sin\left( \inv{2}\left(\psi-\phi\right) \right) \\
%\end{align*}
%
%\begin{align*}
%\begin{bmatrix}
% - \delta^2 + \gamma^2 - \alpha^2 + \beta^2 & - 2  \gamma \delta + 2 \beta \alpha & + 2 \alpha \gamma + 2 \beta \delta \\
% - 2 \gamma \delta - 2 \beta \alpha & + \delta^2 - \gamma^2 - \alpha^2 + \beta^2 & - 2 \alpha \delta + 2 \beta \gamma \\
% + 2 \alpha \gamma - 2 \beta \delta & - 2 \alpha \delta - 2 \beta \gamma & - \delta^2 - \gamma^2 + \alpha^2 + \beta^2 \\
%\end{bmatrix}
%\end{align*}
%
%
%
%
% VERSION 8
%
%
%
%
%\begin{align*}
%R &= \beta - \alpha\Be_{21} - \delta \Be_{32} - \gamma \Be_{31} \\
%\beta &= \cos\left(\frac{\theta}{2}\right) \cos\left( \inv{2}\left(\psi+\phi\right) \right) \\
%\alpha &= -\cos\left(\frac{\theta}{2}\right) \sin\left( \inv{2}\left(\psi+\phi\right) \right) \\
%\delta &= -\sin\left(\frac{\theta}{2}\right) \cos\left( \inv{2}\left(\psi-\phi\right) \right) \\
%\gamma &= -\sin\left(\frac{\theta}{2}\right) \sin\left( \inv{2}\left(\psi-\phi\right) \right) \\
%\end{align*}
%
%\begin{align*}
%\begin{bmatrix}
% - \gamma^2 + \delta^2 - \alpha^2 + \beta^2 & - 2  \delta \gamma + 2 \beta \alpha & + 2 \alpha \delta + 2 \beta \gamma \\
% - 2 \delta \gamma - 2 \beta \alpha & + \gamma^2 - \delta^2 - \alpha^2 + \beta^2 & - 2 \alpha \gamma + 2 \beta \delta \\
% + 2 \alpha \delta - 2 \beta \gamma & - 2 \alpha \gamma - 2 \beta \delta & - \gamma^2 - \delta^2 + \alpha^2 + \beta^2 \\
%\end{bmatrix}
%\end{align*}
%
%
%
%
% VERSION 9
%
%
%
%
%\begin{align*}
%R &= \beta - \gamma\Be_{21} - \delta \Be_{32} - \alpha \Be_{31} \\
%\beta &= \cos\left(\frac{\theta}{2}\right) \cos\left( \inv{2}\left(\psi+\phi\right) \right) \\
%\gamma &= -\cos\left(\frac{\theta}{2}\right) \sin\left( \inv{2}\left(\psi+\phi\right) \right) \\
%\delta &= -\sin\left(\frac{\theta}{2}\right) \cos\left( \inv{2}\left(\psi-\phi\right) \right) \\
%\alpha &= -\sin\left(\frac{\theta}{2}\right) \sin\left( \inv{2}\left(\psi-\phi\right) \right) \\
%\end{align*}
%
%\begin{align*}
%\begin{bmatrix}
% - \alpha^2 + \delta^2 - \gamma^2 + \beta^2 & - 2  \delta \alpha + 2 \beta \gamma & + 2 \gamma \delta + 2 \beta \alpha \\
% - 2 \delta \alpha - 2 \beta \gamma & + \alpha^2 - \delta^2 - \gamma^2 + \beta^2 & - 2 \gamma \alpha + 2 \beta \delta \\
% + 2 \gamma \delta - 2 \beta \alpha & - 2 \gamma \alpha - 2 \beta \delta & - \alpha^2 - \delta^2 + \gamma^2 + \beta^2 \\
%\end{bmatrix}
%\end{align*}
%
%
%
%
% VERSION 10
%
%
%
%
%\begin{align*}
%R &= \beta - \gamma\Be_{21} - \alpha \Be_{32} - \delta \Be_{31} \\
%\beta &= \cos\left(\frac{\theta}{2}\right) \cos\left( \inv{2}\left(\psi+\phi\right) \right) \\
%\gamma &= -\cos\left(\frac{\theta}{2}\right) \sin\left( \inv{2}\left(\psi+\phi\right) \right) \\
%\alpha &= -\sin\left(\frac{\theta}{2}\right) \cos\left( \inv{2}\left(\psi-\phi\right) \right) \\
%\delta &= -\sin\left(\frac{\theta}{2}\right) \sin\left( \inv{2}\left(\psi-\phi\right) \right) \\
%\end{align*}
%
%\begin{align*}
%\begin{bmatrix}
% - \delta^2 + \alpha^2 - \gamma^2 + \beta^2 & - 2  \alpha \delta + 2 \beta \gamma & + 2 \gamma \alpha + 2 \beta \delta \\
% - 2 \alpha \delta - 2 \beta \gamma & + \delta^2 - \alpha^2 - \gamma^2 + \beta^2 & - 2 \gamma \delta + 2 \beta \alpha \\
% + 2 \gamma \alpha - 2 \beta \delta & - 2 \gamma \delta - 2 \beta \alpha & - \delta^2 - \alpha^2 + \gamma^2 + \beta^2 \\
%\end{bmatrix}
%\end{align*}
%
%
%
%
% VERSION 11
%
%
%
%
%\begin{align*}
%R &= \beta - \delta\Be_{21} - \alpha \Be_{32} - \gamma \Be_{31} \\
%\beta &= \cos\left(\frac{\theta}{2}\right) \cos\left( \inv{2}\left(\psi+\phi\right) \right) \\
%\delta &= -\cos\left(\frac{\theta}{2}\right) \sin\left( \inv{2}\left(\psi+\phi\right) \right) \\
%\alpha &= -\sin\left(\frac{\theta}{2}\right) \cos\left( \inv{2}\left(\psi-\phi\right) \right) \\
%\gamma &= -\sin\left(\frac{\theta}{2}\right) \sin\left( \inv{2}\left(\psi-\phi\right) \right) \\
%\end{align*}
%
%\begin{align*}
%\begin{bmatrix}
% - \gamma^2 + \alpha^2 - \delta^2 + \beta^2 & - 2  \alpha \gamma + 2 \beta \delta & + 2 \delta \alpha + 2 \beta \gamma \\
% - 2 \alpha \gamma - 2 \beta \delta & + \gamma^2 - \alpha^2 - \delta^2 + \beta^2 & - 2 \delta \gamma + 2 \beta \alpha \\
% + 2 \delta \alpha - 2 \beta \gamma & - 2 \delta \gamma - 2 \beta \alpha & - \gamma^2 - \alpha^2 + \delta^2 + \beta^2 \\
%\end{bmatrix}
%\end{align*}
%
%
%
%
% VERSION 12
%
%
%
%
%\begin{align*}
%R &= \beta - \delta\Be_{21} - \gamma \Be_{32} - \alpha \Be_{31} \\
%\beta &= \cos\left(\frac{\theta}{2}\right) \cos\left( \inv{2}\left(\psi+\phi\right) \right) \\
%\delta &= -\cos\left(\frac{\theta}{2}\right) \sin\left( \inv{2}\left(\psi+\phi\right) \right) \\
%\gamma &= -\sin\left(\frac{\theta}{2}\right) \cos\left( \inv{2}\left(\psi-\phi\right) \right) \\
%\alpha &= -\sin\left(\frac{\theta}{2}\right) \sin\left( \inv{2}\left(\psi-\phi\right) \right) \\
%\end{align*}
%
%\begin{align*}
%\begin{bmatrix}
% - \alpha^2 + \gamma^2 - \delta^2 + \beta^2 & - 2  \gamma \alpha + 2 \beta \delta & + 2 \delta \gamma + 2 \beta \alpha \\
% - 2 \gamma \alpha - 2 \beta \delta & + \alpha^2 - \gamma^2 - \delta^2 + \beta^2 & - 2 \delta \alpha + 2 \beta \gamma \\
% + 2 \delta \gamma - 2 \beta \alpha & - 2 \delta \alpha - 2 \beta \gamma & - \alpha^2 - \gamma^2 + \delta^2 + \beta^2 \\
%\end{bmatrix}
%\end{align*}
%
%
%
%
% VERSION 13
%
%
%
%
%\begin{align*}
%R &= \gamma - \beta\Be_{21} - \alpha \Be_{32} - \delta \Be_{31} \\
%\gamma &= \cos\left(\frac{\theta}{2}\right) \cos\left( \inv{2}\left(\psi+\phi\right) \right) \\
%\beta &= -\cos\left(\frac{\theta}{2}\right) \sin\left( \inv{2}\left(\psi+\phi\right) \right) \\
%\alpha &= -\sin\left(\frac{\theta}{2}\right) \cos\left( \inv{2}\left(\psi-\phi\right) \right) \\
%\delta &= -\sin\left(\frac{\theta}{2}\right) \sin\left( \inv{2}\left(\psi-\phi\right) \right) \\
%\end{align*}
%
%\begin{align*}
%\begin{bmatrix}
% - \delta^2 + \alpha^2 - \beta^2 + \gamma^2 & - 2  \alpha \delta + 2 \gamma \beta & + 2 \beta \alpha + 2 \gamma \delta \\
% - 2 \alpha \delta - 2 \gamma \beta & + \delta^2 - \alpha^2 - \beta^2 + \gamma^2 & - 2 \beta \delta + 2 \gamma \alpha \\
% + 2 \beta \alpha - 2 \gamma \delta & - 2 \beta \delta - 2 \gamma \alpha & - \delta^2 - \alpha^2 + \beta^2 + \gamma^2 \\
%\end{bmatrix}
%\end{align*}
%
%
%
%
% VERSION 14
%
%
%
%
%\begin{align*}
%R &= \gamma - \beta\Be_{21} - \delta \Be_{32} - \alpha \Be_{31} \\
%\gamma &= \cos\left(\frac{\theta}{2}\right) \cos\left( \inv{2}\left(\psi+\phi\right) \right) \\
%\beta &= -\cos\left(\frac{\theta}{2}\right) \sin\left( \inv{2}\left(\psi+\phi\right) \right) \\
%\delta &= -\sin\left(\frac{\theta}{2}\right) \cos\left( \inv{2}\left(\psi-\phi\right) \right) \\
%\alpha &= -\sin\left(\frac{\theta}{2}\right) \sin\left( \inv{2}\left(\psi-\phi\right) \right) \\
%\end{align*}
%
%\begin{align*}
%-
%\begin{bmatrix}
% + \alpha^2 - \delta^2 + \beta^2 - \gamma^2 & + 2  \delta \alpha - 2 \gamma \beta & - 2 \beta \delta - 2 \gamma \alpha \\
% + 2 \delta \alpha + 2 \gamma \beta & - \alpha^2 + \delta^2 + \beta^2 - \gamma^2 & + 2 \beta \alpha - 2 \gamma \delta \\
% - 2 \beta \delta + 2 \gamma \alpha & + 2 \beta \alpha + 2 \gamma \delta & + \alpha^2 + \delta^2 - \beta^2 - \gamma^2 \\
%\end{bmatrix}
%\end{align*}
%
%
%
%
% VERSION 15
%
%
%
%
%\begin{align*}
%R &= \gamma - \alpha\Be_{21} - \delta \Be_{32} - \beta \Be_{31} \\
%\gamma &= \cos\left(\frac{\theta}{2}\right) \cos\left( \inv{2}\left(\psi+\phi\right) \right) \\
%\alpha &= -\cos\left(\frac{\theta}{2}\right) \sin\left( \inv{2}\left(\psi+\phi\right) \right) \\
%\delta &= -\sin\left(\frac{\theta}{2}\right) \cos\left( \inv{2}\left(\psi-\phi\right) \right) \\
%\beta &= -\sin\left(\frac{\theta}{2}\right) \sin\left( \inv{2}\left(\psi-\phi\right) \right) \\
%\end{align*}
%
%\begin{align*}
%-
%\begin{bmatrix}
% + \beta^2 - \delta^2 + \alpha^2 - \gamma^2 & + 2  \delta \beta - 2 \gamma \alpha & - 2 \alpha \delta - 2 \gamma \beta \\
% + 2 \delta \beta + 2 \gamma \alpha & - \beta^2 + \delta^2 + \alpha^2 - \gamma^2 & + 2 \alpha \beta - 2 \gamma \delta \\
% - 2 \alpha \delta + 2 \gamma \beta & + 2 \alpha \beta + 2 \gamma \delta & + \beta^2 + \delta^2 - \alpha^2 - \gamma^2 \\
%\end{bmatrix}
%\end{align*}
%
%
%
%
% VERSION 16
%
%
%
%
%\begin{align*}
%R &= \gamma - \alpha\Be_{21} - \beta \Be_{32} - \delta \Be_{31} \\
%\gamma &= \cos\left(\frac{\theta}{2}\right) \cos\left( \inv{2}\left(\psi+\phi\right) \right) \\
%\alpha &= -\cos\left(\frac{\theta}{2}\right) \sin\left( \inv{2}\left(\psi+\phi\right) \right) \\
%\beta &= -\sin\left(\frac{\theta}{2}\right) \cos\left( \inv{2}\left(\psi-\phi\right) \right) \\
%\delta &= -\sin\left(\frac{\theta}{2}\right) \sin\left( \inv{2}\left(\psi-\phi\right) \right) \\
%\end{align*}
%
%\begin{align*}
%\begin{bmatrix}
% - \delta^2 + \beta^2 - \alpha^2 + \gamma^2 & - 2  \beta \delta + 2 \gamma \alpha & + 2 \alpha \beta + 2 \gamma \delta \\
% - 2 \beta \delta - 2 \gamma \alpha & + \delta^2 - \beta^2 - \alpha^2 + \gamma^2 & - 2 \alpha \delta + 2 \gamma \beta \\
% + 2 \alpha \beta - 2 \gamma \delta & - 2 \alpha \delta - 2 \gamma \beta & - \delta^2 - \beta^2 + \alpha^2 + \gamma^2 \\
%\end{bmatrix}
%\end{align*}
%
%
%
%
% VERSION 17
%
%
%
%
%\begin{align*}
%R &= \gamma - \delta\Be_{21} - \beta \Be_{32} - \alpha \Be_{31} \\
%\gamma &= \cos\left(\frac{\theta}{2}\right) \cos\left( \inv{2}\left(\psi+\phi\right) \right) \\
%\delta &= -\cos\left(\frac{\theta}{2}\right) \sin\left( \inv{2}\left(\psi+\phi\right) \right) \\
%\beta &= -\sin\left(\frac{\theta}{2}\right) \cos\left( \inv{2}\left(\psi-\phi\right) \right) \\
%\alpha &= -\sin\left(\frac{\theta}{2}\right) \sin\left( \inv{2}\left(\psi-\phi\right) \right) \\
%\end{align*}
%
%\begin{align*}
%\begin{bmatrix}
% - \alpha^2 + \beta^2 - \delta^2 + \gamma^2 & - 2  \beta \alpha + 2 \gamma \delta & + 2 \delta \beta + 2 \gamma \alpha \\
% - 2 \beta \alpha - 2 \gamma \delta & + \alpha^2 - \beta^2 - \delta^2 + \gamma^2 & - 2 \delta \alpha + 2 \gamma \beta \\
% + 2 \delta \beta - 2 \gamma \alpha & - 2 \delta \alpha - 2 \gamma \beta & - \alpha^2 - \beta^2 + \delta^2 + \gamma^2 \\
%\end{bmatrix}
%\end{align*}
%
%
%
%
% VERSION 18
%
%
%
%
%\begin{align*}
%R &= \gamma - \delta\Be_{21} - \alpha \Be_{32} - \beta \Be_{31} \\
%\gamma &= \cos\left(\frac{\theta}{2}\right) \cos\left( \inv{2}\left(\psi+\phi\right) \right) \\
%\delta &= -\cos\left(\frac{\theta}{2}\right) \sin\left( \inv{2}\left(\psi+\phi\right) \right) \\
%\alpha &= -\sin\left(\frac{\theta}{2}\right) \cos\left( \inv{2}\left(\psi-\phi\right) \right) \\
%\beta &= -\sin\left(\frac{\theta}{2}\right) \sin\left( \inv{2}\left(\psi-\phi\right) \right) \\
%\end{align*}
%
%\begin{align*}
%\begin{bmatrix}
% - \beta^2 + \alpha^2 - \delta^2 + \gamma^2 & - 2  \alpha \beta + 2 \gamma \delta & + 2 \delta \alpha + 2 \gamma \beta \\
% - 2 \alpha \beta - 2 \gamma \delta & + \beta^2 - \alpha^2 - \delta^2 + \gamma^2 & - 2 \delta \beta + 2 \gamma \alpha \\
% + 2 \delta \alpha - 2 \gamma \beta & - 2 \delta \beta - 2 \gamma \alpha & - \beta^2 - \alpha^2 + \delta^2 + \gamma^2 \\
%\end{bmatrix}
%\end{align*}
%
%
%
%
% VERSION 19
%
%
%
%
%\begin{align*}
%R &= \delta - \beta\Be_{21} - \gamma \Be_{32} - \alpha \Be_{31} \\
%\delta &= \cos\left(\frac{\theta}{2}\right) \cos\left( \inv{2}\left(\psi+\phi\right) \right) \\
%\beta &= -\cos\left(\frac{\theta}{2}\right) \sin\left( \inv{2}\left(\psi+\phi\right) \right) \\
%\gamma &= -\sin\left(\frac{\theta}{2}\right) \cos\left( \inv{2}\left(\psi-\phi\right) \right) \\
%\alpha &= -\sin\left(\frac{\theta}{2}\right) \sin\left( \inv{2}\left(\psi-\phi\right) \right) \\
%\end{align*}
%
%\begin{align*}
%\begin{bmatrix}
% - \alpha^2 + \gamma^2 - \beta^2 + \delta^2 & - 2  \gamma \alpha + 2 \delta \beta & + 2 \beta \gamma + 2 \delta \alpha \\
% - 2 \gamma \alpha - 2 \delta \beta & + \alpha^2 - \gamma^2 - \beta^2 + \delta^2 & - 2 \beta \alpha + 2 \delta \gamma \\
% + 2 \beta \gamma - 2 \delta \alpha & - 2 \beta \alpha - 2 \delta \gamma & - \alpha^2 - \gamma^2 + \beta^2 + \delta^2 \\
%\end{bmatrix}
%\end{align*}
%
%
%
%
% VERSION 20
%
%
%
%
%\begin{align*}
%R &= \delta - \beta\Be_{21} - \alpha \Be_{32} - \gamma \Be_{31} \\
%\delta &= \cos\left(\frac{\theta}{2}\right) \cos\left( \inv{2}\left(\psi+\phi\right) \right) \\
%\beta &= -\cos\left(\frac{\theta}{2}\right) \sin\left( \inv{2}\left(\psi+\phi\right) \right) \\
%\alpha &= -\sin\left(\frac{\theta}{2}\right) \cos\left( \inv{2}\left(\psi-\phi\right) \right) \\
%\gamma &= -\sin\left(\frac{\theta}{2}\right) \sin\left( \inv{2}\left(\psi-\phi\right) \right) \\
%\end{align*}
%
%\begin{align*}
%\begin{bmatrix}
% - \gamma^2 + \alpha^2 - \beta^2 + \delta^2 & - 2  \alpha \gamma + 2 \delta \beta & + 2 \beta \alpha + 2 \delta \gamma \\
% - 2 \alpha \gamma - 2 \delta \beta & + \gamma^2 - \alpha^2 - \beta^2 + \delta^2 & - 2 \beta \gamma + 2 \delta \alpha \\
% + 2 \beta \alpha - 2 \delta \gamma & - 2 \beta \gamma - 2 \delta \alpha & - \gamma^2 - \alpha^2 + \beta^2 + \delta^2 \\
%\end{bmatrix}
%\end{align*}
%
%
%
%
% VERSION 21
%
%
%
%
%\begin{align*}
%R &= \delta - \gamma\Be_{21} - \alpha \Be_{32} - \beta \Be_{31} \\
%\delta &= \cos\left(\frac{\theta}{2}\right) \cos\left( \inv{2}\left(\psi+\phi\right) \right) \\
%\gamma &= -\cos\left(\frac{\theta}{2}\right) \sin\left( \inv{2}\left(\psi+\phi\right) \right) \\
%\alpha &= -\sin\left(\frac{\theta}{2}\right) \cos\left( \inv{2}\left(\psi-\phi\right) \right) \\
%\beta &= -\sin\left(\frac{\theta}{2}\right) \sin\left( \inv{2}\left(\psi-\phi\right) \right) \\
%\end{align*}
%
%\begin{align*}
%\begin{bmatrix}
% - \beta^2 + \alpha^2 - \gamma^2 + \delta^2 & - 2  \alpha \beta + 2 \delta \gamma & + 2 \gamma \alpha + 2 \delta \beta \\
% - 2 \alpha \beta - 2 \delta \gamma & + \beta^2 - \alpha^2 - \gamma^2 + \delta^2 & - 2 \gamma \beta + 2 \delta \alpha \\
% + 2 \gamma \alpha - 2 \delta \beta & - 2 \gamma \beta - 2 \delta \alpha & - \beta^2 - \alpha^2 + \gamma^2 + \delta^2 \\
%\end{bmatrix}
%\end{align*}
%
%
%
%
% VERSION 22
%
%
%
%
%\begin{align*}
%R &= \delta - \gamma\Be_{21} - \beta \Be_{32} - \alpha \Be_{31} \\
%\delta &= \cos\left(\frac{\theta}{2}\right) \cos\left( \inv{2}\left(\psi+\phi\right) \right) \\
%\gamma &= -\cos\left(\frac{\theta}{2}\right) \sin\left( \inv{2}\left(\psi+\phi\right) \right) \\
%\beta &= -\sin\left(\frac{\theta}{2}\right) \cos\left( \inv{2}\left(\psi-\phi\right) \right) \\
%\alpha &= -\sin\left(\frac{\theta}{2}\right) \sin\left( \inv{2}\left(\psi-\phi\right) \right) \\
%\end{align*}
%
%\begin{align*}
%\begin{bmatrix}
% - \alpha^2 + \beta^2 - \gamma^2 + \delta^2 & - 2  \beta \alpha + 2 \delta \gamma & + 2 \gamma \beta + 2 \delta \alpha \\
% - 2 \beta \alpha - 2 \delta \gamma & + \alpha^2 - \beta^2 - \gamma^2 + \delta^2 & - 2 \gamma \alpha + 2 \delta \beta \\
% + 2 \gamma \beta - 2 \delta \alpha & - 2 \gamma \alpha - 2 \delta \beta & - \alpha^2 - \beta^2 + \gamma^2 + \delta^2 \\
%\end{bmatrix}
%\end{align*}
%
%
%
%
% VERSION 23
%
%
%
%
%\begin{align*}
%R &= \delta - \alpha\Be_{21} - \beta \Be_{32} - \gamma \Be_{31} \\
%\delta &= \cos\left(\frac{\theta}{2}\right) \cos\left( \inv{2}\left(\psi+\phi\right) \right) \\
%\alpha &= -\cos\left(\frac{\theta}{2}\right) \sin\left( \inv{2}\left(\psi+\phi\right) \right) \\
%\beta &= -\sin\left(\frac{\theta}{2}\right) \cos\left( \inv{2}\left(\psi-\phi\right) \right) \\
%\gamma &= -\sin\left(\frac{\theta}{2}\right) \sin\left( \inv{2}\left(\psi-\phi\right) \right) \\
%\end{align*}
%
%\begin{align*}
%\begin{bmatrix}
% - \gamma^2 + \beta^2 - \alpha^2 + \delta^2 & - 2  \beta \gamma + 2 \delta \alpha & + 2 \alpha \beta + 2 \delta \gamma \\
% - 2 \beta \gamma - 2 \delta \alpha & + \gamma^2 - \beta^2 - \alpha^2 + \delta^2 & - 2 \alpha \gamma + 2 \delta \beta \\
% + 2 \alpha \beta - 2 \delta \gamma & - 2 \alpha \gamma - 2 \delta \beta & - \gamma^2 - \beta^2 + \alpha^2 + \delta^2 \\
%\end{bmatrix}
%\end{align*}
%
%
%
%
% VERSION 24
%
%
%
%
%\begin{align*}
%R &= \delta - \alpha\Be_{21} - \gamma \Be_{32} - \beta \Be_{31} \\
%\delta &= \cos\left(\frac{\theta}{2}\right) \cos\left( \inv{2}\left(\psi+\phi\right) \right) \\
%\alpha &= -\cos\left(\frac{\theta}{2}\right) \sin\left( \inv{2}\left(\psi+\phi\right) \right) \\
%\gamma &= -\sin\left(\frac{\theta}{2}\right) \cos\left( \inv{2}\left(\psi-\phi\right) \right) \\
%\beta &= -\sin\left(\frac{\theta}{2}\right) \sin\left( \inv{2}\left(\psi-\phi\right) \right) \\
%\end{align*}
%
%\begin{align*}
%\begin{bmatrix}
% - \beta^2 + \gamma^2 - \alpha^2 + \delta^2 & - 2  \gamma \beta + 2 \delta \alpha & + 2 \alpha \gamma + 2 \delta \beta \\
% - 2 \gamma \beta - 2 \delta \alpha & + \beta^2 - \gamma^2 - \alpha^2 + \delta^2 & - 2 \alpha \beta + 2 \delta \gamma \\
% + 2 \alpha \gamma - 2 \delta \beta & - 2 \alpha \beta - 2 \delta \gamma & - \beta^2 - \gamma^2 + \alpha^2 + \delta^2 \\
%\end{bmatrix}
%\end{align*}

%\subsection{Scripted output.  All parameterization variations}
%
%
%
%
%
%
% VERSION 1
%
%
%
%
%\begin{align*}
%R &= \alpha + \gamma\Be_{21} - \beta \Be_{32} - \delta \Be_{31} \\
%\alpha &= \cos\left(\frac{\theta}{2}\right) \cos\left( \inv{2}\left(\psi+\phi\right) \right) \\
%\gamma &= \cos\left(\frac{\theta}{2}\right) \sin\left( \inv{2}\left(\psi+\phi\right) \right) \\
%\beta &= -\sin\left(\frac{\theta}{2}\right) \cos\left( \inv{2}\left(\psi-\phi\right) \right) \\
%\delta &= -\sin\left(\frac{\theta}{2}\right) \sin\left( \inv{2}\left(\psi-\phi\right) \right) \\
%\end{align*}
%
%\begin{align*}
%\begin{bmatrix}
% - \delta^2 + \beta^2 - \gamma^2 + \alpha^2 & - 2 \beta \delta - 2 \alpha \gamma & - 2 \gamma \beta + 2 \alpha \delta \\
% - 2 \beta \delta + 2 \alpha \gamma & + \delta^2 - \beta^2 - \gamma^2 + \alpha^2 & + 2 \gamma \delta + 2 \alpha \beta \\
% - 2 \gamma \beta - 2 \alpha \delta & + 2 \gamma \delta - 2 \alpha \beta & - \delta^2 - \beta^2 + \gamma^2 + \alpha^2 \\
%\end{bmatrix}
%\end{align*}
%
%
%
%
% VERSION 2
%
%
%
%
%\begin{align*}
%R &= \alpha + \delta\Be_{21} - \beta \Be_{32} - \gamma \Be_{31} \\
%\alpha &= \cos\left(\frac{\theta}{2}\right) \cos\left( \inv{2}\left(\psi+\phi\right) \right) \\
%\delta &= \cos\left(\frac{\theta}{2}\right) \sin\left( \inv{2}\left(\psi+\phi\right) \right) \\
%\beta &= -\sin\left(\frac{\theta}{2}\right) \cos\left( \inv{2}\left(\psi-\phi\right) \right) \\
%\gamma &= -\sin\left(\frac{\theta}{2}\right) \sin\left( \inv{2}\left(\psi-\phi\right) \right) \\
%\end{align*}
%
%
%\begin{align*}
%\begin{bmatrix}
% - \gamma^2 + \beta^2 - \delta^2 + \alpha^2 & - 2 \beta \gamma - 2 \alpha \delta & - 2 \delta \beta + 2 \alpha \gamma \\
% - 2 \beta \gamma + 2 \alpha \delta & + \gamma^2 - \beta^2 - \delta^2 + \alpha^2 & + 2 \delta \gamma + 2 \alpha \beta \\
% - 2 \delta \beta - 2 \alpha \gamma & + 2 \delta \gamma - 2 \alpha \beta & - \gamma^2 - \beta^2 + \delta^2 + \alpha^2 \\
%\end{bmatrix}
%\end{align*}
%
%
%
%
% VERSION 3
%
%
%
%
%\begin{align*}
%R &= \alpha + \delta\Be_{21} - \gamma \Be_{32} - \beta \Be_{31} \\
%\alpha &= \cos\left(\frac{\theta}{2}\right) \cos\left( \inv{2}\left(\psi+\phi\right) \right) \\
%\delta &= \cos\left(\frac{\theta}{2}\right) \sin\left( \inv{2}\left(\psi+\phi\right) \right) \\
%\gamma &= -\sin\left(\frac{\theta}{2}\right) \cos\left( \inv{2}\left(\psi-\phi\right) \right) \\
%\beta &= -\sin\left(\frac{\theta}{2}\right) \sin\left( \inv{2}\left(\psi-\phi\right) \right) \\
%\end{align*}
%
%
%\begin{align*}
%\begin{bmatrix}
% - \beta^2 + \gamma^2 - \delta^2 + \alpha^2 & - 2 \gamma \beta - 2 \alpha \delta & - 2 \delta \gamma + 2 \alpha \beta \\
% - 2 \gamma \beta + 2 \alpha \delta & + \beta^2 - \gamma^2 - \delta^2 + \alpha^2 & + 2 \delta \beta + 2 \alpha \gamma \\
% - 2 \delta \gamma - 2 \alpha \beta & + 2 \delta \beta - 2 \alpha \gamma & - \beta^2 - \gamma^2 + \delta^2 + \alpha^2 \\
%\end{bmatrix}
%\end{align*}
%
%
%
%
% VERSION 4
%
%
%
%
%\begin{align*}
%R &= \alpha + \beta\Be_{21} - \gamma \Be_{32} - \delta \Be_{31} \\
%\alpha &= \cos\left(\frac{\theta}{2}\right) \cos\left( \inv{2}\left(\psi+\phi\right) \right) \\
%\beta &= \cos\left(\frac{\theta}{2}\right) \sin\left( \inv{2}\left(\psi+\phi\right) \right) \\
%\gamma &= -\sin\left(\frac{\theta}{2}\right) \cos\left( \inv{2}\left(\psi-\phi\right) \right) \\
%\delta &= -\sin\left(\frac{\theta}{2}\right) \sin\left( \inv{2}\left(\psi-\phi\right) \right) \\
%\end{align*}
%
%
%\begin{align*}
%\begin{bmatrix}
% - \delta^2 + \gamma^2 - \beta^2 + \alpha^2 & - 2 \gamma \delta - 2 \alpha \beta & - 2 \beta \gamma + 2 \alpha \delta \\
% - 2 \gamma \delta + 2 \alpha \beta & + \delta^2 - \gamma^2 - \beta^2 + \alpha^2 & + 2 \beta \delta + 2 \alpha \gamma \\
% - 2 \beta \gamma - 2 \alpha \delta & + 2 \beta \delta - 2 \alpha \gamma & - \delta^2 - \gamma^2 + \beta^2 + \alpha^2 \\
%\end{bmatrix}
%\end{align*}
%
%
%
%
% VERSION 5
%
%
%
%
%\begin{align*}
%R &= \alpha + \beta\Be_{21} - \delta \Be_{32} - \gamma \Be_{31} \\
%\alpha &= \cos\left(\frac{\theta}{2}\right) \cos\left( \inv{2}\left(\psi+\phi\right) \right) \\
%\beta &= \cos\left(\frac{\theta}{2}\right) \sin\left( \inv{2}\left(\psi+\phi\right) \right) \\
%\delta &= -\sin\left(\frac{\theta}{2}\right) \cos\left( \inv{2}\left(\psi-\phi\right) \right) \\
%\gamma &= -\sin\left(\frac{\theta}{2}\right) \sin\left( \inv{2}\left(\psi-\phi\right) \right) \\
%\end{align*}
%
%
%\begin{align*}
%\begin{bmatrix}
% - \gamma^2 + \delta^2 - \beta^2 + \alpha^2 & - 2 \delta \gamma - 2 \alpha \beta & - 2 \beta \delta + 2 \alpha \gamma \\
% - 2 \delta \gamma + 2 \alpha \beta & + \gamma^2 - \delta^2 - \beta^2 + \alpha^2 & + 2 \beta \gamma + 2 \alpha \delta \\
% - 2 \beta \delta - 2 \alpha \gamma & + 2 \beta \gamma - 2 \alpha \delta & - \gamma^2 - \delta^2 + \beta^2 + \alpha^2 \\
%\end{bmatrix}
%\end{align*}
%
%
%
%
% VERSION 6
%
%
%
%
%\begin{align*}
%R &= \alpha + \gamma\Be_{21} - \delta \Be_{32} - \beta \Be_{31} \\
%\alpha &= \cos\left(\frac{\theta}{2}\right) \cos\left( \inv{2}\left(\psi+\phi\right) \right) \\
%\gamma &= \cos\left(\frac{\theta}{2}\right) \sin\left( \inv{2}\left(\psi+\phi\right) \right) \\
%\delta &= -\sin\left(\frac{\theta}{2}\right) \cos\left( \inv{2}\left(\psi-\phi\right) \right) \\
%\beta &= -\sin\left(\frac{\theta}{2}\right) \sin\left( \inv{2}\left(\psi-\phi\right) \right) \\
%\end{align*}
%
%
%\begin{align*}
%\begin{bmatrix}
% - \beta^2 + \delta^2 - \gamma^2 + \alpha^2 & - 2 \delta \beta - 2 \alpha \gamma & - 2 \gamma \delta + 2 \alpha \beta \\
% - 2 \delta \beta + 2 \alpha \gamma & + \beta^2 - \delta^2 - \gamma^2 + \alpha^2 & + 2 \gamma \beta + 2 \alpha \delta \\
% - 2 \gamma \delta - 2 \alpha \beta & + 2 \gamma \beta - 2 \alpha \delta & - \beta^2 - \delta^2 + \gamma^2 + \alpha^2 \\
%\end{bmatrix}
%\end{align*}
%
%
%
%
% VERSION 7
%
%
%
%
%\begin{align*}
%R &= \beta + \gamma\Be_{21} - \alpha \Be_{32} - \delta \Be_{31} \\
%\beta &= \cos\left(\frac{\theta}{2}\right) \cos\left( \inv{2}\left(\psi+\phi\right) \right) \\
%\gamma &= \cos\left(\frac{\theta}{2}\right) \sin\left( \inv{2}\left(\psi+\phi\right) \right) \\
%\alpha &= -\sin\left(\frac{\theta}{2}\right) \cos\left( \inv{2}\left(\psi-\phi\right) \right) \\
%\delta &= -\sin\left(\frac{\theta}{2}\right) \sin\left( \inv{2}\left(\psi-\phi\right) \right) \\
%\end{align*}
%
%
%\begin{align*}
%\begin{bmatrix}
% - \delta^2 + \alpha^2 - \gamma^2 + \beta^2 & - 2 \alpha \delta - 2 \beta \gamma & - 2 \gamma \alpha + 2 \beta \delta \\
% - 2 \alpha \delta + 2 \beta \gamma & + \delta^2 - \alpha^2 - \gamma^2 + \beta^2 & + 2 \gamma \delta + 2 \beta \alpha \\
% - 2 \gamma \alpha - 2 \beta \delta & + 2 \gamma \delta - 2 \beta \alpha & - \delta^2 - \alpha^2 + \gamma^2 + \beta^2 \\
%\end{bmatrix}
%\end{align*}
%
%
%
%
% VERSION 8
%
%
%
%
%\begin{align*}
%R &= \beta + \delta\Be_{21} - \alpha \Be_{32} - \gamma \Be_{31} \\
%\beta &= \cos\left(\frac{\theta}{2}\right) \cos\left( \inv{2}\left(\psi+\phi\right) \right) \\
%\delta &= \cos\left(\frac{\theta}{2}\right) \sin\left( \inv{2}\left(\psi+\phi\right) \right) \\
%\alpha &= -\sin\left(\frac{\theta}{2}\right) \cos\left( \inv{2}\left(\psi-\phi\right) \right) \\
%\gamma &= -\sin\left(\frac{\theta}{2}\right) \sin\left( \inv{2}\left(\psi-\phi\right) \right) \\
%\end{align*}
%
%
%\begin{align*}
%\begin{bmatrix}
% - \gamma^2 + \alpha^2 - \delta^2 + \beta^2 & - 2 \alpha \gamma - 2 \beta \delta & - 2 \delta \alpha + 2 \beta \gamma \\
% - 2 \alpha \gamma + 2 \beta \delta & + \gamma^2 - \alpha^2 - \delta^2 + \beta^2 & + 2 \delta \gamma + 2 \beta \alpha \\
% - 2 \delta \alpha - 2 \beta \gamma & + 2 \delta \gamma - 2 \beta \alpha & - \gamma^2 - \alpha^2 + \delta^2 + \beta^2 \\
%\end{bmatrix}
%\end{align*}
%
%
%
%
% VERSION 9
%
%
%
%
%\begin{align*}
%R &= \beta + \delta\Be_{21} - \gamma \Be_{32} - \alpha \Be_{31} \\
%\beta &= \cos\left(\frac{\theta}{2}\right) \cos\left( \inv{2}\left(\psi+\phi\right) \right) \\
%\delta &= \cos\left(\frac{\theta}{2}\right) \sin\left( \inv{2}\left(\psi+\phi\right) \right) \\
%\gamma &= -\sin\left(\frac{\theta}{2}\right) \cos\left( \inv{2}\left(\psi-\phi\right) \right) \\
%\alpha &= -\sin\left(\frac{\theta}{2}\right) \sin\left( \inv{2}\left(\psi-\phi\right) \right) \\
%\end{align*}
%
%
%\begin{align*}
%\begin{bmatrix}
% - \alpha^2 + \gamma^2 - \delta^2 + \beta^2 & - 2 \gamma \alpha - 2 \beta \delta & - 2 \delta \gamma + 2 \beta \alpha \\
% - 2 \gamma \alpha + 2 \beta \delta & + \alpha^2 - \gamma^2 - \delta^2 + \beta^2 & + 2 \delta \alpha + 2 \beta \gamma \\
% - 2 \delta \gamma - 2 \beta \alpha & + 2 \delta \alpha - 2 \beta \gamma & - \alpha^2 - \gamma^2 + \delta^2 + \beta^2 \\
%\end{bmatrix}
%\end{align*}
%
%
%
%
% VERSION 10
%
%
%
%
%\begin{align*}
%R &= \beta + \alpha\Be_{21} - \gamma \Be_{32} - \delta \Be_{31} \\
%\beta &= \cos\left(\frac{\theta}{2}\right) \cos\left( \inv{2}\left(\psi+\phi\right) \right) \\
%\alpha &= \cos\left(\frac{\theta}{2}\right) \sin\left( \inv{2}\left(\psi+\phi\right) \right) \\
%\gamma &= -\sin\left(\frac{\theta}{2}\right) \cos\left( \inv{2}\left(\psi-\phi\right) \right) \\
%\delta &= -\sin\left(\frac{\theta}{2}\right) \sin\left( \inv{2}\left(\psi-\phi\right) \right) \\
%\end{align*}
%
%
%\begin{align*}
%\begin{bmatrix}
% - \delta^2 + \gamma^2 - \alpha^2 + \beta^2 & - 2 \gamma \delta - 2 \beta \alpha & - 2 \alpha \gamma + 2 \beta \delta \\
% - 2 \gamma \delta + 2 \beta \alpha & + \delta^2 - \gamma^2 - \alpha^2 + \beta^2 & + 2 \alpha \delta + 2 \beta \gamma \\
% - 2 \alpha \gamma - 2 \beta \delta & + 2 \alpha \delta - 2 \beta \gamma & - \delta^2 - \gamma^2 + \alpha^2 + \beta^2 \\
%\end{bmatrix}
%\end{align*}
%
%
%
%
% VERSION 11
%
%
%
%
%\begin{align*}
%R &= \beta + \alpha\Be_{21} - \delta \Be_{32} - \gamma \Be_{31} \\
%\beta &= \cos\left(\frac{\theta}{2}\right) \cos\left( \inv{2}\left(\psi+\phi\right) \right) \\
%\alpha &= \cos\left(\frac{\theta}{2}\right) \sin\left( \inv{2}\left(\psi+\phi\right) \right) \\
%\delta &= -\sin\left(\frac{\theta}{2}\right) \cos\left( \inv{2}\left(\psi-\phi\right) \right) \\
%\gamma &= -\sin\left(\frac{\theta}{2}\right) \sin\left( \inv{2}\left(\psi-\phi\right) \right) \\
%\end{align*}
%
%
%\begin{align*}
%\begin{bmatrix}
% - \gamma^2 + \delta^2 - \alpha^2 + \beta^2 & - 2 \delta \gamma - 2 \beta \alpha & - 2 \alpha \delta + 2 \beta \gamma \\
% - 2 \delta \gamma + 2 \beta \alpha & + \gamma^2 - \delta^2 - \alpha^2 + \beta^2 & + 2 \alpha \gamma + 2 \beta \delta \\
% - 2 \alpha \delta - 2 \beta \gamma & + 2 \alpha \gamma - 2 \beta \delta & - \gamma^2 - \delta^2 + \alpha^2 + \beta^2 \\
%\end{bmatrix}
%\end{align*}
%
%
%
%
% VERSION 12
%
%
%
%
%\begin{align*}
%R &= \beta + \gamma\Be_{21} - \delta \Be_{32} - \alpha \Be_{31} \\
%\beta &= \cos\left(\frac{\theta}{2}\right) \cos\left( \inv{2}\left(\psi+\phi\right) \right) \\
%\gamma &= \cos\left(\frac{\theta}{2}\right) \sin\left( \inv{2}\left(\psi+\phi\right) \right) \\
%\delta &= -\sin\left(\frac{\theta}{2}\right) \cos\left( \inv{2}\left(\psi-\phi\right) \right) \\
%\alpha &= -\sin\left(\frac{\theta}{2}\right) \sin\left( \inv{2}\left(\psi-\phi\right) \right) \\
%\end{align*}
%
%
%\begin{align*}
%\begin{bmatrix}
% - \alpha^2 + \delta^2 - \gamma^2 + \beta^2 & - 2 \delta \alpha - 2 \beta \gamma & - 2 \gamma \delta + 2 \beta \alpha \\
% - 2 \delta \alpha + 2 \beta \gamma & + \alpha^2 - \delta^2 - \gamma^2 + \beta^2 & + 2 \gamma \alpha + 2 \beta \delta \\
% - 2 \gamma \delta - 2 \beta \alpha & + 2 \gamma \alpha - 2 \beta \delta & - \alpha^2 - \delta^2 + \gamma^2 + \beta^2 \\
%\end{bmatrix}
%\end{align*}
%
%
%
%
% VERSION 13
%
%
%
%
%\begin{align*}
%R &= \gamma + \alpha\Be_{21} - \beta \Be_{32} - \delta \Be_{31} \\
%\gamma &= \cos\left(\frac{\theta}{2}\right) \cos\left( \inv{2}\left(\psi+\phi\right) \right) \\
%\alpha &= \cos\left(\frac{\theta}{2}\right) \sin\left( \inv{2}\left(\psi+\phi\right) \right) \\
%\beta &= -\sin\left(\frac{\theta}{2}\right) \cos\left( \inv{2}\left(\psi-\phi\right) \right) \\
%\delta &= -\sin\left(\frac{\theta}{2}\right) \sin\left( \inv{2}\left(\psi-\phi\right) \right) \\
%\end{align*}
%
%
%\begin{align*}
%\begin{bmatrix}
% - \delta^2 + \beta^2 - \alpha^2 + \gamma^2 & - 2 \beta \delta - 2 \gamma \alpha & - 2 \alpha \beta + 2 \gamma \delta \\
% - 2 \beta \delta + 2 \gamma \alpha & + \delta^2 - \beta^2 - \alpha^2 + \gamma^2 & + 2 \alpha \delta + 2 \gamma \beta \\
% - 2 \alpha \beta - 2 \gamma \delta & + 2 \alpha \delta - 2 \gamma \beta & - \delta^2 - \beta^2 + \alpha^2 + \gamma^2 \\
%\end{bmatrix}
%\end{align*}
%
%
%
%
% VERSION 14
%
%
%
%
%\begin{align*}
%R &= \gamma + \delta\Be_{21} - \beta \Be_{32} - \alpha \Be_{31} \\
%\gamma &= \cos\left(\frac{\theta}{2}\right) \cos\left( \inv{2}\left(\psi+\phi\right) \right) \\
%\delta &= \cos\left(\frac{\theta}{2}\right) \sin\left( \inv{2}\left(\psi+\phi\right) \right) \\
%\beta &= -\sin\left(\frac{\theta}{2}\right) \cos\left( \inv{2}\left(\psi-\phi\right) \right) \\
%\alpha &= -\sin\left(\frac{\theta}{2}\right) \sin\left( \inv{2}\left(\psi-\phi\right) \right) \\
%\end{align*}
%
%
%\begin{align*}
%\begin{bmatrix}
% - \alpha^2 + \beta^2 - \delta^2 + \gamma^2 & - 2 \beta \alpha - 2 \gamma \delta & - 2 \delta \beta + 2 \gamma \alpha \\
% - 2 \beta \alpha + 2 \gamma \delta & + \alpha^2 - \beta^2 - \delta^2 + \gamma^2 & + 2 \delta \alpha + 2 \gamma \beta \\
% - 2 \delta \beta - 2 \gamma \alpha & + 2 \delta \alpha - 2 \gamma \beta & - \alpha^2 - \beta^2 + \delta^2 + \gamma^2 \\
%\end{bmatrix}
%\end{align*}
%
%
%
%
% VERSION 15
%
%
%
%
%\begin{align*}
%R &= \gamma + \delta\Be_{21} - \alpha \Be_{32} - \beta \Be_{31} \\
%\gamma &= \cos\left(\frac{\theta}{2}\right) \cos\left( \inv{2}\left(\psi+\phi\right) \right) \\
%\delta &= \cos\left(\frac{\theta}{2}\right) \sin\left( \inv{2}\left(\psi+\phi\right) \right) \\
%\alpha &= -\sin\left(\frac{\theta}{2}\right) \cos\left( \inv{2}\left(\psi-\phi\right) \right) \\
%\beta &= -\sin\left(\frac{\theta}{2}\right) \sin\left( \inv{2}\left(\psi-\phi\right) \right) \\
%\end{align*}
%
%
%\begin{align*}
%\begin{bmatrix}
% - \beta^2 + \alpha^2 - \delta^2 + \gamma^2 & - 2 \alpha \beta - 2 \gamma \delta & - 2 \delta \alpha + 2 \gamma \beta \\
% - 2 \alpha \beta + 2 \gamma \delta & + \beta^2 - \alpha^2 - \delta^2 + \gamma^2 & + 2 \delta \beta + 2 \gamma \alpha \\
% - 2 \delta \alpha - 2 \gamma \beta & + 2 \delta \beta - 2 \gamma \alpha & - \beta^2 - \alpha^2 + \delta^2 + \gamma^2 \\
%\end{bmatrix}
%\end{align*}
%
%
%
%
% VERSION 16
%
%
%
%
%\begin{align*}
%R &= \gamma + \beta\Be_{21} - \alpha \Be_{32} - \delta \Be_{31} \\
%\gamma &= \cos\left(\frac{\theta}{2}\right) \cos\left( \inv{2}\left(\psi+\phi\right) \right) \\
%\beta &= \cos\left(\frac{\theta}{2}\right) \sin\left( \inv{2}\left(\psi+\phi\right) \right) \\
%\alpha &= -\sin\left(\frac{\theta}{2}\right) \cos\left( \inv{2}\left(\psi-\phi\right) \right) \\
%\delta &= -\sin\left(\frac{\theta}{2}\right) \sin\left( \inv{2}\left(\psi-\phi\right) \right) \\
%\end{align*}
%
%
%\begin{align*}
%\begin{bmatrix}
% - \delta^2 + \alpha^2 - \beta^2 + \gamma^2 & - 2 \alpha \delta - 2 \gamma \beta & - 2 \beta \alpha + 2 \gamma \delta \\
% - 2 \alpha \delta + 2 \gamma \beta & + \delta^2 - \alpha^2 - \beta^2 + \gamma^2 & + 2 \beta \delta + 2 \gamma \alpha \\
% - 2 \beta \alpha - 2 \gamma \delta & + 2 \beta \delta - 2 \gamma \alpha & - \delta^2 - \alpha^2 + \beta^2 + \gamma^2 \\
%\end{bmatrix}
%\end{align*}
%
%
%
%
% VERSION 17
%
%
%
%
%\begin{align*}
%R &= \gamma + \beta\Be_{21} - \delta \Be_{32} - \alpha \Be_{31} \\
%\gamma &= \cos\left(\frac{\theta}{2}\right) \cos\left( \inv{2}\left(\psi+\phi\right) \right) \\
%\beta &= \cos\left(\frac{\theta}{2}\right) \sin\left( \inv{2}\left(\psi+\phi\right) \right) \\
%\delta &= -\sin\left(\frac{\theta}{2}\right) \cos\left( \inv{2}\left(\psi-\phi\right) \right) \\
%\alpha &= -\sin\left(\frac{\theta}{2}\right) \sin\left( \inv{2}\left(\psi-\phi\right) \right) \\
%\end{align*}
%
%
%\begin{align*}
%-
%\begin{bmatrix}
% + \alpha^2 - \delta^2 + \beta^2 - \gamma^2 & + 2 \delta \alpha + 2 \gamma \beta & + 2 \beta \delta - 2 \gamma \alpha \\
% + 2 \delta \alpha - 2 \gamma \beta & - \alpha^2 + \delta^2 + \beta^2 - \gamma^2 & - 2 \beta \alpha - 2 \gamma \delta \\
% + 2 \beta \delta + 2 \gamma \alpha & - 2 \beta \alpha + 2 \gamma \delta & + \alpha^2 + \delta^2 - \beta^2 - \gamma^2 \\
%\end{bmatrix}
%\end{align*}
%
%
%
%
% VERSION 18
%
%
%
%
%\begin{align*}
%R &= \gamma + \alpha\Be_{21} - \delta \Be_{32} - \beta \Be_{31} \\
%\gamma &= \cos\left(\frac{\theta}{2}\right) \cos\left( \inv{2}\left(\psi+\phi\right) \right) \\
%\alpha &= \cos\left(\frac{\theta}{2}\right) \sin\left( \inv{2}\left(\psi+\phi\right) \right) \\
%\delta &= -\sin\left(\frac{\theta}{2}\right) \cos\left( \inv{2}\left(\psi-\phi\right) \right) \\
%\beta &= -\sin\left(\frac{\theta}{2}\right) \sin\left( \inv{2}\left(\psi-\phi\right) \right) \\
%\end{align*}
%
%
%\begin{align*}
%\begin{bmatrix}
% - \beta^2 + \delta^2 - \alpha^2 + \gamma^2 & - 2 \delta \beta - 2 \gamma \alpha & - 2 \alpha \delta + 2 \gamma \beta \\
% - 2 \delta \beta + 2 \gamma \alpha & + \beta^2 - \delta^2 - \alpha^2 + \gamma^2 & + 2 \alpha \beta + 2 \gamma \delta \\
% - 2 \alpha \delta - 2 \gamma \beta & + 2 \alpha \beta - 2 \gamma \delta & - \beta^2 - \delta^2 + \alpha^2 + \gamma^2 \\
%\end{bmatrix}
%\end{align*}
%
%
%
%
% VERSION 19
%
%
%
%
%\begin{align*}
%R &= \delta + \gamma\Be_{21} - \beta \Be_{32} - \alpha \Be_{31} \\
%\delta &= \cos\left(\frac{\theta}{2}\right) \cos\left( \inv{2}\left(\psi+\phi\right) \right) \\
%\gamma &= \cos\left(\frac{\theta}{2}\right) \sin\left( \inv{2}\left(\psi+\phi\right) \right) \\
%\beta &= -\sin\left(\frac{\theta}{2}\right) \cos\left( \inv{2}\left(\psi-\phi\right) \right) \\
%\alpha &= -\sin\left(\frac{\theta}{2}\right) \sin\left( \inv{2}\left(\psi-\phi\right) \right) \\
%\end{align*}
%
%
%\begin{align*}
%\begin{bmatrix}
% - \alpha^2 + \beta^2 - \gamma^2 + \delta^2 & - 2 \beta \alpha - 2 \delta \gamma & - 2 \gamma \beta + 2 \delta \alpha \\
% - 2 \beta \alpha + 2 \delta \gamma & + \alpha^2 - \beta^2 - \gamma^2 + \delta^2 & + 2 \gamma \alpha + 2 \delta \beta \\
% - 2 \gamma \beta - 2 \delta \alpha & + 2 \gamma \alpha - 2 \delta \beta & - \alpha^2 - \beta^2 + \gamma^2 + \delta^2 \\
%\end{bmatrix}
%\end{align*}
%
%
%
%
% VERSION 20
%
%
%
%
%\begin{align*}
%R &= \delta + \alpha\Be_{21} - \beta \Be_{32} - \gamma \Be_{31} \\
%\delta &= \cos\left(\frac{\theta}{2}\right) \cos\left( \inv{2}\left(\psi+\phi\right) \right) \\
%\alpha &= \cos\left(\frac{\theta}{2}\right) \sin\left( \inv{2}\left(\psi+\phi\right) \right) \\
%\beta &= -\sin\left(\frac{\theta}{2}\right) \cos\left( \inv{2}\left(\psi-\phi\right) \right) \\
%\gamma &= -\sin\left(\frac{\theta}{2}\right) \sin\left( \inv{2}\left(\psi-\phi\right) \right) \\
%\end{align*}
%
%
%\begin{align*}
%\begin{bmatrix}
% - \gamma^2 + \beta^2 - \alpha^2 + \delta^2 & - 2 \beta \gamma - 2 \delta \alpha & - 2 \alpha \beta + 2 \delta \gamma \\
% - 2 \beta \gamma + 2 \delta \alpha & + \gamma^2 - \beta^2 - \alpha^2 + \delta^2 & + 2 \alpha \gamma + 2 \delta \beta \\
% - 2 \alpha \beta - 2 \delta \gamma & + 2 \alpha \gamma - 2 \delta \beta & - \gamma^2 - \beta^2 + \alpha^2 + \delta^2 \\
%\end{bmatrix}
%\end{align*}
%
%
%
%
% VERSION 21
%
%
%
%
%\begin{align*}
%R &= \delta + \alpha\Be_{21} - \gamma \Be_{32} - \beta \Be_{31} \\
%\delta &= \cos\left(\frac{\theta}{2}\right) \cos\left( \inv{2}\left(\psi+\phi\right) \right) \\
%\alpha &= \cos\left(\frac{\theta}{2}\right) \sin\left( \inv{2}\left(\psi+\phi\right) \right) \\
%\gamma &= -\sin\left(\frac{\theta}{2}\right) \cos\left( \inv{2}\left(\psi-\phi\right) \right) \\
%\beta &= -\sin\left(\frac{\theta}{2}\right) \sin\left( \inv{2}\left(\psi-\phi\right) \right) \\
%\end{align*}
%
%
%\begin{align*}
%\begin{bmatrix}
% - \beta^2 + \gamma^2 - \alpha^2 + \delta^2 & - 2 \gamma \beta - 2 \delta \alpha & - 2 \alpha \gamma + 2 \delta \beta \\
% - 2 \gamma \beta + 2 \delta \alpha & + \beta^2 - \gamma^2 - \alpha^2 + \delta^2 & + 2 \alpha \beta + 2 \delta \gamma \\
% - 2 \alpha \gamma - 2 \delta \beta & + 2 \alpha \beta - 2 \delta \gamma & - \beta^2 - \gamma^2 + \alpha^2 + \delta^2 \\
%\end{bmatrix}
%\end{align*}
%
%
%
%
% VERSION 22
%
%
%
%
%\begin{align*}
%R &= \delta + \beta\Be_{21} - \gamma \Be_{32} - \alpha \Be_{31} \\
%\delta &= \cos\left(\frac{\theta}{2}\right) \cos\left( \inv{2}\left(\psi+\phi\right) \right) \\
%\beta &= \cos\left(\frac{\theta}{2}\right) \sin\left( \inv{2}\left(\psi+\phi\right) \right) \\
%\gamma &= -\sin\left(\frac{\theta}{2}\right) \cos\left( \inv{2}\left(\psi-\phi\right) \right) \\
%\alpha &= -\sin\left(\frac{\theta}{2}\right) \sin\left( \inv{2}\left(\psi-\phi\right) \right) \\
%\end{align*}
%
%
%\begin{align*}
%-
%\begin{bmatrix}
% + \alpha^2 - \gamma^2 + \beta^2 - \delta^2 & + 2 \gamma \alpha + 2 \delta \beta & + 2 \beta \gamma - 2 \delta \alpha \\
% + 2 \gamma \alpha - 2 \delta \beta & - \alpha^2 + \gamma^2 + \beta^2 - \delta^2 & - 2 \beta \alpha - 2 \delta \gamma \\
% + 2 \beta \gamma + 2 \delta \alpha & - 2 \beta \alpha + 2 \delta \gamma & + \alpha^2 + \gamma^2 - \beta^2 - \delta^2 \\
%\end{bmatrix}
%\end{align*}
%
%
%
%
% VERSION 23
%
%
%
%
%\begin{align*}
%R &= \delta + \beta\Be_{21} - \alpha \Be_{32} - \gamma \Be_{31} \\
%\delta &= \cos\left(\frac{\theta}{2}\right) \cos\left( \inv{2}\left(\psi+\phi\right) \right) \\
%\beta &= \cos\left(\frac{\theta}{2}\right) \sin\left( \inv{2}\left(\psi+\phi\right) \right) \\
%\alpha &= -\sin\left(\frac{\theta}{2}\right) \cos\left( \inv{2}\left(\psi-\phi\right) \right) \\
%\gamma &= -\sin\left(\frac{\theta}{2}\right) \sin\left( \inv{2}\left(\psi-\phi\right) \right) \\
%\end{align*}
%
%
%\begin{align*}
%\begin{bmatrix}
% - \gamma^2 + \alpha^2 - \beta^2 + \delta^2 & - 2 \alpha \gamma - 2 \delta \beta & - 2 \beta \alpha + 2 \delta \gamma \\
% - 2 \alpha \gamma + 2 \delta \beta & + \gamma^2 - \alpha^2 - \beta^2 + \delta^2 & + 2 \beta \gamma + 2 \delta \alpha \\
% - 2 \beta \alpha - 2 \delta \gamma & + 2 \beta \gamma - 2 \delta \alpha & - \gamma^2 - \alpha^2 + \beta^2 + \delta^2 \\
%\end{bmatrix}
%\end{align*}
%
%
%
%
% VERSION 24
%
%
%
%
%\begin{align*}
%R &= \delta + \gamma\Be_{21} - \alpha \Be_{32} - \beta \Be_{31} \\
%\delta &= \cos\left(\frac{\theta}{2}\right) \cos\left( \inv{2}\left(\psi+\phi\right) \right) \\
%\gamma &= \cos\left(\frac{\theta}{2}\right) \sin\left( \inv{2}\left(\psi+\phi\right) \right) \\
%\alpha &= -\sin\left(\frac{\theta}{2}\right) \cos\left( \inv{2}\left(\psi-\phi\right) \right) \\
%\beta &= -\sin\left(\frac{\theta}{2}\right) \sin\left( \inv{2}\left(\psi-\phi\right) \right) \\
%\end{align*}
%
%
%\begin{align*}
%\begin{bmatrix}
% - \beta^2 + \alpha^2 - \gamma^2 + \delta^2 & - 2 \alpha \beta - 2 \delta \gamma & - 2 \gamma \alpha + 2 \delta \beta \\
% - 2 \alpha \beta + 2 \delta \gamma & + \beta^2 - \alpha^2 - \gamma^2 + \delta^2 & + 2 \gamma \beta + 2 \delta \alpha \\
% - 2 \gamma \alpha - 2 \delta \beta & + 2 \gamma \beta - 2 \delta \alpha & - \beta^2 - \alpha^2 + \gamma^2 + \delta^2 \\
%\end{bmatrix}
%\end{align*}

%\subsection{Scripted output.  All parameterization variations}
%
%
%
%
% VERSION 1
%
%
%
%
%
%\begin{align*}
%R &= \alpha - \gamma\Be_{21} + \beta \Be_{32} - \delta \Be_{31} \\
%\alpha &= \cos\left(\frac{\theta}{2}\right) \cos\left( \inv{2}\left(\psi+\phi\right) \right) \\
%\gamma &= -\cos\left(\frac{\theta}{2}\right) \sin\left( \inv{2}\left(\psi+\phi\right) \right) \\
%\beta &= \sin\left(\frac{\theta}{2}\right) \cos\left( \inv{2}\left(\psi-\phi\right) \right) \\
%\delta &= -\sin\left(\frac{\theta}{2}\right) \sin\left( \inv{2}\left(\psi-\phi\right) \right) \\
%\end{align*}
%
%\begin{align*}
%\begin{bmatrix}
% - \delta^2 + \beta^2 - \gamma^2 + \alpha^2 & + 2 \beta \delta + 2 \alpha \gamma & - 2  \gamma \beta + 2 \alpha \delta \\
%+  2 \beta \delta - 2  \alpha \gamma & +  \delta^2 - \beta^2 - \gamma^2 + \alpha^2 & - 2  \gamma \delta - 2  \alpha \beta \\
% - 2  \gamma \beta - 2  \alpha \delta & - 2  \gamma \delta + 2 \alpha \beta & - \delta^2 - \beta^2 + \gamma^2 + \alpha^2 \\
%\end{bmatrix}
%\end{align*}
%
%
%
%
% VERSION 2
%
%
%
%
%
%\begin{align*}
%R &= \alpha - \delta\Be_{21} + \beta \Be_{32} - \gamma \Be_{31} \\
%\alpha &= \cos\left(\frac{\theta}{2}\right) \cos\left( \inv{2}\left(\psi+\phi\right) \right) \\
%\delta &= -\cos\left(\frac{\theta}{2}\right) \sin\left( \inv{2}\left(\psi+\phi\right) \right) \\
%\beta &= \sin\left(\frac{\theta}{2}\right) \cos\left( \inv{2}\left(\psi-\phi\right) \right) \\
%\gamma &= -\sin\left(\frac{\theta}{2}\right) \sin\left( \inv{2}\left(\psi-\phi\right) \right) \\
%\end{align*}
%
%\begin{align*}
%\begin{bmatrix}
% - \gamma^2 + \beta^2 - \delta^2 + \alpha^2 & + 2 \beta \gamma + 2 \alpha \delta & - 2  \delta \beta + 2 \alpha \gamma \\
%+  2 \beta \gamma - 2  \alpha \delta & +  \gamma^2 - \beta^2 - \delta^2 + \alpha^2 & - 2  \delta \gamma - 2  \alpha \beta \\
% - 2  \delta \beta - 2  \alpha \gamma & - 2  \delta \gamma + 2 \alpha \beta & - \gamma^2 - \beta^2 + \delta^2 + \alpha^2 \\
%\end{bmatrix}
%\end{align*}
%
%
%
%
% VERSION 3
%
%
%
%
%
%\begin{align*}
%R &= \alpha - \delta\Be_{21} + \gamma \Be_{32} - \beta \Be_{31} \\
%\alpha &= \cos\left(\frac{\theta}{2}\right) \cos\left( \inv{2}\left(\psi+\phi\right) \right) \\
%\delta &= -\cos\left(\frac{\theta}{2}\right) \sin\left( \inv{2}\left(\psi+\phi\right) \right) \\
%\gamma &= \sin\left(\frac{\theta}{2}\right) \cos\left( \inv{2}\left(\psi-\phi\right) \right) \\
%\beta &= -\sin\left(\frac{\theta}{2}\right) \sin\left( \inv{2}\left(\psi-\phi\right) \right) \\
%\end{align*}
%
%\begin{align*}
%\begin{bmatrix}
% - \beta^2 + \gamma^2 - \delta^2 + \alpha^2 & + 2 \gamma \beta + 2 \alpha \delta & - 2  \delta \gamma + 2 \alpha \beta \\
%+  2 \gamma \beta - 2  \alpha \delta & +  \beta^2 - \gamma^2 - \delta^2 + \alpha^2 & - 2  \delta \beta - 2  \alpha \gamma \\
% - 2  \delta \gamma - 2  \alpha \beta & - 2  \delta \beta + 2 \alpha \gamma & - \beta^2 - \gamma^2 + \delta^2 + \alpha^2 \\
%\end{bmatrix}
%\end{align*}
%
%
%
%
% VERSION 4
%
%
%
%
%
%\begin{align*}
%R &= \alpha - \beta\Be_{21} + \gamma \Be_{32} - \delta \Be_{31} \\
%\alpha &= \cos\left(\frac{\theta}{2}\right) \cos\left( \inv{2}\left(\psi+\phi\right) \right) \\
%\beta &= -\cos\left(\frac{\theta}{2}\right) \sin\left( \inv{2}\left(\psi+\phi\right) \right) \\
%\gamma &= \sin\left(\frac{\theta}{2}\right) \cos\left( \inv{2}\left(\psi-\phi\right) \right) \\
%\delta &= -\sin\left(\frac{\theta}{2}\right) \sin\left( \inv{2}\left(\psi-\phi\right) \right) \\
%\end{align*}
%
%\begin{align*}
%\begin{bmatrix}
% - \delta^2 + \gamma^2 - \beta^2 + \alpha^2 & + 2 \gamma \delta + 2 \alpha \beta & - 2  \beta \gamma + 2 \alpha \delta \\
%+  2 \gamma \delta - 2  \alpha \beta & +  \delta^2 - \gamma^2 - \beta^2 + \alpha^2 & - 2  \beta \delta - 2  \alpha \gamma \\
% - 2  \beta \gamma - 2  \alpha \delta & - 2  \beta \delta + 2 \alpha \gamma & - \delta^2 - \gamma^2 + \beta^2 + \alpha^2 \\
%\end{bmatrix}
%\end{align*}
%
%
%
%
% VERSION 5
%
%
%
%
%
%\begin{align*}
%R &= \alpha - \beta\Be_{21} + \delta \Be_{32} - \gamma \Be_{31} \\
%\alpha &= \cos\left(\frac{\theta}{2}\right) \cos\left( \inv{2}\left(\psi+\phi\right) \right) \\
%\beta &= -\cos\left(\frac{\theta}{2}\right) \sin\left( \inv{2}\left(\psi+\phi\right) \right) \\
%\delta &= \sin\left(\frac{\theta}{2}\right) \cos\left( \inv{2}\left(\psi-\phi\right) \right) \\
%\gamma &= -\sin\left(\frac{\theta}{2}\right) \sin\left( \inv{2}\left(\psi-\phi\right) \right) \\
%\end{align*}
%
%\begin{align*}
%\begin{bmatrix}
% - \gamma^2 + \delta^2 - \beta^2 + \alpha^2 & + 2 \delta \gamma + 2 \alpha \beta & - 2  \beta \delta + 2 \alpha \gamma \\
%+  2 \delta \gamma - 2  \alpha \beta & +  \gamma^2 - \delta^2 - \beta^2 + \alpha^2 & - 2  \beta \gamma - 2  \alpha \delta \\
% - 2  \beta \delta - 2  \alpha \gamma & - 2  \beta \gamma + 2 \alpha \delta & - \gamma^2 - \delta^2 + \beta^2 + \alpha^2 \\
%\end{bmatrix}
%\end{align*}
%
%
%
%
% VERSION 6
%
%
%
%
%
%\begin{align*}
%R &= \alpha - \gamma\Be_{21} + \delta \Be_{32} - \beta \Be_{31} \\
%\alpha &= \cos\left(\frac{\theta}{2}\right) \cos\left( \inv{2}\left(\psi+\phi\right) \right) \\
%\gamma &= -\cos\left(\frac{\theta}{2}\right) \sin\left( \inv{2}\left(\psi+\phi\right) \right) \\
%\delta &= \sin\left(\frac{\theta}{2}\right) \cos\left( \inv{2}\left(\psi-\phi\right) \right) \\
%\beta &= -\sin\left(\frac{\theta}{2}\right) \sin\left( \inv{2}\left(\psi-\phi\right) \right) \\
%\end{align*}
%
%\begin{align*}
%\begin{bmatrix}
% - \beta^2 + \delta^2 - \gamma^2 + \alpha^2 & + 2 \delta \beta + 2 \alpha \gamma & - 2  \gamma \delta + 2 \alpha \beta \\
%+  2 \delta \beta - 2  \alpha \gamma & +  \beta^2 - \delta^2 - \gamma^2 + \alpha^2 & - 2  \gamma \beta - 2  \alpha \delta \\
% - 2  \gamma \delta - 2  \alpha \beta & - 2  \gamma \beta + 2 \alpha \delta & - \beta^2 - \delta^2 + \gamma^2 + \alpha^2 \\
%\end{bmatrix}
%\end{align*}
%
%
%
%
% VERSION 7
%
%
%
%
%
%\begin{align*}
%R &= \beta - \gamma\Be_{21} + \alpha \Be_{32} - \delta \Be_{31} \\
%\beta &= \cos\left(\frac{\theta}{2}\right) \cos\left( \inv{2}\left(\psi+\phi\right) \right) \\
%\gamma &= -\cos\left(\frac{\theta}{2}\right) \sin\left( \inv{2}\left(\psi+\phi\right) \right) \\
%\alpha &= \sin\left(\frac{\theta}{2}\right) \cos\left( \inv{2}\left(\psi-\phi\right) \right) \\
%\delta &= -\sin\left(\frac{\theta}{2}\right) \sin\left( \inv{2}\left(\psi-\phi\right) \right) \\
%\end{align*}
%
%\begin{align*}
%\begin{bmatrix}
% - \delta^2 + \alpha^2 - \gamma^2 + \beta^2 & + 2 \alpha \delta + 2 \beta \gamma & - 2  \gamma \alpha + 2 \beta \delta \\
%+  2 \alpha \delta - 2  \beta \gamma & +  \delta^2 - \alpha^2 - \gamma^2 + \beta^2 & - 2  \gamma \delta - 2  \beta \alpha \\
% - 2  \gamma \alpha - 2  \beta \delta & - 2  \gamma \delta + 2 \beta \alpha & - \delta^2 - \alpha^2 + \gamma^2 + \beta^2 \\
%\end{bmatrix}
%\end{align*}
%
%
%
%
% VERSION 8
%
%
%
%
%
%\begin{align*}
%R &= \beta - \delta\Be_{21} + \alpha \Be_{32} - \gamma \Be_{31} \\
%\beta &= \cos\left(\frac{\theta}{2}\right) \cos\left( \inv{2}\left(\psi+\phi\right) \right) \\
%\delta &= -\cos\left(\frac{\theta}{2}\right) \sin\left( \inv{2}\left(\psi+\phi\right) \right) \\
%\alpha &= \sin\left(\frac{\theta}{2}\right) \cos\left( \inv{2}\left(\psi-\phi\right) \right) \\
%\gamma &= -\sin\left(\frac{\theta}{2}\right) \sin\left( \inv{2}\left(\psi-\phi\right) \right) \\
%\end{align*}
%
%\begin{align*}
%\begin{bmatrix}
% - \gamma^2 + \alpha^2 - \delta^2 + \beta^2 & + 2 \alpha \gamma + 2 \beta \delta & - 2  \delta \alpha + 2 \beta \gamma \\
%+  2 \alpha \gamma - 2  \beta \delta & +  \gamma^2 - \alpha^2 - \delta^2 + \beta^2 & - 2  \delta \gamma - 2  \beta \alpha \\
% - 2  \delta \alpha - 2  \beta \gamma & - 2  \delta \gamma + 2 \beta \alpha & - \gamma^2 - \alpha^2 + \delta^2 + \beta^2 \\
%\end{bmatrix}
%\end{align*}
%
%
%
%
% VERSION 9
%
%
%
%
%
%\begin{align*}
%R &= \beta - \delta\Be_{21} + \gamma \Be_{32} - \alpha \Be_{31} \\
%\beta &= \cos\left(\frac{\theta}{2}\right) \cos\left( \inv{2}\left(\psi+\phi\right) \right) \\
%\delta &= -\cos\left(\frac{\theta}{2}\right) \sin\left( \inv{2}\left(\psi+\phi\right) \right) \\
%\gamma &= \sin\left(\frac{\theta}{2}\right) \cos\left( \inv{2}\left(\psi-\phi\right) \right) \\
%\alpha &= -\sin\left(\frac{\theta}{2}\right) \sin\left( \inv{2}\left(\psi-\phi\right) \right) \\
%\end{align*}
%
%\begin{align*}
%\begin{bmatrix}
% - \alpha^2 + \gamma^2 - \delta^2 + \beta^2 & + 2 \gamma \alpha + 2 \beta \delta & - 2  \delta \gamma + 2 \beta \alpha \\
%+  2 \gamma \alpha - 2  \beta \delta & +  \alpha^2 - \gamma^2 - \delta^2 + \beta^2 & - 2  \delta \alpha - 2  \beta \gamma \\
% - 2  \delta \gamma - 2  \beta \alpha & - 2  \delta \alpha + 2 \beta \gamma & - \alpha^2 - \gamma^2 + \delta^2 + \beta^2 \\
%\end{bmatrix}
%\end{align*}
%
%
%
%
% VERSION 10
%
%
%
%
%
%\begin{align*}
%R &= \beta - \alpha\Be_{21} + \gamma \Be_{32} - \delta \Be_{31} \\
%\beta &= \cos\left(\frac{\theta}{2}\right) \cos\left( \inv{2}\left(\psi+\phi\right) \right) \\
%\alpha &= -\cos\left(\frac{\theta}{2}\right) \sin\left( \inv{2}\left(\psi+\phi\right) \right) \\
%\gamma &= \sin\left(\frac{\theta}{2}\right) \cos\left( \inv{2}\left(\psi-\phi\right) \right) \\
%\delta &= -\sin\left(\frac{\theta}{2}\right) \sin\left( \inv{2}\left(\psi-\phi\right) \right) \\
%\end{align*}
%
%\begin{align*}
%\begin{bmatrix}
% - \delta^2 + \gamma^2 - \alpha^2 + \beta^2 & + 2 \gamma \delta + 2 \beta \alpha & - 2  \alpha \gamma + 2 \beta \delta \\
%+  2 \gamma \delta - 2  \beta \alpha & +  \delta^2 - \gamma^2 - \alpha^2 + \beta^2 & - 2  \alpha \delta - 2  \beta \gamma \\
% - 2  \alpha \gamma - 2  \beta \delta & - 2  \alpha \delta + 2 \beta \gamma & - \delta^2 - \gamma^2 + \alpha^2 + \beta^2 \\
%\end{bmatrix}
%\end{align*}
%
%
%
%
% VERSION 11
%
%
%
%
%
%\begin{align*}
%R &= \beta - \alpha\Be_{21} + \delta \Be_{32} - \gamma \Be_{31} \\
%\beta &= \cos\left(\frac{\theta}{2}\right) \cos\left( \inv{2}\left(\psi+\phi\right) \right) \\
%\alpha &= -\cos\left(\frac{\theta}{2}\right) \sin\left( \inv{2}\left(\psi+\phi\right) \right) \\
%\delta &= \sin\left(\frac{\theta}{2}\right) \cos\left( \inv{2}\left(\psi-\phi\right) \right) \\
%\gamma &= -\sin\left(\frac{\theta}{2}\right) \sin\left( \inv{2}\left(\psi-\phi\right) \right) \\
%\end{align*}
%
%\begin{align*}
%\begin{bmatrix}
% - \gamma^2 + \delta^2 - \alpha^2 + \beta^2 & + 2 \delta \gamma + 2 \beta \alpha & - 2  \alpha \delta + 2 \beta \gamma \\
%+  2 \delta \gamma - 2  \beta \alpha & +  \gamma^2 - \delta^2 - \alpha^2 + \beta^2 & - 2  \alpha \gamma - 2  \beta \delta \\
% - 2  \alpha \delta - 2  \beta \gamma & - 2  \alpha \gamma + 2 \beta \delta & - \gamma^2 - \delta^2 + \alpha^2 + \beta^2 \\
%\end{bmatrix}
%\end{align*}
%
%
%
%
% VERSION 12
%
%
%
%
%
%\begin{align*}
%R &= \beta - \gamma\Be_{21} + \delta \Be_{32} - \alpha \Be_{31} \\
%\beta &= \cos\left(\frac{\theta}{2}\right) \cos\left( \inv{2}\left(\psi+\phi\right) \right) \\
%\gamma &= -\cos\left(\frac{\theta}{2}\right) \sin\left( \inv{2}\left(\psi+\phi\right) \right) \\
%\delta &= \sin\left(\frac{\theta}{2}\right) \cos\left( \inv{2}\left(\psi-\phi\right) \right) \\
%\alpha &= -\sin\left(\frac{\theta}{2}\right) \sin\left( \inv{2}\left(\psi-\phi\right) \right) \\
%\end{align*}
%
%\begin{align*}
%\begin{bmatrix}
% - \alpha^2 + \delta^2 - \gamma^2 + \beta^2 & + 2 \delta \alpha + 2 \beta \gamma & - 2  \gamma \delta + 2 \beta \alpha \\
%+  2 \delta \alpha - 2  \beta \gamma & +  \alpha^2 - \delta^2 - \gamma^2 + \beta^2 & - 2  \gamma \alpha - 2  \beta \delta \\
% - 2  \gamma \delta - 2  \beta \alpha & - 2  \gamma \alpha + 2 \beta \delta & - \alpha^2 - \delta^2 + \gamma^2 + \beta^2 \\
%\end{bmatrix}
%\end{align*}
%
%
%
%
% VERSION 13
%
%
%
%
%
%\begin{align*}
%R &= \gamma - \alpha\Be_{21} + \beta \Be_{32} - \delta \Be_{31} \\
%\gamma &= \cos\left(\frac{\theta}{2}\right) \cos\left( \inv{2}\left(\psi+\phi\right) \right) \\
%\alpha &= -\cos\left(\frac{\theta}{2}\right) \sin\left( \inv{2}\left(\psi+\phi\right) \right) \\
%\beta &= \sin\left(\frac{\theta}{2}\right) \cos\left( \inv{2}\left(\psi-\phi\right) \right) \\
%\delta &= -\sin\left(\frac{\theta}{2}\right) \sin\left( \inv{2}\left(\psi-\phi\right) \right) \\
%\end{align*}
%
%\begin{align*}
%\begin{bmatrix}
% - \delta^2 + \beta^2 - \alpha^2 + \gamma^2 & + 2 \beta \delta + 2 \gamma \alpha & - 2  \alpha \beta + 2 \gamma \delta \\
%+  2 \beta \delta - 2  \gamma \alpha & +  \delta^2 - \beta^2 - \alpha^2 + \gamma^2 & - 2  \alpha \delta - 2  \gamma \beta \\
% - 2  \alpha \beta - 2  \gamma \delta & - 2  \alpha \delta + 2 \gamma \beta & - \delta^2 - \beta^2 + \alpha^2 + \gamma^2 \\
%\end{bmatrix}
%\end{align*}
%
%
%
%
% VERSION 14
%
%
%
%
%
%\begin{align*}
%R &= \gamma - \delta\Be_{21} + \beta \Be_{32} - \alpha \Be_{31} \\
%\gamma &= \cos\left(\frac{\theta}{2}\right) \cos\left( \inv{2}\left(\psi+\phi\right) \right) \\
%\delta &= -\cos\left(\frac{\theta}{2}\right) \sin\left( \inv{2}\left(\psi+\phi\right) \right) \\
%\beta &= \sin\left(\frac{\theta}{2}\right) \cos\left( \inv{2}\left(\psi-\phi\right) \right) \\
%\alpha &= -\sin\left(\frac{\theta}{2}\right) \sin\left( \inv{2}\left(\psi-\phi\right) \right) \\
%\end{align*}
%
%\begin{align*}
%\begin{bmatrix}
% - \alpha^2 + \beta^2 - \delta^2 + \gamma^2 & + 2 \beta \alpha + 2 \gamma \delta & - 2  \delta \beta + 2 \gamma \alpha \\
%+  2 \beta \alpha - 2  \gamma \delta & +  \alpha^2 - \beta^2 - \delta^2 + \gamma^2 & - 2  \delta \alpha - 2  \gamma \beta \\
% - 2  \delta \beta - 2  \gamma \alpha & - 2  \delta \alpha + 2 \gamma \beta & - \alpha^2 - \beta^2 + \delta^2 + \gamma^2 \\
%\end{bmatrix}
%\end{align*}
%
%
%
%
% VERSION 15
%
%
%
%
%
%\begin{align*}
%R &= \gamma - \delta\Be_{21} + \alpha \Be_{32} - \beta \Be_{31} \\
%\gamma &= \cos\left(\frac{\theta}{2}\right) \cos\left( \inv{2}\left(\psi+\phi\right) \right) \\
%\delta &= -\cos\left(\frac{\theta}{2}\right) \sin\left( \inv{2}\left(\psi+\phi\right) \right) \\
%\alpha &= \sin\left(\frac{\theta}{2}\right) \cos\left( \inv{2}\left(\psi-\phi\right) \right) \\
%\beta &= -\sin\left(\frac{\theta}{2}\right) \sin\left( \inv{2}\left(\psi-\phi\right) \right) \\
%\end{align*}
%
%\begin{align*}
%\begin{bmatrix}
% - \beta^2 + \alpha^2 - \delta^2 + \gamma^2 & + 2 \alpha \beta + 2 \gamma \delta & - 2  \delta \alpha + 2 \gamma \beta \\
%+  2 \alpha \beta - 2  \gamma \delta & +  \beta^2 - \alpha^2 - \delta^2 + \gamma^2 & - 2  \delta \beta - 2  \gamma \alpha \\
% - 2  \delta \alpha - 2  \gamma \beta & - 2  \delta \beta + 2 \gamma \alpha & - \beta^2 - \alpha^2 + \delta^2 + \gamma^2 \\
%\end{bmatrix}
%\end{align*}
%
%
%
%
% VERSION 16
%
%
%
%
%
%\begin{align*}
%R &= \gamma - \beta\Be_{21} + \alpha \Be_{32} - \delta \Be_{31} \\
%\gamma &= \cos\left(\frac{\theta}{2}\right) \cos\left( \inv{2}\left(\psi+\phi\right) \right) \\
%\beta &= -\cos\left(\frac{\theta}{2}\right) \sin\left( \inv{2}\left(\psi+\phi\right) \right) \\
%\alpha &= \sin\left(\frac{\theta}{2}\right) \cos\left( \inv{2}\left(\psi-\phi\right) \right) \\
%\delta &= -\sin\left(\frac{\theta}{2}\right) \sin\left( \inv{2}\left(\psi-\phi\right) \right) \\
%\end{align*}
%
%\begin{align*}
%\begin{bmatrix}
% - \delta^2 + \alpha^2 - \beta^2 + \gamma^2 & + 2 \alpha \delta + 2 \gamma \beta & - 2  \beta \alpha + 2 \gamma \delta \\
%+  2 \alpha \delta - 2  \gamma \beta & +  \delta^2 - \alpha^2 - \beta^2 + \gamma^2 & - 2  \beta \delta - 2  \gamma \alpha \\
% - 2  \beta \alpha - 2  \gamma \delta & - 2  \beta \delta + 2 \gamma \alpha & - \delta^2 - \alpha^2 + \beta^2 + \gamma^2 \\
%\end{bmatrix}
%\end{align*}
%
%
%
%
% VERSION 17
%
%
%
%
%
%\begin{align*}
%R &= \gamma - \beta\Be_{21} + \delta \Be_{32} - \alpha \Be_{31} \\
%\gamma &= \cos\left(\frac{\theta}{2}\right) \cos\left( \inv{2}\left(\psi+\phi\right) \right) \\
%\beta &= -\cos\left(\frac{\theta}{2}\right) \sin\left( \inv{2}\left(\psi+\phi\right) \right) \\
%\delta &= \sin\left(\frac{\theta}{2}\right) \cos\left( \inv{2}\left(\psi-\phi\right) \right) \\
%\alpha &= -\sin\left(\frac{\theta}{2}\right) \sin\left( \inv{2}\left(\psi-\phi\right) \right) \\
%\end{align*}
%
%\begin{align*}
%\begin{bmatrix}
% - \alpha^2 + \delta^2 - \beta^2 + \gamma^2 & + 2 \delta \alpha + 2 \gamma \beta & - 2  \beta \delta + 2 \gamma \alpha \\
%+  2 \delta \alpha - 2  \gamma \beta & +  \alpha^2 - \delta^2 - \beta^2 + \gamma^2 & - 2  \beta \alpha - 2  \gamma \delta \\
% - 2  \beta \delta - 2  \gamma \alpha & - 2  \beta \alpha + 2 \gamma \delta & - \alpha^2 - \delta^2 + \beta^2 + \gamma^2 \\
%\end{bmatrix}
%\end{align*}
%
%
%
%
% VERSION 18
%
%
%
%
%
%\begin{align*}
%R &= \gamma - \alpha\Be_{21} + \delta \Be_{32} - \beta \Be_{31} \\
%\gamma &= \cos\left(\frac{\theta}{2}\right) \cos\left( \inv{2}\left(\psi+\phi\right) \right) \\
%\alpha &= -\cos\left(\frac{\theta}{2}\right) \sin\left( \inv{2}\left(\psi+\phi\right) \right) \\
%\delta &= \sin\left(\frac{\theta}{2}\right) \cos\left( \inv{2}\left(\psi-\phi\right) \right) \\
%\beta &= -\sin\left(\frac{\theta}{2}\right) \sin\left( \inv{2}\left(\psi-\phi\right) \right) \\
%\end{align*}
%
%\begin{align*}
%\begin{bmatrix}
% - \beta^2 + \delta^2 - \alpha^2 + \gamma^2 & + 2 \delta \beta + 2 \gamma \alpha & - 2  \alpha \delta + 2 \gamma \beta \\
%+  2 \delta \beta - 2  \gamma \alpha & +  \beta^2 - \delta^2 - \alpha^2 + \gamma^2 & - 2  \alpha \beta - 2  \gamma \delta \\
% - 2  \alpha \delta - 2  \gamma \beta & - 2  \alpha \beta + 2 \gamma \delta & - \beta^2 - \delta^2 + \alpha^2 + \gamma^2 \\
%\end{bmatrix}
%\end{align*}
%
%
%
%
% VERSION 19
%
%
%
%
%
%\begin{align*}
%R &= \delta - \gamma\Be_{21} + \beta \Be_{32} - \alpha \Be_{31} \\
%\delta &= \cos\left(\frac{\theta}{2}\right) \cos\left( \inv{2}\left(\psi+\phi\right) \right) \\
%\gamma &= -\cos\left(\frac{\theta}{2}\right) \sin\left( \inv{2}\left(\psi+\phi\right) \right) \\
%\beta &= \sin\left(\frac{\theta}{2}\right) \cos\left( \inv{2}\left(\psi-\phi\right) \right) \\
%\alpha &= -\sin\left(\frac{\theta}{2}\right) \sin\left( \inv{2}\left(\psi-\phi\right) \right) \\
%\end{align*}
%
%\begin{align*}
%\begin{bmatrix}
% - \alpha^2 + \beta^2 - \gamma^2 + \delta^2 & + 2 \beta \alpha + 2 \delta \gamma & - 2  \gamma \beta + 2 \delta \alpha \\
%+  2 \beta \alpha - 2  \delta \gamma & +  \alpha^2 - \beta^2 - \gamma^2 + \delta^2 & - 2  \gamma \alpha - 2  \delta \beta \\
% - 2  \gamma \beta - 2  \delta \alpha & - 2  \gamma \alpha + 2 \delta \beta & - \alpha^2 - \beta^2 + \gamma^2 + \delta^2 \\
%\end{bmatrix}
%\end{align*}
%
%
%
%
% VERSION 20
%
%
%
%
%
%\begin{align*}
%R &= \delta - \alpha\Be_{21} + \beta \Be_{32} - \gamma \Be_{31} \\
%\delta &= \cos\left(\frac{\theta}{2}\right) \cos\left( \inv{2}\left(\psi+\phi\right) \right) \\
%\alpha &= -\cos\left(\frac{\theta}{2}\right) \sin\left( \inv{2}\left(\psi+\phi\right) \right) \\
%\beta &= \sin\left(\frac{\theta}{2}\right) \cos\left( \inv{2}\left(\psi-\phi\right) \right) \\
%\gamma &= -\sin\left(\frac{\theta}{2}\right) \sin\left( \inv{2}\left(\psi-\phi\right) \right) \\
%\end{align*}
%
%\begin{align*}
%\begin{bmatrix}
% - \gamma^2 + \beta^2 - \alpha^2 + \delta^2 & + 2 \beta \gamma + 2 \delta \alpha & - 2  \alpha \beta + 2 \delta \gamma \\
%+  2 \beta \gamma - 2  \delta \alpha & +  \gamma^2 - \beta^2 - \alpha^2 + \delta^2 & - 2  \alpha \gamma - 2  \delta \beta \\
% - 2  \alpha \beta - 2  \delta \gamma & - 2  \alpha \gamma + 2 \delta \beta & - \gamma^2 - \beta^2 + \alpha^2 + \delta^2 \\
%\end{bmatrix}
%\end{align*}
%
%
%
%
% VERSION 21
%
%
%
%
%
%\begin{align*}
%R &= \delta - \alpha\Be_{21} + \gamma \Be_{32} - \beta \Be_{31} \\
%\delta &= \cos\left(\frac{\theta}{2}\right) \cos\left( \inv{2}\left(\psi+\phi\right) \right) \\
%\alpha &= -\cos\left(\frac{\theta}{2}\right) \sin\left( \inv{2}\left(\psi+\phi\right) \right) \\
%\gamma &= \sin\left(\frac{\theta}{2}\right) \cos\left( \inv{2}\left(\psi-\phi\right) \right) \\
%\beta &= -\sin\left(\frac{\theta}{2}\right) \sin\left( \inv{2}\left(\psi-\phi\right) \right) \\
%\end{align*}
%
%\begin{align*}
%\begin{bmatrix}
% - \beta^2 + \gamma^2 - \alpha^2 + \delta^2 & + 2 \gamma \beta + 2 \delta \alpha & - 2  \alpha \gamma + 2 \delta \beta \\
%+  2 \gamma \beta - 2  \delta \alpha & +  \beta^2 - \gamma^2 - \alpha^2 + \delta^2 & - 2  \alpha \beta - 2  \delta \gamma \\
% - 2  \alpha \gamma - 2  \delta \beta & - 2  \alpha \beta + 2 \delta \gamma & - \beta^2 - \gamma^2 + \alpha^2 + \delta^2 \\
%\end{bmatrix}
%\end{align*}
%
%
%
%
% VERSION 22
%
%
%
%
%
%\begin{align*}
%R &= \delta - \beta\Be_{21} + \gamma \Be_{32} - \alpha \Be_{31} \\
%\delta &= \cos\left(\frac{\theta}{2}\right) \cos\left( \inv{2}\left(\psi+\phi\right) \right) \\
%\beta &= -\cos\left(\frac{\theta}{2}\right) \sin\left( \inv{2}\left(\psi+\phi\right) \right) \\
%\gamma &= \sin\left(\frac{\theta}{2}\right) \cos\left( \inv{2}\left(\psi-\phi\right) \right) \\
%\alpha &= -\sin\left(\frac{\theta}{2}\right) \sin\left( \inv{2}\left(\psi-\phi\right) \right) \\
%\end{align*}
%
%\begin{align*}
%\begin{bmatrix}
% - \alpha^2 + \gamma^2 - \beta^2 + \delta^2 & + 2 \gamma \alpha + 2 \delta \beta & - 2  \beta \gamma + 2 \delta \alpha \\
%+  2 \gamma \alpha - 2  \delta \beta & +  \alpha^2 - \gamma^2 - \beta^2 + \delta^2 & - 2  \beta \alpha - 2  \delta \gamma \\
% - 2  \beta \gamma - 2  \delta \alpha & - 2  \beta \alpha + 2 \delta \gamma & - \alpha^2 - \gamma^2 + \beta^2 + \delta^2 \\
%\end{bmatrix}
%\end{align*}
%
%
%
%
% VERSION 23
%
%
%
%
%
%\begin{align*}
%R &= \delta - \beta\Be_{21} + \alpha \Be_{32} - \gamma \Be_{31} \\
%\delta &= \cos\left(\frac{\theta}{2}\right) \cos\left( \inv{2}\left(\psi+\phi\right) \right) \\
%\beta &= -\cos\left(\frac{\theta}{2}\right) \sin\left( \inv{2}\left(\psi+\phi\right) \right) \\
%\alpha &= \sin\left(\frac{\theta}{2}\right) \cos\left( \inv{2}\left(\psi-\phi\right) \right) \\
%\gamma &= -\sin\left(\frac{\theta}{2}\right) \sin\left( \inv{2}\left(\psi-\phi\right) \right) \\
%\end{align*}
%
%\begin{align*}
%\begin{bmatrix}
% - \gamma^2 + \alpha^2 - \beta^2 + \delta^2 & + 2 \alpha \gamma + 2 \delta \beta & - 2  \beta \alpha + 2 \delta \gamma \\
%+  2 \alpha \gamma - 2  \delta \beta & +  \gamma^2 - \alpha^2 - \beta^2 + \delta^2 & - 2  \beta \gamma - 2  \delta \alpha \\
% - 2  \beta \alpha - 2  \delta \gamma & - 2  \beta \gamma + 2 \delta \alpha & - \gamma^2 - \alpha^2 + \beta^2 + \delta^2 \\
%\end{bmatrix}
%\end{align*}
%
%
%
%
% VERSION 24
%
%
%
%
%
%\begin{align*}
%R &= \delta - \gamma\Be_{21} + \alpha \Be_{32} - \beta \Be_{31} \\
%\delta &= \cos\left(\frac{\theta}{2}\right) \cos\left( \inv{2}\left(\psi+\phi\right) \right) \\
%\gamma &= -\cos\left(\frac{\theta}{2}\right) \sin\left( \inv{2}\left(\psi+\phi\right) \right) \\
%\alpha &= \sin\left(\frac{\theta}{2}\right) \cos\left( \inv{2}\left(\psi-\phi\right) \right) \\
%\beta &= -\sin\left(\frac{\theta}{2}\right) \sin\left( \inv{2}\left(\psi-\phi\right) \right) \\
%\end{align*}
%
%\begin{align*}
%\begin{bmatrix}
% - \beta^2 + \alpha^2 - \gamma^2 + \delta^2 & + 2 \alpha \beta + 2 \delta \gamma & - 2  \gamma \alpha + 2 \delta \beta \\
%+  2 \alpha \beta - 2  \delta \gamma & +  \beta^2 - \alpha^2 - \gamma^2 + \delta^2 & - 2  \gamma \beta - 2  \delta \alpha \\
% - 2  \gamma \alpha - 2  \delta \beta & - 2  \gamma \beta + 2 \delta \alpha & - \beta^2 - \alpha^2 + \gamma^2 + \delta^2 \\
%\end{bmatrix}
%\end{align*}

%\subsection{Scripted output.  All parameterization variations}
%
%
%
%
%
%
% VERSION 1
%
%
%
%
%\begin{align*}
%R &= \alpha - \gamma\Be_{21} - \beta \Be_{32} + \delta \Be_{31} \\
%\alpha &= \cos\left(\frac{\theta}{2}\right) \cos\left( \inv{2}\left(\psi+\phi\right) \right) \\
%\gamma &= -\cos\left(\frac{\theta}{2}\right) \sin\left( \inv{2}\left(\psi+\phi\right) \right) \\
%\beta &= -\sin\left(\frac{\theta}{2}\right) \cos\left( \inv{2}\left(\psi-\phi\right) \right) \\
%\delta &= \sin\left(\frac{\theta}{2}\right) \sin\left( \inv{2}\left(\psi-\phi\right) \right) \\
%\end{align*}
%
%\begin{align*}
%\begin{bmatrix}
% - \delta^2 + \beta^2 - \gamma^2 + \alpha^2 & + 2 \beta \delta + 2 \alpha \gamma & + 2 \gamma \beta - 2 \alpha \delta \\
%+ 2 \beta \delta - 2 \alpha \gamma & + \delta^2 - \beta^2 - \gamma^2 + \alpha^2 & + 2 \gamma \delta + 2 \alpha \beta \\
%+ 2 \gamma \beta + 2 \alpha \delta & + 2 \gamma \delta - 2 \alpha \beta & - \delta^2 - \beta^2 + \gamma^2 + \alpha^2 \\
%\end{bmatrix}
%\end{align*}
%
%
%
%
% VERSION 2
%
%
%
%
%\begin{align*}
%R &= \alpha - \delta\Be_{21} - \beta \Be_{32} + \gamma \Be_{31} \\
%\alpha &= \cos\left(\frac{\theta}{2}\right) \cos\left( \inv{2}\left(\psi+\phi\right) \right) \\
%\delta &= -\cos\left(\frac{\theta}{2}\right) \sin\left( \inv{2}\left(\psi+\phi\right) \right) \\
%\beta &= -\sin\left(\frac{\theta}{2}\right) \cos\left( \inv{2}\left(\psi-\phi\right) \right) \\
%\gamma &= \sin\left(\frac{\theta}{2}\right) \sin\left( \inv{2}\left(\psi-\phi\right) \right) \\
%\end{align*}
%
%\begin{align*}
%\begin{bmatrix}
% - \gamma^2 + \beta^2 - \delta^2 + \alpha^2 & + 2 \beta \gamma + 2 \alpha \delta & + 2 \delta \beta - 2 \alpha \gamma \\
%+ 2 \beta \gamma - 2 \alpha \delta & + \gamma^2 - \beta^2 - \delta^2 + \alpha^2 & + 2 \delta \gamma + 2 \alpha \beta \\
%+ 2 \delta \beta + 2 \alpha \gamma & + 2 \delta \gamma - 2 \alpha \beta & - \gamma^2 - \beta^2 + \delta^2 + \alpha^2 \\
%\end{bmatrix}
%\end{align*}
%
%
%
%
% VERSION 3
%
%
%
%
%\begin{align*}
%R &= \alpha - \delta\Be_{21} - \gamma \Be_{32} + \beta \Be_{31} \\
%\alpha &= \cos\left(\frac{\theta}{2}\right) \cos\left( \inv{2}\left(\psi+\phi\right) \right) \\
%\delta &= -\cos\left(\frac{\theta}{2}\right) \sin\left( \inv{2}\left(\psi+\phi\right) \right) \\
%\gamma &= -\sin\left(\frac{\theta}{2}\right) \cos\left( \inv{2}\left(\psi-\phi\right) \right) \\
%\beta &= \sin\left(\frac{\theta}{2}\right) \sin\left( \inv{2}\left(\psi-\phi\right) \right) \\
%\end{align*}
%
%\begin{align*}
%\begin{bmatrix}
% - \beta^2 + \gamma^2 - \delta^2 + \alpha^2 & + 2 \gamma \beta + 2 \alpha \delta & + 2 \delta \gamma - 2 \alpha \beta \\
%+ 2 \gamma \beta - 2 \alpha \delta & + \beta^2 - \gamma^2 - \delta^2 + \alpha^2 & + 2 \delta \beta + 2 \alpha \gamma \\
%+ 2 \delta \gamma + 2 \alpha \beta & + 2 \delta \beta - 2 \alpha \gamma & - \beta^2 - \gamma^2 + \delta^2 + \alpha^2 \\
%\end{bmatrix}
%\end{align*}
%
%
%
%
% VERSION 4
%
%
%
%
%\begin{align*}
%R &= \alpha - \beta\Be_{21} - \gamma \Be_{32} + \delta \Be_{31} \\
%\alpha &= \cos\left(\frac{\theta}{2}\right) \cos\left( \inv{2}\left(\psi+\phi\right) \right) \\
%\beta &= -\cos\left(\frac{\theta}{2}\right) \sin\left( \inv{2}\left(\psi+\phi\right) \right) \\
%\gamma &= -\sin\left(\frac{\theta}{2}\right) \cos\left( \inv{2}\left(\psi-\phi\right) \right) \\
%\delta &= \sin\left(\frac{\theta}{2}\right) \sin\left( \inv{2}\left(\psi-\phi\right) \right) \\
%\end{align*}
%
%\begin{align*}
%\begin{bmatrix}
% - \delta^2 + \gamma^2 - \beta^2 + \alpha^2 & + 2 \gamma \delta + 2 \alpha \beta & + 2 \beta \gamma - 2 \alpha \delta \\
%+ 2 \gamma \delta - 2 \alpha \beta & + \delta^2 - \gamma^2 - \beta^2 + \alpha^2 & + 2 \beta \delta + 2 \alpha \gamma \\
%+ 2 \beta \gamma + 2 \alpha \delta & + 2 \beta \delta - 2 \alpha \gamma & - \delta^2 - \gamma^2 + \beta^2 + \alpha^2 \\
%\end{bmatrix}
%\end{align*}
%
%
%
%
% VERSION 5
%
%
%
%
%\begin{align*}
%R &= \alpha - \beta\Be_{21} - \delta \Be_{32} + \gamma \Be_{31} \\
%\alpha &= \cos\left(\frac{\theta}{2}\right) \cos\left( \inv{2}\left(\psi+\phi\right) \right) \\
%\beta &= -\cos\left(\frac{\theta}{2}\right) \sin\left( \inv{2}\left(\psi+\phi\right) \right) \\
%\delta &= -\sin\left(\frac{\theta}{2}\right) \cos\left( \inv{2}\left(\psi-\phi\right) \right) \\
%\gamma &= \sin\left(\frac{\theta}{2}\right) \sin\left( \inv{2}\left(\psi-\phi\right) \right) \\
%\end{align*}
%
%\begin{align*}
%\begin{bmatrix}
% - \gamma^2 + \delta^2 - \beta^2 + \alpha^2 & + 2 \delta \gamma + 2 \alpha \beta & + 2 \beta \delta - 2 \alpha \gamma \\
%+ 2 \delta \gamma - 2 \alpha \beta & + \gamma^2 - \delta^2 - \beta^2 + \alpha^2 & + 2 \beta \gamma + 2 \alpha \delta \\
%+ 2 \beta \delta + 2 \alpha \gamma & + 2 \beta \gamma - 2 \alpha \delta & - \gamma^2 - \delta^2 + \beta^2 + \alpha^2 \\
%\end{bmatrix}
%\end{align*}
%
%
%
%
% VERSION 6
%
%
%
%
%\begin{align*}
%R &= \alpha - \gamma\Be_{21} - \delta \Be_{32} + \beta \Be_{31} \\
%\alpha &= \cos\left(\frac{\theta}{2}\right) \cos\left( \inv{2}\left(\psi+\phi\right) \right) \\
%\gamma &= -\cos\left(\frac{\theta}{2}\right) \sin\left( \inv{2}\left(\psi+\phi\right) \right) \\
%\delta &= -\sin\left(\frac{\theta}{2}\right) \cos\left( \inv{2}\left(\psi-\phi\right) \right) \\
%\beta &= \sin\left(\frac{\theta}{2}\right) \sin\left( \inv{2}\left(\psi-\phi\right) \right) \\
%\end{align*}
%
%\begin{align*}
%\begin{bmatrix}
% - \beta^2 + \delta^2 - \gamma^2 + \alpha^2 & + 2 \delta \beta + 2 \alpha \gamma & + 2 \gamma \delta - 2 \alpha \beta \\
%+ 2 \delta \beta - 2 \alpha \gamma & + \beta^2 - \delta^2 - \gamma^2 + \alpha^2 & + 2 \gamma \beta + 2 \alpha \delta \\
%+ 2 \gamma \delta + 2 \alpha \beta & + 2 \gamma \beta - 2 \alpha \delta & - \beta^2 - \delta^2 + \gamma^2 + \alpha^2 \\
%\end{bmatrix}
%\end{align*}
%
%
%
%
% VERSION 7
%
%
%
%
%\begin{align*}
%R &= \beta - \gamma\Be_{21} - \alpha \Be_{32} + \delta \Be_{31} \\
%\beta &= \cos\left(\frac{\theta}{2}\right) \cos\left( \inv{2}\left(\psi+\phi\right) \right) \\
%\gamma &= -\cos\left(\frac{\theta}{2}\right) \sin\left( \inv{2}\left(\psi+\phi\right) \right) \\
%\alpha &= -\sin\left(\frac{\theta}{2}\right) \cos\left( \inv{2}\left(\psi-\phi\right) \right) \\
%\delta &= \sin\left(\frac{\theta}{2}\right) \sin\left( \inv{2}\left(\psi-\phi\right) \right) \\
%\end{align*}
%
%\begin{align*}
%\begin{bmatrix}
% - \delta^2 + \alpha^2 - \gamma^2 + \beta^2 & + 2 \alpha \delta + 2 \beta \gamma & + 2 \gamma \alpha - 2 \beta \delta \\
%+ 2 \alpha \delta - 2 \beta \gamma & + \delta^2 - \alpha^2 - \gamma^2 + \beta^2 & + 2 \gamma \delta + 2 \beta \alpha \\
%+ 2 \gamma \alpha + 2 \beta \delta & + 2 \gamma \delta - 2 \beta \alpha & - \delta^2 - \alpha^2 + \gamma^2 + \beta^2 \\
%\end{bmatrix}
%\end{align*}
%
%
%
%
% VERSION 8
%
%
%
%
%\begin{align*}
%R &= \beta - \delta\Be_{21} - \alpha \Be_{32} + \gamma \Be_{31} \\
%\beta &= \cos\left(\frac{\theta}{2}\right) \cos\left( \inv{2}\left(\psi+\phi\right) \right) \\
%\delta &= -\cos\left(\frac{\theta}{2}\right) \sin\left( \inv{2}\left(\psi+\phi\right) \right) \\
%\alpha &= -\sin\left(\frac{\theta}{2}\right) \cos\left( \inv{2}\left(\psi-\phi\right) \right) \\
%\gamma &= \sin\left(\frac{\theta}{2}\right) \sin\left( \inv{2}\left(\psi-\phi\right) \right) \\
%\end{align*}
%
%\begin{align*}
%\begin{bmatrix}
% - \gamma^2 + \alpha^2 - \delta^2 + \beta^2 & + 2 \alpha \gamma + 2 \beta \delta & + 2 \delta \alpha - 2 \beta \gamma \\
%+ 2 \alpha \gamma - 2 \beta \delta & + \gamma^2 - \alpha^2 - \delta^2 + \beta^2 & + 2 \delta \gamma + 2 \beta \alpha \\
%+ 2 \delta \alpha + 2 \beta \gamma & + 2 \delta \gamma - 2 \beta \alpha & - \gamma^2 - \alpha^2 + \delta^2 + \beta^2 \\
%\end{bmatrix}
%\end{align*}
%
%
%
%
% VERSION 9
%
%
%
%
%\begin{align*}
%R &= \beta - \delta\Be_{21} - \gamma \Be_{32} + \alpha \Be_{31} \\
%\beta &= \cos\left(\frac{\theta}{2}\right) \cos\left( \inv{2}\left(\psi+\phi\right) \right) \\
%\delta &= -\cos\left(\frac{\theta}{2}\right) \sin\left( \inv{2}\left(\psi+\phi\right) \right) \\
%\gamma &= -\sin\left(\frac{\theta}{2}\right) \cos\left( \inv{2}\left(\psi-\phi\right) \right) \\
%\alpha &= \sin\left(\frac{\theta}{2}\right) \sin\left( \inv{2}\left(\psi-\phi\right) \right) \\
%\end{align*}
%
%\begin{align*}
%\begin{bmatrix}
% - \alpha^2 + \gamma^2 - \delta^2 + \beta^2 & + 2 \gamma \alpha + 2 \beta \delta & + 2 \delta \gamma - 2 \beta \alpha \\
%+ 2 \gamma \alpha - 2 \beta \delta & + \alpha^2 - \gamma^2 - \delta^2 + \beta^2 & + 2 \delta \alpha + 2 \beta \gamma \\
%+ 2 \delta \gamma + 2 \beta \alpha & + 2 \delta \alpha - 2 \beta \gamma & - \alpha^2 - \gamma^2 + \delta^2 + \beta^2 \\
%\end{bmatrix}
%\end{align*}
%
%
%
%
% VERSION 10
%
%
%
%
%\begin{align*}
%R &= \beta - \alpha\Be_{21} - \gamma \Be_{32} + \delta \Be_{31} \\
%\beta &= \cos\left(\frac{\theta}{2}\right) \cos\left( \inv{2}\left(\psi+\phi\right) \right) \\
%\alpha &= -\cos\left(\frac{\theta}{2}\right) \sin\left( \inv{2}\left(\psi+\phi\right) \right) \\
%\gamma &= -\sin\left(\frac{\theta}{2}\right) \cos\left( \inv{2}\left(\psi-\phi\right) \right) \\
%\delta &= \sin\left(\frac{\theta}{2}\right) \sin\left( \inv{2}\left(\psi-\phi\right) \right) \\
%\end{align*}
%
%\begin{align*}
%\begin{bmatrix}
% - \delta^2 + \gamma^2 - \alpha^2 + \beta^2 & + 2 \gamma \delta + 2 \beta \alpha & + 2 \alpha \gamma - 2 \beta \delta \\
%+ 2 \gamma \delta - 2 \beta \alpha & + \delta^2 - \gamma^2 - \alpha^2 + \beta^2 & + 2 \alpha \delta + 2 \beta \gamma \\
%+ 2 \alpha \gamma + 2 \beta \delta & + 2 \alpha \delta - 2 \beta \gamma & - \delta^2 - \gamma^2 + \alpha^2 + \beta^2 \\
%\end{bmatrix}
%\end{align*}
%
%
%
%
% VERSION 11
%
%
%
%
%\begin{align*}
%R &= \beta - \alpha\Be_{21} - \delta \Be_{32} + \gamma \Be_{31} \\
%\beta &= \cos\left(\frac{\theta}{2}\right) \cos\left( \inv{2}\left(\psi+\phi\right) \right) \\
%\alpha &= -\cos\left(\frac{\theta}{2}\right) \sin\left( \inv{2}\left(\psi+\phi\right) \right) \\
%\delta &= -\sin\left(\frac{\theta}{2}\right) \cos\left( \inv{2}\left(\psi-\phi\right) \right) \\
%\gamma &= \sin\left(\frac{\theta}{2}\right) \sin\left( \inv{2}\left(\psi-\phi\right) \right) \\
%\end{align*}
%
%\begin{align*}
%\begin{bmatrix}
% - \gamma^2 + \delta^2 - \alpha^2 + \beta^2 & + 2 \delta \gamma + 2 \beta \alpha & + 2 \alpha \delta - 2 \beta \gamma \\
%+ 2 \delta \gamma - 2 \beta \alpha & + \gamma^2 - \delta^2 - \alpha^2 + \beta^2 & + 2 \alpha \gamma + 2 \beta \delta \\
%+ 2 \alpha \delta + 2 \beta \gamma & + 2 \alpha \gamma - 2 \beta \delta & - \gamma^2 - \delta^2 + \alpha^2 + \beta^2 \\
%\end{bmatrix}
%\end{align*}
%
%
%
%
% VERSION 12
%
%
%
%
%\begin{align*}
%R &= \beta - \gamma\Be_{21} - \delta \Be_{32} + \alpha \Be_{31} \\
%\beta &= \cos\left(\frac{\theta}{2}\right) \cos\left( \inv{2}\left(\psi+\phi\right) \right) \\
%\gamma &= -\cos\left(\frac{\theta}{2}\right) \sin\left( \inv{2}\left(\psi+\phi\right) \right) \\
%\delta &= -\sin\left(\frac{\theta}{2}\right) \cos\left( \inv{2}\left(\psi-\phi\right) \right) \\
%\alpha &= \sin\left(\frac{\theta}{2}\right) \sin\left( \inv{2}\left(\psi-\phi\right) \right) \\
%\end{align*}
%
%\begin{align*}
%\begin{bmatrix}
% - \alpha^2 + \delta^2 - \gamma^2 + \beta^2 & + 2 \delta \alpha + 2 \beta \gamma & + 2 \gamma \delta - 2 \beta \alpha \\
%+ 2 \delta \alpha - 2 \beta \gamma & + \alpha^2 - \delta^2 - \gamma^2 + \beta^2 & + 2 \gamma \alpha + 2 \beta \delta \\
%+ 2 \gamma \delta + 2 \beta \alpha & + 2 \gamma \alpha - 2 \beta \delta & - \alpha^2 - \delta^2 + \gamma^2 + \beta^2 \\
%\end{bmatrix}
%\end{align*}
%
%
%
%
% VERSION 13
%
%
%
%
%\begin{align*}
%R &= \gamma - \alpha\Be_{21} - \beta \Be_{32} + \delta \Be_{31} \\
%\gamma &= \cos\left(\frac{\theta}{2}\right) \cos\left( \inv{2}\left(\psi+\phi\right) \right) \\
%\alpha &= -\cos\left(\frac{\theta}{2}\right) \sin\left( \inv{2}\left(\psi+\phi\right) \right) \\
%\beta &= -\sin\left(\frac{\theta}{2}\right) \cos\left( \inv{2}\left(\psi-\phi\right) \right) \\
%\delta &= \sin\left(\frac{\theta}{2}\right) \sin\left( \inv{2}\left(\psi-\phi\right) \right) \\
%\end{align*}
%
%\begin{align*}
%\begin{bmatrix}
% - \delta^2 + \beta^2 - \alpha^2 + \gamma^2 & + 2 \beta \delta + 2 \gamma \alpha & + 2 \alpha \beta - 2 \gamma \delta \\
%+ 2 \beta \delta - 2 \gamma \alpha & + \delta^2 - \beta^2 - \alpha^2 + \gamma^2 & + 2 \alpha \delta + 2 \gamma \beta \\
%+ 2 \alpha \beta + 2 \gamma \delta & + 2 \alpha \delta - 2 \gamma \beta & - \delta^2 - \beta^2 + \alpha^2 + \gamma^2 \\
%\end{bmatrix}
%\end{align*}
%
%
%
%
% VERSION 14
%
%
%
%
%\begin{align*}
%R &= \gamma - \delta\Be_{21} - \beta \Be_{32} + \alpha \Be_{31} \\
%\gamma &= \cos\left(\frac{\theta}{2}\right) \cos\left( \inv{2}\left(\psi+\phi\right) \right) \\
%\delta &= -\cos\left(\frac{\theta}{2}\right) \sin\left( \inv{2}\left(\psi+\phi\right) \right) \\
%\beta &= -\sin\left(\frac{\theta}{2}\right) \cos\left( \inv{2}\left(\psi-\phi\right) \right) \\
%\alpha &= \sin\left(\frac{\theta}{2}\right) \sin\left( \inv{2}\left(\psi-\phi\right) \right) \\
%\end{align*}
%
%\begin{align*}
%\begin{bmatrix}
% - \alpha^2 + \beta^2 - \delta^2 + \gamma^2 & + 2 \beta \alpha + 2 \gamma \delta & + 2 \delta \beta - 2 \gamma \alpha \\
%+ 2 \beta \alpha - 2 \gamma \delta & + \alpha^2 - \beta^2 - \delta^2 + \gamma^2 & + 2 \delta \alpha + 2 \gamma \beta \\
%+ 2 \delta \beta + 2 \gamma \alpha & + 2 \delta \alpha - 2 \gamma \beta & - \alpha^2 - \beta^2 + \delta^2 + \gamma^2 \\
%\end{bmatrix}
%\end{align*}
%
%
%
%
% VERSION 15
%
%
%
%
%\begin{align*}
%R &= \gamma - \delta\Be_{21} - \alpha \Be_{32} + \beta \Be_{31} \\
%\gamma &= \cos\left(\frac{\theta}{2}\right) \cos\left( \inv{2}\left(\psi+\phi\right) \right) \\
%\delta &= -\cos\left(\frac{\theta}{2}\right) \sin\left( \inv{2}\left(\psi+\phi\right) \right) \\
%\alpha &= -\sin\left(\frac{\theta}{2}\right) \cos\left( \inv{2}\left(\psi-\phi\right) \right) \\
%\beta &= \sin\left(\frac{\theta}{2}\right) \sin\left( \inv{2}\left(\psi-\phi\right) \right) \\
%\end{align*}
%
%\begin{align*}
%\begin{bmatrix}
% - \beta^2 + \alpha^2 - \delta^2 + \gamma^2 & + 2 \alpha \beta + 2 \gamma \delta & + 2 \delta \alpha - 2 \gamma \beta \\
%+ 2 \alpha \beta - 2 \gamma \delta & + \beta^2 - \alpha^2 - \delta^2 + \gamma^2 & + 2 \delta \beta + 2 \gamma \alpha \\
%+ 2 \delta \alpha + 2 \gamma \beta & + 2 \delta \beta - 2 \gamma \alpha & - \beta^2 - \alpha^2 + \delta^2 + \gamma^2 \\
%\end{bmatrix}
%\end{align*}
%
%
%
%
% VERSION 16
%
%
%
%
%\begin{align*}
%R &= \gamma - \beta\Be_{21} - \alpha \Be_{32} + \delta \Be_{31} \\
%\gamma &= \cos\left(\frac{\theta}{2}\right) \cos\left( \inv{2}\left(\psi+\phi\right) \right) \\
%\beta &= -\cos\left(\frac{\theta}{2}\right) \sin\left( \inv{2}\left(\psi+\phi\right) \right) \\
%\alpha &= -\sin\left(\frac{\theta}{2}\right) \cos\left( \inv{2}\left(\psi-\phi\right) \right) \\
%\delta &= \sin\left(\frac{\theta}{2}\right) \sin\left( \inv{2}\left(\psi-\phi\right) \right) \\
%\end{align*}
%
%\begin{align*}
%\begin{bmatrix}
% - \delta^2 + \alpha^2 - \beta^2 + \gamma^2 & + 2 \alpha \delta + 2 \gamma \beta & + 2 \beta \alpha - 2 \gamma \delta \\
%+ 2 \alpha \delta - 2 \gamma \beta & + \delta^2 - \alpha^2 - \beta^2 + \gamma^2 & + 2 \beta \delta + 2 \gamma \alpha \\
%+ 2 \beta \alpha + 2 \gamma \delta & + 2 \beta \delta - 2 \gamma \alpha & - \delta^2 - \alpha^2 + \beta^2 + \gamma^2 \\
%\end{bmatrix}
%\end{align*}
%
%
%
%
% VERSION 17
%
%
%
%
%\begin{align*}
%R &= \gamma - \beta\Be_{21} - \delta \Be_{32} + \alpha \Be_{31} \\
%\gamma &= \cos\left(\frac{\theta}{2}\right) \cos\left( \inv{2}\left(\psi+\phi\right) \right) \\
%\beta &= -\cos\left(\frac{\theta}{2}\right) \sin\left( \inv{2}\left(\psi+\phi\right) \right) \\
%\delta &= -\sin\left(\frac{\theta}{2}\right) \cos\left( \inv{2}\left(\psi-\phi\right) \right) \\
%\alpha &= \sin\left(\frac{\theta}{2}\right) \sin\left( \inv{2}\left(\psi-\phi\right) \right) \\
%\end{align*}
%
%\begin{align*}
%\begin{bmatrix}
% - \alpha^2 + \delta^2 - \beta^2 + \gamma^2 & + 2 \delta \alpha + 2 \gamma \beta & + 2 \beta \delta - 2 \gamma \alpha \\
%+ 2 \delta \alpha - 2 \gamma \beta & + \alpha^2 - \delta^2 - \beta^2 + \gamma^2 & + 2 \beta \alpha + 2 \gamma \delta \\
%+ 2 \beta \delta + 2 \gamma \alpha & + 2 \beta \alpha - 2 \gamma \delta & - \alpha^2 - \delta^2 + \beta^2 + \gamma^2 \\
%\end{bmatrix}
%\end{align*}
%
%
%
%
% VERSION 18
%
%
%
%
%\begin{align*}
%R &= \gamma - \alpha\Be_{21} - \delta \Be_{32} + \beta \Be_{31} \\
%\gamma &= \cos\left(\frac{\theta}{2}\right) \cos\left( \inv{2}\left(\psi+\phi\right) \right) \\
%\alpha &= -\cos\left(\frac{\theta}{2}\right) \sin\left( \inv{2}\left(\psi+\phi\right) \right) \\
%\delta &= -\sin\left(\frac{\theta}{2}\right) \cos\left( \inv{2}\left(\psi-\phi\right) \right) \\
%\beta &= \sin\left(\frac{\theta}{2}\right) \sin\left( \inv{2}\left(\psi-\phi\right) \right) \\
%\end{align*}
%
%\begin{align*}
%\begin{bmatrix}
% - \beta^2 + \delta^2 - \alpha^2 + \gamma^2 & + 2 \delta \beta + 2 \gamma \alpha & + 2 \alpha \delta - 2 \gamma \beta \\
%+ 2 \delta \beta - 2 \gamma \alpha & + \beta^2 - \delta^2 - \alpha^2 + \gamma^2 & + 2 \alpha \beta + 2 \gamma \delta \\
%+ 2 \alpha \delta + 2 \gamma \beta & + 2 \alpha \beta - 2 \gamma \delta & - \beta^2 - \delta^2 + \alpha^2 + \gamma^2 \\
%\end{bmatrix}
%\end{align*}
%
%
%
%
% VERSION 19
%
%
%
%
%\begin{align*}
%R &= \delta - \gamma\Be_{21} - \beta \Be_{32} + \alpha \Be_{31} \\
%\delta &= \cos\left(\frac{\theta}{2}\right) \cos\left( \inv{2}\left(\psi+\phi\right) \right) \\
%\gamma &= -\cos\left(\frac{\theta}{2}\right) \sin\left( \inv{2}\left(\psi+\phi\right) \right) \\
%\beta &= -\sin\left(\frac{\theta}{2}\right) \cos\left( \inv{2}\left(\psi-\phi\right) \right) \\
%\alpha &= \sin\left(\frac{\theta}{2}\right) \sin\left( \inv{2}\left(\psi-\phi\right) \right) \\
%\end{align*}
%
%\begin{align*}
%\begin{bmatrix}
% - \alpha^2 + \beta^2 - \gamma^2 + \delta^2 & + 2 \beta \alpha + 2 \delta \gamma & + 2 \gamma \beta - 2 \delta \alpha \\
%+ 2 \beta \alpha - 2 \delta \gamma & + \alpha^2 - \beta^2 - \gamma^2 + \delta^2 & + 2 \gamma \alpha + 2 \delta \beta \\
%+ 2 \gamma \beta + 2 \delta \alpha & + 2 \gamma \alpha - 2 \delta \beta & - \alpha^2 - \beta^2 + \gamma^2 + \delta^2 \\
%\end{bmatrix}
%\end{align*}
%
%
%
%
% VERSION 20
%
%
%
%
%\begin{align*}
%R &= \delta - \alpha\Be_{21} - \beta \Be_{32} + \gamma \Be_{31} \\
%\delta &= \cos\left(\frac{\theta}{2}\right) \cos\left( \inv{2}\left(\psi+\phi\right) \right) \\
%\alpha &= -\cos\left(\frac{\theta}{2}\right) \sin\left( \inv{2}\left(\psi+\phi\right) \right) \\
%\beta &= -\sin\left(\frac{\theta}{2}\right) \cos\left( \inv{2}\left(\psi-\phi\right) \right) \\
%\gamma &= \sin\left(\frac{\theta}{2}\right) \sin\left( \inv{2}\left(\psi-\phi\right) \right) \\
%\end{align*}
%
%\begin{align*}
%\begin{bmatrix}
% - \gamma^2 + \beta^2 - \alpha^2 + \delta^2 & + 2 \beta \gamma + 2 \delta \alpha & + 2 \alpha \beta - 2 \delta \gamma \\
%+ 2 \beta \gamma - 2 \delta \alpha & + \gamma^2 - \beta^2 - \alpha^2 + \delta^2 & + 2 \alpha \gamma + 2 \delta \beta \\
%+ 2 \alpha \beta + 2 \delta \gamma & + 2 \alpha \gamma - 2 \delta \beta & - \gamma^2 - \beta^2 + \alpha^2 + \delta^2 \\
%\end{bmatrix}
%\end{align*}
%
%
%
%
% VERSION 21
%
%
%
%
%\begin{align*}
%R &= \delta - \alpha\Be_{21} - \gamma \Be_{32} + \beta \Be_{31} \\
%\delta &= \cos\left(\frac{\theta}{2}\right) \cos\left( \inv{2}\left(\psi+\phi\right) \right) \\
%\alpha &= -\cos\left(\frac{\theta}{2}\right) \sin\left( \inv{2}\left(\psi+\phi\right) \right) \\
%\gamma &= -\sin\left(\frac{\theta}{2}\right) \cos\left( \inv{2}\left(\psi-\phi\right) \right) \\
%\beta &= \sin\left(\frac{\theta}{2}\right) \sin\left( \inv{2}\left(\psi-\phi\right) \right) \\
%\end{align*}
%
%\begin{align*}
%\begin{bmatrix}
% - \beta^2 + \gamma^2 - \alpha^2 + \delta^2 & + 2 \gamma \beta + 2 \delta \alpha & + 2 \alpha \gamma - 2 \delta \beta \\
%+ 2 \gamma \beta - 2 \delta \alpha & + \beta^2 - \gamma^2 - \alpha^2 + \delta^2 & + 2 \alpha \beta + 2 \delta \gamma \\
%+ 2 \alpha \gamma + 2 \delta \beta & + 2 \alpha \beta - 2 \delta \gamma & - \beta^2 - \gamma^2 + \alpha^2 + \delta^2 \\
%\end{bmatrix}
%\end{align*}
%
%
%
%
% VERSION 22
%
%
%
%
%\begin{align*}
%R &= \delta - \beta\Be_{21} - \gamma \Be_{32} + \alpha \Be_{31} \\
%\delta &= \cos\left(\frac{\theta}{2}\right) \cos\left( \inv{2}\left(\psi+\phi\right) \right) \\
%\beta &= -\cos\left(\frac{\theta}{2}\right) \sin\left( \inv{2}\left(\psi+\phi\right) \right) \\
%\gamma &= -\sin\left(\frac{\theta}{2}\right) \cos\left( \inv{2}\left(\psi-\phi\right) \right) \\
%\alpha &= \sin\left(\frac{\theta}{2}\right) \sin\left( \inv{2}\left(\psi-\phi\right) \right) \\
%\end{align*}
%
%\begin{align*}
%\begin{bmatrix}
% - \alpha^2 + \gamma^2 - \beta^2 + \delta^2 & + 2 \gamma \alpha + 2 \delta \beta & + 2 \beta \gamma - 2 \delta \alpha \\
%+ 2 \gamma \alpha - 2 \delta \beta & + \alpha^2 - \gamma^2 - \beta^2 + \delta^2 & + 2 \beta \alpha + 2 \delta \gamma \\
%+ 2 \beta \gamma + 2 \delta \alpha & + 2 \beta \alpha - 2 \delta \gamma & - \alpha^2 - \gamma^2 + \beta^2 + \delta^2 \\
%\end{bmatrix}
%\end{align*}
%
%
%
%
% VERSION 23
%
%
%
%
%\begin{align*}
%R &= \delta - \beta\Be_{21} - \alpha \Be_{32} + \gamma \Be_{31} \\
%\delta &= \cos\left(\frac{\theta}{2}\right) \cos\left( \inv{2}\left(\psi+\phi\right) \right) \\
%\beta &= -\cos\left(\frac{\theta}{2}\right) \sin\left( \inv{2}\left(\psi+\phi\right) \right) \\
%\alpha &= -\sin\left(\frac{\theta}{2}\right) \cos\left( \inv{2}\left(\psi-\phi\right) \right) \\
%\gamma &= \sin\left(\frac{\theta}{2}\right) \sin\left( \inv{2}\left(\psi-\phi\right) \right) \\
%\end{align*}
%
%\begin{align*}
%\begin{bmatrix}
% - \gamma^2 + \alpha^2 - \beta^2 + \delta^2 & + 2 \alpha \gamma + 2 \delta \beta & + 2 \beta \alpha - 2 \delta \gamma \\
%+ 2 \alpha \gamma - 2 \delta \beta & + \gamma^2 - \alpha^2 - \beta^2 + \delta^2 & + 2 \beta \gamma + 2 \delta \alpha \\
%+ 2 \beta \alpha + 2 \delta \gamma & + 2 \beta \gamma - 2 \delta \alpha & - \gamma^2 - \alpha^2 + \beta^2 + \delta^2 \\
%\end{bmatrix}
%\end{align*}
%
%
%
%
% VERSION 24
%
%
%
%
%\begin{align*}
%R &= \delta - \gamma\Be_{21} - \alpha \Be_{32} + \beta \Be_{31} \\
%\delta &= \cos\left(\frac{\theta}{2}\right) \cos\left( \inv{2}\left(\psi+\phi\right) \right) \\
%\gamma &= -\cos\left(\frac{\theta}{2}\right) \sin\left( \inv{2}\left(\psi+\phi\right) \right) \\
%\alpha &= -\sin\left(\frac{\theta}{2}\right) \cos\left( \inv{2}\left(\psi-\phi\right) \right) \\
%\beta &= \sin\left(\frac{\theta}{2}\right) \sin\left( \inv{2}\left(\psi-\phi\right) \right) \\
%\end{align*}
%
%\begin{align*}
%\begin{bmatrix}
% - \beta^2 + \alpha^2 - \gamma^2 + \delta^2 & + 2 \alpha \beta + 2 \delta \gamma & + 2 \gamma \alpha - 2 \delta \beta \\
%+ 2 \alpha \beta - 2 \delta \gamma & + \beta^2 - \alpha^2 - \gamma^2 + \delta^2 & + 2 \gamma \beta + 2 \delta \alpha \\
%+ 2 \gamma \alpha + 2 \delta \beta & + 2 \gamma \beta - 2 \delta \alpha & - \beta^2 - \alpha^2 + \gamma^2 + \delta^2 \\
%\end{bmatrix}
%\end{align*}
%
%%XX

   \chapter{Spherical polar coordinates}
      %
% Copyright � 2012 Peeter Joot.  All Rights Reserved.
% Licenced as described in the file LICENSE under the root directory of this GIT repository.
%

%
%
\chapter{Spherical polar coordinates}
\index{spherical polar coordinates}
\label{chap:sphericalPolar}
%\date{Nov 13, 2008.  sphericalPolar.tex}

\section{Motivation}

Reading the math intro of \citep{zeilik1998iaa}, I found the statement that the gradient in spherical polar form is:

\begin{equation}\label{eqn:sphericalPolar:20}
\begin{aligned}
\grad &=
\rcap \PD{r}{}
+\thetacap \inv{r} \PD{\theta}{}
+\phicap \inv{r \sin\theta}\PD{\phi}{}
\end{aligned}
\end{equation}

There was no picture or description showing the conventions for measurement of the angles or directions.
To clarify things and leave a margin note I decided to derive the coordinates and unit vector transformation relationships,
gradient, divergence and curl in spherical polar coordinates.

Although details for this particular result can be found in many texts,
including the excellent review article \citep{fleischCoords}, the
exercise of personally working out the details was thought to be
a worthwhile
learning exercise.  Additionally, some related ideas about rotating
frame systems seem worth exploring, and that will be done here.

\section{Notes}
\subsection{Conventions}

\imageFigure{../figures.gabook/spherical_polar}{Angles and lengths in spherical polar coordinates}{fig:spherical_polar}{0.4}

\Cref{fig:spherical_polar} illustrates the conventions used in
these notes.  By inspection, the coordinates can be read off the diagram.

\begin{equation}\label{eqn:sphericalPolar:coordinates}
\begin{aligned}
u &= r \cos\phi \\
x &= u \cos\theta = r \cos\phi \cos\theta \\
y &= u \sin\theta = r \cos\phi \sin\theta \\
z &= r \sin\phi
\end{aligned}
\end{equation}

\subsection{The unit vectors}

To calculate the unit vectors \(\rcap\), \(\thetacap\), \(\phicap\) in the spherical polar frame we need to apply two sets of rotations.  The first is a rotation
in the \(x,y\) plane, and the second in the \(x', z\) plane.

For the intermediate frame after just the \(x,y\) plane rotation we have

\begin{equation}\label{eqn:sphericalPolar:40}
\begin{aligned}
R_\theta &= \exp(-\Be_{12}\theta/2) \\
\Be_i' &= R_\theta \Be_i R_\theta^\dagger
\end{aligned}
\end{equation}

Now for the rotational plane for the \(\phi\) rotation is

\begin{equation}\label{eqn:sphericalPolar:60}
\begin{aligned}
\Be_1' \wedge \Be_3
&= (R_\theta \Be_1 R_\theta^\dagger) \wedge \Be_3 \\
&= \inv{2} ( R_\theta \Be_1 R_\theta^\dagger \Be_3 - \Be_3 R_\theta \Be_1 R_\theta^\dagger ) \\
\end{aligned}
\end{equation}

The rotor (or quaternion) \(R_\theta\) has scalar and \(\Be_{12}\) components, so it commutes with \(\Be_3\) leaving

\begin{equation}\label{eqn:sphericalPolar:80}
\begin{aligned}
\Be_1' \wedge \Be_3
&= R_\theta \inv{2} ( \Be_1 \Be_3 - \Be_3 \Be_1 ) R_\theta^\dagger \\
&= R_\theta \Be_1 \wedge \Be_3 R_\theta^\dagger \\
\end{aligned}
\end{equation}

Therefore the rotor for the second stage rotation is

\begin{equation}\label{eqn:sphericalPolar:100}
\begin{aligned}
R_\phi
&= \exp( - R_\theta \Be_1 \wedge \Be_3 R_\theta^\dagger \phi/2 ) \\
&= \sum \inv{k!} \left( - R_\theta \Be_1 \wedge \Be_3 R_\theta^\dagger \phi/2 \right)^k \\
&= R_\theta \sum \inv{k!} ( - \Be_1 \wedge \Be_3 \phi/2 )^k R_\theta^\dagger \\
&= R_\theta \exp( - \Be_{13} \phi/2 ) R_\theta^\dagger \\
\end{aligned}
\end{equation}

Composing both sets of rotations one has

\begin{equation}\label{eqn:sphericalPolar:120}
\begin{aligned}
R(\Bx)
&= R_\theta \exp( - \Be_{13} \phi/2 ) R_\theta^\dagger R_\theta \Bx R_\theta^\dagger R_\theta \exp( \Be_{13} \phi/2 ) R_\theta^\dagger \\
&= \exp( - \Be_{12} \theta/2 ) \exp( - \Be_{13} \phi/2 ) \Bx \exp( \Be_{13} \phi/2 ) \exp( \Be_{12} \theta/2 ) \\
\end{aligned}
\end{equation}

Or, more compactly

\begin{equation}\label{eqn:sphericalPolar:140}
\begin{aligned}
R(\Bx) &= R \Bx R^\dagger \\
R &= R_\theta R_\phi \\
R_\phi &= \exp(-\Be_{13}\phi/2) \\
R_\theta &= \exp(-\Be_{12}\theta/2)
\end{aligned}
\end{equation}

Application of these to the \(\{\Be_i\}\) basis produces the \(\{\rcap, \thetacap, \phicap\}\) basis.  First application
of \(R_\phi\) yields the basis vectors for the intermediate rotation.

\begin{equation}\label{eqn:sphericalPolar:160}
\begin{aligned}
\begin{array}{l l l}
{R_\phi}\Be_1 {R_\phi}^\dagger &= \Be_1 (\cos\phi + \Be_{13} \sin\phi) &= \Be_1 \cos\phi + \Be_3 \sin\phi \\
{R_\phi}\Be_2 {R_\phi}^\dagger &= \Be_2 R_\phi {R_\phi}^\dagger &= \Be_2 \\
{R_\phi}\Be_3 {R_\phi}^\dagger &= \Be_3 (\cos\phi + \Be_{13} \sin\phi) &= \Be_3 \cos\phi - \Be_1 \sin\phi \\
\end{array}
\end{aligned}
\end{equation}

Applying the second rotation to \(R_\phi(\Be_i)\) we have
\begin{equation}\label{eqn:sphericalPolar:180}
\begin{aligned}
\rcap
&= {R_\theta}( \Be_1 \cos\phi + \Be_3 \sin\phi ) {R_\theta}^\dagger \\
&=
\Be_1 \cos\phi (\cos\theta + \Be_{12} \sin\theta)
+ \Be_3 \sin\phi \\
&=
\Be_1 \cos\phi \cos\theta
+ \Be_2 \cos\phi \sin\theta
+ \Be_3 \sin\phi \\
\thetacap
&= {R_\theta} ( \Be_2 ) {R_\theta}^\dagger \\
&= \Be_2 (\cos\theta + \Be_{12} \sin\theta) \\
&= - \Be_1 \sin\theta + \Be_2 \cos\theta \\
\phicap
&= {R_\theta} ( \Be_3 \cos\phi - \Be_1 \sin\phi ) {R_\theta}^\dagger \\
&= \Be_3 \cos\phi - \Be_1 \sin\phi (\cos\theta + \Be_{12} \sin\theta) \\
&=
- \Be_1 \sin\phi \cos\theta
- \Be_2 \sin\phi \sin\theta
+ \Be_3 \cos\phi
\\
\end{aligned}
\end{equation}

In summary these are

\begin{equation}\label{eqn:sphericalPolar:unitTxSummary}
\begin{aligned}
\rcap &= \Be_1 \cos\phi \cos\theta + \Be_2 \cos\phi \sin\theta + \Be_3 \sin\phi \\
\thetacap &= - \Be_1 \sin\theta + \Be_2 \cos\theta \\
\phicap &= - \Be_1 \sin\phi \cos\theta - \Be_2 \sin\phi \sin\theta + \Be_3 \cos\phi
\end{aligned}
\end{equation}

\subsection{An alternate pictorial derivation of the unit vectors}

Somewhat more directly, \(\rcap\) can be calculated from the coordinate expression of \eqnref{eqn:sphericalPolar:coordinates}

\begin{equation}\label{eqn:sphericalPolar:200}
\begin{aligned}
\rcap
&= \inv{r} (x, y, z),
\end{aligned}
\end{equation}

which was found by inspection of the diagram.

For \(\thetacap\), again from the figure, observe that it lies in an
latitudinal plane (ie: \(x,y\) plane), and is perpendicular to the outwards radial vector in that plane.  That is

\begin{equation}\label{eqn:sphericalPolar:220}
\begin{aligned}
\thetacap
&= (\cos\theta \Be_1 + \sin\theta \Be_2) \Be_1 \Be_2 \\
\end{aligned}
\end{equation}

Lastly, \(\phicap\) can be calculated from the dual of \(\rcap \wedge \thetacap\)

\begin{equation}\label{eqn:sphericalPolar:240}
\begin{aligned}
\phicap
&= - \Be_1 \Be_2 \Be_3 (\rcap \wedge \thetacap) \\
\end{aligned}
\end{equation}

Completing the algebra for the expressions above we have
\begin{equation}\label{eqn:sphericalPolar:260}
\begin{aligned}
\rcap
&=
\cos\phi \cos\theta \Be_1
+ \cos\phi \sin\theta \Be_2
+ \sin\phi \Be_3 \\
\thetacap
&= \cos\theta \Be_2 - \sin\theta \Be_1 \\
\rcap \wedge \thetacap
%&=
%\sin\phi \sin\theta \Be_1 \Be_3
%- \cos\theta \sin\phi \Be_2 \Be_3
%+ ( \cos\phi \sin\theta^2 + \cos\phi \cos\theta^2 ) \Be_1 \Be_2  \\
&=
\sin\phi \sin\theta \Be_1 \Be_3
+ \sin\phi \cos\theta \Be_3 \Be_2
+ \cos\phi \Be_1 \Be_2 \\
\phicap
&=
- \sin\phi \cos\theta \Be_1
- \sin\phi \sin\theta \Be_2
+ \cos\phi \Be_3 %\\
\end{aligned}
\end{equation}

Sure enough this produces the same result as with the rotor logic.

The rotor approach was purely algebraically and does not have
the same reliance on pictures.  That may have
an
additional advantage
since one can then
study any frame transformations of the general form \(\{\Be_i'\} = \{ R \Be_i R^\dagger \}\), and produce results
that apply to
not only spherical polar coordinate systems but others such as the cylindrical polar.

\subsection{Tensor transformation}

Considering a linear transformation providing a mapping from one basis to another of the following form

\begin{equation}\label{eqn:sphericalPolar:280}
\begin{aligned}
f_i = \LL(e_i) = L e_i L^{-1}
\end{aligned}
\end{equation}

The coordinate representation, or Fourier decomposition, of the vectors in each of these frames is

\begin{equation}\label{eqn:sphericalPolar:300}
\begin{aligned}
x = x^i e_i = y^j f_j.
\end{aligned}
\end{equation}

Utilizing a reciprocal frame (ie: not yet requiring an orthonormal frame here), such that \(e^i \cdot e_j = {\delta^i}_j\),
then dot product provide the coordinate transformations
\begin{equation}\label{eqn:sphericalPolar:320}
\begin{aligned}
x^k e_k \cdot e^k &= y^j f_j \cdot e^k \\
y^j f_j \cdot f^i &= x^k e_k \cdot f^i \\
\implies \\
x^i &= y^j f_j \cdot e^i \\
y^i &= x^j e_j \cdot f^i
\end{aligned}
\end{equation}

The transformed reciprocal frame vectors can be expressed directly in terms of the initial reciprocal frame \(f^i = \LL(e^i)\).  Taking
dot products confirms this

\begin{equation}\label{eqn:sphericalPolar:340}
\begin{aligned}
(L e_i L^{-1}) \cdot (L e^j L^{-1})
&= \gpgradezero{ L e_i L^{-1} L e^j L^{-1} } \\
&= \gpgradezero{ L e_i e^j L^{-1} } \\
&= e_i \cdot e^j \gpgradezero{ L L^{-1} } \\
&= e_i \cdot e^j
\end{aligned}
\end{equation}

This implies that the forward and inverse coordinate transformations may be summarized as
\begin{equation}\label{eqn:sphericalPolar:360}
\begin{aligned}
y^i &= x^j e_j \cdot \LL(e^i) \\
x^i &= y^j \LL(e_j) \cdot e^i \\
\end{aligned}
\end{equation}

Or in matrix form
\begin{equation}\label{eqn:sphericalPolar:coordinateTxTensors}
\begin{aligned}
\Lor{i}{j} &= \LL(e^i) \cdot e_j \\
\ILor{i}{j} &= \LL(e_j) \cdot e^i \\
y^i &= \Lor{i}{j} x^j \\
x^i &= \ILor{i}{j} y^j
\end{aligned}
\end{equation}


The use of inverse notation is justified by the following

\begin{equation}\label{eqn:sphericalPolar:380}
\begin{aligned}
x^i &= \ILor{i}{k} y^k \\
&= \ILor{i}{k} \Lor{k}{j} x^j \\
\implies \\
\ILor{i}{k} \Lor{k}{j} &= \delta^i_j
\end{aligned}
\end{equation}

For the special case where the basis is orthonormal (\(e_i \cdot e^j = {\delta_i}^j\)), then it can be observed here that the inverse must also be the
transpose since the forward and reverse transformation tensors then differ only be a swap of indices.

On notation.  Some references such as \citep{MinahanTensors} use \(\Lor{i}{j}\) for both the forward and inverse transformations, with specific conventions
about which index is varied to distinguish the two matrices.  I have found that confusing and have instead used the explicit inverse notation
of \citep{SpenceTensors}.



\subsection{Gradient after change of coordinates}

With the transformation matrices enumerated above we are now equipped to take the gradient expressed in initial frame
\begin{equation}\label{eqn:sphericalPolar:400}
\begin{aligned}
\grad = \sum e^i \PD{x^i}{},
\end{aligned}
\end{equation}

and express it in the transformed frame.  The chain rule is required for the derivatives in terms of the transformed coordinates

\begin{equation}\label{eqn:sphericalPolar:420}
\begin{aligned}
\PD{x^i}{}
&= \PD{x^i}{y^j} \PD{y^j}{} \\
&= \Lor{j}{i} \PD{y^j}{} \\
&= \LL(e^j) \cdot e_i \PD{y^j}{} \\
&= f^j \cdot e_i \PD{y^j}{}
\end{aligned}
\end{equation}

Therefore the gradient is
\begin{equation}\label{eqn:sphericalPolar:440}
\begin{aligned}
\grad &= \sum e^i (f^j \cdot e_i) \PD{y^j}{} \\
      &= \sum f^j \PD{y^j}{} \\
\end{aligned}
\end{equation}

This gets us most of the way towards the desired result for the spherical polar gradient since all that remains is a calculation of the \(\PDi{y^j}{}\)
values for
each of the \(\rcap\), \(\thetacap\), and \(\phicap\) directions.

It is also interesting to observe (as in \citep{DenkerMaxwell}) that the gradient can also be written as

\begin{equation}\label{eqn:sphericalPolar:460}
\begin{aligned}
\grad &= \inv{f_j} \PD{y^j}{} \\
\end{aligned}
\end{equation}

Observe the similarity to the Fourier component decomposition of the vector itself \(x = f_i y^i\).  Thus, roughly speaking, the differential operator
parts of the gradient can be seen to be directional derivatives
along the directions of each of the frame vectors.

This is sufficient to read the elements of distance in each of the directions
off the figure

\begin{equation}\label{eqn:sphericalPolar:480}
\begin{aligned}
\delta \Bx \cdot \rcap &= \delta r \\
\delta \Bx \cdot \thetacap &= r \cos\phi \delta \theta \\
\delta \Bx \cdot \phicap &= r \delta \theta \\
\end{aligned}
\end{equation}

Therefore the gradient is just
\begin{equation}\label{eqn:sphericalPolar:500}
\begin{aligned}
\grad =
\rcap \PD{r}{}
+\thetacap \inv{r \cos\phi} \PD{\theta}{}
+\phicap \inv{r} \PD{\phi}{}
\end{aligned}
\end{equation}

Although this last bit has been
derived graphically, and not analytically, it does
clarify the original question of exactly angle and unit vector
conventions were intended in the text (polar angle measured from the North pole, not equator, and \(\theta\), and \(\phi\) reversed).

This was the long way to that particular result, but this has been
an exploratory treatment of frame rotation concepts that I personally
felt the need to clarity for myself.

There are still some additional details that I will explore before concluding
(including an analytic treatment of the above).

% FIXME: wrong!
%%%\subsection{Element of distance along the curves}
%%%
%%%Using the figure, one can observe that the distances along in each of the spherical polar unit vector directions are obtained in
%%%this particular case by varying each coordinate in turn.
%%%
%%%\begin{itemize}
%%%\item Along \(\rcap\), a vectorial element of distance is just
%%%
%%%\begin{align*}
%%%\delta \Bx_r
%%%&= (r + \delta r)\rcap - r\rcap \\
%%%&= \delta r \rcap.
%%%\end{align*}
%%%
%%%\item Along the \(\phicap\) direction?
%%%
%%%Here one wants the great circle path obtained by fixing \(\theta\) and \(r\).  A difference of such position vectors
%%%(in the standard basis) is
%%%
%%%\begin{align*}
%%%r
%%%\begin{bmatrix}
%%%(\cos(\phi + \delta \phi) - \cos(\phi)) \cos(\theta) \\
%%%(\cos(\phi + \delta \phi) - \cos(\phi)) \sin(\theta) \\
%%%\sin(\phi + \delta \phi) - \sin(\phi) \\
%%%\end{bmatrix}
%%%&\approx
%%%r
%%%\begin{bmatrix}
%%%-\sin(\phi) \cos(\theta) \\
%%%-\sin(\phi) \sin(\theta) \\
%%%\cos(\phi) \\
%%%\end{bmatrix}
%%%\delta \phi
%%%\end{align*}
%%%
%%%This is an unsatisfactory way to express the directed distance, since it is in terms of \(\Be_i\).  We have relationships
%%%for \(\rcap\), \(\thetacap\), \(\phicap\) in terms of \(\Be_i\) and could invert that and multiply it out, but that is going to
%%%make things even messier before things get simpler.
%%%
%%%Instead form the unit bivector for the north south oriented great circle plane through the point of interest
%%%
%%%\begin{align*}
%%%j = \rcap \wedge \phicap
%%%\end{align*}
%%%
%%%For a point \(\Bx\) we want to consider an incremental change in position along the \(\phicap\) direction.  Forming the
%%%projection and rejection from the plane we have
%%%
%%%\begin{align*}
%%%\Bx = \Bx j \inv{j} = (\Bx \cdot j) \inv{j} + (\Bx \wedge j) \inv{j} = \Bx_\parallel + \Bx_\perp
%%%\end{align*}
%%%
%%%So rotating to \(\Bx'\) and taking differences we have
%%%
%%%\begin{align*}
%%%\delta \Bx_\phi
%%%&= \Bx' - \Bx \\
%%%&= \Bx_\parallel' -\Bx_\parallel \\
%%%&= \Bx_\parallel (\exp(j\delta \phi) - 1) \\
%%%&\approx \Bx_\parallel j \delta \phi \\
%%%&= (\Bx \cdot j) \inv{j} j \delta \phi \\
%%%&= (\Bx \cdot j) \delta \phi
%%%\end{align*}
%%%
%%%\item How about along the \(\thetacap\) direction?
%%%Intuitively, one expects the magnitude to be \(r \delta \theta\).  The algebra will be exactly the same as with \(\phicap\) direction
%%%but we have a different bivector for the plane.  Let \(i = \Be_1\Be_2\) we have
%%%
%%%\begin{align*}
%%%\delta \Bx_\theta = (\Bx \cdot i) \delta \theta
%%%\end{align*}
%%%
%%%\end{itemize}
%%%
%%%Now all this is a bit inexact.  What exactly are these \(\delta\) increments?  They have to be small enough that they can be considered to be just along the
%%%time unit vectors for the rotated frame.

\section{Transformation of frame vectors vs. coordinates}

To avoid confusion it is worth noting how the frame vectors vs. the components themselves differ under
rotational transformation.

\subsection{Example.  Two dimensional plane rotation}

Consideration of the example of a pair of orthonormal unit vectors for the plane illustrates this

\begin{equation}\label{eqn:sphericalPolar:520}
\begin{aligned}
\Be_1' &= \Be_1 \exp(\Be_{12}\theta) = \Be_1 \cos\theta + \Be_2 \sin\theta \\
\Be_2' &= \Be_2 \exp(\Be_{12}\theta) = \Be_2 \cos\theta - \Be_1 \sin\theta \\
\end{aligned}
\end{equation}

Forming a matrix for the transformation of these unit vectors we have
\begin{equation}\label{eqn:sphericalPolar:540}
\begin{aligned}
\begin{bmatrix}
\Be_1' \\
\Be_2'
\end{bmatrix}
=
\begin{bmatrix}
\cos\theta & \sin\theta \\
- \sin\theta & \cos\theta \\
\end{bmatrix}
\begin{bmatrix}
\Be_1 \\
\Be_2
\end{bmatrix}
\end{aligned}
\end{equation}

Now compare this to the transformation of a vector in its entirety

\begin{equation}\label{eqn:sphericalPolar:560}
\begin{aligned}
y^1 \Be_1' + y^2 \Be_2'
&= ( x^1 \Be_1 + x^2 \Be_2 ) \exp(\Be_{12}\theta) \\
&= x^1(\Be_1 \cos\theta + \Be_2 \sin\theta)
 + x^2(\Be_2 \cos\theta - \Be_1 \sin\theta) \\
\end{aligned}
\end{equation}

If one uses the standard basis to specify both the rotated point and
the original, then taking dot products with \(\Be_i\) yields the
equivalent matrix representation

\begin{equation}\label{eqn:sphericalPolar:planeRotationAppliedToCoordinates}
\begin{aligned}
\begin{bmatrix}
y_1 \\
y_2
\end{bmatrix}
=
\begin{bmatrix}
\cos\theta & -\sin\theta \\
\sin\theta & \cos\theta \\
\end{bmatrix}
\begin{bmatrix}
x_1 \\
x_2
\end{bmatrix}
\end{aligned}
\end{equation}

Note how this inverts (transposes) the transformation matrix here
compared to the matrix for the transformation of the frame vectors.

%It should also be observed that the use of a matrix here has some subtlies.
%It is really the matrix of the linear transformation which takes
%coordinates from the initial frame to
%coordinates in the rotated frame.  The notation of \citep{damiano1988cla}
%makes this nicely explicit, and an equation like \eqnref{eqn:sphericalPolar:planeRotationAppliedToCoordinates}
%could be written as
%
%\begin{align*}
%{
%\begin{bmatrix}
%x
%\end{bmatrix}
%}_{\alpha'}
%=
%{
%\begin{bmatrix}
%R
%\end{bmatrix}
%}_{\alpha}^{\alpha'}
%{
%\begin{bmatrix}
%x
%\end{bmatrix}
%}_{\alpha}
%\end{align*}
%
%where \({\alpha} = \{\Be_i\}\) is the initial basis
%and \({\alpha}' = \{\Be_i'\}\) is the rotated basis.

\subsection{Inverse relations for spherical polar transformations}

The relations of \eqnref{eqn:sphericalPolar:unitTxSummary} can be summarized in matrix form

\begin{equation}\label{eqn:sphericalPolar:sphericalPolarRotationMatrixForFrameVectors}
\begin{aligned}
\begin{bmatrix}
\rcap \\
\thetacap \\
\phicap \\
\end{bmatrix}
&=
\begin{bmatrix}
%rcap . {e1 e2 e3}
\cos\phi \cos\theta & \cos\phi \sin\theta & \sin\phi \\
%thetacap . {e1 e2 e3}
- \sin\theta & \cos\theta & 0 \\
%phicap . {e1 e2 e3}
- \sin\phi \cos\theta & - \sin\phi \sin\theta & \cos\phi \\
\end{bmatrix}
\begin{bmatrix}
\Be_1 \\
\Be_2 \\
\Be_3 \\
\end{bmatrix}
\end{aligned}
\end{equation}

Or, more compactly
\begin{equation}\label{eqn:sphericalPolar:580}
\begin{aligned}
\begin{bmatrix}
\rcap \\
\thetacap \\
\phicap \\
\end{bmatrix}
=
\BU
\begin{bmatrix}
\Be_1 \\
\Be_2 \\
\Be_3 \\
\end{bmatrix}
\end{aligned}
\end{equation}

This composite rotation can be inverted with a transpose operation, which
becomes clear with the factorization
\begin{equation}\label{eqn:sphericalPolar:600}
\begin{aligned}
\BU
&=
\begin{bmatrix}
\cos\phi & 0 & \sin\phi \\
0 & 1 & 0 \\
-\sin\phi & 0 & \cos\phi
\end{bmatrix}
\begin{bmatrix}
\cos\theta & \sin\theta & 0 \\
-\sin\theta & \cos\theta & 0 \\
0 & 0 & 1 \\
\end{bmatrix}
\end{aligned}
\end{equation}

Thus
\begin{equation}\label{eqn:sphericalPolar:620}
\begin{aligned}
\begin{bmatrix}
\Be_1 \\
\Be_2 \\
\Be_3 \\
\end{bmatrix}
&=
% U^T :
%\begin{bmatrix}
%\cos\theta & -\sin\theta & 0 \\
%\sin\theta & \cos\theta & 0 \\
%0 & 0 & 1 \\
%\end{bmatrix}
%\begin{bmatrix}
%\cos\phi & 0 & -\sin\phi \\
%0 & 1 & 0 \\
%\sin\phi & 0 & \cos\phi
%\end{bmatrix}
%
\begin{bmatrix}
\cos\phi \cos\theta & - \sin\theta & - \sin\phi \cos\theta \\
\cos\phi \sin\theta & \cos\theta & - \sin\phi \sin\theta  \\
\sin\phi & 0 & \cos\phi \\
\end{bmatrix}
\begin{bmatrix}
\rcap \\
\thetacap \\
\phicap \\
\end{bmatrix}
\end{aligned}
\end{equation}

\subsection{Transformation of coordinate vector under spherical polar rotation}

In \eqnref{eqn:sphericalPolar:planeRotationAppliedToCoordinates} the matrix for the rotation of a coordinate vector for the plane rotation
was observed to be the transpose of the matrix that transformed the frame vectors themselves.  This is also the case
in this spherical polar case, as can be seen by forming a general vector and applying equation
\eqnref{eqn:sphericalPolar:sphericalPolarRotationMatrixForFrameVectors} to the standard basis vectors.

\begin{equation}\label{eqn:sphericalPolar:640}
\begin{aligned}
x^1 \Be_1 &\rightarrow x^1 ( \cos\phi \cos\theta \Be_1 + \cos\phi \sin\theta \Be_2 + \sin\phi \Be_3 ) \\
x^2 \Be_2 &\rightarrow x^2 ( - \sin\theta \Be_1 + \cos\theta \Be_2 ) \\
x^3 \Be_3 &\rightarrow x^3 ( - \sin\phi \cos\theta \Be_1 - \sin\phi \sin\theta \Be_2 + \cos\phi \Be_3 )
\end{aligned}
\end{equation}

Summing this and regrouping (ie: a transpose operation) one has:
\begin{equation}\label{eqn:sphericalPolar:660}
\begin{aligned}
x^i \Be_i &\rightarrow y^i \Be_i \\
& \Be_1 ( x^1 \cos\phi \cos\theta - x^2 \sin\theta - x^3 \sin\phi \cos\theta ) \\
& + \Be_2 ( x^1 \cos\phi \sin\theta + x^2 \cos\theta - x^3 \sin\phi \sin\theta ) \\
& + \Be_3 ( x^1 \sin\phi + x^3 \cos\phi ) \\
\end{aligned}
\end{equation}

taking dot products with \(\Be_i\) produces the matrix form
\begin{equation}\label{eqn:sphericalPolar:680}
\begin{aligned}
\begin{bmatrix}
y^1 \\
y^2 \\
y^3 \\
\end{bmatrix}
&=
\begin{bmatrix}
\cos\phi \cos\theta & - \sin\theta & - \sin\phi \cos\theta \\
\cos\phi \sin\theta & \cos\theta & - \sin\phi \sin\theta \\
\sin\phi & 0 & \cos\phi \\
\end{bmatrix}
\begin{bmatrix}
x^1 \\
x^2 \\
x^3 \\
\end{bmatrix} \\
&=
\begin{bmatrix}
\cos\theta & -\sin\theta & 0 \\
\sin\theta & \cos\theta & 0 \\
0 & 0 & 1 \\
\end{bmatrix}
\begin{bmatrix}
\cos\phi & 0 & -\sin\phi \\
0 & 1 & 0 \\
\sin\phi & 0 & \cos\phi
\end{bmatrix}
\begin{bmatrix}
x^1 \\
x^2 \\
x^3 \\
\end{bmatrix} \\
\end{aligned}
\end{equation}

As observed in \chapcite{PJEulerAngle} the matrix for this transformation of the coordinate vector under the composite \(x,y\) rotation followed by
an \(x', z\) rotation ends up expressed as the product of the elementary rotations, but applied in reverse order!

%%%\section{FIX or delete what comes after this}
%%%\subsection{Verification of rotation matrix, using only matrix notation}
%%%
%%%As a verification of \eqnref{eqn:sphericalPolar:sphericalPolarRotationMatrixForFrameVectors} lets calculate that directly.  The initial rotation is in the \(x,y\) plane around the \(-\Be_2' = -\Be_2 \exp(\Be_{12}\theta) = -\Be_2 \cos\theta + \Be_1 \sin\theta\) axis.
%%%
%%%From \citep{PJRotor} we have the rotation matrix for a \(\phi\) rotation about
%%%unit vector \(\Bn = (n_1, n_2, n_3) = (-\cos\theta, \sin\theta, 0) = (-C_\theta, S_\theta, 0)\) is
%%%
%%%\begin{align*}
%%%R_\phi R_\theta
%%%=
%%%\begin{bmatrix}
%%%\cos\phi(1 +{C_\theta}^2) - {C_\theta}^2 & -{C_\theta} {S_\theta} (1-\cos\phi) & {S_\theta} \sin\phi \\
%%%-{C_\theta} {S_\theta} (1-\cos\phi) & \cos\phi(1 -{S_\theta}^2) + {S_\theta}^2 & {C_\theta} \sin\phi \\
%%%-{S_\theta} \sin\phi & {-C_\theta} \sin\phi & \cos\phi \\
%%%\end{bmatrix}
%%%\begin{bmatrix}
%%%C_\theta & -S_\theta & 0 \\
%%%S_\theta & C_\theta & 0 \\
%%%0 & 0 & 1 \\
%%%\end{bmatrix}
%%%\end{align*}
%%%

   \chapter{Rotor interpolation calculation}
      %
% Copyright � 2012 Peeter Joot.  All Rights Reserved.
% Licenced as described in the file LICENSE under the root directory of this GIT repository.
%

%
%
%\chapter{Rotor interpolation calculation}
\index{rotor!interpolation}
\label{chap:slerp}
%\date{Nov 30, 2008.  slerp.tex}

The aim is to compute the interpolating rotor \(r\) that takes an object
from one position to another in \(n\) steps.
Here the initial and final positions are given by two rotors \(R_1\), and \(R_2\)
like so

\begin{equation}\label{eqn:slerp:20}
\begin{aligned}
X_1 &= R_1 X {R_1}^\dagger \\
X_2 &= R_2 X {R_2}^\dagger = r^n R_1 X {R_1}^\dagger {r^n}^\dagger
\end{aligned}
\end{equation}

So, writing

\begin{equation}\label{eqn:slerp:40}
\begin{aligned}
%r^n R_1 = R_2
a = r^n = R_2 \inv{R_1} = \frac{R_2 {R_1}^\dagger}{R_1 {{R_1}^\dagger}} = \cos\theta + I \sin\theta
\end{aligned}
\end{equation}

So,

\begin{equation}\label{eqn:slerp:60}
\begin{aligned}
\frac{\gpgradetwo{a}}{\gpgradezero{a}} &=
\frac{\gpgradetwo{a}}{\Abs{\gpgradetwo{a}}} \frac{\Abs{\gpgradetwo{a}}}{\gpgradezero{a}} \\
&= I \tan\theta
\end{aligned}
\end{equation}

Therefore the interpolating rotor is:
\begin{equation}\label{eqn:slerp:80}
\begin{aligned}
I &= \frac{\gpgradetwo{a}}{\Abs{\gpgradetwo{a}}} \\
\theta &= \atan2\left(\Abs{\gpgradetwo{a}}, \gpgradezero{a}\right) \\
r &= \cos(\theta/n) + I \sin(\theta/n)
\end{aligned}
\end{equation}

In \citep{dorst2007gac}, equation \(10.15\), they have got something like this
for a fractional angle, but then say that they do not use that in software,
instead using \(r\) directly, with a comment about designing more sophisticated
algorithms (bivector splines).  That spline comment in particular sounds
interesting.  Sounds like the details on that are to be found in the journals
mentioned in Further Reading section of chapter 10.

   \chapter{Exponential of a blade}
      %
% Copyright � 2012 Peeter Joot.  All Rights Reserved.
% Licenced as described in the file LICENSE under the root directory of this GIT repository.
%

%
%
\chapter{Exponential of a blade}
\index{blade!exponentiation}
\label{chap:kvectorExponential}
%\date{March 12, 2008.  kvectorExponential.tex}

\section{Motivation}

Exponentials of bivectors and complex numbers are useful as generators of rotations, and exponentials of
square matrices can be used in linear differential equation solution.

How about exponentials of vectors?

Because any power of a vector can be calculated it should be perfectly well defined to use the exponential infinite series with k-vector parameters.  An
exponential function of this form will be expanded explicitly and compared to the real number result.  The first
derivative will also be calculated to examine its form.

In addition for completeness, the bivector and quaternion exponential forms will be examined.

\section{Vector Exponential}

The infinite series representation of the exponential defines a function for any \(x\) that can be repeatedly multiplied
with it self.

\begin{equation}
e^x = \sum_{k=0}^{\infty} \frac{x^k}{k!}
\end{equation}

Depending on the type of the parameter \(x\) this may or may not have properties consistent with
the real number exponential function.
For a vector \(\Bx = \xcap\abs{\Bx}\), after splitting the sum into even and odd terms this infinite series takes the
following form:

\begin{equation*}
e^{\pm\Bx}
=
\sum_{k=0}^{\infty} \frac{\Bx^{2k}}{(2k)!}
\pm
\sum_{k=0}^{\infty} \frac{\abs{\Bx}^{2k}\abs{\Bx}\xcap}{(2k+1)!}
\end{equation*}

\begin{equation}
\implies
e^{\pm\Bx}
=
\cosh\abs{\Bx}
\pm
\xcap \sinh\abs{\Bx}
\end{equation}

One can also employ symmetric and antisymmetric sums to write the hyperbolic functions in terms of the
vector exponentials:

\begin{equation}\label{eqn:kvectorExponential:20}
\cosh\abs{\Bx} = \frac{e^\Bx + e^{-\Bx}}{2}
\end{equation}
\begin{equation}\label{eqn:kvectorExponential:40}
\sinh\abs{\Bx} = \frac{e^\Bx - e^{-\Bx}}{2\xcap}
\end{equation}

\subsection{Vector Exponential derivative}
One of the defining properties of the exponential is that its derivative is related to itself

\begin{equation}\label{eqn:kvectorExponential:60}
\ddu{e^{x}} = \ddu{x}e^{x} = e^{x} \ddu{x}
\end{equation}

For a vector parameter \(\Bx\) one should not generally expect that.  Let us expand this to see the form of this
derivative:

\begin{equation}\label{eqn:kvectorExponential:180}
\begin{aligned}
\ddu{e^{\Bx}}
&= \ddu{}( \cosh\abs{\Bx} + \xcap \sinh\abs{\Bx} ) \\
&= ( \sinh\abs{\Bx} + \xcap \cosh\abs{\Bx} ) \ddu{\abs{\Bx}} + \ddu{\xcap} \sinh\abs{\Bx} \\
\end{aligned}
\end{equation}

Can calculate \(\ddu{\abs{\Bx}}\) with the usual trick:

\begin{equation}\label{eqn:kvectorExponential:80}
\ddu{\abs{\Bx}^2} = 2\abs{\Bx}\ddu{\abs{\Bx}} = \ddu{\Bx}\Bx + \Bx\ddu{\Bx} = 2 \ddu{\Bx} \cdot \Bx
\end{equation}

\begin{equation}\label{eqn:kvectorExponential:100}
\implies
\ddu{\abs{\Bx}} = \ddu{\Bx} \cdot \xcap
\end{equation}

Calculation of \(\ddu{\xcap}\) uses this result:

\begin{equation}\label{eqn:kvectorExponential:200}
\begin{aligned}
\ddu{\xcap}
&= \ddu{}\frac{\Bx}{\abs\Bx}  \\
&= \ddu{\Bx}\inv{\abs\Bx} - \frac{\Bx}{\abs{\Bx}^2}\ddu{\abs{\Bx}} \\
&= \ddu{\Bx}\inv{\abs\Bx} - \frac{\Bx}{\abs{\Bx}^2} \ddu{\Bx} \cdot \xcap \\
&= \inv{\abs\Bx}\left( \ddu{\Bx} - \xcap \left(\ddu{\Bx} \cdot \xcap\right) \right) \\
&= \frac{\xcap}{\abs\Bx} \left(\xcap \wedge \ddu{\Bx} \right) \\
&= \inv{\abs\Bx} {\RejName_{\xcap}\left(\ddu{\Bx}\right)} \\
\end{aligned}
\end{equation}

Putting these together one write the derivative in a few ways:

\begin{equation}\label{eqn:kvectorExponential:220}
\begin{aligned}
\ddu{e^{\Bx}}
&=
\left(\ddu{\Bx} \cdot \xcap\right) \xcap
( \xcap \sinh\abs{\Bx} + \cosh\abs{\Bx} )
 + \frac{\xcap}{\abs\Bx} \left(\xcap \wedge \ddu{\Bx} \right) \sinh\abs{\Bx} \\
&=
\xcap \left(\ddu{\Bx} \cdot \xcap\right)e^{\Bx}
 + \frac{\xcap}{\abs\Bx} \left(\xcap \wedge \ddu{\Bx} \right) \sinh\abs{\Bx} \\
&=
\Proj_{\xcap}\left(\ddu{\Bx}\right) e^{\Bx}
 + \inv{\abs\Bx} \RejName_{\xcap}\left(\ddu{\Bx}\right) \sinh\abs{\Bx} \\
\end{aligned}
\end{equation}

This is considerably different from the real number case.  Only when the vector \(\Bx\) and all its variation
\(\ddu{\Bx}\) are colinear does \(\ddu{\Bx} = \Proj_{\xcap}\left(\ddu{\Bx}\right)\) for the real number like result:

\begin{equation}
\ddu{e^{\Bx}} = \ddu{\Bx} e^{\Bx} = e^{\Bx} \ddu{\Bx}
\end{equation}

Note that the \(\sinh\) term can be explicitly removed

\begin{equation}\label{eqn:kvectorExponential:240}
\begin{aligned}
\ddu{e^{\Bx}}
=
\left(\xcap \left(\ddu{\Bx} \cdot \xcap\right) - \inv{2\abs\Bx}\left(\xcap \wedge \ddu{\Bx} \right) \right) e^{\Bx}
 - \inv{2\abs\Bx}\left(\xcap \wedge \ddu{\Bx} \right) e^{-\Bx} \\
\end{aligned}
\end{equation}

, but without a \(\RejName_{\xcap}\left(\ddu{\Bx}\right) = 0\)
constraint, there will always be a term that is not proportional to \(e^{\Bx}\).

\section{Bivector Exponential}

The bivector exponential can be expanded utilizing its complex number equivalence:

\begin{equation}\label{eqn:kvectorExponential:260}
\begin{aligned}
e^{\BB}
&= e^{\Bcap\Babs} \\
&= \cos{\Babs} + \Bcap\sin{\Babs} \\
\end{aligned}
\end{equation}

So, taking the derivative we have

\begin{equation}\label{eqn:kvectorExponential:280}
\begin{aligned}
(e^{\BB})'
&= \left(-\sin\Babs + \Bcap\cos\Babs\right)\Babs' + \Bcap' \sin\Babs \\
&= \Bcap \left(\Bcap\sin\Babs + \cos\Babs\right)\Babs' + \Bcap' \sin\Babs \\
&= \Bcap e^{\BB} \Babs' + \Bcap' \sin\Babs \\
&= e^{\BB} \Bcap \Babs' + \Bcap' \sin\Babs \\
\end{aligned}
\end{equation}

\subsection{bivector magnitude derivative}

As with the vector case we have got a couple helper derivatives required.  Here is the first:

\begin{equation}\label{eqn:kvectorExponential:300}
\begin{aligned}
({\Babs^2})' &= 2\Babs\Babs' = -(\BB\BB' + \BB'\BB) \\
\implies \\
{\Babs}' &= -\frac{\Bcap\BB' + \BB'\Bcap}{2} \\
\end{aligned}
\end{equation}

Unlike the vector case this last expression is not a bivector dot product \(= -\Bcap\cdot \BB'\) since there could be a
\(\gpgradefour{}\) term that this symmetric sum would also include.
That wedge term would be zero for example if \(\BB = \Bx \wedge \Bk\) for a constant vector \(\Bk\).

\subsection{Unit bivector derivative}

Now calculate \(\Bcap'\):

\begin{equation}\label{eqn:kvectorExponential:320}
\begin{aligned}
\Bcap'
&= \frac{\BB'}{\Babs} - \frac{\BB}{\Babs^2}\Babs' \\
&= \inv{\Babs}\left( \BB' + {\Bcap}\frac{\Bcap\BB' + \BB'\Bcap}{2} \right) \\
&= \inv{2\Babs}\left( \BB' + \Bcap\BB'\Bcap \right) \\
&= \frac{\Bcap}{\Babs}\frac{-\Bcap\BB' + \BB'\Bcap}{2} \\
\end{aligned}
\end{equation}

Thus, the derivative is a scaled bivector rejection:
\begin{equation}
\Bcap' = \inv{\BB}\gpgradetwo{\Bcap\BB'}
\end{equation}

Although this appears different from a unit vector derivative, a slight adjustment highlights the
similarities:

\begin{equation}\label{eqn:kvectorExponential:340}
\begin{aligned}
\rcap'
&= \frac{\rcap}{\abs\Br}\rcap \wedge \Br' \\
&= \inv{\Br}\gpgradetwo{\rcap\Br'} \\
\end{aligned}
\end{equation}

Note however the sign inversion that is built into the bivector inversion.

\subsection{combining results}

Putting the individual results back together we have:

\begin{equation}
(e^{\BB})'
= \inv{\Bcap}\frac{\Bcap\BB' + \BB'\Bcap}{2} e^{\BB} + \inv{\BB}\gpgradetwo{\Bcap\BB'} \sin\Babs
\end{equation}

In general with bivectors we can have two sorts of perpendicularity.  The first is perpendicular but intersecting (generated by the grade 2 term of the product), and perpendicular with no common line (generated by the grade 4 term).
In \R{3} we have only the first sort.

With a restriction that the derivative only changes the bivector enough to introduce the first term, this
exponential derivative is reduced to:

\begin{equation}\label{eqn:kvectorExponential:360}
\begin{aligned}
(e^{\BB})'
&= \inv{\Bcap}\Bcap \cdot \BB' e^{\BB} + \inv{\BB}\gpgradetwo{\Bcap\BB'} \sin\Babs \\
&= \Proj_{\Bcap}(\BB') e^{\BB} + \inv{\Babs}{\RejName_{\Bcap}(\BB')} \sin\Babs \\
\end{aligned}
\end{equation}

Only if the bivector variation is in the same plane as the bivector itself can the \(\gpgradetwo{}\) term be dropped
in which case, since the derivative will equal its projection one has:

\begin{equation}
(e^{\BB})' = \BB' e^{\BB} = e^{\BB} \BB'
\end{equation}

\section{Quaternion exponential derivative}

Using the phrase somewhat loosely a quaternion, or complex number is a multivector of the form

\begin{equation}\label{eqn:kvectorExponential:120}
\alpha + \BB
\end{equation}

Where \(\alpha\) is a scalar, and \(\BB\) is a bivector.

Using the results above, the derivative of a quaternion exponential (ie: a rotation operator) will be

\begin{equation}\label{eqn:kvectorExponential:380}
\begin{aligned}
(e^{\alpha + \BB})'
&= (e^{\alpha})' e^{\BB} + e^{\alpha} (e^{\BB})' \\
&= \alpha' e^{\alpha + \BB} + e^{\alpha}
\inv{\Bcap}\frac{\Bcap\BB' + \BB'\Bcap}{2} e^{\BB} + \inv{\BB}\gpgradetwo{\Bcap\BB'} e^{\alpha} \sin\Babs \\
\end{aligned}
\end{equation}

For the total derivative:
\begin{equation}
(e^{\alpha + \BB})'
= \left(\alpha' + \inv{\Bcap}\frac{\Bcap\BB' + \BB'\Bcap}{2} \right) e^{\alpha + \BB}
+ \inv{\BB}\gpgradetwo{\Bcap\BB'} e^{\alpha} \sin\Babs
\end{equation}

As with the bivector case, the two restrictions
\(\gpgradetwo{\Bcap\BB'} = 0\), and \(\gpgradefour{\Bcap\BB'} = 0\)
are required to get a real number like exponential derivative:

\begin{equation}
(e^{\alpha + \BB})'
= \left(\alpha + \BB\right)' e^{\alpha + \BB}
= e^{\alpha + \BB} \left(\alpha + \BB\right)'
\end{equation}

Note that both of these are true for the important class of multivectors, the complex number.

\subsection{bivector with only one degree of freedom}

For a bivector that includes a constant vector such as \(\BB = \Bx \wedge \Bk\) there will be no
\(\gpgradefour{}\) term.

\begin{equation}\label{eqn:kvectorExponential:140}
\gpgradefour{\Bcap\BB'}
\propto \gpgradefour{\Bx \wedge \Bk \Bx' \wedge \Bk}
= \Bx \wedge \Bk \wedge \Bx' \wedge \Bk
= 0
\end{equation}

Suppose \(\alpha + \BB = \Bx\Bk = \Bx \cdot \Bk + \Bx \wedge \Bk\).  In this case
this quaternion exponential derivative reduces to

\begin{equation}\label{eqn:kvectorExponential:400}
\begin{aligned}
(e^{\Bx\Bk})'
&= \left(\Bx' \cdot \Bk + \inv{\Bx \wedge \Bk} (\Bx \wedge \Bk) \cdot (\Bx' \wedge \Bk) \right) e^{\Bx\Bk} \\
&+ \inv{\Bx \wedge \Bk}\gpgradetwo{\frac{\Bx \wedge \Bk}{\abs{\Bx \wedge \Bk}}\Bx' \wedge \Bk} e^{\Bx \cdot \Bk} \sin\abs{\Bx \wedge \Bk} \\
\end{aligned}
\end{equation}

It is only with the addition restriction that all the bivector variation lies in the plane \(\Bi = \frac{\Bx \wedge \Bk }{ \abs{\Bx \wedge \Bk} }\).  ie:

\begin{equation}\label{eqn:kvectorExponential:160}
\gpgradetwo{\Bx \wedge \Bk \Bx' \wedge \Bk} = 0
\end{equation}

does one have:

\begin{equation}\label{eqn:kvectorExponential:420}
\begin{aligned}
(e^{\Bx\Bk})'
&= \left(\Bx' \cdot \Bk + \Proj_{\Bi}(\Bx' \wedge \Bk) \right) e^{\Bx\Bk} \\
&= \left(\Bx' \cdot \Bk + \Bx' \wedge \Bk \right) e^{\Bx\Bk} \\
\end{aligned}
\end{equation}

Thus with these two restrictions to the variation of the bivector term we have:

\begin{equation}
(e^{\Bx\Bk})' = \Bx'\Bk e^{\Bx\Bk} = e^{\Bx\Bk} \Bx'\Bk
\end{equation}

   \chapter{Generator of rotations in arbitrary dimensions}
      %
% Copyright � 2012 Peeter Joot.  All Rights Reserved.
% Licenced as described in the file LICENSE under the root directory of this GIT repository.
%

%
%
%\input{../peeter_prologue.tex}

\chapter{Generator of rotations in arbitrary dimensions}
\index{rotation!generator}
\label{chap:rotationGenerator}

%\blogpage{http://sites.google.com/site/peeterjoot/math2009/rotationGenerator.pdf}
%\date{Aug 31, 2009}
%\revisionInfo{\(RCSfile: rotationGenerator.tex,v \) Last \(Revision: 1.11 \) \(Date: 2009/10/22 02:07:20 \)}

%\beginArtWithToc
\beginArtNoToc

\section{Motivation}

Eli in his recent \href{http://behindtheguesses.blogspot.com/2009/08/noncommuting-rotation-and-angular.html}{blog post on angular momentum operators} used an exponential operator to generate rotations

\begin{equation}\label{eqn:rotationGen:goo1}
\begin{aligned}
R_{\Delta\theta} = e^{\Delta\theta \ncap \cdot (\Bx \cross \spacegrad)}
\end{aligned}
\end{equation}

This is something I hhad not seen before, but is comparable to the vector shift operator expressed in terms of directional derivatives \(\Bx \cdot \spacegrad\)

\begin{equation}\label{eqn:rotationGen:goo2}
\begin{aligned}
f(\Bx + \Ba) = e^{\Ba \cdot \spacegrad} f(\Bx)
\end{aligned}
\end{equation}

The translation operator of \eqnref{eqn:rotationGen:goo2} translates easily to higher dimensions.  Of particular interest is the Minkowski metric 4D spacetime case, where we can use the four gradient \(\grad = \gamma^\mu \partial_\mu\), and a vector spacetime translation of \(x = x^\mu \gamma_\mu \rightarrow (x^\mu + a^\mu) \gamma_\mu\) to translate ``trivially'' translate this

\begin{equation}\label{eqn:rotationGen:goo3}
\begin{aligned}
f(x + a) = e^{a \cdot \grad} f(x)
\end{aligned}
\end{equation}

Since we do not have a cross product of two vectors in a 4D space, re-expressing \eqnref{eqn:rotationGen:goo1} in a form that is not tied to three dimensions is desirable.  A duality transformation with \(\ncap = i \Be_1 \Be_2 \Be_3\) accomplishes this, where \(i\) is a unit bivector for the plane perpendicular to \(\ncap\) (i.e. product of two perpendicular unit vectors in the plane).  That duality transformation, expressing the rotation direction using an oriented plane instead of the normal to the plane gives us

\begin{equation}\label{eqn:rotationGenerator:64}
\begin{aligned}
\ncap \cdot (\Bx \cross \spacegrad)
&=
\gpgradezero{ \ncap (\Bx \cross \spacegrad) } \\
&=
\gpgradezero{ (i \Be_1 \Be_2 \Be_3) (-\Be_1 \Be_2 \Be_3) (\Bx \wedge \spacegrad) } \\
&=
\gpgradezero{ i (\Bx \wedge \spacegrad) } \\
\end{aligned}
\end{equation}

This is just \(i \cdot (\Bx \wedge \spacegrad)\), so the generator of the rotation in 3D is

\begin{equation}\label{eqn:rotationGen:goo4}
\begin{aligned}
R_{\Delta\theta} = e^{\Delta\theta i \cdot (\Bx \wedge \spacegrad)}
\end{aligned}
\end{equation}

It is reasonable to guess then that we could substitute the spacetime gradient and allow \(i\) to be any 4D unit spacetime bivector, where a spacelike product pair will generate rotations and a spacetime bivector will generate boosts.  That is really just a notational shift, and we would write

\begin{equation}\label{eqn:rotationGen:goo5}
\begin{aligned}
R_{\Delta\theta} = e^{\Delta\theta i \cdot (x \wedge \grad)}
\end{aligned}
\end{equation}

This is very likely correct, but building up to this guess in a logical sequence from a known point will be the aim of this particular exploration.

\section{Setup and conventions}

Rather than expressing the rotation in terms of coordinates, here the rotation will be formulated in terms of dual sided multivector operators (using Geometric Algebra) on vectors.  Then employing the chain rule an examination of the differential change of a multivariable scalar valued function on the underlying rotation will be made.

Following conventions of \citep{doran2003gap} vectors will be undecorated rather than boldface since we are deriving results applicable to four vector (and higher) spaces, and not requiring an Euclidean metric.

\imageFigure{../../figures/gabook/rotationGen}{Rotating vector in the plane with bivector i}{fig:rotationGen:rotationGen}{0.4}

The \cref{fig:rotationGen:rotationGen} has a pair of vectors related by rotation, where the vector \(x(\theta)\) is rotated to \(y(\theta) = x(\theta + \Delta\theta)\). We choose here to express this rotation using a quaternion-ic operator \(R = \alpha + a b\), where \(\alpha\) is a scalar and \(a\), and \(b\) are vectors.

\begin{equation}\label{eqn:rotationGen:foo1}
\begin{aligned}
y = \tilde{R} x R
\end{aligned}
\end{equation}

Required of \(R\) is an invertability property, but without loss of generality we can impose a strictly unitary property \(\tilde{R} R = 1\).  Here \(\tilde{R}\) denotes the multivector reverse of a Geometric product

\begin{equation}\label{eqn:rotationGen:foo2}
\begin{aligned}
(a b)^{\tilde{}} = \tilde{b} \tilde{a}
\end{aligned}
\end{equation}

Where for individual vectors the reverse is itself \(\tilde{a} = a\).  A singly parametrized rotation or boost can be conveniently expressed using the half angle exponential form

\begin{equation}\label{eqn:rotationGen:foo3}
\begin{aligned}
R = e^{i \theta/2}
\end{aligned}
\end{equation}

where \(i = \hat{u}\hat{v}\) is a unit bivector, a product of two perpendicular unit vectors (\(\hat{u} \hat{v} = - \hat{v} \hat{u}\)).  For rotations \(\hat{u}\), and \(\hat{v}\) are both spatial vectors, implying \(i^2 = -1\).  For boosts \(i\) is the product of a unit timelike vector and unit spatial vector, and with a Minkowski metric condition \(\hat{u}^2 \hat{v}^2 = -1\), we have a positive square \(i^2 = 1\) for our spacetime rotation plane \(i\).

A general Lorentz transformation, containing a composition of rotations and boosts can be formed by application of successive transformations

\begin{equation}\label{eqn:rotationGen:foo4}
\begin{aligned}
\LL(x) = (\tilde{U} (\tilde{T} \cdots (\tilde{S} x S) T )\cdots U) = \tilde{U} \tilde{T} \cdots \tilde{S} x S T \cdots U
\end{aligned}
\end{equation}

The composition still has the unitary property \((S T \cdots U)^{\tilde{}} S T \cdots U = 1\), so when the specifics of the parametrization are not required we will allow the rotation operator \(R = S T \cdots U\) to be a general composition of individual rotations and boosts.

We will have brief use of coordinates and employ a reciprocal basis pair \(\{\gamma^\mu\}\) and \(\{\gamma_\nu\}\) where \(\gamma^\mu \cdot \gamma_\nu = {\delta^{\mu}}_\nu\).  A vector, employing summation convention, is then denoted

\begin{equation}\label{eqn:rotationGen:foo5}
\begin{aligned}
x = x^\mu \gamma_\mu = x_\mu \gamma^\mu
\end{aligned}
\end{equation}

Where
\begin{equation}\label{eqn:rotationGen:foo6}
\begin{aligned}
x_\mu &= x \cdot \gamma_\mu \\
x_\mu &= x \cdot \gamma_\mu
\end{aligned}
\end{equation}

Shorthand for partials

\begin{equation}\label{eqn:rotationGen:foo7}
\begin{aligned}
\partial_\mu &\equiv \frac{\partial}{\partial x^\mu} \\
\partial^\mu &\equiv \frac{\partial}{\partial x_\mu}
\end{aligned}
\end{equation}

will allow the gradient to be expressed as

\begin{equation}\label{eqn:rotationGen:foo8}
\begin{aligned}
\grad \equiv \gamma^\mu \partial_\mu = \gamma_\mu \partial^\mu
\end{aligned}
\end{equation}

The perhaps unintuitive mix of upper and lower indices is required to make the indices in the direction derivative come out right when expressed as a dot product

\begin{equation}\label{eqn:rotationGen:foo9}
\begin{aligned}
\lim_{\tau \rightarrow 0} \frac{f(x + a\tau) - f(x)}{\tau} = a^\mu \partial_\mu f(x) = a \cdot \grad f(x)
\end{aligned}
\end{equation}

\section{Rotor examples}

While not attempting to discuss the exponential rotor formulation in any depth, at least illustrating by example for a spatial rotation and Lorentz boost seems called for.

Application of either of these is most easily performed with a split of the vector into components parallel and perpendicular to the ``plane'' of rotation \(i\).  For example suppose we decompose a vector \(x = p + n\) where \(n\) is perpendicular to the rotation plane \(i\) (i.e. \(n i = i n\)), and \(p\) is the components in the plane (\(p i = - i p\)).  A consequence is that \(n\) commutes with \(R\) and \(p\) induces a conjugate effect in the rotor

\begin{equation}\label{eqn:rotationGenerator:84}
\begin{aligned}
\tilde{R} x R
&=
e^{-i \theta/2} (p + n) e^{i \theta/2} \\
&=
p e^{i \theta/2} e^{i \theta/2}
+
n e^{-i \theta/2} e^{i \theta/2} \\
\end{aligned}
\end{equation}

This is then just
\begin{equation}\label{eqn:rotationGen:foo30}
\begin{aligned}
\tilde{R} x R
&=
p e^{i \theta} + n
\end{aligned}
\end{equation}

To expand any further the metric details are required.  The half angle rotors of \eqnref{eqn:rotationGen:foo3} can be expanded in series, where the metric properties of the bivector dictate the behavior.  In the spatial bivector case, where \(i^2 = -1\) we have

\begin{equation}\label{eqn:rotationGen:foo31}
\begin{aligned}
R = e^{i \theta/2} = \cos(\theta/2) + i\sin(\theta/2)
\end{aligned}
\end{equation}

whereas when \(i^2 = 1\), the series expansion yields a hyperbolic pair

\begin{equation}\label{eqn:rotationGen:foo32}
\begin{aligned}
R = e^{i \theta/2} = \cosh(\theta/2) + i\sinh(\theta/2)
\end{aligned}
\end{equation}

To make things more specific, and relate to the familiar, consider a rotation in the Euclidean \(x,y\) plane where we pick \(i = \Be_1 \Be_2\), and rotate \(\Bx = x \Be_1 + y \Be_2 + z \Be_3\).  Applying \eqnref{eqn:rotationGen:foo30}, and \eqnref{eqn:rotationGen:foo31} we have

\begin{equation}\label{eqn:rotationGenerator:104}
\begin{aligned}
\tilde{R} \Bx R
&=
(x \Be_1 + y \Be_2) (\cos\theta + \Be_1 \Be_2 \sin\theta) + z \Be_3 \\
\end{aligned}
\end{equation}

We have \({\Be_1}^2 = {\Be_2}^2 = 1\) and \(\Be_1 \Be_2 = -\Be_1 \Be_2\), so with some rearrangement

\begin{equation}\label{eqn:rotationGenerator:124}
\begin{aligned}
\tilde{R} \Bx R
&=
\Be_1 (x \cos\theta - y \sin\theta )
+\Be_2 (x \sin\theta + y \cos\theta )
+ \Be_3 z
\end{aligned}
\end{equation}

This is the familiar \(x,y\) plane rotation up to a possible sign preference.  Observe that we have the flexibility to adjust the sign of the rotation by altering either \(\theta\) or \(i\) (we could use \(i = \Be_2 \Be_1\) for example).  Because of this Hestenes \citep{hestenes1999nfc} chooses to make the angle bivector valued, so instead of \(i\theta\) writes

\begin{equation}\label{eqn:rotationGen:foo33}
\begin{aligned}
R = e^{B}
\end{aligned}
\end{equation}

where \(B\) is bivector valued, and thus contains the sign or direction of the rotation or boost as well as the orientation.

% cody:
%1 877 977 4601

For completeness lets also expand a rotor application for an x-axis boost in the spacetime plane \(i = \gamma_1 \gamma_0\).  Following \citep{doran2003gap}, we use the \((+,-,-,-)\) metric convention \(1 = {\gamma_0}^2 = -{\gamma_1}^2 = -{\gamma_2}^2 = -{\gamma_3}^2\).  Switching variable conventions to match the norm lets use \(\alpha\) for the rapidity angle, with x-axis boost rotor

\begin{equation}\label{eqn:rotationGen:foo34}
\begin{aligned}
R = e^{\gamma_1 \gamma_0 \alpha/2}
\end{aligned}
\end{equation}

for the rapidity angle \(\alpha\).  The rotor application then gives

\begin{equation}\label{eqn:rotationGenerator:144}
\begin{aligned}
\LL(x)
&= \tilde{R} (x^0 \gamma_0 + x^1 \gamma_1 + x^2 \gamma_2 + x^3 \gamma_3 ) R \\
&=
\tilde{R} (x^0 \gamma_0 + x^1 \gamma_1) R + x^2 \gamma_2 + x^3 \gamma_3 \\
&=
(x^0 \gamma_0 + x^1 \gamma_1) (\cosh(\theta) + \gamma_1 \gamma_0 \sinh(\theta/2)) + x^2 \gamma_0 + x^3 \gamma_3 \\
\end{aligned}
\end{equation}

A final bit of rearrangement yields the familiar

\begin{equation}\label{eqn:rotationGen:foo35}
\begin{aligned}
\LL(x)
&=
\gamma_0 (x^0 \cosh(\theta) - x^1 \sinh(\theta/2)) \\
&\qquad + \gamma_1 ( -x^0 \sinh(\theta/2) + x^1 \cosh(\theta))
+ x^2 \gamma_0 + x^3 \gamma_3
\end{aligned}
\end{equation}

Again observe the flexibility to adjust the sign as desired by either the bivector orientation or the sign of the scalar rapidity angle.

\section{The rotation operator}

Moving on to the guts.  From \eqnref{eqn:rotationGen:foo1} we can express \(x\) in terms of \(y\) using the inverse transformation

\begin{equation}\label{eqn:rotationGen:foo10}
\begin{aligned}
x = R y \tilde{R}
\end{aligned}
\end{equation}

Assuming \(R\) is parametrized by \(\theta\), and that both \(x\) and \(y\) are not directly dependent on \(\theta\), we have

\begin{equation}\label{eqn:rotationGenerator:164}
\begin{aligned}
\frac{dx}{d\theta}
&=
\frac{d R}{d \theta} y \tilde{R} + R y \frac{d \tilde{R} }{d\theta} \\
&=
\left(\frac{d R}{d \theta} \tilde{R} \right) (R y \tilde{R}) + (R y \tilde{R}) \left( R \frac{d \tilde{R} }{d\theta} \right) \\
&=
\left(\frac{d R}{d \theta} \tilde{R} \right) x + x \left( R \frac{d \tilde{R} }{d\theta} \right) \\
\end{aligned}
\end{equation}

Since we also have \(R \tilde{R} = 1\), this product has zero derivative

\begin{equation}\label{eqn:rotationGenerator:184}
\begin{aligned}
0 = \frac{d (R \tilde{R})}{d\theta} = \frac{d R}{d\theta} \tilde{R} + R \frac{d \tilde{R}}{d\theta}
\end{aligned}
\end{equation}

Labeling one of these, say

\begin{equation}\label{eqn:rotationGen:foo11}
\begin{aligned}
\Omega \equiv \frac{d R}{d\theta} \tilde{R}
\end{aligned}
\end{equation}

The multivector \(\Omega\) must in fact be a bivector.  As the product of a grade \(0,2\) multivector with another \(0,2\) multivector, the product may have grades \(0,2,4\).  Since reversing \(\Omega\) negates it, this product can only have grade 2 components.  In particular, employing the exponential representation of \(R\) from \eqnref{eqn:rotationGen:foo3} for a simply parametrized rotation (or boost), we have

\begin{equation}\label{eqn:rotationGen:foo16}
\begin{aligned}
\Omega = \frac{i}{2} e^{i \theta/2} e^{-i \theta/2} = \frac{i}{2}
\end{aligned}
\end{equation}

With this definition we have a

complete description of the incremental (first order) rotational along the curve from \(x\) to \(y\) induced by \(R\) via the commutator of this bivector \(\Omega\) with the initial position vector \(x\).

\begin{equation}\label{eqn:rotationGen:foo12}
\begin{aligned}
\frac{dx}{d\theta} &= \antisymmetric{\Omega}{x} = \inv{2} (i x - x i)
\end{aligned}
\end{equation}

This commutator is in fact the generalize bivector-vector dot product \(\antisymmetric{\Omega}{x} = i \cdot x\), and is vector valued.

Now consider a scalar valued function \(f = f(x(\theta))\).  Employing the chain rule, for the \(theta\) derivative of \(f\) we have a contribution from each coordinate \(x^\mu\).  That is

\begin{equation}\label{eqn:rotationGenerator:204}
\begin{aligned}
\frac{df}{d\theta}
&= \sum_\mu \frac{d x^\mu}{d \theta} \frac{\partial f}{\partial x^\mu}  \\
&= \frac{d x^\mu}{d \theta} \partial_\mu f \\
&= \left( \frac{d x^\mu}{d \theta} \gamma_\mu \right) \cdot \left( \gamma^\nu \partial_\nu \right)  f \\
\end{aligned}
\end{equation}

But this is just
\begin{equation}\label{eqn:rotationGen:foo13}
\begin{aligned}
\frac{df}{d\theta} &= \frac{d x}{d \theta} \cdot \grad f
\end{aligned}
\end{equation}

Or in operator form

\begin{equation}\label{eqn:rotationGen:foo14}
\begin{aligned}
\frac{d}{d\theta} &= (i \cdot x) \cdot \grad
\end{aligned}
\end{equation}

The complete Taylor expansion of \(f(\theta) = f(x(\theta))\) is therefore

\begin{equation}\label{eqn:rotationGenerator:224}
\begin{aligned}
f(x(\theta + \Delta\theta))
&=
\sum_{k=0}^\infty \inv{k!} \left( \Delta\theta \frac{d}{d\theta} \right)^k
f(x(\theta)) \\
&=
\sum_{k=0}^\infty \inv{k!} \left( \Delta\theta (i \cdot x) \cdot \grad \right)^k
f(x(\theta))
\end{aligned}
\end{equation}

Expressing this sum formally as an exponential we have

\begin{equation}\label{eqn:rotationGen:foo20}
\begin{aligned}
f(x(\theta + \Delta\theta)) = e^{\Delta\theta ( i \cdot x) \cdot \grad} f(x(\theta))
\end{aligned}
\end{equation}

In this form, the product \((i \cdot x) \cdot \grad\) does not look much like the cross or wedge product representations of the angular momentum operator that was initially guessed at.  Referring to \cref{fig:rotationGen:bivectorDot} let us make a couple observations about this particular form before translating back to the wedge formulation.

\imageFigure{../../figures/gabook/bivectorDot}{Bivector dot product with vector}{fig:rotationGen:bivectorDot}{0.4}

It is worth pointing out that any bivector has no unique vector factorization.  For example any of the following are equivalent

\begin{equation}\label{eqn:rotationGenerator:244}
\begin{aligned}
i
&= \ucap \wedge \vcap \\
&= (2 \ucap) \wedge (\vcap/2 + \alpha \ucap) \\
&= \inv{\alpha b - \beta a} (\alpha \ucap + \beta \vcap ) \wedge (a \ucap + b \vcap)
\end{aligned}
\end{equation}

For this reason if we factor a bivector into two vectors within the plane we are free to pick one of these in any direction we please and can pick the other in one of the perpendiculars within the plane.  In the figure exactly this was done, factoring the bivector into two perpendicular vectors \(i = \ucap \vcap\), where \(\ucap\) was picked to be in the direction of the projection of the vector \(\Bx\) onto the plane spanned by \(\{\ucap, \vcap\}\).  Suppose that projection of \(\Bx\) onto the plane is \(\alpha \ucap\).  We then have for the bivector vector dot product

\begin{equation}\label{eqn:rotationGenerator:264}
\begin{aligned}
i \cdot \Bx
&=
(\ucap\vcap) \cdot (\alpha \ucap) \\
&=
\alpha \ucap\vcap \ucap \\
&=
-\alpha
\mathLabelBox
[
   labelstyle={below of=m\themathLableNode, below of=m\themathLableNode}
]
{\ucap\ucap}{\(=1\)} \vcap \\
\end{aligned}
\end{equation}

So we have for the dot product \(i \cdot \Bx = - \alpha \vcap\), a rotation in the plane of the projection of the vector \(\Bx\) onto the plane by 90 degrees.  The direction of the rotation is metric dependent, and a spatially positive metric was used in this example.  Observe that the action of a bivector product on a vector, provided that vector is in the plane spanned by the factors of the bivector is very much like the complex imaginary action.  In both cases we have a 90 degree rotation.  This complex number correspondence is not entirely equivalent though, since we also have \(i \cdot \Bx = -\Bx \cdot i\), a negation on reversal of the product ordering, whereas we do not have to worry about commuting the imaginary of complex arithmetic.

This shows how the bivector dot product naturally encodes a rotation.  We could leave things this way, but we also want to see how to put this in a more ``standard'' form.  This is possible by rewriting the scalar product using a scalar grade selection operator.  Also employing the cyclic reordering identity \(\gpgradezero{ a b c } = \gpgradezero{ b c a}\), we have

\begin{equation}\label{eqn:rotationGenerator:284}
\begin{aligned}
( i \cdot x) \cdot \grad
&=
\inv{2} \gpgradezero{ ( i x - x i) \grad } \\
&=
\inv{2} \gpgradezero{ i x \grad - \grad x i } \\
\end{aligned}
\end{equation}

A pause is required to note that this reordering needs to be interpreted with \(x\) fixed with respect to the gradient so that the gradient is acting only to the extreme right.  Then we have

\begin{equation}\label{eqn:rotationGen:foo17}
\begin{aligned}
( i \cdot x) \cdot \grad
&=
\inv{2} \gpgradezero{ i (x \cdot \grad) - (x \cdot \grad) i } + \inv{2} \gpgradezero{ i (x \wedge \grad) + (x \cdot \grad) i }
\end{aligned}
\end{equation}

The rightmost action of the gradient allows the gradient dot and wedge products to be reordered (with interchange of sign for the wedge).  The product in the first scalar selector has only bivector terms, so we are left with

\begin{equation}\label{eqn:rotationGen:foo18}
\begin{aligned}
( i \cdot x) \cdot \grad
&=
i \cdot (x \wedge \grad)
\end{aligned}
\end{equation}

and the rotation operator takes the postulated form

\begin{equation}\label{eqn:rotationGen:foo19}
\begin{aligned}
f(x(\theta + \Delta\theta)) = e^{\Delta\theta i \cdot (x \wedge \grad)} f(x(\theta))
\end{aligned}
\end{equation}

While the cross product formulation of this is fine for 3D, this works in a plane when desired, as well as higher dimensional spaces as well as optionally non-Euclidean spaces like the Minkowski space required for electrodynamics and relativity.

\section{Coordinate expansion}

We have seen the structure of the scalar angular momentum operator of \eqnref{eqn:rotationGen:foo18} in the context of components of the cross product angular momentum operator in 3D spaces.  For a more general space what do we have?

Let \(i = \gamma_\beta \gamma_\alpha\), then we have

\begin{equation}\label{eqn:rotationGenerator:304}
\begin{aligned}
i \cdot (x \wedge \grad)
&=
(\gamma_\beta \wedge \gamma_\alpha) \cdot (\gamma^\mu  \wedge \gamma^\nu) x_\mu \partial_\nu \\
&=
({\delta_\beta}^\nu {\delta_\alpha}^\mu - {\delta_\beta}^\mu {\delta_\alpha}^\nu ) x_\mu \partial_\nu \\
\end{aligned}
\end{equation}

which is

\begin{equation}\label{eqn:rotationGen:foo40}
\begin{aligned}
(\gamma_\beta \wedge \gamma_\alpha) \cdot (x \wedge \grad) &= x_\alpha \partial_\beta -x_\beta \partial_\alpha
\end{aligned}
\end{equation}

In particular, in the four vector Minkowski space, when the pair \(\alpha,\beta\) includes both space and time indices we loose (or gain) negation in this operator sum. For example with \(i = \gamma_1 \gamma_0\), we have

\begin{equation}\label{eqn:rotationGen:foo41}
\begin{aligned}
(\gamma_1 \wedge \gamma_0) \cdot (x \wedge \grad)
&= x^0 \frac{\partial}{\partial x^1} + x^1 \frac{\partial}{\partial x^0}
\end{aligned}
\end{equation}

We can also generalize the coordinate expansion of \eqnref{eqn:rotationGen:foo40} to a more general plane of rotation.  Suppose that \(u\) and \(v\) are two perpendicular unit vectors in the plane of rotation.  For this rotational plane we have \(i=u v = u \wedge v\), and our expansion is

\begin{equation}\label{eqn:rotationGenerator:324}
\begin{aligned}
i \cdot (x \wedge \grad)
&=
(\gamma_\beta \wedge \gamma_\alpha) \cdot (\gamma^\mu  \wedge \gamma^\nu) u^\beta v^\alpha x_\mu \partial_\nu \\
&=
({\delta_\beta}^\nu {\delta_\alpha}^\mu - {\delta_\beta}^\mu {\delta_\alpha}^\nu ) u^\beta v^\alpha x_\mu \partial_\nu \\
\end{aligned}
\end{equation}

So we have

\begin{equation}\label{eqn:rotationGen:foo42}
\begin{aligned}
i \cdot (x \wedge \grad)
&=
(u^\nu v^\mu - u^\mu v^\nu) x_\mu \partial_\nu
\end{aligned}
\end{equation}

This scalar antisymmetric mixed index object is apparently called a vierbien (not a tensor) and written

\begin{equation}\label{eqn:rotationGen:foo43}
\begin{aligned}
\epsilon^{\nu\mu} = (u^\nu v^\mu - u^\mu v^\nu)
\end{aligned}
\end{equation}

It would be slightly prettier to raise the index on \(x^\mu\) (and correspondingly lower the \(\mu\)s in \(\epsilon\)).  We then have a completely non Geometric Algebra representation of the angular momentum operator for higher dimensions (and two dimensions) as well as for the Minkowski (and other if desired) metrics.

\begin{equation}\label{eqn:rotationGen:foo44}
\begin{aligned}
i \cdot (x \wedge \grad)
&=
{\epsilon^{\nu}}_\mu x^\mu \partial_\nu
\end{aligned}
\end{equation}

\section{Matrix treatment}

It should be more accessible to do the same sort of treatment with matrices than the Geometric Algebra approach.  It did not occur to me to try it that way initially, and it is worthwhile to do a comparative derivation.  Setup should be similar

\begin{equation}\label{eqn:rotationGen:boo1}
\begin{aligned}
\By &= R \Bx \\
\Bx &= R^\T \By
\end{aligned}
\end{equation}

Taking derivatives we then have

\begin{equation}\label{eqn:rotationGenerator:344}
\begin{aligned}
\frac{d \Bx}{d\theta}
&= \frac{d R^\T}{d\theta} \By \\
&= \frac{d R^\T}{d\theta} R R^\T \By \\
&= \left( \frac{d R^\T}{d\theta} R \right) \Bx \\
\end{aligned}
\end{equation}

Introducing an \(\Omega = (dR^\T/d\theta) R\) very much like before we can write this

\begin{equation}\label{eqn:rotationGen:boo2}
\begin{aligned}
\frac{d \Bx}{d\theta} = \Omega \Bx
\end{aligned}
\end{equation}

For Euclidean spaces (where \(R^{-1} = R^\T\) as assumed above), we have \(R^\T R = 1\), and thus

\begin{equation}\label{eqn:rotationGen:boo3}
\begin{aligned}
\Omega = \frac{dR^\T}{d\theta} R = -R^\T \frac{dR}{d\theta}
\end{aligned}
\end{equation}

Transposition shows that this matrix \(\Omega\) is completely antisymmetric since we have

\begin{equation}\label{eqn:rotationGen:boo4}
\begin{aligned}
\Omega^\T = -\Omega
\end{aligned}
\end{equation}

Now, is there a convenient formulation for a general plane rotation in matrix form, perhaps like the Geometric exponential form?  Probably can be done, but considering an x,y plane rotation should give the rough idea.

\begin{equation}\label{eqn:rotationGen:boo5}
\begin{aligned}
R_\theta =
\begin{bmatrix}
\cos\theta & -\sin\theta & 0 \\
\sin\theta &  \cos\theta & 0 \\
0 & 0 & 1 \\
\end{bmatrix}
\end{aligned}
\end{equation}

After a bit of algebra we have

\begin{equation}\label{eqn:rotationGen:boo6}
\begin{aligned}
\Omega =
\begin{bmatrix}
0 & 1 & 0 \\
-1 & 0 & 0 \\
0 & 0 & 0 \\
\end{bmatrix}
\end{aligned}
\end{equation}

In general we must have

\begin{equation}\label{eqn:rotationGen:boo7}
\begin{aligned}
\Omega =
\begin{bmatrix}
0 & -c & b \\
c & 0 & -a \\
b & a & 0 \\
\end{bmatrix}
\end{aligned}
\end{equation}

For some \(a,b,c\).  This procedure is not intrinsically three dimension, but in the specific 3D case, we can express this antisymetrization using the cross product.  Writing \(\ncap = (a,b,c)\) for the vector with these components, we have in the 3D case only

\begin{equation}\label{eqn:rotationGen:boo8}
\begin{aligned}
\Omega \Bx = \ncap \cross \Bx
\end{aligned}
\end{equation}

The first order rotation of a function \(f(\Bx(\theta))\) now follows from the chain rule as before

\begin{equation}\label{eqn:rotationGenerator:364}
\begin{aligned}
\frac{df}{d\theta}
&=
\frac{d x^m}{d\theta}
\frac{\partial f}{\partial x^m}
\\
&=
\frac{d \Bx}{d\theta} \cdot \spacegrad f
\\
&=
(\Bn \cross \Bx) \cdot \spacegrad f
\\
\end{aligned}
\end{equation}

We have then for the first order rotation derivative operator in 3D

\begin{equation}\label{eqn:rotationGen:boo9}
\begin{aligned}
\frac{d}{d\theta} &= \Bn \cdot (\Bx \cross \spacegrad)
\end{aligned}
\end{equation}

For higher (or 2D) spaces one cannot use the cross product so a more general expression of the result \eqnref{eqn:rotationGen:boo9} would be

\begin{equation}\label{eqn:rotationGen:boo10}
\begin{aligned}
\frac{d}{d\theta} &= (\Omega \Bx) \cdot \spacegrad
\end{aligned}
\end{equation}

Now, in this outline was a fair amount of cheating.  We know that \(\ncap\) is the unit normal to the rotational plane, but that has not been shown here.  Instead it was a constructed quantity just pulled out of thin air knowing it would be required.  If one were interested in pursuing a treatment of the rotation generator operator strictly using matrix algebra, that would have to be considered.  More troublesome and non-obvious is how this would translate to other metric spaces, where we do not necessarily have the transpose relationships to exploit.

%\EndArticle

   \chapter{Spherical Polar unit vectors in exponential form}
      %
% Copyright � 2012 Peeter Joot.  All Rights Reserved.
% Licenced as described in the file LICENSE under the root directory of this GIT repository.
%

%
%
%\input{../peeter_prologue.tex}

%\chapter{Spherical Polar unit vectors in exponential form}
\index{spherical polar unit vectors}
\label{chap:sphericalPolarUnit}

%\blogpage{http://sites.google.com/site/peeterjoot/math2009/sphericalPolarUnit.pdf?revision=5}
%\date{Sept 20, 2009 \(RCSfile: sphericalPolarUnit.tex,v \) Last \(Revision: 1.10 \) \(Date: 2009/10/26 03:56:52 \)}
%%\date{Sept 20, 2009}
%%\revisionInfo{\(RCSfile: sphericalPolarUnit.tex,v \) Last \(Revision: 1.10 \) \(Date: 2009/10/26 03:56:52 \)}

\beginArtWithToc
%\beginArtNoToc

\section{Motivation}

In \citep{gabookII:qmAngularMom} I blundered on a particularly concise exponential non-coordinate form for the unit vectors in a spherical polar coordinate system.  For future reference outside of a quantum mechanical context here is a separate and more concise iteration of these results.

\section{The rotation and notation}

The spherical polar rotor is a composition of rotations, expressed as half angle exponentials.  Following the normal physics conventions we first apply a \(z,x\) plane rotation by angle theta, then an \(x,y\) plane rotation by angle \(\phi\).  This produces the rotor

\begin{equation}\label{eqn:sphericalPolarUnit:foo1}
\begin{aligned}
R = e^{\Be_{31}\theta/2} e^{\Be_{12}\phi/2}
\end{aligned}
\end{equation}

Our triplet of Cartesian unit vectors is therefore rotated as

\begin{equation}\label{eqn:sphericalPolarUnit:foo2}
\begin{aligned}
\begin{pmatrix}
\rcap \\
\thetacap \\
\phicap \\
\end{pmatrix}
&=
\tilde{R}
\begin{pmatrix}
\Be_3 \\
\Be_1 \\
\Be_2 \\
\end{pmatrix}
R
\end{aligned}
\end{equation}

In the quantum mechanical context it was convenient to denote the \(x,y\) plane unit bivector with the imaginary symbol

\begin{equation}\label{eqn:sphericalPolarUnit:foo3}
\begin{aligned}
i = \Be_1 \Be_2
\end{aligned}
\end{equation}

reserving for the spatial pseudoscalar the capital

\begin{equation}\label{eqn:sphericalPolarUnit:foo4}
\begin{aligned}
I = \Be_1 \Be_2 \Be_3 = \rcap \thetacap \phicap = i \Be_3
\end{aligned}
\end{equation}

Note the characteristic differences between these two ``imaginaries''.  The planar quantity \(i = \Be_1 \Be_2\) commutes with \(\Be_3\), but anticommutes with either \(\Be_1\) or \(\Be_2\).  On the other hand the spatial pseudoscalar \(I\) commutes with any vector, bivector or trivector in the algebra.

\section{Application of the rotor.  The spherical polar unit vectors}

Having fixed notation, lets apply the rotation to each of the unit vectors in sequence, starting with the calculation for \(\phicap\).  This is

\begin{equation}\label{eqn:sphericalPolarUnit:33}
\begin{aligned}
\phicap
&= e^{-i \phi/2} e^{-\Be_{31}\theta/2} (\Be_2) e^{\Be_{31}\theta/2} e^{i\phi/2} \\
&= \Be_2 e^{i\phi}
\end{aligned}
\end{equation}

Here, since \(\Be_2\) commutes with the rotor bivector \(\Be_3 \Be_1\) the innermost exponentials cancel, leaving just the \(i\phi\) rotation.  For \(\rcap\) it is a bit messier, and we have

\begin{equation}\label{eqn:sphericalPolarUnit:53}
\begin{aligned}
\rcap
&= e^{-i \phi/2} e^{-\Be_{31}\theta/2} (\Be_3) e^{\Be_{31}\theta/2} e^{i\phi/2} \\
&= e^{-i \phi/2} \Be_3 e^{\Be_{31}\theta} e^{i\phi/2} \\
&= e^{-i \phi/2} (\Be_3 \cos\theta + \Be_1 \sin\theta) e^{i\phi/2} \\
&= \Be_3 \cos\theta + \Be_1 \sin\theta e^{i\phi} \\
&= \Be_3 \cos\theta + \Be_1 \Be_2 \sin\theta \Be_2 e^{i\phi} \\
&= \Be_3 \cos\theta + i \sin\theta \phicap \\
&= \Be_3 (\cos\theta + \Be_3 i \sin\theta \phicap) \\
&= \Be_3 e^{I\phicap\theta}
\end{aligned}
\end{equation}

Finally for \(\thetacap\), we have a similar messy expansion

\begin{equation}\label{eqn:sphericalPolarUnit:73}
\begin{aligned}
\thetacap
&= e^{-i \phi/2} e^{-\Be_{31}\theta/2} (\Be_1) e^{\Be_{31}\theta/2} e^{i\phi/2} \\
&= e^{-i \phi/2} \Be_1 e^{\Be_{31}\theta} e^{i\phi/2} \\
&= e^{-i \phi/2} (\Be_1 \cos\theta - \Be_3 \sin\theta) e^{i\phi/2} \\
&= \Be_1 \cos\theta e^{i\phi} - \Be_3 \sin\theta \\
&= i \cos\theta \Be_2 e^{i\phi} - \Be_3 \sin\theta \\
&= i \phicap \cos\theta - \Be_3 \sin\theta \\
&= i \phicap (\cos\theta + \phicap i \Be_3 \sin\theta) \\
&= i \phicap e^{I\phicap\theta}
\end{aligned}
\end{equation}

Summarizing the three of these relations we have for the rotated unit vectors

\begin{equation}\label{eqn:sphericalPolarUnit:foo5}
\begin{aligned}
\rcap &= \Be_3 e^{I \phicap \theta} \\
\thetacap &= i \phicap e^{I \phicap \theta} \\
\phicap &= \Be_2 e^{i\phi}
\end{aligned}
\end{equation}

and in particular for the radial position vector from the origin, rotating from the polar axis, we have

\begin{equation}\label{eqn:sphericalPolarUnit:foo6}
\begin{aligned}
\Bx &= r \rcap = r \Be_3 e^{I\phicap \theta}
\end{aligned}
\end{equation}

Compare this to the coordinate representation

\begin{equation}\label{eqn:sphericalPolarUnit:foo7}
\begin{aligned}
\Bx = r(\sin\theta \cos\phi, \sin\theta \sin\phi, \cos\theta)
\end{aligned}
\end{equation}

it is not initially obvious that these \(\theta\) and \(\phi\) rotations admit such a tidy factorization.  In retrospect, this does not seem so surprising, since we can form a quaternion product that acts via multiplication to map a vector to a rotated position.  In fact those quaternions, acting from the right on the initial vectors are

\begin{equation}\label{eqn:sphericalPolarUnit:foo8}
\begin{aligned}
\Be_3 &\rightarrow \rcap = \Be_3 \bigl( e^{I \phicap \theta} \bigr) \\
\Be_1 &\rightarrow \thetacap = \Be_1 \bigl( \Be_2 \phicap e^{I \phicap \theta} \bigr) \\
\Be_2 &\rightarrow \phicap = \Be_2 \bigl( e^{i\phi} \bigr)
\end{aligned}
\end{equation}

FIXME: it should be possible to reduce the quaternion that rotates \(\Be_1 \rightarrow \thetacap\) to a single exponential.  What is it?

\section{A consistency check}

We expect that the dot product between a north pole oriented vector \(\Bz = Z \Be_3\) and the spherically polar rotated vector \(\Bx = r \Be_3 e^{I\phicap \theta}\) is just

\begin{equation}\label{eqn:sphericalPolarUnit:foo13}
\begin{aligned}
\Bx \cdot \Bz = Z r \cos\theta
\end{aligned}
\end{equation}

Lets verify this

\begin{equation}\label{eqn:sphericalPolarUnit:93}
\begin{aligned}
\Bx \cdot \Bz
&=
\gpgradezero{ Z \Be_3 \Be_3 r e^{I\phicap \theta}} \\
&=
Z r \gpgradezero{ \cos\theta + I \phicap \sin\theta} \\
&=
Z r \cos\theta \\
&\qedmarker
\end{aligned}
\end{equation}

\section{Area and volume elements}

Let us use these results to compute the spherical polar volume element.  Pictorially this can be read off simply from a diagram.  If one is less trusting of pictorial means (or want a method more generally applicable), we can also do this particular calculation algebraically, expanding the determinant of partials

\begin{equation}\label{eqn:sphericalPolarUnit:foo9}
\begin{aligned}
\begin{vmatrix}
\frac{\partial \Bx}{\partial r} & \frac{\partial \Bx}{\partial \theta} & \frac{\partial \Bx}{\partial \phi} \\
\end{vmatrix} dr d\theta d\phi
&=
\begin{vmatrix}
\sin\theta \cos\phi & \cos\theta \cos\phi & -\sin\theta \sin\phi \\
\sin\theta \sin\phi & \cos\theta \sin\phi & \sin\theta \cos\phi \\
\cos\theta          & -\sin\theta         & 0                   \\
\end{vmatrix} r^2 dr d\theta d\phi
\end{aligned}
\end{equation}

One can chug through the trig reduction for this determinant with not too much trouble, but it is not particularly fun.

Now compare to the same calculation proceeding directly with the exponential form.  We do still need to compute the partials

\begin{equation}\label{eqn:sphericalPolarUnit:113}
\begin{aligned}
\frac{\partial \Bx}{\partial r} = \rcap
\end{aligned}
\end{equation}

\begin{equation}\label{eqn:sphericalPolarUnit:133}
\begin{aligned}
\frac{\partial \Bx}{\partial \theta}
&= r \Be_3 \frac{\partial }{\partial \theta} e^{I\phicap \theta} \\
&= r \rcap I \phicap \\
&= r \rcap (\rcap \thetacap \phicap) \phicap \\
&= r \thetacap
\end{aligned}
\end{equation}

\begin{equation}\label{eqn:sphericalPolarUnit:153}
\begin{aligned}
\frac{\partial \Bx}{\partial \phi}
&= r \Be_3 \frac{\partial }{\partial \phi} (\cos\theta + I\phicap \sin\theta) \\
&= -r \Be_3 I i \phicap \sin\theta \\
&= r \phicap \sin\theta
\end{aligned}
\end{equation}

So the area element, the oriented area of the parallelogram between the two vectors \(d\theta \partial \Bx/\partial \theta\), and \(d\phi \partial \Bx/\partial \phi\) on the spherical surface at radius \(r\) is

\begin{equation}\label{eqn:sphericalPolarUnit:foo10}
\begin{aligned}
d\BS = \left(d\theta \frac{\partial \Bx}{\partial \theta}\right) \wedge \left( d\phi \frac{\partial \Bx}{\partial \phi} \right)
= r^2 \thetacap \phicap \sin\theta d\theta d\phi
\end{aligned}
\end{equation}

and the volume element in trivector form is just the product
\begin{equation}\label{eqn:sphericalPolarUnit:foo11}
\begin{aligned}
d\BV = \left(dr\frac{\partial \Bx}{\partial r}\right) \wedge d\BS
= r^2 \sin\theta I dr d\theta d\phi
\end{aligned}
\end{equation}

\section{Line element}

The line element for the particle moving on a spherical surface can be calculated by calculating the derivative of the spherical polar unit vector \(\rcap\)

\begin{equation}\label{eqn:sphericalPolarUnit:yoo1a}
\begin{aligned}
\frac{d\rcap}{dt} = \PD{\phi}{\rcap} \frac{d\phi}{dt}
+\PD{\theta}{\rcap} \frac{d\theta}{dt}
\end{aligned}
\end{equation}

than taking the magnitude of this vector.  We can start either in coordinate form

\begin{equation}\label{eqn:sphericalPolarUnit:yoo1}
\begin{aligned}
\rcap
&= \Be_3 \cos\theta + \Be_1 \sin\theta \cos\phi + \Be_2 \sin\theta \sin\phi
\end{aligned}
\end{equation}

or, instead do it the fun way, first grouping this into a complex exponential form.  This factorization was done above, but starting over allows this to be done a bit more effectively for this particular problem.  As above, writing \(i = \Be_1 \Be_2\), the first factorization is

\begin{equation}\label{eqn:sphericalPolarUnit:yoo2}
\begin{aligned}
\rcap
&= \Be_3 \cos\theta + \Be_1 \sin\theta e^{i\phi}
\end{aligned}
\end{equation}

The unit vector \(\Brho = \Be_1 e^{i\phi}\) lies in the \(x,y\) plane perpendicular to \(\Be_3\), so we can form the unit bivector \(\Be_3\Brho\) and further factor the unit vector terms into a \(\cos + i \sin\) form

\begin{equation}\label{eqn:sphericalPolarUnit:173}
\begin{aligned}
\rcap
&= \Be_3 \cos\theta + \Be_1 \sin\theta e^{i\phi} \\
&= \Be_3 (\cos\theta + \Be_3 \Brho \sin\theta) \\
\end{aligned}
\end{equation}

This allows the spherical polar unit vector to be expressed in complex exponential form (really a vector-quaternion product)

\begin{equation}\label{eqn:sphericalPolarUnit:yoo3}
\begin{aligned}
\rcap = \Be_3 e^{\Be_3 \Brho \theta} = e^{-\Be_3 \Brho \theta} \Be_3
\end{aligned}
\end{equation}

Now, calculating the unit vector velocity, we get

\begin{equation}\label{eqn:sphericalPolarUnit:193}
\begin{aligned}
\frac{d\rcap}{dt}
&= \Be_3 \Be_3 \Brho e^{\Be_3 \Brho \theta} \thetadot + \Be_1 \Be_1 \Be_2 \sin\theta e^{i\phi} \phidot \\
&= \Brho e^{\Be_3 \Brho \theta} \left(\thetadot + e^{-\Be_3 \Brho \theta} \Brho \sin\theta e^{-i\phi} \Be_2 \phidot\right) \\
&= \left( \thetadot + \Be_2 \sin\theta e^{i\phi} \phidot \Brho e^{\Be_3 \Brho \theta} \right) e^{-\Be_3 \Brho \theta} \Brho
\end{aligned}
\end{equation}

The last two lines above factor the \(\Brho\) vector and the \(e^{\Be_3 \Brho \theta}\) quaternion to the left and to the right in preparation for squaring this to calculate the magnitude.

\begin{equation}\label{eqn:sphericalPolarUnit:213}
\begin{aligned}
\left( \frac{d\rcap}{dt} \right)^2
&=
\gpgradezero{ \left( \frac{d\rcap}{dt} \right)^2 } \\
&=
\gpgradezero{
\left( \thetadot + \Be_2 \sin\theta e^{i\phi} \phidot \Brho e^{\Be_3 \Brho \theta} \right)
\left(\thetadot + e^{-\Be_3 \Brho \theta} \Brho \sin\theta e^{-i\phi} \Be_2 \phidot\right) } \\
&=
\thetadot^2 + \sin^2\theta \phidot^2
+ \sin\theta \phidot \thetadot
\gpgradezero{
\Be_2 e^{i\phi} \Brho e^{\Be_3 \rho \theta}
+e^{-\Be_3 \rho \theta} \Brho e^{-i\phi} \Be_2
} \\
\end{aligned}
\end{equation}

This last term (\(\in \Span \{\rho\Be_1, \rho\Be_2, \Be_1\Be_3, \Be_2\Be_3\}\)) has only grade two components, so the scalar part is zero.  We are left with the line element

\begin{equation}\label{eqn:sphericalPolarUnit:yoo4}
\begin{aligned}
\left(\frac{d (r\rcap)}{dt}\right)^2 = r^2 \left( \thetadot^2 + \sin^2\theta \phidot^2 \right)
\end{aligned}
\end{equation}

In retrospect, at least once one sees the answer, it seems obvious.  Keeping \(\theta\) constant the length increment moving in the plane is \(ds = \sin\theta d\phi\), and keeping \(\phi\) constant, we have \(ds = d\theta\).  Since these are perpendicular directions we can add the lengths using the Pythagorean theorem.

\subsection{Line element using an angle and unit bivector parameterization}

Parameterizing using scalar angles is not the only approach that we can take to calculate the line element on the unit sphere.  Proceding directly with a alternate polar representation, utilizing a unit bivector \(j\), and scalar angle \(\theta\) is

\begin{equation}\label{eqn:sphericalPolarUnit:zoo1}
\begin{aligned}
\Bx = r \Be_3 e^{j\theta}
\end{aligned}
\end{equation}

For this product to be a vector \(j\) must contain \(\Be_3\) as a factor (\(j = \Be_3 \wedge m\) for some vector m.)  Setting \(r = 1\) for now, the deriviate of \(\Bx\) is

\begin{equation}\label{eqn:sphericalPolarUnit:233}
\begin{aligned}
\dot{\Bx}
&= \Be_3 \frac{d}{dt} \left( \cos\theta + j \sin\theta \right) \\
&= \Be_3 \thetadot \left( -\sin\theta + j \cos\theta \right) + \Be_3 \frac{d j}{dt} \sin\theta  \\
&= \Be_3 \thetadot j \left( j \sin\theta + \cos\theta \right) + \Be_3 \frac{d j}{dt} \sin\theta  \\
\end{aligned}
\end{equation}

This is
\begin{equation}\label{eqn:sphericalPolarUnit:zoo2}
\begin{aligned}
\dot{\Bx} = \Be_3 \left( \frac{d\theta}{dt} j e^{j\theta} + \frac{d j}{dt} \sin\theta \right)
\end{aligned}
\end{equation}

Alternately, we can take derivatives of \(\Bx = r e^{-j\theta} \Be_3\), for

%COMMENT OUT:
%\begin{align*}
%\dot{\Bx}
%&= \frac{d}{dt} \left( \cos\theta - j \sin\theta \right) \Be_3 \\
%&= \thetadot \left( -\sin\theta - j \cos\theta \right) \Be_3 - \frac{d j}{dt} \sin\theta \Be_3 \\
%&= -\thetadot j \left( -j \sin\theta + \cos\theta \right) \Be_3 - \frac{d j}{dt} \sin\theta \Be_3 \\
%\end{align*}

Or
\begin{equation}\label{eqn:sphericalPolarUnit:zoo3}
\begin{aligned}
\dot{\Bx} = -\left( \frac{d\theta}{dt} j e^{-j\theta} + \frac{d j}{dt} \sin\theta \right) \Be_3
\end{aligned}
\end{equation}

Together with \eqnref{eqn:sphericalPolarUnit:zoo2}, the line element for position change on the unit sphere is then

\begin{equation}\label{eqn:sphericalPolarUnit:253}
\begin{aligned}
\dot{\Bx}^2
&= \gpgradezero{
-\left( \frac{d\theta}{dt} j e^{-j\theta} + \frac{d j}{dt} \sin\theta \right) \Be_3 \Be_3 \left( \frac{d\theta}{dt} j e^{j\theta} + \frac{d j}{dt} \sin\theta \right) } \\
&= \gpgradezero{
-\left( \frac{d\theta}{dt} j e^{-j\theta} + \frac{d j}{dt} \sin\theta \right) \left( \frac{d\theta}{dt} j e^{j\theta} + \frac{d j}{dt} \sin\theta \right) } \\
&=
\left(\frac{d\theta}{dt}\right)^2 - \left(\frac{d j}{dt}\right)^2 \sin^2\theta
- \frac{d\theta}{dt} \sin\theta
\gpgradezero{ \frac{dj}{dt} j e^{j\theta} + j e^{-j\theta} \frac{dj}{dt} } \\
\end{aligned}
\end{equation}

Starting with cyclic reordering of the last term, we get zero

\begin{equation}\label{eqn:sphericalPolarUnit:273}
\begin{aligned}
\gpgradezero{ \frac{dj}{dt} j e^{j\theta} + j e^{-j\theta} \frac{dj}{dt} }
&=
\gpgradezero{ \frac{dj}{dt} j \left( e^{j\theta} + e^{-j\theta} \right) }  \\
&=
\gpgradezero{ \frac{dj}{dt} j 2 j \sin\theta }  \\
&=
- 2 \sin\theta \frac{d}{dt}
\mathLabelBox
[
   labelstyle={below of=m\themathLableNode, below of=m\themathLableNode}
]
{\gpgradezero{ j }}{\(=0\)}  \\
\end{aligned}
\end{equation}

The line element (for constant \(r\)) is therefore

\begin{equation}\label{eqn:sphericalPolarUnit:zoo4}
\begin{aligned}
\dot{\Bx}^2
=
r^2 \left( \thetadot^2 - \left(\frac{dj}{dt}\right)^2 \sin^2 \theta \right)
\end{aligned}
\end{equation}

This is essentially the same result as we got starting with an explicit \(r, \theta, \phi\).  Repeating for comparision that was

\begin{equation}\label{eqn:sphericalPolarUnit:zoo5}
\begin{aligned}
\dot{\Bx}^2 = r^2 \left( \thetadot^2 + \sin^2\theta \phidot^2 \right)
\end{aligned}
\end{equation}

The bivector that we have used this time encodes the orientation of the plane of rotation from the polar axis down to the position on the sphere corresponds to the angle \(\phi\) in the
scalar parameterization.  The negation in sign is expected due to the negative bivector square.

Also comparing to previous results it is notable that we can explicitly express this bivector in terms of the scalar angle if desired as

\begin{equation}\label{eqn:sphericalPolarUnit:zoo6}
\begin{aligned}
\Brho &= \Be_1 e^{\Be_1 \Be_2 \phi} = \Be_1 \cos\phi + \Be_2 \sin\phi \\
j &= \Be_3 \wedge \Brho = \Be_3 \Brho
\end{aligned}
\end{equation}

The inverse mapping, expressing the scalar angle using the bivector representation is also possible, but not unique.  The principle angle for that inverse mapping is

\begin{equation}\label{eqn:sphericalPolarUnit:zoo7}
\begin{aligned}
\phi &= -\Be_1 \Be_2 \ln(\Be_1 \Be_3 j)
\end{aligned}
\end{equation}

\subsection{Allowing the magnitude to vary}

Writing a vector in polar form

\begin{equation}\label{eqn:sphericalPolarUnit:xoo1}
\begin{aligned}
\Bx = r \rcap
\end{aligned}
\end{equation}

and also allowing \(r\) to vary, we have

\begin{equation}\label{eqn:sphericalPolarUnit:293}
\begin{aligned}
\left(\frac{d\Bx}{dt}\right)^2
&= \left( \frac{dr}{dt} \rcap + r \frac{d\rcap}{dt} \right)^2 \\
&=
\left( \frac{dr}{dt} \right)^2 + r^2 \left( \frac{d\rcap}{dt} \right)^2
+ 2 r \frac{dr}{dt} \rcap \cdot \frac{d\rcap}{dt}
\end{aligned}
\end{equation}

The squared unit vector derivative was previously calculated to be

\begin{equation}\label{eqn:sphericalPolarUnit:xoo2}
\begin{aligned}
\left(\frac{d \rcap}{dt}\right)^2 = \thetadot^2 + \sin^2\theta \phidot^2
\end{aligned}
\end{equation}

Picturing the geometry is enough to know that \(\dot{\rcap} \cdot \rcap = 0\) since \(\dot{\rcap}\) is always tangential to the sphere.  It should also be possible to algebraically show this, but without going through the effort we at least now know the general line element

\begin{equation}\label{eqn:sphericalPolarUnit:xoo3}
\begin{aligned}
\dot{\Bx}^2 = {\dot{r}}^2 + r^2 \left( \thetadot^2 + \sin^2\theta \phidot^2 \right)
\end{aligned}
\end{equation}

%Observe that we can also write this in matrix form tidily.  Let
%
%\begin{align}\label{eqn:sphericalPolarUnit:xoo4}
%\BTheta &\equiv
%\begin{bmatrix}
%r \\
%\theta \\
%\phi \\
%\end{bmatrix} \\
%Q &\equiv
%\begin{bmatrix}
%1 & 0 & 0 \\
%0 & r^2 & 0 \\
%0 & 0 & r^2 \sin^2\theta \\
%\end{bmatrix}
%\end{align}
%
%Then we have
%
%\begin{align}\label{eqn:sphericalPolarUnit:xoo5}
%\dot{\Bx}^2 = \dot{\BTheta}^\T Q \dot{\BTheta}
%\end{align}
%
%For study of the Hamiltonian equations of the double (or multiple) pendulum in a plane we were able to express the Kinetic term with such a matrix, but work only in \(Q\), so it appears we can probably generalize that treatment to the non-planar case without too much trouble.

%\EndArticle
%%\EndNoBibArticle

   \chapter{Infinitesimal rotations}
      %
% Copyright � 2012 Peeter Joot.  All Rights Reserved.
% Licenced as described in the file LICENSE under the root directory of this GIT repository.
%

%
%
%\input{../peeter_prologue_print.tex}
%\input{../peeter_prologue_widescreen.tex}

%\chapter{Infinitesimal rotations}
\index{rotation!infinitesimal}
\label{chap:infinitesimalRotation}

%\blogpage{http://sites.google.com/site/peeterjoot2/math2012/infinitesimalRotation.pdf}
%\date{Jan 27, 2012}
%\revisionInfo{infinitesimalRotation.tex}

\beginArtWithToc
%\beginArtNoToc

\section{Motivation}

In a classical mechanics lecture (which I audited) Prof. Poppitz made the claim that an infinitesimal rotation in direction \(\ncap\) of magnitude \(\delta \phi\) has the form
%
\begin{equation}\label{eqn:continuumL6:10}
\Bx \rightarrow \Bx + \delta \Bphi \cross \Bx,
\end{equation}
%
where
%
\begin{equation}\label{eqn:continuumL6:30}
\delta \Bphi = \ncap \delta \phi.
\end{equation}
%
I believe he expressed things in terms of the differential displacement
%
\begin{equation}\label{eqn:continuumL6:50}
\delta \Bx = \delta \Bphi \cross \Bx
\end{equation}
%
This was verified for the special case \(\ncap = \zcap\) and \(\Bx = x \xcap\).  Let us derive this in the general case too.

\section{With geometric algebra}

Let us temporarily dispense with the normal notation and introduce two perpendicular unit vectors \(\ucap\), and \(\vcap\) in the plane of the rotation.  Relate these to the unit normal with
%
\begin{equation}\label{eqn:continuumL6:70}
\ncap = \ucap \cross \vcap.
\end{equation}
%
A rotation through an angle \(\phi\) (infinitesimal or otherwise) is then
%
\begin{equation}\label{eqn:continuumL6:90}
\Bx \rightarrow
e^{-\ucap \vcap \phi/2} \Bx e^{\ucap \vcap \phi/2}.
\end{equation}
%
Suppose that we decompose \(\Bx\) into components in the plane and in the direction of the normal \(\ncap\).  We have
%
\begin{equation}\label{eqn:continuumL6:110}
\Bx = x_u \ucap + x_v \vcap + x_n \ncap.
\end{equation}
%
The exponentials commute with the \(\ncap\) vector, and anticommute otherwise, leaving us with
%
\begin{equation}\label{eqn:infinitesimalRotation:150}
\begin{aligned}
\Bx
&\rightarrow
x_n \ncap +
(x_u \ucap + x_v \vcap) e^{\ucap \vcap \phi} \\
&=
x_n \ncap +
(x_u \ucap + x_v \vcap) (\cos\phi + \ucap \vcap \sin\phi) \\
&=
x_n \ncap +
\ucap (x_u \cos\phi - x_v \sin\phi)
+
\vcap (x_v \cos\phi + x_u \sin\phi).
\end{aligned}
\end{equation}
%
In the last line we use \(\ucap^2 = 1\) and \(\ucap \vcap = - \vcap \ucap\).  Making the angle infinitesimal \(\phi \rightarrow \delta \phi\) we have
%
\begin{equation}\label{eqn:infinitesimalRotation:170}
\begin{aligned}
\Bx
&\rightarrow
x_n \ncap +
\ucap (x_u - x_v \delta\phi)
+
\vcap (x_v + x_u \delta\phi)  \\
&=
\Bx + \delta\phi( x_u \vcap - x_v \ucap)
\end{aligned}
\end{equation}
%
We have only to confirm that this matches the assumed cross product representation
%
\begin{equation}\label{eqn:infinitesimalRotation:190}
\begin{aligned}
\ncap \cross \Bx
&=
\begin{vmatrix}
\ucap & \vcap & \ncap \\
0 & 0 & 1 \\
x_u & x_v & x_n
\end{vmatrix} \\
&=
-\ucap x_v + \vcap x_u
\end{aligned}
\end{equation}
%
Taking the two last computations we find
%
\begin{equation}\label{eqn:continuumL6:130}
\delta \Bx = \delta \phi \ncap \cross \Bx = \delta \Bphi \cross \Bx,
\end{equation}
%
as desired.

\section{Without geometric algebra}

We have also done the setup above to verify this result without GA.  Here we wish to apply the rotation to the coordinate vector of \(\Bx\) in the \(\{\ucap, \vcap, \ncap\}\) basis which gives us
%
\begin{equation}\label{eqn:infinitesimalRotation:210}
\begin{aligned}
\begin{bmatrix}
x_u \\
x_v \\
x_n
\end{bmatrix}
&\rightarrow
\begin{bmatrix}
\cos\delta\phi & -\sin\delta\phi & 0 \\
\sin\delta\phi & \cos\delta\phi & 0 \\
0 & 0 & 1
\end{bmatrix}
\begin{bmatrix}
x_u \\
x_v \\
x_n
\end{bmatrix} \\
&\approx
\begin{bmatrix}
1 & -\delta\phi & 0 \\
\delta\phi & 1 & 0 \\
0 & 0 & 1
\end{bmatrix}
\begin{bmatrix}
x_u \\
x_v \\
x_n
\end{bmatrix} \\
&=
\begin{bmatrix}
x_u \\
x_v \\
x_n
\end{bmatrix}
+
\begin{bmatrix}
0 & -\delta\phi & 0 \\
\delta\phi & 0 & 0 \\
0 & 0 & 0
\end{bmatrix}
\begin{bmatrix}
x_u \\
x_v \\
x_n
\end{bmatrix} \\
&=
\begin{bmatrix}
x_u \\
x_v \\
x_n
\end{bmatrix}
+
\delta\phi
\begin{bmatrix}
-x_v \\
x_u \\
0
\end{bmatrix}
\end{aligned}
\end{equation}
%
But as we have shown, this last coordinate vector is just \(\ncap \cross \Bx\), and we get our desired result using plain old fashioned matrix algebra as well.

Really the only difference between this and what was done in class is that there is no assumption here that \(\Bx = x \xcap\).

%%\EndArticle
%\EndNoBibArticle


\part{Calculus}
   \chapter{Developing some intuition for Multivariable and Multivector Taylor Series}\label{chap:PJmultiTaylors}
      %
% Copyright � 2012 Peeter Joot.  All Rights Reserved.
% Licenced as described in the file LICENSE under the root directory of this GIT repository.
%

%
%
%\chapter{Developing some intuition for Multivariable and Multivector Taylor Series}\label{chap:PJmultiTaylors}
\index{Taylor series}
%\date{April 28, 2009.  multivectorTaylors.tex}

The book \citep{doran2003gap} uses Geometric Calculus heavily in its Lagrangian treatment.  In particular it is used in some incomprehensible seeming ways in the stress energy tensor treatment.

In the treatment of transformation of the dependent variables (not the field variables themselves) of field Lagrangians, there is one bit that appears to be the first order linear term from a multivariable Taylor series expansion.  Play with multivariable Taylor series here a bit to develop some intuition with it.

\section{Single variable case, and generalization of it}

For the single variable case, Taylor series takes the form

\begin{equation}\label{eqn:multivectorTaylors:20}
\begin{aligned}
f(x) = \sum \frac{x^k}{k!} \left. \frac{d^k f(x)}{dx^k} \right\vert_{x=0}
\end{aligned}
\end{equation}

or
\begin{equation}\label{eqn:multivectorTaylors:40}
\begin{aligned}
f(x_0 + \epsilon) = \sum \frac{\epsilon^k}{k!} \left. \frac{d^k f(x)}{dx^k} \right\vert_{x=x_0}
\end{aligned}
\end{equation}

As pointed out in \citep{byron1992mca}, this can (as they demonstrated for polynomials) be put into exponential
operator form

\begin{equation}\label{eqn:multivectorTaylors:60}
\begin{aligned}
f(x_0 + \epsilon) = \left. e^{\epsilon d/dx} f(x) \right\vert_{x=x_0}
\end{aligned}
\end{equation}

Without proof, the multivector generalization of this is

\begin{equation}\label{eqn:multivectorTaylors:taylorsExponential}
\begin{aligned}
f(x_0 + \epsilon)
&= \left. e^{\epsilon \cdot \grad} f(x) \right\vert_{x=x_0}
\end{aligned}
\end{equation}

Or in full,

\begin{equation}\label{eqn:multivectorTaylors:taylorMulti}
\begin{aligned}
f(x_0 + \epsilon)
&= \sum \inv{k!} \left. {(\epsilon \cdot \grad)^k} f(x) \right\vert_{x=x_0}
\end{aligned}
\end{equation}

Let us work with this, and develop some comfort with what it means, then revisit the proof.

\section{Directional Derivatives}

First a definition of directional derivative is required.

In
\href{http://tutorial.math.lamar.edu/Classes/CalcIII/DirectionalDeriv.aspx}{standard two variable vector calculus} the directional derivative is defined in one of the following ways
\begin{equation}\label{eqn:multivectorTaylors:80}
\begin{aligned}
\spacegrad_\Bu f(x,y) &= \lim_{h \rightarrow 0} \frac{f(x + a h, y + b h) - f(x,y)}{h} \\
\Bu &= (a,b)
\end{aligned}
\end{equation}

Or \href{http://en.wikipedia.org/wiki/Directional_derivative}{in a more general vector form} as

\begin{equation}\label{eqn:multivectorTaylors:100}
\begin{aligned}
\spacegrad_\Bu f(\Bx) &= \lim_{h \rightarrow 0} \frac{f(\Bx + h\Bu) - f(\Bx)}{h}
\end{aligned}
\end{equation}

Or \href{http://mathworld.wolfram.com/DirectionalDerivative.html}{in terms of the gradient} as
\begin{equation}\label{eqn:multivectorTaylors:120}
\begin{aligned}
\spacegrad_\Bu f(\Bx) &= \frac{\Bu}{\Abs{\Bu}} \cdot \spacegrad f
\end{aligned}
\end{equation}

Each of these was for a vector parametrized scalar function, although the wikipedia article does mention a vector valued form that is identical to that use by \citep{doran2003gap}.  Specifically, that is

\begin{equation}\label{eqn:multivectorTaylors:140}
\begin{aligned}
(\epsilon \cdot \grad) f(x)
&= \lim_{h \rightarrow 0} \frac{f(x + h \epsilon) - f(x)}{h} \\
&= \left. \PD{h}{f(x + h \epsilon)} \right\vert_{h=0}
\end{aligned}
\end{equation}

Observe that this definition as a limit avoids the requirement to define the gradient upfront.  That definition is not necessarily obvious especially for multivector valued functions.

%\section{Work some examples}

\makeexample{First order linear vector polynomial}{example:multivectorTaylors:141}{

Let

\begin{equation}\label{eqn:multivectorTaylors:160}
\begin{aligned}
f(x) = a + x
\end{aligned}
\end{equation}

For this simplest of vector valued vector parametrized functions we have

\begin{equation}\label{eqn:multivectorTaylors:180}
\begin{aligned}
\PD{h}{f(x + h \epsilon)}
&= \PD{h}{} (a + x + h \epsilon) \\
&= \epsilon \\
&= (\epsilon \cdot \grad) f
\end{aligned}
\end{equation}

with no requirement to evaluate at \(h=0\) to complete the directional derivative computation.

The Taylor series expansion about \(0\) is thus

\begin{equation}\label{eqn:multivectorTaylors:200}
\begin{aligned}
f(\epsilon)
&= \left. (\epsilon \cdot \grad)^0 f \right\vert_{x=0} + \left. (\epsilon \cdot \grad)^1 f  \right\vert_{x=0} \\
&= a + \epsilon \\
\end{aligned}
\end{equation}

Nothing else could be expected.
}

\makeexample{Second order vector parametrized multivector polynomial}{example:multivectorTaylors:201}{

Now, step up the complexity slightly, and introduce a multivector valued second degree polynomial, say,

\begin{equation}\label{eqn:multivectorTaylors:secondOrder}
\begin{aligned}
f(x) = \alpha + a + x y + w x + c x^2 + d x e + x g x
\end{aligned}
\end{equation}

Here \(\alpha\) is a scalar, and all the other variables are vectors, so we have grades \(\le 3\).

For the first order partial we have
\begin{equation}\label{eqn:multivectorTaylors:220}
\begin{aligned}
&\PD{h}{f(x + h \epsilon)} \\
&= \PD{h}{} ( \alpha + a + (x + h\epsilon) y + w (x + h\epsilon) + c (x + h\epsilon)^2 + d (x + h\epsilon) e + (x + h\epsilon) g (x + h\epsilon) ) \\
&=
\epsilon y
+ w \epsilon
+ c \epsilon (x + h\epsilon)
+ c (x + h\epsilon) \epsilon
+ c \epsilon
+ d \epsilon e
+ \epsilon g (x + h\epsilon)
+ (x + h\epsilon) g \epsilon \\
\end{aligned}
\end{equation}

Evaluation at \(h=0\) we have

\begin{equation}\label{eqn:multivectorTaylors:240}
\begin{aligned}
(\epsilon \cdot \grad) f
&=
\epsilon y
+ w \epsilon
+ c \epsilon x
+ c x \epsilon
+ c \epsilon
+ d \epsilon e
+ \epsilon g x
+ x g \epsilon \\
\end{aligned}
\end{equation}

By inspection we have

\begin{equation}\label{eqn:multivectorTaylors:260}
\begin{aligned}
(\epsilon \cdot \grad)^2 f
&=
+ 2 c \epsilon^2
+ 2 \epsilon g \epsilon \\
\end{aligned}
\end{equation}

Combining things forming the Taylor series expansion about the origin we should recover our function

\begin{equation}\label{eqn:multivectorTaylors:280}
\begin{aligned}
f(\epsilon)
&= \inv{0!} \left. (\epsilon \cdot \grad)^0 f \right\vert_{x=0}
+ \inv{1!} \left. (\epsilon \cdot \grad)^1 f \right\vert_{x=0}
+ \inv{2} \left. (\epsilon \cdot \grad)^2 f \right\vert_{x=0} \\
&= \inv{1} (\alpha + a) + \inv{1} (\epsilon y + w \epsilon + c \epsilon + d \epsilon e ) + \inv{2}(2 c \epsilon^2 + 2 \epsilon g \epsilon ) \\
&= \alpha + a + \epsilon y + w \epsilon + c \epsilon + d \epsilon e + c \epsilon^2 + \epsilon g \epsilon \\
% cf:
%&f(x) = \alpha + a + x y + w x + c x^2 + d x e + x g x
\end{aligned}
\end{equation}

This should match \eqnref{eqn:multivectorTaylors:secondOrder}, with an \(x = \epsilon\) substitution, and does.  With the vector factors in these functions commutativity assumptions could not be made.  These calculations help provide a small verification that this form of Taylor series does in fact work out fine with such non-commutative variables.

Observe as well that there was really no requirement in this example that \(x\) or any of the other factors to be vectors.  If they were all bivectors or trivectors or some mix the calculations would have had the same results.
}

\section{Proof of the multivector Taylor expansion}

A peek back into \citep{hestenes1999nfc} shows that \eqnref{eqn:multivectorTaylors:taylorMulti} was in fact proved, but it was done in a very sneaky and clever way.  Rather than try to prove treat the multivector parameters explicitly, the following scalar parametrized hybrid function was created

\begin{equation}\label{eqn:multivectorTaylors:300}
\begin{aligned}
G(\tau) &= F(\Bx_0 + \tau\Ba)
\end{aligned}
\end{equation}

The scalar parametrized function \(G(\tau)\) can be Taylor expanded about the origin, and then evaluated at \(1\) resulting in \eqnref{eqn:multivectorTaylors:taylorMulti} in terms of powers of \((\Ba \cdot \grad)\).  I will not reproduce or try to enhance that proof for myself here since it is actually quite clear in the text.  Obviously the trick is non-intuitive enough that when thinking about how to prove this myself it did not occur to me.

\section{Explicit expansion for a scalar function}

Now, despite the \(a \cdot \grad\) notation being unfamiliar seeming, the end result is not.  Explicit expansion of this for a vector to scalar mapping will show this.  In fact this will also account for the \href{http://en.wikipedia.org/wiki/Hessian_matrix}{Hessian matrix}, as in

\begin{equation}\label{eqn:multivectorTaylors:320}
\begin{aligned}
y = f(\mathbf{x}+\Delta\mathbf{x}) \approx f(\mathbf{x}) + J(\mathbf{x}) \Delta \mathbf{x} %+\frac{1}{2} \Delta {\mathbf{x}}^\txtT H(\mathbf{x}) \Delta \mathbf{x}
\end{aligned}
\end{equation}

providing not only the background on where this comes from, but also the so often omitted third order and higher generalizations (most often referred to as \(\cdots\)).  Poking around a bit I see that the \href{http://en.wikipedia.org/wiki/Taylor_expansion}{wikipedia Taylor Series} does explicitly define the higher order case, but if I had seen that before the connection to the Hessian was not obvious.

\makeexample{Two variable case}{example:multivectorTaylors:341}{

Rather than start with the general case, the expansion of the first few powers of \((\Ba \cdot \spacegrad) f\) for the two variable case is enough to show the pattern.  How to further generalize this scalar function case will be clear from inspection.

Starting with the first order term, writing \(\Ba = (a,b)\) we have

\begin{equation}\label{eqn:multivectorTaylors:340}
\begin{aligned}
(\Ba \cdot \spacegrad) f(x,y)
&= \left. \PD{\tau}{} f(x + a\tau, y + b\tau) \right\vert_{\tau=0} \\
&=
\left. \left( \PD{x + a\tau}{} f(x + a\tau, y + b\tau) \PD{\tau}{(x + a\tau)} \right) \right\vert_{\tau=0} \\
&\quad+\left. \left( \PD{y + b\tau}{} f(x + a\tau, y + b\tau) \PD{\tau}{(y + b\tau)} \right) \right\vert_{\tau=0} \\
&=
a \PD{x}{f} +b \PD{y}{f} \\
&=
\Ba \cdot (\spacegrad f)
\end{aligned}
\end{equation}

For the second derivative operation we have
\begin{equation}\label{eqn:multivectorTaylors:360}
\begin{aligned}
(\Ba \cdot \spacegrad)^2 f(x,y)
&=
(\Ba \cdot \spacegrad)
\left( (\Ba \cdot \spacegrad) f(x,y) \right) \\
&=
(\Ba \cdot \spacegrad) \left( a \PD{x}{f} +b \PD{y}{f} \right) \\
&= \left. \PD{\tau}{} \left( a \PD{x}{f}(x + a\tau, y + b\tau) + b \PD{y}{f}(x + a\tau, y + b\tau) \right) \right\vert_{\tau=0} \\
\end{aligned}
\end{equation}

Especially if one makes a temporary substitution of the partials for some other named variables, it is clear this follows as
before, and one gets

\begin{equation}\label{eqn:multivectorTaylors:380}
\begin{aligned}
(\Ba \cdot \spacegrad)^2 f(x,y)
&=
a^2 \PDSq{x}{f} + b a \PDD{y}{x}{f}
+a b \PDD{x}{y}{f} + b^2 \PDSq{y}{f} \\
\end{aligned}
\end{equation}

Similarly the third order derivative operator gives us

\begin{equation}\label{eqn:multivectorTaylors:400}
\begin{aligned}
(\Ba \cdot \spacegrad)^3 f(x,y)
&=
a a a \PD{x}{}\PD{x}{}\PD{x}{} f + a b a \PD{x}{}\PD{y}{}\PD{x}{}{f}  \\
\quad&+a a b \PD{x}{}\PD{y}{}\PD{x}{} f + a b b \PD{x}{}\PD{y}{}\PD{y}{}{f} \\
&\quad+b a a \PD{y}{}\PD{x}{}\PD{x}{} f + b b a \PD{y}{}\PD{y}{}\PD{x}{}{f} \\
&\quad+b a b \PD{y}{}\PD{y}{}\PD{x}{} f + b b b \PD{y}{}\PD{y}{}\PD{y}{}{f} \\
&=
a^3 \frac{\partial^3 f}{\partial x^3}
+ 3 a^2 b \PDSq{x}{}\PD{y}{f}
+ 3 a b^2 \PD{x}{}\PDSq{y}{f}
+b^3 \frac{\partial^3 f}{\partial y^3}
\end{aligned}
\end{equation}

We no longer have the notational nicety of being able to use the gradient notation as was done for the first derivative term.  For the
first and second order derivative operations, one has the
option of using the gradient and Hessian matrix notations

\begin{equation}\label{eqn:multivectorTaylors:420}
\begin{aligned}
(\Ba \cdot \spacegrad) f(x,y) &=
\transpose{\Ba}
\begin{bmatrix}
f_{x} \\
f_{y}
\end{bmatrix}
\\
(\Ba \cdot \spacegrad)^2 f(x,y)
&=
\transpose{\Ba}
\begin{bmatrix}
f_{xx} & f_{xy} \\
f_{yx} & f_{yy}
\end{bmatrix}
\Ba
\end{aligned}
\end{equation}

But this will not be helpful past the second derivative.

Additionally, if we continue to restrict oneself to the two variable case,
it is clear that we have

\begin{equation}\label{eqn:multivectorTaylors:440}
\begin{aligned}
(\Ba \cdot \spacegrad)^n f(x,y)
&=
\sum_{k=0}^{n} \binom{n}{k} a^{n-k} b^{k}
\left( \PD{x}{} \right)^{n-k}
\left( \PD{y}{} \right)^{k} f(x,y)
\end{aligned}
\end{equation}

But it is also clear that if we switch to more than two variables, a binomial
series expansion of derivative powers in this fashion will no longer work.  For
example for three (or more) variables, writing for example \(\Ba = (a_1, a_2, a_3)\),
we have

\begin{equation}\label{eqn:multivectorTaylors:460}
\begin{aligned}
(\Ba \cdot \spacegrad) f(\Bx)
&=
\sum_{i}
\left( a_i \PD{x_i}{} \right)
f(\Bx) \\
(\Ba \cdot \spacegrad)^2 f(\Bx)
&=
\sum_{ij}
\left( a_i \PD{x_i}{} \right)
\left( a_j \PD{x_j}{} \right)
f(\Bx) \\
(\Ba \cdot \spacegrad)^3 f(\Bx)
&=
\sum_{ijk}
\left( a_i \PD{x_i}{} \right)
\left( a_j \PD{x_j}{} \right)
\left( a_k \PD{x_k}{} \right)
f(\Bx)
%\\
%&\vdots
\end{aligned}
\end{equation}

If the partials are all collected into a single indexed object, one really has a tensor.  For the first and second orders we
can represent this tensor in matrix form (as the gradient and Hessian respectively)
}

\section{Gradient with non-Euclidean basis}

The directional derivative has been calculated above for a scalar function.  There is nothing intrinsic to that argument
that requires an orthonormal basis.

Suppose we have a basis \(\{\gamma_\mu\}\), and a reciprocal frame \(\{\gamma^\mu\}\).  Let

\begin{equation}\label{eqn:multivectorTaylors:480}
\begin{aligned}
x &= x^\mu \gamma_\mu = x_\mu \gamma^\mu \\
a &= a^\mu \gamma_\mu = a_\mu \gamma^\mu
\end{aligned}
\end{equation}

The first order directional derivative is then

\begin{equation}\label{eqn:multivectorTaylors:500}
\begin{aligned}
(a \cdot \grad) f(x)
&=
\left. \PD{\tau}{f}(x + \tau a) \right\vert_{\tau=0} \\
\end{aligned}
\end{equation}

This is
\begin{equation}\label{eqn:multivectorTaylors:gradDotF}
\begin{aligned}
(a \cdot \grad) f(x) &= \sum_\mu a^\mu \PD{x^\mu}{f}(x)
\end{aligned}
\end{equation}

Now, we are used to \(\grad\) as a standalone object, and want that operator defined such that we can also write \eqnref{eqn:multivectorTaylors:gradDotF}
as
\begin{equation}\label{eqn:multivectorTaylors:520}
\begin{aligned}
a \cdot (\grad f(x))
&=
\left(a^\mu \gamma_\mu \right) \cdot (\grad f(x))
\end{aligned}
\end{equation}

Comparing these we see that our partials in \eqnref{eqn:multivectorTaylors:gradDotF} do the job provided that we form the vector operator

\begin{equation}\label{eqn:multivectorTaylors:gradForNonOrtho}
\begin{aligned}
\grad &= \sum_\mu \gamma^\mu \PD{x^\mu}{}
\end{aligned}
\end{equation}

The text \citep{doran2003gap} defines \(\grad\) in this fashion, but has no logical motivation of this idea.  One sees quickly enough that this definition works, and is the required form, but building up to the construction in a way that builds on previously established ideas is still desirable.  We see here that this reciprocal frame definition of the gradient follows inevitably from the definition of the directional derivative.  Additionally this is a definition with how the directional derivative is defined in a standard Euclidean space with an orthonormal basis.

\section{Work out Gradient for a few specific multivector spaces}

The directional derivative result expressed in \eqnref{eqn:multivectorTaylors:gradDotF} holds for arbitrarily parametrized multivector spaces, and the image space can also be a generalized one.  However, the corresponding result \eqnref{eqn:multivectorTaylors:gradForNonOrtho} for the gradient itself is good only when the parameters are vectors.  These vector parameters may be non-orthonormal, and the function this is applied to does not have to be a scalar function.

If we switch to functions parametrized by multivector spaces the vector dot gradient notation also becomes misleading.  The natural generalization of the Taylor expansion for such a function, instead of \eqnref{eqn:multivectorTaylors:taylorsExponential}, or \eqnref{eqn:multivectorTaylors:taylorMulti} should instead be

\begin{equation}\label{eqn:multivectorTaylors:taylorsExponentialScalarProd}
\begin{aligned}
f(x_0 + \epsilon)
&= \left. e^{\gpgradezero{\epsilon \grad}} f(x) \right\vert_{x=x_0}
\end{aligned}
\end{equation}

Or in full,

\begin{equation}\label{eqn:multivectorTaylors:taylorMultiScalarProd}
\begin{aligned}
f(x_0 + \epsilon)
&= \sum \inv{k!} \left. {\gpgradezero{\epsilon \grad}^k} f(x) \right\vert_{x=x_0}
\end{aligned}
\end{equation}

One could alternately express this in a notationally less different form using the scalar product operator instead of grade selection, if one writes

\begin{equation}\label{eqn:multivectorTaylors:540}
\begin{aligned}
{\epsilon \stardot \grad} &\equiv \gpgradezero{\epsilon \grad}
\end{aligned}
\end{equation}

However, regardless of the notation used, the fundamental definition is still going to be the same (and the same as in the vector case), which operationally is

\begin{equation}\label{eqn:multivectorTaylors:560}
\begin{aligned}
{\epsilon \conj \grad} f(x) = \gpgradezero{\epsilon \grad} f(x)
= \left. \PD{h}{f(x + h \epsilon)} \right\vert_{h=0}
\end{aligned}
\end{equation}

\makeexample{Complex numbers}{example:multivectorTaylors:581}{
\index{complex numbers}

The simplest grade mixed multivector space is that of the complex numbers.  Let us write out the directional derivative and gradient in this space explicitly.  Writing
\begin{equation}\label{eqn:multivectorTaylors:580}
\begin{aligned}
z_0 &= u + i v \\
z &= x + i y \\
\end{aligned}
\end{equation}

So we have
\begin{equation}\label{eqn:multivectorTaylors:600}
\begin{aligned}
\gpgradezero{z_0 \grad} f(z)
&= u \PD{x}{f} + v \PD{y}{f} \\
&= u \PD{x}{f} + i v \inv{i} \PD{y}{f} \\
&= \gpgradezero{ z_0 \left( \PD{x}{} + \inv{i} \PD{y}{} \right) } f(z) \\
\end{aligned}
\end{equation}

and we can therefore identify the gradient operator as

\begin{equation}\label{eqn:multivectorTaylors:620}
\begin{aligned}
\grad_{0,2} &= \PD{x}{} + \inv{i} \PD{y}{}
\end{aligned}
\end{equation}

Observe the similarity here between the vector gradient for a 2D Euclidean space, where we can form complex numbers by (left) factoring out a unit vector, as in

\begin{equation}\label{eqn:multivectorTaylors:640}
\begin{aligned}
\Bx
&= e_1 x + e_2 y \\
&= e_1 ( x + e_1 e_2 y ) \\
&= e_1 ( x + i y ) \\
&= e_1 z
\end{aligned}
\end{equation}

It appears that we can form this complex gradient, by (right) factoring out of the same unit vector from the vector gradient

\begin{equation}\label{eqn:multivectorTaylors:660}
\begin{aligned}
e_1 \PD{x}{} + e_2 \PD{y}{}
&=
\left( \PD{x}{} + e_2 e_1 \PD{y}{} \right) e_1 \\
&=
\left( \PD{x}{} + \inv{i} \PD{y}{} \right) e_1 \\
&=
\grad_{0,2} e_1 \\
\end{aligned}
\end{equation}

So, if we write \(\spacegrad\) as the \R{2} vector gradient, with \(\Bx = e_1 x + e_2 y = e_1 z\) as above, we have

\begin{equation}\label{eqn:multivectorTaylors:680}
\begin{aligned}
\spacegrad \Bx
&= \grad_{0,2} e_1 e_1 z \\
&= \grad_{0,2} z \\
\end{aligned}
\end{equation}

This is a rather curious equivalence between 2D vectors and complex numbers.

\paragraph{Comparison of contour integral and directional derivative Taylor series}
\index{contour integral}
\index{directional derivative}

Having a complex gradient is not familiar from standard complex variable theory.  Then again, neither is a non-contour integral formulation of complex Taylor series.  The two of these ought to be equivalent, which seems to imply there is a contour integral representation of the gradient in a complex number space too (one of the Hestenes paper's mentioned this but I did not understand the notation).

Let us do an initial comparison of the two.  We need a reminder of the contour integral form of the complex derivative.  For a function \(f(z)\) and its derivatives regular in a neighborhood of a point \(z_0\), we can evaluate

\begin{equation}\label{eqn:multivectorTaylors:700}
\begin{aligned}
\ointctrclockwise \frac{f(z) dz}{(z - z_0)^k}
&=
-\inv{k-1} \ointctrclockwise {f(z) dz}\left( \inv{(z - z_0)^{k-1}} \right)' \\
&=
\inv{k-1} \ointctrclockwise {f'(z) dz}\left( \inv{(z - z_0)^{k-1}} \right) \\
&=
\inv{(k-1)(k-2)} \ointctrclockwise {f^2(z) dz}\left( \inv{(z - z_0)^{k-2}} \right) \\
&=
\inv{(k-1)(k-2)\cdots(k-n)} \ointctrclockwise {f^n(z) dz}\left( \inv{(z - z_0)^{k-n}} \right) \\
% k-n = 1
% n = k-1
&=
\inv{(k-1)(k-2)\cdots(1)} \ointctrclockwise \frac{f^{k-1}(z) dz}{z - z_0} \\
&= \frac{2 \pi i}{(k-1)!} f^{k-1}(z_0)
\end{aligned}
\end{equation}

So we have

\begin{equation}\label{eqn:multivectorTaylors:720}
\begin{aligned}
\left. \frac{d^k}{dz^k} f(z) \right\vert_{z_0}
&=
\frac{k!}{2 \pi i}\ointctrclockwise \frac{f(z) dz}{(z - z_0)^{k+1}}
\end{aligned}
\end{equation}

Given this we now have a few alternate forms of complex Taylor series

\begin{equation}\label{eqn:multivectorTaylors:740}
\begin{aligned}
f(z_0 + \epsilon)
&= \sum \inv{k!} \left. \gpgradezero{\epsilon \grad}^k f(z) \right\vert_{z=z_0} \\
&= \sum \inv{k!} \epsilon^k \left. \frac{d^k}{dz^k} f(z) \right\vert_{z_0} \\
&= \inv{2 \pi i} \sum \epsilon^k \ointctrclockwise \frac{f(z) dz}{(z - z_0)^{k+1}}
\end{aligned}
\end{equation}

Observe that the the \(0,2\) subscript for the gradient has been dropped above (ie: this is the complex gradient, not the vector
form).

\paragraph{Complex gradient compared to the derivative}

A gradient operator has been identified by factoring it out of the directional derivative.  Let us compare this to a plain old complex derivative.

\begin{equation}\label{eqn:multivectorTaylors:760}
\begin{aligned}
f'(z_0) &= \lim_{z \rightarrow z_0} \frac{ f(z) - f(z_0) }{ z - z_0}
\end{aligned}
\end{equation}

In particular, evaluating this limit for \(z = z_0 + h\), approaching \(z_0\) along the x-axis, we have

\begin{equation}\label{eqn:multivectorTaylors:780}
\begin{aligned}
f'(z_0)
&= \lim_{z \rightarrow z_0} \frac{ f(z) - f(z_0) }{ z - z_0} \\
&= \lim_{h \rightarrow 0} \frac{ f(z_0 + h) - f(z_0) }{ h } \\
&= \PD{x}{f}(z_0)
\end{aligned}
\end{equation}

Evaluating this limit for \(z = z_0 + i h\), approaching \(z_0\) along the y-axis, we have

\begin{equation}\label{eqn:multivectorTaylors:800}
\begin{aligned}
f'(z_0)
&= \lim_{h \rightarrow 0} \frac{ f(z_0 + i h) - f(z_0) }{ i h } \\
&= -i \PD{y}{f}(z_0)
\end{aligned}
\end{equation}

We have the Cauchy equations by equating these, and if the derivative exists (ie: independent of path) we require at least

\begin{equation}\label{eqn:multivectorTaylors:820}
\begin{aligned}
\PD{x}{f}(z_0) =
-i \PD{y}{f}(z_0)
\end{aligned}
\end{equation}

Or
\begin{equation}\label{eqn:multivectorTaylors:840}
\begin{aligned}
0
&=
\PD{x}{f}(z_0) + i \PD{y}{f}(z_0) \\
&=
\tilde{\grad} f(z_0)
\end{aligned}
\end{equation}

Premultiplying by \(\grad\) produces the harmonic equation

\begin{equation}\label{eqn:multivectorTaylors:860}
\begin{aligned}
\grad \tilde{\grad} f = \left( \PDSq{x}{} + \PDSq{y}{} \right) f
\end{aligned}
\end{equation}

\paragraph{First order expansion around a point}

The above, while interesting or curious,
does not provide a way to express the differential operator directly in terms of the gradient.

We can write
\begin{equation}\label{eqn:multivectorTaylors:880}
\begin{aligned}
\left. \gpgradezero{ \epsilon \grad } f(z) \right\vert_{z_0}
&=
\frac{\epsilon}{ 2 \pi i } \ointctrclockwise \frac{f(z) dz}{(z - z_0)^{2}} \\
&=
\epsilon f'(z_0)
\end{aligned}
\end{equation}

One can probably integrate this in some circumstances (perhaps when f(z) is regular along the straight path from \(z_0\) to \(z = z_0 + \epsilon\)).  If so, then we have

\begin{equation}\label{eqn:multivectorTaylors:900}
\begin{aligned}
\epsilon \int_{s=z_0}^{z} f'(s) ds &= \int_{s=z_0}^{z} \left. \gpgradezero{ \epsilon \grad } f(z) \right\vert_{z=s} ds
\end{aligned}
\end{equation}

Or
\begin{equation}\label{eqn:multivectorTaylors:920}
\begin{aligned}
f(z) &= f(z_0) + \int_{s=z_0}^{z} \left. \inv{\epsilon}\gpgradezero{ \epsilon \grad } f(z) \right\vert_{z=s} ds
\end{aligned}
\end{equation}

Is there any validity to doing this?  The idea here is to play with some circumstances where we could see where the multivector gradient may show up.  Much more play is required, some of which for discovery and the rest to do things more rigorously.
}

\makeexample{4D scalar plus bivector space}{example:multivectorTaylors:801}{

Suppose we form a scalar, bivector space by factoring out the unit time vector in a Dirac vector representation

\begin{equation}\label{eqn:multivectorTaylors:940}
\begin{aligned}
x
&= x^\mu \gamma_\mu \\
&= \left( x^0 + x^k \gamma_k \gamma_0 \right) \gamma_0 \\
&= \left( x^0 + x^k \sigma_k \right) \gamma_0 \\
&= q \gamma_0 \\
\end{aligned}
\end{equation}

This \(q\) has the structure of a quaternion-like object (scalar, plus bivector), but the bivectors all have positive square.  Our directional derivative, for multivector direction \(Q = Q^0 + Q^k \sigma_k\) is

\begin{equation}\label{eqn:multivectorTaylors:960}
\begin{aligned}
\gpgradezero{Q \grad} f(q)
&= Q^0 \PD{x^0}{f} + \sum_k Q^k \PD{x^k}{f} \\
\end{aligned}
\end{equation}

So, we can write

\begin{equation}\label{eqn:multivectorTaylors:980}
\begin{aligned}
\grad
&= \PD{x^0}{} + \sum_k \sigma_k \PD{x^k}{} \\
\end{aligned}
\end{equation}

We can do something similar for an Euclidean four vector space

\begin{equation}\label{eqn:multivectorTaylors:1000}
\begin{aligned}
x
&= x^\mu e_\mu \\
&= \left( x^0 + x^k e_k e_0 \right) e_0 \\
&= \left( x^0 + x^k i_k \right) e_0 \\
&= q e_0 \\
\end{aligned}
\end{equation}

Here each of the bivectors \(i_k\) have a negative square, much more quaternion-like (and could easily be defined in an isomorphic fashion).  This time we have

\begin{equation}\label{eqn:multivectorTaylors:1020}
\begin{aligned}
\grad
&= \PD{x^0}{} + \sum_k \inv{i_k} \PD{x^k}{} \\
\end{aligned}
\end{equation}
}

   \chapter{Exterior derivative and chain rule components of the gradient}
      %
% Copyright � 2012 Peeter Joot.  All Rights Reserved.
% Licenced as described in the file LICENSE under the root directory of this GIT repository.
%

%
%
%\chapter{Exterior derivative and chain rule components of the gradient}
\index{exterior derivative}
\index{chain rule}
\index{gradient}
\label{chap:gradientAndForms}
%\date{March 31, 2008.  gradientAndForms.tex}

\section{Gradient formulation in terms of reciprocal frames}

We have seen how to calculate reciprocal frames as a method to find
components of a vector with respect to an arbitrary basis (does not have
to be orthogonal).

This can be applied to any vector:
%
\begin{equation}\label{eqn:gradientAndForms:20}
\Bx = \sum \Ba_i (\Ba^i \cdot \Bx) = \sum \Ba^i (\Ba_i \cdot \Bx)
\end{equation}
%
so why not the gradient operator too.
%
\begin{equation}
\grad = \sum \Ba_i (\Ba^i \cdot \grad) = \sum \Ba^i (\Ba_i \cdot \grad)
\end{equation}
%
The dot product part:
%
\begin{equation}
(\Ba \cdot \grad) f(\Bu) = \lim_{\tau \rightarrow 0}
\frac{ f(\Bu + \Ba \tau) - f(\Bu ) }{ \tau }
\end{equation}
%
we know how to calculate explicitly (from NFCM) and is the direction
derivative.

So this gives us an explicit factorization of the gradient into components
in some arbitrary set of directions, all weighted appropriately.

\section{Allowing the basis to vary according to a parametrization}

Now, if one allows the vector basis \({\Ba_i}\) to vary along a curve, it
is interesting to observe the consequences of this to the gradient expressed
as a component along the curve and perpendicular components.

Suppose that one has a parametrization \(\phi(u_1, u_2, \cdots, u_{n-1}) \in \mathbb{R}^n\), defining a generalized surface of degree one less than the space.

Provided these surface direction vectors are linearly independent and non
zero, we can writing \(\Bphi_{u_i} = \frac{\partial \Bphi}{ \partial u_i}\),
and form a basis for the space by extension with a reciprocal frame vector:
%
\begin{equation}\label{eqn:gradientAndForms:40}
\{ \Bphi_{u_1}, \Bphi_{u_2}, \cdots, \Bphi_{u_{n-1}},
(\Bphi_{u_1} \wedge \Bphi_{u_2} \cdots \wedge \Bphi_{u_{n-1}}) \inv{\BI_n}
 \}
\end{equation}
%
\makeexample{Try it with the simplest case}{example:gradientAndForms:100}{

Let us calculate \(\grad\) in this basis.  Intuition says this will
produce something like the exterior derivative from differential forms
for the component that is normal to the surface.

To make things easy, consider the absolutely simplest case, a curve
in \R{2}, with parametrization \(\Br = \Bphi(t)\).  The basis associated
with this curve at some point is
%
\begin{equation}\label{eqn:gradientAndForms:60}
\{ \Ba_1, \Ba_2 \} = \{ \Bphi_t, \BI \Bphi_t \}
\end{equation}
%
with a reciprocal basis of:
\begin{equation}\label{eqn:gradientAndForms:80}
\{ \Ba^1, \Ba^2 \} = \setlr{ \inv{\Bphi_t}, -\inv{\Bphi_t} \BI }
\end{equation}
%
In terms of this components, the gradient along the curve at the specified
point is:
%
\begin{equation}\label{eqn:gradientAndForms:160}
\begin{aligned}
\grad f
&= \left(\Ba^1 \Ba_1 \cdot \grad +\Ba^2 \Ba_2 \cdot \grad \right) f \\
&= \left(\inv{\Bphi_t} \Bphi_t \cdot \grad + \left( \BI \inv{\Bphi_t} \right) \left(\BI \Bphi_t \right) \cdot \grad \right) f \\
&= \inv{\Bphi_t}\left(\Bphi_t \cdot \grad -\BI \left(\BI \cdot \Bphi_t \right) \cdot \grad \right) f \\
&= \inv{\Bphi_t}\left(\Bphi_t \cdot \grad -\BI \left(\BI \cdot \left(\Bphi_t \wedge \grad \right) \right) \right) f \\
&= \inv{\Bphi_t}\left(\Bphi_t \cdot \grad -\BI \left(\BI \left(\Bphi_t \wedge \grad \right) \right) \right) f \\
&= \inv{\Bphi_t}\left(\Bphi_t \cdot \grad + \Bphi_t \wedge \grad \right) f \\
\end{aligned}
\end{equation}
%
Lo and behold, we come full circle through a mass of identities back to the geometric product.
As with many things in math, knowing the answer we can be clever and start from the answer going backwards.  This would have allowed the standard factorization of the gradient vector into
orthogonal components in the usual fashion:
%
\begin{equation}\label{eqn:gradAndForms:grad_dot_wedge}
\grad
= \inv{\Bphi_t}\left( \Bphi_t \grad \right)
= \inv{\Bphi_t}\left( \Bphi_t \cdot \grad + \Bphi_t \wedge \grad \right)
\end{equation}
%
Let us continue writing \(\Bphi(t) = x(t) \Be_1 + y(t) \Be_2\).  Then
%
\begin{equation}\label{eqn:gradientAndForms:100}
\Bphi' = x' \Be_1 + y' \Be_2
\end{equation}
\begin{equation}\label{eqn:gradientAndForms:120}
\Bphi' \cdot \grad =
\frac{dx}{dt} \frac{\partial}{\partial x}
+\frac{dy}{dt} \frac{\partial}{\partial y}
\end{equation}
%
\begin{equation}\label{eqn:gradientAndForms:140}
\Bphi' \wedge \grad =
\BI
\left(\frac{dx}{dt} \frac{\partial}{\partial y} -\frac{dy}{dt} \frac{\partial}{\partial x}\right)
\end{equation}
%
Combining these and inserting back into \eqnref{eqn:gradAndForms:grad_dot_wedge} we have
%
\begin{equation}\label{eqn:gradientAndForms:180}
\begin{aligned}
\grad
&= \frac{\Bphi'}{\left(\Bphi'\right)^2}\left( \Bphi' \cdot \grad + \Bphi' \wedge \grad \right) \\
&=
\inv{(x')^2 + (y')^2}
(x',y')\left(
\frac{dx}{dt} \frac{\partial}{\partial x}
+\frac{dy}{dt} \frac{\partial}{\partial y}
\right)
+
(-y',x')
\left(\frac{dx}{dt} \frac{\partial}{\partial y} -\frac{dy}{dt} \frac{\partial}{\partial x}\right)
\end{aligned}
\end{equation}
%
Now, here it is worth pointing out that the choice of the parametrization can break some of the assumptions made.  In particular
the curve can be completely continuous, but the parametrization could allow it to be zero for some interval since \((x(t), y(t))\)
can be picked to be constant for a ``time'' before continuing.

This problem is eliminated by picking an arc length parametrization.  Provided the curve is not degenerate (ie: a point), then
we have at least one of \(dx/ds \ne 0\), or \(dy/ds \ne 0\).  Additionally, by parametrization using arc length we have
\((dx/ds)^2 + (dy/ds)^2 = (ds/ds)^2 = 1\).  This eliminates the denominator leaving the following decomposition of the \R{2} gradient
%
\begin{equation}
\grad =
\mathLabelBox{(x',y')}{Unit tangent vector}
\mathLabelBox
[
   labelstyle={below of=m\themathLableNode, below of=m\themathLableNode}
]
{
\left(
\frac{dx}{ds} \frac{\partial}{\partial x}
+\frac{dy}{ds} \frac{\partial}{\partial y}
\right)
}{Chain Rule in operator form.}
+
\mathLabelBox{(-y',x')}{Unit normal vector}
\mathLabelBox
[
   labelstyle={below of=m\themathLableNode, below of=m\themathLableNode}
]
{
\left(
\frac{dx}{ds} \frac{\partial}{\partial y}
-\frac{dy}{ds} \frac{\partial}{\partial x}
\right)
}{Exterior derivative operator.}
\end{equation}
%
Thus, loosely speaking we have the chain rule as the scalar component of the unit tangent vector along a parametrized curve,
and we have the exterior derivative as the component of the gradient that lies colinear to a unit normal to the curve (believe this
is the unit normal that points inwards to the curvature if there is any).

} % example

\makeexample{Extension to higher dimensional curves}{example:gradientAndForms:200}{

In \R{3} or above we can perform the same calculation.  The result is similar and straightforward to derive:
%
\begin{equation}
\nabla
=
\mathLabelBox{ \left(\frac{dx_1}{ds}, \frac{dx_2}{ds}, \cdots, \frac{dx_n}{ds}\right) }{Unit tangent.}
\sum_{i=1}^n \frac{dx_i}{ds}\frac{\partial}{\partial x_i}
+ \sum_{1 \le i < j \le n}
\mathLabelBox{
\left(
\mathbf{e}_j \frac{dx_i}{ds}
-\mathbf{e}_i \frac{dx_j}{ds}
\right)
}{Normal to unit tangent.}
\left(
\frac{dx_i}{ds}\frac{\partial}{\partial x_j}
-\frac{dx_j}{ds}\frac{\partial}{\partial x_i}
\right)
\end{equation}
%
Here we have a set of normals to the unit tangent.  For \R{3}, we have \(ij=\{12,13,23\}\).  One of these unit normals must
be linearly dependent on the other two (or zero).  The exterior scalar factors here loose some of their resemblance to the
exterior derivative here.  Perhaps a parametrized (hyper-)surface is required to get the familiar form for \R{3} or above.
} % example

   \chapter{Spherical and hyperspherical parametrization}
      %
% Copyright � 2012 Peeter Joot.  All Rights Reserved.
% Licenced as described in the file LICENSE under the root directory of this GIT repository.
%

%
%
%\chapter{Spherical and hyperspherical parametrization}
\index{spherical parametrization}
\index{hyperspherical parametrization}
\label{chap:nvolume}
%\date{Feb 26, 2009.  nvolume.tex}

\section{Motivation}

In \citep{gabookII:PJ4dFourier} a 4D Fourier transform solution of Maxwell's equation yielded a Green's function of the form

\begin{equation}\label{eqn:nvolume:20}
\begin{aligned}
G(x) = \iiiint \frac{e^{i k_\mu x^\mu}}{k_\nu k^\nu} dk_1 dk_2 dk_3 dk_4
\end{aligned}
\end{equation}

To attempt to ``evaluate'' this integral, as done in
\citep{gabookII:PJpoisson}
to produce the retarded time potentials,
a hypervolume equivalent to spherical polar coordinate
parametrization is probably desirable.

Before attempting to tackle the problem of interest, the basic question
of how to do volume and weighted volume integrals over a hemispherical volumes
must be considered.  Doing this for both Euclidean and Minkowski metrics will have to be covered.

\section{Euclidean n-volume}
\index{n-volume}

\subsection{Parametrization}

The \href{http://en.wikipedia.org/wiki/Hypersphere}{wikipedia article on n-volumes} gives a parametrization, which I will write out explicitly for the first few dimensions

\begin{itemize}
\item 1-sphere (circle)

\begin{equation}\label{eqn:nvolume:40}
\begin{aligned}
x^1 &= r \cos\phi_1 \\
x^2 &= r \sin\phi_1
\end{aligned}
\end{equation}

\item 2-sphere (sphere)

\begin{equation}\label{eqn:nvolume:60}
\begin{aligned}
x^1 &= r \cos\phi_1 \\
x^2 &= r \sin\phi_1 \cos\phi_2 \\
x^3 &= r \sin\phi_1 \sin\phi_2
\end{aligned}
\end{equation}

\item 3-sphere (hypersphere)

\begin{equation}\label{eqn:nvolume:80}
\begin{aligned}
x^1 &= r \cos\phi_1 \\
x^2 &= r \sin\phi_1 \cos\phi_2 \\
x^3 &= r \sin\phi_1 \sin\phi_2 \cos\phi_3 \\
x^4 &= r \sin\phi_1 \sin\phi_2 \sin\phi_3 \\
\end{aligned}
\end{equation}

\end{itemize}

By inspection one can see that we have the desired \(r^2 = \sum_i (x^i)^2\) relation.  Each of these can be vectorized to produce
a parametrized vector that can trace out all the possible points on the volume

\begin{equation}\label{eqn:nvolume:100}
\begin{aligned}
\Br &= \sigma_k x^k
\end{aligned}
\end{equation}

\subsection{Volume elements}

We can form a parallelogram area (or parallelepiped volume, ...) element for any parametrized surface by taking wedge products, as in \cref{fig:tangent_area_form}.  This can also be done for this spherical parametrization too.

\imageFigure{../figures/gabook/tangent_area_form}{Tangent vector along curves of parametrized vector}{fig:tangent_area_form}{0.4}

For example for the circle we have

\begin{equation}\label{eqn:nvolume:120}
\begin{aligned}
dV_{\Rm{2}}
&= \PD{r}{\Br} \wedge \PD{\phi_1}{\Br} dr d\phi_1 \\
&= \left(\PD{r}{}r(\cos\phi_1, \sin\phi_1) \right) \wedge \left(\PD{\phi_1}{}r(\cos\phi_1, \sin\phi_1) \right) dr d\phi_1 \\
&= (\cos\phi_1, \sin\phi_1) \wedge (-\sin\phi_1, \cos\phi_1) r dr d\phi_1 \\
&= (\cos^2\phi_1 \sigma_1 \sigma_2 - \sin^2\phi_1 \sigma_2 \sigma_1 ) r dr d\phi_1 \\
&= r dr d\phi_1 \sigma_1 \sigma_2 \\
\end{aligned}
\end{equation}

And for the sphere
\begin{equation}\label{eqn:nvolume:140}
\begin{aligned}
dV_{\Rm{3}}
&= \PD{r}{\Br} \wedge \PD{\phi_1}{\Br} \wedge \PD{\phi_2}{\Br} dr d\phi_1 d\phi_2 \\
&= (\cos\phi_1, \sin\phi_1 \cos\phi_2, \sin\phi_1 \sin\phi_2)  \\
&\wedge (-\sin\phi_1, \cos\phi_1 \cos\phi_2, \cos\phi_1 \sin\phi_2)  \\
&\wedge (0, -\sin\phi_1 \sin\phi_2, \sin\phi_1 \cos\phi_2) r^2 dr d\phi_1 d\phi_2 \\
&=
\begin{vmatrix}
\cos\phi_1 & \sin\phi_1 \cos\phi_2 & \sin\phi_1 \sin\phi_2  \\
-\sin\phi_1 & \cos\phi_1 \cos\phi_2 & \cos\phi_1 \sin\phi_2  \\
0 & -\sin\phi_1 \sin\phi_2 & \sin\phi_1 \cos\phi_2  \\
\end{vmatrix} r^2 dr d\phi_1 d\phi_2 \sigma_1 \sigma_2 \sigma_3 \\
&=
r^2 dr \sin\phi_1 d\phi_1 d\phi_2 \sigma_1 \sigma_2 \sigma_3 \\
\end{aligned}
\end{equation}

And finally for the hypersphere

\begin{equation}\label{eqn:nvolume:160}
\begin{aligned}
dV_{\Rm{4}} &=
\begin{vmatrix}
\cos\phi_1 & \sin\phi_1 \cos\phi_2 & \sin\phi_1 \sin\phi_2 \cos\phi_3 & \sin\phi_1 \sin\phi_2 \sin\phi_3  \\
-\sin\phi_1 & \cos\phi_1 \cos\phi_2 & \cos\phi_1 \sin\phi_2 \cos\phi_3 & \cos\phi_1 \sin\phi_2 \sin\phi_3  \\
         0 & -\sin\phi_1 \sin\phi_2 & \sin\phi_1 \cos\phi_2 \cos\phi_3 & \sin\phi_1 \cos\phi_2 \sin\phi_3  \\
         0 &                     0 & -\sin\phi_1 \sin\phi_2 \sin\phi_3 & \sin\phi_1 \sin\phi_2 \cos\phi_3  \\
\end{vmatrix} \\
&\quad r^3 dr d\phi_1 d\phi_2 d\phi_3 \sigma_1 \sigma_2 \sigma_3 \sigma_4 \\
&=
%
%
%C_1  &  S_1 C_2 &  S_1 S_2 C_3 & S_1 S_2 S_3  \\
%-S_1 &  C_1 C_2 &  C_1 S_2 C_3 & C_1 S_2 S_3  \\
%   0 & -S_1 S_2 &  S_1 C_2 C_3 & S_1 C_2 S_3  \\
%   0 &        0 & -S_1 S_2 S_3 & S_1 S_2 C_3  \\
%
%
%C_1^2 C_2 S_1^2 C_2 S_2
%+C_1 S_1 S_2^3 C_1 S_1
%+S_1^2 C_2 S_1^2 C_2 S_2
%+S_1^2 S_2 S_1^2 S_2^2
%
%
%+C_1^2 C_2^2 S_1^2 S_2
%+C_1^2 S_1^2 S_2^3
%+S_1^4 C_2^2 S_2
%+S_1^4 S_2^3
%
%
%+ S_1^2 S_2
r^3 dr \sin^2\phi_1 d\phi_1 \sin\phi_2 d\phi_2 d\phi_3 \sigma_1 \sigma_2 \sigma_3 \sigma_4 \\
\end{aligned}
\end{equation}

Each of these is consistent with the result in the wiki page.

\subsection{Some volume computations}

Let us apply the above results to compute the corresponding n-volume's.


\begin{itemize}
\item 1-sphere (circle)
\begin{equation}\label{eqn:nvolume:180}
\begin{aligned}
V_{\Rm{2}}
&= 4 \int_{0}^R r dr \int_{0}^{\pi/2} d\phi_1 \\
&= \pi R^2
\end{aligned}
\end{equation}

\item 2-sphere (sphere)

\begin{equation}\label{eqn:nvolume:200}
\begin{aligned}
V_{\Rm{3}}
&= 8 \int_0^R r^2 dr \int_0^{\pi/2} \sin\phi_1 d\phi_1 \int_0^{\pi/2} d\phi_2 \\
&= 8 \inv{3} R^3 \left( {\left. -\cos\phi_1 \right\vert}_0^{\pi/2} \right) \frac{\pi}{2} \\
&= \frac{4\pi R^3}{3} \\
\end{aligned}
\end{equation}

Okay, so far so good.

\item 3-sphere (hypersphere)

\begin{equation}\label{eqn:nvolume:220}
\begin{aligned}
V_{\Rm{3}}
&= 16 \int_0^R r^3 dr \int_0^{\pi/2} \sin^2\phi_1 d\phi_1 \int_0^{\pi/2} \sin\phi_2 d\phi_2 \int_0^{\pi/2} d\phi_3 \\
&= 2 \pi R^4 \int_0^{\pi/2} \sin^2\phi_1 d\phi_1 \\
&= \pi R^4 \left( {\left.  \phi_1 - \cos\phi_1 \sin\phi_1 \right\vert}_0^{\pi/2} \right) \\
&= \frac{\pi^2 R^4}{2} \\
\end{aligned}
\end{equation}

This is also consistent with the formula supplied in the wiki article.

\end{itemize}

\subsection{Range determination}

What I have done here though it integrate over only one of the quadrants, and multiply by \(2^n\).  This avoided the more tricky issue of what exact range of angles is required for a complete and non-overlapping cover of the surface.

The wiki article says that the range is \([0,2\pi]\) for the last angle and \([0,\pi]\) for the others.
Reevaluating the integrals above shows that this does work, but that is a bit of a cheat, and it is not obvious
to me past \R{3} that this should be the case.

How can this be rationalized?

\begin{itemize}
\item circle
For the case of the circle what are the end points in each of the quadrants?  These are (with \(r=1\))

\begin{equation}\label{eqn:nvolume:240}
\begin{aligned}
(\cos\phi_1, \sin\phi_1)_{\phi_1 = 0}        &= ( 1, 0 ) = \sigma_1 \\
(\cos\phi_1, \sin\phi_1)_{\phi_1 = \pi/2}    &= ( 0, 1 ) = \sigma_2 \\
(\cos\phi_1, \sin\phi_1)_{\phi_1 = \pi}      &= ( -1, 0 ) = -\sigma_1 \\
(\cos\phi_1, \sin\phi_1)_{\phi_1 = 3\pi/2}   &= ( 0, -1 ) = -\sigma_2
\end{aligned}
\end{equation}

As expected, each of the \(\pi/2\) increments traces out the points in successive quadrants.

\item sphere

Again with \(r=1\), some representative points on the circle are

\begin{tabular}{|l|l|l|l|}
\hline
\(\phi_1\) & \(\phi_2\) & \((\cos\phi_1, \sin\phi_1 \cos\phi_2, \sin\phi_1 \sin\phi_2)\) & \(\Br\) \\
\hline
0 & 0 & \((1,0,0)\) & \(\sigma_1\) \\
0 & \(\pi/2\) &  \((1,0,0)\) & \(\sigma_1\) \\
0 & \(\pi\) &  \((1,0,0)\) & \(\sigma_1\) \\
0 & \(3\pi/2\) &  \((1,0,0)\) & \(\sigma_1\) \\
%
% \((0, \cos\phi_2, \sin\phi_2)\)
%
% traces out a circle in y,z plane
%
\(\pi/2\) & 0 &  \((0,1,0)\) & \(\sigma_2\) \\
\(\pi/2\) & \(\pi/2\) &  \((0,0,1)\) & \(\sigma_3\) \\
\(\pi/2\) & \(\pi\) &  \((0,-1,0)\) & \(-\sigma_2\) \\
\(\pi/2\) & \(3\pi/2\) &  \((0,0,-1)\) & \(-\sigma_3\) \\
%
%
% \((-1, 0\cos\phi_2, \sin\phi_1 \sin\phi_2)\) & \(\Br\) \\
%
\(\pi\) & 0 &  \((-1,0,0)\) & \(-\sigma_1\) \\
\(\pi\) & \(\pi/2\) &  \((-1,0,0)\) & \(-\sigma_1\) \\
\(\pi\) & \(\pi\) &  \((-1,0,0)\) & \(-\sigma_1\) \\
\(\pi\) & \(3\pi/2\) & \((-1,0,0)\) & \(-\sigma_1\) \\
\hline
\end{tabular}

The most informative of these is for \(\phi_1 = \pi/2\), where we had \(\Br = (0, \cos\phi_2, \sin\phi_2)\), and our points trace out a path along the unit circle of the \(y,z\) plane.  At \(\phi_1 = 0\) our point \(\Br = \sigma_1\) did not move, and at \(\phi_1 = \pi\) we are at the other end of the sphere, also fixed.  A reasonable guess is that at each \(\phi_1\) we trace out a different circle in the \(y,z\) plane.

We can write, with \(\sigma_{23} = \sigma_1 \wedge \sigma_2 = \sigma_1 \sigma_2\),

\begin{equation}\label{eqn:nvolume:260}
\begin{aligned}
\Br
%&= \cos\phi_1 (1, \tan\phi_1 \cos\phi_2, \tan\phi_1 \sin\phi_2) \\
&= \cos\phi_1 \sigma_1 + \sin\phi_1 ( \cos\phi_2 \sigma_2 + \sin\phi_2 \sigma_3 ) \\
&= \cos\phi_1 \sigma_1 + \sin\phi_1 \sigma_2 ( \cos\phi_2 + \sin\phi_2 \sigma_2 \sigma_3 ) \\
\end{aligned}
\end{equation}

Or, in exponential form

\begin{equation}\label{eqn:nvolume:sphereVec}
\begin{aligned}
\Br &= \cos\phi_1 \sigma_1 + \sin\phi_1 \sigma_2 \exp( \sigma_{23} \phi_2 )
\end{aligned}
\end{equation}

Put this way the effects of the parametrization is clear.   For each fixed \(\phi_1\), the exponential traces out a circle in the \(y,z\) plane, starting at the point \(\Br = \cos\phi_1 \sigma_1 + \sin\phi_1 \sigma_2\).  \(\phi_1\) traces out a semi-circle in the \(x,y\) plane.

FIXME: picture.

This would have been easy enough to understand if starting from a picture and constructing the parametrization.  Seeing what the
geometry is from the algebra requires a bit more (or different) work.  Having done it, are we now prepared to understand the geometry
of the hypersphere parametrization.

\item hypersphere.

The vector form in the spherical case was convenient for extracting geometric properties.  Can we do that here too?

\begin{equation}\label{eqn:nvolume:280}
\begin{aligned}
\Br
&= \sigma_1 \cos\phi_1 + \sigma_2 \sin\phi_1 \cos\phi_2 + \sigma_3 \sin\phi_1 \sin\phi_2 \cos\phi_3 + \sigma_4 \sin\phi_1 \sin\phi_2 \sin\phi_3 \\
&= \sigma_1 \cos\phi_1 + \sigma_2 \sin\phi_1 \cos\phi_2 + \sigma_3 \sin\phi_1 \sin\phi_2 (\cos\phi_3 + \sigma_{34} \sin\phi_3) \\
&= \sigma_1 \cos\phi_1 + \sigma_2 \sin\phi_1 \cos\phi_2 + \sigma_3 \sin\phi_1 \sin\phi_2 \exp(\sigma_{34}\phi_3) \\
&= \sigma_1 \cos\phi_1 + \sigma_2 \sin\phi_1 (\cos\phi_2 + \sigma_{23} \sin\phi_2 \exp(\sigma_{34}\phi_3) ) \\
\end{aligned}
\end{equation}

Observe that if \(\phi_3 = 0\) we have

\begin{equation}\label{eqn:nvolume:300}
\begin{aligned}
\Br
&= \sigma_1 \cos\phi_1 + \sigma_2 \sin\phi_1 \exp( \sigma_{23} \phi_2 ) \\
\end{aligned}
\end{equation}

Which is exactly the parametrization of a half sphere (\(\phi_2 \in [0,\pi]\)).  Contrast this to the semi-circle that \(\phi_1\) traced out in the spherical case.

In the spherical case, the points \(\phi_1 = \pi/2\) were nicely representative.  For the hypersphere those points are

\begin{equation}\label{eqn:nvolume:320}
\begin{aligned}
\Br &= \sigma_2 \cos\phi_2 + \sigma_3 \sin\phi_2 \exp(\sigma_{34}\phi_3) \\
\end{aligned}
\end{equation}

We saw above that this is the parametrization of a sphere.

Also like the spherical case, we have \(\Br = \pm \sigma_1\) at \(\phi_1 = 0\), and \(\phi_1 = \pi\) respectively.

The geometrical conclusion is that for each \(\phi_1 \in [0,\pi/2]\) range the points \(\Br\) trace out increasingly
larger spheres, and after that decreasing sized spheres until we get to a point again at \(\phi_1 = \pi\).

\end{itemize}

\section{Minkowski metric sphere}

\subsection{2D hyperbola}

Our 1-sphere equation was all the points on the curve

\begin{equation}\label{eqn:nvolume:340}
\begin{aligned}
x^2 + y^2 = r^2
\end{aligned}
\end{equation}

The hyperbolic equivalent to this is

\begin{equation}\label{eqn:nvolume:360}
\begin{aligned}
x^2 - y^2 = r^2
\end{aligned}
\end{equation}

Although this is not a closed curve like the circle.  To put this in a
more natural physical context, lets write (with \(c=1\))

\begin{equation}\label{eqn:nvolume:380}
\begin{aligned}
\Br = \gamma_0 t + \gamma_1 x
\end{aligned}
\end{equation}

So the equation of the 1-hyperboloid becomes

\begin{equation}\label{eqn:nvolume:400}
\begin{aligned}
\Br^2 = t^2 - x^2 = r^2
\end{aligned}
\end{equation}

We can parametrize this with complex angles \(i\phi\)

\begin{equation}\label{eqn:nvolume:420}
\begin{aligned}
\Br = r (\gamma_0 \cosh\phi + \gamma_1 \sinh\phi)
\end{aligned}
\end{equation}

This gives us

\begin{equation}\label{eqn:nvolume:440}
\begin{aligned}
\Br^2 = r^2 (\cosh^2\phi - \sinh^2\phi) = r^2
\end{aligned}
\end{equation}

as desired.  Like the circle, writing \(\gamma_{01} = \gamma_0 \wedge \gamma_1\), an exponential form also works nicely

\begin{equation}\label{eqn:nvolume:460}
\begin{aligned}
\Br = r \gamma_0 \exp(\gamma_{01} \phi)
\end{aligned}
\end{equation}

Here the square is

\begin{equation}\label{eqn:nvolume:480}
\begin{aligned}
\Br^2
&= r^2 \gamma_0 \exp(\gamma_{01} \phi) \gamma_0 \exp(\gamma_{01} \phi) \\
&= r^2 \exp(-\gamma_{01} \phi) (\gamma_0)^2 \exp(\gamma_{01} \phi) \\
&= r^2 \exp(-\gamma_{01} \phi) \exp(\gamma_{01} \phi) \\
&= r^2 \\
\end{aligned}
\end{equation}

Again as desired.

\subsection{3D hyperbola}

Unlike the circle, a pure hyperbolic parametrization does not work to construct a Minkowski
square signature.  Consider for example

\begin{equation}\label{eqn:nvolume:500}
\begin{aligned}
\Br
&= \cosh \phi \gamma_0 + \gamma_1 \sinh\phi \cosh \psi + \gamma_2 \sinh\phi \sinh\psi \\
\end{aligned}
\end{equation}

Squaring this we have
\begin{equation}\label{eqn:nvolume:520}
\begin{aligned}
\Br^2
&= \cosh^2 \phi - \sinh^2\phi (\cosh^2 \psi + \sinh^2\psi) \\
\end{aligned}
\end{equation}

We would get the desired result if we chop off the \(h\) in all the \(\psi\) hyperbolic functions.  This
shows that an appropriate parametrization is instead

\begin{equation}\label{eqn:nvolume:540}
\begin{aligned}
\Br
&= \cosh \phi \gamma_0 + \gamma_1 \sinh\phi \cos\psi + \gamma_2 \sinh\phi \sin\psi \\
\end{aligned}
\end{equation}

This now squares to 1.  To see how to extend this to higher dimensions (of which we only need one more)
we can factor out a \(\gamma_0\)

\begin{equation}\label{eqn:nvolume:560}
\begin{aligned}
\Br
&= \gamma_0 (\cosh \phi - \sinh\phi (\mathLabelBox{\sigma_1 \cos\psi + \sigma_2 \sin\psi}{spatial vector parametrization of circle}) ) \\
\end{aligned}
\end{equation}

Now to extend this to three dimensions we have just to substituted the spherical parametrization from \eqnref{eqn:nvolume:sphereVec}

\begin{equation}\label{eqn:nvolume:580}
\begin{aligned}
\Br
&= r \gamma_0 (\cosh \phi_0 - \sinh\phi_0 ( \cos\phi_1 \sigma_1 + \sin\phi_1 \sigma_2 \exp( \sigma_{23} \phi_2 )) ) \\
&= r (\gamma_0 \cosh \phi_0 + \sinh\phi_0 ( \cos\phi_1 \gamma_1 + \sin\phi_1 \gamma_2 \exp( \gamma_{32} \phi_2 )) ) \\
\end{aligned}
\end{equation}

\subsection{Summarizing the hyperbolic vector parametrization}

Our parametrization in two, three, and four dimensions, respectively, are
\begin{equation}\label{eqn:nvolume:600}
\begin{aligned}
\Br_2 &= r (\gamma_0 \cosh \phi_0 + \sinh\phi_0 \gamma_1) \\
\Br_3 &= r (\gamma_0 \cosh \phi_0 + \sinh\phi_0 \gamma_1 \exp( \gamma_{21} \phi_1 ) ) \\
\Br_4 &= r (\gamma_0 \cosh \phi_0 + \sinh\phi_0 ( \cos\phi_1 \gamma_1 + \sin\phi_1 \gamma_2 \exp( \gamma_{32} \phi_2 )) )
\end{aligned}
\end{equation}

\subsection{Volume elements}

What are our volume elements using this parametrization can be calculated as above.

\subsubsection{For one spatial dimension we have}

\begin{equation}\label{eqn:nvolume:620}
\begin{aligned}
dV_2 \gamma_0 \gamma_1
&=
\begin{vmatrix}
\cosh \phi_0 & \sinh\phi_0  \\
\sinh \phi_0 & \cosh\phi_0
\end{vmatrix} r dr d\phi_0 \\
&=
r dr d\phi_0 \\
\end{aligned}
\end{equation}

\subsubsection{For two spatial dimensions we have}

%\begin{align*}
%dV_3 \gamma_0 \gamma_1 \gamma_2
%&= \left(\PD{r}{\Br} \wedge \PD{\phi_0}{\Br} \wedge \PD{\phi_1}{\Br} \right) dr d\phi_0 d\phi_1 \\
%&\langle \\
%& (\gamma_0 \cosh \phi_0 + \sinh\phi_0 \gamma_1 \exp( \gamma_{21} \phi_1 ) ) \times \\
%& (\gamma_0 \sinh \phi_0 + \cosh\phi_0 \gamma_1 \exp( \gamma_{21} \phi_1 ) ) \times \\
%& (\gamma_0 \cosh \phi_0 + \sinh\phi_0 \gamma_1 \gamma_{21} \exp( \gamma_{21} \phi_1 ) )  \\
%\rangle_{3} \\
%& r^2 dr d\phi_0 d\phi_1
%\end{align*}
%
%That first product within the grade selection is
%\begin{align*}
%&(\gamma_0 \cosh \phi_0 + \sinh\phi_0 \gamma_1 \exp( \gamma_{21} \phi_1 ) )
%(\gamma_0 \sinh \phi_0 + \cosh\phi_0 \gamma_1 \exp( \gamma_{21} \phi_1 ) )  \\
%&= \gamma_{01} ( \cosh^2\phi_0 \exp( \gamma_{21} \phi_1)  -\sinh^2\phi_0 \exp( -\gamma_{21} \phi_1 ) ) \\
%&= \gamma_{01} \cos\phi_1
%%+\gamma_{0121} \sin\phi_1 \cosh^2 \phi_0
%+\gamma_{02} \sin\phi_1 \cosh^2 \phi_0
%%-\gamma_{0112} \sin\phi_1 \sinh^2 \phi_0
%+\gamma_{02} \sin\phi_1 \sinh^2 \phi_0 \\
%&= \gamma_{01} \cos\phi_1 +\gamma_{02} \sin\phi_1 (\cosh^2 \phi_0 + \sinh^2 \phi_0) \\
%\end{align*}
%
%and the third factor is
%\begin{align*}
%\gamma_0 \cosh \phi_0 + \sinh\phi_0 \gamma_{2} \exp( \gamma_{21} \phi_1 )
%\end{align*}
%
%So this last product is
%\begin{align*}
%&(\gamma_{01} \cos\phi_1 +\gamma_{02} \sin\phi_1 (\cosh^2 \phi_0 + \sinh^2 \phi_0) )
%(\gamma_0 \cosh \phi_0 + \sinh\phi_0 \gamma_{2} \exp( \gamma_{21} \phi_1 ) ) \\
%&=
%(\gamma_{01} \cos\phi_1 +\gamma_{02} \sin\phi_1 (\cosh^2 \phi_0 + \sinh^2 \phi_0) ) \sinh\phi_0 (\gamma_{2} \cos \phi_1 -\gamma_{1} \sin\phi_1 ) \\
%&=
% \gamma_{012} \cos^2\phi_1
%-\gamma_{021} \sin^2\phi_1 (\cosh^2 \phi_0 + \sinh^2 \phi_0) ) \sinh\phi_0  \\
%&=
% \gamma_{012} (\cos^2\phi_1 +\sin^2\phi_1 (\cosh^2 \phi_0 + \sinh^2 \phi_0) ) \sinh\phi_0 ) \\
%\end{align*}
%
%Note that above in many spots the grade three selection was used to discard terms that have no contribution.  The final result, without much
%potential more for simplification appears to be
%
%\begin{align*}
%dV_3 &= (\cos^2\phi_1 +\sin^2\phi_1 (\cosh^2 \phi_0 + \sinh^2 \phi_0) ) \sinh\phi_0 r^2 dr d\phi_0 d\phi_1
%\end{align*}
%
%I do not trust this result since I expected something simpler.  Let us try again

\begin{equation}\label{eqn:nvolume:640}
\begin{aligned}
\Br_3 &= r (\gamma_0 \cosh \phi_0 + \gamma_1 \sinh\phi_0 \cos\phi_1 + \gamma_2 \sinh\phi_0 \sin\phi_2 ) \\
\end{aligned}
\end{equation}

The derivatives are
\begin{equation}\label{eqn:nvolume:660}
\begin{aligned}
\PD{r}{\Br_3} &= \gamma_0 \cosh \phi_0 + \gamma_1 \sinh\phi_0 \cos\phi_1 + \gamma_2 \sinh\phi_0 \sin\phi_1  \\
\inv{r} \PD{\phi_0}{\Br_3} &= \gamma_0 \sinh \phi_0 + \gamma_1 \cosh\phi_0 \cos\phi_1 + \gamma_2 \cosh\phi_0 \sin\phi_1  \\
\inv{r} \PD{\phi_1}{\Br_3} &= -\gamma_1 \sinh\phi_0 \sin\phi_1 + \gamma_2 \sinh\phi_0 \cos\phi_1
\end{aligned}
\end{equation}

Or
\begin{equation}\label{eqn:nvolume:680}
\begin{aligned}
\PD{r}{\Br_3} &= \gamma_0 \cosh \phi_0 + \gamma_1 \sinh\phi_0 \exp (\gamma_{21} \phi_1 ) \\
\inv{r} \PD{\phi_0}{\Br_3} &= \gamma_0 \sinh \phi_0 + \cosh\phi_0 \exp (-\gamma_{21} \phi_1 ) \gamma_1 \\
\inv{r} \PD{\phi_1}{\Br_3} &= \sinh\phi_0 \gamma_2 \exp(\gamma_{21} \phi_1 )
\end{aligned}
\end{equation}

Multiplying this out, discarding non-grade three terms we have

\begin{equation}\label{eqn:nvolume:700}
\begin{aligned}
&(\gamma_{10} \sinh^2\phi_0 \exp (\gamma_{21} \phi_1 ) +\gamma_{01} \cosh^2 \phi_0 \exp (\gamma_{21} \phi_1 ) ) \sinh\phi_0 \gamma_2 \exp(\gamma_{21} \phi_1 ) \\
&=\gamma_{01} \exp (\gamma_{21} \phi_1 ) \sinh\phi_0 \exp(-\gamma_{21} \phi_1 ) \gamma_2 \\
&=\gamma_{01} \sinh\phi_0 \gamma_2 \\
\end{aligned}
\end{equation}

This gives us

\begin{equation}\label{eqn:nvolume:720}
\begin{aligned}
dV_3 &= r^2 \sinh\phi_0 dr d\phi_0 d\phi_1 \\
\end{aligned}
\end{equation}

\subsubsection{For three spatial dimensions we have}

\begin{equation}\label{eqn:nvolume:740}
\begin{aligned}
\Br_4 &= r (\gamma_0 \cosh \phi_0 + \sinh\phi_0 ( \cos\phi_1 \gamma_1 + \sin\phi_1 \gamma_2 \exp( \gamma_{32} \phi_2 )) )
\end{aligned}
\end{equation}

So our derivatives are

\begin{equation}\label{eqn:nvolume:760}
\begin{aligned}
\PD{r}{\Br_4} &= \gamma_0 \cosh \phi_0 + \sinh\phi_0 ( \cos\phi_1 \gamma_1 + \sin\phi_1 \gamma_2 \exp( \gamma_{32} \phi_2 )) \\
\inv{r}\PD{\phi_0}{\Br_4} &= \gamma_0 \sinh \phi_0 + \cosh\phi_0 ( \cos\phi_1 \gamma_1 + \sin\phi_1 \gamma_2 \exp( \gamma_{32} \phi_2 )) \\
\inv{r}\PD{\phi_1}{\Br_4} &= \sinh\phi_0 ( -\sin\phi_1 \gamma_1 + \cos\phi_1 \gamma_2 \exp( \gamma_{32} \phi_2 )) \\
\inv{r}\PD{\phi_2}{\Br_4} &= \sinh\phi_0 \sin\phi_1 \gamma_3 \exp( \gamma_{32} \phi_2 )
\end{aligned}
\end{equation}

In shorthand, writing \(C\) and \(S\) for the trig and hyperbolic functions as appropriate, we have

\begin{equation}\label{eqn:nvolume:780}
\begin{aligned}
\gamma_0 C_0 + S_0 C_1 \gamma_1 + S_0 S_1 \gamma_2 \exp( \gamma_{32} \phi_2 )  \\
\gamma_0 S_0 + C_0 C_1 \gamma_1 + C_0 S_1 \gamma_2 \exp( \gamma_{32} \phi_2 )  \\
 -S_0 S_1 \gamma_1 + S_0 C_1 \gamma_2 \exp( \gamma_{32} \phi_2 )  \\
S_0 S_1 \gamma_3 \exp( \gamma_{32} \phi_2 )
\end{aligned}
\end{equation}

Multiplying these out and dropping terms that will not contribute grade four bits is needed to calculate the volume element.
The full product for the first two derivatives is

\begin{equation}\label{eqn:nvolume:800}
\begin{aligned}
&\gamma_0 C_0 \gamma_0 S_0
+\gamma_0 C_0 C_0 C_1 \gamma_1
+\gamma_0 C_0 C_0 S_1 \gamma_2 \exp( \gamma_{32} \phi_2 ) \\
&+S_0 C_1 \gamma_1 \gamma_0 S_0
+S_0 C_1 \gamma_1 C_0 C_1 \gamma_1
+S_0 C_1 \gamma_1 C_0 S_1 \gamma_2 \exp( \gamma_{32} \phi_2 ) \\
&+S_0 S_1 \gamma_2 \exp( \gamma_{32} \phi_2 ) \gamma_0 S_0
+S_0 S_1 \gamma_2 \exp( \gamma_{32} \phi_2 ) C_0 C_1 \gamma_1
+S_0 S_1 \gamma_2 \exp( \gamma_{32} \phi_2 ) C_0 S_1 \gamma_2 \exp( \gamma_{32} \phi_2 ) \\
\end{aligned}
\end{equation}

We can discard the scalar terms:
\begin{equation}\label{eqn:nvolume:820}
\begin{aligned}
\gamma_0 C_0 \gamma_0 S_0 +S_0 C_1 \gamma_1 C_0 C_1 \gamma_1 +S_0 S_1 \gamma_2 \exp( \gamma_{32} \phi_2 ) C_0 S_1 \exp( -\gamma_{32} \phi_2 ) \gamma_2
\end{aligned}
\end{equation}

Two of these terms cancel out
\begin{equation}\label{eqn:nvolume:840}
\begin{aligned}
S_0 C_0 C_1 S_1 \gamma_{12} \exp( \gamma_{32} \phi_2 )
+S_0 C_0 C_1 S_1 \gamma_{21} \exp( \gamma_{32} \phi_2 )
\end{aligned}
\end{equation}

and we are left with two bivector contributors to the eventual four-pseudoscalar
\begin{equation}\label{eqn:nvolume:860}
\begin{aligned}
\gamma_{01} C_1
+\gamma_{02} S_1 \exp( \gamma_{32} \phi_2 )
\end{aligned}
\end{equation}

Multiplying out the last two derivatives we have
\begin{equation}\label{eqn:nvolume:880}
\begin{aligned}
-S_0^2 S_1^2 \gamma_{13} \exp( \gamma_{32} \phi_2 )
%+ S_0^2 C_1 S_1 \gamma_2 \exp( \gamma_{32} \phi_2 ) \exp( -\gamma_{32} \phi_2 ) \gamma_3
+ S_0^2 C_1 S_1 \gamma_{23}
\end{aligned}
\end{equation}

Almost there.  A final multiplication of these sets of products gives

\begin{equation}\label{eqn:nvolume:900}
\begin{aligned}
&-\gamma_{01} C_1 S_0^2 S_1^2 \gamma_{13} \exp( \gamma_{32} \phi_2 )
-\gamma_{02} S_1 \exp( \gamma_{32} \phi_2 ) S_0^2 S_1^2 \gamma_{13} \exp( \gamma_{32} \phi_2 ) \\
&+\gamma_{01} C_1 S_0^2 C_1 S_1 \gamma_{23}
+\gamma_{02} S_1 \exp( \gamma_{32} \phi_2 ) S_0^2 C_1 S_1 \gamma_{23} \\
&=
\gamma_{03} C_1 S_0^2 S_1^2 \exp( \gamma_{32} \phi_2 )
-\gamma_{02} \exp( \gamma_{32} \phi_2 ) S_0^2 S_1^3 \exp( -\gamma_{32} \phi_2 ) \gamma_{13} \\
&\qquad +\gamma_{0123} C_1^2 S_0^2 S_1
-\gamma_{03} \exp( \gamma_{32} \phi_2 ) S_0^2 C_1 S_1^2 \\
&=
\gamma_{03} C_1 S_0^2 S_1^2 \exp( \gamma_{32} \phi_2 )
+\gamma_{0123} S_0^2 S_1 (S_1^2 + C_1^2)
-\gamma_{03} \exp( \gamma_{32} \phi_2 ) S_0^2 C_1 S_1^2 \\
&=
\gamma_{03} C_1 S_1^2 (S_0^2 -S_0^2)\exp( \gamma_{32} \phi_2 )
+\gamma_{0123} S_0^2 S_1  \\
\end{aligned}
\end{equation}

Therefore our final result is

\begin{equation}\label{eqn:nvolume:920}
\begin{aligned}
dV_4 &= \sinh^2 \phi_0 \sin\phi_1 r^3 dr d\phi_0 d\phi_1 d\phi_2
\end{aligned}
\end{equation}

\section{Summary}

\subsection{Vector parametrization}

N-Spherical parametrization

\begin{equation}\label{eqn:nvolume:940}
\begin{aligned}
\Br_2 &= \sigma_1 \exp( \sigma_{12} \phi_1 ) \\
\Br_3 &= \cos\phi_1 \sigma_1 + \sin\phi_1 \sigma_2 \exp( \sigma_{23} \phi_2 ) \\
\Br_4 &= \sigma_1 \cos\phi_1 + \sigma_2 \sin\phi_1 (\cos\phi_2 + \sigma_{23} \sin\phi_2 \exp(\sigma_{34}\phi_3) )
\end{aligned}
\end{equation}

N-Hypersphere parametrization

\begin{equation}\label{eqn:nvolume:960}
\begin{aligned}
\Br_2 &= r (\gamma_0 \cosh \phi_0 + \sinh\phi_0 \gamma_1) \\
\Br_3 &= r (\gamma_0 \cosh \phi_0 + \sinh\phi_0 \gamma_1 \exp( \gamma_{21} \phi_1 ) ) \\
\Br_4 &= r (\gamma_0 \cosh \phi_0 + \sinh\phi_0 ( \cos\phi_1 \gamma_1 + \sin\phi_1 \gamma_2 \exp( \gamma_{32} \phi_2 )) )
\end{aligned}
\end{equation}

\subsection{Volume elements}

To summarize the mess of algebra we have shown that our hyperbolic volume elements are given by

\begin{equation}\label{eqn:nvolume:980}
\begin{aligned}
dV_2 &= \left(r dr\right) d\phi_0 \\
dV_3 &= \left(r^2 dr\right) \left(\sinh\phi_0 d\phi_0\right) d\phi_1 \\
dV_4 &= \left(r^3 dr\right) \left(\sinh^2 \phi_0 d\phi_0\right) \left(\sin\phi_1 d\phi_1\right) d\phi_2
\end{aligned}
\end{equation}

Compare this to the volume elements for the n-spheres

\begin{equation}\label{eqn:nvolume:1000}
\begin{aligned}
dV_2 &= \left(r dr\right) d\phi_1 \\
dV_3 &= \left(r^2 dr\right) \left(\sin\phi_1 d\phi_1\right) d\phi_2 \\
dV_4 &= \left(r^3 dr\right) \left(\sin^2\phi_1 d\phi_1\right) \left(\sin\phi_2 d\phi_2\right) d\phi_3
\end{aligned}
\end{equation}

Besides labeling variations the only difference in the form is a switch from trig to hyperbolic functions for the first angle (which has an implied
range difference as well).

   \chapter{Vector Differential Identities}
      %
% Copyright � 2012 Peeter Joot.  All Rights Reserved.
% Licenced as described in the file LICENSE under the root directory of this GIT repository.
%

%
%
%\chapter{Vector Differential Identities}
\index{vector identities}
\label{chap:vectorDifferentialIdentities}
%\date{Jan 05, 2009.  vectorDifferentialIdentities.tex}

\citep{feynman1963flp} electrodynamics \textchapref{II} lists a number of
differential vector identities.

\begin{enumerate}
\item \(\spacegrad \cdot (\spacegrad T) = \spacegrad^2 T = \mbox{a scalar field}\)
\item \(\spacegrad \cross (\spacegrad T) = 0\)
\item \(\spacegrad (\spacegrad \cdot \Bh) = \mbox{a vector field}\)
\item \(\spacegrad \cdot (\spacegrad \cross \Bh) = 0\)
\item \(\spacegrad \cross (\spacegrad \cross \Bh) = \spacegrad(\spacegrad \cdot \Bh) - \spacegrad^2 \Bh\)
\item \((\spacegrad \cdot \spacegrad) \Bh = \mbox{a vector field}\)
\end{enumerate}

Let us see how all these translate to GA form.

\section{Divergence of a gradient}
\index{gradient!divergence}

This one has the same form, but expanding it can be evaluated by grade
selection
%
\begin{equation}\label{eqn:vectorDifferentialIdentities:20}
\begin{aligned}
\spacegrad \cdot (\spacegrad T)
&= \gpgradezero{\spacegrad \spacegrad T} \\
&= (\spacegrad^2) T
\end{aligned}
\end{equation}
%
A less sneaky expansion would be by coordinates
%
\begin{equation}\label{eqn:vectorDifferentialIdentities:40}
\begin{aligned}
\spacegrad \cdot (\spacegrad T)
&= {\sum_{k,j} (\sigma_k \partial_k) \cdot (\sigma_j \partial_j T)} \\
&= \gpgradezero{\sum_{k,j} (\sigma_k \partial_k) (\sigma_j \partial_j T)} \\
&= \gpgradezero{\left(\sum_{k,j} \sigma_k \partial_k \sigma_j \partial_j\right) T} \\
%&= {\left(\sum_{k,j} \sigma_k \partial_k \sigma_j \partial_j\right) } T \\
&= \gpgradezero{\spacegrad^2 T} \\
&= \spacegrad^2 T \\
\end{aligned}
\end{equation}
%
\section{Curl of a gradient is zero}
\index{gradient!curl}

The duality analogue of this is
\begin{equation}\label{eqn:vectorDifferentialIdentities:60}
\begin{aligned}
\spacegrad \cross (\spacegrad T) = -i(\spacegrad \wedge (\spacegrad T))
\end{aligned}
\end{equation}
%
Let us verify that this bivector curl is zero.  This can also be done by grade selection
%
\begin{equation}\label{eqn:vectorDifferentialIdentities:80}
\begin{aligned}
\spacegrad \wedge (\spacegrad T)
&= \gpgradetwo{\spacegrad (\spacegrad T)} \\
&= \gpgradetwo{(\spacegrad \spacegrad) T} \\
&= (\spacegrad \wedge \spacegrad) T \\
&= 0
\end{aligned}
\end{equation}
%
Again, this is sneaky and side steps the continuity requirement for mixed partial equality.  Again by coordinates is better
\begin{equation}\label{eqn:vectorDifferentialIdentities:100}
\begin{aligned}
\spacegrad \wedge (\spacegrad T)
&= \gpgradetwo{\sum_{k,j}\sigma_k \partial_k (\sigma_j \partial_j T)} \\
&= \gpgradetwo{\sum_{k<j}\sigma_k \sigma_j (\partial_k \partial_j - \partial_j \partial_k) T} \\
&= \sum_{k<j} \sigma_k \wedge \sigma_j (\partial_k \partial_j - \partial_j \partial_k) T \\
\end{aligned}
\end{equation}
%
So provided the mixed partials are zero the curl of a gradient is zero.

\section{Gradient of a divergence}
\index{divergence!gradient}

Nothing more to say about this one.

\section{Divergence of curl}
\index{curl!divergence}

This one looks like it will have a dual form using bivectors.
%
\begin{equation}\label{eqn:vectorDifferentialIdentities:120}
\begin{aligned}
\spacegrad \cdot (\spacegrad \cross \Bh)
&= \spacegrad \cdot (-i (\spacegrad \wedge \Bh)) \\
&= \gpgradezero{\spacegrad (-i (\spacegrad \wedge \Bh))} \\
&= \gpgradezero{-i \spacegrad (\spacegrad \wedge \Bh)} \\
&= -(i \spacegrad) \cdot (\spacegrad \wedge \Bh) \\
\end{aligned}
\end{equation}
%
Is this any better than the cross product relationship?

I do not really think so.  They both say the same thing, and only possible value to this duality form is if more than three dimensions are required (in which case the sign of the pseudoscalar \(i\) has to be dealt with more carefully).  Geometrically one has the dual of the gradient (a plane normal to the vector itself) dotted with the plane formed by the gradient and the vector operated on.  The corresponding statement for the cross product form is that we have a dot product of a vector with a vector normal to it, so also intuitively expect a zero.  In either case, because we are talking about operators here
just saying this is zero because of geometrical arguments is not necessarily convincing.  Let us evaluate this explicitly in
coordinates to verify
%
\begin{equation}\label{eqn:vectorDifferentialIdentities:140}
\begin{aligned}
(i \spacegrad) \cdot (\spacegrad \wedge \Bh)
&= \gpgradezero{i \spacegrad (\spacegrad \wedge \Bh) } \\
&= \gpgradezero{i \sum_{k,j,l} \sigma_k \partial_k \left((\sigma_j \wedge \sigma_l) \partial_j h^l\right) } \\
&= -i \sum_{l} \sigma_l \wedge \left( \sum_{k<j} (\sigma_k \wedge \sigma_j) (\partial_k \partial_j -\partial_j \partial_k) h^l \right) \\
\end{aligned}
\end{equation}
%
This inner quantity is zero, again by equality of mixed partials.  While the dual form of this identity was not really any better than the cross
product form, there is nothing in this zero equality proof that was tied to the dimension of the vectors involved, so we do have a more general form
than can be expressed by the cross product, which could be of value in Minkowski space later.

\section{Curl of a curl}
\index{curl!curl}

This will also have a dual form.  That is
%
\begin{equation}\label{eqn:vectorDifferentialIdentities:160}
\begin{aligned}
\spacegrad \cross (\spacegrad \cross \Bh)
&= -i (\spacegrad \wedge (\spacegrad \cross \Bh)) \\
&= -i (\spacegrad \wedge (-i (\spacegrad \wedge \Bh))) \\
&= -i \gpgradetwo{\spacegrad (-i (\spacegrad \wedge \Bh))} \\
&= i \gpgradetwo{i \spacegrad (\spacegrad \wedge \Bh)} \\
&= i^2 \spacegrad \cdot (\spacegrad \wedge \Bh) \\
&= - \spacegrad \cdot (\spacegrad \wedge \Bh) \\
\end{aligned}
\end{equation}
%
Now, let us expand this quantity
\begin{equation}\label{eqn:vectorDifferentialIdentities:180}
\begin{aligned}
\spacegrad \cdot (\spacegrad \wedge \Bh)
\end{aligned}
\end{equation}
%
If the gradient could be treated as a plain old vector we could just do
\begin{equation}\label{eqn:vectorDifferentialIdentities:200}
\begin{aligned}
\Ba \cdot (\Ba \wedge \Bh) &= \Ba^2 \Bh - \Ba(\Ba \cdot \Bh) \\
\end{aligned}
\end{equation}
%
With the gradient substituted this is exactly the desired identity (with the expected sign difference)
%
\begin{equation}\label{eqn:vectorDifferentialIdentities:220}
\begin{aligned}
\spacegrad \cdot (\spacegrad \wedge \Bh) &= \spacegrad^2 \Bh - \spacegrad(\spacegrad \cdot \Bh) \\
\end{aligned}
\end{equation}
%
A coordinate expansion to truly verify that this is valid is logically still required, but having done the others above, it is clear how this
would work out.

\section{Laplacian of a vector}
\index{vector!Laplacian}

This one is not interesting seeming.

\section{Zero curl implies gradient solution}

This theorems is mentioned in the text without proof.

Theorem was
%
\begin{equation*}
\begin{array}{l l l}
\text{If} &          \spacegrad \cross \BA &= 0 \\
\text{there is a } &                  \psi &    \\
\text{such that  } & \BA &= \spacegrad \psi \\
\end{array}
\end{equation*}
%
This appears to be half of an if and only if theorem.  The unstated part is if one has a gradient then the curl is zero
%
\begin{equation}\label{eqn:vectorDifferentialIdentities:240}
\begin{aligned}
\BA = \spacegrad \psi \\
\implies \\
\spacegrad \cross \BA &= \spacegrad \cross \spacegrad \psi = 0
\end{aligned}
\end{equation}
%
This last was proven above, and follows from the assumed mixed partial equality.  Now, the real problem here is to find \(\psi\) given \(\BA\).
First note that we can remove the three dimensionality of the theorem by duality writing \(\spacegrad \cross \BA = -i (\spacegrad \wedge \BA)\).
In one sense changing the theorem to use the wedge instead of cross makes the problem harder since the wedge product is defined not just
for \R{3}.  However, this also allows starting with the simpler \R{2} case, so let us do that one first.

Write
%
\begin{equation}\label{eqn:vecDiffIdent:aDefined}
\begin{aligned}
\BA = \sigma^1 A_1 + \sigma^2 A_2 = \sigma^1 (\partial_1 \psi) + \sigma^2 (\partial_2 \psi)
\end{aligned}
\end{equation}
%
The gradient is
\begin{equation}\label{eqn:vectorDifferentialIdentities:260}
\begin{aligned}
\spacegrad = \sigma^1 \partial_1 + \sigma^2 \partial_2
\end{aligned}
\end{equation}
%
Our curl is then
%
\begin{equation}\label{eqn:vectorDifferentialIdentities:280}
\begin{aligned}
(\sigma^1 \partial_1 + \sigma^2 \partial_2) \wedge (\sigma^1 A_1 + \sigma^2 A_2)
&=
(\sigma^1 \wedge \sigma^2) (\partial_1 A_2 - \partial_2 A_1)
\end{aligned}
\end{equation}
%
So we have
\begin{equation}\label{eqn:vecDiffIdent:curlZero}
\begin{aligned}
\partial_1 A_2 = \partial_2 A_1
\end{aligned}
\end{equation}
%
Now from \eqnref{eqn:vecDiffIdent:aDefined} this means we must have
%
\begin{equation}\label{eqn:vectorDifferentialIdentities:300}
\begin{aligned}
\partial_1 \partial_2 \psi = \partial_2 \partial_1 \psi
\end{aligned}
\end{equation}
%
This is just a statement of mixed partial equality, and does not look particularly useful for solving for \(\psi\).  It really shows that the
is redundancy in the problem, and instead of substituting for both of \(A_1\) and \(A_2\)
in \eqnref{eqn:vecDiffIdent:curlZero}, we can use one or the other.

Doing so we have two equations, either of which we can solve for
\begin{equation}\label{eqn:vectorDifferentialIdentities:320}
\begin{aligned}
\partial_2 \partial_1 \psi &= \partial_1 A_2 \\
\partial_1 \partial_2 \psi &= \partial_2 A_1
\end{aligned}
\end{equation}
%
Integrating once gives
\begin{equation}\label{eqn:vectorDifferentialIdentities:340}
\begin{aligned}
\partial_1 \psi &= \int \partial_1 A_2 dy + B(x) \\
\partial_2 \psi &= \int \partial_2 A_1 dx + C(y)
\end{aligned}
\end{equation}
%
And a second time produces solutions for \(\psi\) in terms of the vector coordinates
\begin{equation}\label{eqn:vecDiffIdent:twoIntegrations}
\begin{aligned}
\psi &= \iint \PD{x}{A_2} dy dx + \int B(x) dx + D(y) \\
\psi &= \iint \PD{y}{A_1} dx dy + \int C(y) dy + E(x)
\end{aligned}
\end{equation}
%
Is there a natural way to merge these so that \(\psi\) can be expressed more symmetrically in terms of both coordinates?
Looking at \eqnref{eqn:vecDiffIdent:twoIntegrations} I am led to guess that its possible to
combine these into a single equation expressing \(\psi\) in terms of both \(A_1\) and \(A_2\).  One way to do so is perhaps just to average the
two as in
%
\begin{equation}\label{eqn:vectorDifferentialIdentities:360}
\begin{aligned}
\psi &= \alpha \iint \PD{x}{A_2} dy dx + (1-\alpha) \iint \PD{y}{A_1} dx dy + \int C(y) dy + E(x) + \int B(x) dx + D(y)
\end{aligned}
\end{equation}
%
But that seems pretty arbitrary.  Perhaps that is the point?

FIXME: work some examples.

%above it looks like these will combine naturally via Green's theorem which introduces a difference in
%sign intrinsically related to the curl.  That is been explored in \chapcite{PJStokes1}, and \chapcite{PJStokes2}, and intuitively one expects that
%a complexification of the area element will produce this result.

%Lets get rid of
%the integration constant functions to start with, pulling them into the integrals.  Starting over this way, the first integration gives us

%\begin{align*}
%\PD{x}{ \psi(x,y)} &= \int_{v = b_1(x)}^{b_2(x)} \PD{x}{A_2(x,v)} dv \\
%\PD{y}{ \psi(x,y)} &= \int_{u = c_1(y)}^{c_2(y)} \PD{y}{A_1(u,y)} du
%\end{align*}
%
%FIXME: second integration gives?
%\begin{align*}
%\psi(x,y) &= \int_{u=d_1(x,y)}^{d_2(x,y)} \left(\int_{v = b_1(u)}^{b_2(u)} \PD{u}{A_2(u,v)} dv \right) du \\
%\psi(x,y) &= \int_{v=e_1(x,y)}^{e_2(x,y)} \left(\int_{u = c_1(v)}^{c_2(v)} \PD{v}{A_1(u,v)} du \right) dv
%\end{align*}
%
%... this does not seem right.  Think it through better.
%

FIXME: look at more than the \R{2} case.

\section{Zero divergence implies curl solution}

This theorems is mentioned in the text without proof.

Theorem was
%
\begin{equation*}
\begin{array}{l l l}
\text{If} &                 \spacegrad \cdot \BD &= 0 \\
\text{there is a } &                         \BC &    \\
\text{such that  } & \BD &= \spacegrad \cross \BC \\
\end{array}
\end{equation*}
%
As above, if \(\BD = \spacegrad \cross \BC\), then we have
%
\begin{equation}\label{eqn:vectorDifferentialIdentities:380}
\begin{aligned}
\spacegrad \cdot \BD &= \spacegrad \cdot (\spacegrad \cross \BC)
\end{aligned}
\end{equation}
%
and this has already been shown to be zero.  So the problem becomes find \(\BC\) given \(\BD\).

Also, as before an equivalent generalized (or de-generalized) problem can be expressed.

That is
\begin{equation}\label{eqn:vectorDifferentialIdentities:400}
\begin{aligned}
\spacegrad \cdot \BD
&= \gpgradezero{\spacegrad \BD} \\
&= \gpgradezero{\spacegrad (\spacegrad \cross \BC)} \\
&= \gpgradezero{\spacegrad -i (\spacegrad \wedge \BC)} \\
&= -\gpgradezero{i\spacegrad \cdot (\spacegrad \wedge \BC)} -\gpgradezero{i\spacegrad \wedge (\spacegrad \wedge \BC)} \\
&= -\gpgradezero{i\spacegrad \cdot (\spacegrad \wedge \BC)} \\
\end{aligned}
\end{equation}
%
So if \(\spacegrad \cdot \BD\) it is also true that \(\spacegrad \cdot (\spacegrad \wedge \BC) = 0\)

Thus the (de)generalized theorem to prove is
%
\begin{equation*}
\begin{array}{l l l}
\text{If} &                 \spacegrad \cdot D &= 0 \\
\text{there is a } &                       C &    \\
\text{such that  } & D &= \spacegrad \wedge C \\
\end{array}
\end{equation*}
%
In the \R{3} case, to prove the original theorem we want a bivector \(D = -i\BD\), and seek a vector \(C\) such that
\(D = \spacegrad \wedge C\) (\(\BD = -i (\spacegrad \wedge C)\)).
%
\begin{equation}\label{eqn:vectorDifferentialIdentities:420}
\begin{aligned}
\spacegrad \cdot D
&= \spacegrad \cdot (\spacegrad \wedge C) \\
&= (\sigma^k \partial_k) \cdot (\sigma^i \wedge \sigma^j \partial_i C_j) \\
&= \sigma^k \cdot (\sigma^i \wedge \sigma^j) \partial_k \partial_i C_j \\
&= ( \sigma^j \delta^{ki} - \sigma^i \delta^{kj} ) \partial_k \partial_i C_j \\
&= \sigma^j \partial_i \partial_i C_j - \sigma^i \partial_j \partial_i C_j \\
&= \sigma^j \partial_i (\partial_i C_j -\partial_j C_i) \\
\end{aligned}
\end{equation}
%
If this is to equal zero we must have the following constraint on C
\begin{equation}\label{eqn:vecDiffIdent:blah}
\begin{aligned}
\partial_{ii} C_j = \partial_{ij} C_i
\end{aligned}
\end{equation}
%
If the following equality was also true
\begin{equation}\label{eqn:vectorDifferentialIdentities:440}
\begin{aligned}
\partial_{i} C_j = \partial_j C_i
\end{aligned}
\end{equation}
%
Then this would also work, but would also mean \(D\) equals zero so that is not an interesting solution.  So, we must go back to
\eqnref{eqn:vecDiffIdent:blah} and solve for \(C_k\) in terms of \(D\).

Suppose we have D explicitly in terms of coordinates
%
\begin{equation}\label{eqn:vectorDifferentialIdentities:460}
\begin{aligned}
D
&= D_{ij} \sigma^i \wedge \sigma^j \\
&= \sum_{i<j} (D_{ij} -D_{ji})\sigma^i \wedge \sigma^j
\end{aligned}
\end{equation}
%
compare this to \(\spacegrad \wedge C\)
%
\begin{equation}\label{eqn:vectorDifferentialIdentities:480}
\begin{aligned}
C &= (\partial_i C_j ) \sigma^i \wedge \sigma^j \\
  &= \sum_{i<j} (\partial_i C_j -\partial_j C_i) \sigma^i \wedge \sigma^j
\end{aligned}
\end{equation}
%
With the identity
\begin{equation}\label{eqn:vectorDifferentialIdentities:500}
\begin{aligned}
\partial_i C_j = D_ij
\end{aligned}
\end{equation}
%
\Eqnref{eqn:vecDiffIdent:blah} becomes
\begin{equation}\label{eqn:vectorDifferentialIdentities:520}
\begin{aligned}
\partial_{ij} C_i &= \partial_{i} D_{ij}  \\
\implies \\
\partial_{j} C_i &= D_{ij} + \alpha_{ij}(x^{k \ne i})
\end{aligned}
\end{equation}
%
Where \(\alpha_{ij}(x^{k \ne i})\) is some function of all the \(x^k \ne x^i\).

Integrating once more we have
\begin{equation}\label{eqn:vectorDifferentialIdentities:540}
\begin{aligned}
C_i &= \int \left(D_{ij} + \alpha_{ij}(x^{k \ne i}) \right) dx^j + \beta_{ij}(x^{k \ne j})
\end{aligned}
\end{equation}

   \chapter{Polar form for the gradient and Laplacian}
   %\chapter{Was Professor Dmitrevsky right about insisting the Laplacian is not generally \texorpdfstring{\(\spacegrad \cdot \spacegrad\)}{gradient dotted with gradient}}
      %
% Copyright � 2012 Peeter Joot.  All Rights Reserved.
% Licenced as described in the file LICENSE under the root directory of this GIT repository.
%

%
%
%\input{../peeter_prologue.tex}

%\chapter{Polar form for the gradient and Laplacian}
\index{gradient!polar form}
\index{Laplacian!polar form}
%\chapter{Was Professor Dmitrevsky right about insisting the Laplacian is not generally \texorpdfstring{\(\spacegrad \cdot \spacegrad\)}{gradient dotted with gradient}}
\label{chap:polarGradAndLaplacian}

%\blogpage{http://sites.google.com/site/peeterjoot/math2009/polarGradAndLaplacian.pdf?revision=1}
%\date{Dec 1, 2009}
%\revisionInfo{\(RCSfile: polarGradAndLaplacian.tex,v \) Last \(Revision: 1.5 \) \(Date: 2009/12/03 02:07:17 \)}

%\beginArtWithToc
%\beginArtNoToc

\section{Dedication}

To all tyrannical old Professors driven to cruelty by an unending barrage of increasingly ill prepared students.

\section{Motivation}

The text \citep{byron1992mca} has an excellent general derivation of a number of forms of the gradient, divergence, curl and Laplacian.

This is actually done, not starting with the usual Cartesian forms, but more general definitions.

\begin{equation}\label{eqn:divGradCurLap:1}
\begin{aligned}
(\text{grad}\  \phi)_i &= \lim_{ds_i \rightarrow 0} \frac{\phi(q_i + dq_i) - \phi(q_i)}{ds_i} \\
\text{div}\  \BV &= \lim_{\Delta \tau \rightarrow 0} \inv{\Delta \tau} \int_\sigma \BV \cdot d\Bsigma \\
(\text{curl}\  \BV) \cdot \Bn &= \lim_{\Delta \sigma \rightarrow 0} \inv{\Delta \sigma} \oint_\lambda \BV \cdot d\Blambda \\
\text{Laplacian}\  \phi &= \text{div} (\text{grad}\ \phi).
\end{aligned}
\end{equation}

These are then shown to imply the usual Cartesian definitions, plus provide the means to calculate the general relationships in whatever coordinate system you like.  All in all one can not beat this approach, and I am not going to try to replicate it, because I can not improve it in any way by doing so.

Given that, what do I have to say on this topic?  Well, way way back in first year electricity and magnetism, my dictator of a prof, the intimidating but diminutive Dmitrevsky, yelled at us repeatedly that one cannot just dot the gradient to form the Laplacian.  As far as he was concerned one can only say

\begin{equation}\label{eqn:divGradCurLap:2}
\begin{aligned}
\text{Laplacian}\  \phi &= \text{div} (\text{grad}\ \phi),
\end{aligned}
\end{equation}

and never never never, the busted way

\begin{equation}\label{eqn:divGradCurLap:3}
\begin{aligned}
\text{Laplacian}\  \phi &= (\spacegrad \cdot \spacegrad) \phi.
\end{aligned}
\end{equation}

Because ``this only works in Cartesian coordinates''.  He probably backed up this assertion with a heartwarming and encouraging statement like ``back in the days when University of Toronto was a real school you would have learned this in kindergarten''.

This detail is actually something that has bugged me ever since, because my assumption was that, provided one was careful, why would a change to an alternate coordinate system matter?  The gradient is still the gradient, so it seems to me that this ought to be a general way to calculate things.

Here we explore the validity of the dictatorial comments of Prof Dmitrevsky.  The key to reconciling intuition and his statement turns out to lie with the fact that one has to let the gradient operate on the unit vectors in the non Cartesian representation as well as the partials, something that was not clear as a first year student.  Provided that this is done, the plain old dot product procedure yields the expected results.

This exploration will utilize a two dimensional space as a starting point, transforming from Cartesian to polar form representation.  I will also utilize a geometric algebra representation of the polar unit vectors.

\section{The gradient in polar form}

Lets start off with a calculation of the gradient in polar form starting with the Cartesian form.  Writing \(\partial_x = \PDi{x}{}\), \(\partial_y = \PDi{y}{}\), \(\partial_r = \PDi{r}{}\), and \(\partial_\theta = \PDi{\theta}{}\), we want to map

\begin{equation}\label{eqn:divGradCurLap:4}
\begin{aligned}
\spacegrad
= \Be_1 \partial_1 + \Be_2 \partial_2
=
\begin{bmatrix}
\Be_1 & \Be_2
\end{bmatrix}
\begin{bmatrix}
\partial_1 \\
\partial_2
\end{bmatrix},
\end{aligned}
\end{equation}

into the same form using \(\rcap, \thetacap, \partial_r\), and \(\partial_\theta\).  With \(i = \Be_1 \Be_2\) we have

\begin{equation}\label{eqn:divGradCurLap:5}
\begin{aligned}
\begin{bmatrix}
\Be_1 \\
\Be_2
\end{bmatrix}
=
e^{i\theta}
\begin{bmatrix}
\rcap \\
\thetacap
\end{bmatrix}.
\end{aligned}
\end{equation}

Next we need to do a chain rule expansion of the partial operators to change variables.  In matrix form that is

\begin{equation}\label{eqn:divGradCurLap:6}
\begin{aligned}
\begin{bmatrix}
\PD{x}{} \\
\PD{y}{}
\end{bmatrix}
=
\begin{bmatrix}
\PD{x}{r} &          \PD{x}{\theta} \\
\PD{y}{r} &          \PD{y}{\theta}
\end{bmatrix}
\begin{bmatrix}
\PD{r}{} \\
\PD{\theta}{}
\end{bmatrix}.
\end{aligned}
\end{equation}

To calculate these partials we drop back to coordinates

\begin{equation}\label{eqn:divGradCurLap:7}
\begin{aligned}
x^2 + y^2 &= r^2 \\
\frac{y}{x} &= \tan\theta \\
\frac{x}{y} &= \cot\theta.
\end{aligned}
\end{equation}

From this we calculate

\begin{equation}\label{eqn:divGradCurLap:8}
\begin{aligned}
\PD{x}{r} &= \cos\theta \\
\PD{y}{r} &= \sin\theta \\
\inv{r\cos\theta} &= \PD{y}{\theta} \inv{\cos^2\theta} \\
\inv{r\sin\theta} &= -\PD{x}{\theta} \inv{\sin^2\theta},
\end{aligned}
\end{equation}

for

\begin{equation}\label{eqn:divGradCurLap:6e}
\begin{aligned}
\begin{bmatrix}
\PD{x}{} \\
\PD{y}{}
\end{bmatrix}
=
\begin{bmatrix}
\cos\theta & -\sin\theta/r \\
\sin\theta & \cos\theta/r
\end{bmatrix}
\begin{bmatrix}
\PD{r}{} \\
\PD{\theta}{}
\end{bmatrix}.
\end{aligned}
\end{equation}

We can now write down the gradient in polar form, prior to final simplification

\begin{equation}\label{eqn:divGradCurLap:4a}
\begin{aligned}
\spacegrad
=
e^{i\theta}
\begin{bmatrix}
\rcap & \thetacap
\end{bmatrix}
\begin{bmatrix}
\cos\theta & -\sin\theta/r \\
\sin\theta & \cos\theta/r
\end{bmatrix}
\begin{bmatrix}
\PD{r}{} \\
\PD{\theta}{}
\end{bmatrix}.
\end{aligned}
\end{equation}

Observe that we can factor a unit vector

\begin{equation}\label{eqn:divGradCurLap:9}
\begin{aligned}
\begin{bmatrix}
\rcap & \thetacap
\end{bmatrix}
=
\rcap
\begin{bmatrix}
1 & i
\end{bmatrix}
=
\begin{bmatrix}
i & 1
\end{bmatrix}
\thetacap
\end{aligned}
\end{equation}

so the \(1,1\) element of the matrix product in the interior is

\begin{equation}\label{eqn:divGradCurLap:9a}
\begin{aligned}
\begin{bmatrix}
\rcap & \thetacap
\end{bmatrix}
\begin{bmatrix}
\cos\theta \\
\sin\theta
\end{bmatrix}
=
\rcap e^{i\theta} = e^{-i\theta}\rcap.
\end{aligned}
\end{equation}

Similarly, the \(1,2\) element of the matrix product in the interior is

\begin{equation}\label{eqn:divGradCurLap:9b}
\begin{aligned}
\begin{bmatrix}
\rcap & \thetacap
\end{bmatrix}
\begin{bmatrix}
-\sin\theta/r \\
\cos\theta/r
\end{bmatrix}
=
\inv{r} e^{-i\theta} \thetacap.
\end{aligned}
\end{equation}

The exponentials cancel nicely, leaving after a final multiplication with the polar form for the gradient

\begin{equation}\label{eqn:divGradCurLap:10}
\begin{aligned}
\spacegrad = \rcap \partial_r + \thetacap \inv{r} \partial_\theta
\end{aligned}
\end{equation}

That was a fun way to get the result, although we could have just looked it up.  We want to use this now to calculate the Laplacian.

\section{Polar form Laplacian for the plane}

We are now ready to look at the Laplacian.  First let us do it the first year electricity and magnetism course way.  We look up the formula for polar form divergence, the one we were supposed to have memorized in kindergarten, and find it to be

\begin{equation}\label{eqn:divGradCurLap:11}
\begin{aligned}
\text{div}\ \BA = \partial_r A_r + \inv{r} A_r + \inv{r} \partial_\theta A_\theta
\end{aligned}
\end{equation}

We can now apply this to the gradient vector in polar form which has components \(\spacegrad_r = \partial_r\), and \(\spacegrad_\theta = (1/r)\partial_\theta\), and get

\begin{equation}\label{eqn:divGradCurLap:12}
\begin{aligned}
\text{div}\ \text{grad} =
\partial_{rr} + \inv{r} \partial_r + \inv{r} \partial_{\theta\theta}
\end{aligned}
\end{equation}

This is the expected result, and what we should get by performing \(\spacegrad \cdot \spacegrad\) in polar form.  Now, let us do it the wrong way, dotting our gradient with itself.

\begin{equation}\label{eqn:polarGradAndLaplacian:34}
\begin{aligned}
\spacegrad \cdot \spacegrad
&= \left(\partial_r, \inv{r} \partial_\theta\right) \cdot \left(\partial_r, \inv{r} \partial_\theta\right) \\
&= \partial_{rr} + \inv{r} \partial_\theta \left(\inv{r} \partial_\theta\right) \\
&= \partial_{rr} + \inv{r^2} \partial_{\theta\theta}
\end{aligned}
\end{equation}

This is wrong!  So is Dmitrevsky right that this procedure is flawed, or do you spot the mistake?  I have also cruelly written this out in a way that obscures the error and highlights the source of the confusion.

The problem is that our unit vectors are functions, and they must also be included in the application of our partials.  Using the coordinate polar form without explicitly putting in the unit vectors is how we go wrong.  Here is the right way

\begin{equation}\label{eqn:polarGradAndLaplacian:54}
\begin{aligned}
\spacegrad \cdot \spacegrad
&=
\left( \rcap \partial_r + \thetacap \inv{r} \partial_\theta \right) \cdot \left( \rcap \partial_r + \thetacap \inv{r} \partial_\theta \right) \\
&=
\rcap \cdot \partial_r \left(\rcap \partial_r \right)
+\rcap \cdot \partial_r \left( \thetacap \inv{r} \partial_\theta \right)
+\thetacap \cdot \inv{r} \partial_\theta \left( \rcap \partial_r \right)
+\thetacap \cdot \inv{r} \partial_\theta \left( \thetacap \inv{r} \partial_\theta \right) \\
\end{aligned}
\end{equation}

Now we need the derivatives of our unit vectors.  The \(\partial_r\) derivatives are zero since these have no radial dependence, but we do have \(\theta\) partials

\begin{equation}\label{eqn:polarGradAndLaplacian:74}
\begin{aligned}
\partial_\theta \rcap
&=
\partial_\theta \left( \Be_1 e^{i\theta} \right) \\
&=
\Be_1 \Be_1 \Be_2 e^{i\theta} \\
&=
\Be_2 e^{i\theta} \\
&=
\thetacap,
\end{aligned}
\end{equation}

and

\begin{equation}\label{eqn:polarGradAndLaplacian:94}
\begin{aligned}
\partial_\theta \thetacap
&=
\partial_\theta \left( \Be_2 e^{i\theta} \right) \\
&=
\Be_2 \Be_1 \Be_2 e^{i\theta} \\
&=
-\Be_1 e^{i\theta} \\
&=
-\rcap.
\end{aligned}
\end{equation}

(One should be able to get the same results if these unit vectors were written out in full as \(\rcap = \Be_1 \cos\theta + \Be_2 \sin\theta\), and \(\thetacap = \Be_2 \cos\theta - \Be_1 \sin\theta\), instead of using the obscure geometric algebra quaterionic rotation exponential operators.)

Having calculated these partials we now have

\begin{equation}\label{eqn:divGradCurLap:14}
\begin{aligned}
(\spacegrad \cdot \spacegrad)
&=
\partial_{rr}
+\inv{r} \partial_r
+\inv{r^2} \partial_{\theta\theta}
\end{aligned}
\end{equation}

Exactly what it should be, and what we got with the coordinate form of the divergence operator when applying the ``Laplacian equals the divergence of the gradient'' rule blindly.  We see that the expectation that \(\spacegrad \cdot \spacegrad\) is the Laplacian in more than the Cartesian coordinate system is not invalid, but that care is required to apply the chain rule to all functions.  We also see that expressing a vector in coordinate form when the basis vectors are position dependent is also a path to danger.

Is this anything that our electricity and magnetism prof did not know?  Unlikely.  Is this something that our prof felt that could not be explained to a mob of first year students?  Probably.

%\EndArticle
%%\EndNoBibArticle

   \chapter{Derivation of the spherical polar Laplacian}
      %
% Copyright � 2012 Peeter Joot.  All Rights Reserved.
% Licenced as described in the file LICENSE under the root directory of this GIT repository.
%

%
%
%\input{../peeter_prologue_print.tex}
%\input{../peeter_prologue_widescreen.tex}

%\chapter{Derivation of the spherical polar Laplacian}
\index{Laplacian!spherical polar}
\label{chap:sphericalPolarLaplacian}

%\blogpage{http://sites.google.com/site/peeterjoot/math2010/sphericalPolarLaplacian.pdf}
%\date{Oct 20, 2010}
%\revisionInfo{sphericalPolarLaplacian.tex}

%\beginArtWithToc
%\beginArtNoToc

\section{Motivation}

In \chapcite{polarGradAndLaplacian} was a Geometric Algebra derivation of the 2D polar Laplacian by squaring the gradient.  In \chapcite{sphericalPolarUnit} was a factorization of the spherical polar unit vectors in a tidy compact form.  Here both these ideas are utilized to derive the spherical polar form for the Laplacian, an operation that is strictly algebraic (squaring the gradient) provided we operate on the unit vectors correctly.

\section{Our rotation multivector}

Our starting point is a pair of rotations.  We rotate first in the \(x,y\) plane by \(\phi\)

\begin{subequations}
\label{eqn:sphericalPolarLaplacian:101}
\begin{equation}\label{eqn:sphericalPolarLaplacian:421}
\begin{aligned}
\Bx &\rightarrow \Bx' = \tilde{R_\phi} \Bx R_\phi \\
i &\equiv \Be_1 \Be_2 \\
R_\phi &= e^{i \phi/2}
\end{aligned}
\end{equation}
\end{subequations}

Then apply a rotation in the \(\Be_3 \wedge (\tilde{R_\phi} \Be_1 R_\phi) = \tilde{R_\phi} \Be_3 \Be_1 R_\phi\) plane

\begin{subequations}
\label{eqn:sphericalPolarLaplacian:102}
\begin{equation}\label{eqn:sphericalPolarLaplacian:441}
\begin{aligned}
\Bx' &\rightarrow \Bx'' = \tilde{R_\theta} \Bx' R_\theta \\
R_\theta &= e^{ \tilde{R_\phi} \Be_3 \Be_1 R_\phi \theta/2 } = \tilde{R_\phi} e^{ \Be_3 \Be_1 \theta/2 } R_\phi
\end{aligned}
\end{equation}
\end{subequations}

The composition of rotations now gives us

\begin{equation}\label{eqn:sphericalPolarLaplacian:461}
\begin{aligned}
\Bx
&\rightarrow \Bx'' = \tilde{R_\theta} \tilde{R_\phi} \Bx R_\phi R_\theta = \tilde{R} \Bx R \\
R &= R_\phi R_\theta = e^{ \Be_3 \Be_1 \theta/2 } e^{ \Be_1 \Be_2 \phi/2 }.
\end{aligned}
\end{equation}

\section{Expressions for the unit vectors}

The unit vectors in the rotated frame can now be calculated.  With \(I = \Be_1 \Be_2 \Be_3\) we can calculate

\begin{subequations}
\label{eqn:sphericalPolarLaplacian:201}
\begin{equation}\label{eqn:sphericalPolarLaplacian:481}
\begin{aligned}
\phicap &= \tilde{R} \Be_2 R  \\
\rcap &= \tilde{R} \Be_3 R  \\
\thetacap &= \tilde{R} \Be_1 R
\end{aligned}
\end{equation}
\end{subequations}

Performing these we get

\begin{equation}\label{eqn:sphericalPolarLaplacian:501}
\begin{aligned}
\phicap
&= e^{ -\Be_1 \Be_2 \phi/2 } e^{ -\Be_3 \Be_1 \theta/2 } \Be_2 e^{ \Be_3 \Be_1 \theta/2 } e^{ \Be_1 \Be_2 \phi/2 } \\
&= \Be_2 e^{ i \phi },
\end{aligned}
\end{equation}

and
\begin{equation}\label{eqn:sphericalPolarLaplacian:521}
\begin{aligned}
\rcap
&= e^{ -\Be_1 \Be_2 \phi/2 } e^{ -\Be_3 \Be_1 \theta/2 } \Be_3 e^{ \Be_3 \Be_1 \theta/2 } e^{ \Be_1 \Be_2 \phi/2 } \\
&= e^{ -\Be_1 \Be_2 \phi/2 } (\Be_3 \cos\theta + \Be_1 \sin\theta ) e^{ \Be_1 \Be_2 \phi/2 } \\
&= \Be_3 \cos\theta +\Be_1 \sin\theta e^{ \Be_1 \Be_2 \phi } \\
&= \Be_3 (\cos\theta + \Be_3 \Be_1 \sin\theta e^{ \Be_1 \Be_2 \phi } ) \\
&= \Be_3 e^{I \phicap \theta},
\end{aligned}
\end{equation}

and

\begin{equation}\label{eqn:sphericalPolarLaplacian:541}
\begin{aligned}
\thetacap
&= e^{ -\Be_1 \Be_2 \phi/2 } e^{ -\Be_3 \Be_1 \theta/2 } \Be_1 e^{ \Be_3 \Be_1 \theta/2 } e^{ \Be_1 \Be_2 \phi/2 } \\
&= e^{ -\Be_1 \Be_2 \phi/2 } ( \Be_1 \cos\theta - \Be_3 \sin\theta ) e^{ \Be_1 \Be_2 \phi/2 } \\
&= \Be_1 \cos\theta e^{ \Be_1 \Be_2 \phi/2 } - \Be_3 \sin\theta \\
&= i \phicap \cos\theta - \Be_3 \sin\theta \\
&= i \phicap (\cos\theta + \phicap i \Be_3 \sin\theta ) \\
&= i \phicap e^{I \phicap \theta}.
\end{aligned}
\end{equation}

Summarizing these are

\begin{subequations}
\label{eqn:sphericalPolarLaplacian:205}
\begin{equation}\label{eqn:sphericalPolarLaplacian:561}
\begin{aligned}
\phicap &= \Be_2 e^{ i \phi } \\
\rcap &= \Be_3 e^{I \phicap \theta} \\
\thetacap &= i \phicap e^{I \phicap \theta}.
\end{aligned}
\end{equation}
\end{subequations}

\section{Derivatives of the unit vectors}

We will need the partials.  Most of these can be computed from \eqnref{eqn:sphericalPolarLaplacian:205} by inspection, and are

\begin{subequations}
\label{eqn:sphericalPolarLaplacian:206}
\begin{equation}\label{eqn:sphericalPolarLaplacian:581}
\begin{aligned}
\partial_r \phicap &= 0 \\
\partial_r \rcap &= 0 \\
\partial_r \thetacap &= 0 \\
\partial_\theta \phicap &= 0 \\
\partial_\theta \rcap &= \rcap I \phicap \\
\partial_\theta \thetacap &= \thetacap I \phicap \\
\partial_\phi \phicap &= \phicap i \\
%\partial_\phi \rcap &= \Be_3 I \phicap i \sin\theta = \phicap \sin\theta \\
\partial_\phi \rcap &= \phicap \sin\theta \\
\partial_\phi \thetacap &= \phicap \cos\theta
\end{aligned}
\end{equation}
\end{subequations}

\section{Expanding the Laplacian}

We note that the line element is \(ds = dr + r d\theta + r\sin\theta d\phi\), so our gradient in spherical coordinates is

\begin{equation}\label{eqn:sphericalPolarLaplacian:300}
\begin{aligned}
\spacegrad &= \rcap \partial_r + \frac{\thetacap}{r} \partial_\theta + \frac{\phicap}{r\sin\theta} \partial_\phi.
\end{aligned}
\end{equation}

We can now evaluate the Laplacian

\begin{equation}\label{eqn:sphericalPolarLaplacian:310}
\begin{aligned}
\spacegrad^2 &=
\left( \rcap \partial_r + \frac{\thetacap}{r} \partial_\theta + \frac{\phicap}{r\sin\theta} \partial_\phi \right) \cdot
\left( \rcap \partial_r + \frac{\thetacap}{r} \partial_\theta + \frac{\phicap}{r\sin\theta} \partial_\phi \right).
\end{aligned}
\end{equation}

Evaluating these one set at a time we have
\begin{equation}\label{eqn:sphericalPolarLaplacian:601}
\begin{aligned}
\rcap \partial_r \cdot \left( \rcap \partial_r + \frac{\thetacap}{r} \partial_\theta + \frac{\phicap}{r\sin\theta} \partial_\phi \right) &= \partial_{rr},
\end{aligned}
\end{equation}

and
\begin{equation}\label{eqn:sphericalPolarLaplacian:621}
\begin{aligned}
\inv{r} \thetacap \partial_\theta \cdot \left( \rcap \partial_r + \frac{\thetacap}{r} \partial_\theta + \frac{\phicap}{r\sin\theta} \partial_\phi \right)
&=
\inv{r} \gpgradezero{
\thetacap \left(
\rcap I \phicap \partial_r + \rcap \partial_{\theta r}
+ \frac{\thetacap}{r} \partial_{\theta\theta} + \inv{r} \thetacap I \phicap \partial_\theta
+ \phicap \partial_\theta \inv{r\sin\theta} \partial_\phi
\right)
} \\
&=
\inv{r} \partial_r
+\inv{r^2} \partial_{\theta\theta},
\end{aligned}
\end{equation}

and

\begin{equation}\label{eqn:sphericalPolarLaplacian:641}
\begin{aligned}
\frac{\phicap}{r\sin\theta} \partial_\phi &\cdot
\left( \rcap \partial_r + \frac{\thetacap}{r} \partial_\theta + \frac{\phicap}{r\sin\theta} \partial_\phi \right) \\
&=
\frac{1}{r\sin\theta}
\gpgradezero{
\phicap
\left(
%\partial_\phi \rcap \partial_r
%+ \rcap \partial_{\phi r}
%+ \partial_\phi \thetacap \frac{1}{r} \partial_\theta
%+ \frac{\thetacap}{r} \partial_{\phi \theta }
%+ \partial_\phi \phicap \frac{1}{r\sin\theta} \partial_\phi
%+ \phicap \frac{1}{r\sin\theta} \partial_{\phi \phi }
\phicap \sin\theta \partial_r
+ \rcap \partial_{\phi r}
+ \phicap \cos\theta \frac{1}{r} \partial_\theta
+ \frac{\thetacap}{r} \partial_{\phi \theta }
+ \phicap i \frac{1}{r\sin\theta} \partial_\phi
+ \phicap \frac{1}{r\sin\theta} \partial_{\phi \phi }
\right)
} \\
&=
\inv{r} \partial_r
+ \frac{\cot\theta}{r^2}\partial_\theta
+ \inv{r^2 \sin^2\theta} \partial_{\phi\phi}
\end{aligned}
\end{equation}

Summing these we have

\begin{equation}\label{eqn:sphericalPolarLaplacian:400}
\begin{aligned}
\spacegrad^2 &=
\partial_{rr}
+ \frac{2}{r} \partial_r
+\inv{r^2} \partial_{\theta\theta}
+ \frac{\cot\theta}{r^2}\partial_\theta
+ \inv{r^2 \sin^2\theta} \partial_{\phi\phi}
\end{aligned}
\end{equation}

This is often written with a chain rule trick to consolidate the \(r\) and \(\theta\) partials

\begin{equation}\label{eqn:sphericalPolarLaplacian:401}
\begin{aligned}
\spacegrad^2 \Psi &=
%\inv{r}\frac{\partial^2 (r \Psi)}{\partial r^2}
%+ \inv{r^2 \sin\theta} \PD{\theta}{} \left( \sin\theta \PD{\theta}{\Psi} \right)
%+ \inv{r^2 \sin^2\theta} \frac{\partial^2 \Psi}{\partial \psi^2}
\inv{r} \partial_{rr} (r \Psi)
+ \inv{r^2 \sin\theta} \partial_\theta \left( \sin\theta \partial_\theta \Psi \right)
+ \inv{r^2 \sin^2\theta} \partial_{\psi\psi} \Psi
\end{aligned}
\end{equation}

It is simple to verify that this is identical to \eqnref{eqn:sphericalPolarLaplacian:400}.

%\EndArticle

   \chapter{Tangent planes and normals in three and four dimensions}
      %
% Copyright � 20123Peeter Joot.  All Rights Reserved.
% Licenced as described in the file LICENSE under the root directory of this GIT repository.
%
% pick one:
%\input{../assignment.tex}
%\input{../blogpost.tex}
%\renewcommand{\basename}{tangentAndNormalVectors}
%\renewcommand{\dirname}{notes/gabook/}
%%\newcommand{\dateintitle}{}
%\newcommand{\keywords}{tangent plane, surface normal, gradient, 3D, 4D, reciprocal basis, duality, pseudoscalar, geometric algebra, bivector, trivector, Minkowski space}
%
%\input{../peeter_prologue_print2.tex}
%
%\beginArtNoToc
%
%\generatetitle{Tangent planes and normals in three and four dimensions}
%\chapter{Tangent planes and normals in three and four dimensions}
\index{tangent plane}
\index{normal}
\label{chap:tangentAndNormalVectors}
\section{Motivation}

I was reviewing the method of Lagrange in my old first year calculus book \citep{salas1990coa} and found that I needed a review of some of the geometry ideas associated with the gradient (that it is normal to the surface).  The approach in the text used 3D level surfaces \(f(x, y, z) = c\), which is general but not the most intuitive.

If we define a surface in the simpler explicit form \(z = f(x, y)\), then how would you show this normal property?  Here we explore this in 3D and 4D, using geometric and wedge products to express the tangent planes and tangent volumes respectively.

In the 4D approach, with a vector \(x\) defined by coordinates \(x^\mu\) and basis \(\{\gamma_\mu\}\) so that

\begin{dmath}\label{eqn:tangentAndNormalVectors:20}
x = \gamma_\mu x^\mu,
\end{dmath}

the reciprocal basis \({\gamma^\mu}\) is defined implicitly by the dot product relations

\begin{dmath}\label{eqn:tangentAndNormalVectors:40}
\gamma^\mu \cdot \gamma_\nu = {\delta^\mu}_\nu.
\end{dmath}

Assuming such a basis makes the result general enough that the 4D (or a trivial generalization to N dimensions) holds for both Euclidean spaces as well as mixed metric (i.e. Minkowski) spaces, and avoids having to detail the specific metric in question.

\section{3D surface}

We start by considering \cref{fig:tangentAndNormalVectors:tangentAndNormalVectorsFig1}.  We wish to determine the bivector for the tangent plane in the neighborhood of the point \(\Bp\)

\imageFigure{../figures.gabook/tangentAndNormalVectorsFig1}{A portion of a surface in 3D}{fig:tangentAndNormalVectors:tangentAndNormalVectorsFig1}{0.3}

\begin{dmath}\label{eqn:tangentAndNormalVectors:60}
\Bp = ( x, y, f(x, y) ),
\end{dmath}

then using a duality transformation (multiplication by the pseudoscalar for the space) determine the normal vector to that plane at this point.  Holding either of the two free parameters constant, we find the tangent vectors on that surface to be

\begin{subequations}
\begin{dmath}\label{eqn:tangentAndNormalVectors:80}
\Bp_1
= \left( dx, 0, \PD{x}{f} dx \right)
\propto \left( 1, 0, \PD{x}{f} \right)
\end{dmath}
\begin{dmath}\label{eqn:tangentAndNormalVectors:100}
\Bp_2
= \left( 0, dy, \PD{y}{f} dy \right)
\propto \left( 0, 1, \PD{y}{f} \right)
\end{dmath}
\end{subequations}

The tangent plane is then

\begin{dmath}\label{eqn:tangentAndNormalVectors:120}
\Bp_1 \wedge \Bp_2 =
\left( 1, 0, \PD{x}{f} \right) \wedge
\left( 0, 1, \PD{y}{f} \right)
=
\left( \Be_1 + \Be_3 \PD{x}{f} \right)
\wedge
\left( \Be_2 + \Be_3 \PD{y}{f} \right)
=
\Be_1 \Be_2
+ \Be_1 \Be_3 \PD{y}{f}
+ \Be_3 \Be_2 \PD{x}{f}.
\end{dmath}

We can factor out the pseudoscalar 3D volume element \(I = \Be_1 \Be_2 \Be_3\), assuming a Euclidean space for which \(\Be_k^2 = 1\).  That is

\begin{dmath}\label{eqn:tangentAndNormalVectors:140}
\Bp_1 \wedge \Bp_2 =
\Be_1 \Be_2 \Be_3 \left(
\Be_3
- \Be_2 \PD{y}{f}
- \Be_1 \PD{x}{f}
\right)
\end{dmath}

Multiplying through by the volume element \(I\) we find that the normal to the surface at this point is

\begin{dmath}\label{eqn:tangentAndNormalVectors:160}
\Bn
\propto -I(\Bp_1 \wedge \Bp_2)
=
\Be_3
- \Be_1 \PD{x}{f}
- \Be_2 \PD{y}{f}.
\end{dmath}

Observe that we can write this as

\boxedEquation{eqn:tangentAndNormalVectors:180}{
\Bn = \spacegrad ( z - f(x, y) ).
}

Let's see how this works in 4D, so that we know how to handle the Minkowski spaces we find in special relativity.

\section{4D surface}

Now, let's move up to one additional direction, with

\begin{dmath}\label{eqn:tangentAndNormalVectors:200}
x^3 = f(x^0, x^1, x^2).
\end{dmath}

the differential of this is

\begin{equation}\label{eqn:tangentAndNormalVectors:220}
dx^3
= \sum_{k=0}^2 \PD{x^k}{f} dx^k
= \sum_{k=0}^2 \partial_k f dx^k .
\end{equation}

We are going to look at the 3-surface in the neighborhood of the point

\begin{equation}\label{eqn:tangentAndNormalVectors:240}
p =
\left(
x^0, x^1, x^2, x^3
\right),
\end{equation}

so that the tangent vectors in the neighborhood of this point are in the span of

\begin{equation}\label{eqn:tangentAndNormalVectors:260}
dp =
\left(
x^0, x^1, x^2, \sum_{k=0}^2 \partial_k dx^k
\right).
\end{equation}

In particular, in each of the directions we have

\begin{subequations}
\begin{equation}\label{eqn:tangentAndNormalVectors:280}
p_0 \propto ( 1, 0, 0, d_0 f)
\end{equation}
\begin{equation}\label{eqn:tangentAndNormalVectors:300}
p_1 \propto ( 0, 1, 0, d_1 f)
\end{equation}
\begin{equation}\label{eqn:tangentAndNormalVectors:320}
p_2 \propto ( 0, 0, 1, d_2 f)
\end{equation}
\end{subequations}

Our tangent volume in this neighborhood is

\begin{dmath}\label{eqn:tangentAndNormalVectors:340}
p_0 \wedge p_1 \wedge p_2
=
\left(
\gamma_0 + \gamma_3 \partial_0 f
\right)
\wedge
\left(
\gamma_1 + \gamma_3 \partial_1 f
\right)
\wedge
\left(
\gamma_2 + \gamma_3 \partial_2 f
\right)
=
\left(
\gamma_0 \gamma_1
+ \gamma_0 \gamma_3 \partial_1 f
+ \gamma_3 \gamma_1 \partial_0 f
\right)
\wedge
\left(
\gamma_2 + \gamma_3 \partial_2 f
\right)
=
\gamma_{012} - \gamma_{023} \partial_1 f + \gamma_{123} \partial_0 f + \gamma_{013} \partial_2 f.
\end{dmath}

Here the shorthand \(\gamma_{ijk} = \gamma_i \gamma_j \gamma_k\) has been used.  Can we factor out a 4D pseudoscalar from this and end up with a coherent result?  We have

\begin{subequations}
\begin{equation}\label{eqn:tangentAndNormalVectors:360}
\gamma_{0123} \gamma^3 = \gamma_{012}
\end{equation}
\begin{equation}\label{eqn:tangentAndNormalVectors:380}
\gamma_{0123} \gamma^1 = \gamma_{023}
\end{equation}
\begin{equation}\label{eqn:tangentAndNormalVectors:400}
\gamma_{0123} \gamma^0 = -\gamma_{123}
\end{equation}
\begin{equation}\label{eqn:tangentAndNormalVectors:420}
\gamma_{0123} \gamma^2 = -\gamma_{013}.
\end{equation}
\end{subequations}

This gives us

\begin{equation}\label{eqn:tangentAndNormalVectors:440}
d^3 p
=
p_0 \wedge p_1 \wedge p_2
=
\gamma_{0123} \left(
\gamma^3
- \gamma^1 \partial_1 f
- \gamma^0 \partial_0 f
- \gamma^2 \partial_2 f
\right).
\end{equation}

With the usual 4d gradient definition (sum implied)

\begin{dmath}\label{eqn:tangentAndNormalVectors:460}
\grad = \gamma^\mu \partial_\mu,
\end{dmath}

we have

\begin{dmath}\label{eqn:tangentAndNormalVectors:480}
\grad x^3
= \gamma^\mu \partial_\mu x^3
= \gamma^\mu {\delta_{\mu}}^3
= \gamma^3,
\end{dmath}

so we can write
\begin{dmath}\label{eqn:tangentAndNormalVectors:500}
d^3 p = \gamma_{0123} \grad \left( x^3 - f(x^0, x^1, x^2) \right),
\end{dmath}

so, finally, the ``normal'' to this surface volume element at this point is

\boxedEquation{eqn:tangentAndNormalVectors:520}{
n =
\grad \left( x^3 - f(x^0, x^1, x^2) \right).
}

This is just like the 3D Euclidean result, with the exception that we need to look at the dual of a 3-volume ``surface'' instead of our normal 2D surface.

It may seem curious that we had to specify an Euclidean metric for the 3D case, but did not here.  That doesn't mean this is a metric free result.  Instead, the metric choice is built into the definition of the gradient \eqnref{eqn:tangentAndNormalVectors:460} and its associated reciprocal basis.  For example with a \(1,3\) metric where \(\gamma_0^2 = 1, \gamma_k^2 = -1\), we have \(\gamma^0 = \gamma_0\) and \(\gamma^k = -\gamma_k\).

% this is to produce the sites.google url and version info and so forth (for blog posts)
%\vcsinfo
%\EndArticle


   \chapter{Stokes' theorem}
      %
% Copyright � 2013 Peeter Joot.  All Rights Reserved.
% Licenced as described in the file LICENSE under the root directory of this GIT repository.
%
\label{chap:stokesTheoremGeometricAlgebra}

The subject of \textAndIndex{differential forms} is one way to obtain an understanding of how to apply \textAndIndex{Stokes theorem} to higher dimensional spaces, non-Euclidean metrics, and \textAndIndex{curvilinear coordinate} systems.
%as covered for example in
%, and \citep{spivak1965calculus}.  However, both of those texts, despite their small size, are intensely scary.
%
The formalism of differential forms requires reexpressing physical quantities as ``forms''.  A notable example, as given in \citep{flanders1989dfa}, introduces a \textAndIndex{two-form} for the Faraday field

\begin{dmath}\label{eqn:stokesTheoremGeometricAlgebraII:5850}
\alpha
=
\lr{ E_1 dx^1 + E_2 dx^2 + E_3 dx^3 } \wedge (c dt)
+
\lr{
 H_1 dx^2 \wedge dx^3
 H_2 dx^3 \wedge dx^1
 H_3 dx^1 \wedge dx^2
}.
\end{dmath}

With a metric for which \((c dt, c dt) = -1\), and a cooresponding notion of \textAndIndex{duality} (*), Maxwell's equations for freespace become

\begin{equation}\label{eqn:stokesTheoremGeometricAlgebraII:5870}
\begin{aligned}
d\alpha &= 0 \\
d\conj\alpha &= 0.
\end{aligned}
\end{equation}

These are equivalent to the usual pair of tensor equations

\begin{equation}\label{eqn:stokesTheoremGeometricAlgebra:5890}
\begin{aligned}
\partial_i F^{ij} &= 0 \\
\epsilon^{ijkl} \partial_j F_{kl} &= 0
\end{aligned},
\end{equation}

or the Geometric Algebra equation
\begin{dmath}\label{eqn:stokesTheoremGeometricAlgebra:5910}
\grad F = 0.
\end{dmath}

An aspect of differential forms that I found unintuitive, is that all physical quantities have to be expressed as forms, even when we have no pressing desire to integrate them.  It also seemed to me that it ought to be possible to express volume and area elements in parameterized spaces directly as wedge products.  For example, given a two parameter surface of all the points that can be traced out on \(\Bx(u, v)\), we can express the (oriented) area of a patch of that surface directly as

\begin{dmath}\label{eqn:stokesTheoremGeometricAlgebra:5930}
dA = \lr{ du \PD{u}{\Bx} } \wedge \lr{ dv \PD{v}{\Bx} }.
\end{dmath}

With such a capability is the abstract notion of a form really required?  Can we stick with the vector notation that we are comfortable with, perhaps just generalizing slightly?  How would something like Stokes theorem, a basic tool needed to tackle so many problems in electromagnetism, be expressed so that it acted on vectors directly?

An answer to this question was found in Denker's straight wire treatment \citep{DenkerWire}, which states that the geometric algebra formulation of Stokes theorem has the form

\begin{equation}\label{eqn:vector_integral_relations:stokesGA}
\int_S \grad \wedge F = \int_{\partial S} F.
\end{equation}

This looks simple enough, but there are some important details left out.  In particular the grades do not match, so there must be some sort of implied projection or dot product.  We also need to understand how to express the hypervolume and hypersurfaces when evaluating these integrals, especially when we want to use curvilinear coordinates.

If one restricts attention to the special case where the dimension of the integration volume also equaled the dimension of the vector space, so that the grade of the curl matches the grade of the space (i.e. integration of a two form \(\int d^2 \Bx \cdot \lr{\spacegrad \wedge \Bf}\) in a two dimensional space), then some of those important details are not difficult to work out.

To treat a more general case, such as the same two form \(\int d^2 \Bx \cdot (\spacegrad \wedge \Bf)\) in a space of dimension greater than two, we need to introduce the notion of tangent space.  That concept can also be found in differential forms, but can also be expressed directly in vector algebra.  Suppose, for example, that \(\Bx(u,v,w)\) parameterizes a subspace, the the tangent space at the point of evaluation is the space that is spanned by \(\setlr{ \PDi{u}{\Bx}, \PDi{v}{\Bx}, \PDi{w}{\Bx} }\).  Stokes theorem is expressed not in terms of the gradient \(\spacegrad\), but in terms of the projection of the gradient onto the tangent space.  This projection denoted by \(\boldpartial\) and called the vector derivative.  The concept of tangent space and and vector derivative are covered thoroughly in \citep{aMacdonaldVAGC}, which also introduces Stokes theorem as a special case of a more fundamental theorem for integration of geometric algebraic objects.

The objective of this chapter is to detail the Geometric Algebra form of Stokes theorem, and not the fundamental theorem of geometric calculus.  Specific aims include the generalization of Stokes theorem to higher dimensional spaces and non-Euclidean metrics and understanding how to properly deal with curvilinear coordinates.  An understanding of both non-Euclidean and higher dimenional spaces is required for
special relativity and electromagnetism. This generalization has the form

%
% Copyright © 2013 Peeter Joot.  All Rights Reserved.
% Licenced as described in the file LICENSE under the root directory of this GIT repository.
%
\maketheorem{Stokes' Theorem}{thm:stokesTheoremGeometricAlgebra:1740}{

For blades \(F \in \bigwedge^{s}\), and \(m\) volume element \(d^k \Bx, s < k\),
%
\begin{equation*}%\label{eqn:stokesTheoremGeometricAlgebra:120}
\int_V d^k \Bx \cdot (\boldpartial \wedge F) = \int_{\partial V} d^{k-1} \Bx \cdot F.
\end{equation*}
%
Here the volume integral is over a \(m\) dimensional surface (manifold).  The derivative operator \(\boldpartial\) is called the vector derviative and is the projection of the gradient onto the tangent space of the manifold.  Integration over the boundary of \(V\) is indicated by \( \partial V \).
}

The vector derivative is defined by
%
\begin{equation}\label{eqn:stokesTheoremGeometricAlgebra:1400}
\boldpartial = \Bx^i \partial_i = \sum_i \Bx_i \PD{u^i}{}.
\end{equation}
%
where \( \Bx^i \) are reciprocal frame vectors dual to the tangent vector basis \( \Bx_i \) associated with the parameters \( u^1, u^2, \cdots \).  
%These will be defined in more detail in the next section.  
Once the volume element, vector product and the other concepts are defined, the proof of
Stokes theorem is really just a statement that

\boxedEquation{eqn:stokesTheoremGeometricAlgebra:2840}{
\int_V d^k \Bx \cdot (\Bx^i \partial_i \wedge F) =
\int_V \lr{ d^k \Bx \cdot \Bx^i } \cdot \partial_i F.
}

This dot product expansion applies to any degree blade and volume element provided the degree of the blade is less than that of the volume element (i.e. \(s < k\)).  That magic follows directly from \cref{thm:stokesTheoremGeometricAlgebra:1420}.


This dot product defines the oriented surface ``area'' elements associated with the ``volume'' element \( d^k \Bx \).  That area element is

\begin{equation}\label{eqn:stokesTheoremGeometricAlgebra:1401}
d^{k-1} \Bx = \sum_i d^k \Bx \cdot \Bx^i.
\end{equation}

The \(i\)th contribution \( d^k \Bx \cdot \Bx^i \) to this area elements is evaluated with the associated parameter \( u^i \) held constant, which allows the right hand side integral of \eqnref{eqn:stokesTheoremGeometricAlgebra:2840} to be evaluated for each \( i\) independently.
This means the area element \(d^{k-1} \Bx \) and the blade \(F\) that it is dotted with in Stokes theorem, are both evaluated at the end points of integration variable \(u^i\) that has been integrated against.

Some work is required to make sense of these concepts and fully detail the meaning of all the terms in this statement of Stokes Theorem.
I will attempt to do so in a gradual fashion, providing a number of examples
such as that of \eqnref{eqn:stokesTheoremGeometricAlgebra:1760}, and \eqnref{eqn:stokesTheoremGeometricAlgebra:1820}, which
illustrate some of the relevant details.

Before we can get there, we need some notation for curvilinear coordinates, and need to define the reciprocal frame vectors \( \Bx^i \) used in the expansion of the vector derivative curl.

\section{Curvilinear coordinates}
%\section{Basic notation}

A finite vector space, not necessarily Euclidean, with basis \(\setlr{\Be_1, \Be_2, \cdots}\) will be assumed to be the generator of the geometric algebra.  A dual or reciprocal basis \(\setlr{\Be^1, \Be^2, \cdots}\) for this basis can be calculated, defined by the property

\begin{equation}\label{eqn:stokesTheoremGeometricAlgebra:20}
\Be_i \cdot \Be^j = {\delta_i}^j.
\end{equation}

This is an Euclidean space when \(\Be_i = \Be^i, \forall i\).

%%% basic notation %%%%%%Implicit summation over repeated indices, typically over mixed upper and lower indices, will be assumed unless otherwise noted.
%%% basic notation %%%%%%For example, the components of a vector \(\Bx\) with respect to the standard or reciprocal bases, are
%%% basic notation %%%%%%
%%% basic notation %%%%%%\begin{equation}\label{eqn:stokesTheoremGeometricAlgebra:40}
%%% basic notation %%%%%%\Bx = \Be_i x^i = \Be^j x_j.
%%% basic notation %%%%%%\end{equation}
%%% basic notation %%%%%%
%%% basic notation %%%%%%The coordinates of the vector follow by taking dot products
%%% basic notation %%%%%%
%%% basic notation %%%%%%\begin{subequations}
%%% basic notation %%%%%%\begin{equation}\label{eqn:stokesTheoremGeometricAlgebra:60}
%%% basic notation %%%%%%\Bx \cdot \Be^j = \lr{ \Be_i x^i } \cdot \Be^j = x^i {\delta_i}^j = x^j
%%% basic notation %%%%%%\end{equation}
%%% basic notation %%%%%%\begin{equation}\label{eqn:stokesTheoremGeometricAlgebra:80}
%%% basic notation %%%%%%\Bx \cdot \Be_j = \lr{ \Be^i x_i } \cdot \Be_j = x_i {\delta^i}_j = x_j
%%% basic notation %%%%%%\end{equation}
%%% basic notation %%%%%%\end{subequations}
%%% basic notation %%%%%%
%%% basic notation %%%%%%Similarly, a bivector \(F\) in coordinate representation has the form
%%% basic notation %%%%%%
%%% basic notation %%%%%%\begin{equation}\label{eqn:stokesTheoremGeometricAlgebra:1180}
%%% basic notation %%%%%%F = \inv{2} \Be^i \wedge \Be^j F_{ij} = \inv{2} \Be_i \wedge \Be_j F^{ij},
%%% basic notation %%%%%%\end{equation}
%%% basic notation %%%%%%
%%% basic notation %%%%%%where
%%% basic notation %%%%%%\begin{equation}\label{eqn:stokesTheoremGeometricAlgebra:1200}
%%% basic notation %%%%%%\begin{aligned}
%%% basic notation %%%%%%F_{ij} &= \lr{ F \cdot \Be_j } \cdot \Be_i \\
%%% basic notation %%%%%%F^{ij} &= \lr{ F \cdot \Be^j } \cdot \Be^i.
%%% basic notation %%%%%%\end{aligned}
%%% basic notation %%%%%%\end{equation}
%%% basic notation %%%%%%
%%% basic notation %%%
%%% basic notation %%%To select from a multivector \(A\) the grade \(k\) portion, say \(A_k\) we write
%%% basic notation %%%
%%% basic notation %%%\begin{equation}\label{eqn:stokesTheoremGeometricAlgebra:1220}
%%% basic notation %%%A_k = \gpgrade{A}{k}.
%%% basic notation %%%\end{equation}
%%% basic notation %%%
%%% basic notation %%%The scalar portion of a multivector \(A\) will be written as
%%% basic notation %%%
%%% basic notation %%%\begin{equation}\label{eqn:stokesTheoremGeometricAlgebra:1240}
%%% basic notation %%%\gpgrade{A}{0} \equiv \gpgradezero{A}.
%%% basic notation %%%\end{equation}
%%% basic notation %%%
%%% basic notation %%%The grade selection operators can be used to define the outer and inner products.  For blades \(U\), and \(V\) of grade \(r\) and \(s\) respectively, these are
%%% basic notation %%%
%%% basic notation %%%\begin{subequations}
%%% basic notation %%%\begin{dmath}\label{eqn:stokesTheoremGeometricAlgebra:300}
%%% basic notation %%%\gpgrade{ U V }{\Abs{r + s}} \equiv U \wedge V
%%% basic notation %%%\end{dmath}
%%% basic notation %%%\begin{dmath}\label{eqn:stokesTheoremGeometricAlgebra:780}
%%% basic notation %%%\gpgrade{ U V }{\Abs{r - s}} \equiv U \cdot V.
%%% basic notation %%%\end{dmath}
%%% basic notation %%%\end{subequations}
%%% basic notation %%%
%%% basic notation %%%Written out explicitly for odd grade blades \(A\) (vector, trivector, ...), and vector \(\Ba\) the dot and wedge products are respectively
%%% basic notation %%%
%%% basic notation %%%\begin{equation}\label{eqn:stokesTheoremGeometricAlgebra:800}
%%% basic notation %%%\begin{aligned}
%%% basic notation %%%\Ba \wedge A &= \inv{2} (\Ba A - A \Ba) \\
%%% basic notation %%%\Ba \cdot A &= \inv{2} (\Ba A + A \Ba).
%%% basic notation %%%\end{aligned}
%%% basic notation %%%\end{equation}
%%% basic notation %%%
%%% basic notation %%%Similarly for even grade blades these are
%%% basic notation %%%
%%% basic notation %%%\begin{equation}\label{eqn:stokesTheoremGeometricAlgebra:820}
%%% basic notation %%%\begin{aligned}
%%% basic notation %%%\Ba \wedge A &= \inv{2} (\Ba A + A \Ba) \\
%%% basic notation %%%\Ba \cdot A &= \inv{2} (\Ba A - A \Ba).
%%% basic notation %%%\end{aligned}
%%% basic notation %%%\end{equation}
%%% basic notation %%%
%%% basic notation %%%It will be useful to employ the cyclic scalar reordering identity for the scalar selection operator
%%% basic notation %%%
%%% basic notation %%%\begin{equation}\label{eqn:stokesTheoremGeometricAlgebra:920}
%%% basic notation %%%\gpgradezero{\Ba \Bb \Bc}
%%% basic notation %%%= \gpgradezero{\Bb \Bc \Ba}
%%% basic notation %%%= \gpgradezero{\Bc \Ba \Bb}.
%%% basic notation %%%\end{equation}
%%% basic notation %%%
%%% basic notation %%%For an \(N\) dimensional vector space, a product of \(N\) orthonormal (up to a sign) unit vectors is referred to as a pseudoscalar for the space, typically denoted by \(I\)
%%% basic notation %%%
%%% basic notation %%%\begin{equation}\label{eqn:stokesTheoremGeometricAlgebra:5790}
%%% basic notation %%%I = \Be_1 \Be_2 \cdots \Be_N.
%%% basic notation %%%\end{equation}
%%% basic notation %%%
%%% basic notation %%%The pseudoscalar may commute or anticommute with other blades in the space.  We may also form a pseudoscalar for a subspace spanned by vectors \(\setlr{\Ba, \Bb, \cdots, \Bc}\) by unit scaling the wedge products of those vectors \(\Ba \wedge \Bb \wedge \cdots \wedge \Bc\).
%%% basic notation %%%
%\section{Curvilinear coordinates}

For our purposes a \textAndIndex{manifold} can be loosely defined as a parameterized surface.  For example, a 2D manifold can be considered a surface in an \(n\) dimensional vector space, parameterized by two variables

\begin{equation}\label{eqn:stokesTheoremGeometricAlgebra:1280}
\Bx = \Bx(a,b) = \Bx(u^1, u^2).
\end{equation}

Note that the indices here do not represent exponentiation.  We can construct a basis for the manifold as

\begin{equation}\label{eqn:stokesTheoremGeometricAlgebra:1300}
\Bx_i = \PD{u^i}{\Bx}.
\end{equation}

On the manifold we can calculate a \textAndIndex{reciprocal basis} \(\setlr{\Bx^i}\), defined by requiring, at each point on the surface

\begin{equation}\label{eqn:stokesTheoremGeometricAlgebra:1320}
\Bx^i \cdot \Bx_j = {\delta^i}_j.
\end{equation}

Associated implicitly with this basis is a curvilinear coordinate representation defined by the projection operation

\begin{equation}\label{eqn:stokesTheoremGeometricAlgebra:1340}
\Bx = x^i \Bx_i,
\end{equation}

(sums over mixed indices are implied).   These coordinates can be calculated by taking dot products with the \textAndIndex{reciprocal frame vectors}

\begin{dmath}\label{eqn:stokesTheoremGeometricAlgebra:1360}
\Bx \cdot \Bx^i
= x^j \Bx_j \cdot \Bx^i
= x^j {\delta_j}^i
= x^i.
\end{dmath}

%Having used the same notation for coordinates with respect to the standard basis and its dual, we must rely on context to disambiguate the two.
In this document all coordinates are with respect to a specific \textAndIndex{curvilinear basis}, and not with respect to the \textAndIndex{standard basis} \(\setlr{\Be_i}\) or its \textAndIndex{dual basis} unless otherwise noted.

Similar to the usual notation for derivatives with respect to the standard basis coordinates we form a lower index partial derivative operator

\begin{dmath}\label{eqn:stokesTheoremGeometricAlgebra:2900}
\PD{u^i}{} \equiv \partial_i,
\end{dmath}

so that when the complete vector space is spanned by \(\setlr{\Bx_i}\) the gradient has the curvilinear representation

\begin{dmath}\label{eqn:stokesTheoremGeometricAlgebra:1380}
\spacegrad = \Bx^i \PD{u^i}{}.
\end{dmath}

%This can be motivated by noting that the directional derivative is defined by
%
%\begin{equation}\label{eqn:stokesTheoremGeometricAlgebra:1260}
%\Ba \cdot \spacegrad f(\Bx) = \lim_{t \rightarrow 0} \frac{f(\Bx + t \Ba) - f(\Bx)}{t}.
%\end{equation}
%%%
%%%the gradient \(\spacegrad\), in terms of the standard and dual bases is
%%%
%%%\begin{equation}\label{eqn:stokesTheoremGeometricAlgebra:100}
%%%\spacegrad \equiv \Be^i \PD{x^i}{}.
%%%%\equiv \Be^i \partial_i,
%%%\end{equation}

When the basis \(\setlr{\Bx_i}\) does not span the space, the projection of the gradient onto the tangent space at the point of evaluation is
given by \eqnref{eqn:stokesTheoremGeometricAlgebra:1400}.

%It will be convient to write this sum over indices \(i\) with the short hand \(\PD{u^i}{} \equiv \partial_i\), or
%
%\begin{dmath}\label{eqn:stokesTheoremGeometricAlgebra:1740}
%\boldpartial = \Bx^i \partial_i.
%\end{dmath}

% redundant:
%See \citep{aMacdonaldVAGC} for a more complete discussion of the gradient and vector derivatives in curvilinear coordinates.

\section{Green's theorem}

%
% Copyright © 2013 Peeter Joot.  All Rights Reserved.
% Licenced as described in the file LICENSE under the root directory of this GIT repository.
%
Given a two parameter (\(u,v\)) surface parameterization, the curvilinear coordinate representation of a vector \(\Bf\) has the form

\begin{dmath}\label{eqn:stokesTheoremGeometricAlgebra:1640}
\Bf = f_u \Bx^u + f_v \Bx^v + f_\perp \Bx^\perp.
\end{dmath}

We assume that the vector space is of dimension two or greater but otherwise unrestricted, and need not have an Euclidean basis.  Here \(f_\perp \Bx^\perp\) denotes the rejection of \(\Bf\) from the tangent space at the point of evaluation.  Green's theorem relates the integral around a closed curve to an ``area'' integral on that surface

\maketheorem{Green's Theorem}{thm:stokesTheoremGeometricAlgebra:1660}{
\index{Green's theorem}
\begin{equation*}
\ointctrclockwise \Bf \cdot d\Bl
=
\iint \lr{
-\PD{v}{f_u}
+\PD{u}{f_v}
}
du dv
\end{equation*}
}

Following the arguments used in \citep{schwartz1987pe} for Stokes theorem in three dimensions, we first evaluate the loop integral along the differential element of the surface at the point \(\Bx(u_0, v_0)\) evaluated over the range \((du, dv)\), as shown in the infinitesimal loop of \cref{fig:loopIntegralInfinitesimal:loopIntegralInfinitesimalFig1}.

\imageFigure{../../physicsplay/figures/gabook/loopIntegralInfinitesimalFig1}{Infinitesimal loop integral}{fig:loopIntegralInfinitesimal:loopIntegralInfinitesimalFig1}{0.2}

Over the infinitesimal area, the loop integral decomposes into

\begin{dmath}\label{eqn:stokesTheoremGeometricAlgebra:1700}
\ointctrclockwise \Bf \cdot d\Bl
=
\int \Bf \cdot d\Bx_1
+\int \Bf \cdot d\Bx_2
+\int \Bf \cdot d\Bx_3
+\int \Bf \cdot d\Bx_4,
\end{dmath}

where the differentials along the curve are

\begin{dmath}\label{eqn:stokesTheoremGeometricAlgebra:1600}
\begin{aligned}
d\Bx_1 &= \evalbar{ \PD{u}{\Bx} }{v = v_0} du \\
d\Bx_2 &= \evalbar{ \PD{v}{\Bx} }{u = u_0 + du} dv \\
d\Bx_3 &= -\evalbar{ \PD{u}{\Bx} }{v = v_0 + dv} du \\
d\Bx_4 &= -\evalbar{ \PD{v}{\Bx} }{u = u_0} dv.
\end{aligned}
\end{dmath}

It is assumed that the parameterization change \((du, dv)\) is small enough that this loop integral can be considered planar (regardless of the dimension of the vector space).  Making use of the fact that \(\Bx^\perp \cdot \Bx_\alpha = 0\) for \(\alpha \in \setlr{u,v}\), the loop integral is

\begin{dmath}\label{eqn:stokesTheoremGeometricAlgebra:1620}
\ointctrclockwise \Bf \cdot d\Bl
=
\int
\lr{
f_u \Bx^u + f_v \Bx^v + f_\perp \Bx^\perp
}
\cdot
\Bigl(
\Bx_u(u, v_0) du - \Bx_u(u, v_0 + dv) du
+\Bx_v(u_0 + du, v) dv - \Bx_v(u_0, v) dv
\Bigr)
=
\int
f_u(u, v_0) du - f_u(u, v_0 + dv) du
+
f_v(u_0 + du, v) dv - f_v(u_0, v) dv
\end{dmath}

With the distances being infinitesimal, these differences can be rewritten as partial differentials

\begin{dmath}\label{eqn:stokesTheoremGeometricAlgebra:1860}
\ointctrclockwise \Bf \cdot d\Bl
=
\iint \lr{
-\PD{v}{f_u}
+\PD{u}{f_v}
}
du dv.
\end{dmath}

We can now sum over a larger area as in \cref{fig:loopIntegralInfinitesimalSum:loopIntegralInfinitesimalSumFig2}

\imageFigure{../../physicsplay/figures/gabook/loopIntegralInfinitesimalSumFig2}{Sum of infinitesimal loops}{fig:loopIntegralInfinitesimalSum:loopIntegralInfinitesimalSumFig2}{0.2}

All the opposing oriented loop elements cancel, so the integral around the complete boundary of the surface \(\Bx(u, v)\) is given by the \(u,v\) area integral of the partials difference.

We will see that Green's theorem is a special case of the Stokes theorem.  This observation will also provide a geometric interpretation of the right hand side area integral of \cref{thm:stokesTheoremGeometricAlgebra:1660}, and allow for a coordinate free representation.

\paragraph{Special case:}

An important special case of Green's theorem is for a Euclidean two dimensional space where the vector function is

\begin{dmath}\label{eqn:stokesTheoremGeometricAlgebra:1720}
\Bf = P \Be_1 + Q \Be_2.
\end{dmath}

Here Green's theorem takes the form

\boxedEquation{eqn:stokesTheoremGeometricAlgebra:1710}{
\ointctrclockwise P dx + Q dy
=
\iint \lr{
\PD{x}{Q}
-\PD{y}{P}
}
dx dy.
}


\section{Stokes theorem, two volume vector field}

Having examined the right hand side of \cref{thm:stokesTheoremGeometricAlgebra:1740} for the very simplest geometric object \(\Bf\), let's look at the right hand side, the area integral in more detail.  We restrict our attention for now to vectors \(\Bf\) still defined by \eqnref{eqn:stokesTheoremGeometricAlgebra:1640}.

First we need to assign a meaning to \(d^2 \Bx\).  By this, we mean the wedge products of the two differential elements.  With

\begin{dmath}\label{eqn:stokesTheoremGeometricAlgebra:1780}
d\Bx_i = du^i \PD{u^i}{\Bx} = du^i \Bx_i,
\end{dmath}

that area element is

\begin{equation}\label{eqn:stokesTheoremGeometricAlgebra:1800}
d^2 \Bx
= d\Bx_1 \wedge d\Bx_2
= du^1 du^2 \Bx_1 \wedge \Bx_2.
\end{equation}

This is the oriented area element that lies in the \textAndIndex{tangent plane} at the point of evaluation, and has the magnitude of the area of that segment of the surface, as depicted in \cref{fig:loopIntegralAreaElements:loopIntegralAreaElementsFig3}.

\imageFigure{../figures/gabook/loopIntegralAreaElementsFig3}{Oriented area element tiling of a surface}{fig:loopIntegralAreaElements:loopIntegralAreaElementsFig3}{0.3}

Observe that we have no requirement to introduce a normal to the surface to describe the direction of the plane.  The wedge product provides the information about the orientation of the place in the space, even when the vector space that our vector lies in has dimension greater than three.

Proceeding with the expansion of the dot product of the area element with the curl, using scalar selection
%\eqnref{eqn:stokesTheoremGeometricAlgebra:800}, \eqnref{eqn:stokesTheoremGeometricAlgebra:820}, and \eqnref{eqn:stokesTheoremGeometricAlgebra:920}, and a scalar selection operation, we have

\begin{dmath}\label{eqn:stokesTheoremGeometricAlgebra:1760}
d^2 \Bx \cdot \lr{ \boldpartial \wedge \Bf }
=
\gpgradezero{
d^2 \Bx \lr{ \boldpartial \wedge \Bf }
}
=
\gpgradezero{
d^2 \Bx
\inv{2}
\lr{ \rboldpartial \Bf - \Bf \lboldpartial }
}
=
\inv{2}
\gpgradezero{
d^2 \Bx \lr{ \Bx^i \lr{ \partial_i \Bf} - \lr{\partial_i \Bf} \Bx^i }
}
=
\inv{2}
\gpgradezero{
\lr{ \partial_i \Bf } d^2 \Bx \,\Bx^i - \lr{ \partial_i \Bf } \Bx^i d^2 \Bx
}
=
\gpgradezero{
\lr{ \partial_i \Bf } \lr{ d^2 \Bx \cdot \Bx^i }
}
=
\partial_i \Bf \cdot
\lr{ d^2 \Bx \cdot \Bx^i }.
\end{dmath}

Let's proceed to expand the inner dot product

\begin{dmath}\label{eqn:stokesTheoremGeometricAlgebra:1820}
d^2 \Bx \cdot \Bx^i
=
du^1 du^2
\lr{ \Bx_1 \wedge \Bx_2 } \cdot \Bx^i
=
du^1 du^2
\lr{
\Bx_2 \cdot \Bx^i \Bx_1
-\Bx_1 \cdot \Bx^i \Bx_2
}
=
du^1 du^2
\lr{
{\delta_2}^i \Bx_1
-{\delta_1}^i \Bx_2
}.
\end{dmath}

The complete curl term is thus

\begin{dmath}\label{eqn:stokesTheoremGeometricAlgebra:1840}
d^2 \Bx \cdot \lr{ \boldpartial \wedge \Bf }
=
du^1 du^2
\lr{
\PD{u^2}{\Bf} \cdot \Bx_1
-\PD{u^1}{\Bf} \cdot \Bx_2
}
\end{dmath}

This almost has the form of \eqnref{eqn:stokesTheoremGeometricAlgebra:1860}, although that is not immediately obvious.  Working backwards, using the shorthand \(u = u^1, v = u^2\), we can show that this coordinate representation can be eliminated

\begin{dmath}\label{eqn:stokesTheoremGeometricAlgebra:1900}
-du dv
\lr{
\PD{u}{f_v} -\PD{v}{f_u}
}
=
du dv
\lr{
\PD{v}{}\lr{\Bf \cdot \Bx_u}
-\PD{u}{}\lr{\Bf \cdot \Bx_v}
}
=
du dv
\lr{
\PD{v}{\Bf} \cdot \Bx_u
-\PD{u}{\Bf} \cdot \Bx_v
+
\Bf \cdot \lr{
\PD{v}{\Bx_u}
-\PD{u}{\Bx_v}
}
}
=
du dv
\lr{
\PD{v}{\Bf} \cdot \Bx_u
-\PD{u}{\Bf} \cdot \Bx_v
+
\Bf \cdot \lr{
\frac{\partial^2 \Bx}{\partial v \partial u}
-\frac{\partial^2 \Bx}{\partial u \partial v}
}
}
=
du dv
\lr{
\PD{v}{\Bf} \cdot \Bx_u
-\PD{u}{\Bf} \cdot \Bx_v
}
=
d^2 \Bx \cdot \lr{ \boldpartial \wedge \Bf }
.
\end{dmath}

This relates the two parameter surface integral of the curl to the loop integral over its boundary

\boxedEquation{eqn:stokesTheoremGeometricAlgebra:1980}{
\int d^2 \Bx \cdot (\boldpartial \wedge \Bf) = \ointclockwise \Bf \cdot d\Bl.
}

This is the very simplest special case of Stokes theorem.   When written in the general form of Stokes \cref{thm:stokesTheoremGeometricAlgebra:1740}

\begin{equation}\label{eqn:stokesTheoremGeometricAlgebra:5770}
\int_A d^2 \Bx \cdot \lr{ \boldpartial \wedge \Bf}
=
\int_{\partial A} d^1 \Bx \cdot \Bf
=
\int_{\partial A} \lr{ d\Bx_1 - d\Bx_2 } \cdot \Bf,
\end{equation}

we must remember (the \(\partial A\) is to remind us of this) that it is implied that both the vector \(\Bf\) and the differential elements are evaluated on the boundaries of the integration ranges respectively.  A more exact statement is

\boxedEquation{eqn:stokesTheoremGeometricAlgebra:3620}{
\int_{\partial A} d^1 \Bx \cdot \Bf
%\ointclockwise \Bf \cdot d\Bl
=
\int
\evalbar{\Bf \cdot d\Bx_1}{\Delta u^2}
-\evalbar{\Bf \cdot d\Bx_2}{\Delta u^1}
=
\int
\evalbar{f_1}{\Delta u^2} du^1
-\evalbar{f_2}{\Delta u^1} du^2.
}

Expanded out in full this is

\begin{equation}\label{eqn:stokesTheoremGeometricAlgebra:5810}
\int
\evalbar{\Bf \cdot d\Bx_1}{u^2(1)}
-\evalbar{\Bf \cdot d\Bx_1}{u^2(0)}
+\evalbar{\Bf \cdot d\Bx_2}{u^1(0)}
-\evalbar{\Bf \cdot d\Bx_2}{u^1(1)},
\end{equation}

which can be cross checked against \cref{fig:clockwiseLoopOnSurface:clockwiseLoopOnSurfaceFig12} to demonstrate that this specifies a clockwise orientation.  For the surface with oriented area \(d\Bx_1 \wedge d\Bx_2\), the clockwise loop is designated with line elements (1)-(4), we see that the contributions around this loop (in boxes) match \eqnref{eqn:stokesTheoremGeometricAlgebra:5810}.

\imageFigure{../figures/gabook/clockwiseLoopOnSurfaceFig12}{Clockwise loop}{fig:clockwiseLoopOnSurface:clockwiseLoopOnSurfaceFig12}{0.3}

%\imageFigure{../figures/gabook/cylindricalPolarParameterizationFig6}{Oriented polar parameterization loop}{fig:cylindricalPolarParameterization:cylindricalPolarParameterizationFig6}{0.15}

\makeexample{Green's theorem, a 2D Cartesian parameterization for a Euclidean space}{example:stokesTheoremGeometricAlgebra:2920}{

For a Cartesian 2D Euclidean parameterization of a vector field and the integration space, Stokes theorem should be equivalent to Green's theorem \eqnref{eqn:stokesTheoremGeometricAlgebra:1710}.  Let's expand both sides of \eqnref{eqn:stokesTheoremGeometricAlgebra:1980} independently to verify equality.  The parameterization is

\begin{dmath}\label{eqn:stokesTheoremGeometricAlgebra:2940}
\Bx(x, y) = x \Be_1 + y \Be_2.
\end{dmath}

Here the dual basis is the basis, and the projection onto the tangent space is just the gradient

\begin{equation}\label{eqn:stokesTheoremGeometricAlgebra:2960}
\boldpartial = \spacegrad
= \Be_1 \PD{x}{}
+ \Be_2 \PD{y}{}.
\end{equation}

The volume element is an area weighted pseudoscalar for the space

\begin{equation}\label{eqn:stokesTheoremGeometricAlgebra:2980}
d^2 \Bx = dx dy \PD{x}{\Bx} \wedge \PD{y}{\Bx} = dx dy \Be_1 \Be_2,
\end{equation}

and the curl of a vector \(\Bf = f_1 \Be_1 + f_2 \Be_2\) is

\begin{dmath}\label{eqn:stokesTheoremGeometricAlgebra:3000}
\boldpartial \wedge \Bf
=
\lr{
\Be_1 \PD{x}{}
+ \Be_2 \PD{y}{}
} \wedge
\lr{
f_1 \Be_1 + f_2 \Be_2
}
=
\Be_1 \Be_2
\lr{
\PD{x}{f_2}
-\PD{y}{f_1}
}.
\end{dmath}

So, the LHS of Stokes theorem takes the coordinate form

\begin{dmath}\label{eqn:stokesTheoremGeometricAlgebra:3020}
\int d^2 \Bx \cdot (\boldpartial \wedge \Bf) =
\iint dx dy
\mathLabelBox{\gpgradezero{\Be_1 \Be_2 \Be_1 \Be_2}}{\(=-1\)}
\lr{
\PD{x}{f_2}
-\PD{y}{f_1}
}.
\end{dmath}

For the RHS, following \cref{fig:euclidean2Dloop:euclidean2DloopFig4}, we have

\begin{dmath}\label{eqn:stokesTheoremGeometricAlgebra:3040}
\ointclockwise \Bf \cdot d\Bx
=
f_2(x_0, y) dy
+
f_1(x, y_1) dx
-
f_2(x_1, y) dy
-
f_1(x, y_0) dx
=
\int dx \lr{
f_1(x, y_1)
-
f_1(x, y_0)
}
-
\int dy \lr{
f_2(x_1, y)
-
f_2(x_0, y)
}.
\end{dmath}

As expected, we can also obtain this by integrating \eqnref{eqn:stokesTheoremGeometricAlgebra:3020}.

\imageFigure{../figures/gabook/euclidean2DloopFig4}{Euclidean 2D loop}{fig:euclidean2Dloop:euclidean2DloopFig4}{0.15}
}

\makeexample{Cylindrical parameterization}{example:stokesTheoremGeometricAlgebra:3060}{

Let's now consider a cylindrical parameterization of a 4D space with Euclidean metric \index{metric!Euclidean} \(++++\) or Minkowski metric \index{metric!Minkowski} \(+++-\).  For such a space let's do a brute force expansion of both sides of Stokes theorem to gain some confidence that all is well.

With \(\kappa = \Be_3 \Be_4\), such a space is conveniently parameterized as illustrated in \cref{fig:cylindricalPolarParameterization:cylindricalPolarParameterizationFig7} as

\begin{dmath}\label{eqn:stokesTheoremGeometricAlgebra:3080}
\Bx(\rho, \theta, h) = x \Be_1 + y \Be_2 + \rho \Be_3 e^{\kappa \theta}.
\end{dmath}

\imageFigure{../figures/gabook/cylindricalPolarParameterizationFig7}{Cylindrical polar parameterization}{fig:cylindricalPolarParameterization:cylindricalPolarParameterizationFig7}{0.15}

%\imageFigure{../figures/gabook/cylindricalPolarParameterizationFig5}{Cylindrical polar parameterization}{fig:cylindricalPolarParameterization:cylindricalPolarParameterizationFig5}{0.15}

Note that the Euclidean case where \(\lr{\Be_4}^2 = 1\) rejection of the non-axial components of \(\Bx\) expands to

\begin{dmath}\label{eqn:stokesTheoremGeometricAlgebra:3100}
\lr{ \lr{ \Bx \wedge \Be_1 \wedge \Be_2} \cdot \Be^2 } \cdot \Be^1 =
\rho \lr{ \Be_3 \cos\theta + \Be_4 \sin \theta },
\end{dmath}

whereas for the Minkowski case where \(\lr{\Be_4}^2 = -1\) the expansion is hyperbolic

\begin{dmath}\label{eqn:stokesTheoremGeometricAlgebra:3120}
\lr{ \lr{ \Bx \wedge \Be_1 \wedge \Be_2} \cdot \Be^2 } \cdot \Be^1 =
\rho \lr{ \Be_3 \cosh\theta + \Be_4 \sinh \theta }.
\end{dmath}

Within such a space consider the surface along \(x = c, y = d\), for which the vectors are parameterized by

\begin{dmath}\label{eqn:stokesTheoremGeometricAlgebra:3140}
\Bx(\rho, \theta) = c \Be_1 + d \Be_2 + \rho \Be_3 e^{\kappa \theta}.
\end{dmath}

The tangent space unit vectors are

\begin{dmath}\label{eqn:stokesTheoremGeometricAlgebra:3160}
\Bx_\rho
= \PD{\rho}{\Bx} = \Be_3 e^{\kappa \theta},
\end{dmath}

and
\begin{dmath}\label{eqn:stokesTheoremGeometricAlgebra:3180}
\Bx_\theta
= \PD{\theta}{\Bx}
= \rho \Be_3 \Be_3 \Be_4 e^{\kappa \theta}
= \rho \Be_4 e^{\kappa \theta}.
\end{dmath}

Observe that both of these vectors have their origin at the point of evaluation, and aren't relative to the absolute origin used to parameterize the complete space. % projective geometry?

We wish to compute the volume element for the tangent plane.  Noting that \(\Be_3\) and \(\Be_4\) both anticommute with \(\kappa\) we have for \(\Ba \in \Span \setlr{\Be_3, \Be_4}\)

\begin{dmath}\label{eqn:stokesTheoremGeometricAlgebra:3380}
\Ba e^{\kappa \theta} = e^{-\kappa \theta} \Ba,
\end{dmath}

so
\begin{dmath}\label{eqn:stokesTheoremGeometricAlgebra:3400}
\Bx_\theta \wedge \Bx_\rho
=
\gpgradetwo{
\Be_3 e^{\kappa \theta} \rho \Be_4 e^{\kappa \theta}
}
=
\rho \gpgradetwo{
\Be_3 e^{\kappa \theta} e^{-\kappa \theta} \Be_4
}
=
\rho \Be_3 \Be_4.
\end{dmath}

The tangent space volume element is thus

\begin{dmath}\label{eqn:stokesTheoremGeometricAlgebra:3420}
d^2 \Bx = \rho d\rho d\theta \Be_3 \Be_4.
\end{dmath}

With the tangent plane vectors both perpendicular we don't need the general \cref{thm:stokesTheoremGeometricAlgebra:3200} to compute the reciprocal basis, but can do so by inspection

\begin{equation}\label{eqn:stokesTheoremGeometricAlgebra:3540}
\Bx^\rho = e^{-\kappa \theta} \Be^3,
\end{equation}

and
\begin{equation}\label{eqn:stokesTheoremGeometricAlgebra:3560}
\Bx^\theta = e^{-\kappa \theta} \Be^4 \inv{\rho}.
\end{equation}

Observe that the latter depends on the metric signature.

The vector derivative, the projection of the gradient on the tangent space, is

\begin{dmath}\label{eqn:stokesTheoremGeometricAlgebra:3580}
\boldpartial
=
\Bx^\rho \PD{\rho}{}
+\Bx^\theta \PD{\theta}{}
=
e^{-\kappa \theta}
\lr{
\Be^3 \partial_\rho
+ \frac{\Be^4}{\rho} \partial_\theta
}.
\end{dmath}

From this we see that acting with the vector derivative on a scalar radial only dependent function \(f(\rho)\) is a vector function that has a radial direction, whereas the action of the vector derivative on an azimuthal only dependent function \(g(\theta)\) is a vector function that has only an azimuthal direction.  The interpretation of the geometric product action of the vector derivative on a vector function is not as simple since the product will be a multivector.

Expanding the curl in coordinates is messier, but yields in the end when tackled with sufficient care

\begin{dmath}\label{eqn:stokesTheoremGeometricAlgebra:3500}
\boldpartial \wedge \Bf
=
\gpgradetwo{
e^{-\kappa \theta}
\lr{
   e^3 \partial_\rho + \frac{e^4}{\rho} \partial_\theta
}
\lr{
   \cancel{e_1 x}
 + \cancel{e_2 y}
 + e_3 e^{\kappa \theta } f_\rho
 + \frac{e^4}{\rho} e^{\kappa \theta } f_\theta
}
}
=
\cancel{\gpgradetwo{
e^{-\kappa \theta}
   e^3 \partial_\rho
\lr{
   e_3 e^{\kappa \theta } f_\rho
}
}
}
+
\gpgradetwo{
\cancel{e^{-\kappa \theta}}
   e^3 \partial_\rho
\lr{
   \frac{e^4}{\rho} \cancel{e^{\kappa \theta }} f_\theta
}
}
+
\gpgradetwo{
e^{-\kappa \theta}
\frac{e^4}{\rho} \partial_\theta
\lr{
   e_3 e^{\kappa \theta } f_\rho
}
}
+
\gpgradetwo{
e^{-\kappa \theta}
\frac{e^4}{\rho} \partial_\theta
\lr{
   \frac{e^4}{\rho} e^{\kappa \theta } f_\theta
}
}
=
\Be^3
\Be^4
\lr{
-\frac{f_\theta}{\rho^2} + \inv{\rho} \partial_\rho f_\theta
- \inv{\rho} \partial_\theta f_\rho
}
+ \inv{\rho^2}
\gpgradetwo{
e^{-\kappa \theta} \lr{\Be^4}^2
\lr{
\Be_3 \Be_4 f_\theta
+ \cancel{\partial_\theta f_\theta}
}
e^{\kappa \theta}
}
=
\Be^3
\Be^4
\lr{
-\frac{f_\theta}{\rho^2} + \inv{\rho} \partial_\rho f_\theta
- \inv{\rho} \partial_\theta f_\rho
}
+ \inv{\rho^2}
\gpgradetwo{
\cancel{e^{-\kappa \theta} }
\Be_3 \Be^4 f_\theta
\cancel{e^{\kappa \theta}}
}
=
\frac{\Be^3
\Be^4
}{\rho}
\lr{
\partial_\rho f_\theta
- \partial_\theta f_\rho
}.
\end{dmath}

After all this reduction, we can now state in coordinates the LHS of Stokes theorem explicitly

\begin{dmath}\label{eqn:stokesTheoremGeometricAlgebra:3600}
\int d^2 \Bx \cdot \lr{ \boldpartial \wedge \Bf }
=
\int \rho d\rho d\theta
\gpgradezero{
\Be_3 \Be_4
\Be^3 \Be^4
}
\inv{\rho}
\lr{
\partial_\rho f_\theta
- \partial_\theta f_\rho
}
=
\int d\rho d\theta
\lr{
\partial_\theta f_\rho
-\partial_\rho f_\theta
}
=
\int d\rho \evalbar{f_\rho}{\Delta \theta}
-
\int d\theta
\evalbar{f_\theta}{\Delta \rho}.
\end{dmath}

Now compare this to the direct evaluation of the loop integral portion of Stokes theorem.  Expressing this using \eqnref{eqn:stokesTheoremGeometricAlgebra:3620}, we have the same result

\begin{equation}\label{eqn:stokesTheoremGeometricAlgebra:3640}
\int d^2 \Bx \cdot \lr{ \boldpartial \wedge \Bf }
=
\int
\evalbar{f_\rho}{\Delta \theta} d\rho
-\evalbar{f_\theta}{\Delta \rho} d\theta
\end{equation}

This example highlights some of the power of Stokes theorem, since the reduction of the volume element differential form was seen to be quite a chore (and easy to make mistakes doing.)
}

\makeexample{Composition of boost and rotation}{example:stokesTheoremGeometricAlgebra:4230}{

Working in a \(\bigwedge^{1,3}\) space with basis \(\setlr{\gamma_0, \gamma_1, \gamma_2, \gamma_3}\) where \(\lr{\gamma_0}^2 = 1\) and \(\lr{\gamma_k}^2 = -1, k \in \setlr{1,2,3}\), an active composition of boost and rotation has the form

\begin{equation}\label{eqn:stokesTheoremGeometricAlgebra:4250}
\begin{aligned}
\Bx' &= e^{i\alpha/2} \Bx_0 e^{-i\alpha/2} \\
\Bx'' &= e^{-j\theta/2} \Bx' e^{j\theta/2}
\end{aligned},
\end{equation}

where \(i\) is a bivector of a timelike unit vector and perpendicular spacelike unit vector, and \(j\) is a bivector of two perpendicular spacelike unit vectors.  For example, \(i = \gamma_0 \gamma_1\) and \(j = \gamma_1 \gamma_2\).  For such \(i,j\) the respective Lorentz transformation matrices are

\begin{equation}\label{eqn:stokesTheoremGeometricAlgebra:4270}
{
\begin{bmatrix}
x^0 \\
x^1 \\
x^2 \\
x^3
\end{bmatrix}
}'
=
\begin{bmatrix}
\cosh\alpha & -\sinh\alpha & 0 & 0 \\
-\sinh\alpha & \cosh\alpha & 0 & 0 \\
0 & 0 & 1 & 0 \\
0 & 0 & 0 & 1
\end{bmatrix}
\begin{bmatrix}
x^0 \\
x^1 \\
x^2 \\
x^3
\end{bmatrix},
\end{equation}

and

\begin{equation}\label{eqn:stokesTheoremGeometricAlgebra:4290}
{
\begin{bmatrix}
x^0 \\
x^1 \\
x^2 \\
x^3
\end{bmatrix}
}''
=
\begin{bmatrix}
1 & 0 & 0 & 0 \\
0 & \cos\theta & \sin\theta & 0  \\
0 & -\sin\theta & \cos\theta & 0  \\
0 & 0 & 0 & 1
\end{bmatrix}
{
\begin{bmatrix}
x^0 \\
x^1 \\
x^2 \\
x^3
\end{bmatrix}
}'.
\end{equation}

Let's calculate the tangent space vectors for this parameterization, assuming that the particle is at an initial spacetime position of \(\Bx_0\).  That is

\begin{equation}\label{eqn:stokesTheoremGeometricAlgebra:4310}
\Bx =
e^{-j\theta/2}
e^{i\alpha/2}
\Bx_0
e^{-i\alpha/2}
e^{j\theta/2}.
\end{equation}

To calculate the tangent space vectors for this subspace we note that

\begin{equation}\label{eqn:stokesTheoremGeometricAlgebra:4330}
\PD{\alpha}{\Bx'} = \frac{i}{2} \Bx_0 - \Bx_0 \frac{i}{2} = i \cdot \Bx_0,
\end{equation}

and
\begin{equation}\label{eqn:stokesTheoremGeometricAlgebra:4350}
\PD{\theta}{\Bx''} = -\frac{j}{2} \Bx' + \Bx' \frac{j}{2} = \Bx' \cdot j.
\end{equation}

The tangent space vectors are therefore

\begin{equation}\label{eqn:stokesTheoremGeometricAlgebra:4370}
\begin{aligned}
\Bx_\alpha &=
e^{-j\theta/2}
\lr{ i \cdot \Bx_0 }
e^{j\theta/2} \\
\Bx_\theta &=
\lr{
e^{i\alpha/2}
\Bx_0
e^{-i\alpha/2}
} \cdot j.
\end{aligned}
\end{equation}

Continuing a specific example where \(i = \gamma_0\gamma_1, j = \gamma_1 \gamma_2\) let's also pick \(\Bx_0 = \gamma_0\), the spacetime position of a particle at the origin of a frame at that frame's \(c t = 1\).  The tangent space vectors for the subspace parameterized by this transformation and this initial position is then reduced to

\begin{dmath}\label{eqn:stokesTheoremGeometricAlgebra:4390}
\Bx_\alpha = -\gamma_1 e^{j \theta} = \gamma_1 \sin\theta + \gamma_2 \cos\theta,
\end{dmath}

and
\begin{dmath}\label{eqn:stokesTheoremGeometricAlgebra:4410}
\Bx_\theta
= \lr{ \gamma_0 e^{-i \alpha} } \cdot j
= \lr{ \gamma_0\lr{ \cosh\alpha - \gamma_0 \gamma_1 \sinh\alpha }} \cdot \lr{ \gamma_1 \gamma_2}
=
\gpgradeone{ \lr{ \gamma_0 \cosh\alpha - \gamma_1 \sinh\alpha } \gamma_1 \gamma_2 }
= \gamma_2 \sinh\alpha.
\end{dmath}

By inspection the dual basis for this parameterization is

\begin{equation}\label{eqn:stokesTheoremGeometricAlgebra:4430}
\begin{aligned}
\Bx^\alpha &= \gamma_1 e^{j \theta} \\
\Bx^\theta &= \frac{\gamma^2}{\sinh\alpha}
\end{aligned}
\end{equation}

So, Stokes theorem, applied to a spacetime vector \(\Bf\), for this subspace is

\begin{equation}\label{eqn:stokesTheoremGeometricAlgebra:4450}
\begin{aligned}
\int d\alpha d\theta \sinh\alpha \sin\theta
&
\lr{ \gamma_1 \gamma_2 } \cdot
\lr{
\lr{
\gamma_1 e^{j \theta} \partial_\alpha
+ \frac{\gamma^2}{\sinh\alpha} \partial_\theta
}
\wedge \Bf
} \\
&=
\int d\alpha \evalrange{
\Bf \cdot \biglr{\gamma^1 e^{j \theta}}
}{\theta_0}{\theta_1}
-
\int d\theta \evalrange{
\Bf \cdot
\biglr{ \gamma_2
\sinh\alpha
}
}{\alpha_0}{\alpha_1}.
\end{aligned}
\end{equation}

Since the point is to avoid the curl integral, we did not actually have to state it explicitly, nor was there any actual need to calculate the dual basis.
}

\makeexample{Dual representation in three dimensions}{example:stokesTheoremGeometricAlgebra:3660}{

It's clear that there is a projective nature to the differential form \(d^2 \Bx \cdot \lr{\boldpartial \wedge \Bf}\).  This projective nature allows us, in three dimensions, to re-express Stokes theorem using the gradient instead of the vector derivative, and to utilize the cross product and a normal direction to the plane.

When we parameterize a normal direction to the tangent space, so that for a 2D tangent space spanned by curvilinear coordinates \(\Bx_1\) and \(\Bx_2\) the vector \(\Bx^3\) is normal to both, we can write our vector as

\begin{equation}\label{eqn:stokesTheoremGeometricAlgebra:3700}
\Bf = f_1 \Bx^1 + f_2 \Bx^2 + f_3 \Bx^3,
\end{equation}

and express the orientation of the tangent space area element in terms of a pseudoscalar that includes this normal direction

\begin{equation}\label{eqn:stokesTheoremGeometricAlgebra:3680}
\Bx_1 \wedge \Bx_2
=
\Bx^3 \cdot \lr{ \Bx_1 \wedge \Bx_2 \wedge \Bx_3 }
=
\Bx^3 \lr{ \Bx_1 \wedge \Bx_2 \wedge \Bx_3 }.
\end{equation}

Inserting this into an expansion of the curl form we have

\begin{dmath}\label{eqn:stokesTheoremGeometricAlgebra:3720}
d^2 \Bx \cdot \lr{ \boldpartial \wedge \Bf }
=
du^1 du^2 \gpgradezero{
\Bx^3 \lr{ \Bx_1 \wedge \Bx_2 \wedge \Bx_3 }
\lr{
\lr{
\sum_{i=1,2} x^i \partial_i
}
\wedge
\Bf
}
}
=
du^1 du^2
\Bx^3 \cdot
\lr{
\lr{ \Bx_1 \wedge \Bx_2 \wedge \Bx_3 }
\cdot \lr{\spacegrad \wedge \Bf}
-
\lr{ \Bx_1 \wedge \Bx_2 \wedge \Bx_3 }
\cdot \lr{\Bx^3 \partial_3 \wedge \Bf}
}.
\end{dmath}

Observe that this last term, the contribution of the component of the gradient perpendicular to the tangent space, has no \(\Bx_3\) components

\begin{dmath}\label{eqn:stokesTheoremGeometricAlgebra:3740}
\lr{ \Bx_1 \wedge \Bx_2 \wedge \Bx_3 }
\cdot \lr{\Bx^3 \partial_3 \wedge \Bf}
=
\lr{ \Bx_1 \wedge \Bx_2 \wedge \Bx_3 }
\cdot \lr{\Bx^3 \wedge \partial_3 \Bf}
=
\lr{
   \lr{ \Bx_1 \wedge \Bx_2 \wedge \Bx_3 } \cdot \Bx^3
}
\cdot \partial_3 \Bf
=
\lr { \Bx_1 \wedge \Bx_2 } \cdot \partial_3 \Bf
=
\Bx_1
\lr{ \Bx_2 \cdot \partial_3 \Bf }
-
\Bx_2
\lr{ \Bx_1 \cdot \partial_3 \Bf },
\end{dmath}

leaving
\begin{dmath}\label{eqn:stokesTheoremGeometricAlgebra:3760}
d^2 \Bx \cdot \lr{ \boldpartial \wedge \Bf }
=
du^1 du^2 \Bx^3 \cdot
\lr{
\lr{ \Bx_1 \wedge \Bx_2 \wedge \Bx_3 } \cdot \lr{ \spacegrad \wedge \Bf}
}.
\end{dmath}

Now scale the normal vector and its dual to have unit norm as follows

\begin{equation}\label{eqn:stokesTheoremGeometricAlgebra:3780}
\begin{aligned}
\Bx^3 &= \alpha \xcap^3 \\
\Bx_3 &= \inv{\alpha} \xcap_3,
\end{aligned}
\end{equation}

so that for \(\beta > 0\), the volume element can be

\begin{dmath}\label{eqn:stokesTheoremGeometricAlgebra:3860}
\Bx_1 \wedge \Bx_2 \wedge \xcap_3 = \beta I.
\end{dmath}

This scaling choice is illustrated in \cref{fig:outwardsNormal:outwardsNormalFig8}, and represents the ``outwards'' normal.  With such a scaling choice we have

\imageFigure{../figures/gabook/outwardsNormalFig8}{Outwards normal}{fig:outwardsNormal:outwardsNormalFig8}{0.15}

\begin{dmath}\label{eqn:stokesTheoremGeometricAlgebra:3800}
\beta du^1 du^2 = dA,
\end{dmath}

and almost have the desired cross product representation

\begin{dmath}\label{eqn:stokesTheoremGeometricAlgebra:3820}
d^2 \Bx \cdot \lr{ \boldpartial \wedge \Bf }
=
dA \xcap^3 \cdot \lr{ I \cdot \lr{\spacegrad \wedge \Bf} }
=
dA \xcap^3 \cdot \lr{ I \lr{\spacegrad \wedge \Bf} }.
\end{dmath}

With the duality identity \(\Ba \wedge \Bb = I \lr{\Ba \cross \Bb}\), we have the traditional 3D representation of Stokes theorem

\boxedEquation{eqn:stokesTheoremGeometricAlgebra:3840}{
\int d^2 \Bx \cdot \lr{ \boldpartial \wedge \Bf }
=
-
\int
dA \xcap^3 \cdot \lr{\spacegrad \cross \Bf}
= \ointclockwise \Bf \cdot d\Bl.
}

Note that the orientation of the loop integral in the traditional statement of the 3D Stokes theorem is counterclockwise instead of clockwise, as written here.
}

\section{Stokes theorem, three variable volume element parameterization}

Let's now proceed to evaluate the area elements defined by \eqnref{eqn:stokesTheoremGeometricAlgebra:2840} for a three variable parameterization, and a vector blade \(\Bf\)

\begin{dmath}\label{eqn:stokesTheoremGeometricAlgebra:1940}
d^3 \Bx \cdot \lr{ \boldpartial \wedge \Bf }
=
\lr{ d^3 \Bx \cdot \Bx^i } \cdot \partial_i \Bf
=
du^1 du^2 du^3
\lr{
\lr{ \Bx_1 \wedge \Bx_2 \wedge \Bx_3 }
 \cdot \Bx^i } \cdot \partial_i \Bf
=
du^1 du^2 du^3
\lr{
\lr{ \Bx_1 \wedge \Bx_2 } {\delta_3}^i
-\lr{ \Bx_1 \wedge \Bx_3 } {\delta_2}^i
+\lr{ \Bx_2 \wedge \Bx_3 } {\delta_1}^i
} \cdot \partial_i \Bf
=
du^1 du^2 du^3
\lr{
\lr{ \Bx_1 \wedge \Bx_2 } \cdot \partial_3 \Bf
-\lr{ \Bx_1 \wedge \Bx_3 } \cdot \partial_2 \Bf
+\lr{ \Bx_2 \wedge \Bx_3 } \cdot \partial_1 \Bf
}.
\end{dmath}

It should not be surprising that this has the structure found in the theory of differential forms.  Using the differentials for each of the parameterization ``directions'', we can write this dot product expansion as

\begin{dmath}\label{eqn:stokesTheoremGeometricAlgebra:2880}
d^3 \Bx \cdot \lr{ \boldpartial \wedge \Bf }
=
 du^3 \lr{ d\Bx_1 \wedge d\Bx_2 } \cdot \partial_3 \Bf
-du^2 \lr{ d\Bx_1 \wedge d\Bx_3 } \cdot \partial_2 \Bf
+du^1 \lr{ d\Bx_2 \wedge d\Bx_3 } \cdot \partial_1 \Bf
.
\end{dmath}

Observe that the sign changes with each element of \(d\Bx_1 \wedge d\Bx_2 \wedge d\Bx_3\) that is skipped.  In differential forms, the wedge product composition of 1-forms \index{1-form} is an abstract quantity.  Here the differentials are just vectors, and their wedge product represents an oriented volume element.  This interpretation is likely available in the theory of differential forms too, but is arguably less obvious.

\makedigression{
As was the case with the loop integral, we expect that the coordinate representation has a representation that can be expressed as a number of antisymmetric terms.  A bit of experimentation shows that such a sum, after dropping the parameter space volume element factor, is

\begin{dmath}\label{eqn:stokesTheoremGeometricAlgebra:1960}
\Bx_1 \lr{ -\partial_2 f_3 + \partial_3 f_2 }
+\Bx_2 \lr{ -\partial_3 f_1 + \partial_1 f_3 }
+\Bx_3 \lr{ -\partial_1 f_2 + \partial_2 f_1 }
=
\Bx_1 \lr{ -\partial_2 \Bf \cdot \Bx_3 + \partial_3 \Bf \cdot \Bx_2 }
+\Bx_2 \lr{ -\partial_3 \Bf \cdot \Bx_1 + \partial_1 \Bf \cdot \Bx_3 }
+\Bx_3 \lr{ -\partial_1 \Bf \cdot \Bx_2 + \partial_2 \Bf \cdot \Bx_1 }
=
\lr{ \Bx_1 \partial_3 \Bf \cdot \Bx_2 -\Bx_2 \partial_3 \Bf \cdot \Bx_1 }
+\lr{ \Bx_3 \partial_2 \Bf \cdot \Bx_1 -\Bx_1 \partial_2 \Bf \cdot \Bx_3 }
+\lr{ \Bx_2 \partial_1 \Bf \cdot \Bx_3 -\Bx_3 \partial_1 \Bf \cdot \Bx_2 }
=
 \lr{ \Bx_1 \wedge \Bx_2 } \cdot \partial_3 \Bf
+\lr{ \Bx_3 \wedge \Bx_1 } \cdot \partial_2 \Bf
+\lr{ \Bx_2 \wedge \Bx_3 } \cdot \partial_1 \Bf.
\end{dmath}
}

To proceed with the integration, we must again consider an infinitesimal volume element, for which the partial can be evaluated as the difference of the endpoints, with all else held constant.  For this three variable parameterization, say, \((u,v,w)\), let's delimit such an infinitesimal volume element by the parameterization ranges \([u_0,u_0 + du]\), \([v_0,v_0 + dv]\), \([w_0,w_0 + dw]\).  The integral is

\begin{equation}\label{eqn:stokesTheoremGeometricAlgebra:2000}
\begin{aligned}
\int_{u = u_0}^{u_0 + du}
\int_{v = v_0}^{v_0 + dv}
\int_{w = w_0}^{w_0 + dw}
d^3 \Bx \cdot \lr{ \boldpartial \wedge \Bf }
&=
\int_{u = u_0}^{u_0 + du}
du
\int_{v = v_0}^{v_0 + dv}
dv
\evalrange{ \biglr{ \lr{ \Bx_u \wedge \Bx_v } \cdot \Bf } }{w = w_0}{w_0 + dw} \\
&-
\int_{u = u_0}^{u_0 + du}
du
\int_{w = w_0}^{w_0 + dw}
dw
\evalrange{\biglr{ \lr{ \Bx_u \wedge \Bx_w } \cdot \Bf } }{v = v_0}{v_0 + dv} \\
&+
\int_{v = v_0}^{v_0 + dv}
dv
\int_{w = w_0}^{w_0 + dw}
dw
\evalrange{
\biglr{ \lr{ \Bx_v \wedge \Bx_w } \cdot \Bf } }{u = u_0}{u_0 + du}
.
\end{aligned}
\end{equation}

Extending this over the ranges \([u_0,u_0 + \Delta u]\), \([v_0,v_0 + \Delta v]\), \([w_0,w_0 + \Delta w]\), we have proved Stokes \cref{thm:stokesTheoremGeometricAlgebra:1740} for vectors and a three parameter volume element, provided we have a surface element of the form

\begin{dmath}\label{eqn:stokesTheoremGeometricAlgebra:2020}
d^2 \Bx =
\evalrange{ \biglr{d\Bx_u \wedge d\Bx_v } }{w = w_0}{w_1}
-
\evalrange{ \biglr{d\Bx_u \wedge d\Bx_w } }{v = v_0}{v_1}
+
\evalrange{ \biglr{d\Bx_v \wedge \Bx_w } }{ u = u_0 }{u_1},
\end{dmath}

where the evaluation of the dot products with \(\Bf\) are also evaluated at the same points.

\makeexample{Euclidean spherical polar parameterization of 3D subspace}{example:stokesTheoremGeometricAlgebra:3880}{

\index{coordinate!spherical}
Consider an Euclidean space where a 3D subspace is parameterized using spherical coordinates, as in

\begin{equation}\label{eqn:stokesTheoremGeometricAlgebra:3900}
\Bx(x, \rho, \theta, \phi) =
\Be_1 x + \Be_4 \rho \exp\lr{ \Be_4 \Be_2 e^{\Be_2 \Be_3 \phi} \theta}
=
\lr{x, \rho \sin\theta \cos\phi, \rho \sin\theta \sin\phi, \rho \cos\theta}.
\end{equation}

The tangent space basis for the subspace situated at some fixed \(x = x_0\), is easy to calculate, and is found to be

\begin{equation}\label{eqn:stokesTheoremGeometricAlgebra:5270}
\begin{aligned}
\Bx_\rho
&= \lr{0, \sin\theta \cos\phi, \sin\theta \sin\phi, \cos\theta}
 =
\Be_4 \exp\lr{ \Be_4 \Be_2 e^{\Be_2 \Be_3 \phi} \theta} \\
\Bx_\theta
&=
\rho
\lr{0,  \cos\theta \cos\phi,  \cos\theta \sin\phi, - \sin\theta}
=
\rho \Be_2 e^{\Be_2 \Be_3 \phi}
\exp\lr{ \Be_4 \Be_2 e^{\Be_2 \Be_3 \phi} \theta } \\
\Bx_\phi
&=
\rho
\lr{0, -\sin\theta \sin\phi,  \sin\theta \cos\phi, 0}
=
\rho \sin\theta \Be_3 e^{\Be_2 \Be_3 \phi}.
\end{aligned}
\end{equation}

While we can use the general relation of \cref{thm:stokesTheoremGeometricAlgebra:4950} to compute the reciprocal basis.  That is

\begin{equation}\label{eqn:stokesTheoremGeometricAlgebra:3940}
\Ba^\conj =
\lr{ \Bb \wedge \Bc } \inv{\Ba \wedge \Bb \wedge \Bc }.
\end{equation}

However, a naive attempt at applying this without algebraic software is a route that requires a lot of care, and is easy to make mistakes doing.  In this case it is really not necessary since the tangent space basis only requires scaling to orthonormalize, satisfying for \(i,j \in \setlr{\rho, \theta, \phi}\)

\begin{equation}\label{eqn:stokesTheoremGeometricAlgebra:3960}
\Bx_i \cdot \Bx_j =
\begin{bmatrix}
 1 & 0 & 0 \\
 0 & \rho^2 & 0 \\
 0 & 0 & \rho^2 \sin^2 \theta
\end{bmatrix}.
\end{equation}

This allows us to read off the dual basis for the tangent volume by inspection

\begin{equation}\label{eqn:stokesTheoremGeometricAlgebra:5290}
\begin{aligned}
\Bx^\rho
&=
\Be_4 \exp\lr{ \Be_4 \Be_2 e^{\Be_2 \Be_3 \phi} \theta} \\
\Bx^\theta
&=
\inv{\rho} \Be_2 e^{\Be_2 \Be_3 \phi}
\exp\lr{ \Be_4 \Be_2 e^{\Be_2 \Be_3 \phi} \theta } \\
\Bx^\phi
&=
\inv{\rho \sin\theta} \Be_3 e^{\Be_2 \Be_3 \phi}.
\end{aligned}
\end{equation}

Should we wish to explicitly calculate the curl on the tangent space, we would need these.  The area and volume elements are also messy to calculate manually.  This expansion can be found in the \textAndIndex{Mathematica} notebook \nbref{sphericalSurfaceAndVolumeElements.nb}, and is

\begin{equation}\label{eqn:stokesTheoremGeometricAlgebra:3980}
\begin{aligned}
\Bx_\theta \wedge \Bx_\phi &=
\rho^2 \sin\theta \left( \Be_4 \Be_2 \sin\theta \sin\phi + \Be_2 \Be_3 \cos\theta + \Be_3 \Be_4 \sin\theta \cos\phi \right) \\
\Bx_\phi \wedge \Bx_\rho &=
\rho \sin\theta \left(
-\Be_2 \Be_3 \sin\theta
-\Be_2 \Be_4 \cos\theta \sin\phi
+
\Be_3 \Be_4
\cos\theta \cos\phi
\right) \\
\Bx_\rho \wedge \Bx_\theta &= -\Be_4 \rho \left(
\Be_2
\cos\phi
+
\Be_3
\sin\phi
\right) \\
\Bx_\rho \wedge \Bx_\theta \wedge \Bx_\phi &= \Be_2 \Be_3 \Be_4 \rho^2 \sin\theta
\end{aligned}
\end{equation}

Those area elements have a Geometric Algebra factorization that are perhaps useful

\begin{equation}\label{eqn:stokesTheoremGeometricAlgebra:3990}
\begin{aligned}
\Bx_\theta \wedge \Bx_\phi &=
-\rho^2 \sin\theta \Be_2 \Be_3 \exp\lr{
-\Be_4 \Be_2 e^{\Be_2 \Be_3 \phi} \theta
} \\
\Bx_\phi \wedge \Bx_\rho &=
\rho \sin\theta \Be_3 \Be_4
e^{\Be_2 \Be_3 \phi}
\exp\lr{
\Be_4 \Be_2 e^{\Be_2 \Be_3 \phi} \theta
} \\
\Bx_\rho \wedge \Bx_\theta &= -
\rho
\Be_4 \Be_2
e^{\Be_2 \Be_3 \phi}
\end{aligned}.
\end{equation}

One of the beauties of Stokes theorem is that we don't actually have to calculate the \textAndIndex{dual basis} on the tangent space to proceed with the integration.  For that calculation above, where we had a normal tangent basis, I still used software was used as an aid, so it is clear that this can generally get pretty messy.

To apply Stokes theorem to a vector field we can use \eqnref{eqn:stokesTheoremGeometricAlgebra:2020} to write down the integral directly

\begin{dmath}\label{eqn:stokesTheoremGeometricAlgebra:4010}
\int_V d^3 \Bx \cdot \lr{ \boldpartial \wedge \Bf }
=
\int_{\partial V} d^2 \Bx \cdot \Bf
=
\int
 \evalrange{ \lr{ \Bx_\theta \wedge \Bx_\phi } \cdot \Bf }{\rho = \rho_0}{\rho_1} d\theta d\phi
+
\int
\evalrange{ \lr{ \Bx_\phi \wedge \Bx_\rho } \cdot \Bf }{\theta = \theta_0}{\theta_1}  d\phi d\rho
+
\int
\evalrange{ \lr{ \Bx_\rho \wedge \Bx_\theta } \cdot \Bf }{\phi = \phi_0}{\phi_1}  d\rho d\theta.
\end{dmath}

Observe that \eqnref{eqn:stokesTheoremGeometricAlgebra:4010} is a vector valued integral that expands to

\begin{equation}\label{eqn:stokesTheoremGeometricAlgebra:4030}
\int
 \evalrange{ \lr{ \Bx_\theta f_\phi - \Bx_\phi f_\theta } }{\rho = \rho_0}{\rho_1} d\theta d\phi
+
\int
 \evalrange{ \lr{ \Bx_\phi f_\rho - \Bx_\rho f_\phi } }{\theta = \theta_0}{\theta_1} d\phi d\rho
+
\int
 \evalrange{ \lr{ \Bx_\rho f_\theta - \Bx_\theta f_\rho } }{\phi = \phi_0}{\phi_1} d\rho d\theta.
\end{equation}

This could easily be a difficult integral to evaluate since the vectors \(\Bx_i\) evaluated at the endpoints are still functions of two parameters.  An easier integral would result from the application of Stokes theorem to a bivector valued field, say \(B\), for which we have

\begin{dmath}\label{eqn:stokesTheoremGeometricAlgebra:4050}
\int_V d^3 \Bx \cdot \lr{ \boldpartial \wedge B }
=
\int_{\partial V} d^2 \Bx \cdot B
=
\int
 \evalrange{ \lr{ \Bx_\theta \wedge \Bx_\phi } \cdot B }{\rho = \rho_0}{\rho_1} d\theta d\phi
+
\int
\evalrange{ \lr{ \Bx_\phi \wedge \Bx_\rho } \cdot B }{\theta = \theta_0}{\theta_1}  d\phi d\rho
+
\int
\evalrange{ \lr{ \Bx_\rho \wedge \Bx_\theta } \cdot B }{\phi = \phi_0}{\phi_1}  d\rho d\theta
=
\int
 \evalrange{ B_{\phi \theta} }{\rho = \rho_0}{\rho_1} d\theta d\phi
+
\int
\evalrange{ B_{\rho \phi} }{\theta = \theta_0}{\theta_1}  d\phi d\rho
+
\int
\evalrange{ B_{\theta \rho} }{\phi = \phi_0}{\phi_1}  d\rho d\theta.
\end{dmath}

There is a geometric interpretation to these oriented area integrals, especially when written out explicitly in terms of the differentials along the parameterization directions.  Pulling out a sign explicitly to match the geometry (as we had to also do for the line integrals in the two parameter volume element case), we can write this as

\begin{equation}\label{eqn:stokesTheoremGeometricAlgebra:4070}
\int_{\partial V} d^2 \Bx \cdot B
=
-\int
 \evalrange{ \lr{ d\Bx_\phi \wedge d\Bx_\theta } \cdot B }{\rho = \rho_0}{\rho_1}
-
\int
\evalrange{ \lr{ d\Bx_\rho \wedge d\Bx_\phi } \cdot B }{\theta = \theta_0}{\theta_1}
-
\int
\evalrange{ \lr{ d\Bx_\theta \wedge d\Bx_\rho } \cdot B }{\phi = \phi_0}{\phi_1}.
\end{equation}

When written out in this differential form, each of the respective area elements is an oriented area along one of the faces of the parameterization volume, much like the line integral that results from a two parameter volume curl integral.  This is visualized in \cref{fig:sphereicalParameterizationBoundaryFaces:sphereicalParameterizationBoundaryFacesFig9}.  In this figure, faces (1) and (3) are ``top faces'', those with signs matching the tops of the evaluation ranges \eqnref{eqn:stokesTheoremGeometricAlgebra:4070}, whereas face (2) is a bottom face with a sign that is correspondingly reversed.

\imageFigure{../figures/gabook/sphereicalParameterizationBoundaryFacesFig9}{Boundary faces of a spherical parameterization region}{fig:sphereicalParameterizationBoundaryFaces:sphereicalParameterizationBoundaryFacesFig9}{0.25}
}
%FIXME: figure titled 'Oriented surface of a three parameter subspace' in older doc looks nicer.

\makeexample{Minkowski hyperbolic-spherical polar parameterization of 3D subspace}{example:stokesTheoremGeometricAlgebra:3920}{

Working with a three parameter volume element in a \textAndIndex{Minkowski space} does not change much.  For example in a 4D space with \(\lr{\Be_4}^2 = -1\), we can employ a hyperbolic-spherical parameterization similar to that used above for the 4D Euclidean space

\begin{equation}\label{eqn:stokesTheoremGeometricAlgebra:4090}
\Bx(x, \rho, \alpha, \phi)
=
\setlr{x, \rho \sinh \alpha \cos\phi, \rho \sinh \alpha \sin\phi, \rho \cosh \alpha}
=
\Be_1 x + \Be_4 \rho \exp\lr{ \Be_4 \Be_2 e^{\Be_2 \Be_3 \phi} \alpha }.
\end{equation}

This has tangent space basis elements

\begin{equation}\label{eqn:stokesTheoremGeometricAlgebra:4110}
\begin{aligned}
\Bx_\rho &= \sinh\alpha \lr{ \cos\phi \Be_2 + \sin\phi \Be_3 } + \cosh\alpha \Be_4 =
\Be_4 \exp\lr{\Be_4 \Be_2 e^{\Be_2 \Be_3 \phi} \alpha} \\
\Bx_\alpha &=
\rho \cosh\alpha \lr{ \cos\phi \Be_2 + \sin\phi \Be_3} + \rho \sinh\alpha \Be_4
=
\rho \Be_2 e^{\Be_2 \Be_3 \phi} \exp\lr{-\Be_4 \Be_2 e^{\Be_2 \Be_3 \phi} \alpha} \\
\Bx_\phi &=
\rho \sinh\alpha \lr{ \Be_3 \cos\phi - \Be_2 \sin\phi} = \rho\sinh\alpha \Be_3 e^{\Be_2 \Be_3 \phi}.
\end{aligned}
\end{equation}

This is a normal basis, but again not orthonormal.  Specifically, for \(i,j \in \setlr{\rho, \theta, \phi}\) we have
\begin{equation}\label{eqn:stokesTheoremGeometricAlgebra:4130}
\Bx_i \cdot \Bx_j =
\begin{bmatrix}
-1 & 0 & 0 \\
 0 & \rho^2 & 0 \\
 0 & 0 & \rho^2 \sinh^2 \alpha
\end{bmatrix},
\end{equation}

where we see that the radial vector \(\Bx_\rho\) is timelike.  We can form the dual basis again by inspection
\begin{equation}\label{eqn:stokesTheoremGeometricAlgebra:4150}
\begin{aligned}
\Bx_\rho &= -\Be_4 \exp\lr{\Be_4 \Be_2 e^{\Be_2 \Be_3 \phi} \alpha} \\
\Bx_\alpha &= \inv{\rho} \Be_2 e^{\Be_2 \Be_3 \phi} \exp\lr{-\Be_4 \Be_2 e^{\Be_2 \Be_3 \phi} \alpha} \\
\Bx_\phi &= \inv{\rho\sinh\alpha} \Be_3 e^{\Be_2 \Be_3 \phi}.
\end{aligned}
\end{equation}

The area elements are

\begin{equation}\label{eqn:stokesTheoremGeometricAlgebra:4190}
\begin{aligned}
\Bx_\alpha \wedge \Bx_\phi &=
\rho^2 \sinh\alpha \left(-\Be_4 \Be_3 \sinh\alpha \cos\phi+\cosh\alpha \Be_2 \Be_3+\sinh\alpha \sin\phi \Be_2 \Be_4\right) \\
\Bx_\phi \wedge \Bx_\rho &=
\rho \sinh\alpha \left(-\Be_2 \Be_3 \sinh\alpha-\Be_2 \Be_4 \cosh\alpha \sin\phi+\cosh\alpha \cos\phi \Be_3 \Be_4\right) \\
\Bx_\rho \wedge \Bx_\alpha &=
-\Be_4 \rho \left(\cos\phi \Be_2+\sin\phi \Be_3\right),
\end{aligned}
\end{equation}

or

\begin{equation}\label{eqn:stokesTheoremGeometricAlgebra:4210}
\begin{aligned}
\Bx_\alpha \wedge \Bx_\phi &=
\rho^2 \sinh\alpha
\Be_2 \Be_3 \exp\lr{ \Be_4 \Be_2 e^{-\Be_2 \Be_3 \phi} \alpha } \\
\Bx_\phi \wedge \Bx_\rho &=
\rho\sinh\alpha \Be_3 \Be_4 e^{\Be_2 \Be_3 \phi} \exp\lr{
\Be_4 \Be_2 e^{\Be_2 \Be_3 \phi} \alpha
} \\
\Bx_\rho \wedge \Bx_\alpha &=
-\Be_4 \Be_2 \rho e^{\Be_2 \Be_3 \phi}.
\end{aligned}
\end{equation}

The volume element also reduces nicely, and is

\begin{equation}\label{eqn:stokesTheoremGeometricAlgebra:4170}
\Bx_\rho \wedge \Bx_\alpha \wedge \Bx_\phi = \Be_2 \Be_3 \Be_4 \rho^2 \sinh\alpha.
\end{equation}

The area and volume element reductions were once again messy, done in software using \nbref{sphericalSurfaceAndVolumeElementsMinkowski.nb}.  However, we really only need \eqnref{eqn:stokesTheoremGeometricAlgebra:4110} to perform the Stokes integration.
}

\section{Stokes theorem, four variable volume element parameterization}

Volume elements for up to four parameters are likely of physical interest, with the four volume elements of interest for \textAndIndex{relativistic physics} in \(\bigwedge^{3,1}\) spaces.  For example, we may wish to use a parameterization \(u^1 = x, u^2 = y, u^3 = z, u^4 = \tau = c t\), with a four volume

\begin{dmath}\label{eqn:stokesTheoremGeometricAlgebra:2040}
d^4 \Bx
=
d\Bx_x \wedge
d\Bx_y \wedge
d\Bx_z \wedge
d\Bx_\tau,
\end{dmath}

We follow the same procedure to calculate the corresponding boundary surface ``area'' element (with dimensions of volume in this case).  This is

\begin{dmath}\label{eqn:stokesTheoremGeometricAlgebra:2060}
d^4 \Bx \cdot \lr{ \boldpartial \wedge \Bf }
=
\lr{ d^4 \Bx \cdot \Bx^i } \cdot \partial_i \Bf
=
du^1 du^2 du^3 du^4
\lr{
\lr{ \Bx_1 \wedge \Bx_2 \wedge \Bx_3 \wedge \Bx_4 }
 \cdot \Bx^i } \cdot \partial_i \Bf
=
du^1 du^2 du^3 du_4
\lr{
\lr{ \Bx_1 \wedge \Bx_2 \wedge \Bx_3 } {\delta_4}^i
-\lr{ \Bx_1 \wedge \Bx_2 \wedge \Bx_4 } {\delta_3}^i
+\lr{ \Bx_1 \wedge \Bx_3 \wedge \Bx_4 } {\delta_2}^i
-\lr{ \Bx_2 \wedge \Bx_3 \wedge \Bx_4 } {\delta_1}^i
} \cdot \partial_i \Bf
=
du^1 du^2 du^3 du^4
\lr{
 \lr{ \Bx_1 \wedge \Bx_2 \wedge \Bx_3 } \cdot \partial_4 \Bf
-\lr{ \Bx_1 \wedge \Bx_2 \wedge \Bx_4 } \cdot \partial_3 \Bf
+\lr{ \Bx_1 \wedge \Bx_3 \wedge \Bx_4 } \cdot \partial_2 \Bf
-\lr{ \Bx_2 \wedge \Bx_3 \wedge \Bx_4 } \cdot \partial_1 \Bf
}.
\end{dmath}

Our boundary value surface element is therefore

\begin{dmath}\label{eqn:stokesTheoremGeometricAlgebra:2080}
d^3 \Bx =
  \Bx_1 \wedge \Bx_2 \wedge \Bx_3
- \Bx_1 \wedge \Bx_2 \wedge \Bx_4
+ \Bx_1 \wedge \Bx_3 \wedge \Bx_4
- \Bx_2 \wedge \Bx_3 \wedge \Bx_4.
\end{dmath}

where it is implied that this (and the dot products with \(\Bf\)) are evaluated on the boundaries of the integration ranges of the omitted index.  This same boundary form can be used for vector, bivector and trivector variations of Stokes theorem.

\section{Duality and its relation to the pseudoscalar.}

Looking to \eqnref{eqn:stokesTheoremGeometricAlgebra:4870} of \cref{thm:stokesTheoremGeometricAlgebra:3200}, and scaling the wedge product \(\Ba \wedge \Bb\)  by its absolute magnitude, we can express duality using that scaled bivector as a \textAndIndex{pseudoscalar} for the plane that spans \(\setlr{\Ba, \Bb}\).  Let's introduce a subscript notation for such scaled blades

\begin{dmath}\label{eqn:stokesTheoremGeometricAlgebra:4910}
I_{\Ba\Bb} = \frac{\Ba \wedge \Bb}{\Abs{\Ba \wedge \Bb}}.
\end{dmath}

This allows us to express the unit vector in the direction of \(\Ba^\conj\) as

\boxedEquation{example:stokesTheoremGeometricAlgebra:4950}{
%\acap^\conj
\aconjcap
= \bcap \frac{\Abs{\Ba \wedge \Bb}}{\Ba \wedge \Bb}
= \bcap \inv{I_{\Ba \Bb}}.
}

Following the pattern of \eqnref{eqn:stokesTheoremGeometricAlgebra:4870}, it is clear how to express the dual vectors for higher dimensional subspaces.  For example

or for the unit vector in the direction of \(\Ba^\conj\),

\begin{equation*}
%\acap^\conj
\aconjcap
= I_{\Bb \Bc} \inv{I_{\Ba \Bb \Bc} }.
\end{equation*}

\section{Divergence theorem.}

When the curl integral is a scalar result we are able to apply duality relationships to obtain the divergence theorem for the corresponding space.  We will be able to show that a relationship of the following form holds

\index{divergence theorem}
\boxedEquation{eqn:stokesTheoremGeometricAlgebra:4810}{
\int_V dV \spacegrad \cdot \Bf = \int_{\partial V} dA_i \ncap^i \cdot \Bf.
}

Here \(\Bf\) is a vector, \(\ncap^i\) is normal to the boundary surface, and \(dA_i\) is the area of this bounding surface element.  We wish to quantify these more precisely, especially because the orientation of the normal vectors are metric dependent.  Working a few specific examples will show the pattern nicely, but it is helpful to first consider some aspects of the general case.

First note that, for a scalar Stokes integral we are integrating the vector derivative curl of a blade \(F \in \bigwedge^{k-1}\) over a k-parameter volume element.  Because the dimension of the space matches the number of parameters, the projection of the gradient onto the tangent space is exactly that gradient

\begin{equation}\label{eqn:stokesTheoremGeometricAlgebra:4550}
\int_V d^k \Bx \cdot (\boldpartial \wedge F)
=
\int_V d^k \Bx \cdot (\spacegrad \wedge F).
%= \int_{\partial V} d^{k-1} \Bx \cdot F.
\end{equation}

Multiplication of \(F\) by the pseudoscalar will always produce a vector.  With the introduction of such a dual vector, as in

\begin{dmath}\label{eqn:stokesTheoremGeometricAlgebra:4570}
F = I \Bf,
\end{dmath}

Stokes theorem takes the form
\begin{dmath}\label{eqn:stokesTheoremGeometricAlgebra:4590}
\int_V d^k \Bx \cdot \gpgrade{\spacegrad I \Bf}{k}
= \int_{\partial V} \gpgradezero{ d^{k-1} \Bx I \Bf},
\end{dmath}

or
\begin{dmath}\label{eqn:stokesTheoremGeometricAlgebra:4610}
\int_V \gpgradezero{ d^k \Bx \spacegrad I \Bf}
= \int_{\partial V} \lr{ d^{k-1} \Bx I} \cdot \Bf,
\end{dmath}

where we will see that the vector \(d^{k-1} \Bx I\) can roughly be characterized as a normal to the boundary surface.  Using primes to indicate the scope of the action of the gradient, cyclic permutation within the scalar selection operator can be used to factor out the pseudoscalar

\begin{dmath}\label{eqn:stokesTheoremGeometricAlgebra:4630}
\int_V \gpgradezero{ d^k \Bx \spacegrad I \Bf}
=
\int_V \gpgradezero{ \Bf' d^k \Bx \spacegrad' I}
=
\int_V \gpgrade{ \Bf' d^k \Bx \spacegrad'}{k} I
=
\int_V
(-1)^{k+1} d^k \Bx \lr{ \spacegrad \cdot \Bf} I
=
(-1)^{k+1}
I^2
\int_V
dV
\lr{ \spacegrad \cdot \Bf}.
\end{dmath}

The second last step uses \cref{thm:stokesTheoremGeometricAlgebra:4690}, and the last writes \(d^k \Bx = I^2 \Abs{d^k \Bx} = I^2 dV\), where we have assumed (without loss of generality) that \(d^k \Bx\) has the same orientation as the pseudoscalar for the space.  We also assume that the parameterization is non-degenerate over the integration volume (i.e. no \(d\Bx_i = 0\)), so the sign of this product cannot change.

Let's now return to the normal vector \(d^{k-1} \Bx I\).  With \(d^{k-1} u_i = du^1 du^2 \cdots du^{i-1} du^{i+1} \cdots du^k\) (the \(i\) indexed differential omitted), and \(I_{ab\cdots c} = (\Bx_a \wedge \Bx_b \wedge \cdots \wedge \Bx_c)/\Abs{\Bx_a \wedge \Bx_b \wedge \cdots \wedge \Bx_c}\), we have

\begin{equation}\label{eqn:stokesTheoremGeometricAlgebra:4750}
\begin{aligned}
d^{k-1} \Bx I
&=
d^{k-1} u_i \lr{ \Bx_1 \wedge \Bx_2 \wedge \cdots \wedge \Bx_k} \cdot \Bx^i I \\
&=
I_{1 2 \cdots (k-1)} I \Abs{d\Bx_1 \wedge d\Bx_2 \wedge \cdots \wedge d\Bx_{k-1} } \\
&\quad -I_{1 \cdots (k-2) k} I \Abs{d\Bx_1 \wedge \cdots \wedge d\Bx_{k-2} \wedge d\Bx_k}
+ \cdots
\end{aligned}
\end{equation}

We've seen in \eqnref{example:stokesTheoremGeometricAlgebra:4950} and \cref{thm:stokesTheoremGeometricAlgebra:4950} that the dual of vector \(\Ba\) with respect to the unit pseudoscalar \(I_{\Bb \cdots \Bc \Bd}\) in a subspace spanned by \(\setlr{\Ba, \cdots \Bc, \Bd}\) is

\begin{equation}\label{eqn:stokesTheoremGeometricAlgebra:5010}
\widehat{\Ba^\conj} = I_{\Bb \cdots \Bc \Bd} \inv{ I_{\Ba \cdots \Bc \Bd} },
\end{equation}

or
\begin{equation}\label{eqn:stokesTheoremGeometricAlgebra:5030}
\widehat{\Ba^\conj} I_{\Ba \cdots \Bc \Bd}^2
=
I_{\Bb \cdots \Bc \Bd}.
\end{equation}

This allows us to write

\begin{equation}\label{eqn:stokesTheoremGeometricAlgebra:5050}
d^{k-1} \Bx I
=
I^2 \sum_i \widehat{\Bx^i} d{A'}_i
\end{equation}

where \(d{A'}_i = \pm dA_i\), and \(dA_i\) is the area of the boundary area element normal to \(\Bx^i\).  Note that the \(I^2\) term will now cancel cleanly from both sides of the divergence equation, taking both the metric and the orientation specific dependencies with it.

This leaves us with

\begin{equation}\label{eqn:stokesTheoremGeometricAlgebra:5070}
\int_V dV \spacegrad \cdot \Bf = (-1)^{k+1} \int_{\partial V} d{A'}_i \widehat{\Bx^i} \cdot \Bf.
\end{equation}

To spell out the details, we have to be very careful with the signs.  However, that is a job best left for specific examples.

\makeexample{2D divergence theorem}{example:stokesTheoremGeometricAlgebra:4690}{
\index{divergence theorem!2D}
Let's start back at

\begin{dmath}\label{eqn:stokesTheoremGeometricAlgebra:5230}
\int_A \gpgradezero{ d^2 \Bx \spacegrad I \Bf } = \int_{\partial A} \lr{ d^1 \Bx I} \cdot \Bf.
\end{dmath}

On the left our integral can be rewritten as

\begin{dmath}\label{eqn:stokesTheoremGeometricAlgebra:5090}
\int_A \gpgradezero{ d^2 \Bx \spacegrad I \Bf }
=
-\int_A \gpgradezero{ d^2 \Bx I \spacegrad \Bf }
=
-\int_A d^2 \Bx I \lr{ \spacegrad \cdot \Bf }
=
- I^2 \int_A dA \spacegrad \cdot \Bf,
\end{dmath}

where \(d^2 \Bx = I dA\) and we pick the \textAndIndex{pseudoscalar} with the same orientation as the volume (area in this case) element \(I = (\Bx_1 \wedge \Bx_2)/\Abs{\Bx_1 \wedge \Bx_2}\).

For the boundary form we have

\begin{dmath}\label{eqn:stokesTheoremGeometricAlgebra:5110}
d^1 \Bx
=
du^2 \lr{ \Bx_1 \wedge \Bx_2 } \cdot \Bx^1
+ du^1 \lr{ \Bx_1 \wedge \Bx_2 } \cdot \Bx^2
=
-du^2 \Bx_2
+du^1 \Bx_1.
\end{dmath}

The duality relations for the tangent space are

\begin{equation}\label{eqn:stokesTheoremGeometricAlgebra:5130}
\begin{aligned}
\Bx^2 &= \Bx_1 \inv{\Bx_2 \wedge \Bx_1} \\
\Bx^1 &= \Bx_2 \inv{\Bx_1 \wedge \Bx_2}
\end{aligned},
\end{equation}

or
\begin{equation}\label{eqn:stokesTheoremGeometricAlgebra:5150}
\begin{aligned}
\widehat{\Bx^2} &= -\widehat{\Bx_1} \inv{I} \\
\widehat{\Bx^1} &= \widehat{\Bx_2} \inv{I}
\end{aligned}.
\end{equation}

Back substitution into the line element gives

\begin{dmath}\label{eqn:stokesTheoremGeometricAlgebra:5170}
d^1 \Bx
=
-du^2 \Abs{\Bx_2} \widehat{\Bx_2}
+du^1 \Abs{\Bx_1} \widehat{\Bx_1}
=
-du^2 \Abs{\Bx_2} \widehat{\Bx^1} I
-du^1 \Abs{\Bx_1} \widehat{\Bx^2} I.
\end{dmath}

Writing (no sum) \(du^i \Abs{\Bx_i} = ds_i\), we have

\begin{dmath}\label{eqn:stokesTheoremGeometricAlgebra:5190}
d^1 \Bx I =
-\lr{ ds_2 \widehat{\Bx^1}
+ds_1 \widehat{\Bx^2} } I^2.
\end{dmath}

This provides us a divergence and normal relationship, with \(-I^2\) terms on each side that can be canceled.  Restoring explicit range evaluation, that is

\begin{dmath}\label{eqn:stokesTheoremGeometricAlgebra:5210}
%\int_{\partial A} \lr{ d^1 \Bx I } \cdot \Bf
\int_A dA \spacegrad \cdot \Bf
=
\int_{\Delta u^2} \evalbar{ ds_2 \widehat{\Bx^1} \cdot \Bf}{\Delta u^1}
+ \int_{\Delta u^1} \evalbar{ ds_1 \widehat{\Bx^2} \cdot \Bf}{\Delta u^2}
=
\int_{\Delta u^2} \evalbar{ ds_2 \widehat{\Bx^1} \cdot \Bf}{u^1(1)}
-\int_{\Delta u^2} \evalbar{ ds_2 \widehat{\Bx^1} \cdot \Bf}{u^1(0)}
+ \int_{\Delta u^1} \evalbar{ ds_1 \widehat{\Bx^2} \cdot \Bf}{u^2(0)}
- \int_{\Delta u^1} \evalbar{ ds_1 \widehat{\Bx^2} \cdot \Bf}{u^2(0)}.
\end{dmath}

Let's consider this graphically for an Euclidean metric as illustrated in \cref{fig:normalsOnSurfaceAreaElement:normalsOnSurfaceAreaElementFig10}.

\imageFigure{../figures/gabook/normalsOnSurfaceAreaElementFig10}{Normals on area element}{fig:normalsOnSurfaceAreaElement:normalsOnSurfaceAreaElementFig10}{0.25}

We see that
\begin{itemize}
\item along \(u^2(0)\) the outwards normal is \(-\widehat{\Bx^2}\),
\item along \(u^2(1)\) the outwards normal is \(\widehat{\Bx^2}\),
\item along \(u^1(0)\) the outwards normal is \(-\widehat{\Bx^1}\), and
\item along \(u^1(1)\) the outwards normal is \(\widehat{\Bx^2}\).
\end{itemize}

Writing that outwards normal as \(\ncap\), we have

\boxedEquation{eqn:stokesTheoremGeometricAlgebra:5250}{
\int_A dA \spacegrad \cdot \Bf
=
\ointctrclockwise ds \ncap \cdot \Bf.
}

Note that we can use the same algebraic notion of outward normal for non-Euclidean spaces, although cannot expect the geometry to look anything like that of the figure.
}

\makeexample{3D divergence theorem}{example:stokesTheoremGeometricAlgebra:4470}{
\index{divergence theorem!3D}
As with the 2D example, let's start back with

\begin{dmath}\label{eqn:stokesTheoremGeometricAlgebra:5330}
\int_V \gpgradezero{ d^3 \Bx \spacegrad I \Bf } = \int_{\partial V} \lr{ d^2 \Bx I} \cdot \Bf.
\end{dmath}

In a 3D space, the pseudoscalar commutes with all grades, so we have

\begin{dmath}\label{eqn:stokesTheoremGeometricAlgebra:5350}
\int_V \gpgradezero{ d^3 \Bx \spacegrad I \Bf }
=
\int_V \lr{ d^3 \Bx I } \spacegrad \cdot \Bf
=
I^2 \int_V dV \spacegrad \cdot \Bf,
\end{dmath}

where \(d^3 \Bx I = dV I^2\), and we have used a pseudoscalar with the same orientation as the volume element

\begin{equation}\label{eqn:stokesTheoremGeometricAlgebra:5390}
\begin{aligned}
I &= \widehat{ \Bx_{123} } \\
\Bx_{123} &= \Bx_1 \wedge \Bx_2 \wedge \Bx_3.
\end{aligned}
\end{equation}

In the boundary integral our dual two form is

\begin{dmath}\label{eqn:stokesTheoremGeometricAlgebra:5370}
d^2 \Bx I
=
 du^1 du^2 \Bx_1 \wedge \Bx_2
+du^3 du^1 \Bx_3 \wedge \Bx_1
+du^2 du^3 \Bx_2 \wedge \Bx_3
=
\lr{
% dA_{3} \widehat{\Bx_1 \wedge \Bx_2} \inv{I}
%+dA_{2} \widehat{\Bx_3 \wedge \Bx_1} \inv{I}
%+dA_{1} \widehat{\Bx_2 \wedge \Bx_3} \inv{I}
 dA_{3} \widehat{ \Bx_{12} } \inv{I}
+dA_{2} \widehat{ \Bx_{31} } \inv{I}
+dA_{1} \widehat{ \Bx_{23} } \inv{I}
} I^2,
\end{dmath}

where \(\Bx_{ij} = \Bx_i \wedge \Bx_j\), and

\begin{equation}\label{eqn:stokesTheoremGeometricAlgebra:5410}
\begin{aligned}
dA_1 &= \Abs{d\Bx_2 \wedge d\Bx_3} \\
dA_2 &= \Abs{d\Bx_3 \wedge d\Bx_1} \\
dA_3 &= \Abs{d\Bx_1 \wedge d\Bx_2}.
\end{aligned}
\end{equation}

Observe that we can do a cyclic permutation of a 3 blade without any change of sign, for example

\begin{equation}\label{eqn:stokesTheoremGeometricAlgebra:5430}
\Bx_1 \wedge \Bx_2 \wedge \Bx_3 =
-\Bx_2 \wedge \Bx_1 \wedge \Bx_3 =
\Bx_2 \wedge \Bx_3 \wedge \Bx_1.
\end{equation}

Because of this we can write the dual two form as we expressed the normals in \cref{thm:stokesTheoremGeometricAlgebra:4950}

\begin{dmath}\label{eqn:stokesTheoremGeometricAlgebra:5450}
d^2 \Bx I = \lr{
   dA_1
   \widehat{\Bx_{23}}
   \inv{\widehat{\Bx_{123}}}
   +
   dA_2
   \widehat{\Bx_{31}}
   \inv{\widehat{\Bx_{231}}}
   +
   dA_3
   \widehat{\Bx_{12}}
   \inv{\widehat{\Bx_{312}}}
} I^2
=
\lr{ dA_1 \widehat{\Bx^1}
+dA_2 \widehat{\Bx^2}
+dA_3 \widehat{\Bx^3} } I^2.
\end{dmath}

We can now state the 3D divergence theorem, canceling out the metric and orientation dependent term \(I^2\) on both sides

\begin{equation}\label{eqn:stokesTheoremGeometricAlgebra:5470}
\int_V dV \spacegrad \cdot \Bf
=
\int dA \ncap \cdot \Bf,
\end{equation}

where (sums implied)

\begin{equation}\label{eqn:stokesTheoremGeometricAlgebra:5490}
dA \ncap = dA_i \widehat{\Bx^i},
\end{equation}

and
\begin{equation}\label{eqn:stokesTheoremGeometricAlgebra:5510}
\begin{aligned}
\evalbar{\ncap}{u^i = u^i(1)} &= \widehat{\Bx^i} \\
\evalbar{\ncap}{u^i = u^i(0)} &= -\widehat{\Bx^i}
\end{aligned}.
\end{equation}

The outwards normals at the upper integration ranges of a three parameter surface are depicted in \cref{fig:normalsOnVolumeAreaElement:normalsOnVolumeAreaElementFig11}.

\imageFigure{../figures/gabook/normalsOnVolumeAreaElementFig11}{Outwards normals on volume at upper integration ranges.}{fig:normalsOnVolumeAreaElement:normalsOnVolumeAreaElementFig11}{0.25}

This sign alternation originates with the two form elements \(\lr{d\Bx_i \wedge d\Bx_j} \cdot F\) from the Stokes boundary integral, which were explicitly evaluated at the endpoints of the integral.  That is, for \(k \ne i,j\),

\begin{equation}\label{eqn:stokesTheoremGeometricAlgebra:5530}
\int_{\partial V} \lr{ d\Bx_i \wedge d\Bx_j } \cdot F
\equiv
\int_{\Delta u^i} \int_{\Delta u^j}
\evalbar{\lr{ \lr{ d\Bx_i \wedge d\Bx_j } \cdot F }}{u^k = u^k(1)}
-\evalbar{\lr{ \lr{ d\Bx_i \wedge d\Bx_j } \cdot F }}{u^k = u^k(0)}
\end{equation}

In the context of the divergence theorem, this means that we are implicitly requiring the dot products \(\widehat{\Bx^k} \cdot \Bf\) to be evaluated specifically at the end points of the integration where \(u^k = u^k(1), u^k = u^k(0)\), accounting for the alternation of sign required to describe the normals as uniformly outwards.
}

\makeexample{4D divergence theorem}{example:stokesTheoremGeometricAlgebra:5270}{
\index{divergence theorem!4D}
Applying Stokes theorem to a trivector \(T = I \Bf\) in the 4D case we find

\begin{dmath}\label{eqn:stokesTheoremGeometricAlgebra:5830}
-I^2 \int_V d^4 x \spacegrad \cdot \Bf =
\int_{\partial V} \lr{ d^3 \Bx I} \cdot \Bf.
\end{dmath}

Here the pseudoscalar has been picked to have the same orientation as the hypervolume element \(d^4 \Bx = I d^4 x\).  Writing \(\Bx_{ij \cdots k} = \Bx_i \wedge \Bx_j \wedge \cdots \Bx_k\) the dual of the three form is

\begin{dmath}\label{eqn:stokesTheoremGeometricAlgebra:5550}
d^3 \Bx I
=
\lr{
 du^1 du^2 du^3 \Bx_{123}
-du^1 du^2 du^4 \Bx_{124}
+du^1 du^3 du^4 \Bx_{134}
-du^2 du^3 du^4 \Bx_{234}
} I
=
\lr{
    dA^{123} \widehat{ \Bx_{123} }
   -dA^{124} \widehat{ \Bx_{124} }
   +dA^{134} \widehat{ \Bx_{134} }
   -dA^{234} \widehat{ \Bx_{234} }
} I
=
\lr{
    dA^{123} \widehat{ \Bx_{123} } \inv{\widehat{\Bx_{1234} }}
   -dA^{124} \widehat{ \Bx_{124} } \inv{\widehat{\Bx_{1234} }}
   +dA^{134} \widehat{ \Bx_{134} } \inv{\widehat{\Bx_{1234} }}
   -dA^{234} \widehat{ \Bx_{234} } \inv{\widehat{\Bx_{1234} }}
} I^2
=
-\lr{
    dA^{123} \widehat{ \Bx_{123} } \inv{\widehat{\Bx_{4123} }}
   +dA^{124} \widehat{ \Bx_{124} } \inv{\widehat{\Bx_{3412} }}
   +dA^{134} \widehat{ \Bx_{134} } \inv{\widehat{\Bx_{2341} }}
   +dA^{234} \widehat{ \Bx_{234} } \inv{\widehat{\Bx_{1234} }}
} I^2
=
-\lr{
    dA^{123} \widehat{ \Bx_{123} } \inv{\widehat{\Bx_{4123} }}
   +dA^{124} \widehat{ \Bx_{412} } \inv{\widehat{\Bx_{3412} }}
   +dA^{134} \widehat{ \Bx_{341} } \inv{\widehat{\Bx_{2341} }}
   +dA^{234} \widehat{ \Bx_{234} } \inv{\widehat{\Bx_{1234} }}
} I^2
=
-\lr{
    dA^{123} \widehat{ \Bx^{4} }
   +dA^{124} \widehat{ \Bx^{3} }
   +dA^{134} \widehat{ \Bx^{2} }
   +dA^{234} \widehat{ \Bx^{1} }
} I^2
\end{dmath}

Here, we've written

\begin{dmath}\label{eqn:stokesTheoremGeometricAlgebra:5570}
dA^{ijk} = \Abs{ d\Bx_i \wedge d\Bx_j \wedge d\Bx_k }.
\end{dmath}

Observe that the dual representation nicely removes the alternation of sign that we had in the Stokes theorem boundary integral, since each alternation of the wedged vectors in the pseudoscalar changes the sign once.

As before, we define the outwards normals as \(\ncap = \pm \widehat{\Bx^i}\) on the upper and lower integration ranges respectively.  The scalar area elements on these faces can be written in a dual form

\begin{equation}\label{eqn:stokesTheoremGeometricAlgebra:5590}
\begin{aligned}
   dA_4 &= dA^{123} \\
   dA_3 &= dA^{124} \\
   dA_2 &= dA^{134} \\
   dA_1 &= dA^{234}
\end{aligned},
\end{equation}

so that the 4D divergence theorem looks just like the 2D and 3D cases

\begin{equation}\label{eqn:stokesTheoremGeometricAlgebra:5610}
\int_V d^4 x \spacegrad \cdot \Bf =
\int_{\partial V} d^3 x \ncap \cdot \Bf.
\end{equation}

Here we define the volume scaled normal as

\begin{equation}\label{eqn:stokesTheoremGeometricAlgebra:5630}
d^3 x \ncap = dA_i \widehat{\Bx^i}.
\end{equation}

As before, we have made use of the implicit fact that the three form (and it's dot product with \(\Bf\)) was evaluated on the boundaries of the integration region, with a toggling of sign on the lower limit of that evaluation that is now reflected in what we have defined as the outwards normal.

We also obtain explicit instructions from this formalism how to compute the ``outwards'' normal for this surface in a 4D space (unit scaling of the dual basis elements), something that we cannot compute using any sort of geometrical intuition.  For free we've obtained a result that applies to both Euclidean and Minkowski (or other non-Euclidean) spaces.
}

\section{Volume integral coordinate representations}

It may be useful to formulate the curl integrals in \textAndIndex{tensor} form.  For vectors \(\Bf\), and bivectors \index{bivector} \(B\), the coordinate representations of those differential forms (\cref{pr:stokesTheoremGeometricAlgebra:1}) are

\begin{subequations}
\begin{equation}\label{eqn:stokesTheoremGeometricAlgebra:2160}
d^2 \Bx \cdot \lr{ \boldpartial \wedge \Bf }
=
- d^2 u \epsilon^{ a b } \partial_a f_b
\end{equation}
\begin{equation}\label{eqn:stokesTheoremGeometricAlgebra:2180}
d^3 \Bx \cdot \lr{ \boldpartial \wedge \Bf }
=
-
%\inv{2}
d^3 u \epsilon^{a b c} \Bx_a \partial_b f_{c}
\end{equation}
\begin{equation}\label{eqn:stokesTheoremGeometricAlgebra:2200}
d^4 \Bx \cdot \lr{ \boldpartial \wedge \Bf }
=
-\inv{2} d^4 u \epsilon^{a b c d} \Bx_a \wedge \Bx_b \partial_{c} f_{d}
\end{equation}
\begin{equation}\label{eqn:stokesTheoremGeometricAlgebra:2220}
d^3 \Bx \cdot \lr{ \boldpartial \wedge B }
=
-
\inv{2}
d^3 u \epsilon^{a b c} \partial_a B_{b c}
\end{equation}
\begin{equation}\label{eqn:stokesTheoremGeometricAlgebra:2240}
d^4 \Bx \cdot \lr{ \boldpartial \wedge B }
=
-\inv{2} d^4 u \epsilon^{a b c d} \Bx_a \partial_b B_{cd}
\end{equation}
\begin{equation}\label{eqn:stokesTheoremGeometricAlgebra:2340}
d^4 \Bx \cdot \lr{ \boldpartial \wedge T }
=
-
d^4 u
\lr{
\partial_4 T_{123}
-\partial_3 T_{124}
+\partial_2 T_{134}
-\partial_1 T_{234}
}.
\end{equation}
\end{subequations}

Here the bivector \(B\) and \textAndIndex{trivector} \(T\) is expressed in terms of their curvilinear components on the tangent space

\begin{subequations}
\begin{equation}\label{eqn:stokesTheoremGeometricAlgebra:2260}
B = \inv{2} \Bx^i \wedge \Bx^j B_{ij} + B_\perp
\end{equation}
\begin{equation}\label{eqn:stokesTheoremGeometricAlgebra:2270}
T = \inv{3!} \Bx^i \wedge \Bx^j \wedge \Bx^k T_{ijk} + T_\perp,
\end{equation}
\end{subequations}

where

\begin{subequations}
\begin{equation}\label{eqn:stokesTheoremGeometricAlgebra:2280}
B_{ij} = \Bx_j \cdot \lr{ \Bx_i \cdot B } = -B_{ji}.
\end{equation}
\begin{equation}\label{eqn:stokesTheoremGeometricAlgebra:2290}
T_{ijk} = \Bx_k \cdot \lr{ \Bx_j \cdot \lr{ \Bx_i \cdot B }}.
\end{equation}
\end{subequations}

For the trivector components are also antisymmetric, changing sign with any interchange of indices.

Note that \eqnref{eqn:stokesTheoremGeometricAlgebra:2220} and \eqnref{eqn:stokesTheoremGeometricAlgebra:2340} appear much different on the surface, but both have the same structure.  This can be seen by writing for former as

\begin{dmath}\label{eqn:stokesTheoremGeometricAlgebra:2640}
d^3 \Bx \cdot \lr{ \boldpartial \wedge B }
=
-
d^3 u
\lr{
    \partial_1 B_{2 3}
  + \partial_2 B_{3 1}
  + \partial_3 B_{1 2}
}
=
-
d^3 u
\lr{
    \partial_3 B_{1 2}
  - \partial_2 B_{1 3}
  + \partial_1 B_{2 3}
}.
\end{dmath}

In both of these we have an alternation of sign, where the tensor index skips one of the volume element indices is sequence.  We've seen in the 4D divergence theorem that this alternation of sign can be related to a duality transformation.

In integral form (no sum over indexes \(i\) in \(du^i\) terms), these are

\begin{subequations}
\begin{equation}\label{eqn:stokesTheoremGeometricAlgebra:5650}
\int d^2 \Bx \cdot \lr{ \boldpartial \wedge \Bf }
=
- \epsilon^{ a b } \int \evalbar{du^b f_b}{\Delta u^a}
\end{equation}
\begin{equation}\label{eqn:stokesTheoremGeometricAlgebra:5670}
\int d^3 \Bx \cdot \lr{ \boldpartial \wedge \Bf }
=
-
\epsilon^{a b c} \int
du^a du^c
\evalbar{\Bx_a f_{c}}{\Delta u^b}
\end{equation}
\begin{equation}\label{eqn:stokesTheoremGeometricAlgebra:5690}
\int d^4 \Bx \cdot \lr{ \boldpartial \wedge \Bf }
=
-\inv{2} \epsilon^{a b c d} \int
du^a du^b du^d
\evalbar{\Bx_a \wedge \Bx_b f_{d}}{\Delta u^c}
\end{equation}
\begin{equation}\label{eqn:stokesTheoremGeometricAlgebra:5710}
\int d^3 \Bx \cdot \lr{ \boldpartial \wedge B }
=
-
\inv{2}
\epsilon^{a b c}
\int
du^b du^c
\evalbar{B_{b c}}{\Delta u^a}
\end{equation}
\begin{equation}\label{eqn:stokesTheoremGeometricAlgebra:5730}
\int d^4 \Bx \cdot \lr{ \boldpartial \wedge B }
=
-\inv{2}
\epsilon^{a b c d}
\int
du^a du^c du^d
\evalbar{\Bx_a B_{cd}}{\Delta u^b}
\end{equation}
\begin{equation}\label{eqn:stokesTheoremGeometricAlgebra:5750}
\begin{aligned}
\int &d^4 \Bx \cdot \lr{ \boldpartial \wedge T }
= \\
&
\int
du^1 du^2 du^3 \evalbar{T_{123}}{\Delta u^4}
-\int
du^1 du^2 du^4 \evalbar{T_{124}}{\Delta u^3} \\
&+\int
du^1 du^3 du^4 \evalbar{T_{134}}{\Delta u^2}
-\int
du^2 du^3 du^4 \evalbar{T_{234}}{\Delta u^1}
.
\end{aligned}
\end{equation}
\end{subequations}

Of these, I suspect that only \eqnref{eqn:stokesTheoremGeometricAlgebra:5650} and \eqnref{eqn:stokesTheoremGeometricAlgebra:5710} are of use.

\section{Final remarks}

Because we have used curvilinear coordinates from the get go, we have arrived naturally at a formulation that works for both Euclidean and non-Euclidean geometries, and have demonstrated that Stokes (and the divergence theorem) holds regardless of the geometry or the parameterization.  We also know explicitly how to formulate both theorems for any parameterization that we choose, something much more valuable than knowledge that this is possible.

For the divergence theorem we have introduced the concept of outwards normal (for example in 3D, \eqnref{eqn:stokesTheoremGeometricAlgebra:5510}), which still holds for non-Euclidean geometries.  We may not be able to form intuitive geometrical interpretations for these normals, but do have an algebraic description of them.

\section{Problems}

\makeproblem{Expand volume elements in coordinates}{pr:stokesTheoremGeometricAlgebra:1}{
Show that the coordinate representation for the volume element dotted with the curl can be represented as a sum of antisymmetric terms.  That is

\makesubproblem{Prove
\eqnref{eqn:stokesTheoremGeometricAlgebra:2160}
}{pr:stokesTheoremGeometricAlgebra:a}
\makesubproblem{Prove
\eqnref{eqn:stokesTheoremGeometricAlgebra:2180}
}{pr:stokesTheoremGeometricAlgebra:b}
\makesubproblem{Prove
\eqnref{eqn:stokesTheoremGeometricAlgebra:2200}
}{pr:stokesTheoremGeometricAlgebra:c}
\makesubproblem{Prove
\eqnref{eqn:stokesTheoremGeometricAlgebra:2220}
}{pr:stokesTheoremGeometricAlgebra:d}
\makesubproblem{Prove
\eqnref{eqn:stokesTheoremGeometricAlgebra:2240}
}{pr:stokesTheoremGeometricAlgebra:e}
\makesubproblem{Prove
\eqnref{eqn:stokesTheoremGeometricAlgebra:2340}
}{pr:stokesTheoremGeometricAlgebra:f}
} % makeproblem

\makeanswer{pr:stokesTheoremGeometricAlgebra:1}{

\makeSubAnswer{Two parameter volume, curl of vector}{pr:stokesTheoremGeometricAlgebra:a}

\begin{dmath}\label{eqn:stokesTheoremGeometricAlgebra:2380}
d^2 \Bx \cdot \lr{ \boldpartial \wedge \Bf }
=
d^2 u
\biglr{ \lr{ \Bx_1 \wedge \Bx_2 } \cdot \Bx^i } \cdot \partial_i \Bf
=
d^2 u
\lr{
\Bx_1 \cdot \partial_2 \Bf
-\Bx_2 \cdot \partial_1 \Bf
}
=
d^2 u
\lr{
\partial_2 f_1
-\partial_1 f_2
}
=
- d^2 u \epsilon^{ab} \partial_{a} f_{b}. \qedmarker
\end{dmath}

\makeSubAnswer{Three parameter volume, curl of vector}{pr:stokesTheoremGeometricAlgebra:b}

\begin{dmath}\label{eqn:stokesTheoremGeometricAlgebra:2400}
d^3 \Bx \cdot \lr{ \boldpartial \wedge \Bf }
=
d^3 u
\biglr{ \lr{ \Bx_1 \wedge \Bx_2 \wedge \Bx_3 } \cdot \Bx^i } \cdot \partial_i \Bf
=
d^3 u
\biglr{
 \lr{ \Bx_1 \wedge \Bx_2 } \cdot \partial_3 \Bf
+\lr{ \Bx_3 \wedge \Bx_1 } \cdot \partial_2 \Bf
+\lr{ \Bx_2 \wedge \Bx_3 } \cdot \partial_1 \Bf
}
=
d^3 u
\biglr{
\lr{ \Bx_1 \partial_3 \Bf \cdot \Bx_2 -\Bx_2 \partial_3 \Bf \cdot \Bx_1 }
+\lr{ \Bx_3 \partial_2 \Bf \cdot \Bx_1 -\Bx_1 \partial_2 \Bf \cdot \Bx_3 }
+\lr{ \Bx_2 \partial_1 \Bf \cdot \Bx_3 -\Bx_3 \partial_1 \Bf \cdot \Bx_2 }
}
=
d^3 u
\biglr{
\Bx_1 \lr{ -\partial_2 \Bf \cdot \Bx_3 + \partial_3 \Bf \cdot \Bx_2 }
+\Bx_2 \lr{ -\partial_3 \Bf \cdot \Bx_1 + \partial_1 \Bf \cdot \Bx_3 }
+\Bx_3 \lr{ -\partial_1 \Bf \cdot \Bx_2 + \partial_2 \Bf \cdot \Bx_1 }
}
=
d^3 u
\biglr{
\Bx_1 \lr{ -\partial_2 f_3 + \partial_3 f_2 }
+\Bx_2 \lr{ -\partial_3 f_1 + \partial_1 f_3 }
+\Bx_3 \lr{ -\partial_1 f_2 + \partial_2 f_1 }
}
= - d^3 u \epsilon^{abc} \partial_b f_c. \qedmarker
\end{dmath}

\makeSubAnswer{Four parameter volume, curl of vector}{pr:stokesTheoremGeometricAlgebra:c}

\begin{equation}\label{eqn:stokesTheoremGeometricAlgebra:2420}
\begin{aligned}
d^4 \Bx \cdot \lr{ \boldpartial \wedge \Bf }
&=
d^4 u
\biglr{ \lr{ \Bx_1 \wedge \Bx_2 \wedge \Bx_3 \wedge \Bx_4 } \cdot \Bx^i } \cdot \partial_i \Bf \\
&=
d^4 u
\biglr{
\lr{ \Bx_1 \wedge \Bx_2 \wedge \Bx_3 } \cdot \partial_4 \Bf
-\lr{ \Bx_1 \wedge \Bx_2 \wedge \Bx_4 } \cdot \partial_3 \Bf
+\lr{ \Bx_1 \wedge \Bx_3 \wedge \Bx_4 } \cdot \partial_2 \Bf
-\lr{ \Bx_2 \wedge \Bx_3 \wedge \Bx_4 } \cdot \partial_1 \Bf
} \\
&=
d^4 u
\biglr{  \\
&\quad\quad \lr{ \Bx_1 \wedge \Bx_2 } \Bx_3 \cdot \partial_4 \Bf
-\lr{ \Bx_1 \wedge \Bx_3 } \Bx_2 \cdot \partial_4 \Bf
+\lr{ \Bx_2 \wedge \Bx_3 } \Bx_1 \cdot \partial_4 \Bf \\
%
&\quad-\lr{ \Bx_1 \wedge \Bx_2 } \Bx_4 \cdot \partial_3 \Bf
+\lr{ \Bx_1 \wedge \Bx_4 } \Bx_2 \cdot \partial_3 \Bf
-\lr{ \Bx_2 \wedge \Bx_4 } \Bx_1 \cdot \partial_3 \Bf \\
%
&\quad+ \lr{ \Bx_1 \wedge \Bx_3 } \Bx_4 \cdot \partial_2 \Bf
-\lr{ \Bx_1 \wedge \Bx_4 } \Bx_3 \cdot \partial_2 \Bf
+\lr{ \Bx_3 \wedge \Bx_4 } \Bx_1 \cdot \partial_2 \Bf \\
%
&\quad-\lr{ \Bx_2 \wedge \Bx_3 } \Bx_4 \cdot \partial_1 \Bf
+\lr{ \Bx_2 \wedge \Bx_4 } \Bx_3 \cdot \partial_1 \Bf
-\lr{ \Bx_3 \wedge \Bx_4 } \Bx_2 \cdot \partial_1 \Bf \\
&\qquad
} \\
&=
d^4 u
\biglr{
\Bx_1 \wedge \Bx_2 \partial_{[4} f_{3]}
+\Bx_1 \wedge \Bx_3 \partial_{[2} f_{4]}
+\Bx_1 \wedge \Bx_4 \partial_{[3} f_{2]}
+\Bx_2 \wedge \Bx_3 \partial_{[4} f_{1]}
+\Bx_2 \wedge \Bx_4 \partial_{[1} f_{3]}
+\Bx_3 \wedge \Bx_4 \partial_{[2} f_{1]}
} \\
&=
- \inv{2} d^4 u \epsilon^{abcd} \Bx_a \wedge \Bx_b \partial_{c} f_{d}. \qedmarker
\end{aligned}
\end{equation}

\makeSubAnswer{Three parameter volume, curl of bivector}{pr:stokesTheoremGeometricAlgebra:d}

\begin{equation}\label{eqn:stokesTheoremGeometricAlgebra:2440}
\begin{aligned}
d^3 \Bx \cdot \lr{ \boldpartial \wedge B }
&=
d^3 u
\biglr{ \lr{ \Bx_1 \wedge \Bx_2 \wedge \Bx_3 } \cdot \Bx^i } \cdot \partial_i B \\
&=
d^3 u
\biglr{
 \lr{ \Bx_1 \wedge \Bx_2 } \cdot \partial_3 B
+\lr{ \Bx_3 \wedge \Bx_1 } \cdot \partial_2 B
+\lr{ \Bx_2 \wedge \Bx_3 } \cdot \partial_1 B
} \\
&=
\inv{2} d^3 u
\biglr{ \Bx_1 \cdot \lr{ \Bx_2 \cdot \partial_3 B } -\Bx_2 \cdot \lr{ \Bx_1 \cdot \partial_3 B } \\
&\qquad +\Bx_3 \cdot \lr{ \Bx_1 \cdot \partial_2 B } -\Bx_1 \cdot \lr{ \Bx_3 \cdot \partial_2 B } \\
&\qquad +\Bx_2 \cdot \lr{ \Bx_3 \cdot \partial_1 B } -\Bx_3 \cdot \lr{ \Bx_2 \cdot \partial_1 B }
} \\
&=
\inv{2} d^3 u
\biglr{ \Bx_1 \cdot \lr{ \Bx_2 \cdot \partial_3 B - \Bx_3 \cdot \partial_2 B } \\
&\qquad +\Bx_2 \cdot \lr{ \Bx_3 \cdot \partial_1 B - \Bx_1 \cdot \partial_3 B } \\
&\qquad +\Bx_3 \cdot \lr{ \Bx_1 \cdot \partial_2 B - \Bx_2 \cdot \partial_1 B }
} \\
&=
\inv{2} d^3 u
\biglr{
\Bx_1 \cdot \lr{ \partial_3 \lr{ \Bx_2 \cdot B} - \partial_2 \lr{ \Bx_3 \cdot B} } \\
&\qquad +\Bx_2 \cdot \lr{ \partial_1 \lr{ \Bx_3 \cdot B} - \partial_3 \lr{ \Bx_1 \cdot B} } \\
&\qquad +\Bx_3 \cdot \lr{ \partial_2 \lr{ \Bx_1 \cdot B} - \partial_1 \lr{ \Bx_2 \cdot B} }
} \\
&=
\inv{2} d^3 u
\biglr{
\partial_2 \lr{ \Bx_3 \cdot \lr{ \Bx_1 \cdot B} } - \partial_3 \lr{ \Bx_2 \cdot \lr{ \Bx_1 \cdot B} } \\
%
&\qquad+ \partial_3 \lr{ \Bx_1 \cdot \lr{ \Bx_2 \cdot B} } - \partial_1 \lr{ \Bx_3 \cdot \lr{ \Bx_2 \cdot B} } \\
%
&\qquad+ \partial_1 \lr{ \Bx_2 \cdot \lr{ \Bx_3 \cdot B} } - \partial_2 \lr{ \Bx_1 \cdot \lr{ \Bx_3 \cdot B} }
} \\
&=
\inv{2} d^3 u
\biglr{
\partial_2 B_{13} - \partial_3 B_{12}
+\partial_3 B_{21} - \partial_1 B_{23}
+\partial_1 B_{32} - \partial_2 B_{31}
} \\
&=
d^3 u
\biglr{
\partial_2 B_{13}
+\partial_3 B_{21}
+\partial_1 B_{32}
} \\
&= - \inv{2} d^3 u \epsilon^{abc} \partial_a B_{bc}. \qedmarker
\end{aligned}
\end{equation}

\makeSubAnswer{Four parameter volume, curl of bivector}{pr:stokesTheoremGeometricAlgebra:e}

To start, we require \cref{thm:stokesTheoremGeometricAlgebra:2460}.  For convenience lets also write our wedge products as a single indexed quantity, as in \(\Bx_{abc}\) for \(\Bx_a \wedge \Bx_b \wedge \Bx_c\).  The expansion is

\begin{equation}\label{eqn:stokesTheoremGeometricAlgebra:2500}
\begin{aligned}
d^4 \Bx \cdot \lr{ \boldpartial \wedge B }
&=
d^4 u
\lr{ \Bx_{1234} \cdot \Bx^i } \cdot \partial_i B  \\
&=
d^4 u
\lr{
\Bx_{123} \cdot \partial_4 B
-\Bx_{124} \cdot \partial_3 B
+\Bx_{134} \cdot \partial_2 B
-\Bx_{234} \cdot \partial_1 B
} \\
&=
d^4 u
\biglr{
\Bx_1 \lr{ \Bx_{23} \cdot \partial_4 B }
+\Bx_2 \lr{ \Bx_{32} \cdot \partial_4 B }
+\Bx_3 \lr{ \Bx_{12} \cdot \partial_4 B } \\
%
&\qquad -\Bx_1 \lr{ \Bx_{24} \cdot \partial_3 B }
-\Bx_2 \lr{ \Bx_{41} \cdot \partial_3 B }
-\Bx_4 \lr{ \Bx_{12} \cdot \partial_3 B } \\
%
&\qquad +\Bx_1 \lr{ \Bx_{34} \cdot \partial_2 B }
+\Bx_3 \lr{ \Bx_{41} \cdot \partial_2 B }
+\Bx_4 \lr{ \Bx_{13} \cdot \partial_2 B } \\
%
&\qquad -\Bx_2 \lr{ \Bx_{34} \cdot \partial_1 B }
-\Bx_3 \lr{ \Bx_{42} \cdot \partial_1 B }
-\Bx_4 \lr{ \Bx_{23} \cdot \partial_1 B }
} \\
&=
d^4 u
\biglr{
\Bx_1 \lr{
 \Bx_{23} \cdot \partial_4 B
+\Bx_{42} \cdot \partial_3 B
+\Bx_{34} \cdot \partial_2 B
} \\
%
&\qquad +\Bx_2 \lr{
 \Bx_{32} \cdot \partial_4 B
+\Bx_{14} \cdot \partial_3 B
+\Bx_{43} \cdot \partial_1 B
} \\
%
&\qquad +\Bx_3 \lr{
 \Bx_{12} \cdot \partial_4 B
+\Bx_{41} \cdot \partial_2 B
+\Bx_{24} \cdot \partial_1 B
} \\
%
&\qquad +\Bx_4 \lr{
 \Bx_{21} \cdot \partial_3 B
+\Bx_{13} \cdot \partial_2 B
+\Bx_{32} \cdot \partial_1 B
}
} \\
&=
-\inv{2} d^4 u \epsilon^{a b c d} \Bx_a \partial_b B_{c d}.  \qedmarker
\end{aligned}
\end{equation}

This last step uses an intermediate result from the \eqnref{eqn:stokesTheoremGeometricAlgebra:2440} expansion above, since each of the four terms has the same structure we have previously observed.

\makeSubAnswer{Four parameter volume, curl of trivector}{pr:stokesTheoremGeometricAlgebra:f}

Using the \(\Bx_{ijk}\) shorthand again, the initial expansion gives

\begin{dmath}\label{eqn:stokesTheoremGeometricAlgebra:2520}
d^4 \Bx \cdot \lr{ \boldpartial \wedge T }
=
d^4 u
\lr{
\Bx_{123} \cdot \partial_4 T
-\Bx_{124} \cdot \partial_3 T
+\Bx_{134} \cdot \partial_2 T
-\Bx_{234} \cdot \partial_1 T
}.
\end{dmath}

Applying \cref{thm:stokesTheoremGeometricAlgebra:2560} to expand the inner products within the braces we have

\begin{equation}\label{eqn:stokesTheoremGeometricAlgebra:2720}
\begin{aligned}
\Bx_{123} \cdot \partial_4 T
-&\Bx_{124} \cdot \partial_3 T
+\Bx_{134} \cdot \partial_2 T
-\Bx_{234} \cdot \partial_1 T \\
&=
\Bx_1 \cdot \lr{ \Bx_2 \cdot \lr{ \Bx_3 \cdot \partial_4 T } }
-\Bx_1 \cdot \lr{ \Bx_2 \cdot \lr{ \Bx_4 \cdot \partial_3 T } }  \\
&\quad +
\mathLabelBox
[
   labelstyle={xshift=2cm},
   linestyle={out=270,in=90, latex-}
]
{
\Bx_1 \cdot \lr{ \Bx_3 \cdot \lr{ \Bx_4 \cdot \partial_2 T } }
-\Bx_2 \cdot \lr{ \Bx_3 \cdot \lr{ \Bx_4 \cdot \partial_1 T } }
}{Apply cyclic permutations \index{cyclic permutation}}
\\
&=
\Bx_1 \cdot \lr{ \Bx_2 \cdot \lr{ \Bx_3 \cdot \partial_4 T } }
-\Bx_1 \cdot \lr{ \Bx_2 \cdot \lr{ \Bx_4 \cdot \partial_3 T } }  \\
&\quad +\Bx_3 \cdot \lr{ \Bx_4 \cdot \lr{ \Bx_1 \cdot \partial_2 T } }
-\Bx_3 \cdot \lr{ \Bx_4 \cdot \lr{ \Bx_2 \cdot \partial_1 T } }  \\
&=
\Bx_1 \cdot \lr{ \Bx_2 \cdot
\lr{
\Bx_3 \cdot \partial_4 T
-\Bx_4 \cdot \partial_3 T
} }  \\
&\quad +\Bx_3 \cdot \lr{ \Bx_4 \cdot \lr{
\Bx_1 \cdot \partial_2 T
-\Bx_2 \cdot \partial_1 T
} }  \\
&=
\Bx_1 \cdot \lr{ \Bx_2 \cdot
\lr{
\partial_4 \lr{ \Bx_3 \cdot T }
-\partial_3 \lr{ \Bx_4 \cdot T }
} }  \\
&\quad +\Bx_3 \cdot \lr{ \Bx_4 \cdot \lr{
\partial_2 \lr{ \Bx_1 \cdot T }
-\partial_1 \lr{ \Bx_2 \cdot T }
} }  \\
&=
\Bx_1 \cdot \partial_4 \lr{ \Bx_2 \cdot \lr{ \Bx_3 \cdot T } }
+\Bx_2 \cdot \partial_3 \lr{ \Bx_1 \cdot \lr{ \Bx_4 \cdot T } } \\
&\quad +\Bx_3 \cdot \partial_2 \lr{ \Bx_4 \cdot \lr{ \Bx_1 \cdot T } }
+\Bx_4 \cdot \partial_1 \lr{ \Bx_3 \cdot \lr{ \Bx_2 \cdot T } } \\
&
-\Bx_1 \cdot \lr{ \lr{ \partial_4 \Bx_2} \cdot \lr{ \Bx_3 \cdot T } }
-\Bx_2 \cdot \lr{ \lr{ \partial_3 \Bx_1} \cdot \lr{ \Bx_4 \cdot T } }  \\
&\quad -\Bx_3 \cdot \lr{ \lr{ \partial_2 \Bx_4} \cdot \lr{ \Bx_1 \cdot T } }
-\Bx_4 \cdot \lr{ \lr{ \partial_1 \Bx_3} \cdot \lr{ \Bx_2 \cdot T } }  \\
&=
\Bx_1 \cdot \partial_4 \lr{ \Bx_2 \cdot \lr{ \Bx_3 \cdot T } }
+\Bx_2 \cdot \partial_3 \lr{ \Bx_1 \cdot \lr{ \Bx_4 \cdot T } } \\
&\quad +\Bx_3 \cdot \partial_2 \lr{ \Bx_4 \cdot \lr{ \Bx_1 \cdot T } }
+\Bx_4 \cdot \partial_1 \lr{ \Bx_3 \cdot \lr{ \Bx_2 \cdot T } } \\
&
+
\frac{\partial^2 \Bx}{\partial u^4 \partial u^2}
\cdot
\cancel{
\lr{
\Bx_1 \cdot \lr{ \Bx_3 \cdot T }
+\Bx_3 \cdot \lr{ \Bx_1 \cdot T }
}
} \\
&\quad +
\frac{\partial^2 \Bx}{\partial u^1 \partial u^3}
\cdot
\cancel{
\lr{
\Bx_2 \cdot \lr{ \Bx_4 \cdot T }
+\Bx_4 \cdot \lr{ \Bx_2 \cdot T }
}
}.
\end{aligned}
\end{equation}

We can cancel those last terms using \cref{thm:stokesTheoremGeometricAlgebra:2740}.  Using the same reverse chain rule expansion once more we have

\begin{equation}\label{eqn:stokesTheoremGeometricAlgebra:2800}
\begin{aligned}
\Bx_{123} \cdot \partial_4 T
-&\Bx_{124} \cdot \partial_3 T
+\Bx_{134} \cdot \partial_2 T
-\Bx_{234} \cdot \partial_1 T \\
&=
\partial_4 \lr{ \Bx_1 \cdot \lr{ \Bx_2 \cdot \lr{ \Bx_3 \cdot T } } }
+\partial_3 \lr{ \Bx_2 \cdot \lr{ \Bx_1 \cdot \lr{ \Bx_4 \cdot T } } }
+\partial_2 \lr{ \Bx_3 \cdot \lr{ \Bx_4 \cdot \lr{ \Bx_1 \cdot T } } }
+\partial_1 \lr{ \Bx_4 \cdot \lr{ \Bx_3 \cdot \lr{ \Bx_2 \cdot T } } } \\
&-
\lr{ \partial_4 \Bx_1}
\cdot
\cancel{
\lr{
\Bx_2 \cdot \lr{ \Bx_3 \cdot T }
+\Bx_3 \cdot \lr{ \Bx_2 \cdot T }
}
}
-\lr{ \partial_3 \Bx_2} \cdot
\cancel{
\lr{
\Bx_1 \cdot \lr{ \Bx_4 \cdot T }
\Bx_4 \cdot \lr{ \Bx_1 \cdot T }
}
},
\end{aligned}
\end{equation}

or
\begin{dmath}\label{eqn:stokesTheoremGeometricAlgebra:2820}
d^4 \Bx \cdot \lr{ \boldpartial \wedge T }
=
d^4 u
\biglr{
\partial_4 T_{3 2 1}
+\partial_3 T_{4 1 2}
+\partial_2 T_{1 4 3}
+\partial_1 T_{2 3 4}
}.
\end{dmath}

The final result follows after permuting the indices slightly.

%%%Expanding \(6 \times\) the portion in the braces, we have
%%%
%%%\begin{equation}\label{eqn:stokesTheoremGeometricAlgebra:2540}
%%%\begin{aligned}
%%%6 \biglr{ \Bx_{123} \cdot \partial_4 T
%%%-\Bx_{124} \cdot \partial_3 T
%%%+\Bx_{134} \cdot \partial_2 T
%%%-\Bx_{234} \cdot \partial_1 T
%%%}
%%%&=
%%%\Bx_1 \cdot \lr{ \Bx_2 \cdot \lr{ \Bx_3 \cdot \partial_4 T }}
%%%-\Bx_1 \cdot \lr{ \Bx_3 \cdot \lr{ \Bx_2 \cdot \partial_4 T }} \\
%%%&\quad +\Bx_2 \cdot \lr{ \Bx_3 \cdot \lr{ \Bx_1 \cdot \partial_4 T }}
%%%-\Bx_2 \cdot \lr{ \Bx_1 \cdot \lr{ \Bx_3 \cdot \partial_4 T }} \\
%%%&\quad +\Bx_3 \cdot \lr{ \Bx_1 \cdot \lr{ \Bx_2 \cdot \partial_4 T }}
%%%-\Bx_3 \cdot \lr{ \Bx_2 \cdot \lr{ \Bx_1 \cdot \partial_4 T }} \\
%%%%
%%%&
%%%-\Bx_1 \cdot \lr{ \Bx_2 \cdot \lr{ \Bx_4 \cdot \partial_3 T }}
%%%+\Bx_1 \cdot \lr{ \Bx_4 \cdot \lr{ \Bx_2 \cdot \partial_3 T }} \\
%%%&\quad -\Bx_2 \cdot \lr{ \Bx_4 \cdot \lr{ \Bx_1 \cdot \partial_3 T }}
%%%+\Bx_2 \cdot \lr{ \Bx_1 \cdot \lr{ \Bx_4 \cdot \partial_3 T }} \\
%%%&\quad -\Bx_4 \cdot \lr{ \Bx_1 \cdot \lr{ \Bx_2 \cdot \partial_3 T }}
%%%+\Bx_4 \cdot \lr{ \Bx_2 \cdot \lr{ \Bx_1 \cdot \partial_3 T }} \\
%%%%
%%%&
%%%+ \Bx_1 \cdot \lr{ \Bx_3 \cdot \lr{ \Bx_4 \cdot \partial_2 T }}
%%%- \Bx_1 \cdot \lr{ \Bx_4 \cdot \lr{ \Bx_3 \cdot \partial_2 T }} \\
%%%&\quad +\Bx_3 \cdot \lr{ \Bx_4 \cdot \lr{ \Bx_1 \cdot \partial_2 T }}
%%%-\Bx_3 \cdot \lr{ \Bx_1 \cdot \lr{ \Bx_4 \cdot \partial_2 T }} \\
%%%&\quad +\Bx_4 \cdot \lr{ \Bx_1 \cdot \lr{ \Bx_3 \cdot \partial_2 T }}
%%%-\Bx_4 \cdot \lr{ \Bx_3 \cdot \lr{ \Bx_1 \cdot \partial_2 T }} \\
%%%%
%%%&- \Bx_2 \cdot \lr{ \Bx_3 \cdot \lr{ \Bx_4 \cdot \partial_1 T }}
%%%+\Bx_2 \cdot \lr{ \Bx_4 \cdot \lr{ \Bx_3 \cdot \partial_1 T }} \\
%%%&\quad -\Bx_3 \cdot \lr{ \Bx_4 \cdot \lr{ \Bx_2 \cdot \partial_1 T }}
%%%+\Bx_3 \cdot \lr{ \Bx_2 \cdot \lr{ \Bx_4 \cdot \partial_1 T }} \\
%%%&\quad -\Bx_4 \cdot \lr{ \Bx_2 \cdot \lr{ \Bx_3 \cdot \partial_1 T }}
%%%+\Bx_4 \cdot \lr{ \Bx_3 \cdot \lr{ \Bx_2 \cdot \partial_1 T }}
%%%\end{aligned}
%%%\end{equation}

%%%This can be combined pairwise as before, allowing for the cancelation of asymmetric derivatives of \(\Bx_a\) terms.  That first grouping yields
%%%
%%%\begin{equation}\label{eqn:stokesTheoremGeometricAlgebra:2620}
%%%\begin{aligned}
%%%%6 \biglr{ \Bx_{123} \cdot \partial_4 T
%%%%-\Bx_{124} \cdot \partial_3 T
%%%%+\Bx_{134} \cdot \partial_2 T
%%%%-\Bx_{234} \cdot \partial_1 T
%%%%}
%%%%&=
%%%&\Bx_1 \cdot \lr{ \Bx_2 \cdot \lr{ \partial_{[4} \Bx_{3]} \cdot T }}
%%%+\Bx_1 \cdot \lr{ \Bx_3 \cdot \lr{ \partial_{[2} \Bx_{4]} \cdot T }}
%%%+\Bx_1 \cdot \lr{ \Bx_4 \cdot \lr{ \partial_{[3} \Bx_{2]} \cdot T }} \\
%%%&+\Bx_2 \cdot \lr{ \Bx_1 \cdot \lr{ \partial_{[3} \Bx_{4]} \cdot T }}
%%%+\Bx_2 \cdot \lr{ \Bx_3 \cdot \lr{ \partial_{[4} \Bx_{1]} \cdot T }}
%%%+\Bx_2 \cdot \lr{ \Bx_4 \cdot \lr{ \partial_{[1} \Bx_{3]} \cdot T }} \\
%%%&+\Bx_3 \cdot \lr{ \Bx_1 \cdot \lr{ \partial_{[4} \Bx_{2]} \cdot T }}
%%%+\Bx_3 \cdot \lr{ \Bx_2 \cdot \lr{ \partial_{[1} \Bx_{4]} \cdot T }}
%%%+\Bx_3 \cdot \lr{ \Bx_4 \cdot \lr{ \partial_{[2} \Bx_{1]} \cdot T }} \\
%%%&+\Bx_4 \cdot \lr{ \Bx_1 \cdot \lr{ \partial_{[2} \Bx_{3]} \cdot T }}
%%%+\Bx_4 \cdot \lr{ \Bx_2 \cdot \lr{ \partial_{[3} \Bx_{1]} \cdot T }}
%%%+\Bx_4 \cdot \lr{ \Bx_2 \cdot \lr{ \partial_{[1} \Bx_{2]} \cdot T }}
%%%\end{aligned}
%%%\end{equation}
%%%
%%%FIXME: complete (preferably finding a smarter way).
%Regrouping these
%
%\begin{equation}\label{eqn:stokesTheoremGeometricAlgebra:2620}
%\begin{aligned}
%\Bx_1 \cdot \lr{ \Bx_2 \cdot \lr{ +\partial_4 \Bx_3 \cdot T }}
%\Bx_2 \cdot \lr{ \Bx_1 \cdot \lr{ -\partial_4 \Bx_3 \cdot T }}
%\Bx_1 \cdot \lr{ \Bx_3 \cdot \lr{ -\partial_4 \Bx_2 \cdot T }}
%\Bx_3 \cdot \lr{ \Bx_1 \cdot \lr{ +\partial_4 \Bx_2 \cdot T }}
%\Bx_2 \cdot \lr{ \Bx_3 \cdot \lr{ +\partial_4 \Bx_1 \cdot T }}
%\Bx_3 \cdot \lr{ \Bx_2 \cdot \lr{ -\partial_4 \Bx_1 \cdot T }}
%
%\Bx_1 \cdot \lr{ \Bx_2 \cdot \lr{ -\partial_3 \Bx_4 \cdot T }}
%\Bx_1 \cdot \lr{ \Bx_4 \cdot \lr{ +\partial_3 \Bx_2 \cdot T }}
%\Bx_2 \cdot \lr{ \Bx_1 \cdot \lr{ +\partial_3 \Bx_4 \cdot T }}
%\Bx_4 \cdot \lr{ \Bx_2 \cdot \lr{ +\partial_3 \Bx_1 \cdot T }}
%\Bx_4 \cdot \lr{ \Bx_1 \cdot \lr{ -\partial_3 \Bx_2 \cdot T }}
%\Bx_2 \cdot \lr{ \Bx_4 \cdot \lr{ -\partial_3 \Bx_1 \cdot T }}
%
%\Bx_1 \cdot \lr{ \Bx_3 \cdot \lr{ +\partial_2 \Bx_4 \cdot T }}
%\Bx_1 \cdot \lr{ \Bx_4 \cdot \lr{ -\partial_2 \Bx_3 \cdot T }}
%\Bx_4 \cdot \lr{ \Bx_2 \cdot \lr{ -\partial_2 \Bx_1 \cdot T }}
%\Bx_4 \cdot \lr{ \Bx_1 \cdot \lr{ +\partial_2 \Bx_3 \cdot T }}
%\Bx_3 \cdot \lr{ \Bx_4 \cdot \lr{ +\partial_2 \Bx_1 \cdot T }}
%\Bx_3 \cdot \lr{ \Bx_1 \cdot \lr{ -\partial_2 \Bx_4 \cdot T }}
%
%\Bx_2 \cdot \lr{ \Bx_3 \cdot \lr{ -\partial_1 \Bx_4 \cdot T }}
%\Bx_2 \cdot \lr{ \Bx_4 \cdot \lr{ +\partial_1 \Bx_3 \cdot T }}
%\Bx_3 \cdot \lr{ \Bx_2 \cdot \lr{ +\partial_1 \Bx_4 \cdot T }}
%\Bx_3 \cdot \lr{ \Bx_4 \cdot \lr{ -\partial_1 \Bx_2 \cdot T }}
%\Bx_4 \cdot \lr{ \Bx_2 \cdot \lr{ +\partial_1 \Bx_2 \cdot T }}
%\Bx_4 \cdot \lr{ \Bx_2 \cdot \lr{ -\partial_1 \Bx_3 \cdot T }}
%\end{aligned}
%\end{equation}

} % makeanswer

      %\section{Problems} % already a problems section above
         %
% Copyright � 2016 Peeter Joot.  All Rights Reserved.
% Licenced as described in the file LICENSE under the root directory of this GIT repository.
%
%{
%\input{../blogpost.tex}
%\renewcommand{\basename}{stokesCorollariesGriffiths}
%%\renewcommand{\dirname}{notes/phy1520/}
%\renewcommand{\dirname}{notes/ece1228-electromagnetic-theory/}
%%\newcommand{\dateintitle}{}
%%\newcommand{\keywords}{}
%
%\input{../peeter_prologue_print2.tex}
%
%\usepackage{peeters_layout_exercise}
%\usepackage{peeters_braket}
%\usepackage{peeters_figures}
%\usepackage{siunitx}
%\usepackage{macros_qed}
%\usepackage{txfonts} % \ointclockwise
%
%\beginArtNoToc
%
%\generatetitle{Corollaries to Stokes and Divergence theorems}
%\chapter{Corollaries to Stokes and Divergence theorems}
%
%In \citep{griffiths1999introduction} a few problems are set to prove some variations of Stokes theorem.  He gives some cool tricks to prove each one using just the classic 3D Stokes and divergence theorems.  We can also do them directly from the more general Stokes theorem \( \int d^k \Bx \cdot (\spacegrad \wedge F) = \oint d^{k-1} \Bx \cdot F \).
%
\makeoproblem{Stokes theorem on scalar function.}{problem:stokesCorollariesGriffiths:1}{\citep{griffiths1999introduction} pr. 1.60a}{
Prove
\begin{equation}\label{eqn:stokesCorollariesGriffiths:20}
\int \spacegrad T dV = \oint T d\Ba.
\end{equation}
} % problem
\makeanswer{problem:stokesCorollariesGriffiths:1}{
The direct way to prove this is to apply Stokes theorem
\begin{equation}\label{eqn:stokesCorollariesGriffiths:80}
\int d^3 \Bx \cdot (\spacegrad \wedge T) = \oint d^2 \Bx \cdot T
\end{equation}

Here \( d^3 \Bx = d\Bx_1 \wedge d\Bx_2 \wedge d\Bx_3 \), a pseudoscalar (trivector) volume element, and the wedge and dot products take their most general meanings.  For \(k\)-blade \( F \), and \(k'\)-blade \( F' \), that is
\begin{equation}\label{eqn:stokesCorollariesGriffiths:100}
\begin{aligned}
F \wedge F' &= \gpgrade{F F'}{k+k'} \\
F \cdot F' &= \gpgrade{F F'}{\Abs{k-k'}}
\end{aligned}
\end{equation}
With \( d^3\Bx = I dV \), and \( d^2 \Bx = I \ncap dA = I d\Ba \), we have
\begin{equation}\label{eqn:stokesCorollariesGriffiths:120}
\int I dV \spacegrad T = \oint I d\Ba T.
\end{equation}
Cancelling the factors of \( I \) proves the result.

Griffith's trick to do this was to let \( \Bv = \Bc T \), where \( \Bc \) is a constant.  For this, the divergence theorem integral is
\begin{equation}\label{eqn:stokesCorollariesGriffiths:160}
\begin{aligned}
\int dV \spacegrad \cdot (\Bc T)
&=
\int dV \Bc \cdot \spacegrad T \\
&=
\Bc \cdot \int dV \spacegrad T \\
&=
\oint d\Ba \cdot (\Bc T) \\
&=
\Bc \cdot \oint d\Ba T.
\end{aligned}
\end{equation}

This is true for any constant \( \Bc \), so is also true for the unit vectors.  This allows for summing projections in each of the unit directions
\begin{equation}\label{eqn:stokesCorollariesGriffiths:180}
\begin{aligned}
\int dV \spacegrad T
&=
\sum \Be_k \lr{ \Be_k \cdot \int dV \spacegrad T } \\
&=
\sum \Be_k \lr{ \Be_k \cdot \oint d\Ba T } \\
&=
\oint d\Ba T.\qedmarker
\end{aligned}
\end{equation}
} % answer
\makeoproblem{}{problem:stokesCorollariesGriffiths:2}{\citep{griffiths1999introduction} pr. 1.60b}{
Prove
\begin{equation}\label{eqn:stokesCorollariesGriffiths:40}
\int \spacegrad \cross \Bv dV = -\oint \Bv \cross d\Ba.
\end{equation}
} % problem
\makeanswer{problem:stokesCorollariesGriffiths:2}{
This also follows directly from the general Stokes theorem
\begin{equation}\label{eqn:stokesCorollariesGriffiths:200}
\int d^3 \Bx \cdot \lr{ \spacegrad \wedge \Bv } = \oint d^2 \Bx \cdot \Bv.
\end{equation}

The volume integrand is
\begin{equation}\label{eqn:stokesCorollariesGriffiths:220}
\begin{aligned}
d^3 \Bx \cdot \lr{ \spacegrad \wedge \Bv }
&=
\gpgradeone{ I dV I \spacegrad \cross \Bv } \\
&=
-dV \spacegrad \cross \Bv,
\end{aligned}
\end{equation}
and the surface integrand is
\begin{equation}\label{eqn:stokesCorollariesGriffiths:240}
\begin{aligned}
d^2 \Bx \cdot \Bv
&=
\gpgradeone{ I d\Ba \Bv } \\
&=
\gpgradeone{ I (d\Ba \wedge \Bv) } \\
&=
I^2 (d\Ba \cross \Bv) \\
&=
-d\Ba \cross \Bv \\
&=
\Bv \cross d\Ba.
\end{aligned}
\end{equation}

Plugging these into \cref{eqn:stokesCorollariesGriffiths:200} proves the result.

Griffiths trick for the same is to apply the divergence theorem to \( \Bv \cross \Bc \).  Such a volume integral is
\begin{equation}\label{eqn:stokesCorollariesGriffiths:260}
\begin{aligned}
\int dV \spacegrad \cdot (\Bv \cross \Bc)
&=
\int dV \Bc \cdot (\spacegrad \cross \Bv) \\
&=
\Bc \cdot \int dV \spacegrad \cross \Bv.
\end{aligned}
\end{equation}

This must equal
\begin{equation}\label{eqn:stokesCorollariesGriffiths:280}
\begin{aligned}
\oint d\Ba \cdot (\Bv \cross \Bc)
&=
\Bc \cdot \oint d\Ba \cross \Bv \\
&=
-\Bc \cdot \oint \Bv \cross d\Ba
\end{aligned}
\end{equation}

Again, assembling projections, we have
\begin{equation}\label{eqn:stokesCorollariesGriffiths:300}
\begin{aligned}
\int dV \spacegrad \cross \Bv
&=
\sum \Be_k \lr{ \Be_k \cdot \int dV \spacegrad \cross \Bv } \\
&=
-\sum \Be_k \lr{ \Be_k \cdot \oint \Bv \cross d\Ba } \\
&=
-\oint \Bv \cross d\Ba.
\end{aligned}
\end{equation}
} % answer
\makeoproblem{}{problem:stokesCorollariesGriffiths:3}{\citep{griffiths1999introduction} pr. 1.60e}{
Prove
\begin{equation}\label{eqn:stokesCorollariesGriffiths:60}
\int \spacegrad T \cross d\Ba = -\ointctrclockwise T d\Bl.
\end{equation}
} % problem
\makeanswer{problem:stokesCorollariesGriffiths:3}{
This one follows from
\begin{equation}\label{eqn:stokesCorollariesGriffiths:320}
\int d^2 \Bx \cdot \lr{ \spacegrad \wedge T } = \oint d^1 \Bx \cdot T.
\end{equation}

The surface integrand can be written
\begin{equation}\label{eqn:stokesCorollariesGriffiths:340}
\begin{aligned}
d^2 \Bx \cdot \lr{ \spacegrad \wedge T }
&=
\gpgradeone{ I d\Ba \spacegrad T } \\
&=
I (d\Ba \wedge \spacegrad T ) \\
&=
I^2 ( d\Ba \cross \spacegrad T ) \\
&=
-d\Ba \cross \spacegrad T.
\end{aligned}
\end{equation}

The line integrand is
\begin{equation}\label{eqn:stokesCorollariesGriffiths:360}
d^1 \Bx \cdot T = d^1 \Bx T.
\end{equation}

Given a two parameter representation of the surface area element \( d^2 \Bx = d\Bx_1 \wedge d\Bx_2 \), the line element representation is
\begin{equation}\label{eqn:stokesCorollariesGriffiths:380}
\begin{aligned}
d^1 \Bx
&= (\Bx_1 \wedge d\Bx_2) \cdot \Bx^1 + (d\Bx_1 \wedge \Bx_2) \cdot \Bx^2 \\
&= -d\Bx_2 + d\Bx_1,
\end{aligned}
\end{equation}
giving
\begin{equation}\label{eqn:stokesCorollariesGriffiths:400}
\begin{aligned}
-\int d\Ba \cross \spacegrad T
&=
\int
-\evalbar{\lr{ \PD{u_2}{\Bx} T }}{\Delta u_1} du_2
+\evalbar{\lr{ \PD{u_1}{\Bx} T }}{\Delta u_2} du_1 \\
&=
\ointclockwise d\Bl T,
\end{aligned}
\end{equation}
or
\begin{equation}\label{eqn:stokesCorollariesGriffiths:420}
\int \spacegrad T \cross d\Ba
=
-\ointctrclockwise d\Bl T. \qedmarker
\end{equation}

Griffiths trick for the same is to use \( \Bv = \Bc T \) for constant \( \Bc \) in (the usual 3D) Stokes' theorem.  That is
\begin{equation}\label{eqn:stokesCorollariesGriffiths:440}
\begin{aligned}
\int d\Ba \cdot (\spacegrad \cross (\Bc T))
%=
%\int d\Ba \cdot (\Bc \cross \spacegrad T)
&=
\Bc \cdot \int d\Ba \cross \spacegrad T \\
&=
-\Bc \cdot \int \spacegrad T \cross d\Ba \\
&=
\ointctrclockwise d\Bl \cdot (\Bc T) \\
&=
\Bc \cdot \ointctrclockwise d\Bl T.
\end{aligned}
\end{equation}
} % answer

Again assembling projections we have
\begin{equation}\label{eqn:stokesCorollariesGriffiths:460}
\begin{aligned}
\int \spacegrad T \cross d\Ba
&=
\sum \Be_k \lr{ \Be_k \cdot \int \spacegrad T \cross d\Ba} \\
&=
-\sum \Be_k \lr{ \Be_k \cdot  \ointctrclockwise d\Bl T } \\
&=
-\ointctrclockwise d\Bl T.\qedmarker
\end{aligned}
\end{equation}
%It's a bit more work to translate the algebraic enumeration \( d^1 \Bx \)
%}
%\EndArticle

   \chapter{Fundamental Theorem of Geometric Calculus}
      \section{Fundamental Theorem of Geometric Calculus}
         %
% Copyright � 2016 Peeter Joot.  All Rights Reserved.
% Licenced as described in the file LICENSE under the root directory of this GIT repository.
%
%{
%\input{../blogpost.tex}
%\renewcommand{\basename}{fundamentalTheoremOfCalculus}
%\renewcommand{\dirname}{notes/phy1520/}
%%\newcommand{\dateintitle}{}
%%\newcommand{\keywords}{}
%
%\input{../peeter_prologue_print2.tex}
%
%\usepackage{peeters_layout_exercise}
%\usepackage{peeters_braket}
%\usepackage{peeters_figures}
%\usepackage{siunitx}
%
%\beginArtNoToc
%
%\generatetitle{Fundamental theorem of geometric calculus}
%\label{chap:fundamentalTheoremOfCalculus}


\paragraph{Definitions}

The Fundamental Theorem of (Geometric) Calculus is a generalization of Stokes theorem \cref{thm:stokesTheoremGeometricAlgebra:1740} to multivector integrals.  Notationally, it looks like Stokes theorem with all the dot and wedge products removed.

\maketheorem{Fundamental Theorem of Geometric Calculus}{thm:fundamentalTheoremOfCalculus:1}{
For blades \(F, G \), and a volume element \(d^k \Bx\),
%
\begin{equation*}
\int_V F d^k \Bx \boldpartial G = \oint_{\partial V} F d^{k-1} \Bx G.
\end{equation*}
}

The Fundamental Theorem as stated here generalized Stokes' theorem in a couple ways.  One generalization is that the blade grade restriction of Stokes' theorem are missing in action.  This theorem is a multivector result, whereas the LHS and RHS of Stokes' are of equal grade.  Additional, the fundamental theorem promotes the action of the vector derivative to a bidirectional operator.  Not all formulations of this theorem include that generalization.  For example \citep{aMacdonaldVAGC} states the theorem as
\begin{equation}\label{eqn:fundamentalTheoremOfCalculus:180}
\int_V d^k \Bx \boldpartial F = \oint_{\partial V} d^{k-1} \Bx F.
\end{equation}
whereas \citep{doran2003gap} points out there is a variant with the vector derivative acting to the left
\begin{equation}\label{eqn:fundamentalTheoremOfCalculus:200}
\int_V F d^k \Bx \boldpartial = \oint_{\partial V} F d^{k-1} \Bx,
\end{equation}
but does not promote the vector derivative to a bidirectional operator.  Here I follow \citep{sobczyk2011fundamental} which uses a bidirectional vector derivative, providing the most general expression of the Fundamental Theorem of (Geometric) Calculus.

This theorem and Stokes' theorem, both single formulas, are loaded and abstract statement, requiring many definitions to make them useful.
Most of those definitions can be found in the previous Stokes' theorem chapter, but are summarized here

\begin{itemize}
\item The volume integral is over a \(m\) dimensional surface (manifold).
\item Integration over the boundary of the manifold \(V\) is indicated by \( \partial V \).
\item This manifold is assumed to be spanned by a parameterized vector \( \Bx(u^1, u^2, \cdots, u^k) \).
\item A curvilinear coordinate basis \( \setlr{ \Bx_i } \) can be defined on the manifold by
\begin{equation}\label{eqn:fundamentalTheoremOfCalculus:40}
\Bx_i \equiv \PD{u^i}{\Bx} \equiv \partial_i \Bx.
\end{equation}
\item A dual basis \( \setlr{\Bx^i} \) reciprocal to the tangent vector basis \( \Bx_i \) can be calculated subject to the requirement \( \Bx_i \cdot \Bx^j = \delta_i^j \).
\item The vector derivative \(\boldpartial\), the projection of the gradient onto the tangent space of the manifold, is defined by
\begin{equation}\label{eqn:fundamentalTheoremOfCalculus:100}
\boldpartial = \Bx^i \partial_i = \sum_{i=1}^k \Bx_i \PD{u^i}{}.
\end{equation}
\item The volume element is defined by
\begin{equation}\label{eqn:fundamentalTheoremOfCalculus:60}
d^k \Bx = d\Bx_1 \wedge d\Bx_2 \cdots \wedge d\Bx_k,
\end{equation}
where
\begin{equation}\label{eqn:fundamentalTheoremOfCalculus:80}
d\Bx_k = \Bx_k du^k,\qquad \text{(no sum)}.
\end{equation}
\item The volume element is non-zero on the manifold, or \( \Bx_1 \wedge \cdots \wedge \Bx_k \ne 0 \).
\item The surface area element \( d^{k-1} \Bx \), is defined by
\begin{equation}\label{eqn:fundamentalTheoremOfCalculus:120}
%d^{k-1} \Bx = \sum_{i = 1}^k (-1)^{k-i} d\Bx_1 \wedge d\Bx_2 \cdots \wyhat{d\Bx_i} \cdots \wedge d\Bx_k,
d^{k-1} \Bx = \sum_{i = 1}^k (-1)^{k-i} d\Bx_1 \wedge d\Bx_2 \cdots \wedge d\Bx_{i-1} \wedge d\Bx_{i+1} \wedge \cdots \wedge d\Bx_k.
\end{equation}

%where \( \wyhat{d\Bx_i} \) indicates the omission of \( d\Bx_i \).
\item My proof for this theorem was restricted to a simple ``rectangular'' volume parameterized by the ranges
   \(
   [u^1(0), u^1(1) ] \directproduct
   [u^2(0), u^2(1) ] \directproduct \cdots \directproduct
   [u^k(0), u^k(1) ] \)
\item The precise meaning that should be given to oriented area integral is
\begin{equation}\label{eqn:fundamentalTheoremOfCalculus:140}
\oint_{\partial V} d^{k-1} \Bx \cdot F
=
\sum_{i = 1}^k (-1)^{k-i} \int \evalrange{
   %\lr{ \lr{ d\Bx_1 \wedge d\Bx_2 \cdots \wyhat{d\Bx_i} \cdots \wedge d\Bx_k } \cdot F }
   \Biglr{ \lr{ d\Bx_1 \wedge d\Bx_2 \cdots \wedge d\Bx_{i-1} \wedge d\Bx_{i+1} \wedge \cdots \wedge d\Bx_k } \cdot F }
}{u^i = u^i(0)}{u^i(1)},
\end{equation}
where both the a area form and the blade \( F \) are evaluated at the end points of the parameterization range.
\item
The bidirectional vector derivative operator acts both to the left and right on \( F \) and \( G \).  The specific action of this operator is
\begin{equation}\label{eqn:fundamentalTheoremOfCalculus:240}
\begin{aligned}
F \boldpartial G
&=
(F \boldpartial) G
+
F (\boldpartial G) \\
&=
(\partial_i F) \Bx^i G
+
F \Bx^i (\partial_i G).
\end{aligned}
\end{equation}
\end{itemize}

For Stokes' theorem \cref{thm:stokesTheoremGeometricAlgebra:1740}, after the work of stating exactly what is meant by this theorem, most of the proof follows from the fact that for \( s < k \) the volume curl dot product can be expanded as
\begin{equation}\label{eqn:fundamentalTheoremOfCalculus:160}
\begin{aligned}
\int_V d^k \Bx \cdot (\boldpartial \wedge F) 
&= \int_V d^k \Bx \cdot (\Bx^i \wedge \partial_i F)  \\
&= \int_V \lr{ d^k \Bx \cdot \Bx^i } \cdot \partial_i F.
\end{aligned}
\end{equation}

Each of the \(du^i\) integrals can be evaluated directly, since each of the remaining \(d\Bx_j = du^j \PDi{u^j}{}, i \ne j \) is calculated with \( u^i \) held fixed.  This allows for the integration over a ``rectangular'' parameterization region, proving the theorem for such a volume parameterization.  A more general proof requires a triangulation of the volume and surface, but the basic principle of the theorem is evident, without that additional work.
\paragraph{Fundamental Theorem of Calculus}
The fundamental theorem can be demonstrated by direct expansion.  With the vector derivative \( \boldpartial \) and its partials \( \partial_i \) acting bidirectionally, that is
\begin{equation}\label{eqn:fundamentalTheoremOfCalculus:260}
\begin{aligned}
\int_V F d^k \Bx \boldpartial G
&= \int_V F d^k \Bx \Bx^i \partial_i G \\
&= \int_V F \lr{ d^k \Bx \cdot \Bx^i + d^k \Bx \wedge \Bx^i } \partial_i G.
\end{aligned}
\end{equation}

Both the reciprocal frame vectors and the curvilinear basis span the tangent space of the manifold, since we can write any reciprocal frame vector as a set of projections in the curvilinear basis
\begin{equation}\label{eqn:fundamentalTheoremOfCalculus:280}
\Bx^i = \sum_j \lr{ \Bx^i \cdot \Bx^j } \Bx_j,
\end{equation}
so \( \Bx^i \in \Span \setlr{ \Bx_j, j \in [1,k] } \).
This means that \( d^k \Bx \wedge \Bx^i = 0 \), and
%Writing \( d^k u = du^1 du^2 \cdots du^k \), we have
%
%FIXME: bug in latex?  widehat is duplicating the thing that it is covering:
%
\begin{equation}\label{eqn:fundamentalTheoremOfCalculus:300}
\begin{aligned}
\int_V F d^k \Bx \boldpartial G
&=
\int_V F \lr{ d^k \Bx \cdot \Bx^i } \partial_i G \\
&=
\sum_{i = 1}^{k}
\int_V
%du^1 du^2 \cdots \wyhat{ du^i} \cdots du^k
du^1 du^2 \cdots du^{i-1} du^{i+1} \cdots du^k
F \lr{
(-1)^{k-i}
 \Bx_1 \wedge \Bx_2 \cdots \wedge \Bx_{i-1} \wedge \Bx_{i+1} \wedge \cdots \wedge \Bx_k } \partial_i G du^i
 %\Bx_1 \wedge \Bx_2 \cdots \wyhat{\Bx_i} \cdots \wedge \Bx_k } \partial_i G du^i
 \\
 &=
\sum_{i = 1}^{k}
(-1)^{k-i}
\int_{u^1}
\int_{u^2}
\cdots
\int_{u^{i-1}}
\int_{u^{i+1}}
\cdots
\int_{u^k}
\evalrange{ \Biglr{
%F d\Bx_1 \wedge d\Bx_2 \cdots \wyhat{d\Bx_i} \cdots \wedge d\Bx_k G
F d\Bx_1 \wedge d\Bx_2 \cdots \wedge d\Bx_{i-1} \wedge d\Bx_{i+1} \wedge \cdots \wedge d\Bx_k G
}
}{u^i = u^i(0)}{u^i(1)}.
\end{aligned}
\end{equation}
Adding in the same notational sugar that we used in Stokes theorem, this proves the Fundamental theorem
%\eqnref{eqn:fundamentalTheoremOfCalculus:220}
for ``rectangular'' parameterizations.  Note that such a parameterization need not actually be rectangular.
%}
%\EndArticle

         %
% Copyright © 2016 Peeter Joot.  All Rights Reserved.
% Licenced as described in the file LICENSE under the root directory of this GIT repository.
%

\makeexample{Application to Maxwell's equation}{example:fundamentalTheoremOfCalculus:1}{

Maxwell's equation is an example of a first order gradient equation

\begin{dmath}\label{eqn:fundamentalTheoremOfCalculus:320}
\grad F = \inv{\epsilon_0 c} J.
\end{dmath}

Integrating over a four-volume (where the vector derivative equals the gradient), and applying the Fundamental theorem, we have

\begin{dmath}\label{eqn:fundamentalTheoremOfCalculus:340}
\inv{\epsilon_0 c} \int d^4 x J = \oint d^3 x F.
\end{dmath}

Observe that the surface area element product with \( F \) has both vector and trivector terms.  This can be demonstrated by considering some examples

\begin{dmath}\label{eqn:fundamentalTheoremOfCalculus:360}
\begin{aligned}
\gamma_{012} \gamma_{01} &\propto \gamma_2 \\
\gamma_{012} \gamma_{23} &\propto \gamma_{023}.
\end{aligned}
\end{dmath}

On the other hand, the four volume integral of \( J \) has only trivector parts.  This means that the integral can be split into a pair of same-grade equations

\begin{dmath}\label{eqn:fundamentalTheoremOfCalculus:380}
\begin{aligned}
\inv{\epsilon_0 c} \int d^4 x \cdot J &=
\oint \gpgradethree{ d^3 x F} \\
0 &=
\oint d^3 x \cdot F.
\end{aligned}
\end{dmath}

The first can be put into a slightly tidier form using a duality transformation
\begin{dmath}\label{eqn:fundamentalTheoremOfCalculus:400}
\gpgradethree{ d^3 x F}
=
-\gpgradethree{ d^3 x I^2 F}
=
\gpgradethree{ I d^3 x I F}
=
(I d^3 x) \wedge (I F).
\end{dmath}

Letting \( n \Abs{d^3 x} = I d^3 x \), this gives

\begin{dmath}\label{eqn:fundamentalTheoremOfCalculus:420}
\oint \Abs{d^3 x} n \wedge (I F) = \inv{\epsilon_0 c} \int d^4 x \cdot J.
\end{dmath}

Note that this normal is normal to a three-volume subspace of the spacetime volume.  For example, if one component of that spacetime surface area element is \( \gamma_{012} c dt dx dy \), then the normal to that area component is \( \gamma_3 \).

A second set of duality transformations

\begin{dmath}\label{eqn:fundamentalTheoremOfCalculus:440}
n \wedge (IF)
=
\gpgradethree{ n I F}
=
-\gpgradethree{ I n F}
=
-\gpgradethree{ I (n \cdot F)}
=
-I (n \cdot F),
\end{dmath}

and
\begin{dmath}\label{eqn:fundamentalTheoremOfCalculus:460}
I d^4 x \cdot J
=
\gpgradeone{ I d^4 x \cdot J }
=
\gpgradeone{ I d^4 x J }
=
\gpgradeone{ (I d^4 x) J }
=
(I d^4 x) J,
\end{dmath}

can further tidy things up, leaving us with

%\begin{dmath}\label{eqn:fundamentalTheoremOfCalculus:500}
\boxedEquation{eqn:fundamentalTheoremOfCalculus:500}{
\begin{aligned}
\oint \Abs{d^3 x} n \cdot F &= \inv{\epsilon_0 c} \int (I d^4 x) J \\
\oint d^3 x \cdot F &= 0.
\end{aligned}
}
%\end{dmath}

The Fundamental theorem of calculus immediately provides relations between normal projections of the Faraday bivector \( F \) and the four-current \( J \), as well as boundary value constraints on \( F \) coming from the source free components of Maxwell's equation.
} % example

      \section{Green's function for the gradient in Euclidean spaces.}
         %
% Copyright � 2016 Peeter Joot.  All Rights Reserved.
% Licenced as described in the file LICENSE under the root directory of this GIT repository.
%
%{
%\input{../blogpost.tex}
%\renewcommand{\basename}{gradientGreensFunction}
%\renewcommand{\dirname}{notes/ece1228-electromagnetic-theory/}
%%\newcommand{\dateintitle}{}
%%\newcommand{\keywords}{}
%
%\input{../peeter_prologue_print2.tex}
%
%\usepackage{peeters_layout_exercise}
%\usepackage{peeters_braket}
%\usepackage{peeters_figures}
%\usepackage{siunitx}
%
%\beginArtNoToc
%
%\generatetitle{Green's function for the gradient in Euclidean spaces.}
%\label{chap:gradientGreensFunction}

In \citep{doran2003gap} it is stated that the Green's function for the gradient is

\begin{dmath}\label{eqn:gradientGreensFunction:20}
   G(x, x') = \inv{S_n} \frac{x - x'}{\Abs{x-x'}^n},
\end{dmath}

where \( n \) is the dimension of the space, \( S_n \) is the area of the unit sphere, and
\begin{equation}\label{eqn:gradientGreensFunction:40}
   \grad G = \grad \cdot G = \delta(x - x').
\end{equation}

What I'd like to do here is verify that this Green's function operates as asserted.
Here, as in some parts of the text, I am following a convention where vectors are written without boldface.

Let's start with checking that the gradient of the Green's function is zero everywhere that \( x \ne x' \)

\begin{dmath}\label{eqn:gradientGreensFunction:100}
\spacegrad \inv{\Abs{x - x'}^n}
=
-\frac{n}{2} \frac{e^\nu \partial_\nu (x_\mu - x_\mu')(x^\mu - {x^\mu}')}{\Abs{x - x'}^{n+2}}
=
-\frac{n}{2} 2 \frac{e^\nu (x_\mu - x_\mu') \delta_\nu^\mu }{\Abs{x - x'}^{n+2}}
=
-n \frac{ x - x'}{\Abs{x - x'}^{n+2}}.
\end{dmath}

This means that we have, everywhere that \( x \ne x' \)

\begin{dmath}\label{eqn:gradientGreensFunction:120}
\spacegrad \cdot G
=
\inv{S_n} \lr{ \frac{\spacegrad \cdot \lr{x - x'}}{\Abs{x - x'}^{n}} + \lr{ \spacegrad \inv{\Abs{x - x'}^{n}} } \cdot \lr{ x - x'} }
=
\inv{S_n} \lr{ \frac{n}{\Abs{x - x'}^{n}} + \lr{ -n \frac{x - x'}{\Abs{x - x'}^{n+2} } \cdot \lr{ x - x'} } }
= 0.
\end{dmath}

Next, consider the curl of the Green's function.
Zero curl will mean that we have \( \grad G = \grad \cdot G = G \lgrad \).

\begin{dmath}\label{eqn:gradientGreensFunction:140}
S_n (\grad \wedge G)
=
\frac{\grad \wedge (x-x')}{\Abs{x - x'}^{n}}
+
\grad \inv{\Abs{x - x'}^{n}} \wedge (x-x')
=
\frac{\grad \wedge (x-x')}{\Abs{x - x'}^{n}}
- \cancel{n
\frac{x - x'}{\Abs{x - x'}^{n}} \wedge (x-x')}.
\end{dmath}

However,

\begin{dmath}\label{eqn:gradientGreensFunction:160}
\grad \wedge (x-x')
=
\grad \wedge x
=
e^\mu \wedge e_\nu \partial_\mu x^\nu
=
e^\mu \wedge e_\nu \delta_\mu^\nu
=
e^\mu \wedge e_\mu.
\end{dmath}

For any metric where \( e_\mu \propto e^\mu \), which is the case in all the ones with physical interest (i.e. \R{3} and Minkowski space), \( \grad \wedge G \) is zero.

Having shown that the gradient of the (presumed) Green's function is zero everywhere that \( x \ne x' \), the guts of the
demonstration can now proceed.  We wish to evaluate the gradient weighted convolution of the Green's function using the Fundamental Theorem of (Geometric) Calculus.  Here the gradient acts bidirectionally on both the gradient and the test function.  Working in primed coordinates so that the final result is in terms of the unprimed, we have

\begin{dmath}\label{eqn:gradientGreensFunction:60}
   \int_V G(x,x') d^n x' \lrgrad' F(x')
   = \int_{\partial V} G(x,x') d^{n-1} x' F(x').
\end{dmath}

Let \( d^n x' = dV' I \), \( d^{n-1} x' n = dA' I \), where \( n = n(x') \) is the outward normal to the area element \( d^{n-1} x' \).
From this point on, lets restrict attention to Euclidean spaces, where \( n^2 = 1 \).  In that case

\begin{dmath}\label{eqn:gradientGreensFunction:80}
\int_V dV' G(x,x') \lrgrad' F(x')
=
\int_V dV' \lr{G(x,x') \lgrad'} F(x')
+
\int_V dV' G(x,x') \lr{ \rgrad' F(x') }
= \int_{\partial V} dA' G(x,x') n F(x').
\end{dmath}

Here, the pseudoscalar \( I \) has been factored out by commuting it with \( G \), using \( G I = (-1)^{n-1} I G \), and then pre-multiplication with \( 1/((-1)^{n-1} I ) \).

Each of these integrals can be considered in sequence.  A convergence bound is required of the multivector test function \( F(x') \) on the infinite surface \( \partial V \).  Since it's true that

\begin{dmath}\label{eqn:gradientGreensFunction:180}
\Abs{ \int_{\partial V} dA' G(x,x') n F(x') }
\ge
\int_{\partial V} dA' \Abs{ G(x,x') n F(x') },
\end{dmath}

then it is sufficient to require that

\begin{dmath}\label{eqn:gradientGreensFunction:200}
\lim_{x' \rightarrow \infty} \Abs{ \frac{x -x'}{\Abs{x - x'}^n} n(x') F(x') } \rightarrow 0,
\end{dmath}

in order to kill off the surface integral.  Evaluating the integral on a hypersphere centred on \( x \) where \( x' - x = n \Abs{x - x'} \), that is

\begin{dmath}\label{eqn:gradientGreensFunction:260}
\lim_{x' \rightarrow \infty} \frac{ \Abs{F(x')}}{\Abs{x - x'}^{n-1}} \rightarrow 0.
\end{dmath}

Given such a constraint, that leaves

\begin{dmath}\label{eqn:gradientGreensFunction:220}
\int_V dV' \lr{G(x,x') \lgrad'} F(x')
=
-\int_V dV' G(x,x') \lr{ \rgrad' F(x') }.
\end{dmath}

The LHS is zero everywhere that \( x \ne x' \) so it can be restricted to a spherical ball around \( x \), which allows the test function \( F \) to be pulled out of the integral, and a second application of the Fundamental Theorem to be applied.

\begin{dmath}\label{eqn:gradientGreensFunction:240}
\int_V dV' \lr{G(x,x') \lgrad'} F(x')
=
\lim_{\epsilon \rightarrow 0}
\int_{\Abs{x - x'} < \epsilon} dV' \lr{G(x,x') \lgrad'} F(x')
=
\lr{ \lim_{\epsilon \rightarrow 0}
I^{-1} \int_{\Abs{x - x'} < \epsilon} I dV' \lr{G(x,x') \lgrad'}
} F(x)
=
\lr{ \lim_{\epsilon \rightarrow 0}
(-1)^{n-1} I^{-1} \int_{\Abs{x - x'} < \epsilon} G(x,x') d^n x' \lgrad'
} F(x)
=
\lr{ \lim_{\epsilon \rightarrow 0}
(-1)^{n-1} I^{-1} \int_{\Abs{x - x'} = \epsilon} G(x,x') d^{n-1} x'
} F(x)
=
\lr{ \lim_{\epsilon \rightarrow 0}
(-1)^{n-1} I^{-1} \int_{\Abs{x - x'} = \epsilon} G(x,x') dA' I n
} F(x)
=
\lr{ \lim_{\epsilon \rightarrow 0}
\int_{\Abs{x - x'} = \epsilon} dA' G(x,x') n
} F(x)
=
\lr{ \lim_{\epsilon \rightarrow 0}
\int_{\Abs{x - x'} = \epsilon} dA' \frac{\epsilon (-n)}{S_n \epsilon^n} n
} F(x)
=
-\lim_{\epsilon \rightarrow 0}
\frac{F(x)}{S_n \epsilon^{n-1}}
\int_{\Abs{x - x'} = \epsilon} dA'
=
-\lim_{\epsilon \rightarrow 0}
\frac{F(x)}{S_n \epsilon^{n-1}}
S_n \epsilon^{n-1}
=
-F(x).
\end{dmath}

This essentially calculates the divergence integral around an infinitesimal hypersphere, without assuming that the gradient commutes with the gradient in this infinitesimal region.  So, provided the test function is constrained by \cref{eqn:gradientGreensFunction:260}, we have

\begin{dmath}\label{eqn:gradientGreensFunction:280}
F(x) = \int_V dV' G(x,x') \lr{ \grad' F(x') }.
\end{dmath}

In particular, should we have a first order gradient equation

\begin{dmath}\label{eqn:gradientGreensFunction:300}
\spacegrad' F(x') = M(x'),
\end{dmath}

the inverse of this equation is given by
%\begin{dmath}\label{eqn:gradientGreensFunction:320}
\boxedEquation{eqn:gradientGreensFunction:320}{
F(x) = \int_V dV' G(x,x') M(x').
}
%\end{dmath}

Note that the sign of the Green's function is explicitly tied to the definition of the convolution integral that is used.
%There is a slightly annoying negative sign in this convolution integral.
This is important since since the conventions for the sign of the Green's function or the parameters in the convolution integral often vary.

%For this definition, it seems desirable to eliminate that negation by variable substitution
%
%\begin{dmath}\label{eqn:gradientGreensFunction:340}
%F(y)
%= -\int_V dV' G(y,x') M(x')
%= -\int_V dV G(y,x) M(x),
%\end{dmath}
%
%so the inverse of \( \grad F = M \) can be written as
%
%%\begin{dmath}\label{eqn:gradientGreensFunction:360}
%\boxedEquation{eqn:gradientGreensFunction:360}{
%F(y)
%= \int_V dV G(x, y) M(x).
%}
%\end{dmath}

What's cool about this result is that it applies not only to gradient equations in Euclidean spaces, but also to multivector (or even just vector) fields \( F \), instead of the usual scalar functions that we usually apply Green's functions to.

%}
%\EndArticle

         %
% Copyright © 2016 Peeter Joot.  All Rights Reserved.
% Licenced as described in the file LICENSE under the root directory of this GIT repository.
%
\makeexample{Electrostatics}{example:gradientGreensFunction:1}{
As a check of the sign consider the electrostatics equation

\begin{dmath}\label{eqn:gradientGreensFunction:380}
\spacegrad \BE = \frac{\rho}{\epsilon_0},
\end{dmath}

for which we have after substitution into \cref{eqn:gradientGreensFunction:320}
\begin{dmath}\label{eqn:gradientGreensFunction:400}
\BE(\Bx) = \inv{4 \pi \epsilon_0} \int_V dV' \frac{\Bx - \Bx'}{\Abs{\Bx - \Bx'}^3} \rho(\Bx').
\end{dmath}

This matches the sign found in a trusted reference such as \citep{jackson1975cew}.
} % example


      % example:
         %
% Copyright � 2016 Peeter Joot.  All Rights Reserved.
% Licenced as described in the file LICENSE under the root directory of this GIT repository.
%
%{
%\input{../blogpost.tex}
%\renewcommand{\basename}{biotSavartGreens}
%%\renewcommand{\dirname}{notes/phy1520/}
%\renewcommand{\dirname}{notes/ece1228-electromagnetic-theory/}
%%\newcommand{\dateintitle}{}
%%\newcommand{\keywords}{}
%
%\input{../peeter_prologue_print2.tex}
%
%\usepackage{peeters_layout_exercise}
%\usepackage{peeters_braket}
%\usepackage{peeters_figures}
%\usepackage{siunitx}
%
%\beginArtNoToc
%
%\generatetitle{Green's function inversion of magnetostatic equation}
%\chapter{Green's function inversion of magnetostatic equation}
%\label{chap:biotSavartGreens}
% \citep{sakurai2014modern} pr X.Y
% \citep{pozar2009microwave}
% \citep{qftLectureNotes}
% \citep{doran2003gap}
% \citep{jackson1975cew}
% \citep{griffiths1999introduction}

\makeexample{Magnetostatics.}{example:biotSavartGreens:1}{

The magnetostatics equation in linear media has the Geometric Algebra form
%A previous example of inverting a gradient equation was the electrostatics equation.  We can do the same for the magnetostatics equation, which has the following Geometric Algebra form in linear media

\begin{dmath}\label{eqn:biotSavartGreens:20}
\spacegrad I \BB = - \mu \BJ.
\end{dmath}

The Green's inversion of this is
\begin{dmath}\label{eqn:biotSavartGreens:40}
I \BB(\Bx)
= \int_V dV' G(\Bx, \Bx') \spacegrad' I \BB(\Bx')
= \gpgradeone{ \int_V dV' G(\Bx, \Bx') \spacegrad' I \BB(\Bx') }
= \int_V dV' \gpgradeone{ G(\Bx, \Bx') (-\mu \BJ(\Bx')) }
= \inv{4\pi} \int_V dV' \frac{\Bx - \Bx'}{ \Abs{\Bx - \Bx'}^3 } \wedge (-\mu \BJ(\Bx'))
= \frac{\mu}{4\pi} \int_V dV' \BJ(\Bx') \wedge \frac{\Bx - \Bx'}{ \Abs{\Bx - \Bx'}^3 }.
\end{dmath}

A duality transformation can be used to obtain the usual cross product form of the Biot-Savart law if desired.

Note that freedom to insert a no-op vector grade selection was utilized to simplify the calculation above.
It can be demonstrated that the scalar component of this integral is explicitly zero with some of the usual trickery
\begin{dmath}\label{eqn:biotSavartGreens:60}
-\frac{\mu}{4\pi} \int_V dV' \frac{\Bx - \Bx'}{ \Abs{\Bx - \Bx'}^3 } \cdot \BJ(\Bx')
= \frac{\mu}{4\pi} \int_V dV' \lr{ \spacegrad \inv{ \Abs{\Bx - \Bx'} }} \cdot \BJ(\Bx')
= -\frac{\mu}{4\pi} \int_V dV' \lr{ \spacegrad' \inv{ \Abs{\Bx - \Bx'} }} \cdot \BJ(\Bx')
= -\frac{\mu}{4\pi} \int_V dV' \lr{
\spacegrad' \cdot \frac{\BJ(\Bx')}{ \Abs{\Bx - \Bx'} }
-
\frac{\spacegrad' \cdot \BJ(\Bx')}{ \Abs{\Bx - \Bx'} }
}.
\end{dmath}

By premultiplying \cref{eqn:biotSavartGreens:20} by the gradient, we have

\begin{dmath}\label{eqn:biotSavartGreens:80}
\spacegrad^2 I \BB = -\mu \spacegrad \BJ,
\end{dmath}

showing that the current \( \BJ \) is not unconstrained.  In particular, since
the LHS is a bivector, the gradient of the current must also be a bivector
\( \spacegrad \BJ = \spacegrad \wedge \BJ \),
or equivaalently the divergence of the current must be zero 
\( \spacegrad \cdot \BJ = 0 \).  This kills the \( \spacegrad' \cdot \BJ(\Bx') \) integrand numerator in \cref{eqn:biotSavartGreens:60}, leaving

\begin{dmath}\label{eqn:biotSavartGreens:100}
-\frac{\mu}{4\pi} \int_V dV' \frac{\Bx - \Bx'}{ \Abs{\Bx - \Bx'}^3 } \cdot \BJ(\Bx')
= -\frac{\mu}{4\pi} \int_V dV' \spacegrad' \cdot \frac{\BJ(\Bx')}{ \Abs{\Bx - \Bx'} }
= -\frac{\mu}{4\pi} \int_{\partial V} dA' \ncap \cdot \frac{\BJ(\Bx')}{ \Abs{\Bx - \Bx'} }.
\end{dmath}

Provided the normal component of \( \BJ/\Abs{\Bx - \Bx'} \) vanishes on the boundary of the infinite sphere, we see that the 
the scalar selection of the convolution integral is zero, justifying the vector selection operation.
%Observe that the traditional vector form of the Biot-Savart law can be obtained by premultiplying both sides with \( -I \), leaving
%
%\begin{dmath}\label{eqn:biotSavartGreens:140}
%\BB(\Bx)
%= \frac{\mu}{4\pi} \int_V dV' \BJ(\Bx') \cross \frac{\Bx - \Bx'}{ \Abs{\Bx - \Bx'}^3 }.
%\end{dmath}
%
%This checks against a trusted source such as \citep{griffiths1999introduction} (eq. 5.39).
} % example

%}
%\EndArticle

      \section{Problems}
         % convert to example?
         %
% Copyright � 2016 Peeter Joot.  All Rights Reserved.
% Licenced as described in the file LICENSE under the root directory of this GIT repository.
%
%{
%\input{../blogpost.tex}
%\renewcommand{\basename}{helmholtzDerviationMultivector}
%%\renewcommand{\dirname}{notes/phy1520/}
%\renewcommand{\dirname}{notes/ece1228-electromagnetic-theory/}
%%\newcommand{\dateintitle}{}
%%\newcommand{\keywords}{}
%
%\input{../peeter_prologue_print2.tex}
%
%\usepackage{peeters_layout_exercise}
%\usepackage{peeters_braket}
%\usepackage{peeters_figures}
%\usepackage{siunitx}
%
%\beginArtNoToc
%
%\generatetitle{Helmholtz theorem}
%\chapter{Helmholtz theorem}
%\label{chap:helmholtzDerviationMultivector}
% \citep{sakurai2014modern} pr X.Y
% \citep{pozar2009microwave}
% \citep{qftLectureNotes}
% \citep{doran2003gap}
% \citep{jackson1975cew}
% \citep{griffiths1999introduction}

%\section{Appendix IV.  2nd Geometric Algebra solution to problem 5.}

%This is a problem from ece1228.  I attempted solutions in a number of ways.  One using Geometric Algebra, one devoid of that algebra, and then this method, which combined aspects of both.  Of the three methods I tried to obtain this result, this is the most compact and elegant.  It does however, require a fair bit of Geometric Algebra knowledge, including the Fundamental Theorem of Geometric Calculus, as detailed in \citep{doran2003gap}, \citep{sobczyk2011fundamental} and \citep{aMacdonaldVAGC}.

\makeproblem{Helmholtz theorem}{problem:helmholtzTakeIII}{
Prove the first Helmholtz's theorem.
%
% Copyright © 2016 Peeter Joot.  All Rights Reserved.
% Licenced as described in the file LICENSE under the root directory of this GIT repository.
%
\maketheorem{Helmholtz first theorem.}{thm:helmholtzDerviationMultivectorStatement:1}{
If vector \(\BM\) is defined by its divergence
\begin{equation}\label{eqn:helmholtzDerviationMultivector:20}
\spacegrad \cdot \BM = s
\end{equation}
and its curl
\begin{equation}\label{eqn:helmholtzDerviationMultivector:40}
\spacegrad \cross \BM = \BC
\end{equation}
within a region and its normal component \( \BM_{\txtn} \) over the boundary, then \( \BM \) is uniquely specified.
} % theorem

%Note: Assume there is a vector \( \BN \) with its divergence and curl equal to \( s \) and \( \BC \) respectively, then show that \( \BM = \BN \) .
} % makeproblem

\makeanswer{problem:helmholtzTakeIII}{
%%I attempted this problem in two different ways.  My first approach assembled the divergence and curl relations above into a single (Geometric Algebra) multivector gradient equation and applied the vector valued Green's function for the gradient to invert that equation.  That approach logically led from the differential equation for \( \BM \) to the solution for \( \BM \) in terms of \( s \) and \( \BC \).  However, this strategy introduced some complexities
%%that make me doubt the correctness of the associated boundary analysis.
%%
%%Even if the details of the boundary handling in my multivector approach is not correct, I thought that approach was interesting enough to share, and have placed it in an appendix to this problem set.  It is accompanied with a primer on Geometric Algebra that is hopefully enough to allow the reader to grasp the basic ideas of the approach, but is probably not sufficient to understand all the details without further study.
%%
%%The answer obtained in that first attempt at this problem, when \( \Abs{\BM}/r^2 \), and \( \Abs{\BC}/r \) both vanish on an infinite sphere, is that the field has a unique value determined by \( s \) and \( \BC \), namely
%
%\begin{equation}\label{eqn:helmholtzDerviationMultivector:60}
%\BM =
%-\spacegrad \int_V dV' \frac{ s(\Bx')}{ 4 \pi \Abs{\Bx - \Bx'} }
%+\spacegrad \cross \int_V dV' \frac{ \BC(\Bx') }{ 4 \pi \Abs{\Bx - \Bx'} }.
%\end{equation}
%
%It's possible to work backwards from this result to obtain second order gradient terms applied to \( \BM(\Bx')/\Abs{\Bx - \Bx'} \) .  This suggests that a Laplacian (i.e. scalar) representation of the delta function may be a superior way to tackle this problem, perhaps also yielding a simpler result for the boundary term.  This is in fact the case, and the logical starting point is a convolution representation of the vector function \( \BM \)
%

%
% Copyright © 2016 Peeter Joot.  All Rights Reserved.
% Licenced as described in the file LICENSE under the root directory of this GIT repository.
%
The gradient of the vector \( \BM \) can be written as a single even grade multivector

\begin{equation}\label{eqn:helmholtzDerviationMultivector:60}
\spacegrad \BM
= \spacegrad \cdot \BM + I \spacegrad \cross \BM
= s + I \BC.
\end{equation}

%Observe that the Laplacian of \( \BM \) is vector valued
%
%\begin{dmath}\label{eqn:helmholtzDerviationMultivector:760}
%\spacegrad^2 \BM = \spacegrad s + I \spacegrad \BC.
%\end{dmath}
%
%This means that \( \spacegrad \BC \) must be a bivector \( \spacegrad \BC = \spacegrad \wedge \BC \), or that \( \BC \) has zero divergence
%
%\begin{dmath}\label{eqn:helmholtzDerviationMultivector:780}
%\spacegrad \cdot \BC = 0.
%\end{dmath}

This can be used to attempt to discover the relation between the vector \( \BM \) and its divergence and curl.  
The vector \( \BM \) can be expressed at the point of interest as a convolution with the delta function at all other points in space

\begin{dmath}\label{eqn:helmholtzDerviationMultivector:80}
\BM(\Bx) = \int_V dV' \delta(\Bx - \Bx') \BM(\Bx').
\end{dmath}

The Laplacian representation of the delta function in \R{3} is

\begin{dmath}\label{eqn:helmholtzDerviationMultivector:100}
\delta(\Bx - \Bx') = -\inv{4\pi} \spacegrad^2 \inv{\Abs{\Bx - \Bx'}},
\end{dmath}

so \( \BM \) can be represented as the following convolution

\begin{dmath}\label{eqn:helmholtzDerviationMultivector:120}
\BM(\Bx) = -\inv{4\pi} \int_V dV' \spacegrad^2 \inv{\Abs{\Bx - \Bx'}} \BM(\Bx').
\end{dmath}

%As noted in \cref{eqn:helmholtzDerviationMultivector:460} the Laplacian of a vector can be factored as
%
%\begin{dmath}\label{eqn:helmholtzDerviationMultivector:140}
%\spacegrad^2 \Ba
%=
%\spacegrad (\spacegrad \cdot \Ba)
%-
%\spacegrad \cross (\spacegrad \cross \Ba).
%\end{dmath}
%
%Note that the last term can be written in cross product notation using \( \Bc \cdot (\Ba \wedge \Bb) = -\Bc \cross (\Ba \cross \Bb) \) if desired.

Using this relation and proceeding with a few applications of the chain rule, plus the fact that \( \spacegrad 1/\Abs{\Bx - \Bx'} = -\spacegrad' 1/\Abs{\Bx - \Bx'} \), we find
%
%I previously posted a Geometric Algebra attack on the Helmholtz theorem.  Here is
%
%Here's a third way of deriving the Helmholtz theorem inversion relation.  This is a refinement of the traditional vector algebra solution that led to \cref{eqn:helmholtzDerviationMultivector:200}, that uses a factorization of the Laplacian directly, deferring any expansion in terms of dot and cross (or wedge) products until the very end.
%
%Starting from the first line of \cref{eqn:helmholtzDerviationMultivector:160}, we have

\begin{dmath}\label{eqn:helmholtzDerviationMultivector:720}
-4 \pi \BM(\Bx)
= \int_V dV' \spacegrad^2 \inv{\Abs{\Bx - \Bx'}} \BM(\Bx')
= \gpgradeone{\int_V dV' \spacegrad^2 \inv{\Abs{\Bx - \Bx'}} \BM(\Bx')}
= -\gpgradeone{\int_V dV' \spacegrad \lr{ \spacegrad' \inv{\Abs{\Bx - \Bx'}}} \BM(\Bx')}
= -\gpgradeone{\spacegrad \int_V dV' \lr{
\spacegrad' \frac{\BM(\Bx')}{\Abs{\Bx - \Bx'}}
-\frac{\spacegrad' \BM(\Bx')}{\Abs{\Bx - \Bx'}}
} }
=
-\gpgradeone{\spacegrad \int_{\partial V} dA'
\ncap \frac{\BM(\Bx')}{\Abs{\Bx - \Bx'}}
 }
+\gpgradeone{\spacegrad \int_V dV'
\frac{s(\Bx') + I\BC(\Bx')}{\Abs{\Bx - \Bx'}}
 }
=
-\gpgradeone{\spacegrad \int_{\partial V} dA'
\ncap \frac{\BM(\Bx')}{\Abs{\Bx - \Bx'}}
 }
+\spacegrad \int_V dV'
\frac{s(\Bx')}{\Abs{\Bx - \Bx'}}
+\spacegrad \cdot \int_V dV'
\frac{I\BC(\Bx')}{\Abs{\Bx - \Bx'}}.
\end{dmath}

By inserting a no-op grade selection operation in the second step, the trivector terms that would show up in subsequent steps are automatically filtered out.
%the troublesome trivector term that shows up in my first purely Geometric Algebra
%attempt is eliminated.
This leaves us with a boundary term dependent on the surface and the normal and tangential components of \( \BM \).  Added to that is a pair of volume integrals that provide the unique dependence of \( \BM \) on its divergence and curl.
When the surface is taken to infinity, which requires \( \Abs{\BM}/\Abs{\Bx - \Bx'} \rightarrow 0 \), then the dependence of \( \BM \) on its divergence and curl is unique.

In order to express final result in traditional vector algebra form, a couple transformations are required.  The first is that

\begin{equation}\label{eqn:helmholtzDerviationMultivector:800}
\gpgradeone{ \Ba I \Bb } = I^2 \Ba \cross \Bb = -\Ba \cross \Bb.
\end{equation}

For the grade selection in the boundary integral, note that

\begin{dmath}\label{eqn:helmholtzDerviationMultivector:740}
\gpgradeone{ \spacegrad \ncap \BX }
=
\gpgradeone{ \spacegrad (\ncap \cdot \BX) }
+
\gpgradeone{ \spacegrad (\ncap \wedge \BX) }
=
\spacegrad (\ncap \cdot \BX)
+
\gpgradeone{ \spacegrad I (\ncap \cross \BX) }
=
\spacegrad (\ncap \cdot \BX)
-
\spacegrad \cross (\ncap \cross \BX).
\end{dmath}

These give

%\begin{dmath}\label{eqn:helmholtzDerviationMultivector:721}
\boxedEquation{eqn:helmholtzDerviationMultivector:721}{
\begin{aligned}
\BM(\Bx)
&=
\spacegrad \inv{4\pi} \int_{\partial V} dA' \ncap \cdot \frac{\BM(\Bx')}{\Abs{\Bx - \Bx'}}
-
\spacegrad \cross \inv{4\pi} \int_{\partial V} dA' \ncap \cross \frac{\BM(\Bx')}{\Abs{\Bx - \Bx'}} \\
&-\spacegrad \inv{4\pi} \int_V dV'
\frac{s(\Bx')}{\Abs{\Bx - \Bx'}}
+\spacegrad \cross \inv{4\pi} \int_V dV'
\frac{\BC(\Bx')}{\Abs{\Bx - \Bx'}}.
\end{aligned}
}
%\end{dmath}


%we recover the non-boundary integral of \cref{eqn:helmholtzDerviationMultivector:200}.  The boundary term is seen to have a particularly simple form using this technique.  Note that the dot and double cross product expression obtained with the vector algebra approach can be recovered from this directly if desired using an expansion of the following form

%Using this expansion in \cref{eqn:helmholtzDerviationMultivector:720} recovers \cref{eqn:helmholtzDerviationMultivector:200}.
%Of the three methods I tried to obtain this result, this is the most compact and elegant of all three solution attempts.
%, but also requires full knowledge of the Geometric Algebra toolbox to understand.
}

%\EndArticle

%\paragraph{Future thought.}
%
%Does this Green's function also work for mixed metric spaces?  If so, in such a metric, what does it mean to
%calculate the surface area of a unit sphere in a mixed signature space?

\part{General Physics}
   \chapter{Angular Velocity and Acceleration (Again)}\label{chap:PJAngAcc}
      %
% Copyright � 2012 Peeter Joot.  All Rights Reserved.
% Licenced as described in the file LICENSE under the root directory of this GIT repository.
%

%
%
\chapter{Angular Velocity and Acceleration (Again)}\label{chap:PJAngAcc}
\index{angular velocity}
\index{angular acceleration}
%\date{June 10, 2008.  angularAcc.tex}

A more coherent derivation of angular velocity and acceleration than
my initial attempt while first learning geometric algebra.

\section{Angular velocity}

The goal is to take first and second derivatives of a vector expressed radially:

\begin{equation}
\Br = r \rcap.
\end{equation}

The velocity is the derivative of our position vector, which in terms of radial components is:

\begin{equation}\label{eqn:angular_acc:velocityasrcapprime}
\Bv = \Br' = r' \rcap + r \rcap'.
\end{equation}

We can also calculate the projection and rejection of the velocity by multiplication by \(1 = \rcap^2\), and expanding
this product in an alternate order taking advantage of the associativity of the geometric product:

\begin{equation}\label{eqn:angularAcc:380}
\begin{aligned}
\Bv &= \rcap \rcap \Bv \\
    &= \rcap \left ( \rcap \cdot \Bv + \rcap \wedge \Bv \right) \\
\end{aligned}
\end{equation}

Since \(\rcap \wedge (\rcap \wedge \Bv) = 0\), the total velocity in terms of radial components is:

\begin{equation}\label{eqn:angular_acc:velocityprojrej}
\Bv = \rcap \left(\rcap \cdot \Bv\right) + \rcap \cdot \left(\rcap \wedge \Bv\right).
\end{equation}

Here the first component above is the projection of the vector in the radial direction:

\begin{equation}\label{eqn:angularAcc:20}
\Proj_{\Br}(\Bv) = \rcap \left(\rcap \cdot \Bv\right)
\end{equation}

This projective term can also be rewritten in terms of magnitude:

\begin{equation}\label{eqn:angularAcc:40}
\left(r^2\right)' = 2 r r' = \left(\Br \cdot \Br\right)' = 2 \Br \cdot \Bv.
\end{equation}

So the magnitude variation can be expressed the radial coordinate of the velocity:

\begin{equation}\label{eqn:angular_acc:rprime}
r' = \rcap \cdot \Bv
\end{equation}

The remainder is the rejection of the radial component from the velocity, leaving just the part
portion perpendicular to the radial direction.

\begin{equation}\label{eqn:angularAcc:60}
\RejName_{\Br}(\Bv) = \rcap \cdot \left(\rcap \wedge \Bv\right)
\end{equation}

It is traditional to introduce an angular velocity vector normal to the plane of rotation
that describes this rejective component using a triple cross product.  With the formulation above,
one can see that it is more natural to directly use an angular velocity bivector instead:

\begin{equation}
\BOmega = \frac{\Br \wedge \Bv}{r^2}
\end{equation}

This bivector encodes the
angular velocity as a plane directly.  The
product of a vector with the bivector that contains it produces another vector
in the plane.  That product is a scaled and rotated by 90 degrees, much like the
multiplication by a unit complex imaginary.  That is no coincidence since
the square of a bivector is negative and directly encodes this complex structure
of an arbitrarily oriented plane.

Using this angular velocity bivector we have the following radial expression for velocity:

\begin{equation}\label{eqn:angular_acc:velocityomega}
\Bv = \rcap r' + \Br \cdot \BOmega.
\end{equation}

A little thought will show that \(\rcap'\) is also entirely perpendicular to \(\rcap\).  The \(\rcap\) vector describes
a path traced out on the unit sphere, and any variation of that vector must be tangential to the sphere.
It is thus not surprising that we can also express \(\rcap'\) using the rejective term of equation
\eqnref{eqn:angular_acc:velocityprojrej}.  Using the angular velocity bivector this is:

\begin{equation}\label{eqn:angular_acc:rcapprime}
\rcap' = \rcap \cdot \BOmega.
\end{equation}

This identity will be useful below for the calculation of angular acceleration.

\section{Angular acceleration}

Next we want the second derivatives of position

\begin{equation}\label{eqn:angularAcc:400}
\begin{aligned}
\Ba
&= \Br'' \\
&= r'' \rcap + 2r' \rcap' + r \rcap'' \\
&= r'' \rcap + \inv{r}\left( r^2 \rcap' \right)' \\
\end{aligned}
\end{equation}

This last step I found scribbled in a margin note in
my old mechanics book.  It is a trick that somebody clever once noticed and it simplifies this derivation to use it
since it avoids the generation of a number of terms that will just cancel out anyways after more tedious manipulation
(see examples section).

Expanding just this last derivative:

\begin{equation}\label{eqn:angularAcc:420}
\begin{aligned}
\left( r^2 \rcap' \right)'
&= \left( r^2 \rcap \cdot \BOmega \right)' \\
&= \left( \rcap \cdot \left(\Br \wedge \Bv\right) \right)' \\
&= \left( \rcap \cdot \left(\Br \wedge \Bv\right) \right)' \\
&= \rcap' \cdot \left(\Br \wedge \Bv\right) +\rcap \cdot (\mathLabelBox{\Bv \wedge \Bv}{\(=0\)}) + \rcap \cdot \left(\Br \wedge \Ba\right) \\
\end{aligned}
\end{equation}

Thus the acceleration is:
\begin{equation*}
\Ba = r'' \rcap + \left(\Br \cdot \BOmega\right) \cdot \BOmega + \rcap \cdot \left(\rcap \wedge \Ba\right)
\end{equation*}


Note that the action of taking two successive dot products with the plane bivector \(\BOmega\) just acts to rotate the
vector by 180 degrees (as well as scale it).

One can verify this explicitly using grade selection operators.  This allows the total acceleration to be expressed
in the final form:

\begin{equation*}
\Ba = r'' \rcap + \Br \BOmega^2 + \rcap \cdot \left(\rcap \wedge \Ba\right)
\end{equation*}

Note that the squared bivector \(\BOmega^2\) is a negative scalar, so the first two terms are radially directed.
The last term is perpendicular to the acceleration, in the plane formed by the vector and its second derivative.

Given the acceleration, the force on a particle is thus:

\begin{equation}\label{eqn:angularAcc:80}
\BF = m\Ba = m\rcap r'' + m \Br \BOmega^2 + \frac{\Br}{r^2} \left(\Br \wedge \Bp\right)'
\end{equation}

Writing the angular momentum as:

\begin{equation}\label{eqn:angularAcc:100}
\BL = \Br \wedge \Bp = m r^2 \BOmega
\end{equation}
%m \BOmega^2 = \BL^2/m r^4

the force is thus:

\begin{equation}\label{eqn:angularAcc:120}
\BF = m\Ba = m\rcap r'' + m \Br \BOmega^2 + \frac{1}{\Br} \cdot \frac{d\BL}{dt}
\end{equation}

The derivative of the angular momentum is called the torque \(\Btau\), also a bivector:

\begin{equation}\label{eqn:angularAcc:140}
\Btau = \frac{d\BL}{dt}
\end{equation}


When \(\Br\) is constant this has the radial arm times force form that we expect of torque:

\begin{equation}\label{eqn:angularAcc:160}
\Btau = \Br \wedge \frac{d \Bp}{dt} = \Br \wedge \BF
\end{equation}

%We can also write the equation of motion in terms of angular momentum and torque, in which case we have:
%
%\[
%\BF = m\rcap r'' + \inv{m\Br^3} \BL^2 + \frac{1}{\Br} \cdot \Btau
%\]
We can also write the equation of motion in terms of torque, in which case we have:

\begin{equation}\label{eqn:angularAcc:180}
\BF = m\rcap r'' + m \Br \Omega^2 + \frac{1}{\Br} \cdot \Btau
\end{equation}

As with all these plane quantities (angular velocity, momentum, acceleration), the torque as well is a bivector as it is natural to express this as a planar quantity.  This makes
more sense in many ways than a cross product, since all of these quantities should be perfectly well defined in a plane (or in spaces of degree greater than three), whereas the
cross product is a strictly three dimensional entity.

\section{Expressing these using traditional form (cross product)}

To compare with traditional results to see if I got things right, remove the geometric algebra constructs
(wedge products and bivector/vector products) in favor of cross products.  Do this by
using the duality relationships, multiplication by the three dimensional pseudoscalar
\(i = \Be_1\Be_2\Be_3\), to convert bivectors to vectors and wedge products to cross and dot products
(\(\Bu \wedge \Bv = \Bu \cross \Bv i\)).

First define some vector quantities in terms of the corresponding bivectors:

\begin{equation}\label{eqn:angularAcc:200}
\Bomega = \BOmega / i = \frac{\Br \wedge \Bv}{r^2 i} = \frac{\Br \cross \Bv}{r^2}
\end{equation}

\begin{equation}\label{eqn:angularAcc:220}
\Br \cdot \BOmega = \inv{2}(\Br \Bomega i - \Bomega i \Br ) = \Br \wedge \Bomega i = \Bomega \cross \Br
\end{equation}

Thus the velocity is:

\begin{equation}\label{eqn:angularAcc:240}
\Bv = \rcap r' + \Bomega \cross \Br.
\end{equation}

In the same way, write the angular momentum vector as the dual of the angular momentum bivector:

\begin{equation}\label{eqn:angularAcc:260}
\Bl = \BL /i = \Br \cross \Bp = m r^2 \Bomega
\end{equation}

And the torque vector \(\BN\) as the dual of the torque bivector \(\Btau\)

\begin{equation}\label{eqn:angularAcc:280}
\BN = \Btau /i = \frac{d\Bl}{dt} = \frac{d}{dt} \left(\Br \cross \Bp \right)
\end{equation}

The equation of motion for a single particle then becomes:

% r . t = r . N i = 1/2(r N i - N i r) = r ^ N i = r x N i^2 = N x r
\begin{equation}\label{eqn:angularAcc:300}
\BF = m\rcap r'' - m \Br \norm{\Bomega}^2 + \BN \cross \frac{\Br}{r^2}
\end{equation}

\section{Examples (perhaps future exercises?)}

\subsection{Unit vector derivative}
\index{derivative!unit vector}

Demonstrate by direct calculation the result of \eqnref{eqn:angular_acc:rcapprime}.

\begin{equation}\label{eqn:angularAcc:440}
\begin{aligned}
\rcap'
&= \left(\frac{\Br}{r}\right)' \\
&= \frac{\Br'}{r} - \frac{\Br r'}{r^2} \\
&= \inv{r} \left( \Bv - \rcap \left(\rcap \cdot \Bv \right) \right) \\
&= \frac{\rcap}{r} \left( \rcap \Bv - \rcap \cdot \Bv \right) \\
&= \frac{\rcap}{r} \left( \rcap \wedge \Bv \right) \\
\end{aligned}
\end{equation}

\subsection{Direct calculation of acceleration}
\index{acceleration}

It is more natural to calculate this acceleration directly by taking derivatives of \eqnref{eqn:angular_acc:velocityomega}, but as noted above this is messier.  Here is exactly that calculation for
comparison.

Taking second derivatives of the velocity we have:

\begin{equation}\label{eqn:angularAcc:320}
\Bv' = \Ba = \left(\rcap r' + \frac{\Br}{r^2} \left(\Br \wedge \Bv\right)\right)'
\end{equation}

%\rcap' = \inv{r^3} \Br \Br \wedge \Bv
\begin{equation}\label{eqn:angularAcc:460}
\begin{aligned}
\Ba
&= \rcap' r' + \rcap r'' + \frac{\Br}{r^2}
\mathLabelBox
[
   labelstyle={below of=m\themathLableNode, below of=m\themathLableNode}
]
{\left(\Bv \wedge \Bv\right)}{\(=0\)} + \frac{\Br}{r^2} \left(\Br \wedge \Ba\right) + \left(\frac{\rcap}{r}\right)' \left(\Br \wedge \Bv\right) \\
&=
\rcap r''
+\rcap'\left( r'  + \frac{1}{r} \Br \wedge \Bv \right)
- r' \frac{\rcap}{r^2} \left(\Br \wedge \Bv\right)
+ \rcap \left(\rcap \wedge \Ba\right)  \\
&=
\rcap r''
+\inv{r^3} \Br\left( \Br \wedge \Bv\right) \left( r'  + \frac{1}{r} \Br \wedge \Bv \right)
- r' \frac{\rcap}{r^2} \left(\Br \wedge \Bv\right)
+ \rcap \left(\rcap \wedge \Ba\right)  \\
\end{aligned}
\end{equation}

The \(r'\) terms cancel out, leaving just:

\begin{equation}\label{eqn:angularAcc:340}
\Ba = \rcap r'' + \Br \BOmega^2 +
\rcap \left(\rcap \wedge \Ba\right)
\end{equation}

\subsection{Expand the omega omega triple product}

\begin{equation}\label{eqn:angularAcc:480}
\begin{aligned}
\left(\Br \cdot \BOmega\right) \cdot \BOmega
&= \gpgradeone{ \left(\Br \cdot \BOmega\right) \BOmega } \\
&= \inv{2} \gpgradeone{ \Br \BOmega^2 - \BOmega \Br \BOmega } \\
&= \inv{2} \Br \BOmega^2 - \inv{2}\gpgradeone{ \BOmega \Br \BOmega } \\
&= \inv{2} \Br \BOmega^2 + \inv{2}\gpgradeone{ \Br \BOmega \BOmega } \\
&= \inv{2} \Br \BOmega^2 + \inv{2}\Br \BOmega^2 \\
&= \Br \BOmega^2 \\
\end{aligned}
\end{equation}

Also used above implicitly was the following:

\begin{equation}\label{eqn:angularAcc:360}
\Br \BOmega = \Br \cdot \BOmega + \mathLabelBox{\Br \wedge \BOmega}{\(=0\)} = -\BOmega \cdot \Br = -\BOmega \Br
\end{equation}

(ie: a vector anticommutes with a bivector describing a plane that contains it).

   \chapter{Cross product Radial decomposition}\label{chap:PJAngAccCross}
      %
% Copyright � 2012 Peeter Joot.  All Rights Reserved.
% Licenced as described in the file LICENSE under the root directory of this GIT repository.
%

%
%
%\mychapter{Cross product Radial decomposition}
%\label{chap:PJAngAccCross}
%\date{July 8, 2008.  angularAccCross.tex}

We have seen how to use GA constructs to perform a radial
decomposition of a velocity and acceleration vector.  Is it that
much harder to do this with straight vector algebra.  This shows that
the answer is no, but we need to at least assume some additional
identities that can take work to separately prove.  Here is a quick
demonstration for comparision purposes how a radial decomposition
can be performed entirely without any GA usage.

\section{Starting point}

Starting point is taking derivatives of:
%
\begin{equation}\label{eqn:angularAccCross:20}
\Br = r \rcap
\end{equation}
%
\begin{equation}\label{eqn:angularAccCross:40}
\Bv = r' \rcap + r \rcap'
\end{equation}
%
It can be shown without any Geometric Algebra use (see for example \citep{salas1990coa}) that the unit vector derivative can be expressed using the cross product:
%
\begin{equation}\label{eqn:angularAccCross:60}
\rcap' = \inv{r} \left(\rcap \cross \frac{d\Br}{dt}\right) \cross \rcap.
\end{equation}
%
Now, one can express \(r'\) in terms of \(\Br\) as well as follows:
%
\begin{equation}\label{eqn:angularAccCross:80}
\left(\Br \cdot \Br\right)' = 2 \Bv \cdot \Br = 2 r r'.
\end{equation}
%
Thus the derivative of the vector magnitude is part of a projective term:
%
\begin{equation}\label{eqn:angularAccCross:100}
r' = \rcap \cdot \Bv.
\end{equation}
%
Putting this together one has velocity in terms of projective and rejective
components along a radial direction:
%
\begin{equation}\label{eqn:angularAccCross:120}
\Bv = \left(\rcap \cdot \Bv\right) \rcap + \left(\rcap \cross \frac{d\Br}{dt}\right) \cross \rcap.
\end{equation}
%
Now \(\Bomega = \frac{\Br \cross \Bv}{r^2}\) term is what we call the angular velocity.  The magnitude of this
is the rate of change of the angle between the radial arm and the direction of rotation.  The direction of this
cross product is normal to the plane of rotation and encodes both the rotational plane and the direction of the
rotation.  Putting these together one has the total velocity expressed radially:
%
\begin{equation}\label{eqn:angularAccCross:140}
\Bv = \left(\rcap \cdot \Bv\right) \rcap + \Bomega \cross \Br.
\end{equation}
%
\section{Acceleration}

Acceleration follows in the same fashion.
%
\begin{equation}\label{eqn:angularAccCross:160}
\begin{aligned}
\Bv'
&= (
\mathLabelBox{\left(\rcap \cdot \Bv\right) \rcap}{\(r'\rcap\)}
)' + (
\mathLabelBox{\Bomega \cross \Br}{\((\Br \cross \Bv) \cross \frac{\Br}{r^2}\)}
)' \\
&= r'' \rcap
 + r' \frac{\Bomega \cross \Br}{r}
 + (\rcap \cross \Ba) \cross \rcap
 + (\mathLabelBox{\Bv \cross \Bv)}{\(=0\)} \cross \frac{\Br}{r^2}
 + (\Br \cross \Bv) \cross \left(\frac{\Br}{r^2}\right)'
\end{aligned}
\end{equation}
%
That last derivative is
%
\begin{equation}\label{eqn:angularAccCross:180}
\begin{aligned}
\left(\frac{\Br}{r^2}\right)'
&= \left(\frac{\rcap}{r}\right)' \\
&= \frac{\rcap'}{r} - \frac{\rcap r'}{r^2} \\
&= \frac{\Bomega \cross \Br}{r^2} - \frac{\rcap r'}{r^2},
\end{aligned}
\end{equation}
%
and back substitution gives:
\begin{equation}\label{eqn:angularAccCross:200}
\begin{aligned}
\Bv'
&= r'' \rcap
 + r' \frac{\Bomega \cross \Br}{r}
 + (\rcap \cross \Ba) \cross \rcap
 + (\Br \cross \Bv) \cross \left( \frac{\Bomega \cross \Br}{r^2} - \frac{\rcap r'}{r^2} \right).
\end{aligned}
\end{equation}
%
Canceling terms and collecting we have the final result for acceleration expressed radially:
%
\begin{equation}
\Bv' = \Ba = r'' \rcap + \Bomega \cross \left( {\Bomega \cross \Br} \right) + (\rcap \cross \Ba) \cross \rcap
\end{equation}
%
Now, applying the angular velocity via cross product takes the vector back to the original plane, but inverts it.  Thus we can write the acceleration completely in terms of the radially directed components, and the perpendicular component.
%
\begin{equation}
\Ba = r'' \rcap -\Br \omega^2 + (\rcap \cross \Ba) \cross \rcap
\end{equation}
%
An alternate way to express this is in terms of radial scalar acceleration:
%
\begin{equation}
\Ba \cdot \rcap = r'' - r\omega^2.
\end{equation}
%
This is the acceleration analogue of the scalar radial velocity component demonstrated above:
\begin{equation}
\Bv \cdot \rcap = r'.
\end{equation}

   \chapter{Kinetic Energy in rotational frame}\label{chap:PJKeRot}
      %
% Copyright � 2012 Peeter Joot.  All Rights Reserved.
% Licenced as described in the file LICENSE under the root directory of this GIT repository.
%

%
%
\chapter{Kinetic Energy in rotational frame}\label{chap:PJKeRot}
\index{kinetic energy}
\index{rotational frame}
%\date{April 30, 2008.  keRotation.tex}

\section{Motivation}

Fill in the missing details of the rotational Kinetic Energy derivation in
\citep{TongDynamics}
%Tong's classical
%dynamics paper
and contrast matrix and GA approach.

Generalize acceleration in terms
of rotating frame coordinates without unproved extrapolation that the z axis result
of Tong's paper is good unconditionally (his cross products are kind of pulled out of
a magic hat and this write up will show a couple ways to see where they come from).

Given coordinates for a point in a rotating frame \(\Br'\), the coordinate vector for that point
in a rest frame is:

\begin{equation}\label{eqn:keRot:rotcoord}
\Br = R \Br'
\end{equation}

Where the rotating frame moves according to the following z-axis rotation matrix:

\begin{equation}\label{eqn:keRotation:20}
R =
\begin{bmatrix}
\cos \theta & -\sin \theta & 0 \\
\sin \theta & \cos \theta & 0 \\
0 & 0 & 1 \\
\end{bmatrix}
\end{equation}

To compute the Lagrangian we want to re-express the
kinetic energy of a particle:

\begin{equation}\label{eqn:keRotation:40}
K =
\inv{2} m \DotT{\Br}^2
\end{equation}

in terms of the rotating frame coordinate system.

\section{With matrix formulation}

The Tong paper does this for a z axis rotation with \(\theta = \omega t\).
Constant angular frequency is assumed.

First we calculate our position vector in terms of the rotational frame

\begin{equation}\label{eqn:keRotation:60}
\Br = R\Br'
\end{equation}

%Where
%
%\[
%R_\theta^{-1} = R_{-\theta} =
%\begin{bmatrix}
%\cos \theta & \sin \theta & 0 \\
%-\sin \theta & \cos \theta & 0 \\
%0 & 0 & 1 \\
%\end{bmatrix}
%\]

The rest frame velocity is:

\begin{equation}\label{eqn:keRotation:80}
\DotT{\Br} = \DotT{R}_{\theta} \Br' + R_{\theta} \DotT{\Br'}.
\end{equation}

Taking the matrix time derivative we have:

\begin{equation}\label{eqn:keRotation:100}
\DotT{R}_{\theta} =
-\DotT{\theta}
\begin{bmatrix}
\sin \theta & \cos \theta & 0 \\
-\cos \theta & \sin \theta & 0 \\
0 & 0 & 0 \\
\end{bmatrix}.
\end{equation}

Taking magnitudes of the velocity we have three terms

\begin{equation}\label{eqn:keRotation:800}
\begin{aligned}
\DotT{\Br}^2
&=
(\DotT{R}_{\theta} \Br') \cdot (\DotT{R}_{\theta} \Br')
+2 (\DotT{R}_{\theta} \Br') \cdot (R_{\theta} \DotT{\Br'})
+(R_{\theta} \DotT{\Br'}) \cdot (R_{\theta} \DotT{\Br'}) \\
&=
\transpose{\Br'}\transpose{\DotT{R}_{\theta}} \DotT{R}_{\theta} \Br'
+2 \transpose{\Br'} \transpose{\DotT{R}_{\theta}} R_{\theta} \DotT{\Br'}
+\DotT{\Br'}^2 \\
\end{aligned}
\end{equation}

We need to calculate all the intermediate matrix products.  The last was
identity, and the first is:

\begin{equation}\label{eqn:keRotation:120}
\transpose{\DotT{R}_{\theta}} \DotT{R}_{\theta}
=
{\DotT{\theta}}^2
\begin{bmatrix}
\sin \theta & -\cos \theta & 0 \\
\cos \theta & \sin \theta & 0 \\
0 & 0 & 0 \\
\end{bmatrix}
\begin{bmatrix}
\sin \theta & \cos \theta & 0 \\
-\cos \theta & \sin \theta & 0 \\
0 & 0 & 0 \\
\end{bmatrix}
\end{equation}
\begin{equation}\label{eqn:keRotation:140}
=
{\DotT{\theta}}^2
\begin{bmatrix}
1 & 0 & 0 \\
0 & 1 & 0 \\
0 & 0 & 0 \\
\end{bmatrix}
\end{equation}

This leaves just the mixed term

\begin{equation}\label{eqn:keRotation:160}
\transpose{\DotT{R}_{\theta}} {R_{\theta}}
=
-{\DotT{\theta}}
\begin{bmatrix}
\sin \theta & -\cos \theta & 0 \\
\cos \theta & \sin \theta & 0 \\
0 & 0 & 0 \\
\end{bmatrix}
\begin{bmatrix}
\cos \theta & -\sin \theta & 0 \\
\sin \theta & \cos \theta & 0 \\
0 & 0 & 1 \\
\end{bmatrix}
\end{equation}
\begin{equation}\label{eqn:keRotation:180}
=
-{\DotT{\theta}}
\begin{bmatrix}
0 & -1 & 0 \\
1 & 0 & 0 \\
0 & 0 & 0 \\
\end{bmatrix}
\end{equation}

With \(\DotT{\theta} = \omega\), the total magnitude of the velocity is thus

\begin{equation}\label{eqn:keRotation:200}
\DotT{\Br}^2 =
\transpose{\Br'}
\omega^2
\begin{bmatrix}
1 & 0 & 0 \\
0 & 1 & 0 \\
0 & 0 & 0 \\
\end{bmatrix}
\Br'
-2 \omega \transpose{{\Br'}}
\begin{bmatrix}
0 & -1 & 0 \\
1 & 0 & 0 \\
0 & 0 & 0 \\
\end{bmatrix}
\DotT{\Br'}
+ {\DotT{\Br'}}^2
\end{equation}

Tong's paper presents this expanded out in terms of coordinates:

\begin{equation}\label{eqn:keRotation:220}
\DotT{\Br}^2 =
\omega^2\left( {x'}^{2} + {y'}^{2} \right)
+ 2 \omega \left( x' \DotT{y'} -y' \DotT{x'} \right)
+ \left( \DotT{x'}^{2} + \DotT{y'}^{2} + \DotT{z'}^{2} \right)
\end{equation}

Or,
\begin{equation}\label{eqn:keRot:vmagwithmatrix}
\DotT{\Br}^2 =
\left( -\omega y' + \DotT{x'} \right)^2
+\left( \omega x' + \DotT{y'} \right)^2
+ \DotT{z'}^2
\end{equation}

He also then goes on to
show that this can be written, with \(\Bomega = \omega \zcap\), as

\begin{equation}\label{eqn:keRotation:240}
\DotT{\Br}^2 = ( \DotT{\Br'} + \Bomega \cross \Br')^2
\end{equation}

The implication here is that this is a valid result for any rotating
coordinate system.   How to prove this in the general rotation case, is shown much later
in his treatment of rigid bodies.

\section{With rotor}

The equivalent to \eqnref{eqn:keRot:rotcoord} using a rotor is:

\begin{equation}
\Br' = R^\dagger \Br R
\end{equation}

Where \(R = \exp( i\theta/2 )\).

Unlike the
matrix formulation above we are free to pick any constant unit bivector
for \(i\) if we want to generalize this to any rotational axis, but if we
want an equivalent to the above rotation matrix we just have to take
\(i = \Be_1 \wedge \Be_2\).

We need a double sided inversion to get our unprimed vector:

\begin{equation}\label{eqn:keRotation:260}
\Br = R \Br' R^\dagger
\end{equation}

and can then take derivatives:

\begin{equation}\label{eqn:keRotation:280}
\DotT{\Br} =
\DotT{R} \Br' R^\dagger
+{R} {\Br'} \DotT{R}^\dagger
+{R} \DotT{\Br'} R^\dagger
\end{equation}
\begin{equation}\label{eqn:keRotation:300}
=
i\omega \inv{2} {R} \Br' R^\dagger
- {R} \Br' R^\dagger i\omega\inv{2}
+{R} \DotT{\Br'} R^\dagger
\end{equation}
\begin{equation}\label{eqn:keRot:velocityfixedrotplane}
\implies
\DotT{\Br} = \omega i \cdot ({R} \Br' R^\dagger) +  {R} \DotT{\Br'} R^\dagger
\end{equation}

One can put this into the traditional cross product form by introducing
a normal vector for the rotational axis in the usual way:

\begin{equation}\label{eqn:keRotation:320}
\BOmega = \omega i
\end{equation}
\begin{equation}\label{eqn:keRotation:340}
\Bomega = \BOmega / \BI_3
\end{equation}

We can describe the angular velocity by a scaled normal vector \((\Bomega)\) to the rotational plane, or by a scaled bivector for the plane itself (\(\BOmega\)).

\begin{equation}\label{eqn:keRotation:820}
\begin{aligned}
\BOmega \cdot ({R} \Br' R^\dagger)
&= \gpgradeone{ \BOmega {R} \Br' R^\dagger } \\
&= \gpgradeone{ {R} \BOmega \Br' R^\dagger } \\
&= {R} \BOmega \cdot \Br' R^\dagger \\
&= {R} (\Bomega \BI_3) \cdot \Br' R^\dagger \\
&= {R} (\Bomega \cross \Br') R^\dagger \\
\end{aligned}
\end{equation}

Note that here as before this is valid only when the rotational plane orientation is constant (ie: no wobble), since only then can we assume \(i\), and thus \(\BOmega\) will commute with the rotor \(R\).

Summarizing, we can write our velocity using rotational frame components
as:
\begin{equation}\label{eqn:keRot:vrotcross}
\DotT{\Br} = {R} \left( \Bomega \cross \Br' + \DotT{\Br'} \right) R^\dagger
\end{equation}
Or
\begin{equation}
\DotT{\Br} = {R} \left( \BOmega \cdot \Br' + \DotT{\Br'} \right) R^\dagger
\end{equation}

Using the result above from \eqnref{eqn:keRot:vrotcross}, we can calculate
the squared magnitude directly:

\begin{equation}\label{eqn:keRotation:840}
\begin{aligned}
\DotT{\Br} ^2
&= \gpscalargrade{
{R} \left( \Bomega \cross \Br' + \DotT{\Br'} \right) R^\dagger
{R} \left( \Bomega \cross \Br' + \DotT{\Br'} \right) R^\dagger
} \\
&= \gpscalargrade{
{R} ( \Bomega \cross \Br' + \DotT{\Br'} ) ^2 R^\dagger
} \\
&= ( \Bomega \cross \Br' + \DotT{\Br'} ) ^2 \\
\end{aligned}
\end{equation}

We are able to go straight to the end result this way without the mess
of sine and cosine terms in the rotation matrix.  This is something that
we can expand by components if desired:

\begin{equation}\label{eqn:keRotation:860}
\begin{aligned}
\Bomega \cross \Br' + \DotT{\Br'}
&=
\begin{vmatrix}
\Be_1 & \Be_2 & \Be_3 \\
0 & 0 & \omega \\
x' & y' & z' \\
\end{vmatrix}
+ \DotT{\Br'} \\
&=
\begin{bmatrix}
-\omega y' + \DotT{x'} \\
\omega x' + \DotT{y'} \\
 \DotT{z'} \\
\end{bmatrix}
\end{aligned}
\end{equation}

This verifies the second part of Tong's equation 2.19, also consistent with the
derivation of \eqnref{eqn:keRot:vmagwithmatrix}.

\section{Acceleration in rotating coordinates}

Having calculated velocity in terms of rotational frame coordinates, acceleration is the next
logical step.

The starting point is the velocity

\begin{equation}\label{eqn:keRotation:360}
\DotT{\Br} = R ( \BOmega \cdot \Br' + \DotT{\Br}' ) R^\dagger
\end{equation}

Taking derivatives we have
\begin{equation}\label{eqn:keRotation:380}
\DDotT{\Br} = i \omega /2 \DotT{\Br} - \DotT{\Br} i \omega /2 + R \left( \dot{\BOmega} \cdot \Br' + \BOmega \cdot \DotT{\Br}' + \DDotT{\Br}' \right) R^\dagger
\end{equation}

The first two terms are a bivector vector dot product and we can simplify this as follows

\begin{equation}\label{eqn:keRotation:880}
\begin{aligned}
i \omega /2 \DotT{\Br} - \DotT{\Br} i \omega /2
&= \BOmega /2 \DotT{\Br} - \DotT{\Br} \BOmega \\
&= \BOmega \cdot \DotT{\Br} \\
&= \gpgradeone{ \BOmega R ( \BOmega \cdot \Br' + \DotT{\Br}' ) R^\dagger } \\
&= \gpgradeone{ R ( \BOmega (\BOmega \cdot \Br' + \DotT{\Br}') ) R^\dagger } \\
&= R ( \BOmega \cdot (\BOmega \cdot \Br') + \BOmega \cdot \DotT{\Br}') R^\dagger \\
\end{aligned}
\end{equation}

Thus the total acceleration is

\begin{equation}
\DDotT{\Br} = R \left( \BOmega \cdot (\BOmega \cdot \Br') +\dot{\BOmega} \cdot \Br' + 2 \BOmega \cdot \DotT{\Br}' + \DDotT{\Br}' \right) R^\dagger
\end{equation}

Or, in terms of cross products, and angular velocity and acceleration vectors \(\Bomega\), and \(\Balpha\) respectively, this is

\begin{equation}\label{eqn:keRot:accelerationfixedrotplane}
\DDotT{\Br} = R \left( \Bomega \cross (\Bomega \cross \Br') + \Balpha \cross \Br' + 2 \Bomega \cross \DotT{\Br}' + \DDotT{\Br}' \right) R^\dagger
\end{equation}

\section{Allow for a wobble in rotational plane}

A calculation similar to this can be found in GAFP, but for strictly rigid motion.  It does not take too much to combine the two for a generalized result that
expresses the total acceleration expressed in rotating frame coordinates, but also allowing for general rotation where the frame rotation and the angular velocity
bivector do not have to be coplanar (ie: commute as above).

Since the primes and dots are kind of cumbersome switch to the GAFP notation where the position of a particle is expressed in terns of a rotational component \(\Bx\)
and origin translation \(\Bx_0\):

\begin{equation}\label{eqn:keRotation:400}
\By = R \Bx R^\dagger + \Bx_0
\end{equation}

Taking derivatives for velocity

\begin{equation}\label{eqn:keRot:velocity}
\DotT{\By} = \DotT{R} \Bx R^\dagger +R \Bx \DotT{R^\dagger} +R \DotT{\Bx} R^\dagger + \DotT{\Bx}_0
\end{equation}

Now use the same observation that the derivative of \(R R^\dagger = 1\) is zero:

\begin{equation*}
\frac{d (R R^\dagger)}{dt} = \DotT{R}R^\dagger + R \DotT{R^\dagger} = 0
\end{equation*}
\begin{equation}\label{eqn:keRot:rotdtrot}
\implies
\DotT{R}R^\dagger =
%-R \DotT{R^\dagger} =
 - R \DotT{R}^\dagger = -{\left( \DotT{R} R^\dagger \right)}^\dagger
\end{equation}

Since \(R\) has only grade 0 and 2 terms, so does its derivative.  Thus the product of the two has grade 0, 2, and 4 terms, but
\eqnref{eqn:keRot:rotdtrot} shows that the product \(\DotT{R} R^\dagger\) has a value that is the negative of its reverse, so it must have
only grade 2 terms (the reverse of the grade 0 and 4 terms would not change sign).

As in \eqnref{eqn:keRot:velocityfixedrotplane} we want to write \(\DotT{R}\) as a bivector/rotor product and \eqnref{eqn:keRot:rotdtrot} gives us a means to do so.
This would have been clearer in GAFP if they had done the simple example first with the orientation of the rotational plane fixed.

So, write:

\begin{equation}\label{eqn:keRotation:420}
\DotT{R}R^\dagger = \inv{2}\BOmega
\end{equation}
\begin{equation}\label{eqn:keRotation:440}
\DotT{R} = \inv{2}\BOmega R
\end{equation}
\begin{equation}\label{eqn:keRotation:460}
\DotT{R}^\dagger = - \inv{2} R^\dagger \BOmega
\end{equation}

(including the \(1/2\) here is a bit of a cheat ... it is here because having done the calculation on paper first one sees that it is natural to do so).

With this we can substitute back into \eqnref{eqn:keRot:velocity}, writing \(\By_0 = \By - \Bx_0\) :

\begin{equation}\label{eqn:keRotation:900}
\begin{aligned}
\DotT{\By}
&= \inv{2} \BOmega {R} \Bx R^\dagger - \inv{2} R \Bx {R^\dagger} \BOmega +R \DotT{\Bx} R^\dagger + \DotT{\Bx}_0 \\
&= \inv{2}\left(\BOmega \By_ - \By_0 \BOmega\right) +R \DotT{\Bx} R^\dagger + \DotT{\Bx}_0 \\
&= \BOmega \cdot \By_0 +R \DotT{\Bx} R^\dagger + \DotT{\Bx}_0 \\
\end{aligned}
\end{equation}

We also want to pull in this \(\BOmega\) into the rotor as in the fixed orientation
case, but cannot use commutativity this time since the rotor and angular velocity bivector are not necessarily in the same plane.

This is where GAFP introduces their body angular velocity, which applies an inverse rotation to the angular velocity.

Let:
\begin{equation}\label{eqn:keRotation:480}
\BOmega = R \BOmega_B R^\dagger
\end{equation}

Computing this bivector dot product with \(\By\) we have

\begin{equation}\label{eqn:keRotation:920}
\begin{aligned}
\BOmega \cdot \By_0
&= (R \BOmega_B R^\dagger) \cdot (R \Bx R^\dagger) \\
&= \gpgradeone{R \BOmega_B R^\dagger R \Bx R^\dagger} \\
&= \gpgradeone{R \BOmega_B \Bx R^\dagger} \\
&= \gpgradeone{R (\BOmega_B \cdot \Bx + \BOmega_B \wedge \Bx) R^\dagger} \\
&= {R \BOmega_B \cdot \Bx R^\dagger} \\
\end{aligned}
\end{equation}

Thus the total velocity is:

\begin{equation}
\DotT{\By} = {R (\BOmega_B \cdot \Bx + \DotT{\Bx} )R^\dagger} + \DotT{\Bx}_0
\end{equation}

Thus given any vector \(\Bx\) in the rotating frame coordinate system, we have the relationship for the inertial frame velocity.  We can apply this a second
time to compute the inertial (rest frame) acceleration in terms of rotating coordinates.  Write \(\Bv = \BOmega_B \cdot \Bx + \DotT{\Bx}\),

\begin{equation*}
\DotT{\By} = {R \Bv R^\dagger} + \DotT{\Bx}_0
\end{equation*}
\begin{equation*}
\implies
\DDotT{\By} = {R (\BOmega_B \cdot \Bv + \DotT{\Bv} ) R^\dagger} + \DDotT{\Bx}_0
\end{equation*}

\begin{equation}\label{eqn:keRotation:500}
\DotT{\Bv} =
\DotT{\BOmega}_B \cdot \Bx
+\BOmega_B \cdot \DotT{\Bx}
+ \DDotT{\Bx}
\end{equation}

Combining these we have:
\begin{equation}\label{eqn:keRotation:940}
\begin{aligned}
\DDotT{\By}
&= {R (\BOmega_B \cdot ( \BOmega_B \cdot \Bx + \DotT{\Bx} ) + \DotT{\BOmega}_B \cdot \Bx +\BOmega_B \cdot \DotT{\Bx} + \DDotT{\Bx}) R^\dagger} + \DDotT{\Bx}_0 \\
\end{aligned}
\end{equation}

\begin{equation}
\implies
\DDotT{\By}
= {R (\BOmega_B \cdot ( \BOmega_B \cdot \Bx ) + \DotT{\BOmega}_B \cdot \Bx + 2\BOmega_B \cdot \DotT{\Bx} + \DDotT{\Bx}) R^\dagger} + \DDotT{\Bx}_0
\end{equation}

This generalizes \eqnref{eqn:keRot:accelerationfixedrotplane}, providing the rest frame acceleration in terms of rotational frame coordinates, with centrifugal acceleration, Euler force acceleration, and Coriolis force acceleration terms that accompany the plain old acceleration term \(\DDotT{\Bx}\).  The only
requirement for the generality of allowing the orientation of the rotational plane to potentially vary is the use of the ``body angular velocity''
\(\BOmega_B\), replacing the angular velocity as seen from the rest frame \(\BOmega\).

\subsection{Body angular acceleration in terms of rest frame}
\index{body angular acceleration}
\index{rest frame}

Since we know the relationship between the body angular velocity \(\BOmega_B\) with the Rotor (rest frame) angular velocity bivector, for
completeness, lets compute the body angular acceleration bivector \(\DotT{\BOmega}_B\) in terms of the rest frame angular acceleration \(\DotT{\BOmega}\).

\begin{equation}\label{eqn:keRotation:520}
\BOmega_B = R^\dagger \BOmega R
\end{equation}
\begin{equation}\label{eqn:keRotation:960}
\begin{aligned}
\implies
\DotT{\BOmega}_B
&= \DotT{R}^\dagger \BOmega R +R^\dagger \DotT{\BOmega} R +R^\dagger \BOmega \DotT{R} \\
&= -\inv{2}R^\dagger \BOmega^2 R +R^\dagger \DotT{\BOmega} R +R^\dagger \BOmega^2 R \inv{2} \\
&= \inv{2} \left(R^\dagger \BOmega^2 R - R^\dagger \BOmega^2 R\right) +R^\dagger \DotT{\BOmega} R \\
&= R^\dagger \DotT{\BOmega} R \\
\end{aligned}
\end{equation}

This shows that the body angular acceleration is just an inverse rotation of the rest frame angular acceleration like the angular velocities are.

\section{Revisit general rotation using matrices}

Having fully calculated velocity and acceleration in terms of rotating frame coordinates, lets
go back and revisit this with matrices and see how one would do the same for a general rotation.

Following GAFP express the rest frame coordinates for a point \(\By\) in terms of a rotation
applied to a rotating frame position \(\Bx\) (this is easier than the mess of primes and dots
used in Tong's paper).  Also omit the origin translation (that can be added in later if desired
easily enough)

\begin{equation}\label{eqn:keRotation:540}
\By = R \Bx
\end{equation}

Thus the derivative is:

\begin{equation}\label{eqn:keRotation:560}
\DotT{\By} = \DotT{R} \Bx + R \DotT{\Bx}.
\end{equation}

As in the GA case we want to factor this so that we have a rotation applied to a something
that is completely specified in the rotating frame.  This
is quite easy with matrices, as we just have to factor out a rotation matrix from \(\DotT{R}\):

\begin{equation}\label{eqn:keRotation:980}
\begin{aligned}
\DotT{\By}
&= R \transpose{R}\DotT{R} \Bx + R \DotT{\Bx} \\
&= R \left(\transpose{R}\DotT{R} \Bx + \DotT{\Bx} \right) \\
\end{aligned}
\end{equation}

This new product \(\transpose{R}\DotT{R} \Bx\) we have seen above in the special case of z-axis
rotation as a cross product.  In the GA general rotation case, we have seen that this as a
bivector-vector dot product.  Both of these are fundamentally antisymmetric operations,
so we expect this of the matrix operator too.  Verification of this antisymmetry follows
in almost the same fashion as the GA case, by observing that the derivative of an identity
matrix \(I = \transpose{R}R\) is zero:

\begin{equation}\label{eqn:keRotation:580}
\DotT{I} = 0
\end{equation}
\begin{equation}\label{eqn:keRotation:600}
\implies
\transpose{\DotT{R}}R + \transpose{R}\DotT{R} = 0
\end{equation}
\begin{equation}\label{eqn:keRotation:620}
\implies
\transpose{R}\DotT{R} = -\transpose{\DotT{R}}R = -\transpose{\transpose{R}\DotT{R}}
\end{equation}

Thus if one writes:

\begin{equation}\label{eqn:keRot:bodyangularvelocitymatrix}
\BOmega = \transpose{R}\DotT{R}
\end{equation}

the antisymmetric property of this matrix can be summarized as:

\begin{equation}\label{eqn:keRotation:640}
\BOmega = -\transpose{\BOmega}.
\end{equation}

Let us write out the form of this matrix in the first few dimensions:

\begin{itemize}
\item \R{2}

\begin{equation}\label{eqn:keRotation:660}
\BOmega =
\begin{bmatrix}
0 & -a \\
a & 0  \\
\end{bmatrix}
\end{equation}

For some \(a\).

\item \R{3}

\begin{equation}\label{eqn:keRotation:680}
\BOmega =
\begin{bmatrix}
0 & -a & -b \\
a &  0 & -c \\
b &  c &  0 \\
\end{bmatrix}
\end{equation}

For some \(a, b, c\).

\item \R{4}

\begin{equation}\label{eqn:keRotation:700}
\BOmega =
\begin{bmatrix}
0 & -a & -b & -d \\
a & 0 & -c & -e \\
b & c & 0 & -f \\
d & e & f & 0 \\
\end{bmatrix}
\end{equation}

For some \(a, b, c, d, e, f\).
\end{itemize}

For \R{N} we have \((N^2-N)/2\) degrees of freedom.  It is noteworthy to observe that this is exactly the number of basis elements of a bivector.  For example, in \R{4}, such a bivector basis is
\(\Be_{12}, \Be_{13}, \Be_{14}, \Be_{23}, \Be_{24}, \Be_{34}\).

For \R{3} we have three degrees of freedom and because of the antisymmetry
can express this matrix-vector product using the cross product.  Let

\begin{equation}\label{eqn:keRotation:720}
(a,b,c) = (\omega_3, -\omega_2, \omega_1)
\end{equation}

One has:

\begin{equation}\label{eqn:keRotation:740}
\BOmega \Bx =
\begin{bmatrix}
0 & -\omega_3 & \omega_2 \\
\omega_3 &  0 & -\omega_1 \\
-\omega_2 & \omega_1 &  0 \\
\end{bmatrix}
\begin{bmatrix}
x_1 \\
x_2 \\
x_3 \\
\end{bmatrix}
=
\begin{bmatrix}
-\omega_3 x_2 +\omega_2 x_3 \\
+\omega_3 x_1 -\omega_1 x_3 \\
-\omega_2 x_1 +\omega_1 x_2 \\
\end{bmatrix}
= \Bomega \cross \Bx
\end{equation}

Summarizing the velocity result we have, using \(\BOmega\) from \eqnref{eqn:keRot:bodyangularvelocitymatrix}:

\begin{equation}
\DotT{\By} = R \left( \BOmega \Bx + \DotT{\Bx} \right)
\end{equation}

Or, for \R{3}, we can define a body angular velocity vector

\begin{equation}
\Bomega =
\begin{bmatrix}
\BOmega_{32} \\
\BOmega_{13} \\
\BOmega_{21} \\
\end{bmatrix}
\end{equation}

and thus write the velocity as:

\begin{equation}
\DotT{\By} = R \left( \Bomega \cross \Bx + \DotT{\Bx} \right)
\end{equation}

This, like the GA result is good for general rotations.  Then do not have to be constant
rotation rates, and it allows for arbitrarily
oriented as well as wobbly motion of the rotating frame.

As with the GA general velocity calculation, this general form also allows us to calculate
the squared velocity easily, since the rotation matrices will
cancel after transposition:

\begin{equation}\label{eqn:keRotation:760}
\DotT{\By}^2 =
\left(R \left( \Bomega \cross \Bx + \DotT{\Bx} \right)\right) \cdot
\left(R \left( \Bomega \cross \Bx + \DotT{\Bx} \right)\right)
=
\transpose{\left( \Bomega \cross \Bx + \DotT{\Bx} \right)} \transpose{R}
R \left( \Bomega \cross \Bx + \DotT{\Bx} \right)
\end{equation}
\begin{equation}\label{eqn:keRotation:780}
\implies
\DotT{\By}^2 =
{\left( \Bomega \cross \Bx + \DotT{\Bx} \right)}^2
\end{equation}

\section{Equations of motion from Lagrange partials}

TBD.  Do this using the Rotor formulation.  How?

   \chapter{Polar velocity and acceleration}
      %
% Copyright � 2012 Peeter Joot.  All Rights Reserved.
% Licenced as described in the file LICENSE under the root directory of this GIT repository.
%

%
%
%\chapter{Polar velocity and acceleration}
\index{velocity!polar}
\index{acceleration!polar}
\label{chap:radial}
%\date{Jan 13, 2009.  radial.tex}

\section{Motivation}

Have previously worked out the radial velocity and acceleration components a pile of different ways in
\chapcite{PJAngAcc},
\chapcite{PJAngAccCross},
\chapcite{PJAngVel},
\chapcite{PJKeRot},
\chapcite{PJRadialDer}, and
\chapcite{PJUnitDer}.

So, what is a couple more?

When the motion is strictly restricted to a plane we can get away with doing this either in complex numbers
(used in a number of the Tong Lagrangian solutions), or with a polar form \R{2} vector (a polar representation
I have not seen since High School).

\section{With complex numbers}

Let
\begin{equation}\label{eqn:radial:20}
\begin{aligned}
z = r e^{i\theta}
\end{aligned}
\end{equation}

So our velocity is

\begin{equation}\label{eqn:radial:40}
\begin{aligned}
\zdot = \rdot e^{i\theta} + i r \thetadot e^{i\theta}
\end{aligned}
\end{equation}

and the acceleration is
\begin{equation}\label{eqn:radial:60}
\begin{aligned}
\ddot{z}
&= \ddot{r} e^{i\theta} + i \dot{r} \thetadot e^{i\theta}
 + i \rdot \thetadot e^{i\theta}
 + i r \ddot{\theta} e^{i\theta}
 - r \thetadot^2 e^{i\theta} \\
&= (\ddot{r} - r \thetadot^2 ) e^{i\theta} + (2 \dot{r} \thetadot + r \ddot{\theta} ) i e^{i\theta}
\end{aligned}
\end{equation}

\section{Plane vector representation}

Also can do this with polar vector representation directly (without involving the complexity of rotation matrices or anything fancy)

\begin{equation}\label{eqn:radial:80}
\begin{aligned}
\Br
&= r
\begin{bmatrix}
\cos\theta \\
\sin\theta
\end{bmatrix}
\end{aligned}
\end{equation}

Velocity is then
\begin{equation}\label{eqn:radial:100}
\begin{aligned}
\Bv
&=
\rdot
\begin{bmatrix}
\cos\theta \\
\sin\theta
\end{bmatrix}
+r \thetadot
\begin{bmatrix}
-\sin\theta \\
\cos\theta
\end{bmatrix}
\end{aligned}
\end{equation}

and for acceleration we have

\begin{equation}\label{eqn:radial:120}
\begin{aligned}
\Ba
&=
\ddot{r}
\begin{bmatrix}
\cos\theta \\
\sin\theta
\end{bmatrix}
+\rdot \thetadot
\begin{bmatrix}
-\sin\theta \\
\cos\theta
\end{bmatrix}
+\rdot \thetadot
\begin{bmatrix}
-\sin\theta \\
\cos\theta
\end{bmatrix}
+r \ddot{\theta}
\begin{bmatrix}
-\sin\theta \\
\cos\theta
\end{bmatrix}
-r \thetadot^2
\begin{bmatrix}
\cos\theta \\
\sin\theta
\end{bmatrix} \\
&=
(\ddot{r} -r \thetadot^2)
\begin{bmatrix}
\cos\theta \\
\sin\theta
\end{bmatrix}
+(2\rdot \thetadot +r \ddot{\theta})
\begin{bmatrix}
-\sin\theta \\
\cos\theta
\end{bmatrix}
\end{aligned}
\end{equation}

   \chapter{Gradient and tensor notes}
      %
% Copyright � 2012 Peeter Joot.  All Rights Reserved.
% Licenced as described in the file LICENSE under the root directory of this GIT repository.
%

%
%
%\chapter{Gradient and tensor notes}
\index{gradient}
\index{tensor}
\label{chap:tensor}
%\date{June 6, 2008.  tensor.tex}

\section{Motivation}

Some notes on tensors filling in assumed details covered in
\citep{doran2003gap}.

%in response to having to clarify
%, mostly from Geometric Algebra for Physicists (GAFP).  Write up enough notes for myself that I can understand the topics, filling in details
%omitted from GAFP.  Goal for this is to ensure I can explain the ideas to myself, since if I can not then I do not understand them sufficiently.

Conclude with the
solution of problem 6.1 to demonstrate the frame independence of the
vector derivative.
%covariant derivative.

Despite being notes associated with a Geometric Algebra text, there is no
GA content.  Outside of the eventual GA application of the gradient as
described in this form, the only GA connection is the fact that the
the reciprocal frame vectors can be thought of as a result of a duality
calculation.  That connection is not necessary though since one can just as
easily define the reciprocal frame in terms of matrix operations.  As an example, for a Euclidean metric the reciprocal frame vectors are the columns of
\(F(F^{\text{T}}F)^{-1}\) where the columns of \(F\) are the vectors in question.

These notes may not stand well on their own without the text, at least
as learning material.

\subsection{Raised and lowered indices. Coordinates of vectors with non-orthonormal frames}

Let \(\{ e_i \}\) represent a frame of not necessarily orthonormal basis vectors for a metric space, and \(\{ e^i \}\) represent the reciprocal frame.

The reciprocal frame vectors are defined by the relation:

\begin{equation}
e_i \cdot e^j = {\delta_i}^j.
\end{equation}

Lets compute the coordinates of a vector \(x\) in terms of both frames:

\begin{equation}\label{eqn:tensor:20}
x = \sum \alpha_j e_j = \sum \beta_j e^j
\end{equation}

Forming \(x \cdot e^i\), and \(x \cdot e_i\) respectively solves for the \(\alpha\), and \(\beta\) coefficients

\begin{equation}\label{eqn:tensor:40}
x \cdot e^i = \sum \alpha_j e_j \cdot e^i = \sum \alpha_j {\delta_j}^i = \alpha_i
\end{equation}

\begin{equation}\label{eqn:tensor:60}
x \cdot e_i = \sum \beta_j e^j \cdot e_i = \sum \beta_j {\delta_i}^j = \beta_i
\end{equation}

Thus, the reciprocal frame vectors allow for simple determination of coordinates for an arbitrary frame. We can summarize this as follows:

\begin{equation}\label{eqn:tensor:80}
x = \sum ( x \cdot e^i ) e_i = \sum ( x \cdot e_i ) e^i
\end{equation}

Now, for orthonormal frames, where \(e_i = e^i\) we are used to writing:

\begin{equation}\label{eqn:tensor:100}
x = \sum x_i e_i,
\end{equation}

however for non-orthonormal frames the convention is to mix raised and lowered indices as follows:

\begin{equation}\label{eqn:tensor:120}
x = \sum x^i e_i = \sum x_i e^i.
\end{equation}

Where, as demonstrated above these generalized coordinates have the values, \(x^i = x \cdot e^i\), and \(x_i = x \cdot e_i\).  This is a strange seeming notation at
first especially since most of linear algebra is done with always lowered (or always upper for some authors) indices.  However one quickly gets used to it, especially after seeing how powerful the reciprocal frame concept is for dealing with non-orthonormal frames.  The alternative is probably the use of matrices and their inverses to express the same vector decompositions.

\subsection{Metric tensor}
\index{metric tensor}

It is customary in tensor formulations of physics to utilize a metric tensor to express the dot product.

Compute the dot product using the coordinate vectors

\begin{equation}\label{eqn:tensor:140}
x \cdot y = \left(\sum x^i e_i \right)\left(\sum y^j e_j \right) = \sum x^i y^j \left( e_i \cdot e_j \right)
\end{equation}

\begin{equation}\label{eqn:tensor:160}
x \cdot y = \left(\sum x_i e^i \right)\left(\sum y_j e^j \right) = \sum x_i y_j \left( e^i \cdot e^j \right)
\end{equation}

Introducing second rank (symmetric) tensors for the dot product pairs \( e_i \cdot e_j = g_{ij}\), and \( g^{ij} = e^i \cdot e^j \) we have

\begin{equation}\label{eqn:tensor:180}
x \cdot y = \sum x_i y_j g^{ij} = \sum x^i y^j g_{ij} = \sum x_i y^i = \sum x^i y_i
\end{equation}

We see that the metric tensor provides a way to specify the dot product in index notation, and removes the explicit references to the original frame vectors.  Mixed indices also removes the references to the original frame vectors, but additionally eliminates the need for either of the metric tensors.

Note that it is also common to see Einstein summation convention employed, which omits the \(\sum\):

\begin{equation}\label{eqn:tensor:200}
x \cdot y = x_i y_j g^{ij} = x^i y^j g_{ij} = x^i y_i = x_i y^i
\end{equation}

Here summation over all matched upper, lower index pairs is implied.

\subsection{Metric tensor relations to coordinates}

Given a coordinate expression of a vector, we dot that with the frame vectors to observe the relation between coordinates and the metric tensor:

\begin{equation}\label{eqn:tensor:220}
x \cdot e_i = \sum x^j e_j \cdot e_i = \sum x^j g_{ij}
\end{equation}

\begin{equation}\label{eqn:tensor:240}
x \cdot e^i = \sum x_j e^j \cdot e^i = \sum x_j g^{ij}
\end{equation}

The metric tensors can therefore be used be used to express the relations between the upper and lower index coordinates:

\begin{align}
x_i &= \sum g_{ij} x^j \label{eqn:tensor:metric_upper_to_lower} \\
x^i &= \sum g^{ij} x_j \label{eqn:tensor:metric_lower_to_upper}
\end{align}

It is therefore apparent that the matrix of the index lowered metric tensor \(g_{ij}\) is the inverse of the matrix for the raised index metric tensor \(g^{ij}\).

Expressed more exactly,

\begin{equation}\label{eqn:tensor:300}
\begin{aligned}
x_i
&= \sum_{ij} g_{ij} x^j \\
&= \sum_{ijk} g_{ij} g^{jk} x_k \\
&= \sum_{ik} x_k \sum_j g_{ij} g^{jk} \\
\end{aligned}
\end{equation}

Since the left and right hand sides are equal for any \(x_i, x_k\), we have:

\begin{equation}
{\delta_{i}}^j = \sum_m g_{im} g^{mj} \\
\end{equation}

Demonstration of the inverse property required for summation on other set of indices too for completeness, but since these functions are symmetric, there
is no potential that this would have a ``left'' or ``right'' inverse type of action.

\subsection{Metric tensor as a Jacobian}
\index{Jacobian}

The relations of equations \eqnref{eqn:tensor:metric_upper_to_lower}, and \eqnref{eqn:tensor:metric_lower_to_upper} show that the metric tensor can be expressed in terms of partial derivatives:

\begin{equation}\label{eqn:tensor:320}
\begin{aligned}
\frac{\partial x_i }{\partial x^j } &= g_{ij} \\
\frac{\partial x^i }{\partial x_j } &= g^{ij}
\end{aligned}
\end{equation}

Therefore the metric tensors can also be expressed as Jacobian matrices (not Jacobian determinants) :

\begin{equation}\label{eqn:tensor:340}
\begin{aligned}
g_{ij} &= \frac{\partial (x_1, \cdots, x_n) }{\partial (x^1, \cdots, x^n) } \\
g^{ij} &= \frac{\partial (x^1, \cdots, x^n) }{\partial (x_1, \cdots, x_n) }
\end{aligned}
\end{equation}

Will this be useful in any way?

\subsection{Change of basis}
\index{change of basis}

To perform a change of basis from one non-orthonormal basis \(\{e_i\}\) to a second \(\{f_i\}\), relations between the sets of vectors
are required.  Using Greek indices for the \(f\) frame, and English for the \(e\) frame, those are:

\begin{equation}\label{eqn:tensor:360}
\begin{aligned}
e_i 		&= \sum f^{\mu} e_i \cdot f_{\mu} 	= \sum f_{\mu} e_i \cdot f^{\mu} \\
f_{\alpha} 	&= \sum e^k f_{\alpha} \cdot e_k 	= \sum e_k f_{\alpha} \cdot e^k \\
e^i 		&= \sum f^{\mu} e^i \cdot f_{\mu} 	= \sum f_{\mu} e^i \cdot f^{\mu} \\
f^{\alpha} 	&= \sum e^k f^{\alpha} \cdot e_k 	= \sum e_k f^{\alpha} \cdot e^k
\end{aligned}
\end{equation}

Following GAFP we can write the dot product terms as a second order tensors \(f\) (ie: matrix relation) :

\begin{equation}\label{eqn:tensor:380}
\begin{aligned}
e_i 		&= \sum f^{\mu} f_{i\mu}  	= \sum f_{\mu} {f_i}^{\mu} \\
f_{\alpha} 	&= \sum e^k f_{k\alpha} 	= \sum e_k {f^k}_{\alpha} \\
e^i 		&= \sum f^{\mu} {f^i}_{\mu} 	= \sum f_{\mu} f^{i \mu} \\
f^{\alpha} 	&= \sum e^k {f_k}^{\alpha}  	= \sum e_k f^{k\alpha}
\end{aligned}
\end{equation}

Note that all these various tensors are related to each other using the metric tensors for \(f\) and \(e\).  FIXME: show example.  Also note that using this notation the metric tensors \(g_{ij}\) and \(g_{\alpha\beta}\) are two completely different linear functions, and careful use of the index conventions are required to keep these straight.

\subsection{Inverse relationships}

Looking at these relations in pairs, such as

\begin{equation}\label{eqn:tensor:400}
\begin{aligned}
f_{\alpha} 	&= \sum e^k f_{k\alpha} \\
e^i 		&= \sum f_{\mu} f^{i \mu}
\end{aligned}
\end{equation}

and

\begin{equation}\label{eqn:tensor:420}
\begin{aligned}
e_i 		&= \sum f^{\mu} f_{i\mu} \\
f^{\alpha} 	&= \sum e_k f^{k\alpha}
\end{aligned}
\end{equation}

It is clear that \(f_{i\alpha}\) is the inverse of \(f^{i\alpha}\).

To be more precise
\begin{equation}\label{eqn:tensor:440}
\begin{aligned}
f_{\alpha}
&= \sum e^k f_{k\alpha} \\
&= \sum f_{\mu} f^{k \mu} f_{k\alpha} \\
\end{aligned}
\end{equation}

Thus
\begin{equation}
\sum_k f^{k \beta} f_{k\alpha} = \delta_{\alpha}^{\beta}
\end{equation}

To verify both ``left'' and ``right'' inverse properties we also need:

\begin{equation}\label{eqn:tensor:460}
\begin{aligned}
e^i
&= \sum f_{\mu} f^{i \mu}  \\
&= \sum e^k f_{k\mu} f^{i \mu} \\
\end{aligned}
\end{equation}

which shows that summation on the Greek indices also yields an inverse:

\begin{equation}
\sum_{\mu} f_{i\mu} f^{j \mu} = \delta_{i}^j
\end{equation}

There are also inverse relationships for the mixed index tensors above.  Specifically,

\begin{equation}\label{eqn:tensor:480}
\begin{aligned}
x^{\alpha}
&= \sum x^i f_i^{\alpha} \\
&= \sum x^{\beta} f_{\beta}^i f_i^{\alpha} \\
\end{aligned}
\end{equation}

Thus,
\begin{equation}
\sum_i f_{\beta}^i f_i^{\alpha} = \delta_{\beta}^{\alpha}
\end{equation}

And,
\begin{equation}\label{eqn:tensor:500}
\begin{aligned}
x^{i}
&= \sum x^{\beta} f_{\beta}^i \\
&= \sum x^{j} f_j^{\beta} f_{\beta}^i \\
\end{aligned}
\end{equation}

Thus,
\begin{equation}
\sum_{\alpha} f_j^{\alpha} f_{\alpha}^i = \delta_j^i
\end{equation}

This completely demonstrates the inverse relationship.

\subsection{Vector derivative}
\index{vector derivative}

GAFP exercise 6.1.  Show that the vector derivative:

\begin{equation}
\nabla = \sum e^i \frac{\partial}{\partial x^i}
\end{equation}

is not frame dependent.

To show this we will need to utilize the chain rule to rewrite the partials in terms of the alternate frame:

\begin{equation}\label{eqn:tensor:520}
\begin{aligned}
\frac{\partial}{\partial x^i} &= \sum \frac{\partial x^\alpha}{\partial x^i} \frac{\partial}{\partial x^\alpha}
\end{aligned}
\end{equation}

To evaluate the first partial here, we write the coordinates of a vector in terms of both, and take dot products:

\begin{equation}\label{eqn:tensor:540}
\begin{aligned}
\left(\sum x^{\gamma} f_{\gamma}\right) \cdot f^{\alpha} = \left(\sum x^i e_i\right) \cdot f^{\alpha} \\
\end{aligned}
\end{equation}
\begin{equation}\label{eqn:tensor:560}
\begin{aligned}
x^{\alpha} &= \sum x^i {f_i}^{\alpha} \\
\end{aligned}
\end{equation}
\begin{equation}\label{eqn:tensor:580}
\begin{aligned}
\frac{\partial x^{\alpha}}{\partial x^i} &= {f_i}^{\alpha}
\end{aligned}
\end{equation}

Similar expressions for the other change of basis tensors is also possible, but
not required for this problem.

With this result we have the partial re-expressed in terms of coordinates
in the new frame.

\begin{equation}\label{eqn:tensor:600}
\begin{aligned}
\frac{\partial}{\partial x^i} &= \sum {f_i}^{\alpha} \frac{\partial}{\partial x^\alpha}
\end{aligned}
\end{equation}

Combine this with the alternate contra-variant frame vector as calculated above:

\begin{equation}\label{eqn:tensor:260}
e^i = \sum f^{\mu} {f^i}_{\mu}
\end{equation}

and we have:

\begin{equation}\label{eqn:tensor:620}
\begin{aligned}
\sum_i e^i \frac{\partial}{\partial x^i}
&= \sum_i \left(\sum_{\mu} f^{\mu} {f^i}_{\mu} \right) \left( \sum_{\alpha} {f_i}^{\alpha} \frac{\partial}{\partial x^\alpha}\right) \\
&= \sum_{\mu \alpha} \left(f^{\mu} \frac{\partial}{\partial x^\alpha} \right) \sum_i {f^i}_{\mu} {f_i}^{\alpha} \\
&= \sum_{\mu \alpha} \left(f^{\mu} \frac{\partial}{\partial x^\alpha} \right) {\delta_{\mu}}^{\alpha} \\
&= \sum_{\alpha} f^{\alpha} \frac{\partial}{\partial x^\alpha} \\
\end{aligned}
\end{equation}

%Note that my original paper derivation of the above used only the tensors \(f_{i\alpha}\), and \(f^{i\alpha}\) instead of the mixed index versions used here.  That worked but also required a pair of metric tensors, and one more step to sum over those tensors to get at the final result.

\subsection{Why a preference for index upper vector and coordinates in the gradient?}

We can express the gradient in terms of index lower variables and vectors too, as follows:

\begin{equation*}
\frac{\partial}{\partial x^i} = \sum \frac{\partial x_j}{\partial x^i} \frac{\partial}{\partial x_j}
\end{equation*}

Employing the coordinate relations we have:
\begin{equation}\label{eqn:tensor:640}
\begin{aligned}
\sum x_i e^i \cdot e_j = x_j = \sum x^i e_i \cdot e_j = \sum x^i g_{ij}
\end{aligned}
\end{equation}

and can thus calculate the partials:
\begin{equation}\label{eqn:tensor:660}
\begin{aligned}
\frac{\partial x_j}{\partial x^i} = g_{ij},
\end{aligned}
\end{equation}

and can use that to do the change of variables to index lower coordinates:

\begin{equation*}
\frac{\partial}{\partial x^i} = \sum g_{ij} \frac{\partial}{\partial x_j}
\end{equation*}

Now we also can write the reciprocal frame vectors:

\begin{equation*}
e^i = \sum e_j e^i \cdot e^j = \sum e_j g^{ij}
\end{equation*}

Thus the gradient is:

\begin{equation}\label{eqn:tensor:680}
\begin{aligned}
\sum_i e^i \frac{\partial}{\partial x^i}
&= \sum_{ijk} e_j g^{ij} g_{ik} \frac{\partial}{\partial x_k} \\
&= \sum_{jk} \left(e_j \frac{\partial}{\partial x_k} \right) \sum_i g^{ij} g_{ik} \\
&= \sum_{jk} \left(e_j \frac{\partial}{\partial x_k} \right) \delta_k^j \\
&= \sum_{i} e_i \frac{\partial}{\partial x_i} \\
\end{aligned}
\end{equation}

My conclusion is that there is not any preference for the index upper form of the gradient in GAFP.  Both should be equivalent.  That said consistency is likely required.  FIXME: To truly get a feel for why index upper is used in this definition one likely needs to step back and look at the defining directional derivative relation for the gradient.

   \chapter{Inertial Tensor}
      %
% Copyright � 2012 Peeter Joot.  All Rights Reserved.
% Licenced as described in the file LICENSE under the root directory of this GIT repository.
%

%
%
%\chapter{Inertial Tensor}
\index{inertial tensor}
\label{chap:inertialTensor}
%\date{Feb 15, 2008.  inertialTensor.tex}

\citep{doran2003gap}
derives the angular momentum for rotational motion in the following form

\begin{equation}\label{eqn:inertialTensor:20}
L = R \left( \int \Bx \wedge (\Bx \cdot \Omega_B) dm \right) R^\dagger
\end{equation}

and calls the integral part, the inertia tensor

\begin{equation}\label{eqn:inertialTensor:40}
\emph{I}(B) = \int \Bx \wedge (\Bx \cdot \Omega_B) dm
\end{equation}

which is a linear mapping from bivectors to bivectors.  To understand the
form of this I found it helpful to expanding the wedge product
part of this explicitly for the \R{3} case.

Ignoring the sum in this expansion write

\begin{equation}\label{eqn:inertialTensor:60}
f(B) = \Bx \wedge (\Bx \cdot B)
\end{equation}

And writing \(\Be_{ij} = \Be_i \Be_j\) introduce a basis

\begin{equation}\label{eqn:inertialTensor:80}
b = \{ \Be_1 I, \Be_2 I, \Be_3 I \} = \{ \Be_{23}, \Be_{31}, \Be_{12} \}
\end{equation}

for the \R{3} bivector product space.

Now calculate \(f(B)\) for each of the basis vectors

\begin{equation}\label{eqn:inertialTensor:520}
\begin{aligned}
f(\Be_1 I)
&= \Bx \wedge (\Bx \cdot \Be_{23}) \\
&= ( x_1 \Be_1 + x_2 \Be_2 + x_3 \Be_3) \wedge (x_2 \Be_3 - x_3 \Be_2) \\
\end{aligned}
\end{equation}

Completing this calculation for each of the unit basic bivectors, we have
%%+ (           + x_2 \Be_2            ) \wedge (x_2 \Be_3            ) \\
%%+ (                       + x_3 \Be_3) \wedge (          - x_3 \Be_2) \\
%+ ( x_1 \Be_1                        ) \wedge (x_2 \Be_3            ) \\
%+ ( x_1 \Be_1                        ) \wedge (          - x_3 \Be_2) \\
%%+ (           + x_2 \Be_2            ) \wedge (          - x_3 \Be_2) \\ %% = 0
%%+ (                       + x_3 \Be_3) \wedge (x_2 \Be_3            ) \\ %% = 0
\begin{equation}\label{eqn:inertialTensor:540}
\begin{aligned}
f(\Be_1 I) &= (x_2^2 + x_3^2) \Be_{23} - (x_1 x_2) \Be_{31} - (x_1 x_3) \Be_{12} \\
f(\Be_2 I) &= -(x_1 x_2) \Be_{23} + (x_1^2 + x_3^2) \Be_{31} - (x_2 x_3) \Be_{12} \\
f(\Be_3 I) &= -(x_1 x_3) \Be_{23} - (x_2 x_3) \Be_{31} + (x_1^2 + x_2^2) \Be_{12} \\
\end{aligned}
\end{equation}

Observe that taking dot products with \((\Be_i I)^\dagger\) will select just the \(\Be_i I\) term of the result, so one can
form the matrix of this linear transformation that maps bivectors in basis \(b\) to image vectors also in basis \(b\) as follows

\begin{equation}\label{eqn:inertialTensor:100}
\matrixoftx{\emph{I}(B)}{b}{b}
=
\coordvec{\emph{I}(\Be_i I) \cdot (\Be_j I)^\dagger}{ij}
=
\int {\begin{bmatrix}
x_2^2 + x_3^2  & - x_1 x_2  & - x_1 x_3 \\
-x_1 x_2  & x_1^2 + x_3^2  & - x_2 x_3  \\
-x_1 x_3  & -x_2 x_3  & x_1^2 + x_2^2  \\
\end{bmatrix}} dm
\end{equation}

Here the notation \(\matrixoftx{A}{b}{c}\) is borrowed from
\citep{damiano1988cla} for the matrix of a linear transformation that
takes one from basis \(b\) to \(c\).

Observe that this (\R{3} specific expansion) can also be written in a more
typical tensor notation with \(\matrixoftx{\emph{I}}{b}{b} = \coordvec{I_{ij}}{ij}\)

\begin{equation}\label{eqn:inertialTensor:120}
I_{ij}
= \emph{I}(\Be_i I) \cdot (\Be_j I)^\dagger
= \int (\delta_{ij} \Bx^2 - x_i x_j) dm
\end{equation}

Where, as usual for tensors, the meaning of the indices and whether summation is required is implied.  In this case
the coordinate transformation matrix for this linear transformation has components \(I_{ij}\) (and no summation).

\section{orthogonal decomposition of a function mapping a blade to a blade}

Arriving at this result without explicit expansion is also possible by observing that an orthonormal decomposition of a
function can be written in terms of an orthogonal basis \(\{\sigma_i\}\) as follows:

\begin{equation}
f(B) = \sum_i (f(B) \cdot \sigma_i) \cdot \frac{1}{\sigma_i}
\end{equation}\label{eqn:itensor:bladeOrthogDecomp}

The dot product is required since the general product of two bivectors has grade-0, grade-2, and grade-4 terms (with a similar mix of higher grade terms for k-blades).

Perhaps unobviously since one is not normally used to seeing a scalar-vector dot product, this formula is not only true for bivectors, but any grade blade, including
vectors.  To verify this recall that the
general definition of the dot product is the lowest grade term of the geometric product of two blades.  For example with grade \(i,j\) blades \(a\), and \(b\) respectively
the dot product is:

\begin{equation}\label{eqn:inertialTensor:140}
a \cdot b = \langle a b \rangle_{\abs{i-j}}
\end{equation}

So, for a scalar-vector dot product is just the scalar product of the two

\begin{equation}\label{eqn:inertialTensor:160}
a \cdot \Bx = \langle{ a \Bx }\rangle_1 = a \Bx
\end{equation}

The inverse in \eqnref{eqn:itensor:bladeOrthogDecomp} can be removed by reversion, and for a grade-r blade this sum of projective terms then becomes:

\begin{equation}
f(B) = (-1)^{r(r-1)/2} \frac{1}{{\abs{\sigma_i}}^2}\sum_i (f(B) \cdot \sigma_i) \cdot {\sigma_i}
\end{equation}

For an orthonormal basis we have

\begin{equation}\label{eqn:inertialTensor:180}
\sigma_i \sigma_i^\dagger = \abs{\sigma_i}^2 = 1
\end{equation}

Which allows for a slightly simpler set of projective terms:

\begin{equation}
f(B) = (-1)^{r(r-1)/2} \sum_i (f(B) \cdot \sigma_i) \cdot {\sigma_i}
\end{equation}\label{eqn:itensor:OrthonormalDecomp}

\section{coordinate transformation matrix for a couple other linear transformations}

Seeing a function of a bivector for the first time is kind of intriguing.  We can form the matrix of such a linear transformation
from a basis of the bivector space to the space spanned by function.  For fun, let us calculate that matrix for the basis \(b\) above
for the following function:

\begin{equation}\label{eqn:inertialTensor:200}
f(B) = \Be_1 \wedge (\Be_2 \cdot B)
\end{equation}

For this function operating on \R{3} bivectors we have:

\begin{equation}\label{eqn:inertialTensor:560}
\begin{aligned}
f(\Be_{23}) &= \Be_1 \wedge (\Be_2 \cdot \Be_{23}) = -\Be_{31} \\
f(\Be_{31}) &= \Be_1 \wedge (\Be_2 \cdot \Be_{31}) = 0 \\
f(\Be_{12}) &= \Be_1 \wedge (\Be_2 \cdot \Be_{12}) = 0 \\
\end{aligned}
\end{equation}

So

\begin{equation}\label{eqn:inertialTensor:220}
\matrixoftx{f}{b}{b}
=
\begin{bmatrix}
-1 & 0 & 0 \\
0 & 0 & 0 \\
0 & 0 & 0 \\
\end{bmatrix}
\end{equation}

For \R{4} one orthonormal basis is

\begin{equation}\label{eqn:inertialTensor:240}
b = \{
\Be_{12},
\Be_{13},
\Be_{14},
\Be_{23},
\Be_{24},
\Be_{34}
\}
\end{equation}

A basis for the span of \(f\) is $b' = \{
\Be_{13},
\Be_{14}
\}$.  Like any other coordinate transformation associated with a linear transformation we can write the matrix of the transformation that
takes a coordinate vector in one basis into a coordinate vector for the basis for the image:

\begin{equation}\label{eqn:inertialTensor:260}
\coordvec{f(x)}{b'}
=
\matrixoftx{f}{b}{b'}
\coordvec{x}{b}
\end{equation}

For this function \(f\) and these pair of basis bivectors we have:

\begin{equation}\label{eqn:inertialTensor:280}
\matrixoftx{f}{b}{b'}
=
\begin{bmatrix}
0 & 0 & 0 & 1 & 0 & 0 \\
0 & 0 & 0 & 0 & 1 & 0 \\
\end{bmatrix}
\end{equation}

\section{Equation 3.126 details}

This statement from GAFP deserves expansion (or at least an exercise):

\begin{equation}\label{eqn:inertialTensor:300}
A \cdot ( \Bx \wedge (\Bx \cdot B) )
= \langle{A \Bx (\Bx \cdot B)}\rangle
= \langle (A \cdot \Bx) \Bx B \rangle
= B \cdot ( \Bx \wedge (\Bx \cdot A) )
\end{equation}

Perhaps this is obvious to the author, but was not to me.  To clarify this observe the following product

\begin{equation}\label{eqn:inertialTensor:320}
\Bx ( \Bx \cdot B ) = \Bx \cdot ( \Bx \cdot B ) + \Bx \wedge ( \Bx \cdot B )
\end{equation}

By writing \(B = \Bb \wedge \Bc\) we can show that the dot product part of this product is zero:

\begin{equation}\label{eqn:inertialTensor:580}
\begin{aligned}
\Bx \cdot ( \Bx \cdot B )
&= \Bx \cdot ( (\Bx \cdot \Bb) \Bc - (\Bx \cdot \Bc) \Bb ) \\
&= (\Bx \cdot \Bc) (\Bx \cdot \Bb) - (\Bx \cdot \Bb) (\Bx \cdot \Bc)) \\
&= 0
\end{aligned}
\end{equation}

This provides the justification for the wedge product removal in the text, since
one can write

\begin{equation}
\Bx \wedge ( \Bx \cdot B ) = \Bx ( \Bx \cdot B )
\end{equation}\label{eqn:itensor:inertiaWedgeToProduct}

Although it was not stated in the text \eqnref{eqn:itensor:inertiaWedgeToProduct}, can
be used to put this inertia product in a pure dot product form

\begin{equation}\label{eqn:inertialTensor:600}
\begin{aligned}
A^\dagger \cdot (\Bx \wedge (\Bx \cdot B) )
&= -\langle {A \Bx (\Bx \cdot B)} \rangle \\
&= \langle (\Bx \cdot A - A \wedge \Bx)(\Bx \cdot B) \rangle \\
\end{aligned}
\end{equation}

The trivector-vector part of this product has only vector and trivector components
\begin{equation}\label{eqn:inertialTensor:340}
(A \wedge \Bx)(\Bx \cdot B) = \langle{ (A \wedge \Bx)(\Bx \cdot B)}\rangle_1 + \langle{(A \wedge \Bx)(\Bx \cdot B)}\rangle_3
\end{equation}

So \(\langle{(A \wedge \Bx)(\Bx \cdot B)}\rangle_0 = 0\), and one can write

\begin{equation}
A^\dagger \cdot (\Bx \wedge (\Bx \cdot B) ) = (\Bx \cdot A) \cdot (\Bx \cdot B)
\end{equation}\label{eqn:itensor:iTensorTripleDot}

As pointed out in the text this is symmetric.  That can not be more clear than in \eqnref{eqn:itensor:iTensorTripleDot}.

\section{Just for fun.  General dimension component expansion of inertia tensor terms}

This triple dot product expansion allows for a more direct component expansion of the component form of the inertia tensor.
There are three general cases to consider.

\begin{itemize}
\item The diagonal terms:

\begin{equation}\label{eqn:inertialTensor:360}
(\Bx \cdot \sigma_i) \cdot (\Bx \cdot \sigma_i)
= (\Bx \cdot \sigma_i)^2
\end{equation}

Writing \(\sigma_i = \Be_{st}\) where \(s \ne t\), we have

\begin{equation}\label{eqn:inertialTensor:620}
\begin{aligned}
(\Bx \cdot \Be_{st})^2
&=
( (\Bx \cdot \Be_s) \Be_t -(\Bx \cdot \Be_t) \Be_s )^2 \\
&=
x_s^2 + x_t^2 - 2 x_s x_t \Be_t \cdot \Be_s \\
&=
x_s^2 + x_t^2
\end{aligned}
\end{equation}

\item Off diagonal terms where basis bivectors have a line of intersection (always true for \R{3}).

Here, ignoring the potential variation in sign, we can write the two basis bivectors as \(\sigma_i = \Be_{si}\) and \(\sigma_j = \Be_{ti}\), where \(s \ne t \ne i\).  Computing the products we have

\begin{equation}\label{eqn:inertialTensor:640}
\begin{aligned}
(\Bx \cdot \sigma_i) \cdot (\Bx \cdot \sigma_j)
&= (\Bx \cdot \Be_{si}) \cdot (\Bx \cdot \Be_{ti})  \\
&= ((\Bx \cdot \Be_s) \Be_i - (\Bx \cdot \Be_i) \Be_s) \cdot ((\Bx \cdot \Be_t) \Be_i - (\Bx \cdot \Be_i) \Be_t) \\
&= (x_s \Be_i - x_i \Be_s) \cdot (x_t \Be_i - x_i \Be_t) \\
&= x_s x_t \\
\end{aligned}
\end{equation}

\item Off diagonal terms where basis bivectors have no intersection.

An example from \R{4} are the two bivectors \(\Be_1 \wedge \Be_2\) and \(\Be_3 \wedge \Be_4\)

In general, again ignoring the potential variation in sign, we can write the two basis bivectors as \(\sigma_i = \Be_{su}\) and \(\sigma_j = \Be_{tv}\), where \(s \ne t \ne u \ne v\).  Computing the products we have

\begin{equation}\label{eqn:inertialTensor:660}
\begin{aligned}
(\Bx \cdot \sigma_i) \cdot (\Bx \cdot \sigma_j)
&= (\Bx \cdot \Be_{su}) \cdot (\Bx \cdot \Be_{tv})  \\
&= ((\Bx \cdot \Be_s) \Be_u - (\Bx \cdot \Be_u) \Be_s) \cdot ((\Bx \cdot \Be_t) \Be_v - (\Bx \cdot \Be_v) \Be_t) \\
&= 0 \\
\end{aligned}
\end{equation}

\end{itemize}

For example, choosing basis \(\sigma = \{ \Be_{12}, \Be_{13}, \Be_{14}, \Be_{23}, \Be_{24}, \Be_{34} \}\) the coordinate transformation matrix can be written out

\begin{equation}\label{eqn:inertialTensor:380}
\matrixoftx{f}{\sigma}{\sigma}
=
\begin{bmatrix}
x_1^2 + x_2^2     &     x_2 x_3      &     x_2 x_4    &     -x_1 x_3      &    -x_1 x_4    &     0 \\
x_2 x_3           & x_1^2 + x_3^2    &     x_3 x_4      &     x_1 x_2    &            0  & -x_1 x_4 \\
x_2 x_4           &  x_3 x_4         & x_1^2 + x_4^2    &         0    &      x_1 x_2    & x_1 x_3 \\
-x_1 x_3          &  x_1 x_2         &          0     & x_2^2 + x_3^2  &      x_3 x_4    & -x_2 x_4 \\
-x_1 x_4          &  0             &     x_1 x_2      &     x_3 x_4    & x_1^2 + x_4^2   & x_2 x_3 \\
   0            &  -x_1 x_4        &     x_1 x_3      &    -x_2 x_4    &      x_2 x_3    &  x_3^2 + x_4^2 \\
\end{bmatrix}
\end{equation}

\section{Example calculation.  Masses in a line}

Pick some points on the x-axis, \(\Br^{(i)}\) with masses \(m_i\).
The (\R{3}) inertia tensor with respect to basis \(\{\Be_i I\}\), is

\begin{equation}\label{eqn:inertialTensor:400}
\sum_i {
\begin{bmatrix}
 0 & 0               & 0      \\
 0 & (r^{(i)}_1)^2     & 0      \\
 0 & 0               & (r^{(i)}_1)^2  \\
\end{bmatrix}
} m_i
= \sum{ m_i \Br_i^2}
\begin{bmatrix}
0 & 0 & 0 \\
0 & 1 & 0 \\
0 & 0 & 1 \\
\end{bmatrix}
\end{equation}

Observe that in this case the inertia tensor here only has components in the \(zx\) and \(xy\) planes (no component in the yz plane that is perpendicular to the line).

\section{Example calculation.  Masses in a plane}

Let \(x = re^{i\theta}\Be_1\), where \(i=\Be_1 \wedge \Be_2\) be a set of points in the \(xy\) plane, and use \(\sigma = \{\sigma_i = \Be_i I\}\) as the basis for the \R{3} bivector space.

We need to compute

\begin{equation}\label{eqn:inertialTensor:680}
\begin{aligned}
\Bx \cdot \sigma_i
&= r(e^{i\theta} \Be_1) \cdot ( \Be_i I ) \\
&= r\langle{ e^{i\theta} \Be_1 \Be_i I }\rangle \\
\end{aligned}
\end{equation}

Calculation of the inertia tensor components has three cases, depending on the value of \(i\)

\begin{itemize}
\item \(i=1\)

\begin{equation}\label{eqn:inertialTensor:700}
\begin{aligned}
\frac{1}{r}(\Bx \cdot \sigma_i)
&= {\langle e^{i\theta} I\rangle}_1 \\
&= i \sin\theta I \\
&= - \Be_3 \sin\theta \\
\end{aligned}
\end{equation}

\item \(i=2\)

\begin{equation}\label{eqn:inertialTensor:720}
\begin{aligned}
\frac{1}{r}(\Bx \cdot \sigma_i)
&= {\langle e^{i\theta} \Be_1 \Be_2 I\rangle}_1 \\
&= -{\langle e^{i\theta} \Be_3 \rangle}_1 \\
&= - \Be_3 \cos\theta \\
\end{aligned}
\end{equation}

\item \(i=3\)

\begin{equation}\label{eqn:inertialTensor:740}
\begin{aligned}
\frac{1}{r}(\Bx \cdot \sigma_i)
&= {\langle e^{i\theta} \Be_1 \Be_3 (\Be_3 \Be_1 \Be_2) \rangle}_1 \\
&= {\langle e^{i\theta} \Be_2 \rangle}_1 \\
&= e^{i\theta} \Be_2 \\
\end{aligned}
\end{equation}

\end{itemize}

Thus for \(i=\{1, 2, 3\}\), the diagonal terms are
\begin{equation}\label{eqn:inertialTensor:420}
(\Bx \cdot \sigma_i)^2 = r^2 \{ \sin^2 \theta, \cos^2 \theta, 1\}
\end{equation}

and the non-diagonal terms are
\begin{equation}\label{eqn:inertialTensor:440}
(\Bx \cdot \sigma_1) \cdot (\Bx \cdot \sigma_2) = r^2 \sin\theta \cos\theta
\end{equation}
\begin{equation}\label{eqn:inertialTensor:460}
(\Bx \cdot \sigma_1) \cdot (\Bx \cdot \sigma_3) = 0
\end{equation}
\begin{equation}\label{eqn:inertialTensor:480}
(\Bx \cdot \sigma_2) \cdot (\Bx \cdot \sigma_3) = 0
\end{equation}

Thus, with indices implied (\(r=\Br_i\), \(\theta = \theta_i\), and \(m = m_i\), the inertia tensor is

\begin{equation}\label{eqn:inertialTensor:760}
\begin{aligned}
\matrixoftx{\emph{I}}{\sigma}{\sigma}
&=
\sum m r^2
{
\begin{bmatrix}
\sin^2 \theta  & \sin\theta \cos \theta & 0 \\
\sin\theta \cos \theta & \cos^2 \theta & 0 \\
0 & 0 & 1 \\
\end{bmatrix}
} \\
\end{aligned}
\end{equation}

It is notable that this can be put into double angle form
\begin{equation}\label{eqn:inertialTensor:780}
\begin{aligned}
\matrixoftx{\emph{I}}{\sigma}{\sigma}
&=
\sum
m r^2
{
\begin{bmatrix}
\frac{1}{2} (1 - \cos 2\theta)    &   \frac{1}{2} \sin 2\theta      &     0 \\
\frac{1}{2} \sin 2\theta      &     \frac{1}{2}(1 + \cos 2\theta)    &    0 \\
0 & 0 & 1 \\
\end{bmatrix}
} \\
&=
\frac{1}{2}
\sum
m r^2
\left(
I +
{
\begin{bmatrix}
- \cos 2\theta    &    \sin 2\theta     &    0 \\
\sin 2\theta      &     \cos 2\theta    &    0 \\
0 & 0 & 1 \\
\end{bmatrix}
}
\right) \\
\end{aligned}
\end{equation}

So if grouping masses along each distinct line in the plane, those components of the inertia tensor can be thought of as
functions of twice the angle.  This is natural in terms of a rotor interpretation, which is likely possible since each of
these groups of masses in a line can be diagonalized with a rotation.

It can be verified that the following \(xy\) plane rotation diagonalizes all the terms of constant angle.  Writing

\begin{equation}\label{eqn:inertialTensor:500}
R_\theta =
\begin{bmatrix}
\cos\theta & -\sin\theta & 0 \\
\sin\theta & \cos\theta & 0 \\
0 & 0 & 1 \\
\end{bmatrix}
\end{equation}

We have
\begin{equation}\label{eqn:inertialTensor:800}
\begin{aligned}
\matrixoftx{\emph{I}}{\sigma}{\sigma}
&=
\sum
m r^2
R_{-\theta}
\begin{bmatrix}
0 & 0 & 0 \\
0 & 1 & 0 \\
0 & 0 & 1 \\
\end{bmatrix}
R_\theta
\end{aligned}
\end{equation}

   \chapter{Satellite triangulation over sphere}
      %
% Copyright � 2012 Peeter Joot.  All Rights Reserved.
% Licenced as described in the file LICENSE under the root directory of this GIT repository.
%

%
%
%\chapter{Satellite triangulation over sphere}
\index{triangulation}
\label{chap:locateSatellite}
%\date{April 13, 2008.  locateSatellite.tex}

\section{Motivation and preparation}

Was playing around with what is probably traditionally a spherical trig type problem using geometric algebra (locate satellite position using angle measurements from two well separated points).  Origin of the problem was just me looking at my Feynman Lectures introduction where there is a diagram illustrating how triangulation could be used to locate "Sputnik" and thought I had try such a calculation, but in a way that I thought was more realistic.

\imageFigure{../gabook-figures/satellite}{Satellite location by measuring direction from two points}{fig:satellite}{0.4}

\Cref{fig:satellite} illustrates the problem I attempted to solve.  Pick two arbitrary points \(P_1\), and \(P_2\) on the globe, separated far enough that the curvature of the earth may be a factor.
For this problem it is assumed that the angles to the satellite will be measured concurrently.

Place a fixed reference frame at the center of the earth.  In the figure this is shown translated to the \((0,0)\) point (equator and prime meridian intersection).  I have picked \(\Be_1\) facing east, \(\Be_2\) facing north, and \(\Be_3\) facing outwards from the core.

Each point \(P_i\) can be located by a rotation along the equatorial plane by angle \(\lambda_i\) (measured with an east facing orientation (direction of \(\Be_1\)), and a rotation \(\psi_i\) towards the north (directed towards \(\Be_2\)).

To identify a point on the surface we translate our \((0,0)\) reference frame to that point using the
following rotor equation:

\begin{equation}\label{eqn:locSat:northrotor}
R_{\psi} = \exp(-\Be_{32}\psi/2) = \cos(\psi/2) - \Be_{32}\sin(\psi/2)
\end{equation}
\begin{equation}\label{eqn:locSat:eastrotor}
R_{\lambda} = \exp(-\Be_{31}\lambda/2) = \cos(\lambda/2) - \Be_{31}\sin(\lambda/2)
\end{equation}
\begin{equation}
R(x) = R_{\psi} R_{\lambda} x R_{\lambda}^\dagger R_{\psi}^\dagger
\end{equation}

To verify that I got the sign of these rotations right, I applied them to the unit vectors using a \(\pi/2\) rotation.  We want the following for the equatorial plane rotation:

\begin{equation*}
R_{\lambda}(\pi/2)
\begin{bmatrix}
\Be_1 \\
\Be_2 \\
\Be_3 \\
\end{bmatrix}
R_{\lambda}(\pi/2)^\dagger
=
\begin{bmatrix}
-\Be_3 \\
\Be_2 \\
\Be_1 \\
\end{bmatrix}
\end{equation*}

And for the northwards rotation:

\begin{equation*}
R_{\psi}(\pi/2)
\begin{bmatrix}
\Be_1 \\
\Be_2 \\
\Be_3 \\
\end{bmatrix}
R_{\psi}(\pi/2)^\dagger
=
\begin{bmatrix}
\Be_1 \\
\Be_2 \\
-\Be_3 \\
\end{bmatrix}
\end{equation*}

Verifying this is simple enough using the explicit sine and cosine expansion of the rotors in \eqnref{eqn:locSat:northrotor} and \eqnref{eqn:locSat:eastrotor}.

Once we have the ability to translate our reference frame to each point on the Earth, we can use the inverse rotation to translate our measured unit vector
to the satellite at that point back to the reference frame.

Suppose one calculates a local unit vector \(\alpha'\) towards the satellite by measuring direction cosines in our local reference frame (ie: angle from gravity opposing (up facing) direction, east, and north directions at that point).
Once that is done, that unit vector \(\alpha\) in our reference frame is obtained by inverse rotation:

\begin{equation}
\alpha = R_{\lambda_i}^\dagger R_{\psi_i}^\dagger \alpha' R_{\psi_i} R_{\lambda_i}
\end{equation}

The other place we need this rotation for is to calculate the points \(P_i\) in our reference from (treating this now as being at the core of the earth).  This is just:

\begin{equation}
P_i = R_{\psi} R_{\lambda} A_i \Be_3 R_{\lambda}^\dagger R_{\psi}^\dagger
\end{equation}

Where \(A_i\) is the altitude (relative to the center of the earth) at the point of interest.

\section{Solution}

Solving for the position of the satellite \(P_s\) we have:

\begin{equation}\label{eqn:locateSatellite:20}
P_s = a_1 \alpha_1 + P_1 = a_2 \alpha_2 + P_2
\end{equation}

Solution of this follows directly by taking wedge products.  Solve for \(a_1\) for example, we wedge with \(\alpha_2\) :

\begin{equation}\label{eqn:locateSatellite:40}
a_1 \alpha_1 \wedge \alpha_2 + P_1 \wedge \alpha_2 = a_2 \mathLabelBox{\alpha_2 \wedge \alpha_2}{\(=0\)} + P_2 \wedge \alpha_2
\end{equation}

Provided the points are far enough apart to get distinct \(\alpha_i\) measurements, then we have:
\begin{equation}\label{eqn:locateSatellite:60}
a_1 = \frac{(P_2-P_1) \wedge \alpha_2}{ \alpha_1 \wedge \alpha_2 }.
\end{equation}

Thus the position vector from the core of earth reference frame to the satellite is:

\begin{equation}
P_s = \left(\frac{(P_2-P_1) \wedge \alpha_2}{ \alpha_1 \wedge \alpha_2 }\right) \alpha_1 + P_1
\end{equation}

Notice how all the trigonometry is encoded directly in the rotor equations.  If one had to calculate all this using the spherical trigonometry generalized triangle relations I expect that you would have an ungodly mess of sine and cosines here.

This demonstrates two very distinct applications of the wedge product.  The first was to define an oriented plane, and was used as a generator of rotations (very much like the unit imaginary).  This second application, to solve linear equations takes advantage of \(a \wedge a = 0\) property of the wedge product.  It was convenient as it allowed simple simultaneous solution of the three equations (one for each component) and two unknowns problem in this particular case.

\section{matrix formulation}

Instead of solving with the wedge product one could formulate this as a matrix equation:

\begin{equation}\label{eqn:locSat:twopointsmatrix}
\begin{bmatrix}
\alpha_1 & -\alpha_2 \\
\end{bmatrix}
\begin{bmatrix}
a_1 \\
a_2 \\
\end{bmatrix}
=
\begin{bmatrix}
P_2 - P_1 \\
\end{bmatrix}
\end{equation}

This highlights the fact that the equations are over-specified, which is more obvious still when this is written out in component form:

\begin{equation}
\begin{bmatrix}
\alpha_{11} & -\alpha_{21} \\
\alpha_{12} & -\alpha_{22} \\
\alpha_{13} & -\alpha_{23} \\
\end{bmatrix}
\begin{bmatrix}
a_1 \\
a_2 \\
\end{bmatrix}
=
\begin{bmatrix}
P_{21} - P_{11} \\
P_{22} - P_{12} \\
P_{23} - P_{13} \\
\end{bmatrix}
\end{equation}

We have one more equation than we need to actually solve it, and cannot use matrix inversion directly (Gaussian elimination or a generalized inverse is required).

Recall the figure in the Feynman lectures when the observation points and the satellite are all in the same plane.  For that all that was needed was two angles, whereas we have measured six for each of the direction cosines used above, so the fact that our equations can include more info than required to solve the problem is not unexpected.

We could also generalize this, perhaps to remove measurement error, by utilizing more than two observation points.  This will compound the over-specification of the equations, and makes it clear that we likely want a least squares approach to solve it.
Here is an example of the matrix to solve for three points:

\begin{equation}\label{eqn:locSat:threepointsmatrix}
\begin{bmatrix}
\alpha_1 & -\alpha_2 & 0 \\
-\alpha_1 & 0 & \alpha_3 \\
0 & \alpha_2 & -\alpha_3 \\
\end{bmatrix}
\begin{bmatrix}
a_1 \\
a_2 \\
a_3 \\
\end{bmatrix}
=
\begin{bmatrix}
P_2 - P_1 \\
P_1 - P_3 \\
P_3 - P_2 \\
\end{bmatrix}
\end{equation}

Since the \(\alpha_i\) are vectors, this matrix of rotated direction cosines has dimensions 9 by 3 (just as \eqnref{eqn:locSat:twopointsmatrix} is a 3 by 2 matrix).

\section{Question.  Order of latitude and longitude rotors?}

Looking at a globe, it initially seemed clear to me that these "perpendicular" (abusing the word) rotations could be applied in either order, but their rotors definitely do not commute, so I assume that together the non-commutative bits of the rotors "cancel out".

Question, is it actually true that the end effect of applying these rotors in either order is the same?

\begin{equation}\label{eqn:locateSatellite:80}
x' = R_{\psi} R_{\lambda} x R_{\lambda}^\dagger R_{\psi}^\dagger = R_{\lambda} R_{\psi} x R_{\psi}^\dagger R_{\lambda}^\dagger
\end{equation}

Attempting to show this is true or false by direct brute force expansion is not productive (perhaps would be okay with a symbolic GA calculator).  However, such a direct expansion
of just the rotor products in either order allows for a comparison:

\begin{equation}\label{eqn:locateSatellite:140}
\begin{aligned}
R_{\psi} R_{\lambda}
&= ( \cos(\psi/2) - \Be_{32}\sin(\psi/2) )(\cos(\lambda/2) - \Be_{31}\sin(\lambda/2)) \\
&= \cos(\psi/2)\cos(\lambda/2) - \Be_{32}\sin(\psi/2)\cos(\lambda/2) - \Be_{31} \cos(\psi/2) \sin(\lambda/2) - \Be_{21} \sin(\psi/2) \sin(\lambda/2) \\
\end{aligned}
\end{equation}
\begin{equation}\label{eqn:locateSatellite:160}
\begin{aligned}
R_{\lambda} R_{\psi}
&= (\cos(\lambda/2) - \Be_{31}\sin(\lambda/2)) ( \cos(\psi/2) - \Be_{32}\sin(\psi/2) ) \\
&= \mathLabelBox{\cos(\psi/2)\cos(\lambda/2)}{\(a_0\)} \\
&\qquad + \mathLabelBox{-\Be_{32}\sin(\psi/2)\cos(\lambda/2) - \Be_{31} \cos(\psi/2) \sin(\lambda/2)}{\(A\)} \\
&\qquad + \mathLabelBox{\Be_{21} \sin(\psi/2) \sin(\lambda/2)}{\(B\)}
\end{aligned}
\end{equation}

Observe that these are identical except for an inversion of sign of the \(\Be_{21}\) term.  Using the shorthand above the respective rotations are:

\begin{equation}\label{eqn:locateSatellite:100}
R_{\lambda,\psi}(x) = R_{\psi} R_{\lambda} x R_{\lambda}^\dagger R_{\psi}^\dagger = (a_0 + A - B) x (a_0 -A +B)
\end{equation}

And
\begin{equation}\label{eqn:locateSatellite:120}
R_{\psi,\lambda}(x) = R_{\lambda} R_{\psi} x R_{\psi}^\dagger R_{\lambda}^\dagger = (a_0 + A + B) x (a_0 -A -B)
\end{equation}

And this can be used to disprove the general rotation commutativity.  We take the difference between these two rotation results, and see if it can be shown to equal zero.
Taking differences, also temporarily writing \(a = a_0 + A\), and exploiting a grade one filter since the final result must be a vector we have:

\begin{equation}\label{eqn:locateSatellite:180}
\begin{aligned}
R_{\lambda,\psi}(x) - R_{\psi,\lambda}(x)
&= \gpgradeone{R_{\lambda,\psi}(x) - R_{\psi,\lambda}(x)} \\
&= \gpgradeone{ (a - B) x (a^\dagger +B) -(a + B) x (a^\dagger -B) } \\
&= \gpgradeone{ ( a x a^\dagger -B x B -B x a^\dagger +a x B ) + ( - a x a^\dagger +B x B +a x B -B x a^\dagger ) } \\
&= \gpgradeone{ ( -B x a^\dagger +a x B ) + ( +a x B -B x a^\dagger ) } \\
&= 2\gpgradeone{ -B x a^\dagger +a x B } \\
&= 2\gpgradeone{ -B x (a_0 - A) + (a_0 + A) x B } \\
&= 2 a_0 ( -B x + x B ) + 2 \gpgradeone{ B x A + A x B } \\
&= 4 a_0 x \cdot B + 2 \gpgradeone{ B x A + A x B } \\
&= 4 a_0 x \cdot B + 2 \gpgradeone{ B \cdot x A - A B \cdot x } + 2 \gpgradeone{ B \wedge x A + A x \wedge B }  \\
&= 4 a_0 x \cdot B + 4 (B \cdot x) \cdot A + 2 (B \wedge x) \cdot A + 2 A \cdot (B \wedge x )  \\
&= 4 a_0 x \cdot B + 4 (B \cdot x) \cdot A + 2 (B \wedge x) \cdot A - 2 (B \wedge x ) \cdot A \\
&= 4 a_0 x \cdot B + 4 (B \cdot x) \cdot A \\
&= 4 (B \cdot x) \cdot (-a_0 + A) \\
&= -4 (B \cdot x) \cdot a^\dagger \\
\end{aligned}
\end{equation}

Evaluate this for \(x = \Be_1\) we do not have zero (a vector with \(\Be_2\) and \(\Be_3\) components), and for \(x = \Be_2\) this difference has \(\Be_1\), and \(\Be_3\) components.  However, for \(x = \Be_3\) this is zero.  Thus these
rotations only commute when applied to a vector that is completely normal to the sphere.  This is what messes up the intuition.  Rotating a point (represented by a vector) in either order works fine, but rotating a frame located at the surface back to a different point on the surface, and maintaining the east and north orientations we have to be careful which orientation to use.

So which order is right?  It has to be rotate first in the equatorial plane (\(\lambda)\), then the northwards rotation, where both are great circle rotations.

A numeric confirmation of this is likely prudent.

   \chapter{Exponential Solutions to Laplace Equation in \texorpdfstring{\R{N}}{ND}}
      %
% Copyright � 2012 Peeter Joot.  All Rights Reserved.
% Licenced as described in the file LICENSE under the root directory of this GIT repository.
%

%
%
%\chapter{Exponential Solutions to Laplace Equation in \texorpdfstring{\R{N}}{ND}}
\index{Laplace equation}
\label{chap:laplace}
%\date{Feb 28, 2008.  laplace.tex}

\section{The problem}

Want solutions of

\begin{equation}\label{eqn:gaLaplacianSol:laplacian}
\laplacian f = \sum_k \dsqxj{f}{k} = 0
\end{equation}

For real f.

\subsection{One dimension}

Here the problem is easy, just integrate twice:

\begin{equation}\label{eqn:laplace:22}
f = cx + d.
\end{equation}

\subsection{Two dimensions}

For the two dimensional case we want to solve:

\begin{equation}\label{eqn:laplace:42}
\dsqxj{f}{1} + \dsqxj{f}{2} = 0
\end{equation}

Using separation of variables one can find solutions of the form \(f = X(x_1)Y(x_2)\).  Differentiating we have:

\begin{equation}\label{eqn:laplace:62}
X''Y + XY'' = 0
\end{equation}

So, for \(X \ne 0\), and \(Y \ne 0\):
\begin{equation}\label{eqn:laplace:82}
\frac{X''}{X} = -\frac{Y''}{Y} = k^2
\end{equation}

\begin{equation}\label{eqn:laplace:102}
\implies
X = e^{kx}
\end{equation}
\begin{equation}\label{eqn:laplace:122}
Y = e^{k\Bi y}
\end{equation}

\begin{equation}\label{eqn:laplace:142}
\implies
f = XY = e^{k(x + \Bi y)}
\end{equation}

Here \(\Bi\) is anything that squares to -1.  Traditionally this is the
complex unit imaginary, but we are also free to use a geometric product unit bivector such as \(\Bi = \Be_1 \wedge \Be_2 = \Be_1\Be_2 = \Be_{12}\), or \(\Bi = \Be_{21}\).

With \(\Bi = \Be_{12}\) for example we have:

\begin{equation}\label{eqn:laplace:282}
\begin{aligned}
f = XY = e^{k(x + \Bi y)}
&= e^{k(x + \Be_{12} y)} \\
&= e^{k(x\Be_{1}\Be_1 + \Be_{12} y)} \\
&= e^{k\Be_1(x\Be_1 + \Be_2 y)} \\
\end{aligned}
\end{equation}

Writing \(\Bx = \sum x_i \Be_i\), all of the following are solutions
of the Laplacian

\begin{equation}\label{eqn:laplace:302}
\begin{aligned}
e^{k\Be_1\Bx} \\
e^{\Bx k\Be_1} \\
e^{k\Be_2\Bx} \\
e^{\Bx k\Be_2} \\
\end{aligned}
\end{equation}

Now there is not anything special about the use of the x and y axis so it is reasonable to expect that, given any constant vector \(\Bk\),
the following may also be solutions to the two dimensional Laplacian problem

\begin{equation}\label{eqn:gaLaplacianSol:expgeo1}
e^{\Bx\Bk} = e^{\Bx \cdot \Bk + \Bx \wedge \Bk}
\end{equation}
\begin{equation}\label{eqn:gaLaplacianSol:expgeo2}
e^{\Bk\Bx} = e^{\Bx \cdot \Bk - \Bx \wedge \Bk}
\end{equation}

\subsection{Verifying it is a solution}

To verify that equations \eqnref{eqn:gaLaplacianSol:expgeo1} and \eqnref{eqn:gaLaplacianSol:expgeo2} are Laplacian solutions, start with taking the first order partial with one of the coordinates.
Since there are conditions where this form of solution works in \R{N},
a two dimensional Laplacian will not be assumed here.

\begin{equation}\label{eqn:laplace:162}
\dxj{}{j}e^{\Bx\Bk}
\end{equation}

This can be evaluated without any restrictions, but introducing the restriction that the bivector part of \(\Bx\Bk\)
is coplanar with its derivative simplifies the result considerably.  That is introduce a restriction:

\begin{equation}\label{eqn:laplace:182}
\gpgradetwo{ \Bx \wedge \Bk \dxj{\Bx \wedge \Bk}{j} } = \gpgradetwo{ \Bx \wedge \Bk \Be_j \wedge \Bk } = 0
\end{equation}

With such a restriction we have

\begin{equation}\label{eqn:laplace:202}
\dxj{}{j}e^{\Bx\Bk} = \Be_j\Bk e^{\Bx\Bk} = e^{\Bx\Bk} \Be_j\Bk
\end{equation}

Now, how does one enforce a restriction of this form in general?  Some thought will show that one way to do so
is to require that
both \(\Bx\) and \(\Bk\) have only two components.  Say, components \(j\), and \(m\).  Then, summing second partials
we have:

\begin{equation}\label{eqn:laplace:322}
\begin{aligned}
\sum_{u=j,m}\dsqxj{}{u}e^{\Bx\Bk}
&= \left( \Be_j\Bk \Be_j\Bk + \Be_m\Bk \Be_m\Bk \right) e^{\Bx\Bk} \\
&= \left( \Be_j\Bk (-\Bk\Be_j + 2 \Bk \cdot \Be_j) + \Be_m\Bk (-\Bk\Be_m + 2 \Be_m \cdot \Bk) \right) e^{\Bx\Bk} \\
&= \left( -2\Bk^2 + 2 k_j^2 + 2 k_m k_j \Be_{jm} + 2 k_m^2 + 2 k_j k_m \Be_{mj} \right) e^{\Bx\Bk} \\
&= \left( -2\Bk^2 + 2 \Bk^2 + 2 k_j k_m (\Be_{mj} + \Be_{jm}) \right) e^{\Bx\Bk} \\
&= 0 \\
\end{aligned}
\end{equation}

This proves the result, but essentially just says that this form of
solution is only
valid when the constant parametrization vector \(\Bk\) and \(\Bx\) and its
variation are restricted to a specific plane.  That result could have
been obtained in much simpler ways, but I learned a lot about bivector
geometry in the approach! (not all listed here since it caused serious
digressions)

\subsection{Solution for an arbitrarily oriented plane}

Because the solution above is coordinate free, one would expect that this
works for any solution that is restricted to the plane with bivector \(\Bi\)
even when those do not line up with any specific pair of two coordinates.
This can be verified by performing a rotational
coordinate transformation of the
Laplacian operator, since one can always pick a pair of mutually orthogonal
basis vectors with corresponding coordinate vectors that lie in the plane
defined by such a bivector.

Given two arbitrary vectors in the space when both are projected onto the plane
with constant bivector \(\Bi\) their product is:

\begin{equation}\label{eqn:laplace:222}
\left(\Bx \cdot \Bi \inv{\Bi}\right)\left(\inv{\Bi} \Bi \cdot \Bk\right)
=
(\Bx \cdot \Bi)(\Bk \cdot \Bi)
\end{equation}

Thus one can express the general equation for a planar solution to the
homogeneous Laplace equation in the form

\begin{equation}
\exp((\Bx \cdot \Bi)(\Bk \cdot \Bi))
=
\exp((\Bx \cdot \Bi) \cdot (\Bk \cdot \Bi) +
     (\Bx \cdot \Bi) \wedge (\Bk \cdot \Bi) )
\end{equation}

\subsection{Characterization in real numbers}

Now that it has been verified that equations \eqnref{eqn:gaLaplacianSol:expgeo1} and \eqnref{eqn:gaLaplacianSol:expgeo2} are solutions
of \eqnref{eqn:gaLaplacianSol:laplacian} let us characterize this in terms of real numbers.

If \(\Bx\), and \(\Bk\) are colinear, the solution has the form

\begin{equation}
e^{\pm\Bx \cdot \Bk}
\end{equation}

(ie: purely hyperbolic solutions).

Whereas with \(\Bx\) and \(\Bk\) orthogonal we have can employ the unit bivector for the plane spanned by these vectors
\(\Bi = \frac{\Bx \wedge \Bk}{\abs{\Bx \wedge \Bk}}\):

\begin{equation}
e^{\pm\Bx \wedge \Bk} = \cos\abs{\Bx \wedge \Bk} \pm \Bi\sin\abs{\Bx \wedge \Bk}
\end{equation}

Or:
\begin{equation}
e^{\pm\Bx \wedge \Bk} = \cos\left(\frac{\Bx \wedge \Bk}{\Bi}\right) \pm \Bi\sin\left(\frac{\Bx \wedge \Bk}{\Bi}\right)
\end{equation}

(ie: purely trigonometric solutions)

Provided \(\Bx\), and \(\Bk\) are not colinear, the wedge product component of the above can be written in terms of a unit bivector
\(\Bi = \frac{\Bx \wedge \Bk}{\abs{\Bx \wedge \Bk}}\):

\begin{equation}\label{eqn:laplace:342}
\begin{aligned}
e^{\Bx\Bk} &= e^{\Bx \cdot \Bk + \Bx \wedge \Bk} \\
&= e^{\Bx \cdot \Bk} \left( \cos{\abs{\Bx \wedge \Bk}} + \Bi \sin{\abs{\Bx \wedge \Bk}} \right) \\
&= e^{\Bx \cdot \Bk} \left( \cos\left(\frac{\Bx \wedge \Bk}{\Bi}\right) + \Bi \sin\left(\frac{\Bx \wedge \Bk}{\Bi}\right) \right) \\
\end{aligned}
\end{equation}

And, for the reverse:
\begin{equation}\label{eqn:laplace:362}
\begin{aligned}
(e^{\Bx\Bk})^\dagger = e^{\Bk\Bx}
&= e^{\Bx \cdot \Bk} \left( \cos{\abs{\Bx \wedge \Bk}} - \Bi \sin\left(\abs{\Bx \wedge \Bk}\right) \right) \\
&= e^{\Bx \cdot \Bk} \left( \cos\left(\frac{\Bx \wedge \Bk}{\Bi}\right) - \Bi \sin\left(\frac{\Bx \wedge \Bk}{\Bi}\right) \right) \\
\end{aligned}
\end{equation}

This exponential however has both scalar and bivector parts, and we are looking for a strictly scalar result, so we can use linear combinations of the
exponential and its reverse to form a strictly real sum for the \(\Bx \wedge \Bk \ne 0\) cases:

\begin{equation}\label{eqn:laplace:382}
\begin{aligned}
\inv{2}\left(e^{\Bx\Bk} + e^{\Bk\Bx}\right) = e^{\Bx\cdot\Bk}\cos\left(\frac{\Bx \wedge \Bk}{\Bi}\right) \\
\inv{2\Bi}\left(e^{\Bx\Bk} - e^{\Bk\Bx}\right) = e^{\Bx\cdot\Bk}\sin{\frac{\Bx \wedge \Bk}{\Bi}} \\
\end{aligned}
\end{equation}

Also note that further linear combinations (with positive and negative variations of \(\Bk\)) can be taken, so we can
combine equations \eqnref{eqn:gaLaplacianSol:expgeo1} and \eqnref{eqn:gaLaplacianSol:expgeo2} into the following real valued, coordinate free, form:

\begin{equation}\label{eqn:laplace:402}
\begin{aligned}
\cosh(\Bx\cdot\Bk)\cos\left({\frac{\Bx \wedge \Bk}{\Bi}}\right) \\
\sinh(\Bx\cdot\Bk)\cos\left({\frac{\Bx \wedge \Bk}{\Bi}}\right) \\
\cosh(\Bx\cdot\Bk)\sin\left({\frac{\Bx \wedge \Bk}{\Bi}}\right) \\
\sinh(\Bx\cdot\Bk)\sin\left({\frac{\Bx \wedge \Bk}{\Bi}}\right)
\end{aligned}
\end{equation}

Observe that the ratio \(\frac{\Bx \wedge \Bk}{\Bi}\) is just a scalar
determinant

\begin{equation}\label{eqn:laplace:242}
\frac{\Bx \wedge \Bk}{\Bi}
=
x_j k_m - x_m k_j
\end{equation}

So one is free to choose \(k' = k_m \Be_j - k_j \Be_m\), in which case the
solution takes the alternate form:

\begin{equation}\label{eqn:laplace:422}
\begin{aligned}
\cos(\Bx\cdot\Bk')\cosh\left({\frac{\Bx \wedge \Bk'}{\Bi}}\right) \\
\sin(\Bx\cdot\Bk')\cosh\left({\frac{\Bx \wedge \Bk'}{\Bi}}\right) \\
\cos(\Bx\cdot\Bk')\sinh\left({\frac{\Bx \wedge \Bk'}{\Bi}}\right) \\
\sin(\Bx\cdot\Bk')\sinh\left({\frac{\Bx \wedge \Bk'}{\Bi}}\right)
\end{aligned}
\end{equation}

These sets of equations and the exponential form both remove the explicit reference to the pair of coordinates used in the original restriction

\begin{equation}\label{eqn:laplace:262}
\gpgradetwo{ \Bx \wedge \Bk \Be_j \wedge \Bk } = 0
\end{equation}

that was used in the proof that \(e^{\Bx\Bk}\) was a solution.

   \chapter{Hyper complex numbers and symplectic structure}
      %
% Copyright � 2012 Peeter Joot.  All Rights Reserved.
% Licenced as described in the file LICENSE under the root directory of this GIT repository.
%

%
%
%\chapter{Hyper complex numbers and symplectic structure}
\index{hypercomplex numbers}
\index{symplectic structure}
\label{chap:complex}
%\date{November 8, 2008.  complex.tex}

\section{On 4.2 Hermitian Norms and Unitary Groups}

These are some rather rough notes filling in some details
on the treatment of \citep{DoranHamiltonian}.

Expanding equation 4.17

\begin{equation}\label{eqn:complex:20}
\begin{aligned}
J &= e_i \wedge f_i \\
a &= u_i e_i + v_i f_i \\
b &= x_i e_i + y_i f_i \\
B &= a \wedge b + (a \cdot J) \wedge (b \cdot J)  \\
\end{aligned}
\end{equation}

\begin{equation}\label{eqn:complex:40}
\begin{aligned}
a \wedge b
&= (u_i e_i + v_i f_i) \wedge (x_j e_j + y_j f_j) \\
&=
u_i x_j e_i \wedge e_j
+ u_i y_j e_i \wedge f_j
+ v_i x_j f_i \wedge e_j
+ v_i y_j f_i \wedge f_j \\
\end{aligned}
\end{equation}

\begin{equation}\label{eqn:complex:60}
\begin{aligned}
a \cdot J
&=
u_i e_i \cdot ( e_j \wedge f_j )
+ v_i f_i \cdot ( e_j \wedge f_j ) \\
&= u_j f_j - v_j e_j
\end{aligned}
\end{equation}

Search and replace for \(b \cdot J\) gives

\begin{equation}\label{eqn:complex:80}
\begin{aligned}
b \cdot J
&=
x_i e_i \cdot ( e_j \wedge f_j )
+ y_i f_i \cdot ( e_j \wedge f_j ) \\
&= x_j f_j - y_j e_j
\end{aligned}
\end{equation}

So we have

\begin{equation}\label{eqn:complex:100}
\begin{aligned}
(a \cdot J) \wedge (b \cdot J)
&= (u_i f_i - v_i e_i) \wedge (x_j f_j - y_j e_j) \\
&=
 u_i x_j f_i \wedge f_j
-u_i y_j f_i \wedge e_j
- v_i x_j e_i \wedge f_j
+ v_i y_j e_i \wedge e_j
\end{aligned}
\end{equation}

For
\begin{equation}\label{eqn:complex:120}
\begin{aligned}
a \wedge b + (a \cdot J) \wedge (b \cdot J)
&=
 ( u_i y_j - v_i x_j ) (e_i \wedge f_j - f_i \wedge e_j)
+ ( u_i x_j + v_i y_j ) (e_i \wedge e_j + f_i \wedge f_j)
\end{aligned}
\end{equation}

This shows why the elements were picked as a basis
\begin{equation}\label{eqn:complex:140}
\begin{aligned}
e_i \wedge f_j - f_i \wedge e_j
\end{aligned}
\end{equation}
\begin{equation}\label{eqn:complex:160}
\begin{aligned}
e_i \wedge e_j + f_i \wedge f_j
\end{aligned}
\end{equation}

The first of which is a multiple of \(J_i = e_i \wedge f_i\) when \(i=j\), and the second of which is zero if \(i=j\).

\section{5.1 Conservation Theorems and Flows}

equation 5.10 is

\begin{equation}\label{eqn:complex:180}
\begin{aligned}
\fdot = \xdot \cdot \grad f = (\grad f \wedge \grad H) \cdot J
\end{aligned}
\end{equation}

This one is not obvious to me.  For \(\fdot\) we have

\begin{equation}\label{eqn:complex:200}
\begin{aligned}
\fdot = \PD{p_i}{f} \pdot_i +\PD{q_i}{f} \qdot_i + \mathLabelBox{\PD{t}{f}}{\(=0\)}
\end{aligned}
\end{equation}

compare to

\begin{equation}\label{eqn:complex:220}
\begin{aligned}
\xdot \cdot \grad f
&= (\pdot_i e_i + \qdot_i f_i) \cdot (e_j \PD{p_j}{f} + f_j \PD{q_j}{f}) \\
&= \pdot_i \PD{p_i}{f} + \qdot_i \PD{q_i}{f}
\end{aligned}
\end{equation}

Okay, this part matches the first part of (5.10).  Writing this in terms of the Hamiltonian relation (5.9) \(\xdot = \grad H \cdot J\) we have

\begin{equation}\label{eqn:complex:240}
\begin{aligned}
\fdot
&= (\grad H \cdot J) \cdot \grad f \\
&= \grad f \cdot (\grad H \cdot J) \\
\end{aligned}
\end{equation}

The relation \(a \cdot (b \cdot (c \wedge d)) = (a \wedge b) \cdot (c \wedge d)\),
can be used here to factor out the \(J\), we have
\begin{equation}\label{eqn:complex:260}
\begin{aligned}
\fdot
&= \grad f \cdot (\grad H \cdot J) \\
&= (\grad f \wedge \grad H) \cdot J \\
\end{aligned}
\end{equation}

which completes (5.10).

Also with \(f=H\) since H was also specified as having no explicit time dependence, one has

\begin{equation}\label{eqn:complex:280}
\begin{aligned}
\dot{H} &= (\grad H \wedge \grad H) \cdot J = 0 \cdot J = 0
\end{aligned}
\end{equation}

%\bibliographystyle{plainnat}
%\bibliography{myrefs}

%\end{document}

   \chapter{Newton's method for intersection of curves in a plane}
      %
% Copyright � 2012 Peeter Joot.  All Rights Reserved.
% Licenced as described in the file LICENSE under the root directory of this GIT repository.
%

%
%
%\input{../peeter_prologue_print.tex}
%\input{../peeter_prologue_widescreen.tex}

%\chapter{Newton's method for intersection of curves in a plane}
\index{Newton's method}
\index{intersection}
\label{chap:intersectionNewton}

%\blogpage{http://sites.google.com/site/peeterjoot/math2010/intersectionNewton.pdf}
%\date{Mar 7, 2010}
%\revisionInfo{intersectionNewton.tex}

%\beginArtWithToc
\beginArtNoToc

\section{Motivation}

Reading the blog post \href{http://www.cafelinear.com/2010/02/problem-solving-artificial-intelligence-and-linear-algebra/}{Problem solving, artificial intelligence and computational linear algebra} some variations of Newton's method for finding local minimums and maximums are given.

While I had seen the Hessian matrix eons ago in the context of back propagation feedback methods, Newton's method itself I remember as a first order root finding method.  Here I refresh my memory what that simpler Newton's method was about, and build on that slightly to find the form of the solution for the intersection of an arbitrarily oriented line with a curve, and finally the problem of refining an approximation for the intersection of two curves using the same technique.

\section{Root finding as the intersection with a horizontal}

\imageFigure{../gabook-figures/newtonsIntersectionHorizontal}{Refining an approximate horizontal intersection}{fig:newtonsIntersectionHorizontal}{0.5}

The essence of Newton's method for finding roots is following the tangent from the point of first guess down to the line that one wants to intersect with the curve.  This is illustrated in \cref{fig:newtonsIntersectionHorizontal}.

Algebraically, the problem is that of finding the point \(x_1\), which is given by the tangent

\begin{align}\label{eqn:intersectionNewton:10}
\frac{f(x_0) - b}{x_0 - x_1} = f'(x_0).
\end{align}

Rearranging and solving for \(x_1\), we have

\begin{align}\label{eqn:intersectionNewton:11}
x_1 = x_0 - \frac{f(x_0) - b}{f'(x_0)}
\end{align}

If one presumes convergence, something not guaranteed, then a first guess, if good enough, will get closer and closer to the target with each iteration.  If this first guess is far from the target, following the tangent line could ping pong you to some other part of the curve, and it is possible not to find the root, or to find some other one.

\section{Intersection with a line}

\imageFigure{../gabook-figures/newtonsIntersectionAnyOrientation}{Refining an approximation for the intersection with an arbitrarily oriented line}{fig:newtonsIntersectionAnyOrientation}{0.5}

The above pictorial treatment works nicely for the intersection of a horizontal line with a curve.  Now consider the intersection of an arbitrarily oriented line with a curve, as illustrated in \cref{fig:newtonsIntersectionAnyOrientation}.  Here it is useful to setup the problem algebraically from the beginning.  Our problem is really still just that of finding the intersection of two lines.  The curve itself can be considered the set of end points of the vector

\begin{align}\label{eqn:intersectionNewton:20}
\Br(x) = x \Be_1 + f(x) \Be_2,
\end{align}

for which the tangent direction vector is

\begin{align}\label{eqn:intersectionNewton:21}
\Bt(x) = \frac{d\Br}{dx} = \Be_1 + f'(x) \Be_2.
\end{align}

The set of points on this tangent, taken at the point \(x_0\), can also be written as a vector, namely

\begin{align}\label{eqn:intersectionNewton:22}
(x_0, f(x)) + \alpha \Bt(x_0).
\end{align}

For the line to intersect this, suppose we have one point on the line \(\Bp_0\), and a direction vector for that line \(\ucap\).  The points on this line are therefore all the endpoints of

\begin{align}\label{eqn:intersectionNewton:23}
\Bp_0 + \beta \ucap.
\end{align}

Provided that the tangent and the line of intersection do in fact intersect then our problem becomes finding \(\alpha\) or \(\beta\) after equating \eqnref{eqn:intersectionNewton:22} and \eqnref{eqn:intersectionNewton:23}.  This is the solution of

\begin{align}\label{eqn:intersectionNewton:24}
(x_0, f(x_0)) + \alpha \Bt(x_0) = \Bp_0 + \beta \ucap.
\end{align}

Since we do not care which of \(\alpha\) or \(\beta\) we solve for, setting this up as a matrix equation in two variables is not the best approach.  Instead we wedge both sides with \(\Bt(x_0)\) (or \(\ucap\)), essentially using Cramer's method.  This gives

\begin{align}\label{eqn:intersectionNewton:25}
\left((x_0, f(x_0)) -\Bp_0 \right) \wedge \Bt(x_0) = \beta \ucap \wedge \Bt(x_0).
\end{align}

If the lines are not parallel, then both sides are scalar multiples of \(\Be_1 \wedge \Be_2\), and dividing out one gets

\begin{align}\label{eqn:intersectionNewton:26}
\beta = \frac{\left((x_0, f(x_0)) -\Bp_0 \right) \wedge \Bt(x_0)}{\ucap \wedge \Bt(x_0)}.
\end{align}

Writing out \(\Bt(x_0) = \Be_1 + f'(x_0) \Be_2\), explicitly, this is

\begin{align}\label{eqn:intersectionNewton:26b}
\beta = \frac{\left((x_0, f(x_0)) -\Bp_0 \right) \wedge \left(\Be_1 + f'(x_0) \Be_2\right)}{\ucap \wedge \left(\Be_1 + f'(x_0) \Be_2\right)}.
\end{align}

Further, dividing out the common \(\Be_1 \wedge \Be_2\) bivector, we have a ratio of determinants

\begin{align}\label{eqn:intersectionNewton:26c}
\beta =
\frac{
\begin{vmatrix}
x_0 -\Bp_0 \cdot \Be_1 & f(x_0) - \Bp_0 \cdot \Be_2 \\
1 & f'(x_0) \\
\end{vmatrix}
}
{
\begin{vmatrix}
\ucap \cdot \Be_1 & \ucap \cdot \Be_2 \\
1 & f'(x_0) \\
\end{vmatrix}
}.
\end{align}

The final step in the solution is noting that the point of intersection is just

\begin{align}\label{eqn:intersectionNewton:27}
\Bp_0 + \beta \ucap,
\end{align}

and in particular, the \(x\) coordinate of this is the desired result of one step of iteration

\begin{align}\label{eqn:intersectionNewton:28}
x_1 = \Bp_0 \cdot \Be_1 + (\ucap \cdot \Be_1)
\frac{
\begin{vmatrix}
x_0 -\Bp_0 \cdot \Be_1 & f(x_0) - \Bp_0 \cdot \Be_2 \\
1 & f'(x_0) \\
\end{vmatrix}
}
{
\begin{vmatrix}
\ucap \cdot \Be_1 & \ucap \cdot \Be_2 \\
1 & f'(x_0) \\
\end{vmatrix}
}.
\end{align}

This looks a whole lot different than the original \(x_1\) for the horizontal from back at \eqnref{eqn:intersectionNewton:11}, but substitution of \(\ucap = \Be_1\), and \(\Bp_0 = b \Be_2\), shows that these are identical.

\section{Intersection of two curves}

Can we generalize this any further?  It seems reasonable that we would be able to use this Newton's method technique of following the tangent to refine an approximation for the intersection point of two general curves.  This is not expected to be much harder, and the geometric idea is illustrated in \cref{fig:newtonsIntersectionTwoCurves}

\imageFigure{../gabook-figures/newtonsIntersectionTwoCurves}{Refining an approximation for the intersection of two curves in a plane}{fig:newtonsIntersectionTwoCurves}{0.5}

The task at hand is to setup this problem algebraically.  Suppose the two curves \(s(x)\), and \(r(x)\) are parametrized as vectors

\begin{align}\label{eqn:intersectionNewton:40}
\Bs(x) &= x \Be_1 + s(x) \Be_2 \\
\Br(x) &= x \Be_1 + r(x) \Be_2.
\end{align}

Tangent direction vectors at the point \(x_0\) are then

\begin{align}\label{eqn:intersectionNewton:41}
\Bs'(x_0) &= \Be_1 + s'(x_0) \Be_2 \\
\Br'(x_0) &= \Be_1 + r'(x_0) \Be_2.
\end{align}

The intersection of interest is therefore the solution of

\begin{align}\label{eqn:intersectionNewton:42}
(x_0, s(x_0)) + \alpha \Bs' = (x_0, r(x_0)) + \beta \Br'.
\end{align}

Wedging with one of tangent vectors \(\Bs'\) or \(\Br'\) provides our solution.  Solving for \(\alpha\) this is

\begin{align}\label{eqn:intersectionNewton:43}
\alpha = \frac{(0, r(x_0) - s(x_0)) \wedge \Br'}{\Bs' \wedge \Br'}
=
\frac
{
\begin{vmatrix}
0 & r(x_0) - s(x_0) \\
\Br' \cdot \Be_1 & \Br' \cdot \Be_2
\end{vmatrix}
}
{
\begin{vmatrix}
\Bs' \cdot \Be_1 & \Bs' \cdot \Be_2 \\
\Br' \cdot \Be_1 & \Br' \cdot \Be_2
\end{vmatrix}
}
=
-
\frac
{
r(x_0) - s(x_0)
}
{
r'(x_0) - s'(x_0)
}.
\end{align}

To finish things off, we just have to calculate the new \(x\) coordinate on the line for this value of \(\alpha\), which gives us

\begin{align}\label{eqn:intersectionNewton:44}
x_1 = x_0 -
\frac
{
r(x_0) - s(x_0)
}
{
r'(x_0) - s'(x_0)
}.
\end{align}

It is ironic that generalizing things to two curves leads to a tidier result than the more specific line and curve result from \eqnref{eqn:intersectionNewton:28}.  With a substitution of \(r(x) = f(x)\), and \(s(x) = b\), we once again recover the result \eqnref{eqn:intersectionNewton:11}, for the horizontal line intersecting a curve.

\section{Followup}

Having completed the play that I set out to do, the next logical step would be to try the min/max problem that leads to the Hessian.  That can be for another day.

%%\EndArticle
%\EndNoBibArticle

   \chapter{Center of mass of a toroidal segment}
      %
% Copyright � 2012 Peeter Joot.  All Rights Reserved.
% Licenced as described in the file LICENSE under the root directory of this GIT repository.
%

%
%
%\input{../peeter_prologue_print.tex}
%\input{../peeter_prologue_widescreen.tex}

\chapter{Center of mass of a toroidal segment}
\index{center of mass!toroidal segment}
\label{chap:torusCenterOfMass}

%\blogpage{http://sites.google.com/site/peeterjoot/math2010/torusCenterOfMass.pdf}
%\date{May 15, 2010}
%\revisionInfo{torusCenterOfMass.tex}

%\beginArtWithToc
\beginArtNoToc

\section{Motivation}

In \href{http://samjshah.com/2010/05/05/i-love-when-kids-stump-me/}{I love when kids stump me}, the center of mass of a toroidal segment is desired, and the simpler problem of a circular ring segment is considered.

Let us try the solid torus problem for fun using the geometric algebra toolbox.  To setup the problem, it seems reasonable to introduce two angle, plus radius, toroidal parametrization as shown in \cref{fig:toriodalSegment}.

\imageFigure{../figures.gabook/toriodalSegment}{Toroidal parametrization}{fig:toriodalSegment}{0.5}

Our position vector to a point within the torus is then

\begin{subequations}
\begin{align}\label{eqn:torusCenterOfMass:1}
\Br(\rho, \theta, \phi) &= e^{-j\theta/2} \left( \rho \Be_1 e^{ i \phi } + R \Be_3 \right) e^{j \theta/2} \\
i &= \Be_1 \Be_3 \\
j &= \Be_3 \Be_2
\end{align}
\end{subequations}

Here \(i\) and \(j\) for the bivectors are labels picked at random.  They happen to have the quaternion-ic properties \(i j = -j i\), and \(i^2 = j^2 = -1\) which can be verified easily.

\section{Volume element}

Before we can calculate the center of mass, we will need the volume element.  I do not recall having ever seen such a volume element, so let us calculate it from scratch.

We want

\begin{align}\label{eqn:torusCenterOfMass:2}
dV = \pm \Be_1 \Be_2 \Be_3 \left( \PD{\rho}{\Br} \wedge \PD{\theta}{\Br} \wedge \PD{\phi}{\Br} \right) d\rho d\theta d\phi,
\end{align}

so the first order of business is calculation of the partials.  After some regrouping those are

\begin{subequations}
\begin{align}\label{eqn:torusCenterOfMass:3}
\PD{\rho}{\Br} &= e^{-j\theta/2} \Be_1 e^{ i \phi } e^{j \theta/2} \\
\PD{\theta}{\Br}
%&= e^{-j\theta/2} \left( \rho \inv{2} \left( -\Be_3 \Be_2 \Be_1 e^{ i \phi } + \Be_1 e^{ i \phi } \Be_3 \Be_2 \right) + R \Be_2 \right) e^{j \theta/2} \\
&= e^{-j\theta/2} \left( R + \rho \sin\phi \right) \Be_2 e^{j \theta/2} \\
\PD{\phi}{\Br} &= e^{-j\theta/2} \rho \Be_3 e^{ i \phi } e^{j \theta/2}.
\end{align}
\end{subequations}

For the volume element we want the wedge of each of these, and can instead select the trivector grades of the products, which conveniently wipes out a number of the interior exponentials

\begin{align}\label{eqn:torusCenterOfMass:4}
\PD{\rho}{\Br} \wedge \PD{\theta}{\Br} \wedge \PD{\phi}{\Br}
&=
\rho \left( R + \rho \sin\phi \right) \gpgradethree{ e^{-j\theta/2} \Be_1 e^{ i \phi } \Be_2 \Be_3 e^{ i \phi } e^{j \theta/2} }
\end{align}

Note that \(\Be_1\) commutes with \(j = \Be_3 \Be_2\), so also with \(e^{-j\theta/2}\).  Also \(\Be_2 \Be_3 = -j\) anticommutes with \(i\), so we have a conjugate commutation effect \(e^{i\phi} j = j e^{-i\phi}\).  Together the trivector grade selection reduces almost magically to just

\begin{align}\label{eqn:torusCenterOfMass:5}
\PD{\rho}{\Br} \wedge \PD{\theta}{\Br} \wedge \PD{\phi}{\Br}
&=
\rho \left( R + \rho \sin\phi \right) \Be_1 \Be_2 \Be_3
\end{align}

Thus the volume element, after taking the positive sign, is

\begin{align}\label{eqn:torusCenterOfMass:6}
dV = \rho \left( R + \rho \sin\phi \right) d\rho d\theta d\phi.
\end{align}

As a check we should find that we can use this to calculate the volume of the complete torus, and obtain the expected \(V = (2 \pi R) (\pi r^2)\) result.  That volume is

\begin{align}\label{eqn:torusCenterOfMass:7}
V = \int_{\rho=0}^r \int_{\theta=0}^{2\pi} \int_{\phi=0}^{2\pi} \rho \left( R + \rho \sin\phi \right) d\rho d\theta d\phi.
\end{align}

The sine term conveniently vanishes over the \(2\pi\) interval, leaving just
\begin{align}\label{eqn:torusCenterOfMass:8}
V = \inv{2} r^2 R (2 \pi)(2 \pi),
\end{align}

as expected.

\section{Center of mass}

With the prep done, we are ready to move on to the original problem.  Given a toroidal segment over angle \(\theta \in [-\Delta \theta/2, \Delta \theta/2]\), then the volume of that segment is

\begin{align}\label{eqn:torusCenterOfMass:9}
\Delta V = r^2 R \pi \Delta \theta.
\end{align}

Our center of mass position vector is then located at

\begin{align}\label{eqn:torusCenterOfMass:10}
\BR \Delta V
&=
\int_{\rho=0}^r \int_{\theta=-\Delta \theta/2}^{\Delta \theta/2} \int_{\phi=0}^{2\pi}
e^{-j\theta/2} \left( \rho \Be_1 e^{ i \phi } + R \Be_3 \right) e^{j \theta/2}
\rho \left( R + \rho \sin\phi \right) d\rho d\theta d\phi.
\end{align}

Evaluating the \(\phi\) integrals we loose the \(\int_0^{2\pi} e^{i\phi}\) and \(\int_0^{2\pi} \sin\phi\) terms and are left with \(\int_0^{2\pi} e^{i\phi} \sin\phi d\phi = i \pi /2\) and \(\int_0^{2\pi} d\phi = 2 \pi\).  This leaves us with

\begin{align}\label{eqn:torusCenterOfMass:11}
\BR \Delta V
&=
\int_{\rho=0}^r \int_{\theta=-\Delta \theta/2}^{\Delta \theta/2}
\left( e^{-j\theta/2} \rho^3 \Be_3 \frac{\pi}{2} e^{j \theta/2} + 2 \pi \rho R^2 \Be_3 e^{j \theta}  \right) d\rho d\theta \\
&=
\int_{\theta=-\Delta \theta/2}^{\Delta \theta/2}
\left( e^{-j\theta/2} r^4 \Be_3 \frac{\pi}{8} e^{j \theta/2} + 2\pi \inv{2} r^2 R^2 \Be_3 e^{j \theta}  \right) d\theta \\
&=
\int_{\theta=-\Delta \theta/2}^{\Delta \theta/2}
\left( e^{-j\theta/2} r^4 \Be_3 \frac{\pi}{8} e^{j \theta/2} + \pi r^2 R^2 \Be_3 e^{j \theta}  \right) d\theta.
\end{align}

Since \(\Be_3 j = -j \Be_3\), we have a conjugate commutation with the \(e^{-j \theta/2}\) for just

\begin{align}\label{eqn:torusCenterOfMass:12}
\BR \Delta V
&=
\pi r^2 \left( \frac{r^2}{8} + R^2 \right) \Be_3
\int_{\theta=-\Delta \theta/2}^{\Delta \theta/2}
e^{j \theta} d\theta \\
&=
\pi r^2 \left( \frac{r^2}{8} + R^2 \right) \Be_3
2 \sin(\Delta \theta/2).
\end{align}

A final reassembly, provides the desired final result for the center of mass vector

\begin{align}\label{eqn:torusCenterOfMass:13}
\BR &= \Be_3 \inv{R} \left( \frac{r^2}{8} + R^2 \right) \frac{\sin(\Delta \theta/2)}{ \Delta \theta/2 }.
\end{align}

Presuming no algebraic errors have been made, how about a couple of sanity checks to see if the correctness of this seems plausible.

We are pointing in the \(z\)-axis direction as expected by symmetry.  Good.  For \(\Delta \theta = 2 \pi\), our center of mass vector is at the origin.  Good, that is also what we expected.  If we let \(r \rightarrow 0\), and \(\Delta \theta \rightarrow 0\), we have \(\BR = R \Be_3\) as also expected for a tiny segment of ``wire'' at that position.  Also good.

\section{Center of mass for a circular wire segment}

As an additional check for the correctness of the result above, we should be able to compare with the center of mass of a circular wire segment, and get the same result in the limit \(r \rightarrow 0\).

For that we have

\begin{align}\label{eqn:torusCenterOfMass:14}
Z (R \Delta \theta) = \int_{\theta=-\Delta \theta/2}^{\Delta \theta/2} R i e^{-i\theta} R d\theta
\end{align}

So we have

\begin{align}\label{eqn:torusCenterOfMass:15}
Z
&= \inv{\Delta \theta} R i \inv{-i} (e^{-i\Delta \theta/2} - e^{i\Delta\theta/2}).
\end{align}

Observe that this is

\begin{align}\label{eqn:torusCenterOfMass:16}
Z &= R i \frac{\sin(\Delta\theta/2)}{\Delta\theta/2},
\end{align}

which is consistent with the previous calculation for the solid torus when we let that solid diameter shrink to zero.

In particular, for \(3/4\) of the torus, we have \(\Delta \theta = 2 \pi (3/4) = 3 \pi/2\), and

\begin{align}\label{eqn:torusCenterOfMass:17}
Z = R i \frac{4 \sin(3\pi/4)}{3 \pi} = R i \frac{2 \sqrt{2}}{3 \pi} \approx 0.3 R i.
\end{align}

We are a little bit up the imaginary axis as expected.

I had initially somehow thought I had been off by a factor of two compared to \href{http://samjshah.com/2010/05/05/i-love-when-kids-stump-me/#comment-2349}{the result by The Virtuosi}, without seeing a mistake in either.  But that now appears not to be the case, and I just screwed up plugging in the numbers.  Once again, I should go to my eight year old son when I have arithmetic problems, and restrict myself to just the calculus and algebra bits.

%%\EndArticle
%\EndNoBibArticle


%%
% Copyright � 2016 Peeter Joot.  All Rights Reserved.
% Licenced as described in the file LICENSE under the root directory of this GIT repository.
%
%{
\input{../latex/blogpost.tex}
\renewcommand{\basename}{whatsTheDifference}
%\renewcommand{\dirname}{notes/phy1520/}
\renewcommand{\dirname}{notes/ece1228-electromagnetic-theory/}
%\newcommand{\dateintitle}{}
%\newcommand{\keywords}{}

\input{../latex/peeter_prologue_print2.tex}

\usepackage{peeters_layout_exercise}
\usepackage{peeters_braket}
\usepackage{peeters_figures}
\usepackage{siunitx}
%\usepackage{txfonts} % \ointclockwise

\beginArtNoToc

\generatetitle{XXX}
%\chapter{XXX}
%\label{chap:whatsTheDifference}

Exterior algebra defines an antisymmetric wedge product.  An example of the wedge product of two vectors, called a two-form (unit vectors in this case) is
\begin{equation}\label{eqn:whatsTheDifference:20}
\Be_1 \wedge \Be_2 = -\Be_2 \wedge \Be_1.
\end{equation}

An example of a wedge product of three (unit) vectors, a three-form, is
\begin{equation}\label{eqn:whatsTheDifference:40}
\begin{aligned}
\Be_1 \wedge \Be_2 \wedge \Be_3
&= -\Be_2 \wedge \Be_1 \wedge \Be_3 \\
&= \Be_2 \wedge \Be_3 \wedge \Be_1 \\
&= -\Be_3 \wedge \Be_2 \wedge \Be_1.
\end{aligned}
\end{equation}

A consequence of this antisymmetry is that any wedge product where one of the wedged vectors is colinear with another is zero.  Exterior algebra also has the concept of duality, which provides a mapping between k-forms and N-k forms, where N is the dimension of the underlying vector space.  For example, in a three dimensional Euclidean space the dual of the two form \( \Be_1 \wedge \Be_2 \), denoted \( *\lr{ \Be_1 \wedge \Be_2} \) is the quantity
\begin{equation}\label{eqn:whatsTheDifference:60}
*\lr{\Be_1 \wedge \Be_2} \wedge \lr{ \Be_1 \wedge \Be_2} = \Be_1 \wedge \Be_2 \wedge \Be_3,
\end{equation}
so
\begin{equation}\label{eqn:whatsTheDifference:80}
*\lr{\Be_1 \wedge \Be_2} = \Be_3.
\end{equation}

In an exterior algebra, one can add k-forms to other k-forms, but would not add forms of different rank.  This restriction is relaxed in Geometric Algebra (GA), where a quantity such as
\begin{equation}\label{eqn:whatsTheDifference:100}
1 + 2 \Be_1 + 3 \Be_2 \wedge \Be_4 + 5 \Be_1 \wedge \Be_2 \wedge \Be_4,
\end{equation}
is perfectly well formed.  The Geometric Algebra is built up of products of vectors, where the vector product is defined as an associative product
\begin{equation}\label{eqn:whatsTheDifference:120}
\Ba (\Bb \Bc) = (\Ba \Bb) \Bc = \Ba \Bb \Bc,
\end{equation}
and where the product of a vector with itself is defined as the squared length of that vector
\begin{equation}\label{eqn:whatsTheDifference:140}
\Ba \Ba = \Ba \cdot \Ba = \Abs{\Ba}^2.
\end{equation}

In an Euclidean space such length is always positive, but that mixed sign length metrics (such as that of the Minkowski space used in special relativity) are also allowed.

The product of two non-colinear vectors can be factored as
\begin{equation}\label{eqn:whatsTheDifference:160}
\Ba \Bb = \inv{2} \lr{ \Ba \Bb + \Bb \Ba } + \inv{2} \lr{ \Ba \Bb - \Bb \Ba }.
\end{equation}

The first (symmetric) term can be identified with the dot-product, whereas the second completely antisymmetric term can be identified as with the wedge product, so this complete vector product is denoted
\begin{equation}\label{eqn:whatsTheDifference:180}
\Ba \Bb = \Ba \cdot \Bb + \Ba \wedge \Bb.
\end{equation}

This is one of the simplest examples of what is called a multivector in GA, containing the sum of a scalar (grade zero) and a bivector (grade two).  There are a number of other consequences of the product axioms of GA.  One such consequence is that the product of two perpendicular vectors is antisymmetric, and that any unit vector has a unit square.  A number of specific algebraic structures can be represented with Geometric Algebras.  For example, one can identify the algebra spanned by a scalar and unit bivector, such as
\begin{equation}\label{eqn:whatsTheDifference:200}
\Span \setlr{ 1, \Be_1 \Be_2 }
\end{equation}
with complex numbers.  This is because any unit bivector of this form (in a Euclidean space) squares to unity
\begin{equation}\label{eqn:whatsTheDifference:220}
\begin{aligned}
(\Be_1 \Be_2)^2
&= (\Be_1 \Be_2)(\Be_1 \Be_2) \\
&= \Be_1 (\Be_2 \Be_1) \Be_2 \\
&= -\Be_1 (\Be_1 \Be_2) \Be_2 \\
&= -(\Be_1 \Be_1) (\Be_2 \Be_2) \\
&= - (1)(1) \\
&= -1.
\end{aligned}
\end{equation}

Other examples of algebraic structures that can have GA representations include quaternions, the Pauli (spin) algebra of quantum mechanics, and the Dirac algebra from QED.

The GA representation of dual vectors is through multiplication by a (unit) pseudoscalar, often denoted \( I \), for the vector space (an ordered product of all the unit vectors of the space).  For example, negative multiplication by the three dimensional pseudoscalar has the duality property illustrated in the exterior algebra duality example
\begin{equation}\label{eqn:whatsTheDifference:240}
\begin{aligned}
-I \Be_1
&=
-\Be_1 \Be_2 \Be_3 \Be_1 \\
&=
\Be_1 \Be_2 \Be_1 \Be_3 \\
&=
\Be_2 \Be_3,
\end{aligned}
\end{equation}

\begin{equation}\label{eqn:whatsTheDifference:260}
\begin{aligned}
-I
\Be_2 \Be_3
&=
- \Be_1 \Be_2 \Be_3 \Be_2 \Be_3 \\
&=
\Be_1 \Be_2 \Be_2 \Be_3 \Be_3 \\
&=
\Be_1.
\end{aligned}
\end{equation}

A number of fundamental geometric operations, such as projection, rotation, and reflection can all be represented using GA multivector product operations.

In GA the basis vectors for the space are typically real valued vectors.  Clifford algebras provide a further generalization, allowing those basis vectors to reside in a complex vector space.

All of these algebras are linear algebras.  In an exterior algebra
\begin{equation}\label{eqn:whatsTheDifference:280}
\begin{aligned}
(c \Ba) \wedge \Bb
&=
c (\Ba \wedge \Bb) \\
&=
\Ba \wedge (c \Bb),
\end{aligned}
\end{equation}
and
\begin{equation}\label{eqn:whatsTheDifference:300}
\Ba \wedge (b \Bb + c \Bc) = b \Ba \wedge \Bb + c \Ba \wedge \Bc,
\end{equation}
or in GA
\begin{equation}\label{eqn:whatsTheDifference:320}
\Ba \lr{ b \Bb + c \Bc \Bd }
=
b \Ba \Bb + c \Ba \Bc \Bd.
\end{equation}
%}
\EndArticle
%\EndNoBibArticle

