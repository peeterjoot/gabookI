%
% Copyright © 2016 Peeter Joot.  All Rights Reserved.
% Licenced as described in the file LICENSE under the root directory of this GIT repository.
%
\maketheorem{Distribution of inner products}{thm:stokesTheoremGeometricAlgebra:1420}{
Given two blades \(A_s, B_r\) with grades subject to \(s > r > 0\), and a vector \(b\), the inner product distributes according to
\begin{equation*}
A_s \cdot \lr{ b \wedge B_r } = \lr{ A_s \cdot b } \cdot B_r.
\end{equation*}
}

This will allow us, for example, to expand a general inner product of the form \(d^k \Bx \cdot (\boldpartial \wedge F)\).

The proof is straightforward, but also mechanical.  Start by expanding the wedge and dot products within a grade selection operator

\begin{dmath}\label{eqn:stokesTheoremGeometricAlgebra:1460}
A_s \cdot \lr{ b \wedge B_r }
=
\gpgrade{A_s (b \wedge B_r)}{s - (r + 1)}
=
\inv{2} \gpgrade{A_s \lr{b B_r + (-1)^{r} B_r b} }{s - (r + 1)}
\end{dmath}

Solving for \(B_r b\) in

\begin{dmath}\label{eqn:stokesTheoremGeometricAlgebra:1480}
2 b \cdot B_r = b B_r - (-1)^{r} B_r b,
\end{dmath}

we have

\begin{dmath}\label{eqn:stokesTheoremGeometricAlgebra:1500}
A_s \cdot \lr{ b \wedge B_r }
=
\inv{2} \gpgrade{ A_s b B_r + A_s \lr{ b B_r - 2 b \cdot B_r } }{s - (r + 1)}
=
\gpgrade{ A_s b B_r }{s - (r + 1)}
-
\cancel{\gpgrade{ A_s \lr{ b \cdot B_r } }{s - (r + 1)}}.
\end{dmath}

The last term above is zero since we are selecting the \(s - r - 1\) grade element of a multivector with grades \(s - r + 1\) and \(s + r - 1\), which has no terms for \(r > 0\).  Now we can expand the \(A_s b\) multivector product, for

\begin{dmath}\label{eqn:stokesTheoremGeometricAlgebra:1520}
A_s \cdot \lr{ b \wedge B_r }
=
\gpgrade{ \lr{ A_s \cdot b + A_s \wedge b} B_r }{s - (r + 1)}.
\end{dmath}

The latter multivector (with the wedge product factor) above has grades \(s + 1 - r\) and \(s + 1 + r\), so this selection operator finds nothing.  This leaves

\begin{dmath}\label{eqn:stokesTheoremGeometricAlgebra:1540}
A_s \cdot \lr{ b \wedge B_r }
=
\gpgrade{
\lr{ A_s \cdot b } \cdot B_r
+ \lr{ A_s \cdot b } \wedge B_r
}{s - (r + 1)}.
\end{dmath}

The first dot products term has grade \(s - 1 - r\) and is selected, whereas the wedge term has grade \(s - 1 + r \ne s - r - 1\) (for \(r > 0\)).  \(\qedmarker\)

%Next consider an expansion that we cannot do above, but require

