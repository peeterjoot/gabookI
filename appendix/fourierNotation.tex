%
% Copyright � 2012 Peeter Joot.  All Rights Reserved.
% Licenced as described in the file LICENSE under the root directory of this GIT repository.
%

%
%
\chapter{A cheatsheet for Fourier transform conventions}
\index{Fourier transform}
\label{chap:fourierNotation}
%\date{Jan 21, 2009.  fourierNotation.tex}

\section{A cheatsheet for different Fourier integral notations}

Damn.  There are too many different notations for the Fourier transform.  Examples are:

\begin{equation}\label{eqn:fourierNotation:20}
\begin{aligned}
\tilde{f}(k) &= \IIinf f(x) \exp\left( - 2 \pi i k x \right) dx \\
\tilde{f}(k) &= \inv{\sqrt{2\pi}} \IIinf f(x) \exp\left( -i k x \right) dx \\
\tilde{f}(p) &= \sqrt{\frac{1}{2\pi\Hbar}} \IIinf f(x) \exp\left( \frac{-i p x}{\Hbar} \right) dx \\
\end{aligned}
\end{equation}

There are probably many more, with other variations such as using hats over things instead of twiddles, and so forth.

Unfortunately each of these have different numeric factors for the inverse transform.
Having just been bitten by rogue factors of \(2 \pi\) after innocently switching notations, it seems
worthwhile
to express the Fourier transform with a general fudge factor in the exponential.  Then it can be
seen at a
glance what constants are required in the inverse transform given anybody's particular choice of the transform
definition.

Where to put all the factors can actually be seen from the QM formulation since one is free to treat \(\Hbar\)
as an arbitrary constant, but let us do it from scratch in a mechanical fashion without having to think
back to QM as a fundamental.

Suppose we define the Fourier transform as

\begin{equation}\label{eqn:fourierNotation:40}
\begin{aligned}
\tilde{f}(s) = \kappa \IIinf f(x) \exp\left( - i \alpha s x \right) dx \\
{f}(x) = \kappa' \IIinf \tilde{f}(s) \exp\left( i \alpha x s \right) ds \\
\end{aligned}
\end{equation}

Now, what factor do we need in the inverse transform to make things work out right?  With the Rigor
Police on holiday, let us expand the inverse transform integral in terms of the original transform
and see what these numeric factors must then be to make this work out.

Omitting temporarily the \(\kappa\) factors to be determined we have

\begin{equation}\label{eqn:fourierNotation:60}
\begin{aligned}
f(x)
&\propto \IIinf \left( \IIinf f(u) \exp\left( - i \alpha s u \right) du \right) \exp\left( i \alpha x s \right) ds \\
&= \IIinf f(u) du \IIinf \exp\left( i \alpha s (x-u) \right) ds \\
&= \IIinf f(u) du \lim_{R\rightarrow \infty} 2 \pi \inv{ \pi \alpha(x-u) }\sin\left( \alpha R (x-u) \right) \\
&= \IIinf f(u) du 2 \pi \delta\left( \alpha (x-u) \right) \\
&= \inv{\alpha} \IIinf f(v/\alpha) dv 2 \pi \delta\left( \alpha x - v \right) \\
&= \frac{2 \pi}{\alpha} f((\alpha x)/\alpha) \\
&= \frac{2 \pi}{\alpha} f(x)
\end{aligned}
\end{equation}

Note that to get the result above, after switching order of integration, and assuming that we can take the principle value of the integrals,
the usual ad-hoc \(\sinc\) and exponential integral identification of the delta function was made

\begin{equation}\label{eqn:fourierNotation:80}
\begin{aligned}
\PV \inv{2\pi} \IIinf \exp\left( i  s  x \right) ds
&= \lim_{R \rightarrow \infty} \inv{2\pi} \int_{-R}^R \exp\left( i  s  x \right) ds \\
&= \lim_{R\rightarrow \infty} \frac{\sin\left(  R  x \right)}{ \pi  x } \\
&\equiv \delta\left( x \right) \\
\end{aligned}
\end{equation}

The end result is that we will need to fix

\begin{equation}\label{eqn:fourierNotation:100}
\begin{aligned}
\kappa \kappa' = \frac{\alpha}{2\pi}
\end{aligned}
\end{equation}

to have the transform pair produce the desired result.  Our transform pair is therefore

\begin{equation}\label{eqn:fourier_notation:blah}
\begin{aligned}
\tilde{f}(s) = \kappa \IIinf f(x) \exp\left( - i \alpha s x \right) dx \Leftrightarrow f(x) = {\frac{ \alpha} {2 \pi \kappa} } \IIinf \tilde{f}(s) \exp\left( i \alpha s x \right) ds
\end{aligned}
\end{equation}

\section{A survey of notations}

From \eqnref{eqn:fourier_notation:blah} we can
express the required numeric factors that accompany all the various forward transforms conventions.  Let us do a quick survey of the bookshelf, ignoring differences in the \(i\)'s and \(j\)'s, differences in the transform variables, and so forth.

From my old systems and signals course, with the book \citep{haykin1994cs} we have, \(\kappa = 1\), and \(\alpha = 2 \pi\)

\begin{equation}\label{eqn:fourierNotation:120}
\begin{aligned}
\tilde{f}(s) &= \IIinf f(x) \exp\left( - 2 \pi i s x \right) dx \\
f(x) &= \IIinf \tilde{f}(s) \exp\left( 2 \pi i s x \right) ds \\
\end{aligned}
\end{equation}

The mathematician's preference, and that of
\citep{bohm1989qt}, and \citep{byron1992mca} appears to be the nicely symmetrical version, with \(\kappa = 1/\sqrt{2\pi}\), and \(\alpha = 1\)

\begin{equation}\label{eqn:fourierNotation:140}
\begin{aligned}
\tilde{f}(s) &= \inv{\sqrt{2\pi}} \IIinf f(x) \exp\left( - i  s x \right) dx \\
f(x) &= \inv{\sqrt{2 \pi}} \IIinf \tilde{f}(s) \exp\left( i  s x \right) ds \\
\end{aligned}
\end{equation}

From the old circuits course using \citep{irwin2007bec}, and also in the excellent text \citep{lepage1980cva}, we have \(\kappa = 1\), and \(\alpha = 1\)

\begin{equation}\label{eqn:fourierNotation:160}
\begin{aligned}
\tilde{f}(s) &= \IIinf f(x) \exp\left( - i  s x \right) dx  \\
f(x) &= \inv{{2 \pi}} \IIinf \tilde{f}(s) \exp\left( i  s x \right) ds \\
\end{aligned}
\end{equation}

and finally, the QM specific version from \citep{mcmahon2005qmd}, with \(\alpha = p/\Hbar\), and \(\kappa = 1/\sqrt{2\pi \Hbar}\) we have

\begin{equation}\label{eqn:fourierNotation:180}
\begin{aligned}
\tilde{f}(p) &= \inv{\sqrt{2\pi \Hbar}} \IIinf f(x) \exp\left( - \frac{i  p x}{\Hbar} \right) dx \\
f(x) &= \inv{\sqrt{2 \pi \Hbar}} \IIinf \tilde{f}(p) \exp\left( \frac{i  p x}{\Hbar} \right) dp \\
\end{aligned}
\end{equation}
