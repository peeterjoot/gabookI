%
% Copyright � 2012 Peeter Joot.  All Rights Reserved.
% Licenced as described in the file LICENSE under the root directory of this GIT repository.
%

%
%
\chapter{Projection with generalized dot product}\label{chap:PJprojGen}
\index{projection}
%\date{May 17, 2008.  projGeneralizedDotProd.tex}

\imageFigure{../figures.gabook/visualize_subspace_projection}{Visualizing projection onto a subspace}{fig:Projection_subspace}{0.4}

We can geometrically visualize the projection problem
as in \cref{fig:Projection_subspace}.  Here
the subspace can be pictured
as a plane containing a set of mutually perpendicular basis vectors, as if
one has visually projected all the higher dimensional vectors onto a plane.

For a vector \(\Bx\) that contains some part not in the space we want to find
the component in the space \(\Bp\), or characterize the projection operation
that produces this vector, and also find the space of vectors that lie
perpendicular to the space.

Expressed in terms of
the Euclidean dot product this perpendicularity can be expressed explicitly as
\(U^\T \Bn = 0\).
This is why we say that \(\Bn\) is in the null space of \(U^\T\),
\(N(U^\T)\) not the null space of \(U\) itself (\(N(U)\)).  One perhaps could say this is in the null
or perpendicular space of the set \(\{u_i\}\), but the typical preference to use columns as
vectors makes this not entirely unnatural.

In a complex vector space with \(\Bu \cdot \Bv = \Bu^* \Bv\) transposition no longer expresses this
null space concept, so the null space
is the set of \(\Bn\), such that \(U^* \Bn = 0\), so one would say \(\Bn \in N(U^*)\).

One can generalize this projection and nullity to more general dot products.  Let us examine the projection
matrix calculation with respect to a more arbitrary inner product.  For an inner product that is conjugate linear in the first variable, and linear in second variable
we can write:

\begin{equation}\label{eqn:projGeneralizedDotProd:20}
\innerprod{\Bu}{\Bv} = \Bu^* A \Bv
\end{equation}

This is the most general complex bilinear form, and can thus represent any complex dot product.

The problem is the same as above.  We want to repeat the projection derivation
done with the Euclidean dot product, but
be more careful with ordering of terms since we now using a non-commutative dot (inner) product.

We are looking for vectors \(\Bp = \sum a_i \Bu_i\), and \(\Be\) such that
\begin{equation}\label{eqn:projGeneralizedDotProd:40}
\Bx = \Bp + \Be
\end{equation}

If the inner product defines the projection operation we have for any \(\Bu_i\)

\begin{equation}\label{eqn:projGeneralizedDotProd:60}
\begin{aligned}
0 &= \innerprod{\Bu_i}{\Be} \\
  &= \innerprod{\Bu_i}{\Bx - \Bp} \\
\implies \\
\innerprod{\Bu_i}{\Bx} &= \innerprod{\Bu_i}{\Bp} \\
                       &= \innerprod{\Bu_i}{\sum_j a_j \Bu_j} \\
                       &= \sum_j a_j \innerprod{\Bu_i}{\Bu_j} \\
\end{aligned}
\end{equation}

In matrix form, this is
\begin{equation*}
{
\begin{bmatrix}
\innerprod{\Bu_i}{\Bx}
\end{bmatrix}
}_i
=
{
\begin{bmatrix}
\innerprod{\Bu_i}{\Bu_j}
\end{bmatrix}
}_{ij}
[a_i]_i
\end{equation*}

Or
\begin{equation*}
A = [a_i]_i =
\inv{
   {
   \begin{bmatrix}
   \innerprod{\Bu_i}{\Bu_j}
   \end{bmatrix}
   }_{ij}
}
{
\begin{bmatrix}
\innerprod{\Bu_i}{\Bx}
\end{bmatrix}
}_i
\end{equation*}

We can also write our projection in terms of \(A\):

\begin{equation*}
\Bp =
\begin{bmatrix}
\Bu_1 & \Bu_2 & \cdots & \Bu_k
\end{bmatrix}
A
= U A
\end{equation*}

Thus the projection vector can be written:

\begin{equation*}
\Bp = U
\inv{
   {
   \begin{bmatrix}
   \innerprod{\Bu_i}{\Bu_j}
   \end{bmatrix}
   }_{ij}
}
{
\begin{bmatrix}
\innerprod{\Bu_i}{\Bx}
\end{bmatrix}
}_i
\end{equation*}

In matrix form this is:

\begin{equation}
\Proj_U(\Bx) = \left(U \inv{U^* A U} U^* A\right) \Bx
\end{equation}

Writing \(W^* = U^* A\), this is

\begin{equation*}
\Proj_U(\Bx) = \left(U \inv{W^* U} W^* A\right) \Bx
\end{equation*}

which is what the wikipedia article on projection calls an oblique projection.  Q: Can any oblique projection be expressed using just an alternate dot product?
