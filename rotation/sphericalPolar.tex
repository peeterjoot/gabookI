%
% Copyright � 2012 Peeter Joot.  All Rights Reserved.
% Licenced as described in the file LICENSE under the root directory of this GIT repository.
%

%
%
\chapter{Spherical polar coordinates}
\index{spherical polar coordinates}
\label{chap:sphericalPolar}
%\date{Nov 13, 2008.  sphericalPolar.tex}

\section{Motivation}

Reading the math intro of \citep{zeilik1998iaa}, I found the statement that the gradient in spherical polar form is:

\begin{equation}\label{eqn:sphericalPolar:20}
\begin{aligned}
\grad &=
\rcap \PD{r}{}
+\thetacap \inv{r} \PD{\theta}{}
+\phicap \inv{r \sin\theta}\PD{\phi}{}
\end{aligned}
\end{equation}

There was no picture or description showing the conventions for measurement of the angles or directions.
To clarify things and leave a margin note I decided to derive the coordinates and unit vector transformation relationships,
gradient, divergence and curl in spherical polar coordinates.

Although details for this particular result can be found in many texts,
including the excellent review article \citep{fleischCoords}, the
exercise of personally working out the details was thought to be
a worthwhile
learning exercise.  Additionally, some related ideas about rotating
frame systems seem worth exploring, and that will be done here.

\section{Notes}
\subsection{Conventions}

\imageFigure{../../figures/gabook/spherical_polar}{Angles and lengths in spherical polar coordinates}{fig:spherical_polar}{0.4}

\Cref{fig:spherical_polar} illustrates the conventions used in
these notes.  By inspection, the coordinates can be read off the diagram.

\begin{equation}\label{eqn:sphericalPolar:coordinates}
\begin{aligned}
u &= r \cos\phi \\
x &= u \cos\theta = r \cos\phi \cos\theta \\
y &= u \sin\theta = r \cos\phi \sin\theta \\
z &= r \sin\phi
\end{aligned}
\end{equation}

\subsection{The unit vectors}

To calculate the unit vectors \(\rcap\), \(\thetacap\), \(\phicap\) in the spherical polar frame we need to apply two sets of rotations.  The first is a rotation
in the \(x,y\) plane, and the second in the \(x', z\) plane.

For the intermediate frame after just the \(x,y\) plane rotation we have

\begin{equation}\label{eqn:sphericalPolar:40}
\begin{aligned}
R_\theta &= \exp(-\Be_{12}\theta/2) \\
\Be_i' &= R_\theta \Be_i R_\theta^\dagger
\end{aligned}
\end{equation}

Now for the rotational plane for the \(\phi\) rotation is

\begin{equation}\label{eqn:sphericalPolar:60}
\begin{aligned}
\Be_1' \wedge \Be_3
&= (R_\theta \Be_1 R_\theta^\dagger) \wedge \Be_3 \\
&= \inv{2} ( R_\theta \Be_1 R_\theta^\dagger \Be_3 - \Be_3 R_\theta \Be_1 R_\theta^\dagger ) \\
\end{aligned}
\end{equation}

The rotor (or quaternion) \(R_\theta\) has scalar and \(\Be_{12}\) components, so it commutes with \(\Be_3\) leaving

\begin{equation}\label{eqn:sphericalPolar:80}
\begin{aligned}
\Be_1' \wedge \Be_3
&= R_\theta \inv{2} ( \Be_1 \Be_3 - \Be_3 \Be_1 ) R_\theta^\dagger \\
&= R_\theta \Be_1 \wedge \Be_3 R_\theta^\dagger \\
\end{aligned}
\end{equation}

Therefore the rotor for the second stage rotation is

\begin{equation}\label{eqn:sphericalPolar:100}
\begin{aligned}
R_\phi
&= \exp( - R_\theta \Be_1 \wedge \Be_3 R_\theta^\dagger \phi/2 ) \\
&= \sum \inv{k!} \left( - R_\theta \Be_1 \wedge \Be_3 R_\theta^\dagger \phi/2 \right)^k \\
&= R_\theta \sum \inv{k!} ( - \Be_1 \wedge \Be_3 \phi/2 )^k R_\theta^\dagger \\
&= R_\theta \exp( - \Be_{13} \phi/2 ) R_\theta^\dagger \\
\end{aligned}
\end{equation}

Composing both sets of rotations one has

\begin{equation}\label{eqn:sphericalPolar:120}
\begin{aligned}
R(\Bx)
&= R_\theta \exp( - \Be_{13} \phi/2 ) R_\theta^\dagger R_\theta \Bx R_\theta^\dagger R_\theta \exp( \Be_{13} \phi/2 ) R_\theta^\dagger \\
&= \exp( - \Be_{12} \theta/2 ) \exp( - \Be_{13} \phi/2 ) \Bx \exp( \Be_{13} \phi/2 ) \exp( \Be_{12} \theta/2 ) \\
\end{aligned}
\end{equation}

Or, more compactly

\begin{equation}\label{eqn:sphericalPolar:140}
\begin{aligned}
R(\Bx) &= R \Bx R^\dagger \\
R &= R_\theta R_\phi \\
R_\phi &= \exp(-\Be_{13}\phi/2) \\
R_\theta &= \exp(-\Be_{12}\theta/2)
\end{aligned}
\end{equation}

Application of these to the \(\{\Be_i\}\) basis produces the \(\{\rcap, \thetacap, \phicap\}\) basis.  First application
of \(R_\phi\) yields the basis vectors for the intermediate rotation.

\begin{equation}\label{eqn:sphericalPolar:160}
\begin{aligned}
\begin{array}{l l l}
{R_\phi}\Be_1 {R_\phi}^\dagger &= \Be_1 (\cos\phi + \Be_{13} \sin\phi) &= \Be_1 \cos\phi + \Be_3 \sin\phi \\
{R_\phi}\Be_2 {R_\phi}^\dagger &= \Be_2 R_\phi {R_\phi}^\dagger &= \Be_2 \\
{R_\phi}\Be_3 {R_\phi}^\dagger &= \Be_3 (\cos\phi + \Be_{13} \sin\phi) &= \Be_3 \cos\phi - \Be_1 \sin\phi \\
\end{array}
\end{aligned}
\end{equation}

Applying the second rotation to \(R_\phi(\Be_i)\) we have
\begin{equation}\label{eqn:sphericalPolar:180}
\begin{aligned}
\rcap
&= {R_\theta}( \Be_1 \cos\phi + \Be_3 \sin\phi ) {R_\theta}^\dagger \\
&=
\Be_1 \cos\phi (\cos\theta + \Be_{12} \sin\theta)
+ \Be_3 \sin\phi \\
&=
\Be_1 \cos\phi \cos\theta
+ \Be_2 \cos\phi \sin\theta
+ \Be_3 \sin\phi \\
\thetacap
&= {R_\theta} ( \Be_2 ) {R_\theta}^\dagger \\
&= \Be_2 (\cos\theta + \Be_{12} \sin\theta) \\
&= - \Be_1 \sin\theta + \Be_2 \cos\theta \\
\phicap
&= {R_\theta} ( \Be_3 \cos\phi - \Be_1 \sin\phi ) {R_\theta}^\dagger \\
&= \Be_3 \cos\phi - \Be_1 \sin\phi (\cos\theta + \Be_{12} \sin\theta) \\
&=
- \Be_1 \sin\phi \cos\theta
- \Be_2 \sin\phi \sin\theta
+ \Be_3 \cos\phi
\\
\end{aligned}
\end{equation}

In summary these are

\begin{equation}\label{eqn:sphericalPolar:unitTxSummary}
\begin{aligned}
\rcap &= \Be_1 \cos\phi \cos\theta + \Be_2 \cos\phi \sin\theta + \Be_3 \sin\phi \\
\thetacap &= - \Be_1 \sin\theta + \Be_2 \cos\theta \\
\phicap &= - \Be_1 \sin\phi \cos\theta - \Be_2 \sin\phi \sin\theta + \Be_3 \cos\phi
\end{aligned}
\end{equation}

\subsection{An alternate pictorial derivation of the unit vectors}

Somewhat more directly, \(\rcap\) can be calculated from the coordinate expression of \eqnref{eqn:sphericalPolar:coordinates}

\begin{equation}\label{eqn:sphericalPolar:200}
\begin{aligned}
\rcap
&= \inv{r} (x, y, z),
\end{aligned}
\end{equation}

which was found by inspection of the diagram.

For \(\thetacap\), again from the figure, observe that it lies in an
latitudinal plane (ie: \(x,y\) plane), and is perpendicular to the outwards radial vector in that plane.  That is

\begin{equation}\label{eqn:sphericalPolar:220}
\begin{aligned}
\thetacap
&= (\cos\theta \Be_1 + \sin\theta \Be_2) \Be_1 \Be_2 \\
\end{aligned}
\end{equation}

Lastly, \(\phicap\) can be calculated from the dual of \(\rcap \wedge \thetacap\)

\begin{equation}\label{eqn:sphericalPolar:240}
\begin{aligned}
\phicap
&= - \Be_1 \Be_2 \Be_3 (\rcap \wedge \thetacap) \\
\end{aligned}
\end{equation}

Completing the algebra for the expressions above we have
\begin{equation}\label{eqn:sphericalPolar:260}
\begin{aligned}
\rcap
&=
\cos\phi \cos\theta \Be_1
+ \cos\phi \sin\theta \Be_2
+ \sin\phi \Be_3 \\
\thetacap
&= \cos\theta \Be_2 - \sin\theta \Be_1 \\
\rcap \wedge \thetacap
%&=
%\sin\phi \sin\theta \Be_1 \Be_3
%- \cos\theta \sin\phi \Be_2 \Be_3
%+ ( \cos\phi \sin\theta^2 + \cos\phi \cos\theta^2 ) \Be_1 \Be_2  \\
&=
\sin\phi \sin\theta \Be_1 \Be_3
+ \sin\phi \cos\theta \Be_3 \Be_2
+ \cos\phi \Be_1 \Be_2 \\
\phicap
&=
- \sin\phi \cos\theta \Be_1
- \sin\phi \sin\theta \Be_2
+ \cos\phi \Be_3 %\\
\end{aligned}
\end{equation}

Sure enough this produces the same result as with the rotor logic.

The rotor approach was purely algebraically and does not have
the same reliance on pictures.  That may have
an
additional advantage
since one can then
study any frame transformations of the general form \(\{\Be_i'\} = \{ R \Be_i R^\dagger \}\), and produce results
that apply to
not only spherical polar coordinate systems but others such as the cylindrical polar.

\subsection{Tensor transformation}

Considering a linear transformation providing a mapping from one basis to another of the following form

\begin{equation}\label{eqn:sphericalPolar:280}
\begin{aligned}
f_i = \LL(e_i) = L e_i L^{-1}
\end{aligned}
\end{equation}

The coordinate representation, or Fourier decomposition, of the vectors in each of these frames is

\begin{equation}\label{eqn:sphericalPolar:300}
\begin{aligned}
x = x^i e_i = y^j f_j.
\end{aligned}
\end{equation}

Utilizing a reciprocal frame (ie: not yet requiring an orthonormal frame here), such that \(e^i \cdot e_j = {\delta^i}_j\),
then dot product provide the coordinate transformations
\begin{equation}\label{eqn:sphericalPolar:320}
\begin{aligned}
x^k e_k \cdot e^k &= y^j f_j \cdot e^k \\
y^j f_j \cdot f^i &= x^k e_k \cdot f^i \\
\implies \\
x^i &= y^j f_j \cdot e^i \\
y^i &= x^j e_j \cdot f^i
\end{aligned}
\end{equation}

The transformed reciprocal frame vectors can be expressed directly in terms of the initial reciprocal frame \(f^i = \LL(e^i)\).  Taking
dot products confirms this

\begin{equation}\label{eqn:sphericalPolar:340}
\begin{aligned}
(L e_i L^{-1}) \cdot (L e^j L^{-1})
&= \gpgradezero{ L e_i L^{-1} L e^j L^{-1} } \\
&= \gpgradezero{ L e_i e^j L^{-1} } \\
&= e_i \cdot e^j \gpgradezero{ L L^{-1} } \\
&= e_i \cdot e^j
\end{aligned}
\end{equation}

This implies that the forward and inverse coordinate transformations may be summarized as
\begin{equation}\label{eqn:sphericalPolar:360}
\begin{aligned}
y^i &= x^j e_j \cdot \LL(e^i) \\
x^i &= y^j \LL(e_j) \cdot e^i \\
\end{aligned}
\end{equation}

Or in matrix form
\begin{equation}\label{eqn:sphericalPolar:coordinateTxTensors}
\begin{aligned}
\Lor{i}{j} &= \LL(e^i) \cdot e_j \\
\ILor{i}{j} &= \LL(e_j) \cdot e^i \\
y^i &= \Lor{i}{j} x^j \\
x^i &= \ILor{i}{j} y^j
\end{aligned}
\end{equation}


The use of inverse notation is justified by the following

\begin{equation}\label{eqn:sphericalPolar:380}
\begin{aligned}
x^i &= \ILor{i}{k} y^k \\
&= \ILor{i}{k} \Lor{k}{j} x^j \\
\implies \\
\ILor{i}{k} \Lor{k}{j} &= \delta^i_j
\end{aligned}
\end{equation}

For the special case where the basis is orthonormal (\(e_i \cdot e^j = {\delta_i}^j\)), then it can be observed here that the inverse must also be the
transpose since the forward and reverse transformation tensors then differ only be a swap of indices.

On notation.  Some references such as \citep{MinahanTensors} use \(\Lor{i}{j}\) for both the forward and inverse transformations, with specific conventions
about which index is varied to distinguish the two matrices.  I have found that confusing and have instead used the explicit inverse notation
of \citep{SpenceTensors}.



\subsection{Gradient after change of coordinates}

With the transformation matrices enumerated above we are now equipped to take the gradient expressed in initial frame
\begin{equation}\label{eqn:sphericalPolar:400}
\begin{aligned}
\grad = \sum e^i \PD{x^i}{},
\end{aligned}
\end{equation}

and express it in the transformed frame.  The chain rule is required for the derivatives in terms of the transformed coordinates

\begin{equation}\label{eqn:sphericalPolar:420}
\begin{aligned}
\PD{x^i}{}
&= \PD{x^i}{y^j} \PD{y^j}{} \\
&= \Lor{j}{i} \PD{y^j}{} \\
&= \LL(e^j) \cdot e_i \PD{y^j}{} \\
&= f^j \cdot e_i \PD{y^j}{}
\end{aligned}
\end{equation}

Therefore the gradient is
\begin{equation}\label{eqn:sphericalPolar:440}
\begin{aligned}
\grad &= \sum e^i (f^j \cdot e_i) \PD{y^j}{} \\
      &= \sum f^j \PD{y^j}{} \\
\end{aligned}
\end{equation}

This gets us most of the way towards the desired result for the spherical polar gradient since all that remains is a calculation of the \(\PDi{y^j}{}\)
values for
each of the \(\rcap\), \(\thetacap\), and \(\phicap\) directions.

It is also interesting to observe (as in \citep{DenkerMaxwell}) that the gradient can also be written as

\begin{equation}\label{eqn:sphericalPolar:460}
\begin{aligned}
\grad &= \inv{f_j} \PD{y^j}{} \\
\end{aligned}
\end{equation}

Observe the similarity to the Fourier component decomposition of the vector itself \(x = f_i y^i\).  Thus, roughly speaking, the differential operator
parts of the gradient can be seen to be directional derivatives
along the directions of each of the frame vectors.

This is sufficient to read the elements of distance in each of the directions
off the figure

\begin{equation}\label{eqn:sphericalPolar:480}
\begin{aligned}
\delta \Bx \cdot \rcap &= \delta r \\
\delta \Bx \cdot \thetacap &= r \cos\phi \delta \theta \\
\delta \Bx \cdot \phicap &= r \delta \theta \\
\end{aligned}
\end{equation}

Therefore the gradient is just
\begin{equation}\label{eqn:sphericalPolar:500}
\begin{aligned}
\grad =
\rcap \PD{r}{}
+\thetacap \inv{r \cos\phi} \PD{\theta}{}
+\phicap \inv{r} \PD{\phi}{}
\end{aligned}
\end{equation}

Although this last bit has been
derived graphically, and not analytically, it does
clarify the original question of exactly angle and unit vector
conventions were intended in the text (polar angle measured from the North pole, not equator, and \(\theta\), and \(\phi\) reversed).

This was the long way to that particular result, but this has been
an exploratory treatment of frame rotation concepts that I personally
felt the need to clarity for myself.

There are still some additional details that I will explore before concluding
(including an analytic treatment of the above).

% FIXME: wrong!
%%%\subsection{Element of distance along the curves}
%%%
%%%Using the figure, one can observe that the distances along in each of the spherical polar unit vector directions are obtained in
%%%this particular case by varying each coordinate in turn.
%%%
%%%\begin{itemize}
%%%\item Along \(\rcap\), a vectorial element of distance is just
%%%
%%%\begin{align*}
%%%\delta \Bx_r
%%%&= (r + \delta r)\rcap - r\rcap \\
%%%&= \delta r \rcap.
%%%\end{align*}
%%%
%%%\item Along the \(\phicap\) direction?
%%%
%%%Here one wants the great circle path obtained by fixing \(\theta\) and \(r\).  A difference of such position vectors
%%%(in the standard basis) is
%%%
%%%\begin{align*}
%%%r
%%%\begin{bmatrix}
%%%(\cos(\phi + \delta \phi) - \cos(\phi)) \cos(\theta) \\
%%%(\cos(\phi + \delta \phi) - \cos(\phi)) \sin(\theta) \\
%%%\sin(\phi + \delta \phi) - \sin(\phi) \\
%%%\end{bmatrix}
%%%&\approx
%%%r
%%%\begin{bmatrix}
%%%-\sin(\phi) \cos(\theta) \\
%%%-\sin(\phi) \sin(\theta) \\
%%%\cos(\phi) \\
%%%\end{bmatrix}
%%%\delta \phi
%%%\end{align*}
%%%
%%%This is an unsatisfactory way to express the directed distance, since it is in terms of \(\Be_i\).  We have relationships
%%%for \(\rcap\), \(\thetacap\), \(\phicap\) in terms of \(\Be_i\) and could invert that and multiply it out, but that is going to
%%%make things even messier before things get simpler.
%%%
%%%Instead form the unit bivector for the north south oriented great circle plane through the point of interest
%%%
%%%\begin{align*}
%%%j = \rcap \wedge \phicap
%%%\end{align*}
%%%
%%%For a point \(\Bx\) we want to consider an incremental change in position along the \(\phicap\) direction.  Forming the
%%%projection and rejection from the plane we have
%%%
%%%\begin{align*}
%%%\Bx = \Bx j \inv{j} = (\Bx \cdot j) \inv{j} + (\Bx \wedge j) \inv{j} = \Bx_\parallel + \Bx_\perp
%%%\end{align*}
%%%
%%%So rotating to \(\Bx'\) and taking differences we have
%%%
%%%\begin{align*}
%%%\delta \Bx_\phi
%%%&= \Bx' - \Bx \\
%%%&= \Bx_\parallel' -\Bx_\parallel \\
%%%&= \Bx_\parallel (\exp(j\delta \phi) - 1) \\
%%%&\approx \Bx_\parallel j \delta \phi \\
%%%&= (\Bx \cdot j) \inv{j} j \delta \phi \\
%%%&= (\Bx \cdot j) \delta \phi
%%%\end{align*}
%%%
%%%\item How about along the \(\thetacap\) direction?
%%%Intuitively, one expects the magnitude to be \(r \delta \theta\).  The algebra will be exactly the same as with \(\phicap\) direction
%%%but we have a different bivector for the plane.  Let \(i = \Be_1\Be_2\) we have
%%%
%%%\begin{align*}
%%%\delta \Bx_\theta = (\Bx \cdot i) \delta \theta
%%%\end{align*}
%%%
%%%\end{itemize}
%%%
%%%Now all this is a bit inexact.  What exactly are these \(\delta\) increments?  They have to be small enough that they can be considered to be just along the
%%%time unit vectors for the rotated frame.

\section{Transformation of frame vectors vs. coordinates}

To avoid confusion it is worth noting how the frame vectors vs. the components themselves differ under
rotational transformation.

\subsection{Example.  Two dimensional plane rotation}

Consideration of the example of a pair of orthonormal unit vectors for the plane illustrates this

\begin{equation}\label{eqn:sphericalPolar:520}
\begin{aligned}
\Be_1' &= \Be_1 \exp(\Be_{12}\theta) = \Be_1 \cos\theta + \Be_2 \sin\theta \\
\Be_2' &= \Be_2 \exp(\Be_{12}\theta) = \Be_2 \cos\theta - \Be_1 \sin\theta \\
\end{aligned}
\end{equation}

Forming a matrix for the transformation of these unit vectors we have
\begin{equation}\label{eqn:sphericalPolar:540}
\begin{aligned}
\begin{bmatrix}
\Be_1' \\
\Be_2'
\end{bmatrix}
=
\begin{bmatrix}
\cos\theta & \sin\theta \\
- \sin\theta & \cos\theta \\
\end{bmatrix}
\begin{bmatrix}
\Be_1 \\
\Be_2
\end{bmatrix}
\end{aligned}
\end{equation}

Now compare this to the transformation of a vector in its entirety

\begin{equation}\label{eqn:sphericalPolar:560}
\begin{aligned}
y^1 \Be_1' + y^2 \Be_2'
&= ( x^1 \Be_1 + x^2 \Be_2 ) \exp(\Be_{12}\theta) \\
&= x^1(\Be_1 \cos\theta + \Be_2 \sin\theta)
 + x^2(\Be_2 \cos\theta - \Be_1 \sin\theta) \\
\end{aligned}
\end{equation}

If one uses the standard basis to specify both the rotated point and
the original, then taking dot products with \(\Be_i\) yields the
equivalent matrix representation

\begin{equation}\label{eqn:sphericalPolar:planeRotationAppliedToCoordinates}
\begin{aligned}
\begin{bmatrix}
y_1 \\
y_2
\end{bmatrix}
=
\begin{bmatrix}
\cos\theta & -\sin\theta \\
\sin\theta & \cos\theta \\
\end{bmatrix}
\begin{bmatrix}
x_1 \\
x_2
\end{bmatrix}
\end{aligned}
\end{equation}

Note how this inverts (transposes) the transformation matrix here
compared to the matrix for the transformation of the frame vectors.

%It should also be observed that the use of a matrix here has some subtlies.
%It is really the matrix of the linear transformation which takes
%coordinates from the initial frame to
%coordinates in the rotated frame.  The notation of \citep{damiano1988cla}
%makes this nicely explicit, and an equation like \eqnref{eqn:sphericalPolar:planeRotationAppliedToCoordinates}
%could be written as
%
%\begin{align*}
%{
%\begin{bmatrix}
%x
%\end{bmatrix}
%}_{\alpha'}
%=
%{
%\begin{bmatrix}
%R
%\end{bmatrix}
%}_{\alpha}^{\alpha'}
%{
%\begin{bmatrix}
%x
%\end{bmatrix}
%}_{\alpha}
%\end{align*}
%
%where \({\alpha} = \{\Be_i\}\) is the initial basis
%and \({\alpha}' = \{\Be_i'\}\) is the rotated basis.

\subsection{Inverse relations for spherical polar transformations}

The relations of \eqnref{eqn:sphericalPolar:unitTxSummary} can be summarized in matrix form

\begin{equation}\label{eqn:sphericalPolar:sphericalPolarRotationMatrixForFrameVectors}
\begin{aligned}
\begin{bmatrix}
\rcap \\
\thetacap \\
\phicap \\
\end{bmatrix}
&=
\begin{bmatrix}
%rcap . {e1 e2 e3}
\cos\phi \cos\theta & \cos\phi \sin\theta & \sin\phi \\
%thetacap . {e1 e2 e3}
- \sin\theta & \cos\theta & 0 \\
%phicap . {e1 e2 e3}
- \sin\phi \cos\theta & - \sin\phi \sin\theta & \cos\phi \\
\end{bmatrix}
\begin{bmatrix}
\Be_1 \\
\Be_2 \\
\Be_3 \\
\end{bmatrix}
\end{aligned}
\end{equation}

Or, more compactly
\begin{equation}\label{eqn:sphericalPolar:580}
\begin{aligned}
\begin{bmatrix}
\rcap \\
\thetacap \\
\phicap \\
\end{bmatrix}
=
\BU
\begin{bmatrix}
\Be_1 \\
\Be_2 \\
\Be_3 \\
\end{bmatrix}
\end{aligned}
\end{equation}

This composite rotation can be inverted with a transpose operation, which
becomes clear with the factorization
\begin{equation}\label{eqn:sphericalPolar:600}
\begin{aligned}
\BU
&=
\begin{bmatrix}
\cos\phi & 0 & \sin\phi \\
0 & 1 & 0 \\
-\sin\phi & 0 & \cos\phi
\end{bmatrix}
\begin{bmatrix}
\cos\theta & \sin\theta & 0 \\
-\sin\theta & \cos\theta & 0 \\
0 & 0 & 1 \\
\end{bmatrix}
\end{aligned}
\end{equation}

Thus
\begin{equation}\label{eqn:sphericalPolar:620}
\begin{aligned}
\begin{bmatrix}
\Be_1 \\
\Be_2 \\
\Be_3 \\
\end{bmatrix}
&=
% U^T :
%\begin{bmatrix}
%\cos\theta & -\sin\theta & 0 \\
%\sin\theta & \cos\theta & 0 \\
%0 & 0 & 1 \\
%\end{bmatrix}
%\begin{bmatrix}
%\cos\phi & 0 & -\sin\phi \\
%0 & 1 & 0 \\
%\sin\phi & 0 & \cos\phi
%\end{bmatrix}
%
\begin{bmatrix}
\cos\phi \cos\theta & - \sin\theta & - \sin\phi \cos\theta \\
\cos\phi \sin\theta & \cos\theta & - \sin\phi \sin\theta  \\
\sin\phi & 0 & \cos\phi \\
\end{bmatrix}
\begin{bmatrix}
\rcap \\
\thetacap \\
\phicap \\
\end{bmatrix}
\end{aligned}
\end{equation}

\subsection{Transformation of coordinate vector under spherical polar rotation}

In \eqnref{eqn:sphericalPolar:planeRotationAppliedToCoordinates} the matrix for the rotation of a coordinate vector for the plane rotation
was observed to be the transpose of the matrix that transformed the frame vectors themselves.  This is also the case
in this spherical polar case, as can be seen by forming a general vector and applying equation
\eqnref{eqn:sphericalPolar:sphericalPolarRotationMatrixForFrameVectors} to the standard basis vectors.

\begin{equation}\label{eqn:sphericalPolar:640}
\begin{aligned}
x^1 \Be_1 &\rightarrow x^1 ( \cos\phi \cos\theta \Be_1 + \cos\phi \sin\theta \Be_2 + \sin\phi \Be_3 ) \\
x^2 \Be_2 &\rightarrow x^2 ( - \sin\theta \Be_1 + \cos\theta \Be_2 ) \\
x^3 \Be_3 &\rightarrow x^3 ( - \sin\phi \cos\theta \Be_1 - \sin\phi \sin\theta \Be_2 + \cos\phi \Be_3 )
\end{aligned}
\end{equation}

Summing this and regrouping (ie: a transpose operation) one has:
\begin{equation}\label{eqn:sphericalPolar:660}
\begin{aligned}
x^i \Be_i &\rightarrow y^i \Be_i \\
& \Be_1 ( x^1 \cos\phi \cos\theta - x^2 \sin\theta - x^3 \sin\phi \cos\theta ) \\
& + \Be_2 ( x^1 \cos\phi \sin\theta + x^2 \cos\theta - x^3 \sin\phi \sin\theta ) \\
& + \Be_3 ( x^1 \sin\phi + x^3 \cos\phi ) \\
\end{aligned}
\end{equation}

taking dot products with \(\Be_i\) produces the matrix form
\begin{equation}\label{eqn:sphericalPolar:680}
\begin{aligned}
\begin{bmatrix}
y^1 \\
y^2 \\
y^3 \\
\end{bmatrix}
&=
\begin{bmatrix}
\cos\phi \cos\theta & - \sin\theta & - \sin\phi \cos\theta \\
\cos\phi \sin\theta & \cos\theta & - \sin\phi \sin\theta \\
\sin\phi & 0 & \cos\phi \\
\end{bmatrix}
\begin{bmatrix}
x^1 \\
x^2 \\
x^3 \\
\end{bmatrix} \\
&=
\begin{bmatrix}
\cos\theta & -\sin\theta & 0 \\
\sin\theta & \cos\theta & 0 \\
0 & 0 & 1 \\
\end{bmatrix}
\begin{bmatrix}
\cos\phi & 0 & -\sin\phi \\
0 & 1 & 0 \\
\sin\phi & 0 & \cos\phi
\end{bmatrix}
\begin{bmatrix}
x^1 \\
x^2 \\
x^3 \\
\end{bmatrix} \\
\end{aligned}
\end{equation}

As observed in \chapcite{PJEulerAngle} the matrix for this transformation of the coordinate vector under the composite \(x,y\) rotation followed by
an \(x', z\) rotation ends up expressed as the product of the elementary rotations, but applied in reverse order!

%%%\section{FIX or delete what comes after this}
%%%\subsection{Verification of rotation matrix, using only matrix notation}
%%%
%%%As a verification of \eqnref{eqn:sphericalPolar:sphericalPolarRotationMatrixForFrameVectors} lets calculate that directly.  The initial rotation is in the \(x,y\) plane around the \(-\Be_2' = -\Be_2 \exp(\Be_{12}\theta) = -\Be_2 \cos\theta + \Be_1 \sin\theta\) axis.
%%%
%%%From \citep{PJRotor} we have the rotation matrix for a \(\phi\) rotation about
%%%unit vector \(\Bn = (n_1, n_2, n_3) = (-\cos\theta, \sin\theta, 0) = (-C_\theta, S_\theta, 0)\) is
%%%
%%%\begin{align*}
%%%R_\phi R_\theta
%%%=
%%%\begin{bmatrix}
%%%\cos\phi(1 +{C_\theta}^2) - {C_\theta}^2 & -{C_\theta} {S_\theta} (1-\cos\phi) & {S_\theta} \sin\phi \\
%%%-{C_\theta} {S_\theta} (1-\cos\phi) & \cos\phi(1 -{S_\theta}^2) + {S_\theta}^2 & {C_\theta} \sin\phi \\
%%%-{S_\theta} \sin\phi & {-C_\theta} \sin\phi & \cos\phi \\
%%%\end{bmatrix}
%%%\begin{bmatrix}
%%%C_\theta & -S_\theta & 0 \\
%%%S_\theta & C_\theta & 0 \\
%%%0 & 0 & 1 \\
%%%\end{bmatrix}
%%%\end{align*}
%%%
