%
% Copyright � 2012 Peeter Joot.  All Rights Reserved.
% Licenced as described in the file LICENSE under the root directory of this GIT repository.
%

%
%
%\chapter{Rotor Notes}\label{chap:rotor}
\index{rotor}
%\date{Feb 19, 2008.  rotor.tex}

\section{Rotations strictly in a plane}

For a plane rotation, a rotation does not have to
be expressed in terms of left and right half angle rotations, as is the case
with complex numbers.  Starting with this ``natural'' one sided rotation
we will see why the half angle double sided Rotor formula works.

\subsection{Identifying a plane with a bivector.  Justification}
Given a bivector \(\BB\), we can say this defines the orientation of a plane
(through the origin)
since for any vector in the plane we have \(\BB \wedge \Bx = 0\), or any vector
strictly normal to the plane \(\BB \cdot \Bx = 0\).

Note that this naturally compares
to the equation of a line (through the origin) expressed in terms of a
direction vector \(\Bb\),
where \(\Bb \wedge \Bx=0\) if \(\Bx\) lies on the line, and \(\Bb \cdot \Bx = 0\)
if \(\Bx\) is normal to the line.

Given this it is not unreasonable to identify the plane with its bivector.  This
will be done below, and it should be clear that
loose language such as ``the plane \(\BB\)'', should really be interpreted
as ``the plane with direction bivector \(\BB\)'', where the direction bivector
has the wedge and dot product properties noted above.

\subsection{Components of a vector in and out of a plane}

To calculate the components of a vector in and out of a plane, we can form
the product

\begin{equation}\label{eqn:rotor:20}
\Bx = \Bx \BB \inv{\BB} = \Bx \cdot \BB \inv{\BB} + \Bx \wedge \BB \inv{\BB}
\end{equation}

This is an orthogonal decomposition of the vector \(\Bx\) where the first
part is the projective term onto the plane \(\BB\), and the second is the rejective
term, the component not in the plane.  Let us verify this.

Write \(\Bx = \Bx_\parallel + \Bx_\perp\), where \(\Bx_\parallel\), and \(\Bx_\perp\) are the components of \(\Bx\) parallel and perpendicular to the plane.  Also write
\(\BB = \Bb_1 \wedge \Bb_2\), where \(\Bb_i\) are non-colinear vectors in the plane \(\BB\).

If \(\Bx = \Bx_\parallel\), a vector entirely in the plane \(\BB\), then one can
write

\begin{equation}\label{eqn:rotor:40}
\Bx = a_1\Bb_1 + a_2\Bb_2
\end{equation}

and the wedge product term is zero

\begin{equation}\label{eqn:rotor:740}
\begin{aligned}
\Bx \wedge \BB
&= \left( a_1\Bb_1 + a_2\Bb_2 \right) \wedge \Bb_1 \wedge \Bb_2 \\
&= a_1 ( \Bb_1 \wedge \Bb_1 ) \wedge \Bb_2
 - a_2 ( \Bb_2 \wedge \Bb_2 ) \wedge \Bb_1 \\
&= 0
\end{aligned}
\end{equation}

Thus the component parallel to the plane is composed strictly of the dot
product term

\begin{equation}
\Bx_\parallel = \Bx \cdot \BB \inv{\BB}
\end{equation}

Or for a general vector not necessarily in the plane the component
of that vector in the plane, its projection onto the plane is,

\begin{equation}\label{eqn:rotor:60}
\Proj_{\BB}(\Bx) = \Bx \cdot \BB \inv{\BB}
= \inv{\abs{\BB}^2}(\BB \cdot \Bx)\BB
= (\hat{\BB} \cdot \Bx)\hat{\BB}
\end{equation}

Now, for a vector that lies completely perpendicular to the plane \(\Bx = \Bx_\perp\), the dot product term with the plane is zero.  To verify this observe

\begin{equation}\label{eqn:rotor:760}
\begin{aligned}
\Bx_\perp \cdot \BB
&= \Bx_\perp \cdot (\Bb_1 \wedge \Bb_2) \\
&= (\Bx_\perp \cdot \Bb_1) \Bb_2 - (\Bx_\perp \cdot \Bb_2) \Bb_1 \\
\end{aligned}
\end{equation}

Each of these dot products are zero since \(\Bx\) has no components that lie
in the plane (those components if they existed could be expressed as linear
combinations of \(\Bb_i\)).

Thus only the component perpendicular to the plane is composed strictly of the
wedge product term

\begin{equation}
\Bx_\perp = \Bx \wedge \BB \inv{\BB}
\end{equation}

And again for a general vector the component that lies out
of the plane as, the rejection of the plane from the vector is

\begin{equation}\label{eqn:rotor:80}
\RejName_{\BB}(\Bx)
= \Bx \wedge \BB \inv{\BB}
= -\inv{\abs{\BB}^2} \Bx \wedge \BB {\BB}
= -\Bx \wedge \hat{\BB} \hat{\BB}
\end{equation}

\section{Rotation around normal to arbitrarily oriented plane through origin}

Having established the preliminaries, we can now express a rotation around
the normal to a plane (with the plane and that normal through the origin).

\imageFigure{../figures/gabook/rotor}{Rotation of Vector}{fig:rotor}{0.4}

Such a rotation is illustrated in \cref{fig:rotor}
preserves all components of the vector that are perpendicular
to the plane, and operates only on the components parallel to the plane.

Expressed in terms of exponentials and the projective and rejective decompositions above, this is

\begin{equation}\label{eqn:rotor:780}
\begin{aligned}
R_\theta(\Bx)
&= \Bx \wedge \BB \inv{\BB} + \left(\Bx \cdot \BB \inv{\BB}\right)e^{\hat{\BB}\theta} \\
&= \Bx \wedge \BB \inv{\BB} + e^{-\hat{\BB}\theta}\left(\Bx \cdot \BB \inv{\BB}\right) \\
\end{aligned}
\end{equation}

Where we have made explicit note that a plane rotation does not commute with a vector in a plane (its reverse is required).

To demonstrate this write \(i = \Be_2 \Be_1\), a unit bivector in some plane with unit vectors \(\Be_i\) also in the plane.  If a vector
lies in that plane we can write the rotation

\begin{equation}\label{eqn:rotor:800}
\begin{aligned}
\Bx e^{i\theta}
&= \left(a_1\Be_1 + a_2\Be_2\right)\left(\cos\theta + i\sin\theta\right) \\
&= \cos\theta\left(a_1\Be_1 + a_2\Be_2\right) + \left(a_1\Be_1 + a_2\Be_2\right)\left(\Be_2 \Be_1\sin\theta\right) \\
&= \cos\theta\left(a_1\Be_1 + a_2\Be_2\right) + \sin\theta \left(-a_1\Be_2 + a_2\Be_1\right) \\
&= \cos\theta\left(a_1\Be_1 + a_2\Be_2\right) -\Be_2 \Be_1\sin\theta \left(a_1\Be_1 + a_2\Be_2\right) \\
&= e^{-i\theta}\Bx \\
\end{aligned}
\end{equation}

Similarly for a vector that lies outside of the plane we can write

\begin{equation}\label{eqn:rotor:820}
\begin{aligned}
\Bx e^{i\theta}
&= (\sum_{j \ne 1,2} a_j \Be_j)(\cos\theta + \Be_2 \Be_1\sin\theta) \\
&= (\cos\theta + \Be_2 \Be_1\sin\theta) (\sum_{j \ne 1,2} a_j \Be_j) \\
&= e^{i\theta}\Bx
\end{aligned}
\end{equation}

The multivector for a rotation in a plane perpendicular to a vector commutes with that vector.  The properties of the
exponential allow us to factor a rotation

\begin{equation}\label{eqn:rotor:100}
R(\theta) = R(\alpha\theta) R((1-\alpha)\theta)
\end{equation}

where \(\alpha <= 1\), and in particular we can set \(\alpha = 1/2\), and write

\begin{equation}\label{eqn:rotor:840}
\begin{aligned}
R_\theta(\Bx)
&= \Bx \wedge \BB \inv{\BB} + \left(\Bx \cdot \BB \inv{\BB}\right)e^{\hat{\BB}\theta} \\
&= \left(\Bx \wedge \BB \inv{\BB}\right) e^{-\hat{\BB}\theta/2} e^{\hat{\BB}\theta/2}
 + \left(\Bx \cdot \BB \inv{\BB} \right) e^{\hat{\BB}\theta/2} e^{\hat{\BB}\theta/2} \\
&= e^{-\hat{\BB}\theta/2} \left(\Bx \wedge \BB \inv{\BB}\right) e^{\hat{\BB}\theta/2}
+ e^{-\hat{\BB}\theta/2} \left(\Bx \cdot \BB \inv{\BB}\right)e^{\hat{\BB}\theta/2} \\
&= e^{-\hat{\BB}\theta/2} \left(\Bx \wedge \BB + \Bx \cdot \BB\right) \inv{\BB} e^{\hat{\BB}\theta/2} \\
&= e^{-\hat{\BB}\theta/2} \left(\Bx \BB \inv{\BB} \right) e^{\hat{\BB}\theta/2}
\end{aligned}
\end{equation}

This takes us full circle from dot and wedge products back to \(\Bx\), and allows us to express the rotated vector as:

\begin{equation}\label{eqn:rotor:rotor}
R_\theta(\Bx)
= e^{-\hat{\BB}\theta/2} \Bx e^{\hat{\BB}\theta/2}
\end{equation}

Only when the vector lies in the plane (\(\Bx = \Bx_\parallel\), or \(\Bx \wedge \BB = 0\)) can be written using the familiar left or right ``full angle'' rotation exponential that we are used to from complex arithmetic:

\begin{equation}\label{eqn:rotor:120}
R_\theta(\Bx) = e^{-\hat{\BB}\theta} \Bx = \Bx e^{\hat{\BB}\theta}
\end{equation}

\section{Rotor equation in terms of normal to plane}

The rotor equation above is valid for any number of dimensions.  For \R{3} we can alternatively parametrize the plane in terms of
a unit normal \(\Bn\):

\begin{equation}\label{eqn:rotor:140}
\BB = k i\Bn
\end{equation}

Here \(i\) is the \R{3} pseudoscalar \(\Be_1 \Be_2 \Be_3\).

Thus we can write

\begin{equation}\label{eqn:rotor:160}
\hat{\BB} = i\Bn
\end{equation}

and expressing \eqnref{eqn:rotor:rotor} in terms of the unit normal becomes trivial

\begin{equation}
R_\theta(\Bx)
= e^{- i {\Bn}\theta/2} \Bx e^{i{\Bn}\theta/2}
\end{equation}

Expressing this in terms of components and the unit normal is a bit harder

\begin{equation}\label{eqn:rotor:860}
\begin{aligned}
R_\theta(\Bx)
&= \Bx \wedge \BB \inv{\BB} + \left(\Bx \cdot \BB \inv{\BB}\right)e^{\hat{\BB}\theta} \\
&= \Bx \wedge (i\Bn) \inv{i\Bn} + \left(\Bx \cdot (i\Bn) \inv{i\Bn}\right)e^{{i\Bn}\theta} \\
\end{aligned}
\end{equation}

Now,

\begin{equation}\label{eqn:rotor:880}
\begin{aligned}
\Bx \wedge (i\Bn)
&= \inv{2}(\Bx i \Bn + i \Bn \Bx) \\
&= \frac{i}{2}(\Bx \Bn + \Bn \Bx) \\
&= (\Bx \cdot \Bn) i
\end{aligned}
\end{equation}

And

\begin{equation}\label{eqn:rotor:900}
\begin{aligned}
\inv{i\Bn}
&= \inv{i\Bn} \inv{\Bn i} \Bn i \\
&= - i \Bn \\
\end{aligned}
\end{equation}

So the rejective term becomes
\begin{equation}\label{eqn:rotor:920}
\begin{aligned}
\Bx \wedge \BB \inv{\BB}
&= \Bx \wedge (i\Bn) \inv{i\Bn} \\
&= \Bx \wedge (i\Bn) \inv{i\Bn} \\
&= (\Bx \cdot \Bn) i (-i) \Bn \\
&= (\Bx \cdot \Bn) \Bn \\
&= \Proj_{\Bn}(\Bx) \\
\end{aligned}
\end{equation}

Now, for the dot product with the plane term, we have

\begin{equation}\label{eqn:rotor:940}
\begin{aligned}
\Bx \cdot \BB
&= \Bx \cdot (i \Bn) \\
&= \inv{2}(\Bx i \Bn - i \Bn \Bx) \\
&= (\Bx \wedge \Bn)i \\
\end{aligned}
\end{equation}

Putting it all together we have

\begin{equation}\label{eqn:rotor:rotexp}
R_\theta(\Bx)
= (\Bx \cdot \Bn) \Bn + (\Bx \wedge \Bn)\Bn e^{{i\Bn}\theta}
\end{equation}

In terms of explicit sine and cosine terms this is (observe that \((i\Bn)^2 = -1\)),

\begin{equation}\label{eqn:rotor:960}
\begin{aligned}
R_\theta(\Bx)
&= \left(\Bx \cdot \Bn\right) \Bn + \left(\Bx \wedge \Bn\right)\Bn \left(\cos\theta + i\Bn \sin\theta\right) \\
\end{aligned}
\end{equation}

\begin{equation}\label{eqn:rotor:rotnorm}
R_\theta(\Bx) =
\left(\Bx \cdot \Bn\right) \Bn + \left(\Bx \wedge \Bn\right)\Bn \cos\theta + (\Bx \wedge \Bn) i \sin\theta
\end{equation}

\imageFigure{../figures/gabook/normalRot}{Direction vectors associated with rotation}{fig:normalRot}{0.4}

This triplet of mutually orthogonal direction vectors,
\(\Bn\), \((\Bx \wedge \Bn)\Bn\), and \((\Bx \wedge \Bn) i\)
are illustrated in \cref{fig:normalRot}.  The component of the vector in the direction of the normal
\(\Proj_\Bn(\Bx) = \Bx \cdot \Bn \Bn\) is unaltered by the rotation.
The rotation is applied to the remaining component of \(\Bx\), \(\RejName_{\Bn}(\Bx) = (\Bx \wedge \Bn)\Bn\), and we rotate
in the direction \((\Bx \wedge \Bn) i\)

\subsection{Vector rotation in terms of dot and cross products only}

Expression of this rotation formula \eqnref{eqn:rotor:rotnorm} in terms of ``vector'' relations is also possible, by removing the wedge
products and the pseudoscalar references.

First the rejective term

\begin{equation}\label{eqn:rotor:980}
\begin{aligned}
(\Bx \wedge \Bn) \Bn
&= ((\Bx \cross \Bn) i) \Bn \\
&= ((\Bx \cross \Bn) i) \cdot \Bn \\
&= \inv{2} ( ((\Bx \cross \Bn) i) \Bn - \Bn ((\Bx \cross \Bn) i)) \\
&= \frac{i}{2} ( (\Bx \cross \Bn) \Bn - \Bn (\Bx \cross \Bn) ) \\
&= i ( (\Bx \cross \Bn) \wedge \Bn ) \\
&= i^2 ( (\Bx \cross \Bn) \cross \Bn ) \\
&= \Bn \cross (\Bx \cross \Bn) \\
\end{aligned}
\end{equation}

The next term expressed in terms of the cross product is

\begin{equation}\label{eqn:rotor:1000}
\begin{aligned}
(\Bx \wedge \Bn) i
&=
(\Bx \cross \Bn) i^2 \\
&= \Bn \cross \Bx \\
\end{aligned}
\end{equation}

And putting it all together we have

\begin{equation}\label{eqn:rotor:rotcross}
R_\theta(\Bx) =
\left(\Bx \cdot \Bn\right) \Bn
 + \left(\Bn \cross \Bx\right) \cross \Bn \cos\theta
 + \Bn \cross \Bx \sin\theta
\end{equation}

Compare \eqnref{eqn:rotor:rotcross} to \eqnref{eqn:rotor:rotnorm} and \eqnref{eqn:rotor:rotexp}, and then back to \eqnref{eqn:rotor:rotor}.

\section{Giving a meaning to the sign of the bivector}

For a rotation between two vectors in the plane containing those vectors, we can write the rotation
in terms of the exponential as either a left or right rotation operator:

\begin{equation}\label{eqn:rotor:180}
\Bb = \Ba e^{\Bi\theta} = e^{-\Bi\theta}\Ba
\end{equation}
\begin{equation}\label{eqn:rotor:200}
\Bb = e^{\Bj\theta}\Ba = \Ba e^{-\Bj\theta/2}
\end{equation}

Here both \(\Bi\) and \(\Bj=-\Bi\) are unit bivectors with the property \(\Bi^2 = \Bj^2 = -1\).
Thus in order to write a rotation in exponential form a meaning must be assigned to the sign of the unit bivector that describes the
plane and the orientation of the rotation.

Consider for example the case of a rotation by \(\pi/2\).  For this is the exponential is:

\begin{equation}\label{eqn:rotor:220}
e^{\Bi\pi/2} = \cos(\pi/2) + \Bi \sin(\pi/2) = \Bi
\end{equation}

Thus for perpendicular unit vectors \(\Bu\) and \(\Bv\), if we wish \(\Bi\) to act as a \(\pi/2\) rotation left acting operator on \(\Bu\)
towards \(\Bv\) its value must be:

\begin{equation}\label{eqn:rotor:240}
\Bi = \Bu \wedge \Bv
\end{equation}
\begin{equation}\label{eqn:rotor:260}
\Bu\Bi = \Bu \Bu \wedge \Bv = \Bu\Bu\Bv = \Bv
\end{equation}

For that same rotation if the bivector is employed as a right acting operator, the reverse is required:

\begin{equation}\label{eqn:rotor:280}
\Bj = \Bv \wedge \Bu
\end{equation}
\begin{equation}\label{eqn:rotor:300}
\Bj\Bu = \Bv \wedge \Bu = \Bv\Bu\Bu = \Bv
\end{equation}

\imageFigure{../figures/gabook/imaginaryorientation}{Orientation of unit imaginary}{fig:imaginaryorientation}{0.4}

In general, for any two vectors, one can find an angle \(\theta\) in the range \(0 \le \theta \le \pi\) between those vectors.
If one lets that angle define the orientation of the rotation between the vectors, and implicitly
define a sort of ``imaginary axis'' for that plane, that imaginary axis will have direction

\begin{equation}\label{eqn:rotor:320}
\inv{\Ba} \Ba \wedge \Bb = \Bb \wedge \Ba \inv {\Ba}.
\end{equation}

This is illustrated in \cref{fig:imaginaryorientation}.

Thus the bivector

\begin{equation}\label{eqn:rotor:340}
\Bi = \frac{\Ba \wedge \Bb}{\abs{\Ba \wedge \Bb}}
\end{equation}

When acting as an operator to the left (\(\Ba \Bi\)) with a vector in the plane can be interpreted as acting as a rotation by \(\pi/2\) towards \(\Bb\).

Similarly the bivector

\begin{equation}\label{eqn:rotor:360}
\Bj = \Bi^\dagger = -\Bi = \frac{\Bb \wedge \Ba}{\abs{\Bb \wedge \Ba}}
\end{equation}

also applied to a vector in the plane produces the same rotation when
acting as an operator to the right.  Thus, in general we can write
a rotation by theta in the plane containing non-colinear vectors \(\Ba\) and \(\Bb\) in the direction of minimal angle
from \(\Ba\) towards \(\Bb\) in one of the three forms:

\begin{equation}\label{eqn:rotor:380}
R_{\theta : \Ba \rightarrow \Bb}(\Ba)
= \Ba e^{ \frac{\Ba \wedge \Bb}{\abs{\Ba \wedge \Bb}} \theta }
= e^{ \frac{\Bb \wedge \Ba}{\abs{\Bb \wedge \Ba}} \theta } \Ba
\end{equation}

Or,
\begin{equation}\label{eqn:rotor:400}
R_{\theta : \Ba \rightarrow \Bb}(\Bx)
= e^{ \frac{\Bb \wedge \Ba}{\abs{\Bb \wedge \Ba}} \theta/2 } \Bx e^{ \frac{\Ba \wedge \Bb}{\abs{\Ba \wedge \Bb}} \theta/2 }
\end{equation}

This last (writing \(\Bx\) instead of \(\Ba\) since it also applies to vectors that lie outside of the \(\Ba \wedge \Bb\) plane),
is our rotor formula \eqnref{eqn:rotor:rotor}, reexpressed in a way that removes the sign ambiguity of the bivector \(\Bi\) in that equation.

\section{Rotation between two unit vectors}

\imageFigure{../figures/gabook/parallelogramvec}{Sum of unit vectors bisects angle between}{fig:parallelogramvec}{0.4}

As illustrated in \cref{fig:parallelogramvec}, when the angle between two vectors is less than \(\pi\)
the fact that the sum of two arbitrarily oriented unit vectors bisects those vectors provides a convenient
way to compute the half angle rotation exponential.

Thus we can write

\begin{equation*}
\frac{\Ba + \Bb}{\abs{\Ba + \Bb}} = \Ba e^{\Bi\theta/2} = e^{\Bj\theta/2} \Ba
\end{equation*}

Where \(\Bi = \Bj^\dagger\) are unit bivectors of appropriate sign.  Multiplication through by \(\Ba\) gives

\begin{equation*}
e^{\Bi\theta/2} =
\frac{1 + \Ba\Bb}{\abs{\Ba + \Bb}}
\end{equation*}

Or,
\begin{equation*}
e^{\Bj\theta/2} =
\frac{1 + \Bb\Ba}{\abs{\Ba + \Bb}}
\end{equation*}

Thus we can write the total rotation from \(\Ba\) to \(\Bb\) as

\begin{equation*}
\Bb
= e^{-\Bi\theta/2} \Ba e^{\Bi\theta/2}
= e^{\Bj\theta/2} \Ba e^{-\Bj\theta/2}
= \left(\frac{1 + \Bb\Ba}{\abs{\Ba + \Bb}}\right) \Ba \left(\frac{1 + \Ba\Bb}{\abs{\Ba + \Bb}}\right)
\end{equation*}

For the case where the rotation is through an angle \(\theta\) where \(\pi < \theta < 2\pi\), again employing a left acting
exponential operator we have

\begin{equation}\label{eqn:rotor:1020}
\begin{aligned}
\frac{\Ba + \Bb}{\abs{\Ba + \Bb}}
&= \Bb e^{\Bi(2\pi - \theta)/2} \\
&= \Bb e^{\Bi \pi} e^{- \Bi\theta/2} \\
&= -\Bb e^{- \Bi\theta/2} \\
\end{aligned}
\end{equation}

Or,
\begin{equation}\label{eqn:rotor:420}
e^{- \Bi\theta/2} = -\frac{\Bb\Ba + 1}{\abs{\Ba + \Bb}}
\end{equation}

Thus

\begin{equation}\label{eqn:rotor:rotunit}
\Bb = e^{- \Bi\theta/2} \Ba e^{ \Bi\theta/2} =
\left(-\frac{1 + \Bb\Ba}{\abs{\Ba + \Bb}}\right) \Ba \left(-\frac{1 + \Ba\Bb}{\abs{\Ba + \Bb}}\right)
\end{equation}

Note that the two negatives cancel, giving the same result as in the \(\theta < \pi\) case.  Thus \eqnref{eqn:rotor:rotunit} is valid for all vectors \(\Ba \ne -\Bb\) (this can be verified by direct multiplication.)

These
half angle exponentials are called rotors, writing the rotor as

\begin{equation}\label{eqn:rotor:440}
R = \frac{1 + \Ba\Bb}{\abs{\Ba + \Bb}}
\end{equation}

and the rotation in terms of rotors is:

\begin{equation}\label{eqn:rotor:460}
\Bb = R^\dagger \Ba R
\end{equation}

The angle associated with this rotor \(R\) is the minimal angle between the two vectors (\(0 < \theta < \pi\)), and is directed from \(\Ba\) to \(\Bb\).  Inverting the rotor will not change the net effect of the rotation, but has the geometric meaning that the rotation from \(\Ba\) to \(\Bb\)
rotates in the opposite direction through the larger angle (\(\pi < \theta < 2\pi\)) between the vectors.

\section{Eigenvalues, vectors and coordinate vector and matrix of the rotation linear transformation}

Given the plane containing two orthogonal vectors \(\Bu\) and \(\Bv\), we can form a unit bivector for the plane

\begin{equation}\label{eqn:rotor:480}
\BB = \Bu\Bv
\end{equation}

A normal to this plane is \(\Bn = \Bv\Bu I\).

The rotation operator for a rotation around \(\Bn\) in that plane (directed from \(\Bu\) towards \(\Bv\)) is

\begin{equation}\label{eqn:rotor:500}
R_\theta(\Bx) = e^{\Bv\Bu \theta/2} \Bx e^{\Bu\Bv \theta/2}
\end{equation}

To form the matrix of this linear transformation assume an orthonormal basis \(\sigma = \{ \Be_i \}\).

In terms of these basis vectors we can write

\begin{equation}\label{eqn:rotor:520}
R_\theta(\Be_j) =
e^{-\Bv\Bu \theta/2} \Be_j e^{\Bu\Bv \theta/2}
=
\sum_i \left(e^{-\Bv\Bu \theta/2} \Be_j e^{\Bu\Bv \theta/2}\right) \cdot \Be_i \Be_i
\end{equation}

Thus the coordinate vector for this basis is

\begin{equation}\label{eqn:rotor:540}
{
\begin{bmatrix}
R_\theta(\Be_j)
\end{bmatrix}
}_\sigma
=
\begin{bmatrix}
\left(e^{-\Bv\Bu \theta/2} \Be_j e^{\Bu\Bv \theta/2}\right) \cdot \Be_1 \\
\vdots \\
\left(e^{-\Bv\Bu \theta/2} \Be_j e^{\Bu\Bv \theta/2}\right) \cdot \Be_n \\
\end{bmatrix}
\end{equation}

We can use this to form the matrix for the linear operator that takes coordinate vectors from
the basis \(\sigma\) to \(\sigma\):

\begin{equation}\label{eqn:rotor:560}
{
\begin{bmatrix}
R_\theta(\Bx)
\end{bmatrix}
}_\sigma
=
{
\begin{bmatrix}
R_\theta
\end{bmatrix}
}_\sigma^\sigma
{
\begin{bmatrix}
\Bx
\end{bmatrix}
}_\sigma
\end{equation}

Where
\begin{equation}\label{eqn:rotor:rotcoords}
{
\begin{bmatrix}
R_\theta
\end{bmatrix}
}_\sigma^\sigma
=
\begin{bmatrix}
{
\begin{bmatrix}
R_\theta(\Be_1)
\end{bmatrix}
}_\sigma
\hdots
{
\begin{bmatrix}
R_\theta(\Be_n)
\end{bmatrix}
}_\sigma
\end{bmatrix}
=
{
\begin{bmatrix}
\left(e^{-\Bv\Bu \theta/2} \Be_j e^{\Bu\Bv \theta/2}\right) \cdot \Be_i \\
\end{bmatrix}
}_{ij}
\end{equation}

If one uses the plane and its normal to form an alternate orthonormal basis
\(\alpha = \{\Bu, \Bv, \Bn\}\).

The transformation matrix for coordinate vectors in this basis is

\begin{equation}\label{eqn:rotor:580}
{
\begin{bmatrix}
R_\theta
\end{bmatrix}
}_\alpha^\alpha
=
\begin{bmatrix}
\left(\Bu e^{\Bu\Bv \theta}\right) \cdot \Bu & \left(\Bv e^{\Bu\Bv \theta}\right) \cdot \Bu & 0 \\
\left(\Bu e^{\Bu\Bv \theta}\right) \cdot \Bv & \left(\Bv e^{\Bu\Bv \theta}\right) \cdot \Bv & 0 \\
0 & 0 & \Bn\cdot\Bn \\
\end{bmatrix}
=
\begin{bmatrix}
\cos\theta & -\sin\theta & 0 \\
\sin\theta & \cos\theta & 0 \\
0 & 0 & 1 \\
\end{bmatrix}
\end{equation}

This matrix has eigenvalues \(e^{i\theta}, e^{-i\theta}, 1\), with (coordinate) eigenvectors

\begin{equation}\label{eqn:rotor:600}
\inv{\sqrt{2}}
\begin{bmatrix}
1 \\
-i \\
0 \\
\end{bmatrix},
\inv{\sqrt{2}}
\begin{bmatrix}
1 \\
i \\
0 \\
\end{bmatrix},
\begin{bmatrix}
0 \\
0 \\
1 \\
\end{bmatrix}
\end{equation}

Its interesting to observe that without introducing coordinate vectors an eigensolution is possible directly from
the linear transformation itself.

The rotation linear operator has right and left eigenvalues \(e^{\Bu\Bv \theta}\), \(e^{\Bv\Bu \theta}\) (respectively), where the eigenvectors for these are any vectors in the plane.  There is also a scalar eigenvalue \(1\) (both left and right eigenvalue), for the eigenvector \(\Bn\):

\begin{equation}\label{eqn:rotor:1040}
\begin{aligned}
R_\theta(\Bu) &= e^{\Bv \Bu \theta} \Bx = \Bx e^{\Bu \Bv \theta} \\
R_\theta(\Bu) &= e^{\Bv \Bu \theta} \Bx = \Bx e^{\Bu \Bv \theta} \\
R_\theta(\Bn) &= \Bn (1) \\
\end{aligned}
\end{equation}

Observe that the eigenvalues here are not all scalars, which is likely related
to the fact that the coordinate matrix was not diagonalizable with real vectors.

the matrix of the linear transformation.
Given this, one can write:

\begin{equation}\label{eqn:rotor:1060}
\begin{aligned}
\begin{bmatrix}
R_\theta(\Bu) & R_\theta(\Bv) & R_\theta(\Bn) \\
\end{bmatrix}
&=
\begin{bmatrix}
\Bu & \Bv & \Bn \\
\end{bmatrix}
\begin{bmatrix}
e^{\Bu \Bv \theta} & 0 & 0 \\
0 & e^{\Bu \Bv \theta} & 0 \\
0 & 0 & 1 \\
\end{bmatrix} \\
&=
\begin{bmatrix}
e^{\Bv \Bu \theta} & 0 & 0 \\
0 & e^{\Bv \Bu \theta} & 0 \\
0 & 0 & 1 \\
\end{bmatrix}
\begin{bmatrix}
\Bu & \Bv & \Bn \\
\end{bmatrix}
\end{aligned}
\end{equation}

But neither of these can be used to diagonalize the matrix of the transformation.  To do that
we require dot products that span the matrix product to form the coordinate vector columns.

Observe that interestingly
enough the left and right eigenvalues of the operator in the plane are of complex exponential form (\(e^{\pm \Bn I \theta}\)) just as the eigenvalues for
coordinate vectors restricted to the plane are complex exponentials (\(e^{\pm i\theta}\)).
%This suggests that a basis for a quaternion
%like space (0-2 multivectors) will be required to diagonalize a rotation operator.

\section{matrix for rotation linear transformation}

Let us expand the terms in \eqnref{eqn:rotor:rotcoords} to calculate explicitly the rotation matrix for an arbitrary
rotation.  Also, as before, write \(\Bn = \Bv\Bu I\), and parametrize the Rotor as follows:

\begin{equation}\label{eqn:rotor:620}
R = e^{\Bn I \theta/2} = \cos{\theta/2} + \Bn I \sin{\theta/2} = \alpha + I\Bbeta
\end{equation}

Thus the \(ij\) terms in the matrix are:

\begin{equation}\label{eqn:rotor:1080}
\begin{aligned}
\Be_i \cdot \left(e^{-\Bn I \theta/2} \Be_j e^{\Bn I \theta/2}\right)
&= \langle{ \Be_i (\alpha -I\Bbeta) \Be_j (\alpha +I\Bbeta) } \rangle \\
&= \langle{ \Be_i (\Be_j \alpha -I\Bbeta\Be_j) (\alpha +I\Bbeta) } \rangle \\
&= \langle{ \Be_i \left( \Be_j \alpha^2 -I\alpha(\Bbeta\Be_j - \Be_j\Bbeta) + \Bbeta\Be_j\Bbeta \right) } \rangle \\
&= \delta_{ij}\alpha^2 + \langle{ \Be_i \left( -2I\alpha(\Bbeta \wedge \Be_j) + \Bbeta\Be_j\Bbeta \right) } \rangle \\
&= \delta_{ij}\alpha^2 + 2\alpha \Be_i \cdot (\Bbeta \cross \Be_j) + \langle{ \Be_i \Bbeta \Be_j \Bbeta } \rangle \\
\end{aligned}
\end{equation}

Lets expand the last term separately:
\begin{equation}\label{eqn:rotor:1100}
\begin{aligned}
\langle{ \Be_i \Bbeta \Be_j \Bbeta } \rangle
&= \langle{ ( \Be_i \cdot \Bbeta + \Be_i \wedge \Bbeta) ( \Be_j \cdot \Bbeta + \Be_j \wedge \Bbeta) } \rangle  \\
&= (\Be_i \cdot \Bbeta)(\Be_j \cdot \Bbeta) + \langle{ (\Be_i \wedge \Bbeta) ( \Be_j \wedge \Bbeta) } \rangle  \\
\end{aligned}
\end{equation}

And once more considering first the \(i=j\) case (writing \(s \ne i \ne t\)).

\begin{equation}\label{eqn:rotor:1120}
\begin{aligned}
\langle{ (\Be_i \wedge \Bbeta)^2 }\rangle
&= \lr{ \sum_{k \ne i}{ \Be_{ik} \beta_k} }^2 \\
&= ( \Be_{is} \beta_s + \Be_{it} \beta_t ) ( \Be_{is} \beta_s + \Be_{it} \beta_t ) \\
&= -\beta_s^2 -\beta_t^2 -  \Be_{st} \beta_s \beta_t + \Be_{ts} \beta_t \beta_s  \\
&= -\beta_s^2 -\beta_t^2 \\
&= -\Bbeta^2 + \beta_i^2 \\
\end{aligned}
\end{equation}

For the \(i \ne j\) term, writing \(i \ne j \ne k\)
\begin{equation}\label{eqn:rotor:1140}
\begin{aligned}
\langle{(\Be_i \wedge \Bbeta) (\Be_j \wedge \Bbeta)}\rangle
&= \langle{\sum_{s \ne i} \Be_{is} \beta_s\sum_{t \ne i} \Be_{it} \beta_t}\rangle \\
&= \langle{( \Be_{ij} \beta_j + \Be_{ik} \beta_k) ( \Be_{ji} \beta_i + \Be_{jk} \beta_k)}\rangle \\
&= \beta_i\beta_j + \langle{ \Be_{ji} \beta_k^2 +\Be_{ik} \beta_j \beta_k +\Be_{kj} \beta_k \beta_i }\rangle \\
&= \beta_i\beta_j \\
\end{aligned}
\end{equation}

Thus
\begin{equation}\label{eqn:rotor:640}
\langle{ (\Be_i \wedge \Bbeta) ( \Be_j \wedge \Bbeta) } \rangle
= \delta_{ij}(-\Bbeta^2 + \beta_i^2) + (1-\delta_{ij})\beta_i\beta_j
= \beta_i\beta_j -\delta_{ij}\Bbeta^2
\end{equation}

And putting it all back together
\begin{equation}\label{eqn:rotor:rotmgreek}
\Be_i \cdot \left(e^{-\Bn I \theta/2} \Be_j e^{\Bn I \theta/2}\right)
= \delta_{ij}(\alpha^2 -\Bbeta^2) + 2\alpha \Be_i \cdot (\Bbeta \cross \Be_j) + 2\beta_i\beta_j
\end{equation}


The \(\alpha\) and \(\beta\) terms can be expanded in terms of \(\theta\).
we see that The \(\delta_{ij}\) coefficient is

\begin{equation}\label{eqn:rotor:660}
\alpha^2 -\Bbeta^2 = 2{\cos}^2{\theta} -1 = \cos\theta.
\end{equation}

The triple product \(\Be_i \cdot (\Bbeta \cross \Be_j)\) is zero along the diagonal where \(i=j\) since an \(\Be_j=\Be_i\) cross has no \(\Be_i\) component, so
for \(k \ne i \ne j\), the triple product term is

\begin{equation}\label{eqn:rotor:1160}
\begin{aligned}
2\alpha \Be_i \cdot (\Bbeta \cross \Be_j)
&= 2\alpha \beta_k \Be_i \cdot (\Be_k \cross \Be_j) \\
&= 2\alpha \beta_k \Sgn{\pi_{ikj}} \\
&= 2 n_k \cos({\theta/2})\sin({\theta/2}) \Sgn{\pi_{ikj}} \\
&= n_k \sin{\theta} \Sgn{\pi_{ikj}} \\
\end{aligned}
\end{equation}

The last term is:
\begin{equation}\label{eqn:rotor:680}
2\beta_i\beta_j
= 2 n_i n_j {\sin}^2({\theta/2})
= n_i n_j (1-\cos\theta)
\end{equation}

Thus we can alternatively write \eqnref{eqn:rotor:rotmgreek}

\begin{equation}\label{eqn:rotor:rotmn}
\Be_i \cdot \left(e^{-\Bn I \theta/2} \Be_j e^{\Bn I \theta/2}\right)
= \delta_{ij}\cos\theta
+ n_k \sin{\theta} \epsilon_{ikj} + n_i n_j (1-\cos\theta)
\end{equation}

This is enough to easily and explicitly write out the complete rotation matrix for a rotation about unit vector \(\Bn = (n_1, n_2, n_3)\):
(with basis \(\sigma = \{\Be_i\}\)):

\begin{equation}\label{eqn:rotor:700}
[
R_\theta
]_\sigma^\sigma
=
\begin{bmatrix}
\cos\theta(1 -n_1^2) + n_1^2 & n_1 n_2 (1-\cos\theta) - n_3 \sin\theta & n_1 n_3 (1-\cos\theta) + n_2 \sin\theta \\
n_1 n_2 (1-\cos\theta) + n_3 \sin\theta & \cos\theta(1 -n_2^2) + n_2^2 & n_2 n_3 (1-\cos\theta) - n_1 \sin\theta \\
n_1 n_3 (1-\cos\theta) - n_2 \sin\theta & n_2 n_3 (1-\cos\theta) + n_1 \sin\theta & \cos\theta(1 -n_3^2) + n_3^2 \\
\end{bmatrix}
\end{equation}

Note also that the \(n_i\) terms are the direction cosines of the unit normal for the rotation, so all the terms above
are really strictly sums of sine and cosine products, so we have the rotation matrix completely described in terms of four
angles.  Also observe how much additional complexity we have to express a rotation in terms of the matrix.  This representation also
does not work for plane rotations, just vectors (whereas that is not the case for the rotor form).

It is actually somewhat simpler looking to leave things in terms of the \(\alpha\), and \(\beta\) parameters.  We can rewrite
\eqnref{eqn:rotor:rotmgreek} as:

\begin{equation}
\Be_i \cdot \left(e^{-\Bn I \theta/2} \Be_j e^{\Bn I \theta/2}\right)
= \delta_{ij}(2\alpha^2 -1)
+2\alpha \beta_k \epsilon_{ikj} + 2\beta_i\beta_j
\end{equation}

and the rotation matrix:

\begin{equation}\label{eqn:rotor:720}
[
R_\theta
]_\sigma^\sigma
=
2
\begin{bmatrix}
\alpha^2 -\frac{1}{2} + \beta_1^2 & \beta_1 \beta_2  - \beta_3 \alpha & \beta_1 \beta_3  + \beta_2 \alpha \\
\beta_1 \beta_2  + \beta_3 \alpha & \alpha^2 -\frac{1}{2} + \beta_2^2 & \beta_2 \beta_3  - \beta_1 \alpha \\
\beta_1 \beta_3  - \beta_2 \alpha & \beta_2 \beta_3  + \beta_1 \alpha & \alpha^2 -\frac{1}{2} + \beta_3^2 \\
\end{bmatrix}
\end{equation}

Not really that much simpler, but a bit.  The trade off is that the similarity to the standard \(2x2\) rotation matrix is not obvious.

