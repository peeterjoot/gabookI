%
% Copyright � 2012 Peeter Joot.  All Rights Reserved.
% Licenced as described in the file LICENSE under the root directory of this GIT repository.
%

%
%
%\input{../peeter_prologue.tex}

%\chapter{Generator of rotations in arbitrary dimensions}
\index{rotation!generator}
\label{chap:rotationGenerator}

%\blogpage{http://sites.google.com/site/peeterjoot/math2009/rotationGenerator.pdf}
%\date{Aug 31, 2009}
%\revisionInfo{\(RCSfile: rotationGenerator.tex,v \) Last \(Revision: 1.11 \) \(Date: 2009/10/22 02:07:20 \)}

%\beginArtWithToc
\beginArtNoToc

\section{Motivation}

Eli in his recent \href{http://behindtheguesses.blogspot.com/2009/08/noncommuting-rotation-and-angular.html}{blog post on angular momentum operators} used an exponential operator to generate rotations

\begin{equation}\label{eqn:rotationGen:goo1}
\begin{aligned}
R_{\Delta\theta} = e^{\Delta\theta \ncap \cdot (\Bx \cross \spacegrad)}
\end{aligned}
\end{equation}

This is something I hhad not seen before, but is comparable to the vector shift operator expressed in terms of directional derivatives \(\Bx \cdot \spacegrad\)

\begin{equation}\label{eqn:rotationGen:goo2}
\begin{aligned}
f(\Bx + \Ba) = e^{\Ba \cdot \spacegrad} f(\Bx)
\end{aligned}
\end{equation}

The translation operator of \eqnref{eqn:rotationGen:goo2} translates easily to higher dimensions.  Of particular interest is the Minkowski metric 4D spacetime case, where we can use the four gradient \(\grad = \gamma^\mu \partial_\mu\), and a vector spacetime translation of \(x = x^\mu \gamma_\mu \rightarrow (x^\mu + a^\mu) \gamma_\mu\) to translate ``trivially'' translate this

\begin{equation}\label{eqn:rotationGen:goo3}
\begin{aligned}
f(x + a) = e^{a \cdot \grad} f(x)
\end{aligned}
\end{equation}

Since we do not have a cross product of two vectors in a 4D space, re-expressing \eqnref{eqn:rotationGen:goo1} in a form that is not tied to three dimensions is desirable.  A duality transformation with \(\ncap = i \Be_1 \Be_2 \Be_3\) accomplishes this, where \(i\) is a unit bivector for the plane perpendicular to \(\ncap\) (i.e. product of two perpendicular unit vectors in the plane).  That duality transformation, expressing the rotation direction using an oriented plane instead of the normal to the plane gives us

\begin{equation}\label{eqn:rotationGenerator:64}
\begin{aligned}
\ncap \cdot (\Bx \cross \spacegrad)
&=
\gpgradezero{ \ncap (\Bx \cross \spacegrad) } \\
&=
\gpgradezero{ (i \Be_1 \Be_2 \Be_3) (-\Be_1 \Be_2 \Be_3) (\Bx \wedge \spacegrad) } \\
&=
\gpgradezero{ i (\Bx \wedge \spacegrad) } \\
\end{aligned}
\end{equation}

This is just \(i \cdot (\Bx \wedge \spacegrad)\), so the generator of the rotation in 3D is

\begin{equation}\label{eqn:rotationGen:goo4}
\begin{aligned}
R_{\Delta\theta} = e^{\Delta\theta i \cdot (\Bx \wedge \spacegrad)}
\end{aligned}
\end{equation}

It is reasonable to guess then that we could substitute the spacetime gradient and allow \(i\) to be any 4D unit spacetime bivector, where a spacelike product pair will generate rotations and a spacetime bivector will generate boosts.  That is really just a notational shift, and we would write

\begin{equation}\label{eqn:rotationGen:goo5}
\begin{aligned}
R_{\Delta\theta} = e^{\Delta\theta i \cdot (x \wedge \grad)}
\end{aligned}
\end{equation}

This is very likely correct, but building up to this guess in a logical sequence from a known point will be the aim of this particular exploration.

\section{Setup and conventions}

Rather than expressing the rotation in terms of coordinates, here the rotation will be formulated in terms of dual sided multivector operators (using Geometric Algebra) on vectors.  Then employing the chain rule an examination of the differential change of a multivariable scalar valued function on the underlying rotation will be made.

Following conventions of \citep{doran2003gap} vectors will be undecorated rather than boldface since we are deriving results applicable to four vector (and higher) spaces, and not requiring an Euclidean metric.

\imageFigure{../figures/gabook/rotationGen}{Rotating vector in the plane with bivector i}{fig:rotationGen:rotationGen}{0.4}

The \cref{fig:rotationGen:rotationGen} has a pair of vectors related by rotation, where the vector \(x(\theta)\) is rotated to \(y(\theta) = x(\theta + \Delta\theta)\). We choose here to express this rotation using a quaternion-ic operator \(R = \alpha + a b\), where \(\alpha\) is a scalar and \(a\), and \(b\) are vectors.

\begin{equation}\label{eqn:rotationGen:foo1}
\begin{aligned}
y = \tilde{R} x R
\end{aligned}
\end{equation}

Required of \(R\) is an invertability property, but without loss of generality we can impose a strictly unitary property \(\tilde{R} R = 1\).  Here \(\tilde{R}\) denotes the multivector reverse of a Geometric product

\begin{equation}\label{eqn:rotationGen:foo2}
\begin{aligned}
(a b)^{\tilde{}} = \tilde{b} \tilde{a}
\end{aligned}
\end{equation}

Where for individual vectors the reverse is itself \(\tilde{a} = a\).  A singly parametrized rotation or boost can be conveniently expressed using the half angle exponential form

\begin{equation}\label{eqn:rotationGen:foo3}
\begin{aligned}
R = e^{i \theta/2}
\end{aligned}
\end{equation}

where \(i = \hat{u}\hat{v}\) is a unit bivector, a product of two perpendicular unit vectors (\(\hat{u} \hat{v} = - \hat{v} \hat{u}\)).  For rotations \(\hat{u}\), and \(\hat{v}\) are both spatial vectors, implying \(i^2 = -1\).  For boosts \(i\) is the product of a unit timelike vector and unit spatial vector, and with a Minkowski metric condition \(\hat{u}^2 \hat{v}^2 = -1\), we have a positive square \(i^2 = 1\) for our spacetime rotation plane \(i\).

A general Lorentz transformation, containing a composition of rotations and boosts can be formed by application of successive transformations

\begin{equation}\label{eqn:rotationGen:foo4}
\begin{aligned}
\LL(x) = (\tilde{U} (\tilde{T} \cdots (\tilde{S} x S) T )\cdots U) = \tilde{U} \tilde{T} \cdots \tilde{S} x S T \cdots U
\end{aligned}
\end{equation}

The composition still has the unitary property \((S T \cdots U)^{\tilde{}} S T \cdots U = 1\), so when the specifics of the parametrization are not required we will allow the rotation operator \(R = S T \cdots U\) to be a general composition of individual rotations and boosts.

We will have brief use of coordinates and employ a reciprocal basis pair \(\{\gamma^\mu\}\) and \(\{\gamma_\nu\}\) where \(\gamma^\mu \cdot \gamma_\nu = {\delta^{\mu}}_\nu\).  A vector, employing summation convention, is then denoted

\begin{equation}\label{eqn:rotationGen:foo5}
\begin{aligned}
x = x^\mu \gamma_\mu = x_\mu \gamma^\mu
\end{aligned}
\end{equation}

Where
\begin{equation}\label{eqn:rotationGen:foo6}
\begin{aligned}
x_\mu &= x \cdot \gamma_\mu \\
x_\mu &= x \cdot \gamma_\mu
\end{aligned}
\end{equation}

Shorthand for partials

\begin{equation}\label{eqn:rotationGen:foo7}
\begin{aligned}
\partial_\mu &\equiv \frac{\partial}{\partial x^\mu} \\
\partial^\mu &\equiv \frac{\partial}{\partial x_\mu}
\end{aligned}
\end{equation}

will allow the gradient to be expressed as

\begin{equation}\label{eqn:rotationGen:foo8}
\begin{aligned}
\grad \equiv \gamma^\mu \partial_\mu = \gamma_\mu \partial^\mu
\end{aligned}
\end{equation}

The perhaps unintuitive mix of upper and lower indices is required to make the indices in the direction derivative come out right when expressed as a dot product

\begin{equation}\label{eqn:rotationGen:foo9}
\begin{aligned}
\lim_{\tau \rightarrow 0} \frac{f(x + a\tau) - f(x)}{\tau} = a^\mu \partial_\mu f(x) = a \cdot \grad f(x)
\end{aligned}
\end{equation}

\section{Rotor examples}

While not attempting to discuss the exponential rotor formulation in any depth, at least illustrating by example for a spatial rotation and Lorentz boost seems called for.

Application of either of these is most easily performed with a split of the vector into components parallel and perpendicular to the ``plane'' of rotation \(i\).  For example suppose we decompose a vector \(x = p + n\) where \(n\) is perpendicular to the rotation plane \(i\) (i.e. \(n i = i n\)), and \(p\) is the components in the plane (\(p i = - i p\)).  A consequence is that \(n\) commutes with \(R\) and \(p\) induces a conjugate effect in the rotor

\begin{equation}\label{eqn:rotationGenerator:84}
\begin{aligned}
\tilde{R} x R
&=
e^{-i \theta/2} (p + n) e^{i \theta/2} \\
&=
p e^{i \theta/2} e^{i \theta/2}
+
n e^{-i \theta/2} e^{i \theta/2} \\
\end{aligned}
\end{equation}

This is then just
\begin{equation}\label{eqn:rotationGen:foo30}
\begin{aligned}
\tilde{R} x R
&=
p e^{i \theta} + n
\end{aligned}
\end{equation}

To expand any further the metric details are required.  The half angle rotors of \eqnref{eqn:rotationGen:foo3} can be expanded in series, where the metric properties of the bivector dictate the behavior.  In the spatial bivector case, where \(i^2 = -1\) we have

\begin{equation}\label{eqn:rotationGen:foo31}
\begin{aligned}
R = e^{i \theta/2} = \cos(\theta/2) + i\sin(\theta/2)
\end{aligned}
\end{equation}

whereas when \(i^2 = 1\), the series expansion yields a hyperbolic pair

\begin{equation}\label{eqn:rotationGen:foo32}
\begin{aligned}
R = e^{i \theta/2} = \cosh(\theta/2) + i\sinh(\theta/2)
\end{aligned}
\end{equation}

To make things more specific, and relate to the familiar, consider a rotation in the Euclidean \(x,y\) plane where we pick \(i = \Be_1 \Be_2\), and rotate \(\Bx = x \Be_1 + y \Be_2 + z \Be_3\).  Applying \eqnref{eqn:rotationGen:foo30}, and \eqnref{eqn:rotationGen:foo31} we have

\begin{equation}\label{eqn:rotationGenerator:104}
\begin{aligned}
\tilde{R} \Bx R
&=
(x \Be_1 + y \Be_2) (\cos\theta + \Be_1 \Be_2 \sin\theta) + z \Be_3 \\
\end{aligned}
\end{equation}

We have \({\Be_1}^2 = {\Be_2}^2 = 1\) and \(\Be_1 \Be_2 = -\Be_1 \Be_2\), so with some rearrangement

\begin{equation}\label{eqn:rotationGenerator:124}
\begin{aligned}
\tilde{R} \Bx R
&=
\Be_1 (x \cos\theta - y \sin\theta )
+\Be_2 (x \sin\theta + y \cos\theta )
+ \Be_3 z
\end{aligned}
\end{equation}

This is the familiar \(x,y\) plane rotation up to a possible sign preference.  Observe that we have the flexibility to adjust the sign of the rotation by altering either \(\theta\) or \(i\) (we could use \(i = \Be_2 \Be_1\) for example).  Because of this Hestenes \citep{hestenes1999nfc} chooses to make the angle bivector valued, so instead of \(i\theta\) writes

\begin{equation}\label{eqn:rotationGen:foo33}
\begin{aligned}
R = e^{B}
\end{aligned}
\end{equation}

where \(B\) is bivector valued, and thus contains the sign or direction of the rotation or boost as well as the orientation.

% cody:
%1 877 977 4601

For completeness lets also expand a rotor application for an x-axis boost in the spacetime plane \(i = \gamma_1 \gamma_0\).  Following \citep{doran2003gap}, we use the \((+,-,-,-)\) metric convention \(1 = {\gamma_0}^2 = -{\gamma_1}^2 = -{\gamma_2}^2 = -{\gamma_3}^2\).  Switching variable conventions to match the norm lets use \(\alpha\) for the rapidity angle, with x-axis boost rotor

\begin{equation}\label{eqn:rotationGen:foo34}
\begin{aligned}
R = e^{\gamma_1 \gamma_0 \alpha/2}
\end{aligned}
\end{equation}

for the rapidity angle \(\alpha\).  The rotor application then gives

\begin{equation}\label{eqn:rotationGenerator:144}
\begin{aligned}
\LL(x)
&= \tilde{R} (x^0 \gamma_0 + x^1 \gamma_1 + x^2 \gamma_2 + x^3 \gamma_3 ) R \\
&=
\tilde{R} (x^0 \gamma_0 + x^1 \gamma_1) R + x^2 \gamma_2 + x^3 \gamma_3 \\
&=
(x^0 \gamma_0 + x^1 \gamma_1) (\cosh(\theta) + \gamma_1 \gamma_0 \sinh(\theta/2)) + x^2 \gamma_0 + x^3 \gamma_3 \\
\end{aligned}
\end{equation}

A final bit of rearrangement yields the familiar

\begin{equation}\label{eqn:rotationGen:foo35}
\begin{aligned}
\LL(x)
&=
\gamma_0 (x^0 \cosh(\theta) - x^1 \sinh(\theta/2)) \\
&\qquad + \gamma_1 ( -x^0 \sinh(\theta/2) + x^1 \cosh(\theta))
+ x^2 \gamma_0 + x^3 \gamma_3
\end{aligned}
\end{equation}

Again observe the flexibility to adjust the sign as desired by either the bivector orientation or the sign of the scalar rapidity angle.

\section{The rotation operator}

Moving on to the guts.  From \eqnref{eqn:rotationGen:foo1} we can express \(x\) in terms of \(y\) using the inverse transformation

\begin{equation}\label{eqn:rotationGen:foo10}
\begin{aligned}
x = R y \tilde{R}
\end{aligned}
\end{equation}

Assuming \(R\) is parametrized by \(\theta\), and that both \(x\) and \(y\) are not directly dependent on \(\theta\), we have

\begin{equation}\label{eqn:rotationGenerator:164}
\begin{aligned}
\frac{dx}{d\theta}
&=
\frac{d R}{d \theta} y \tilde{R} + R y \frac{d \tilde{R} }{d\theta} \\
&=
\left(\frac{d R}{d \theta} \tilde{R} \right) (R y \tilde{R}) + (R y \tilde{R}) \left( R \frac{d \tilde{R} }{d\theta} \right) \\
&=
\left(\frac{d R}{d \theta} \tilde{R} \right) x + x \left( R \frac{d \tilde{R} }{d\theta} \right) \\
\end{aligned}
\end{equation}

Since we also have \(R \tilde{R} = 1\), this product has zero derivative

\begin{equation}\label{eqn:rotationGenerator:184}
\begin{aligned}
0 = \frac{d (R \tilde{R})}{d\theta} = \frac{d R}{d\theta} \tilde{R} + R \frac{d \tilde{R}}{d\theta}
\end{aligned}
\end{equation}

Labeling one of these, say

\begin{equation}\label{eqn:rotationGen:foo11}
\begin{aligned}
\Omega \equiv \frac{d R}{d\theta} \tilde{R}
\end{aligned}
\end{equation}

The multivector \(\Omega\) must in fact be a bivector.  As the product of a grade \(0,2\) multivector with another \(0,2\) multivector, the product may have grades \(0,2,4\).  Since reversing \(\Omega\) negates it, this product can only have grade 2 components.  In particular, employing the exponential representation of \(R\) from \eqnref{eqn:rotationGen:foo3} for a simply parametrized rotation (or boost), we have

\begin{equation}\label{eqn:rotationGen:foo16}
\begin{aligned}
\Omega = \frac{i}{2} e^{i \theta/2} e^{-i \theta/2} = \frac{i}{2}
\end{aligned}
\end{equation}

With this definition we have a

complete description of the incremental (first order) rotational along the curve from \(x\) to \(y\) induced by \(R\) via the commutator of this bivector \(\Omega\) with the initial position vector \(x\).

\begin{equation}\label{eqn:rotationGen:foo12}
\begin{aligned}
\frac{dx}{d\theta} &= \antisymmetric{\Omega}{x} = \inv{2} (i x - x i)
\end{aligned}
\end{equation}

This commutator is in fact the generalize bivector-vector dot product \(\antisymmetric{\Omega}{x} = i \cdot x\), and is vector valued.

Now consider a scalar valued function \(f = f(x(\theta))\).  Employing the chain rule, for the \(theta\) derivative of \(f\) we have a contribution from each coordinate \(x^\mu\).  That is

\begin{equation}\label{eqn:rotationGenerator:204}
\begin{aligned}
\frac{df}{d\theta}
&= \sum_\mu \frac{d x^\mu}{d \theta} \frac{\partial f}{\partial x^\mu}  \\
&= \frac{d x^\mu}{d \theta} \partial_\mu f \\
&= \left( \frac{d x^\mu}{d \theta} \gamma_\mu \right) \cdot \left( \gamma^\nu \partial_\nu \right)  f \\
\end{aligned}
\end{equation}

But this is just
\begin{equation}\label{eqn:rotationGen:foo13}
\begin{aligned}
\frac{df}{d\theta} &= \frac{d x}{d \theta} \cdot \grad f
\end{aligned}
\end{equation}

Or in operator form

\begin{equation}\label{eqn:rotationGen:foo14}
\begin{aligned}
\frac{d}{d\theta} &= (i \cdot x) \cdot \grad
\end{aligned}
\end{equation}

The complete Taylor expansion of \(f(\theta) = f(x(\theta))\) is therefore

\begin{equation}\label{eqn:rotationGenerator:224}
\begin{aligned}
f(x(\theta + \Delta\theta))
&=
\sum_{k=0}^\infty \inv{k!} \left( \Delta\theta \frac{d}{d\theta} \right)^k
f(x(\theta)) \\
&=
\sum_{k=0}^\infty \inv{k!} \left( \Delta\theta (i \cdot x) \cdot \grad \right)^k
f(x(\theta))
\end{aligned}
\end{equation}

Expressing this sum formally as an exponential we have

\begin{equation}\label{eqn:rotationGen:foo20}
\begin{aligned}
f(x(\theta + \Delta\theta)) = e^{\Delta\theta ( i \cdot x) \cdot \grad} f(x(\theta))
\end{aligned}
\end{equation}

In this form, the product \((i \cdot x) \cdot \grad\) does not look much like the cross or wedge product representations of the angular momentum operator that was initially guessed at.  Referring to \cref{fig:rotationGen:bivectorDot} let us make a couple observations about this particular form before translating back to the wedge formulation.

\imageFigure{../figures/gabook/bivectorDot}{Bivector dot product with vector}{fig:rotationGen:bivectorDot}{0.4}

It is worth pointing out that any bivector has no unique vector factorization.  For example any of the following are equivalent

\begin{equation}\label{eqn:rotationGenerator:244}
\begin{aligned}
i
&= \ucap \wedge \vcap \\
&= (2 \ucap) \wedge (\vcap/2 + \alpha \ucap) \\
&= \inv{\alpha b - \beta a} (\alpha \ucap + \beta \vcap ) \wedge (a \ucap + b \vcap)
\end{aligned}
\end{equation}

For this reason if we factor a bivector into two vectors within the plane we are free to pick one of these in any direction we please and can pick the other in one of the perpendiculars within the plane.  In the figure exactly this was done, factoring the bivector into two perpendicular vectors \(i = \ucap \vcap\), where \(\ucap\) was picked to be in the direction of the projection of the vector \(\Bx\) onto the plane spanned by \(\{\ucap, \vcap\}\).  Suppose that projection of \(\Bx\) onto the plane is \(\alpha \ucap\).  We then have for the bivector vector dot product

\begin{equation}\label{eqn:rotationGenerator:264}
\begin{aligned}
i \cdot \Bx
&=
(\ucap\vcap) \cdot (\alpha \ucap) \\
&=
\alpha \ucap\vcap \ucap \\
&=
-\alpha
\mathLabelBox
[
   labelstyle={below of=m\themathLableNode, below of=m\themathLableNode}
]
{\ucap\ucap}{\(=1\)} \vcap \\
\end{aligned}
\end{equation}

So we have for the dot product \(i \cdot \Bx = - \alpha \vcap\), a rotation in the plane of the projection of the vector \(\Bx\) onto the plane by 90 degrees.  The direction of the rotation is metric dependent, and a spatially positive metric was used in this example.  Observe that the action of a bivector product on a vector, provided that vector is in the plane spanned by the factors of the bivector is very much like the complex imaginary action.  In both cases we have a 90 degree rotation.  This complex number correspondence is not entirely equivalent though, since we also have \(i \cdot \Bx = -\Bx \cdot i\), a negation on reversal of the product ordering, whereas we do not have to worry about commuting the imaginary of complex arithmetic.

This shows how the bivector dot product naturally encodes a rotation.  We could leave things this way, but we also want to see how to put this in a more ``standard'' form.  This is possible by rewriting the scalar product using a scalar grade selection operator.  Also employing the cyclic reordering identity \(\gpgradezero{ a b c } = \gpgradezero{ b c a}\), we have

\begin{equation}\label{eqn:rotationGenerator:284}
\begin{aligned}
( i \cdot x) \cdot \grad
&=
\inv{2} \gpgradezero{ ( i x - x i) \grad } \\
&=
\inv{2} \gpgradezero{ i x \grad - \grad x i } \\
\end{aligned}
\end{equation}

A pause is required to note that this reordering needs to be interpreted with \(x\) fixed with respect to the gradient so that the gradient is acting only to the extreme right.  Then we have

\begin{equation}\label{eqn:rotationGen:foo17}
\begin{aligned}
( i \cdot x) \cdot \grad
&=
\inv{2} \gpgradezero{ i (x \cdot \grad) - (x \cdot \grad) i } + \inv{2} \gpgradezero{ i (x \wedge \grad) + (x \cdot \grad) i }
\end{aligned}
\end{equation}

The rightmost action of the gradient allows the gradient dot and wedge products to be reordered (with interchange of sign for the wedge).  The product in the first scalar selector has only bivector terms, so we are left with

\begin{equation}\label{eqn:rotationGen:foo18}
\begin{aligned}
( i \cdot x) \cdot \grad
&=
i \cdot (x \wedge \grad)
\end{aligned}
\end{equation}

and the rotation operator takes the postulated form

\begin{equation}\label{eqn:rotationGen:foo19}
\begin{aligned}
f(x(\theta + \Delta\theta)) = e^{\Delta\theta i \cdot (x \wedge \grad)} f(x(\theta))
\end{aligned}
\end{equation}

While the cross product formulation of this is fine for 3D, this works in a plane when desired, as well as higher dimensional spaces as well as optionally non-Euclidean spaces like the Minkowski space required for electrodynamics and relativity.

\section{Coordinate expansion}

We have seen the structure of the scalar angular momentum operator of \eqnref{eqn:rotationGen:foo18} in the context of components of the cross product angular momentum operator in 3D spaces.  For a more general space what do we have?

Let \(i = \gamma_\beta \gamma_\alpha\), then we have

\begin{equation}\label{eqn:rotationGenerator:304}
\begin{aligned}
i \cdot (x \wedge \grad)
&=
(\gamma_\beta \wedge \gamma_\alpha) \cdot (\gamma^\mu  \wedge \gamma^\nu) x_\mu \partial_\nu \\
&=
({\delta_\beta}^\nu {\delta_\alpha}^\mu - {\delta_\beta}^\mu {\delta_\alpha}^\nu ) x_\mu \partial_\nu \\
\end{aligned}
\end{equation}

which is

\begin{equation}\label{eqn:rotationGen:foo40}
\begin{aligned}
(\gamma_\beta \wedge \gamma_\alpha) \cdot (x \wedge \grad) &= x_\alpha \partial_\beta -x_\beta \partial_\alpha
\end{aligned}
\end{equation}

In particular, in the four vector Minkowski space, when the pair \(\alpha,\beta\) includes both space and time indices we loose (or gain) negation in this operator sum. For example with \(i = \gamma_1 \gamma_0\), we have

\begin{equation}\label{eqn:rotationGen:foo41}
\begin{aligned}
(\gamma_1 \wedge \gamma_0) \cdot (x \wedge \grad)
&= x^0 \frac{\partial}{\partial x^1} + x^1 \frac{\partial}{\partial x^0}
\end{aligned}
\end{equation}

We can also generalize the coordinate expansion of \eqnref{eqn:rotationGen:foo40} to a more general plane of rotation.  Suppose that \(u\) and \(v\) are two perpendicular unit vectors in the plane of rotation.  For this rotational plane we have \(i=u v = u \wedge v\), and our expansion is

\begin{equation}\label{eqn:rotationGenerator:324}
\begin{aligned}
i \cdot (x \wedge \grad)
&=
(\gamma_\beta \wedge \gamma_\alpha) \cdot (\gamma^\mu  \wedge \gamma^\nu) u^\beta v^\alpha x_\mu \partial_\nu \\
&=
({\delta_\beta}^\nu {\delta_\alpha}^\mu - {\delta_\beta}^\mu {\delta_\alpha}^\nu ) u^\beta v^\alpha x_\mu \partial_\nu \\
\end{aligned}
\end{equation}

So we have

\begin{equation}\label{eqn:rotationGen:foo42}
\begin{aligned}
i \cdot (x \wedge \grad)
&=
(u^\nu v^\mu - u^\mu v^\nu) x_\mu \partial_\nu
\end{aligned}
\end{equation}

This scalar antisymmetric mixed index object is apparently called a vierbien (not a tensor) and written

\begin{equation}\label{eqn:rotationGen:foo43}
\begin{aligned}
\epsilon^{\nu\mu} = (u^\nu v^\mu - u^\mu v^\nu)
\end{aligned}
\end{equation}

It would be slightly prettier to raise the index on \(x^\mu\) (and correspondingly lower the \(\mu\)s in \(\epsilon\)).  We then have a completely non Geometric Algebra representation of the angular momentum operator for higher dimensions (and two dimensions) as well as for the Minkowski (and other if desired) metrics.

\begin{equation}\label{eqn:rotationGen:foo44}
\begin{aligned}
i \cdot (x \wedge \grad)
&=
{\epsilon^{\nu}}_\mu x^\mu \partial_\nu
\end{aligned}
\end{equation}

\section{Matrix treatment}

It should be more accessible to do the same sort of treatment with matrices than the Geometric Algebra approach.  It did not occur to me to try it that way initially, and it is worthwhile to do a comparative derivation.  Setup should be similar

\begin{equation}\label{eqn:rotationGen:boo1}
\begin{aligned}
\By &= R \Bx \\
\Bx &= R^\T \By
\end{aligned}
\end{equation}

Taking derivatives we then have

\begin{equation}\label{eqn:rotationGenerator:344}
\begin{aligned}
\frac{d \Bx}{d\theta}
&= \frac{d R^\T}{d\theta} \By \\
&= \frac{d R^\T}{d\theta} R R^\T \By \\
&= \left( \frac{d R^\T}{d\theta} R \right) \Bx \\
\end{aligned}
\end{equation}

Introducing an \(\Omega = (dR^\T/d\theta) R\) very much like before we can write this

\begin{equation}\label{eqn:rotationGen:boo2}
\begin{aligned}
\frac{d \Bx}{d\theta} = \Omega \Bx
\end{aligned}
\end{equation}

For Euclidean spaces (where \(R^{-1} = R^\T\) as assumed above), we have \(R^\T R = 1\), and thus

\begin{equation}\label{eqn:rotationGen:boo3}
\begin{aligned}
\Omega = \frac{dR^\T}{d\theta} R = -R^\T \frac{dR}{d\theta}
\end{aligned}
\end{equation}

Transposition shows that this matrix \(\Omega\) is completely antisymmetric since we have

\begin{equation}\label{eqn:rotationGen:boo4}
\begin{aligned}
\Omega^\T = -\Omega
\end{aligned}
\end{equation}

Now, is there a convenient formulation for a general plane rotation in matrix form, perhaps like the Geometric exponential form?  Probably can be done, but considering an x,y plane rotation should give the rough idea.

\begin{equation}\label{eqn:rotationGen:boo5}
\begin{aligned}
R_\theta =
\begin{bmatrix}
\cos\theta & -\sin\theta & 0 \\
\sin\theta &  \cos\theta & 0 \\
0 & 0 & 1 \\
\end{bmatrix}
\end{aligned}
\end{equation}

After a bit of algebra we have

\begin{equation}\label{eqn:rotationGen:boo6}
\begin{aligned}
\Omega =
\begin{bmatrix}
0 & 1 & 0 \\
-1 & 0 & 0 \\
0 & 0 & 0 \\
\end{bmatrix}
\end{aligned}
\end{equation}

In general we must have

\begin{equation}\label{eqn:rotationGen:boo7}
\begin{aligned}
\Omega =
\begin{bmatrix}
0 & -c & b \\
c & 0 & -a \\
b & a & 0 \\
\end{bmatrix}
\end{aligned}
\end{equation}

For some \(a,b,c\).  This procedure is not intrinsically three dimension, but in the specific 3D case, we can express this antisymetrization using the cross product.  Writing \(\ncap = (a,b,c)\) for the vector with these components, we have in the 3D case only

\begin{equation}\label{eqn:rotationGen:boo8}
\begin{aligned}
\Omega \Bx = \ncap \cross \Bx
\end{aligned}
\end{equation}

The first order rotation of a function \(f(\Bx(\theta))\) now follows from the chain rule as before

\begin{equation}\label{eqn:rotationGenerator:364}
\begin{aligned}
\frac{df}{d\theta}
&=
\frac{d x^m}{d\theta}
\frac{\partial f}{\partial x^m}
\\
&=
\frac{d \Bx}{d\theta} \cdot \spacegrad f
\\
&=
(\Bn \cross \Bx) \cdot \spacegrad f
\\
\end{aligned}
\end{equation}

We have then for the first order rotation derivative operator in 3D

\begin{equation}\label{eqn:rotationGen:boo9}
\begin{aligned}
\frac{d}{d\theta} &= \Bn \cdot (\Bx \cross \spacegrad)
\end{aligned}
\end{equation}

For higher (or 2D) spaces one cannot use the cross product so a more general expression of the result \eqnref{eqn:rotationGen:boo9} would be

\begin{equation}\label{eqn:rotationGen:boo10}
\begin{aligned}
\frac{d}{d\theta} &= (\Omega \Bx) \cdot \spacegrad
\end{aligned}
\end{equation}

Now, in this outline was a fair amount of cheating.  We know that \(\ncap\) is the unit normal to the rotational plane, but that has not been shown here.  Instead it was a constructed quantity just pulled out of thin air knowing it would be required.  If one were interested in pursuing a treatment of the rotation generator operator strictly using matrix algebra, that would have to be considered.  More troublesome and non-obvious is how this would translate to other metric spaces, where we do not necessarily have the transpose relationships to exploit.

%\EndArticle
