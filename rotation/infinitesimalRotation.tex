%
% Copyright � 2012 Peeter Joot.  All Rights Reserved.
% Licenced as described in the file LICENSE under the root directory of this GIT repository.
%

%
%
%\input{../peeter_prologue_print.tex}
%\input{../peeter_prologue_widescreen.tex}

%\chapter{Infinitesimal rotations}
\index{rotation!infinitesimal}
\label{chap:infinitesimalRotation}

%\blogpage{http://sites.google.com/site/peeterjoot2/math2012/infinitesimalRotation.pdf}
%\date{Jan 27, 2012}
%\revisionInfo{infinitesimalRotation.tex}

\beginArtWithToc
%\beginArtNoToc

\section{Motivation}

In a classical mechanics lecture (which I audited) Prof. Poppitz made the claim that an infinitesimal rotation in direction \(\ncap\) of magnitude \(\delta \phi\) has the form

\begin{equation}\label{eqn:continuumL6:10}
\Bx \rightarrow \Bx + \delta \Bphi \cross \Bx,
\end{equation}

where

\begin{equation}\label{eqn:continuumL6:30}
\delta \Bphi = \ncap \delta \phi.
\end{equation}

I believe he expressed things in terms of the differential displacement

\begin{equation}\label{eqn:continuumL6:50}
\delta \Bx = \delta \Bphi \cross \Bx
\end{equation}

This was verified for the special case \(\ncap = \zcap\) and \(\Bx = x \xcap\).  Let us derive this in the general case too.

\section{With geometric algebra}

Let us temporarily dispense with the normal notation and introduce two perpendicular unit vectors \(\ucap\), and \(\vcap\) in the plane of the rotation.  Relate these to the unit normal with

\begin{equation}\label{eqn:continuumL6:70}
\ncap = \ucap \cross \vcap.
\end{equation}

A rotation through an angle \(\phi\) (infinitesimal or otherwise) is then

\begin{equation}\label{eqn:continuumL6:90}
\Bx \rightarrow
e^{-\ucap \vcap \phi/2} \Bx e^{\ucap \vcap \phi/2}.
\end{equation}

Suppose that we decompose \(\Bx\) into components in the plane and in the direction of the normal \(\ncap\).  We have

\begin{equation}\label{eqn:continuumL6:110}
\Bx = x_u \ucap + x_v \vcap + x_n \ncap.
\end{equation}

The exponentials commute with the \(\ncap\) vector, and anticommute otherwise, leaving us with

\begin{equation}\label{eqn:infinitesimalRotation:150}
\begin{aligned}
\Bx
&\rightarrow
x_n \ncap +
(x_u \ucap + x_v \vcap) e^{\ucap \vcap \phi} \\
&=
x_n \ncap +
(x_u \ucap + x_v \vcap) (\cos\phi + \ucap \vcap \sin\phi) \\
&=
x_n \ncap +
\ucap (x_u \cos\phi - x_v \sin\phi)
+
\vcap (x_v \cos\phi + x_u \sin\phi).
\end{aligned}
\end{equation}

In the last line we use \(\ucap^2 = 1\) and \(\ucap \vcap = - \vcap \ucap\).  Making the angle infinitesimal \(\phi \rightarrow \delta \phi\) we have

\begin{equation}\label{eqn:infinitesimalRotation:170}
\begin{aligned}
\Bx
&\rightarrow
x_n \ncap +
\ucap (x_u - x_v \delta\phi)
+
\vcap (x_v + x_u \delta\phi)  \\
&=
\Bx + \delta\phi( x_u \vcap - x_v \ucap)
\end{aligned}
\end{equation}

We have only to confirm that this matches the assumed cross product representation

\begin{equation}\label{eqn:infinitesimalRotation:190}
\begin{aligned}
\ncap \cross \Bx
&=
\begin{vmatrix}
\ucap & \vcap & \ncap \\
0 & 0 & 1 \\
x_u & x_v & x_n
\end{vmatrix} \\
&=
-\ucap x_v + \vcap x_u
\end{aligned}
\end{equation}

Taking the two last computations we find

\begin{equation}\label{eqn:continuumL6:130}
\delta \Bx = \delta \phi \ncap \cross \Bx = \delta \Bphi \cross \Bx,
\end{equation}

as desired.

\section{Without geometric algebra}

We have also done the setup above to verify this result without GA.  Here we wish to apply the rotation to the coordinate vector of \(\Bx\) in the \(\{\ucap, \vcap, \ncap\}\) basis which gives us

\begin{equation}\label{eqn:infinitesimalRotation:210}
\begin{aligned}
\begin{bmatrix}
x_u \\
x_v \\
x_n
\end{bmatrix}
&\rightarrow
\begin{bmatrix}
\cos\delta\phi & -\sin\delta\phi & 0 \\
\sin\delta\phi & \cos\delta\phi & 0 \\
0 & 0 & 1
\end{bmatrix}
\begin{bmatrix}
x_u \\
x_v \\
x_n
\end{bmatrix} \\
&\approx
\begin{bmatrix}
1 & -\delta\phi & 0 \\
\delta\phi & 1 & 0 \\
0 & 0 & 1
\end{bmatrix}
\begin{bmatrix}
x_u \\
x_v \\
x_n
\end{bmatrix} \\
&=
\begin{bmatrix}
x_u \\
x_v \\
x_n
\end{bmatrix}
+
\begin{bmatrix}
0 & -\delta\phi & 0 \\
\delta\phi & 0 & 0 \\
0 & 0 & 0
\end{bmatrix}
\begin{bmatrix}
x_u \\
x_v \\
x_n
\end{bmatrix} \\
&=
\begin{bmatrix}
x_u \\
x_v \\
x_n
\end{bmatrix}
+
\delta\phi
\begin{bmatrix}
-x_v \\
x_u \\
0
\end{bmatrix}
\end{aligned}
\end{equation}

But as we have shown, this last coordinate vector is just \(\ncap \cross \Bx\), and we get our desired result using plain old fashioned matrix algebra as well.

Really the only difference between this and what was done in class is that there is no assumption here that \(\Bx = x \xcap\).

%%\EndArticle
%\EndNoBibArticle
