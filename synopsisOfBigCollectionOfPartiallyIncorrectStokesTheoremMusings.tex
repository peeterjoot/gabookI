%
% Copyright � 2013 Peeter Joot.  All Rights Reserved.
% Licenced as described in the file LICENSE under the root directory of this GIT repository.
%
\input{../latex/blogpost.tex}
\renewcommand{\basename}{stokesTheoremGeometricAlgebra}
\renewcommand{\dirname}{notes/gabook/}
\newcommand{\keywords}{Stokes theorem, Geometric algebra, Clifford algebra, gradient, divergence, wedge product}

\input{../latex/peeter_prologue_print2.tex}
\usepackage{peeters_layout_exercise}
\usepackage{peeters_figures}

% ointctr...
\usepackage{txfonts}

\beginArtNoToc

\generatetitle{Synposis of old notes on Stokes theorem in Geometric algebra}
%\chapter{Stokes theorem in Geometric algebra}
\label{chap:stokesTheoremGeometricAlgebra}

The generalization of Stokes theorem to higher dimesional spaces, expressed in the formalism of geometric algebra takes the form

\begin{equation}\label{eqn:stokesTheoremGeometricAlgebra:120}
\int_V d^k x \cdot (\grad \wedge f) = \int_{\partial V} d^{k-1} x \cdot f.
\end{equation}

To give this enough specific meaning to be useful takes some work.  That will be attempted here.

\section{Notation}

A finite vector space with basis \(\{\gamma_1, \gamma_2, \cdots\}\) will be assumed.  This need not be a Euclidean vector space.  A dual or reciprocal basis \(\{\gamma^1, \gamma^2, \cdots\}\) for this basis can be calculated, defined by the property

\begin{equation}\label{eqn:stokesTheoremGeometricAlgebra:20}
\gamma_i \cdot \gamma^j = {\delta_i}^j.
\end{equation}

Implicit summation over repeated indices will be employed unless otherwise noted.  For example, the components of a vector \(x\) with respect to the standard or reciprocal bases, are

\begin{equation}\label{eqn:stokesTheoremGeometricAlgebra:40}
x = \gamma_i x^i = \gamma^j x_j.
\end{equation}

The coordinates of the vector follow by taking dot products

\begin{subequations}
\begin{equation}\label{eqn:stokesTheoremGeometricAlgebra:60}
x \cdot \gamma^j = \lr{ \gamma_i x^i } \cdot \gamma^j = x^i {\delta_i}^j = x^j
\end{equation}
\begin{equation}\label{eqn:stokesTheoremGeometricAlgebra:80}
x \cdot \gamma_j = \lr{ \gamma^i x_i } \cdot \gamma_j = x_i {\delta^i}_j = x_j
\end{equation}
\end{subequations}

Similarly, a bivector \(f\) in coordinate representation has the form

\begin{equation}\label{eqn:stokesTheoremGeometricAlgebra:1180}
f = \inv{2} \gamma^i \wedge \gamma^j f_{ij} = \inv{2} \gamma_i \wedge \gamma_j f^{ij},
\end{equation}

where
\begin{equation}\label{eqn:stokesTheoremGeometricAlgebra:1200}
\begin{aligned}
f_{ij} &= \lr{ f \cdot \gamma_j } \cdot \gamma_i \\
f^{ij} &= \lr{ f \cdot \gamma^j } \cdot \gamma^i.
\end{aligned}
\end{equation}

The gradient will be expressed in mixed coordinates as

\begin{equation}\label{eqn:stokesTheoremGeometricAlgebra:100}
\grad \equiv \gamma^i \PD{x^i}{} \equiv \gamma^i \partial_i.
\end{equation}

To select from a multivector \(a\) the grade \(k\) portion, say \(a_k\) we write

\begin{equation}\label{eqn:stokesTheoremGeometricAlgebra:1220}
a_k = \gpgrade{a}{k}.
\end{equation}

The scalar portion of a multivector \(a\) will be written as

\begin{equation}\label{eqn:stokesTheoremGeometricAlgebra:1240}
\gpgrade{a}{0} \equiv \gpgradezero{a}.
\end{equation}

The grade selection operators can be used to define the outer and inner products.  For blades \(u\), and \(v\) of grade \(r\) and \(s\) respectively, these are

\begin{subequations}
\begin{dmath}\label{eqn:stokesTheoremGeometricAlgebra:300}
\gpgrade{ u v }{\Abs{r + s}} \equiv u \wedge v
\end{dmath}
\begin{dmath}\label{eqn:stokesTheoremGeometricAlgebra:780}
\gpgrade{ u v }{\Abs{r - s}} \equiv u \cdot v.
\end{dmath}
\end{subequations}

Written out explicitly for odd grade blades \(A\) (vector, trivector, ...), and vector \(a\) the dot and wedge products are respectively

\begin{equation}\label{eqn:stokesTheoremGeometricAlgebra:800}
\begin{aligned}
a \wedge A &= \inv{2} (a A - A a) \\
a \cdot A &= \inv{2} (a A + A a).
\end{aligned}
\end{equation}

Similarly for even grade blades these are

\begin{equation}\label{eqn:stokesTheoremGeometricAlgebra:820}
\begin{aligned}
a \wedge A &= \inv{2} (a A + A a) \\
a \cdot A &= \inv{2} (a A - A a).
\end{aligned}
\end{equation}

It will be useful to employ the cyclic scalar reordering identity for the scalar selection operator

\begin{equation}\label{eqn:stokesTheoremGeometricAlgebra:920}
\gpgradezero{a b c}
= \gpgradezero{b c a}
= \gpgradezero{c a b}.
\end{equation}

\section{Fundamental theorem of calculus, Stokes theorem for scalar functions.}

% from vectorIntegralRelations.tex

The fundamental theorem of calculus, in its vector formulation, is the simplest specific example of Stokes theorem, relating the line integral of the gradient of a scalar function to the value of that function at the end points of the curve.

Given any curve \(C\) with end points \(x_1\) and \(x_2\), and a vector \(f(x)\), this theorem states

\maketheorem{Fundamental theorem of calculus}{thm:stokesTheoremGeometricAlgebra:140}{
\begin{equation*}
\int_C dx \cdot \grad f = f(x_2) - f(x_1).
\end{equation*}
}

The theorem follows by introducing a parameterization \(\lambda\) for the points along the curve.  With such a parameterization, the differential element is

\begin{equation}\label{eqn:stokesTheoremGeometricAlgebra:260}
dx = \gamma_i \frac{d x^i}{d\lambda} d\lambda.
\end{equation}

The differential dotted with the gradient is

\begin{dmath}\label{eqn:stokesTheoremGeometricAlgebra:280}
dx
\cdot
\grad f
=
\left(\gamma_{j} \frac{d x^{j}}{d\lambda} \right) d\lambda
\cdot
\left(\gamma^{i} \partial_{i} f\right)
= {\delta^{i}}_{j} \PD{x^{i}}{f} \frac{d x^{j}}{d\lambda} d\lambda
= \PD{x^{i}}{f} \frac{d x^{i}}{d\lambda} d\lambda
= \frac{d f}{d \lambda} d\lambda.
\end{dmath}

Integration clearly proves \cref{thm:stokesTheoremGeometricAlgebra:140} for this specific parameterization.  This result, however, is independent of the parameterization.  This can be shown by considering any other curve \(C'\), parameterized by \(\sigma\).  As above, for this curve the differential dotted with the gradient is

\begin{dmath}\label{eqn:stokesTheoremGeometricAlgebra:320}
dx \cdot
\grad f
= \frac{d f}{d \sigma} d\sigma.
\end{dmath}

Utilizing a change of variables, this can be expressed in terms of the \(\lambda\) parameterization

\begin{equation}\label{eqn:stokesTheoremGeometricAlgebra:340}
\frac{d f}{d \sigma} d\sigma
= \frac{d f}{d \lambda} \frac{d\lambda}{d\sigma} d\sigma
= \frac{d f}{d \lambda} d\lambda.
\end{equation}

Thus, after integrating, regardless of the parameterization, this integral is dependent on only the end points.  There is probably something that should be said here about the region itself (i.e. an open region), but I will neglect that dtail here.

Finally, but not intuitively, this line integral can be written in outer product notation.  Application of \eqnref{eqn:stokesTheoremGeometricAlgebra:300} to the gradient and the scalar function gives

\begin{equation}\label{eqn:stokesTheoremGeometricAlgebra:360}
\grad \wedge f = \gpgrade{\grad f}{1} = \grad f.
\end{equation}

This allows the fundamental theorem to be expressed in Stokes form

\begin{equation}\label{eqn:stokesTheoremGeometricAlgebra:380}
\int_C dx \cdot \lr{ \grad \wedge f} = f(x_2) - f(x_1).
\end{equation}

\section{Stokes theorem for vector functions on a surface}

% based on stokesGradeTwo.tex

The next task is to assign meaning to the Stokes theorem applied to vector functions \(f\)

\maketheorem{Stokes theorem for vector functions}{thm:stokesTheoremGeometricAlgebra:540}{
\begin{equation*}
\int_S d^2 x \cdot \lr{ \grad \wedge f } = \ointclockwise dx \cdot f.
\end{equation*}
}

There is an orientation to both the area element and the boundary integral.  These are illustrated in \cref{fig:stokesTheoremGeometricAlgebraAreaIntegral:stokesTheoremGeometricAlgebraAreaIntegralFig1}.

FIXME: used subscript for the parameter ranges in the figures.

\imageFigure{../../figures/gabook/stokesTheoremGeometricAlgebraAreaIntegralFig1}{Two variable surface integral parameterization}{fig:stokesTheoremGeometricAlgebraAreaIntegral:stokesTheoremGeometricAlgebraAreaIntegralFig1}{0.3}

Given a two parameter surface of points \(x(\alpha, \beta)\), the area element at a point can be formed by wedging two differential elements.  Let

\begin{equation}\label{eqn:stokesTheoremGeometricAlgebra:540}
\begin{aligned}
dx_\alpha &= \PD{\alpha}{x} d\alpha \\
dx_\beta &= \PD{\beta}{x} d\beta,
\end{aligned}
\end{equation}

so that the area element is

\begin{dmath}\label{eqn:stokesTheoremGeometricAlgebra:560}
d^2 x
= dx_\alpha \wedge dx_\beta
=
\gamma_i \wedge \gamma_j \PD{\alpha}{x^i} \PD{\beta}{x^j} d\alpha d\beta.
\end{dmath}

Presuming the area element is made small enough, and the surface is sufficiently smooth, this area element will lie in the subspace of the surface.

Expressed in terms of coordinates, the curl is

\begin{dmath}\label{eqn:stokesTheoremGeometricAlgebra:580}
\grad \wedge f
=
\lr{ \gamma^i \partial_i } \wedge \lr{ \gamma^j f_j }
=
\gamma^i \wedge \gamma^j \partial_i f_j.
\end{dmath}

The curl and the area element dotted together provide the differential form for the surface integral

\begin{equation}\label{eqn:stokesTheoremGeometricAlgebra:400}
\begin{aligned}
d^2 x \cdot \lr{ \grad \wedge f }
&=
\lr{ \gamma_i \wedge \gamma_j \PD{\alpha}{x^i} \PD{\beta}{x^j} d\alpha d\beta }
\cdot
\lr{
\gamma^r \wedge \gamma^s \partial_r f_s
}
\\
&=
\partial_r f_s \PD{\alpha}{x^i} \PD{\beta}{x^j} (\gamma^r \wedge \gamma^s) \cdot (\gamma_i \wedge \gamma_j)
d\alpha d\beta \\
&=
\partial_r f_s \PD{\alpha}{x^i} \PD{\beta}{x^j} ( {\delta^r}_j {\delta^s}_i - {\delta^r}_i {\delta^s}_j )
d\alpha d\beta \\
&=
\partial_r f_s \left( \PD{\alpha}{x^s} \PD{\beta}{x^r} - \PD{\alpha}{x^r} \PD{\beta}{x^s} \right)
d\alpha d\beta \\
\end{aligned}
\end{equation}

As a sum over all the Jacobian factors that is

\begin{equation}\label{eqn:stokesTheoremGeometricAlgebra:420}
d^2 x  \cdot
\lr{ \grad \wedge f }
= -\partial_r f_s \frac{\partial (x^r, x^s)}{\partial (\alpha, \beta)} d\alpha d\beta.
\end{equation}

Now, consider the loop integral, integrating clockwise

\begin{equation}\label{eqn:stokesTheoremGeometricAlgebra:440}
\begin{aligned}
\ointclockwise f \cdot dx
&=
I_1 + I_2 + I_3 + I_4 \\
&=
\int_{\beta^0}^{\beta^1} dx_\beta \cdot
f(\alpha^0, \beta)
+\int_{\alpha^0}^{\alpha^1} dx_\alpha \cdot
f(\alpha, \beta^1)
-\int_{\beta^0}^{\beta^1} dx_\beta \cdot
f(\alpha^1, \beta)
-\int_{\alpha^0}^{\alpha^1} dx_\alpha \cdot
f(\alpha, \beta^0) \\
&=
\int_{\alpha^0}^{\alpha^1} dx_\alpha \cdot
\lr{
f(\alpha, \beta^1)
-
f(\alpha, \beta^0)
}
-\int_{\beta^0}^{\beta^1} dx_\beta \cdot \lr{
f(\alpha^1, \beta) - f(\alpha^0, \beta)
} \\
&=
\int_{\alpha^0}^{\alpha^1} d\alpha \PD{\alpha}{x^i}
\lr{
f_i(\alpha, \beta^1)
-
f_i(\alpha, \beta^0)
}
-\int_{\beta^0}^{\beta^1} d\beta \PD{\beta}{x^i}
\lr{
f_i(\alpha^1, \beta) - f_i(\alpha^0, \beta)
} \\
&=
\int_{\alpha^0}^{\alpha^1} d\alpha
\PD{\alpha}{x^i}
\int_{\beta^0}^{\beta^1} d\beta
\PD{\beta}{f_i}
-
\int_{\beta^0}^{\beta^1} d\beta
\PD{\beta}{x^i}
\int_{\alpha^0}^{\alpha^1} d\alpha
\PD{\alpha}{f_i}
\\
&=
\int d\alpha d\beta
\lr{
\PD{\alpha}{x^i}
\PD{\beta}{f_i}
-
\PD{\beta}{x^i}
\PD{\alpha}{f_i}
}
\\
&=
\int d\alpha d\beta
\lr{
\PD{\alpha}{x^i}
\PD{x^j}{f_i}
\PD{\beta}{x^j}
-
\PD{\beta}{x^i}
\PD{x^j}{f_i}
\PD{\alpha}{x^j}
}
\\
&=
\int d\alpha d\beta
\partial_j f_i
\PD{(\alpha, \beta)}{(x^i, x^j)}.
\end{aligned}
\end{equation}

A change of variables and integration of \eqnref{eqn:stokesTheoremGeometricAlgebra:420} produces this same result, demonstrating this theorem for this particular parameterization.  Again, assuming an open region, a change of variables can be made to produce the same result for any other surface.

\section{Stokes theorem for bivector functions over a volume}
% also based on stokesGradeTwo.tex

Moving on to bivector functions \(f\), the task is to give specific meaning to

\maketheorem{Stokes theorem for bivector functions}{thm:stokesTheoremGeometricAlgebra:600}{
\begin{equation*}
\int_V d^3 x \cdot \lr{ \grad \wedge f } = \iint_{\partial V} d^2 x \cdot f.
\end{equation*}
}

A subspace of all the points \(x(\alpha, \beta, \sigma)\) is illustrated in \cref{fig:stokesTheoremGeometricAlgebraAreaIntegral:stokesTheoremGeometricAlgebraAreaIntegralFig2}.

\imageFigure{../../figures/gabook/stokesTheoremGeometricAlgebraAreaIntegralFig2}{Oriented surface of a three parameter subspace}{fig:stokesTheoremGeometricAlgebraAreaIntegral:stokesTheoremGeometricAlgebraAreaIntegralFig2}{0.3}

This space is spanned by the differentials

\begin{equation}\label{eqn:stokesTheoremGeometricAlgebra:620}
\begin{aligned}
dx_\alpha &= \PD{\alpha}{x} d\alpha \\
dx_\beta &= \PD{\beta}{x} d\beta \\
dx_\sigma &= \PD{\sigma}{x} d\sigma.
\end{aligned}
\end{equation}

Oriented area elements for the front, right and top faces respectively are

\begin{equation}\label{eqn:stokesTheoremGeometricAlgebra:640}
\begin{aligned}
dA_F &= dx_\alpha \wedge dx_\beta \\
dA_R &= dx_\beta \wedge dx_\sigma \\
dA_T &= dx_\sigma \wedge dx_\alpha.
\end{aligned}
\end{equation}


These faces of the subspace are the surfaces of constant parametrization, respectively, \(\sigma = \sigma^1\), \(\alpha = \alpha^1\), and \(\beta = \beta^1\).  For a bivector \(f\), the flux through the surface is

\begin{equation}\label{eqn:stokesTheoremGeometricAlgebra:660}
\begin{aligned}
\iint d^2 x \cdot f
&=
\int_{\alpha^0}^{\alpha^1}
\int_{\beta^0}^{\beta^1}
\lr{ dx_\alpha \wedge dx_\beta } \cdot
\lr{
f(\alpha, \beta, \sigma^1)
- f(\alpha, \beta, \sigma^0)
} \\
&+
\int_{\sigma^0}^{\sigma^1}
\int_{\beta^0}^{\beta^1}
\lr{ dx_\beta \wedge dx_\sigma } \cdot
\lr{
f(\alpha^1, \beta, \sigma)
- f(\alpha^0, \beta, \sigma)
} \\
&+
\int_{\sigma^0}^{\sigma^1}
\int_{\alpha^0}^{\alpha^1}
\lr{ dx_\sigma \wedge dx_\alpha } \cdot
\lr{
f(\alpha, \beta^1, \sigma)
- f(\alpha, \beta^0, \sigma)
} \\
&=
\int_{\alpha^0}^{\alpha^1}
\int_{\beta^0}^{\beta^1}
\int_{\sigma^0}^{\sigma^1}
\lr{
\lr{ dx_\alpha \wedge dx_\beta } \cdot
\PD{\sigma}{f}
d\sigma
+
\lr{ dx_\beta \wedge dx_\sigma } \cdot
\PD{\alpha}{f}
d\alpha
+
\lr{ dx_\sigma \wedge dx_\alpha } \cdot
\PD{\beta}{f}
d\beta
} \\
&=
\int_{\alpha^0}^{\alpha^1}
\int_{\beta^0}^{\beta^1}
\int_{\sigma^0}^{\sigma^1}
d\alpha d\beta d\sigma \\
&
\qquad
\lr{
\PD{\alpha}{x^i}
\PD{\beta}{x^j}
\PD{\sigma}{x^k}
\lr{ \gamma_i \wedge \gamma_j } \cdot \PD{x^k}{f}
+
\PD{\beta}{x^j}
\PD{\sigma}{x^k}
\PD{\alpha}{x^i}
\lr{ \gamma_j \wedge \gamma_k } \cdot \PD{x^i}{f}
+
\PD{\alpha}{x^i}
\PD{\sigma}{x^k}
\PD{\beta}{x^j}
\lr{ \gamma_k \wedge \gamma_i } \cdot \PD{x^j}{f} } \\
&=
\int_{\alpha^0}^{\alpha^1}
\int_{\beta^0}^{\beta^1}
\int_{\sigma^0}^{\sigma^1}
d\alpha d\beta d\sigma
\PD{\alpha}{x^i}
\PD{\beta}{x^j}
\PD{\sigma}{x^k}
\lr{
\lr{ \gamma_i \wedge \gamma_j } \cdot \PD{x^k}{f}
+
\lr{ \gamma_j \wedge \gamma_k } \cdot \PD{x^i}{f}
+
\lr{ \gamma_k \wedge \gamma_i } \cdot \PD{x^j}{f}
}
\end{aligned}
\end{equation}

Expanding \(\int_V d^3 x \cdot (\grad \wedge f)\) to compare

\begin{dmath}\label{eqn:stokesTheoremGeometricAlgebra:760}
d^3 x \cdot \lr{ \grad \wedge f }
=
\gpgradezero{
d^3 x \lr{ \gamma^a \wedge \partial_a f }
}
=
\inv{2} \gpgradezero{
d^3 x \lr{
\gamma^a \partial_a f
+
\partial_a f
\gamma^a
}
}
=
\inv{2} \gpgradezero{
d^3 x \gamma^a \partial_a f
+
\gamma^a
d^3 x \partial_a f
}
=
\inv{2}
\partial_a f  \cdot
\gpgradetwo{
d^3 x \gamma^a
+
\gamma^a
d^3 x
}
=
\partial_a f  \cdot
\lr{
d^3 x \cdot \gamma^a
}
=
d\alpha d\beta d\sigma
\PD{\alpha}{x^i}
\PD{\beta}{x^j}
\PD{\sigma}{x^k}
\partial_a f \cdot
\lr{
\lr{ \gamma_i \wedge \gamma_j \wedge \gamma_k }
\cdot
\gamma^a
}
=
d\alpha d\beta d\sigma
\PD{\alpha}{x^i}
\PD{\beta}{x^j}
\PD{\sigma}{x^k}
\partial_a f \cdot
\lr{
\lr{ \gamma_i \wedge \gamma_j } {\delta_k}^a
-\lr{ \gamma_i \wedge \gamma_k } {\delta_j}^a
+\lr{ \gamma_j \wedge \gamma_k } {\delta_i}^a
}
=
d\alpha d\beta d\sigma
\PD{\alpha}{x^i}
\PD{\beta}{x^j}
\PD{\sigma}{x^k}
\lr{
\lr{ \gamma_i \wedge \gamma_j } \cdot \partial_k f
-\lr{ \gamma_i \wedge \gamma_k } \cdot \partial_j f
+\lr{ \gamma_j \wedge \gamma_k } \cdot \partial_i f
}
\end{dmath}

Integrating this yields \eqnref{eqn:stokesTheoremGeometricAlgebra:660}, demonstrating Stokes theorem for this volume parameterization.  Given an open region a change of variables will show that this is true for any parameterization of the subspace.

Like the vector case, there is a requirement to be very specific about the meaning given to the oriented surfaces, and the corresponding oriented volume element.  That volume may be a subspace of a greater than three dimensional space.

\section{Stokes theorem for trivector functions over a four volume}
% from: stokesNoTensor.tex

For integration of the curl of a trivector function, an integration over a four volume subspace is required.
The geometrical intuition required to understand, a-priori, how to express the oriented boundary of such a subspace may not come readily, but fortunately algebra is there to come to our aid.

The workhorse of the Stokes proof for these higher grade objects is

\begin{lemma}\label{thm:stokesTheoremGeometricAlgebra:1260}
For a blade \(f\) of grade \(k-1\)

\begin{equation*}
( \grad \wedge f ) \cdot d^k x = \partial_i f \cdot \lr{ d^k x \cdot \gamma^i }.
\end{equation*}
\end{lemma}

Writing the curl in coordinates and then the dot product as a scalar grade selection, this differential form is

\begin{equation}\label{eqn:stokesTheoremGeometricAlgebra:840}
\begin{aligned}
( \grad \wedge f ) \cdot d^k x
&=
( \gamma^i \wedge \partial_i f ) \cdot d^k x \\
&=
\gpgradezero{ ( \gamma^i \wedge \partial_i f ) d^k x } \\
\end{aligned}
\end{equation}

For odd grades \(f\), using \eqnref{eqn:stokesTheoremGeometricAlgebra:820}, and \eqnref{eqn:stokesTheoremGeometricAlgebra:920}, this is

\begin{equation}\label{eqn:stokesTheoremGeometricAlgebra:900}
\begin{aligned}
( \grad \wedge f ) \cdot d^k x
&=
\inv{2} \gpgradezero{ \gamma^i \partial_i f d^k x } - \inv{2} \gpgradezero{ \partial_i f \gamma^i d^k x } \\
&=
\inv{2} \gpgradezero{ \partial_i f (d^k x \gamma^i - \gamma^i d^k x)} \\
&=
\inv{2} \gpgradezero{ \partial_i f (d^k x \cdot \gamma^i - \gamma^i d^k x)} \\
&=
\gpgradezero{ \partial_i f (d^k x \cdot \gamma^i)}.
\end{aligned}
\end{equation}

This is just \cref{thm:stokesTheoremGeometricAlgebra:1260} written as a scalar selection.  For even grades \(f\), using \eqnref{eqn:stokesTheoremGeometricAlgebra:800}, and \eqnref{eqn:stokesTheoremGeometricAlgebra:920}, this is

\begin{equation}\label{eqn:stokesTheoremGeometricAlgebra:910}
\begin{aligned}
( \grad \wedge f ) \cdot d^k x
&=
\inv{2} \gpgradezero{ \gamma^i \partial_i f d^k x } + \inv{2} \gpgradezero{ \partial_i f \gamma^i d^k x } \\
&=
\inv{2} \gpgradezero{ \partial_i f (d^k x \gamma^i + \gamma^i d^k x)} \\
&=
\inv{2} \gpgradezero{ \partial_i f (d^k x \cdot \gamma^i + \gamma^i d^k x)} \\
&=
\gpgradezero{ \partial_i f (d^k x \cdot \gamma^i)},
\end{aligned}
\end{equation}

completing the proof of this lemma.

Return now to the trivector function \(f\) of interest, for which we assume a subspace parameterization \(f = f(a_1, a_2, a_3, a_4)\).  The points within space \(x = x(a_1, a_2, a_3, a_4)\) generate vector differentials

\begin{equation}\label{eqn:stokesTheoremGeometricAlgebra:n}
dx_m = da_m \PD{a_m}{x}.
\end{equation}

The volume element dot product is

\begin{equation}\label{eqn:stokesTheoremGeometricAlgebra:1000}
\begin{aligned}
d^4 x \cdot \gamma^i
&=
( dx_1 \wedge dx_2 \wedge dx_3 \wedge dx_4 ) \cdot \gamma^i \\
&= ( dx_1 \wedge dx_2 \wedge dx_3 ) dx_4 \cdot \gamma^i \\
&-( dx_1 \wedge dx_2 \wedge dx_4 ) dx_3 \cdot \gamma^i \\
&+( dx_1 \wedge dx_3 \wedge dx_4 ) dx_2 \cdot \gamma^i \\
&-( dx_2 \wedge dx_3 \wedge dx_4 ) dx_1 \cdot \gamma^i  \\
\end{aligned}
\end{equation}

Each of the vector differential dot products have the scalar values

\begin{equation}\label{eqn:stokesTheoremGeometricAlgebra:1020}
\begin{aligned}
dx_m \cdot \gamma^i
&=
da_m \PD{a_m}{x^j} \gamma_j \cdot \gamma^i \\
&=
da_m \PD{a_m}{x^i},
\end{aligned}
\end{equation}

so that

\begin{equation}\label{eqn:stokesTheoremGeometricAlgebra:1040}
\begin{aligned}
\partial_i f da_m \PD{a_m}{x^i}
&=
da_m \PD{a_m}{x^i} \PD{x^i}{f} \\
&=
da_m \PD{a_m}{f}.
\end{aligned}
\end{equation}

With very little work, the differential form \eqnref{eqn:stokesTheoremGeometricAlgebra:1000} is reduced to

\begin{equation}\label{eqn:stokesTheoremGeometricAlgebra:1060}
\begin{aligned}
( \grad \wedge f ) \cdot d^4 x
&= da_4 \PD{a_4}{f} \cdot ( dx_1 \wedge dx_2 \wedge dx_3 ) \\
&- da_3 \PD{a_3}{f} \cdot ( dx_1 \wedge dx_2 \wedge dx_4 ) \\
&+ da_2 \PD{a_2}{f} \cdot ( dx_1 \wedge dx_3 \wedge dx_4 ) \\
&- da_1 \PD{a_1}{f} \cdot ( dx_2 \wedge dx_3 \wedge dx_4 ) \\
\end{aligned}
\end{equation}

Writing \(a_m^1\) and \(a_m^0\) for the end points of the parameters \(a_m\), this is directly integrable

\begin{equation}\label{eqn:stokesTheoremGeometricAlgebra:1100}
\begin{aligned}
\int ( \grad \wedge f ) \cdot d^4 x
&= \int \lr{f(a_1, a_2, a_3, a_4^1) - f(a_1, a_2, a_3, a_4^0)} \cdot ( dx_1 \wedge dx_2 \wedge dx_3 ) \\
&- \int \lr{f(a_1, a_2, a_3^1, a_4) - f(a_1, a_2, a_3^0, a_4)} \cdot ( dx_1 \wedge dx_2 \wedge dx_4 ) \\
&+ \int \lr{f(a_1, a_2^1, a_3, a_4) - f(a_1, a_2^0, a_3, a_4)} \cdot ( dx_1 \wedge dx_3 \wedge dx_4 ) \\
&- \int \lr{f(a_1^1, a_2, a_3, a_4) - f(a_1^0, a_2, a_3, a_4)} \cdot ( dx_2 \wedge dx_3 \wedge dx_4 ).
\end{aligned}
\end{equation}

\section{Fixme}

The differentials \(dx_m\) of \eqnref{eqn:stokesTheoremGeometricAlgebra:1100} are functions of all the free parameters.  There is something similar in \eqnref{eqn:stokesTheoremGeometricAlgebra:640}, which should really be

\begin{equation}\label{eqn:stokesTheoremGeometricAlgebra:641}
\begin{aligned}
dA_F &= \evalbar{\lr{dx_\alpha \wedge dx_\beta}}{\sigma = \sigma^1} \\
dA_R &= \evalbar{\lr{dx_\beta \wedge dx_\sigma}}{\alpha = \alpha^1} \\
dA_T &= \evalbar{\lr{dx_\sigma \wedge dx_\alpha}}{\beta = \beta^1}.
\end{aligned}
\end{equation}

This assumption is used to turn the differences of \(f\) on the faces into an integral without requiring the two forms to be functions of the surface parameters.  The other issue here is that the two forms for the back and front, or left and right, or top and bottom faces could have different orientations.

There is an implicit assumption here that the wedge products of these one forms, which are functions of all the parameters, have been evaluated on the surfaces.
This problem is also to be found in the one form boundary integrals way back in  \eqnref{eqn:stokesTheoremGeometricAlgebra:440} too.  For example \(dx_\beta\) is a function of both both \(\alpha\) and \(\beta\) even if it is defined as a \(\beta\) partial.  It will have different values at the end points \(\beta^0, \beta^1\).

Have to think all this through.  Does this make all the proofing above useless unless the regions and coordinates are restricted to rectangular?

%\EndArticle
\EndNoBibArticle
