%
% Copyright � 2012 Peeter Joot.  All Rights Reserved.
% Licenced as described in the file LICENSE under the root directory of this GIT repository.
%

%
%
%\chapter{Angle between geometric elements}
\index{vectors!angle between}
\label{chap:angleBetweenLineAndPlane}
%\date{Mar 17, 2008.  angleBetweenLineAndPlane.tex}

Have the calculation for the angle between bivectors done elsewhere

\begin{equation}\label{eqn:angleLinePlane:bivectorangle}
\cos\theta = - \frac{\BA \cdot \BB}{\abs{\BA} \abs{\BB} }
\end{equation}

For \(\theta \in [0,\pi]\).

The vector/vector result is well known and also works fine in \R{N}

\begin{equation}\label{eqn:angleLinePlane:vectorangle}
\cos\theta = \frac{\Bu \cdot \Bv}{\abs{\Bu} \abs{\Bv} }
\end{equation}

\section{Calculation for a line and a plane}

Given a line with unit direction vector \(\Bu\), and plane with unit direction bivector \(\BA\), the component of that
vector in the plane is:

\begin{equation}\label{eqn:angleBetweenLineAndPlane:20}
-\Bu \cdot \BA \BA.
\end{equation}

So the direction cosine is available immediately

\begin{equation}\label{eqn:angleBetweenLineAndPlane:40}
\cos\theta = \Bu \cdot \frac{-\Bu \cdot \BA \BA}{\abs{\Bu \cdot \BA \BA}}
\end{equation}

However, this can be reduced significantly.  Start with the denominator

\begin{equation}\label{eqn:angleBetweenLineAndPlane:60}
\begin{aligned}
\abs{\Bu \cdot \BA \BA}^2
&= (\Bu \cdot \BA \BA)(\BA \BA \cdot \Bu) \\
&= (\Bu \cdot \BA )^2. \\
\end{aligned}
\end{equation}

And in the numerator we have:

\begin{equation}\label{eqn:angleBetweenLineAndPlane:80}
\begin{aligned}
\Bu \cdot (\Bu \cdot \BA \BA)
&= \inv{2}(
  \Bu (\Bu \cdot \BA \BA)
+ (\Bu \cdot \BA \BA) \Bu
) \\
&= \inv{2}(
  (\Bu \Bu \cdot \BA) \BA
+ (\Bu \cdot \BA) \BA \Bu
) \\
&= \inv{2}(
  (\BA \cdot \Bu \Bu) \BA
- (\BA \cdot \Bu) \BA \Bu
) \\
&= (\BA \cdot \Bu) \inv{2}( \Bu \BA - \BA \Bu ) \\
&= -(\BA \cdot \Bu)^2.
\end{aligned}
\end{equation}

Putting things back together

\begin{equation*}
\cos\theta
= \frac{(\BA \cdot \Bu)^2}{\abs{\Bu \cdot \BA}} = \abs{\Bu \cdot \BA}
\end{equation*}

The strictly positive value here is consistent with the fact that theta as calculated is in the \([0,\pi/2]\) range.

Restated for consistency with equations \eqnref{eqn:angleLinePlane:vectorangle} and \eqnref{eqn:angleLinePlane:bivectorangle} in terms of not necessarily
unit vector and bivectors \(\Bu\) and \(\BA\), we have

\begin{equation}
\cos\theta =
\frac{\abs{\Bu \cdot \BA}}{ \abs{\Bu} \abs{\BA} }
\end{equation}
