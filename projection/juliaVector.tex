%
% Copyright � 2012 Peeter Joot.  All Rights Reserved.
% Licenced as described in the file LICENSE under the root directory of this GIT repository.
%

%
%
%\input{../peeter_prologue_print.tex}
%\input{../peeter_prologue_widescreen.tex}

%\chapter{Vector form of Julia fractal}
\index{Julia fractal}
\label{chap:juliaVector}

%\blogpage{http://sites.google.com/site/peeterjoot/math2010/juliaVector.pdf}
%\date{Dec 27, 2010}
%\revisionInfo{juliaVector.tex}

%\beginArtWithToc
\beginArtNoToc

\section{Motivation}

As outlined in \citep{dorst2007gac}, 2-D and N-D Julia fractals can be computed using the geometric product, instead of complex numbers.  Explore a couple of details related to that here.

\section{Guts}

Fractal patterns like the Mandelbrot and Julia sets are typically using iterative computations in the complex plane.  For the Julia set, our iteration has the form
%
\begin{equation}\label{eqn:juliaFractal:10}
Z \rightarrow Z^p + C
\end{equation}
%
where \(p\) is an integer constant, and \(Z\), and \(C\) are complex numbers.  For \(p=2\) I believe we obtain the Mandelbrot set.  Given the isomorphism between complex numbers and vectors using the geometric product, we can use write
%
\begin{equation}\label{eqn:juliaFractal:20}
\begin{aligned}
Z &= \Bx \ncap \\
C &= \Bc \ncap,
\end{aligned}
\end{equation}
%
and re-express the Julia iterator as
%
\begin{equation}\label{eqn:juliaFractal:30}
\Bx \rightarrow (\Bx \ncap)^p \ncap + \Bc
\end{equation}
%
It is not obvious that the RHS of this equation is a vector and not a multivector, especially when the vector \(\Bx\) lies in \R{3} or higher dimensional space.  To get a feel for this, let us start by write this out in components for \(\ncap = \Be_1\) and \(p=2\).  We obtain for the product term
%
\begin{equation}\label{eqn:juliaVector:80}
\begin{aligned}
(\Bx \ncap)^p \ncap
&= \Bx \ncap \Bx \ncap \ncap \\
&= \Bx \ncap \Bx \\
&=
(
x_1 \Be_1
+ x_2 \Be_2  )
\Be_1
(
x_1 \Be_1
+ x_2 \Be_2  ) \\
&=
(
x_1
+ x_2 \Be_2 \Be_1 )
(
x_1 \Be_1
+ x_2 \Be_2  ) \\
&=
(
x_1^2 - x_2^2 ) \Be_1 + 2 x_1 x_2 \Be_2
\end{aligned}
\end{equation}
%
Looking at the same square in coordinate representation for the \R{n} case (using summation notation unless otherwise specified), we have
%
\begin{equation}\label{eqn:juliaVector:100}
\begin{aligned}
\Bx \ncap \Bx
&=
x_k \Be_k
\Be_1
x_m \Be_m  \\
&=
\left(x_1 + \sum_{k>1} x_k \Be_k \Be_1\right)
x_m \Be_m  \\
&=
x_1 x_m \Be_m
+
\sum_{k>1} x_k x_m \Be_k \Be_1 \Be_m \\
&=
x_1 x_m \Be_m
+
\sum_{k>1} x_k x_1 \Be_k
+\sum_{k>1,m>1} x_k x_m \Be_k \Be_1 \Be_m \\
&=
\left(x_1^2
-\sum_{k>1} x_k^2\right) \Be_1
+
2 \sum_{k>1} x_1 x_k \Be_k
+\sum_{1 < k < m, 1 < m < k} x_k x_m \Be_k \Be_1 \Be_m \\
\end{aligned}
\end{equation}
%
This last term is zero since \(\Be_k \Be_1 \Be_m = -\Be_m \Be_1 \Be_k\), and we are left with
%
\begin{equation}\label{eqn:juliaFractal:50}
\Bx \ncap \Bx =
\left(x_1^2
-\sum_{k>1} x_k^2\right) \Be_1
+
2 \sum_{k>1} x_1 x_k \Be_k,
\end{equation}
%
a vector, even for non-planar vectors.  How about for an arbitrary orientation of the unit vector in \R{n}?  For that we get
%
\begin{equation}\label{eqn:juliaVector:120}
\begin{aligned}
\Bx \ncap \Bx
&=
(\Bx \cdot \ncap \ncap + \Bx \wedge \ncap \ncap ) \ncap \Bx  \\
&=
(\Bx \cdot \ncap + \Bx \wedge \ncap ) (\Bx \cdot \ncap \ncap + \Bx \wedge \ncap \ncap )
  \\
&=
((\Bx \cdot \ncap)^2 + (\Bx \wedge \ncap)^2) \ncap
+ 2 (\Bx \cdot \ncap) (\Bx \wedge \ncap) \ncap
\end{aligned}
\end{equation}
%
We can read \eqnref{eqn:juliaFractal:50} off of this result by inspection for the \(\ncap = \Be_1\) case.

It is now straightforward to show that the product \((\Bx \ncap)^p \ncap\) is a vector for integer \(p \ge 2\).  We have covered the \(p=2\) case, justifying an assumption that this product has the following form
%
\begin{equation}\label{eqn:juliaFractal:60}
(\Bx \ncap)^{p-1} \ncap = a \ncap + b (\Bx \wedge \ncap) \ncap,
\end{equation}
%
for scalars \(a\) and \(b\).  The induction test becomes
%
\begin{equation}\label{eqn:juliaVector:140}
\begin{aligned}
(\Bx \ncap)^{p} \ncap
&= (\Bx \ncap)^{p-1} (\Bx \ncap) \ncap \\
&= (\Bx \ncap)^{p-1} \Bx \\
&= (a + b (\Bx \wedge \ncap) ) ((\Bx \cdot \ncap )\ncap + (\Bx \wedge \ncap) \ncap) \\
&=
( a(\Bx \cdot \ncap )^2 - b (\Bx \wedge \ncap)^2 ) \ncap
+ ( a + b(\Bx \cdot \ncap ) ) (\Bx \wedge \ncap) \ncap.
\end{aligned}
\end{equation}
%
Again we have a vector split nicely into projective and rejective components, so for any integer power of \(p\) our iterator \eqnref{eqn:juliaFractal:30} employing the geometric product is a mapping from vectors to vectors.

There is a striking image in the text of such a Julia set for such a 3D iterator, and an exercise left for the adventurous reader to attempt to code that based on the 2D \(p=2\) sample code they provide.

%\EndArticle
