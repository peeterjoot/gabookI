%
% Copyright � 2012 Peeter Joot.  All Rights Reserved.
% Licenced as described in the file LICENSE under the root directory of this GIT repository.
%

%
%
\chapter{Projection and Moore-Penrose vector inverse}
\index{projection}
\index{Moore-Penrose inverse}
\label{chap:projectionAndMoorePenroseVectorInverse}
%\date{May 16, 2008.  projectionAndMoorePenroseVectorInverse.tex}

\section{Projection and Moore-Penrose vector inverse}

One can observe that the Moore Penrose left vector inverse \(\Bv^+\) shows up in the projection matrix for a projection onto a line with a direction vector \(\Bv\):

\begin{equation}
\Proj_\Bv(\Bx) = \Bv \mathLabelBox{\inv{ \Bv^\T \Bv} \Bv^\T}{\(\Bv^+\)} \Bx
\end{equation}

I do not know of any other ``application'' of this Moore-Penrose vector inverse in traditional matrix algebra.  As stated it is an interesting mathematical curiosity that yes one can define a vector inverse, however what would you do with it?

In geometric algebra we also have a vector inverse, but it plays a much more fundamental role, and does not have the restriction of only acting from the left and
producing a scalar result.  As an example consider the projection, and rejection decomposition of a vector:

\begin{equation}\label{eqn:projectionAndMoorePenroseVectorInverse:120}
\begin{aligned}
\Bx
&= \Bv \inv{\Bv} \Bx \\
&= \Bv \left(\inv{\Bv} \cdot \Bx\right) + \Bv \left(\inv{\Bv} \wedge \Bx\right) \\
&= \Bv
\left(
\frac{\Bv}{\Bv^2} \cdot
 \Bx\right)
 + \Bv \left(\frac{\Bv}{\Bv^2} \wedge \Bx\right) \\
\end{aligned}
\end{equation}

In the above, \(\frac{\Bv}{\Bv^2} \cdot = \frac{\Bv^\T}{\Bv^\T \Bv} = \Bv^+\).  We can therefore describe the Moore Penrose vector left inverse as the matrix of the GA linear transformation \(\inv{\Bv} \cdot\).

Unlike the GA vector inverse, whos associativity allowed for the projection/rejection derivation above, this Moore-Penrose vector inverse has only left action, so in the above, you can not further write:

\begin{equation}\label{eqn:projectionAndMoorePenroseVectorInverse:20}
\Bv \Bv^{+} = 1
\end{equation}

(ie: \(\Bv \Bv^{+}\) is a projection matrix not scalar or matrix unity).

\subsection{matrix of wedge project transformation?}

Q: What is the matrix of the linear transformation \(\inv{\Bv} \wedge\)?

In rigid body dynamics we see the matrix of the linear transformation \(T_\Bv(\Bx) = (\Bv \cross)(\Bx)\).  This is the completely antisymmetric matrix as follows:

\begin{equation}
\Bv \times \Bx =
\begin{bmatrix}
0 & -v_3 & v_2 \\
v_3 & 0 & -v_1 \\
-v_2 & v_1 & 0 \\
\end{bmatrix}
\begin{bmatrix}
x_1 \\
x_2 \\
x_3 \\
\end{bmatrix}
\end{equation}

In order to specify the matrix of a vector-vector wedge product linear transformation we must introduce bivector coordinate vectors.  For the matrix of the cross product linear transformation the standard vector basis was the obvious choice.

Let us pick the following orthonormal basis:

\begin{equation}\label{eqn:projectionAndMoorePenroseVectorInverse:40}
\sigma = \{ \sigma_{ij} = \Be_i \wedge \Be_j \}_{i<j}
\end{equation}

and construct the matrix of the wedge project \(T_\Bv : \mathbb{R}^N \rightarrow {\bigwedge}^2\)

\begin{equation}\label{eqn:projectionAndMoorePenroseVectorInverse:60}
T_\Bv(\Bx) = \Bv \wedge \Bx = \sum_{\mu = ij, i<j} \DETuvij{v}{x}{i}{j} \sigma_{\mu}
\end{equation}
\begin{equation}\label{eqn:projectionAndMoorePenroseVectorInverse:80}
\implies
T_\Bv(\Be_k) \cdot {\sigma_{ij}}^\dagger =
\sum_{k \in ij, i<j} \DETuvij{v}{x}{i}{j}
= %\sum_{k \in ij, i<j}
v_i \delta_{kj} - v_j \delta_{ki}
\end{equation}

Since \(k\) cannot be simultaneously equal to both \(i\), and \(j\), this is:

\begin{equation}\label{eqn:projectionAndMoorePenroseVectorInverse:100}
T_\Bv(\Be_k) \cdot {\sigma_{ij}}^\dagger =
\left\{
\begin{array}{rl}
v_i & k=j \\
-v_j & k=i \\
0 & k \ne i,j \\
\end{array}
\right\}
\end{equation}

Unlike the left Moore-Penrose vector inverse that we find as the matrix of the linear transformation \(v \cdot ( \cdot )\), except for \R{3} where we have the cross product, I do not recognize this as the matrix of any common linear transformation.
