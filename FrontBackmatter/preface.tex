%
% Copyright � 2012 Peeter Joot.  All Rights Reserved.
% Licenced as described in the file LICENSE under the root directory of this GIT repository.
%

%
%
%\chapter{Preface}
% this suppresses an explicit chapter number for the preface.
\chapter*{Preface}%\normalsize
  \thispagestyle{empty}
  \addcontentsline{toc}{chapter}{Preface}

This is an exploratory collection of notes containing worked examples of a number of introductory applications of Geometric Algebra (GA), also known as Clifford Algebra.  This writing is focused on undergraduate level physics concepts, with a target audience of somebody with an undergraduate engineering background.

These notes are more journal than book.  You'll find lots of duplication, since I reworked some topics from scratch a number of times.  In many places I was attempting to learn both the basic physics concepts as well as playing with how to express many of those concepts using GA formalisms.  The page count proves that I did a very poor job of weeding out all the duplication.

These notes are (dis)organized into the following chapters

\begin{itemize}
\item Basics and Geometry.
This chapter covers a hodge-podge collection of topics, including GA forms for traditional vector identities, Quaterions, Cauchy equations, Legendre polynomials, wedge product representation of a plane, bivector and trivector geometry, torque and more.  A couple attempts at producing an introduction to GA concepts are included (none of which I was ever happy with.)
\item Projection.
Here the concept of reciprocal frame vectors, using GA and traditional matrix formalisms is developed.  Projection, rejection and Moore-Penrose (generalized inverse) operations are discussed.
\item Rotation.
GA Rotors, Euler angles, spherical coordinates, blade exponentials, rotation generators, and infinitesimal rotations are all examined from a GA point of view.
\item Calculus.
Here GA equivalents for a number of vector calculus relations are developed, spherical and hyperspherical volume parameterizations are derived, some questions about the structure of divergence and curl are examined, and tangent planes and normals in 3 and 4 dimensions are examined.
Wrapping up this chapter is a complete GA formulation of the general Stokes theorem for curvilinear coordinates in Euclidean or non-Euclidean spaces.
\item General Physics.
This chapter introduces a bivector form of angular momentum (instead of a cross product), examines the components of radial velocity and acceleration, kinetic energy, symplectic structure, Newton's method, and a center of mass problem for a toroidal segment.
\end{itemize}

% The use of this algebra in Physics could be said to be still in its infancy.  There is a fair amount Geometric Algebra in advanced treatments like the work of the Cambridge group \citep{doran2003gap}.  There is much less that is easily accessible to someone with undergrad level education.  Even a text like Hestenes's New Foundations \citep{hestenes1999nfc}, which has a more elementary target audience is fairly difficult to read.  These notes attempt to bridge some of that gap.
% %Reading these leaves one having to do a fair amount of figure it out yourself.
% %This prompted a fair number of the notes in this compilation.  Somebody who has studied Physics instead of engineering would probably be better equipt for the subject as it is currently presented.

I can not promise that I have explained things in a way that is good for anybody else.  My audience was essentially myself as I existed at the time of writing, so the prerequisites, both for the mathematics and the physics, have evolved continually.

Peeter Joot  \quad peeterjoot@pm.me
